\documentclass[english,hidelinks,pdftex, 11 pt, class=report,crop=false]{standalone}
\usepackage[T1]{fontenc}
\usepackage[utf8]{luainputenc}
\usepackage{lmodern} % load a font with all the characters
\usepackage{geometry}
\geometry{verbose,a4paper, inner=0cm, outer=0 cm, bmargin=2cm, tmargin=1cm}
%\textwidth=12cm
\setlength{\parindent}{0bp}
\usepackage{import}
\usepackage[subpreambles=false]{standalone}
\usepackage{amsmath}
\usepackage{amssymb}
\usepackage{esint}
\usepackage{babel}
\usepackage{tabu}
\usepackage[dvipsnames, table]{xcolor}
\usepackage{cancel}
\makeatother
\makeatletter
\usepackage{datetime2}
\usepackage{titlesec}
\usepackage[many]{tcolorbox}

% Eheter
\newcommand{\enh}[1]{\,\textrm{#1}}
%referances
\newcommand{\net}[2]{{\color{blue}\href{#1}{#2}}}

%Spaces
\newcommand{\vsk}{\\[12pt]}
\newcommand{\vs}{\vspace{-12pt}}

% Tabell for opplegg

\newcommand{\ovlist}[1]{
\vspace{-16pt}
\begin{itemize}
	#1
\end{itemize}
}

% Chapters and sections
\titleformat{\section}[block]{\bfseries}{\hspace{3cm}\thesection}{5pt}{}
\titleformat{\subsection}[block]{\bfseries}{\hspace{3cm}\thesection}{5pt}{}
\newcommand{\sectionbreak}{\clearpage} % New page on each section
 

\newlength{\mywidth}
\setlength{\mywidth}{14cm}

\newcommand{\cont}[1]{
\begin{tcolorbox}[center, boxrule=0.0 mm, width=\mywidth,arc=0mm,enhanced jigsaw,,colback=white,breakable]
#1	
\end{tcolorbox}
}

\newcommand{\info}[5]{
\begin{tcolorbox}[center, boxrule=0.1 mm, width=\mywidth,arc=0mm,enhanced jigsaw,breakable,colback=yellow!5]	
	
	\footnotesize
	\textbf{Øvingsområde}\\[5pt] #1 
	
	\textbf{Utstyr}\\ #2  \\
	
	\begin{tabular}{@{} p{4cm} p{4cm} l} 
		\textbf{Tid} & \textbf{Elevinndeling} & \textbf{Læringsarena} \\
		#3  & #4 & #5
	\end{tabular} 
\end{tcolorbox}	
}

\newcommand{\gjen}[1]{\begin{tcolorbox}[center,boxrule=0.1 mm, width=\mywidth,arc=0mm,colback=blue!3] {\large \textbf{Gjennomføring} \vspace{5 pt}}\newline #1  \end{tcolorbox}\vspace{-5pt}}
\newcommand{\eks}[1]{\begin{tcolorbox}[center,boxrule=0.1 mm, width=\mywidth,arc=0mm,colback=green!3] {\large \textbf{Eksempel} \vspace{5 pt}}\newline #1  \end{tcolorbox}\vspace{-5pt}}

\newcounter{opl}
%\numberwithin{opl}{article}


\newcommand{\opl}[1]{
\newpage
{\refstepcounter{opl} %\phantomsection 
\large \textbf{\theopl \;#1} \vsk}
}

% Headlines
\newcommand{\fork}{\textbf{Forkunnskapar}\\}
\newcommand{\forb}{\textbf{Forberedelsar}\\}
\newcommand{\opgvr}{\textbf{Oppgaver}}



%colors
\newcommand{\colr}[1]{{\color{red} #1}}
\newcommand{\colb}[1]{{\color{blue} #1}}
\newcommand{\colo}[1]{{\color{orange} #1}}
\newcommand{\colc}[1]{{\color{cyan} #1}}
\definecolor{projectgreen}{cmyk}{100,0,100,0}
\newcommand{\colg}[1]{{\color{projectgreen} #1}}

% Lister med bokstavar
\usepackage[inline]{enumitem}
% Opg
\newcommand{\abc}[1]{
	\begin{enumerate}[label=\alph*),leftmargin=18pt]
		#1
	\end{enumerate}
}

\usepackage[]{hyperref}

\begin{document}
\newpage
\section{Likskapsteiknet, mengder og tallinjer}
\index{tal}
\subsection*{\likteikn}
Som namnet tilseier, viser \textit{likskapsteiknet} \index{likskapsteiknet} \sym{$ = $} til at noko er likt. I kva grad og når ein kan seie at noko er likt er ein filosofisk diskusjon, og innleiingsvis er vi berre prisgitt dette: Kva likskap \sym{$=$} sikter til må bli forstått ut ifrå konteksten teiknet blir brukt i. Med denne forståinga av \sym{=} kan vi studere nokre grunnleggande eigenskaper for tala våre, og så komme tilbake til meir presise tydingar av teiknet. \regv
\spr{
Vanlege måtar å seie \sym{$=$} på er
\begin{itemize}
	\item ''er lik'' \\
	\item ''er det same som''
\end{itemize}
}
\subsection*{Mengder og tallinjer}
Tal kan representere så mangt. I denne boka skal vi halde oss til to måtar å tolke tala på; tal som ei \textsl{mengde} og tal som ei \textsl{plassering på ei linje}. Alle representasjonar av tal tek eigentleg utganspunkt i kva forståinga er av tala 0 og 1.

\subsubsection*{Tal som mengde}
	Når vi snakkar om ei mengde, vil talet 0 vere\footnote{I \hrs{Rekneartane}{kapittel} skal vi sjå at det også er andre tolkingar av 0.} knytt til ''ingenting''. Ein figur der det ikkje er noko til stades vil slik vere det same som 0:
	\[ =0 \]
	1 vil vi teikne som ei rute:
	\fig{rut1}

Andre tal vil da vere definert ut ifrå kor mange einarruter (einarar) ein har:
	\fig{rut2}
\newpage	
\subsubsection*{Tal som plassering på ei linje}
	Når vi plasserer tal på ei linje, vil 0 vere utgangspunket vårt:
	\fig{lin0a}
	Så plasserer vi 1 ei viss lengde til høgre for 0:
	\fig{lin0b}
	Andre tal vil da vere definert ut ifrå kor mange einarlengder (einarar) vi er unna 0:
	\fig{lin1}
\subsection*{Positive heiltal}
Vi skal straks sjå at tal ikkje naudsynleg treng å vere \textsl{heile} antal einarar, men tala som \textsl{er} det har eit eige namn:\regv

\reg[Positive heiltal]{
Tal som er eit heilt antal einarar kallast \textit{positive\footnotemark heiltal}\index{positive heiltal}. Dei positive heiltala er
\[ 1, 2, 3, 4, 5 \text{ og så vidare.} \]
Positive heiltal blir også kalla \textit{naturlege tal}\index{tal!naturlege}.
}
\info{Kva med 0?}{
Nokre forfattarar inkluderer også 0 i omgrepet naturlege tal. I nokre samanhengar vil dette lønne seg, i andre ikkje.
}
\footnotetext{Kva ordet positiv inneber skal vi gjere greie for i \hrs{Negtal}{kapittel}.}

\newpage
\section{\talsifverd}
Tala våre er bygd opp av \textit{siffera}\index{siffer} $ 0, 1, 2 , 3, 4, 5, 6, 7, 8 $ og $ 9 $, og \textsl{plasseringa} av dei. Siffera og deira plassering definerer\footnote{Etterkvart skal vi også sjå at \textit{forteikn} er med på å definere verdien til talet (sjå \hrs{Negtal}{kapittel}).} \textit{verdien} \index{verdi} til talet.
\subsection*{Heiltal større enn 10}
La oss som eit eksempel skrive talet \textsl{fjorten} ved hjelp av sifra våre.
\fig{tel}
Vi kan no lage ei gruppe med 10 einarar, i tillegg har vi da 4 einarar. Da skriv vi fjorten slik:
\[ \text{fjorten}=14 \]
\fig{tel2}\vsk

\fig{tel2t}
\newpage
\subsection*{Desimaltal}
I mange tilfelle har vi ikkje eit heilt antal einarar, og da vil det vere behov for å dele 1 inn i mindre bitar. La oss starte med å teikne ein einar:
\fig{maal}
\fig{des1}
Så deler vi einaren vår inn i 10 mindre bitar:
\fig{maal1}
\fig{des1a}
Sidan vi har delt 1 inn i 10 bitar, kallar vi ein slik bit for \textit{ein tidel}:
\fig{maal1a}
\fig{des1b}
\begin{comment}
\eks{\vs
	\fig{maal2}
	\fig{des2}
}\vsk
\end{comment}
Tidelar skriv vi ved hjelp av  \textit{desimalteiknet} \sym{,}  :
\fig{maal1b}
\fig{des1c}
\eks[]{\vs 
	\fig{maal2a}
	\fig{des3}
}\regv
\spr{
På engelsk bruker ein punktum \sym{.} som desimalteikn i staden for komma \sym{,}\,:\vsb
\alg{
	3,5&\quad(norsk) \\
	3.5&\quad (english)
}
}
\newpage
\subsection*{Titalssystemet}
Vi har no sett korleis vi kan uttrykke verdien til tal ved å plassere \\siffer etter antal tiarar, einarar og tidelar, og det stoppar sjølvsagt ikkje der: \regv

\reg[Titalssystemet \label{titalsys}]{
Verdien til eit tal er gitt av siffera $ 0, 1, 2, 3, 4, 5, 6, 7, 8  $ og $ 9 $, og plasseringa av dei. Med sifferet som angir einarar som utgangspunkt vil
\begin{itemize}
	\item siffer til venstre (i rekkefølge) indikere antal tiarar, \\hundrarar, tusenar osv.
	\item siffer til høgre (i rekkefølge) indikere antal tidelar, \\hundredelar, tusendelar osv.
\end{itemize}
}
\eks[1]{\vs 
	\fig{maal3}
}
\eks[2]{ \vs \vs
\fig{titalsys}
}
\newpage
\reg[Partal og oddetal \label{parogodd}]{
	Heiltal som har 0, 2, 6 eller 8 på einarplassen kallast \textit{partal} \index{partal}.\vsk
	
	Heiltall som har 1, 3, 5, 7 eller 9 på einarplassen kallast \textit{oddetal} \index{oddetal}.
}
\eks{
	Dei ti første (positive) partala er
	\[ \text{0, 2, 4, 6, 8, 10, 12, 14, 16, og 18} \]
	De ti første (positive) oddetala er
	\[ \text{1, 3, 5, 7, 9, 11, 13, 15, 17, og 19} \]
}

\newpage
\section{\koordsys \label{Koord}}

\prbxl{0.5}{I mange tilfelle er det nyttig å bruke to tallinjer samtidig. Dette kallar vi eit \textit{koordinatsystem}\index{koordinatsystem}. Vi plasserer da éi tallinje som går \textsl{horisontalt} og éi som går \textsl{vertikalt}. Ei plassering i eit koordinatsystem kallar vi eit \textit{punkt}\index{punkt}. 
 }\qquad
\prbxr{0.4}{Strengt tatt fins det mange typar koordinatsystem, men i denne boka bruker vi ordet om berre éin sort, nemleg det \textit{kartesiske koordinatsystem}. Det er oppkalt etter den franske filosofen og matematikaren René Descartes.}
\st{Eit punkt skriv vi som to tal inni ein parentes. Dei to tala blir kalla \textit{førstekoordinaten} og \textit{andrekoordinaten}.
	\begin{itemize}
		\item Førstekoordinaten fortel oss kor langt vi skal gå langs horisontalaksen.
		\item Andrekoordinaten fortel oss kor langt vi skal gå langs vertikalaksen.
\end{itemize}
I figuren ser vi punkta $ (2, 3) $, $ (5, 1) $ og $ (0, 0) $. Punktet der aksane møtast, altså $ (0, 0) $, kallast \textit{origo}.
\fig{kord}
}

\end{document}