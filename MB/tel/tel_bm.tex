\documentclass[english,hidelinks,pdftex, 11 pt, class=report,crop=false]{standalone}
\usepackage[T1]{fontenc}
\usepackage[utf8]{luainputenc}
\usepackage{lmodern} % load a font with all the characters
\usepackage{geometry}
\geometry{verbose,a4paper, inner=0cm, outer=0 cm, bmargin=2cm, tmargin=1cm}
%\textwidth=12cm
\setlength{\parindent}{0bp}
\usepackage{import}
\usepackage[subpreambles=false]{standalone}
\usepackage{amsmath}
\usepackage{amssymb}
\usepackage{esint}
\usepackage{babel}
\usepackage{tabu}
\usepackage[dvipsnames, table]{xcolor}
\usepackage{cancel}
\makeatother
\makeatletter
\usepackage{datetime2}
\usepackage{titlesec}
\usepackage[many]{tcolorbox}

% Eheter
\newcommand{\enh}[1]{\,\textrm{#1}}
%referances
\newcommand{\net}[2]{{\color{blue}\href{#1}{#2}}}

%Spaces
\newcommand{\vsk}{\\[12pt]}
\newcommand{\vs}{\vspace{-12pt}}

% Tabell for opplegg

\newcommand{\ovlist}[1]{
\vspace{-16pt}
\begin{itemize}
	#1
\end{itemize}
}

% Chapters and sections
\titleformat{\section}[block]{\bfseries}{\hspace{3cm}\thesection}{5pt}{}
\titleformat{\subsection}[block]{\bfseries}{\hspace{3cm}\thesection}{5pt}{}
\newcommand{\sectionbreak}{\clearpage} % New page on each section
 

\newlength{\mywidth}
\setlength{\mywidth}{14cm}

\newcommand{\cont}[1]{
\begin{tcolorbox}[center, boxrule=0.0 mm, width=\mywidth,arc=0mm,enhanced jigsaw,,colback=white,breakable]
#1	
\end{tcolorbox}
}

\newcommand{\info}[5]{
\begin{tcolorbox}[center, boxrule=0.1 mm, width=\mywidth,arc=0mm,enhanced jigsaw,breakable,colback=yellow!5]	
	
	\footnotesize
	\textbf{Øvingsområde}\\[5pt] #1 
	
	\textbf{Utstyr}\\ #2  \\
	
	\begin{tabular}{@{} p{4cm} p{4cm} l} 
		\textbf{Tid} & \textbf{Elevinndeling} & \textbf{Læringsarena} \\
		#3  & #4 & #5
	\end{tabular} 
\end{tcolorbox}	
}

\newcommand{\gjen}[1]{\begin{tcolorbox}[center,boxrule=0.1 mm, width=\mywidth,arc=0mm,colback=blue!3] {\large \textbf{Gjennomføring} \vspace{5 pt}}\newline #1  \end{tcolorbox}\vspace{-5pt}}
\newcommand{\eks}[1]{\begin{tcolorbox}[center,boxrule=0.1 mm, width=\mywidth,arc=0mm,colback=green!3] {\large \textbf{Eksempel} \vspace{5 pt}}\newline #1  \end{tcolorbox}\vspace{-5pt}}

\newcounter{opl}
%\numberwithin{opl}{article}


\newcommand{\opl}[1]{
\newpage
{\refstepcounter{opl} %\phantomsection 
\large \textbf{\theopl \;#1} \vsk}
}

% Headlines
\newcommand{\fork}{\textbf{Forkunnskapar}\\}
\newcommand{\forb}{\textbf{Forberedelsar}\\}
\newcommand{\opgvr}{\textbf{Oppgaver}}



%colors
\newcommand{\colr}[1]{{\color{red} #1}}
\newcommand{\colb}[1]{{\color{blue} #1}}
\newcommand{\colo}[1]{{\color{orange} #1}}
\newcommand{\colc}[1]{{\color{cyan} #1}}
\definecolor{projectgreen}{cmyk}{100,0,100,0}
\newcommand{\colg}[1]{{\color{projectgreen} #1}}

% Lister med bokstavar
\usepackage[inline]{enumitem}
% Opg
\newcommand{\abc}[1]{
	\begin{enumerate}[label=\alph*),leftmargin=18pt]
		#1
	\end{enumerate}
}

\usepackage[]{hyperref}

\newcommand{\note}{Merk}
\newcommand{\notesm}[1]{{\footnotesize \textsl{\note:} #1}}
\newcommand{\ekstitle}{Eksempel }
\newcommand{\sprtitle}{Språkboksen}
\newcommand{\expl}{forklaring}
\newcommand{\pyt}{Pytagoras' setning}
\newcommand\sv{\vsk \textbf{Svar} \vspace{4 pt}\\}

%references
\newcommand{\reftab}[1]{\hrs{#1}{tabell}}
\newcommand{\rref}[1]{\hrs{#1}{regel}}
\newcommand{\dref}[1]{\hrs{#1}{definisjon}}
\newcommand{\refkap}[1]{\hrs{#1}{kapittel}}
\newcommand{\refsec}[1]{\hrs{#1}{seksjon}}
\newcommand{\refdsec}[1]{\hrs{#1}{delseksjon}}
\newcommand{\refved}[1]{\hrs{#1}{vedlegg}}
\newcommand{\eksref}[1]{\textsl{#1}}
\newcommand\fref[2][]{\hyperref[#2]{\textsl{figur \ref*{#2}#1}}}
\newcommand{\refop}[1]{{\color{blue}Oppgave \ref{#1}}}
\newcommand{\refops}[1]{{\color{blue}oppgave \ref{#1}}}


%Algebra
\newcommand{\kvadset}{Kvadratsetningene}
\newcommand{\aenato}{Sum-produkt-metoden}

% Geometry
\newcommand{\hlikb}{Midtnormalen i en likebeint trekant}
\newcommand{\arealsetn}{Arealsetningen}
\newcommand{\trkmedian}{Median}
\newcommand{\midtrk}{Midtnormal (i trekant)}
\newcommand{\innskrsirk}{Innskrevet sirkel}
\newcommand{\cossetn}{Cosinussetningen}
\newcommand{\perfvink}{Sentral- og periferivinkel}
\newcommand{\tang}{Tangent}

% Derivative
\newcommand{\derel}{Den deriverte av elementære funksjoner}
\newcommand{\divder}{Divisjonsregelen}
\newcommand{\kjernereg}{Kjerneregelen}
\newcommand{\prodregder}{Produktregelen}
\newcommand{\lhop}{L'Hopitals regel}

% Funksjonsdrofting
\newcommand{\monder}{Monotoniegenskaper og den deriverte}
\newcommand{\fderekstr}{$ \bm{f'=0} $ for lokale ektstremalpunkt}
\newcommand{\andredertest}{Andrederiverttesten}

% Vectors
\newcommand{\detar}{Arealformler med determinanter}
\newcommand{\avstpunktlin}{Avstand mellom punkt og linje}

%Appendix
\newcommand{\rolle}{Rolles teorem}
\newcommand{\meanval}{Middelverdisetningen}

% Solutions manual
\newcommand{\selos}{Se løsningsforslag.}

\begin{document}
\newpage
\section{Likhetstegnet, mengder og tallinjer}
\index{tall}
\subsection*{\likteikn}
Som navnet tilsier, viser \outl{likhetstegnet} \index{likhetstegnet} \sym{$ = $} til at noe er likt. I hvilken grad og når man kan si at noe er likt er en filosofisk diskusjon, og innledningsvis er vi bare prisgitt dette: \textsl{Hvilken likhet \sym{$=$} sikter til må bli forstått ut ifra konteksten tegnet blir brukt i}. Med denne forstelsen av \sym{$ = $} kan vi studere noen grunnleggende egenskaper for tallene våre, og så komme tilbake til mer presise betydninger av tegnet. \regv
\spr{
Vanlige måter å si \sym{$=$} på er
\begin{itemize}
	\item ''er lik'' \\
	\item ''er det samme som''
\end{itemize}
}
\subsection*{Mengder og tallinjer}
Tall kan representere så mangt. I denne boka skal vi holde oss til to måter å tolke tallene på; tall som en \textsl{mengde} og tall som en \textsl{plassering på en linje}. Alle representasjoner av tall tar utganspunkt i hva forståelsen er av tallene 0 og 1.

\subsubsection*{Tall som mengde}
	Når vi snakkar om en mengde, vil tallet 0 være\footnote{I \hrs{Rekneartane}{kapittel} skal vi se at det også er andre tolkninger av 0.} knyttet til ''ingenting''. En figur der det ikke er noe til stede vil slik være det samme som 0:
	\[ =0 \]
	1 vil vi tegne som en rute:
	\fig{rut1}

Andre tall vil da være definert ut ifra hvor mange ''enerruter'' (''enere'') vi har:
	\fig{rut2}
\newpage	
\subsubsection*{Tall som plassering på ei linje}
	Når vi plasserer tall på en linje, vil 0 være utgangspunket vårt:
	\fig{lin0a}
	Så plasserer vi 1 en viss lengde til høyre for 0:
	\fig{lin0b}
	Andre tall vil da være definert ut ifra hvor mange enerlengder (enere) vi er unna 0:
	\fig{lin1}
\subsection*{Positive heltall}
Vi skal straks se at tall ikke nødvendigvis trenger å være \textsl{hele} antall enere, men tallene som \textsl{er} det har et eget navn:\regv

\reg[Positive heltall]{
Tall som er et helt antall enere kalles \outl{positive\footnotemark heltall}\index{positive heltall}. De\\ positive heltallene er
\[ 1, 2, 3, 4, 5 \text{ og så videre.} \]
Positive heltall blir også kalt \outl{naturlige tal}\index{tall!naturlige}.
}
\info{Hva med 0?}{
Noen forfattere inkluderer også 0 i begrepet naturlige tal. I noen sammenhenger vil dette lønne seg, i andre ikke.
}
\footnotetext{Hva ordet 'positiv' innebærer skal vi se i \hrs{Negtal}{kapittel}.}

\newpage
\section{\talsifverd}
Tallene våre er bygd opp av \outl{sifrene}\index{siffer} $ 0, 1, 2 , 3, 4, 5, 6, 7, 8 $ og $ 9 $, og \textsl{plasseringen} av dem. Sifrene og deres plassering definerer\footnote{Etter hvert skal vi også se at \textit{fortegn} er med på å definere verdien til tallet (se \hrs{Negtal}{kapittel}).} \outl{verdien} \index{verdi} til tallet.
\subsection*{Heltall større enn 9}
La oss som et eksempel skrive tallet 'fjorten' ved hjelp av sifrene våre.
\fig{tel}
En gruppe med ti enere kaller vi en ''tier''. Av fjorten kan vi lage 1 tier, og i tillegg har vi da 4 enere. Da skriver vi 'fjorten' slik:
\[ \text{fjorten}=14 \]
\fig{tel2_bm}\vsk

\fig{tel2t}
\newpage
\subsection*{Desimaltall}
I mange tilfeller har vi ikke et helt antall enere, og da vil det være behov for å dele 1 inn i mindre biter. La oss starte med å tegne en ener:
\fig{maal}
\fig{des1}
Så deler vi eneren vår inn i 10 mindre biter:
\fig{maal1}
\fig{des1a}
Siden vi har delt 1 inn i 10 biter, kaller vi en slik bit for ''en tidel'':
\fig{maal1a_bm}
\fig{des1b_bm}
\begin{comment}
\eks{\vs
	\fig{maal2}
	\fig{des2}
}\vsk
\end{comment}
Tideler skriver vi ved hjelp av  \outl{desimaltegnet} \sym{,}  :
\fig{maal1b}
\fig{des1c}
\eks[]{\vs 
	\fig{maal2a}
	\fig{des3}
}\regv
\spr{
På engelsk bruker man punktum \sym{.} som desimaltegn:
\alg{
	3,5&\quad(norsk) \\
	3.5&\quad (english)
}
}
\newpage
\subsection*{Titallsystemet}
Vi har nå sett hvordan vi kan uttrykke verdien til tall ved å plassere \\siffer etter antall tiere, enere og tideler, og det stopper selvsagt ikke der: \regv

\reg[Titallsystemet \label{titalsys}]{
Verdien til et tall er gitt av sifrene $ 0, 1, 2, 3, 4, 5, 6, 7, 8  $ og $ 9 $, og plasseringen av dem. Med sifferet som angir enere som utgangspunkt vil
\begin{itemize}
	\item siffer til venstre (i rekkefølge) indikere antall tiere, \\hundrere, tusener og så videre.
	\item siffer til høyre (i rekkefølge) indikere antall tideler, \\hundredeler, tusendeler og så videre.
\end{itemize}
}
\eks[1]{\vs 
	\fig{maal3_bm}
}
\eks[2]{ \vs \vs
\fig{titalsys_bm}
}
\newpage
\reg[Partall og oddetall \label{parogodd}]{
Heltall som har 0, 2, 6 eller 8 på enerplassen kalles \outl{partall}\index{partall}.\vsk

Heltall som har 1, 3, 5, 7 eller 9 på enerplassen kalles \outl{oddetall}\index{oddetall}.
}
\eks{
De ti første (positive) partallene er
\[ \text{0, 2, 4, 6, 8, 10, 12, 14, 16, og 18} \]
De ti første (positive) oddetallene er
\[ \text{1, 3, 5, 7, 9, 11, 13, 15, 17, og 19} \]
}

\newpage
\section{\koordsys \label{Koord}}

\prbxl{0.5}{I mange tilfeller er det nyttig å bruke to tallinjer samtidig. Dette kaller vi et \outl{koordinatsystem}\index{koordinatsystem}. Vi plasserer da én tallinje (en akse) som går \textsl{horisontalt} og én som går \textsl{vertikalt}. En plassering i et koordinatsystem kaller vi et \outl{punkt}\index{punkt}. 
 }\qquad
\prbxr{0.4}{Strengt tatt finnes det mange typer koordinatsystem, men i denne boka bruker vi ordet om bare én sort, nemlig det \outl{kartesiske koordinatsystem}. Det er oppkalt etter den franske filosofen og matematikeren René Descartes.}
\st{Et punkt skriver vi som to tall inni en parentes. De to tallene blir kalt \outl{førstekoordinaten} og \outl{andrekoordinaten} til punktet.
	\begin{itemize}
		\item Førstekoordinaten forteller oss hvor langt vi skal gå langs horisontalaksen.
		\item Andrekoordinaten forteller oss hvor langt vi skal gå langs vertikalaksen.
\end{itemize}
I figuren ser vi punktene $ (2, 3) $, $ (5, 1) $ og $ (0, 0) $. Punktet der aksene møtes, altså $ (0, 0) $, kalles \outl{origo}\index{origo}.
\fig{kord}
}

\end{document}