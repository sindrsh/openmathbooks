\documentclass[english,hidelinks,xetex, 11 pt, class=report,crop=false]{standalone}
\input{../../preamb_cyrillic}

% note
\newcommand{\note}{Note}
\newcommand{\notesm}[1]{{\footnotesize \textsl{\note:} #1}}
\newcommand{\selos}{See the solutions manual.}

\newcommand{\texandasy}{The text is written in \LaTeX\ and the figures are made with the aid of Asymptote.}

\newcommand{\rknut}{Calculate.}
\newcommand\sv{\vsk \textbf{Answer} \vspace{4 pt}\\}
\newcommand{\ekstitle}{Example }
\newcommand{\sprtitle}{The language box}
\newcommand{\expl}{explanation}

% answers
\newcommand{\mulansw}{\notesm{Multiple possible answers.}}	
\newcommand{\faskap}{Chapter}

% exercises
\newcommand{\opgt}{\newpage \phantomsection \addcontentsline{toc}{section}{Exercises} \section*{Exercises for Chapter \thechapter}\vs \setcounter{section}{1}}

% references
\newcommand{\reftab}[1]{\hrs{#1}{Table}}
\newcommand{\rref}[1]{\hrs{#1}{Rule}}
\newcommand{\dref}[1]{\hrs{#1}{Definition}}
\newcommand{\refkap}[1]{\hrs{#1}{Chapter}}
\newcommand{\refsec}[1]{\hrs{#1}{Section}}
\newcommand{\refdsec}[1]{\hrs{#1}{Subsection}}
\newcommand{\refved}[1]{\hrs{#1}{Appendix}}
\newcommand{\eksref}[1]{\textsl{#1}}
\newcommand\fref[2][]{\hyperref[#2]{\textsl{Figure \ref*{#2}#1}}}
\newcommand{\refop}[1]{{\color{blue}Exercise \ref{#1}}}
\newcommand{\refops}[1]{{\color{blue}Exercise \ref{#1}}}

%%% SECTION HEADLINES %%%

% Our numbers
\newcommand{\likteikn}{The equal sign}
\newcommand{\talsifverd}{Numbers, digits and values}
\newcommand{\koordsys}{Coordinate systems}

% Calculations
\newcommand{\adi}{Addition}
\newcommand{\sub}{Subtraction}
\newcommand{\gong}{Multiplication}
\newcommand{\del}{Division}

%Factorization and order of operations
\newcommand{\fak}{Factorization}
\newcommand{\rrek}{Order of operations}

%Fractions
\newcommand{\brgrpr}{Introduction}
\newcommand{\brvu}{Values, expanding and simplifying}
\newcommand{\bradsub}{Addition and subtraction}
\newcommand{\brgngheil}{Fractions multiplied by integers}
\newcommand{\brdelheil}{Fractions divided by integers}
\newcommand{\brgngbr}{Fractions multiplied by fractions}
\newcommand{\brkans}{Cancelation of fractions}
\newcommand{\brdelmbr}{Division by fractions}
\newcommand{\Rasjtal}{Rational numbers}

%Negative numbers
\newcommand{\negintro}{Introduction}
\newcommand{\negrekn}{The elementary operations}
\newcommand{\negmeng}{Negative numbers as amounts}

%Calculation methods
\newcommand{\delmedtihundre}{Deling med 10, 100, 1\,000 osv.}

% Geometry 1
\newcommand{\omgr}{Terms}
\newcommand{\eignsk}{Attributes of triangles and quadrilaterals}
\newcommand{\omkr}{Perimeter}
\newcommand{\area}{Area}

%Algebra 
\newcommand{\algintro}{Introduction}
\newcommand{\pot}{Powers}
\newcommand{\irrasj}{Irrational numbers}

%Equations
\newcommand{\ligintro}{Introduction}
\newcommand{\liglos}{Solving with the elementary operations}
\newcommand{\ligloso}{Solving with elementary operations summarized}

%Functions
\newcommand{\fintro}{Introduction}
\newcommand{\lingraf}{Linear functions and graphs}

%Geometry 2
\newcommand{\geoform}{Formulas of area and perimeter}
\newcommand{\kongogsim}{Congruent and similar triangles}
\newcommand{\geofork}{Explanations}

% Names of rules
\newcommand{\adkom}{Addition is commutative}
\newcommand{\gangkom}{Multiplication is commutative}
\newcommand{\brdef}{Fractions as rewriting of division}
\newcommand{\brtbr}{Fractions multiplied by fractions}
\newcommand{\delmbr}{Fractions divided by fractions}
\newcommand{\gangpar}{Distributive law}
\newcommand{\gangparsam}{Paranthesis multiplied together}
\newcommand{\gangmnegto}{Multiplication by negative numbers I}
\newcommand{\gangmnegtre}{Multiplication by negative numbers II}
\newcommand{\konsttre}{Unique construction of triangles}
\newcommand{\kongtre}{Congruent triangles}
\newcommand{\topv}{Vertical angles}
\newcommand{\trisum}{The sum of angles in a triangle}
\newcommand{\firsum}{The sum of angles in a quadrilateral}
\newcommand{\potgang}{Multiplication by powers}
\newcommand{\potdivpot}{Division by powers}
\newcommand{\potanull}{The special case of \boldmath $a^0$}
\newcommand{\potneg}{Powers with negative exponents}
\newcommand{\potbr}{Fractions as base}
\newcommand{\faktgr}{Factors as base}
\newcommand{\potsomgrunn}{Powers as base}
\newcommand{\arsirk}{The area of a circle}
\newcommand{\artrap}{The area of a trapezoid}
\newcommand{\arpar}{The area of a parallelogram}
\newcommand{\pyt}{Pythagoras's theorem}
\newcommand{\forform}{Ratios in similar triangles}
\newcommand{\vilkform}{Terms of similar triangles}
\newcommand{\omkrsirk}{The perimeter of a circle (and the value of $ \bm \pi $)}
\newcommand{\artri}{The area of a triangle}
\newcommand{\arrekt}{The area of a rectangle}
\newcommand{\liknflyt}{Moving terms across the equal sign}
\newcommand{\funklin}{Linear functions}


\begin{document}
	\setcounter{chapter}{1}
	\section{\ligintro}
	Незважаючи на те, що кожний математичний вираз, що включає \sym{$=$}, є \textit{рівнянням}\index{equation}, традиційно це слово тісно пов'язане з наявністю \textit{невідомого} числа.\vsk
	
	Припустимо, ми хочемо знайти число, яке, будучи доданим до $4$, дає в результаті $7$. Назву цього невідомого числа можна вибрати довільно, але зазвичай його називають $ x $. Наше рівняння тепер можна записати як
	\[ x+4=7 \]
	Значення $ x $\footnote{У інших випадках може бути кілька значень.}, яке робить значення з обох сторін знака рівності однаковими, є \textit{розв'язком} рівняння.
	Нічого поганого в тому, щоб просто спостерігати, яким має бути значення $ x $. Ймовірно, ви вже зрозуміли, що $ {x=3} $ є розв'язком рівняння, оскільки
	\[ 3+4=7 \]
	Проте, більшість рівнянь важко розв'язати, просто спостерігаючи, тому мудро користуватися більш загальними методами. Насправді, існує лише один принцип, якому слід слідувати: \regv
	\st{ \label{principle}
		Ми завжди можемо виконати математичну операцію на одній із сторін знака рівності, якщо ми виконуємо операцію й на іншій стороні.}
	Математичні операції, представлені в цій книзі, є чотирма елементарними операціями. Стосовно цих принцип звучить так:\regv
	\st{Ми завжди можемо додавати, віднімати, множити чи ділити на число з однієї сторони знака рівності, якщо ми також робимо це з іншої сторони.}\regv
	Принцип випливає зі значення \sym{$=$}. Коли два вирази мають однакову вартість, їхні значення залишаються однаковими, доки ми виконуємо ідентичні математичні операції з ними. У наступному розділі ми конкретизуємо цей принцип для кожної окремої елементарної операції. Якщо вам вже все зрозуміло, ви можете без великої втрати розуміння перейти до \refsec{ligsaml}.
	\section{\liglos \label{likloysfire}}
	\textit{На малюнках цього розділу ми зрозуміємо рівняння з того, що ми називаємо принципом ваги. У цьому випадку \sym{$=$} вказує\footnote{\sym{$\neq$} символи ''не дорівнює''.} на те, що вага (або значення) зліва і справа є однаковою.} \vs
	\begin{figure}
		\centering
		\includegraphics[]{\figp{lig}}\qquad
		\includegraphics[]{\figp{lig1}}
	\end{figure} \vs \vspace{-3pt}
	\subsection*{Додавання та віднімання; перенесення членів}
	\subsubsection*{Перший приклад} \vspace{-3pt}
	Ми вже знайшли розв'язок цього рівняння, але зараз давайте розв'яжемо його іншим способом\footnote{\mer На попередніх малюнках розмір коробок відповідав (абсолютній) величині числа, яке вони представляли. Це не стосується коробок, які представляють $ x $.}: \vs
	\[ x+4=7 \]
	\fig{lig2}
	Значення $ x $ стає очевидним, якщо $ x $ залишиться сам на одній із сторін, і ми можемо ізолювати $ x $ з лівого боку, видаляючи 4. Але якщо ми збираємося видалити 4 з лівого боку, ми також повинні видалити 4 з правого боку, щоб зберегти рівні значення з обох сторін. \vspace{-3pt}
	\[ x+4-{\color{red}4}=7-{\color{red}4}  \]
	\fig{lig2b}
	Оскільки $ 4-{\color{red}4}=0 $ і $ 7-{\color{red}4}=3 $, ми отримуємо 
	\[ x=3 \]
	\fig{lig2c}
	Це можна записати стисліше так
	
	\prbxl{0.5}{
		\alg{
			x+4 &= 7 \\
			x&= 7-4 \\
			x &= 3
	}}
	\prbxr{0.5}{Між першою і другою лініями зазвичай кажуть, що \textsl{4 змінило сторону і тому також знак (з $ + $ на $ - $).}}
	\textbf{Другий приклад}\os
	Перейдемо до трохи складнішого рівняння\footnote{Зверніть увагу, що малюнок ілюструє $ {4x+(-2)} $ (див. \refsec{negmeng}) з лівого боку. Однак, $ {4x+(-2)} $ дорівнює $ {4x-2} $ (див. \refsec{rekmneg}).}:
	\[ 4x-2=3x+5 \]
	Щоб отримати вираз з $ x $ виключно на одному боці, ми видаляємо $ 3x $ з обох сторін:
	\[ 4x-2-{\color{red}3x}=3x+5-{\color{red}3x} \]
	Тепер,
	\[ x-2=5 \]
	Щоб ізолювати $ x $, ми додаємо 2 з лівого боку. Тоді ми також повинні додати $ 2 $ з правого боку:
	\[ x-2+{\color{blue}2}=5+{\color{blue}2} \]
	Отже
	\[ x=7 \]
	Кроки, які ми зробили, можна підсумувати так:
	\begin{flalign*}
		&& 4x-2&=3x+5 && \llap{1. малюнок} \\
		&& 4x-{\color{red}3x}-2&=3x-{\color{red}3x}+5 &&  \llap{2. малюнок} \\
		&& x -2 &= 5 &&\llap{3. малюнок}\\
		&& x-2+\color{blue}2&=  5+\color{blue}2 &&\llap{4. малюнок}\\
		&& x &= 7 &&\llap{5. малюнок}
	\end{flalign*}
	Стисліше можна записати
	\alg{
		4x-2&=3x+5 \\
		4x-{\color{red}3x}&=5+\color{blue}2\\
		x &= 7
	}
	
	\reg[Переміщення чисел через знак рівності \label{bytt}]{Щоб розв'язати рівняння, ми збираємо всі $x$-члени і всі відомі члени на відповідних сторонах знака рівності. Член, який переходить на іншу сторону, також змінює знак.}
	
	\eks[1]{Розв'яжіть рівняння
		\[ 3x+5 =2x+9 \]
		\sv	
		\vs \vs \vs \vs
		\begin{align*}
			3x-2x &=9-5 \\
			x &=4
		\end{align*}  \vspace{-20pt}}
	
	\eks[2]{Розв'яжіть рівняння
		\[ -4x-3 =-5x+12  \]	
		\sv
		\vs \vs \vs \vs
		\begin{align*}
			-4x+5x &=12+3 \\
			x &=15
	\end{align*}}
	\subsection*{Множення та ділення}
	
	\subsubsection{Ділення}
	Досі ми вивчали рівняння, які призводили до одного випадку $x$ на одній стороні знаку рівності. Часто $x$ зустрічається кілька разів, як, наприклад, у рівнянні
	\[ 3x=6 \]
	\fig{lig8}
	Якщо ми розділимо лівий бік на три рівні групи, то отримаємо одне $x$ в кожній групі. І, розділяючи праву сторону на три рівні групи, всі групи мають однакову вартість
	\[ \frac{3x}{3}=\frac{6}{3} \]
	\fig{lig9}
	Отже
	\[ x=2 \]
	\fig{lig10}
	Підсумуємо наші обчислення:
	\begin{flalign*}
		&& 3x&=6 && \llap{1. малюнок} \br
		&& \frac{\cancel{3}x}{\cancel{3}}&=\frac{6}{3} && \llap{2. малюнок} \br
		&& x&=2 && \llap{3. малюнок}
	\end{flalign*}
	\newpage
	\reg[Ділення обох сторін рівняння \label{ligdel}]{Ми можемо ділити обидві сторони рівняння на одне і те ж число.}
	\eks[1]{ Розв'язати рівняння
		\[ 	4x = 20  \]
		\sv \vs \vs \vsb
		\begin{align*}
			\frac{\cancel{4}x}{\cancel{4}}&=\frac{20}{4} \\
			x &=5
		\end{align*}
		\vspace{-20 pt}
	}
	
	\eks[2]{Розв'язати рівняння
		\[2x+6 =3x-2 \]
		\sv \vs \vs \vs \vs
		\begin{flalign*}
			&& 2x-3x &= -2-6 &&\\
			&&-x &= -8&& \\
			&& \frac{\cancel{-1}x}{\cancel{-1}} &= \frac{-8}{-1} &&\cm{($-x=-1x$)}\\
			&& x &= 8&&
	\end{flalign*}}
	
	\subsubsection*{Множення}
	Давайте розв'яжемо рівняння
	\[ \frac{x}{3}=4 \]
	\fig{lig11}
	Ми можемо отримати одиничне $ x $ зліва, якщо додамо ще два випадки $ \frac{x}{3} $. Рівняння вказує, що $ \frac{x}{3} $ дорівнює 4, це означає, що на кожний випадок $ \frac{x}{3} $, який ми додаємо зліва, ми повинні додати 4 справа, щоб зберегти баланс.
	\[ \frac{x}{3}+\frac{x}{3}+\frac{x}{3}=4+4+4 \]
	\fig{lig12}
	Тепер ми помічаємо, що \y{\frac{x}{3}+\frac{x}{3}+\frac{x}{3}=\frac{x}{3}\cdot3} і \y{4+4+4=4\cdot3}:
	\[ \frac{x}{3}\cdot 3 = 4\cdot 3 \]
	\fig{lig12b}
	Оскільки $ \frac{x}{3}\cdot3=x $ і $ 4\cdot3=12 $, ми маємо
	\[ x=12 \]
	\fig{lig13}
	Наші кроки можна підсумувати таким чином:
	\begin{flalign*}
		&& \frac{x}{3}&=4 && \llap{1. малюнок} \br 
		&& \frac{x}{3}+\frac{x}{3}+\frac{x}{3} &= 4+4+4  &&\llap{2. малюнок} \br
		&& \frac{x}{3}\cdot 3&=4\cdot3 && \llap{3. малюнок} \\
		&& x&=12 && \llap{4. малюнок}
	\end{flalign*}
	У стислішій формі це можна записати як
	\alg{
		\frac{x}{3}&= 4 \br 
		\frac{x}{\cancel{3}}\cdot \cancel{3} &= 4\cdot 3 \br
		x &= 12
	}
	\newpage
	\reg[Множення обох сторін рівняння]{
		Ми можемо множити обидві сторони рівняння на одне і те ж число.
	}
	\eks[1]{
		Розв'яжіть рівняння
		\[ \frac{x}{5}=2 \]
		\sv \vsb \vs
		
		\algv{
			\frac{x}{\cancel{5}}\cdot\cancel{5}&=2\cdot5 \\
			x &= 10
		}
	}
	\eks[2]{
		Розв'яжіть рівняння
		\[ \frac{7x}{10}-5=13+\frac{x}{10} \]
		\sv \vsb \vs
		
		\algv{
			\frac{7x}{10}-\frac{x}{10}&=13+5\br
			\frac{6x}{10}&=18 \br
			\frac{6x}{\cancel{10}}\cdot \cancel{10}&=18\cdot10 \\
			6x&=180 \br
			\frac{\cancel{6}x}{\cancel{6}}&=\frac{180}{6}\br
			x&=30
		}
	}
	
\newpage
\section{\ligloso \label{ligsaml}}
\reg[Методи розв'язання з використанням елементарних операцій \label{lsmlig}]{
	Ми завжди можемо
	\begin{itemize}
		\item додавати або віднімати однакове число з обох боків рівняння. Це еквівалентно перенесенню терміну з одного боку на інший, також змінюючи знак терміну.
		\item множити або ділити обидва боки рівняння на одне і те ж число.
	\end{itemize}
}
\eks[1]{
	Розв'яжіть рівняння
	\[ 3x-4=6+2x \]
	
	\sv \vs \vs
	
	\algv{
		3x-2x&=6+4 \\
		x&=10
	}
}
\eks[2]{
	Розв'яжіть рівняння
	\[ 9-7x=-8x+3 \]
	\sv \vs \vs
	\algv{
		8x-7x&=3-9 \\
		x&=-6
	}
}
\newpage
\eks[3]{
	Розв'яжіть рівняння
	\[ 10x-20=7x-5 \]
	\sv \vs \vsb
	\algv{
		10x-7x&=20-5 \\
		3x&=15 \\
		\frac{\cancel{3}x}{\cancel{3}}&=\frac{15}{3} \\
		x&=5
	}
}
\eks[4]{
	Розв'яжіть рівняння
	\[ 15-4x=x+5 \]
	
	\sv \vs \vs \vs
	\alg{
		15-5&=x+4x \\
		10&=5x\\
		\frac{10}{5}&=\frac{\cancel{5}x}{\cancel{5}}\\
		2&=x
	}
	{\footnotesize \mer У інших прикладах ми вирішили збирати випадки $ x $ з лівого боку рівняння, але ми можемо так само зібрати їх і з правого боку. Зробивши це тут, ми уникли обчислень з від'ємними числами.}
}	

\eks[5]{
	Розв'яжіть рівняння
	\[ \frac{4x}{9}-20=8-\frac{3x}{9} \]
	
	\sv \vs \vsb
	\alg{
		\frac{4x}{9}+\frac{3x}{9}&=20+8 \\
		\frac{\cancel{7}x}{9\cdot \cancel{7}}&=\frac{28}{7} \br
		\frac{x}{\cancel{9}}\cdot \cancel{9}&=4\cdot 9\\
		x&=36
	}
}

\newpage
\eks[6]{
	Розв'яжіть рівняння
	\[ \frac{1}{3}x+\frac{1}{6}=\frac{5}{12}x+2  \vs\]
	
	\sv
	Щоб уникнути дробів, множимо обидві сторони на спільний знаменник 12:
	\begin{align}
		\left(\frac{1}{3}x+\frac{1}{6}\right)12&=\left(\frac{5}{12}x+2\right)12 \br
		\frac{1}{3}x\cdot12+\frac{1}{6}\cdot12&=\frac{5}{12}x\cdot12+2\cdot12 \tag{$ \ast $}\br
		4x+2 &= 5x+24 \\
		4x-5x &= 24-2 \\
		-x &= 22 \\
		\frac{\cancel{-1}\,x}{\cancel{-1}} &= \frac{22}{-1} \\
		x &= -22
	\end{align}
}
\info{Порада}{
	Деякі вважають за краще правило, що ''{ми можемо множити або ділити всі терміни на одне й те ж число}''. У цьому випадку ми могли б перейти до другого рядка у обчисленнях згаданого прикладу.
}
\eks[7]{
	Розв'яжіть рівняння
	\[ 3-\frac{6}{x}= 2+\frac{5}{2x} \vs\]
	
	\sv
	Множимо обидві сторони на спільний знаменник $ 2x $:
	\alg{
		2x\left(3-\frac{6}{x}\right)&= 2x\left(2+\frac{5}{2x}\right) \br
		6x -12 &= 4x+5\\
		6x-4x &= 5+12\\
		2x &= 17\\
		x&=\frac{17}{2}
	}
}
\section{Рівняння зі ступенями}
Давайте розв'яжемо рівняння
\[ x^2=9 \]
Це називається \textit{ступеневим рівнянням}\index{power equation}. Загалом, ступеневі рівняння важко розв'язати, застосовуючи лише чотири елементарні операції. Застосовуючи правила ступенів, підносимо обидві сторони до ступеня, оберненого до показника ступеня при $ x $:
\[ \left(x^2\right)^\frac{1}{2}=9^\frac{1}{2} \]
За \rref{potsomgrunn}, маємо
\alg{
	x^{2\cdot\frac{1}{2}}&=9^\frac{1}{2} \\
	x&=9^\frac{1}{2}
}
Оскільки $ 3^2=9 $, маємо $ 9^\frac{1}{2}=3 $. Тепер зверніть увагу: \vsk

\textit{Принцип, викладений на сторінці \pageref{principle} каже, що ми можемо, як ми щойно зробили, виконувати математичну операцію на обох сторонах рівняння. Однак, дотримання цього принципу не гарантує, що будуть знайдені всі розв'язки.} \\ \vsk

Стосовно нашого рівняння, ми знаємо, що $ {x=3} $ є розв'язком. Заради цього ми можемо підтвердити це обчисленням
\[ 3^2=3\cdot3=9 \]
Але ми також маємо
\[ (-3)^2=(-3)(-3)=9 \]
Отже, $ -3 $ також є розв'язком нашого оригінального рівняння!

\reg[Ступеневі рівняння]{
	Рівняння, яке можна записати як
	\[ x^a=b \]
	де $ a $ і $ b $ є сталими,
	є \textit{ступеневим} рівнянням. \vsk
	
	Рівняння має $ a $ різних розв'язків.
}
\newpage
\eks[1]{
	Розв'яжіть рівняння
	\[ x^2+5= 21\]
	\sv \vs \vs \vs
	\algv{
		x^2+5&= 21\\
		x^2 &= 21-5\\
		x^2 &= 16
	}
	Оскільки $ {4\cdot4 =16} $ і $ {(-4)\cdot(-4)=16} $, маємо
	\[ x=4\qquad\vee\qquad x=-4 \]
}
\eks[2]{
	Розв'яжіть рівняння
	\[ 3x^2+1=7 \]
	\sv \vs \vs \vs
	\alg{
		3x^2&=7-1 \\
		3x^2&=6 \\
		\frac{\cancel{3}x^2}{\cancel{3}}&=\frac{6}{3}\\
		x^2&=2}
	Отже,
	\[ x=\sqrt{2}\qquad\vee\qquad x=-\sqrt{2} \]
}
\info{\note}{
	Хоча рівняння
	\[ x^a=b \]
	має $ a $ розв'язків, вони не обов'язково всі \textit{дійсні}\footnote{Як зазначалося раніше, \textit{дійсні} та \textit{уявні} числа виходять за рамки цієї книги.}. Стосовно цієї книги, це означає, що ми задовольняємося знаходженням усіх раціональних або ірраціональних чисел, які розв'язують рівняння. Наприклад,
	\[ x^3=8 \]
	має 3 розв'язки, але ми задовольняємося з розв'язком $ x=2 $.
}

\end{document}


