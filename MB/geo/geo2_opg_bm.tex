\documentclass[english,hidelinks,pdftex, 11 pt, class=report,crop=false]{standalone}
\usepackage[T1]{fontenc}
\usepackage[utf8]{luainputenc}
\usepackage{lmodern} % load a font with all the characters
\usepackage{geometry}
\geometry{verbose,a4paper, inner=0cm, outer=0 cm, bmargin=2cm, tmargin=1cm}
%\textwidth=12cm
\setlength{\parindent}{0bp}
\usepackage{import}
\usepackage[subpreambles=false]{standalone}
\usepackage{amsmath}
\usepackage{amssymb}
\usepackage{esint}
\usepackage{babel}
\usepackage{tabu}
\usepackage[dvipsnames, table]{xcolor}
\usepackage{cancel}
\makeatother
\makeatletter
\usepackage{datetime2}
\usepackage{titlesec}
\usepackage[many]{tcolorbox}

% Eheter
\newcommand{\enh}[1]{\,\textrm{#1}}
%referances
\newcommand{\net}[2]{{\color{blue}\href{#1}{#2}}}

%Spaces
\newcommand{\vsk}{\\[12pt]}
\newcommand{\vs}{\vspace{-12pt}}

% Tabell for opplegg

\newcommand{\ovlist}[1]{
\vspace{-16pt}
\begin{itemize}
	#1
\end{itemize}
}

% Chapters and sections
\titleformat{\section}[block]{\bfseries}{\hspace{3cm}\thesection}{5pt}{}
\titleformat{\subsection}[block]{\bfseries}{\hspace{3cm}\thesection}{5pt}{}
\newcommand{\sectionbreak}{\clearpage} % New page on each section
 

\newlength{\mywidth}
\setlength{\mywidth}{14cm}

\newcommand{\cont}[1]{
\begin{tcolorbox}[center, boxrule=0.0 mm, width=\mywidth,arc=0mm,enhanced jigsaw,,colback=white,breakable]
#1	
\end{tcolorbox}
}

\newcommand{\info}[5]{
\begin{tcolorbox}[center, boxrule=0.1 mm, width=\mywidth,arc=0mm,enhanced jigsaw,breakable,colback=yellow!5]	
	
	\footnotesize
	\textbf{Øvingsområde}\\[5pt] #1 
	
	\textbf{Utstyr}\\ #2  \\
	
	\begin{tabular}{@{} p{4cm} p{4cm} l} 
		\textbf{Tid} & \textbf{Elevinndeling} & \textbf{Læringsarena} \\
		#3  & #4 & #5
	\end{tabular} 
\end{tcolorbox}	
}

\newcommand{\gjen}[1]{\begin{tcolorbox}[center,boxrule=0.1 mm, width=\mywidth,arc=0mm,colback=blue!3] {\large \textbf{Gjennomføring} \vspace{5 pt}}\newline #1  \end{tcolorbox}\vspace{-5pt}}
\newcommand{\eks}[1]{\begin{tcolorbox}[center,boxrule=0.1 mm, width=\mywidth,arc=0mm,colback=green!3] {\large \textbf{Eksempel} \vspace{5 pt}}\newline #1  \end{tcolorbox}\vspace{-5pt}}

\newcounter{opl}
%\numberwithin{opl}{article}


\newcommand{\opl}[1]{
\newpage
{\refstepcounter{opl} %\phantomsection 
\large \textbf{\theopl \;#1} \vsk}
}

% Headlines
\newcommand{\fork}{\textbf{Forkunnskapar}\\}
\newcommand{\forb}{\textbf{Forberedelsar}\\}
\newcommand{\opgvr}{\textbf{Oppgaver}}



%colors
\newcommand{\colr}[1]{{\color{red} #1}}
\newcommand{\colb}[1]{{\color{blue} #1}}
\newcommand{\colo}[1]{{\color{orange} #1}}
\newcommand{\colc}[1]{{\color{cyan} #1}}
\definecolor{projectgreen}{cmyk}{100,0,100,0}
\newcommand{\colg}[1]{{\color{projectgreen} #1}}

% Lister med bokstavar
\usepackage[inline]{enumitem}
% Opg
\newcommand{\abc}[1]{
	\begin{enumerate}[label=\alph*),leftmargin=18pt]
		#1
	\end{enumerate}
}

\usepackage[]{hyperref}

\newcommand{\note}{Merk}
\newcommand{\notesm}[1]{{\footnotesize \textsl{\note:} #1}}
\newcommand{\ekstitle}{Eksempel }
\newcommand{\sprtitle}{Språkboksen}
\newcommand{\expl}{forklaring}
\newcommand{\pyt}{Pytagoras' setning}
\newcommand\sv{\vsk \textbf{Svar} \vspace{4 pt}\\}

%references
\newcommand{\reftab}[1]{\hrs{#1}{tabell}}
\newcommand{\rref}[1]{\hrs{#1}{regel}}
\newcommand{\dref}[1]{\hrs{#1}{definisjon}}
\newcommand{\refkap}[1]{\hrs{#1}{kapittel}}
\newcommand{\refsec}[1]{\hrs{#1}{seksjon}}
\newcommand{\refdsec}[1]{\hrs{#1}{delseksjon}}
\newcommand{\refved}[1]{\hrs{#1}{vedlegg}}
\newcommand{\eksref}[1]{\textsl{#1}}
\newcommand\fref[2][]{\hyperref[#2]{\textsl{figur \ref*{#2}#1}}}
\newcommand{\refop}[1]{{\color{blue}Oppgave \ref{#1}}}
\newcommand{\refops}[1]{{\color{blue}oppgave \ref{#1}}}


%Algebra
\newcommand{\kvadset}{Kvadratsetningene}
\newcommand{\aenato}{Sum-produkt-metoden}

% Geometry
\newcommand{\hlikb}{Midtnormalen i en likebeint trekant}
\newcommand{\arealsetn}{Arealsetningen}
\newcommand{\trkmedian}{Median}
\newcommand{\midtrk}{Midtnormal (i trekant)}
\newcommand{\innskrsirk}{Innskrevet sirkel}
\newcommand{\cossetn}{Cosinussetningen}
\newcommand{\perfvink}{Sentral- og periferivinkel}
\newcommand{\tang}{Tangent}

% Derivative
\newcommand{\derel}{Den deriverte av elementære funksjoner}
\newcommand{\divder}{Divisjonsregelen}
\newcommand{\kjernereg}{Kjerneregelen}
\newcommand{\prodregder}{Produktregelen}
\newcommand{\lhop}{L'Hopitals regel}

% Funksjonsdrofting
\newcommand{\monder}{Monotoniegenskaper og den deriverte}
\newcommand{\fderekstr}{$ \bm{f'=0} $ for lokale ektstremalpunkt}
\newcommand{\andredertest}{Andrederiverttesten}

% Vectors
\newcommand{\detar}{Arealformler med determinanter}
\newcommand{\avstpunktlin}{Avstand mellom punkt og linje}

%Appendix
\newcommand{\rolle}{Rolles teorem}
\newcommand{\meanval}{Middelverdisetningen}

% Solutions manual
\newcommand{\selos}{Se løsningsforslag.}

\begin{document}
	
\opgt

\op{geoopgformlvink}
Trekantene er formlike. Bestem verdien til $ \angle ACB $. \vs
\fig{geoopg2}

\op{geoopgsamsv}
Trekantene er formlike. Finn de tre parene med samsvarende sider.
\fig{geoopg3} 

\op{geofinnlen}
Trekantene er formlike. Finn lengden til $ EF $ og lengden til $ DF $.
\fig{geoopg4}
\newpage
\op{geofinnlen2}
Trekantene er formlike. Finn lengden til $ AC $ og lengden til $ DF $.
\fig{geoopg5}

\op{opggeofinnforml}
Finn alle formlike trekanter definert av $ A $, $ B $, $ C $ og $ D $.
\fig{opggeofinnforml}

\op{opggeoarforhold}
$ \triangle ABC $ og $ \triangle DEF $ er formlike. Hva er forholdet mellom arealet til $ \triangle DEF $ og arealet til $ \triangle ABC $ hvis $ h_1=2 $ og $ h_2=6 $? \vsk

(Se også \grubr{opggeoarforhold2})
\fig{opggeoarforh}

\newpage
\op{opggeototilen}
Finn lengden til $ c $ uttrykt ved $ a $.
\fig{opggeototilen}


\op{opggeovolkjegl}
En kjegle har radius 10 og høgde 4.
\abc{
\item Finn grunnflaten til kjeglen.
\item Finn volumet til kjeglen.
}

\op{opggepvolforhold} \vs
\abc{
	\item En kule har radius $ 2 $ og en annen kule har radius $ 6 $. Hva er forholdet mellom volumet til den største kula og volumet til den minste kula?
	\item En kule har radius $ r $ og en annen kule har radius $ ar $, hvor $ {a>1} $. Hva er forholdet mellom volumet til den største kula og volumet til den minste kula?
}

\newpage

\grubop{opggeo263}
Finn lengden til den røde linja.
\fig{opggeo263}


\grubop{opgegeotrapes}
Finn arealet til det grønne området.
\fig{opggeotrapes}


\grubop{opggeoarforhold2}
$ \triangle ABC $ og $ \triangle DEF $ er formlike. Gitt et tall $ a $, og at $ {h_2=a h_1} $. Finn $ \frac{A_{\triangle DEF}}{A_{\triangle ABC} } $ uttrykt ved $ a $.
\fig{opggeoarforh}

\newpage
\grubop{GV21D1opg12}
(GV21D1) \os
Figuren under viser en regulær\footnote{I regulære mangekanter har alle sidene lik lengde.} sekskant. Bestem hvor mange grader $  v $ er.
\fig{opggv21d1opg12}



\grubop{opglikbmidtn}
Gitt en likebeint trekant $\triangle {ABC} $ hvor $ AC=BC $. Vis at halveringslinja\footnote{Definisjonen av halveringslinja til en vinkel og midtnormalen til ei linje finner du i \tmen.} til $ \angle ACB $ er midtnormalen til $ AB $.

\grubop{opggeoeqlheight}
$ \triangle ABC $ er likesidet og har sidelengde $ s $. 
\fig{opggeoeqlheight}
\abc{
\item Vis at i en trekant med vinklene $ 30^\circ $, $ 60^\circ $, $ 90^\circ $, så er hypotenusen dobbelt så lang som den korteste kateten.
\item Vis at høgda i $ \triangle ABC $ er $ \frac{\sqrt{3}}{2}s $.
}

\newpage
\grubop{opggeovispyt}
\abc{
\item Finn $ AD $ uttrykt ved $ a $, $ b $ og $ c $.
\item Finn $ DB $ uttrykt ved $ a $, $ b $ og $ c $.
\item Bruk uttrykkene du fant til å bevise Pytagoras' setning.
}
\fig{opggeovispyt}

\grubop{opggeopolvinksum}
Gitt en mangekant med $ n $ sider. Finn en formel for vinkelsummen til mangekanten uttrykt ved $ n $.


\grubop{opggeopytar}
En rettvinklet trekant har omkrets 40. Den ene siden i trekanten har lengde 8. Finn de to andre sidelengdene til trekanten.

\newpage
\grubop{opggeosekt}
Overflaten til en (vilkårlig) kjegle består av en sektor med radius $ s $ og en sirkel med radius $ r $.
\abc{
\item Skriv $ s $ som et uttrykk av $ r $ og høyden $ h $ til kjeglen.	
\item Vis at overflatearealet $ A_O $ til kjeglen er gitt ved formelen
\[ A_O = \pi r (r+s) \]
}

\fig{opggeosekt}



\grubop{opgdoublear}
Vis at det doble arealet til $ \triangle ABC $ er gitt som
\[ AE\cdot BD + CE\cdot AD \]
\fig{opggeoqrst}


\grubop{opgmedian}
En \outl{median} i en trekant er et linjestykke som går fra et hjørne til midten av den motstående siden. 
\fig{geoopg1}
Gitt en vilkårlig trekant $ \triangle ABC $ med medianer $ AE $, $ BF $ og  $ CD $. 
\abc{
\item Vis at $ AE $, $ BF $ og $ CD $ skjærer hverandre i samme punkt ($ G $ på figuren).
\item Vis at
\[ \frac{GC}{DG}=\frac{GB}{FG}=\frac{GA}{EG}=2 \]
}
\mers{Oppgave b) er nok lettere enn oppgave a).}

\newpage
\grubop{opgtresirkar}
De tre sirklene har radius $ 2 $, og $ A $, $ B $ og $ C $ ligger på linje.
Finn arealet til det røde området.
\fig{opggeotresirk}
{\footnotesize Hint: Her kan du nok få bruk for at arealet til en sektor med vinkel $ v $ utgjør $ \frac{a}{360^\circ} $ av arealet til sirkelen med samme radius.}


\grubop{opggeolikar}
De fargede områdene utgjør et kvadrat, og $ F $, $ G $, $ H $ og $ I $ er de respektive midpunktene på sidene til dette kvadratet.\os

Vi at arealet til det blåfargede området er det samme som arealet til det grønnfargede området.
\fig{opggeolikar}

\grubop{opggeosirkfirk}
Kvadratet har sidelengde 4. Finn radien til sirkelen.
\fig{opggeosirkfirk}

\newpage
\grubop{opggeosin15inv}
\abc{
\item Vis at $ \frac{a}{b}=\sqrt{2}+\sqrt{6} $.\\
\fig{opggeosin15inv}
\mers{For å løse denne oppgaven er det mulig (men ikke nødvendigvis) du vil få bruk for \textit{abc}-formelen, som du finner i \tmen.}
\item $ AD=BC $. Bestem verdien til $ \angle A $.
\fig{opggeo1530}
}
\newpage
\grubop{opggeotrekulik}
\mers{Denne oppgaven tar for seg resultater som intuitivt virker helt opplagte, men som kan være krevende å bevise. 
}
\abc{
\item Vis at hvis $ AC=BC $, er $ \angle A =\angle B$.
\fig{opggeotrekulikisc}
\mers{Vi har tidligere erklært at en likebeint trekant har to vinkler som er like store, men strengt tatt kan vi ikke bare gå ut ifra at det er slik.
}
\item Vis at hvis $ AC>BC $, er $ \angle B>\angle C $.
\fig{opggeotrekulikvink}
\item Gitt $ \triangle ABC $, hvor $ AB $ er den lengste siden. Vis at når $ AB $ er grunnlinje, ligger høgden inni trekanten.
\item I figuren under er $ c $ den lengste siden i trekanten. 
\fig{opggeotrekulik}
Bevis at  
\[ c>a+b\qquad,\qquad b+c>a\qquad,\qquad a+c>b  \]
\mers{Disse tre ulikhetene samlet kalles gjerne \outl{trekantulikheten}.
}

}


\end{document}

