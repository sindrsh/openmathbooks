\documentclass[english,hidelinks,pdftex, 11 pt, class=report,crop=false]{standalone}
\usepackage[T1]{fontenc}
\usepackage[utf8]{luainputenc}
\usepackage{lmodern} % load a font with all the characters
\usepackage{geometry}
\geometry{verbose,a4paper, inner=0cm, outer=0 cm, bmargin=2cm, tmargin=1cm}
%\textwidth=12cm
\setlength{\parindent}{0bp}
\usepackage{import}
\usepackage[subpreambles=false]{standalone}
\usepackage{amsmath}
\usepackage{amssymb}
\usepackage{esint}
\usepackage{babel}
\usepackage{tabu}
\usepackage[dvipsnames, table]{xcolor}
\usepackage{cancel}
\makeatother
\makeatletter
\usepackage{datetime2}
\usepackage{titlesec}
\usepackage[many]{tcolorbox}

% Eheter
\newcommand{\enh}[1]{\,\textrm{#1}}
%referances
\newcommand{\net}[2]{{\color{blue}\href{#1}{#2}}}

%Spaces
\newcommand{\vsk}{\\[12pt]}
\newcommand{\vs}{\vspace{-12pt}}

% Tabell for opplegg

\newcommand{\ovlist}[1]{
\vspace{-16pt}
\begin{itemize}
	#1
\end{itemize}
}

% Chapters and sections
\titleformat{\section}[block]{\bfseries}{\hspace{3cm}\thesection}{5pt}{}
\titleformat{\subsection}[block]{\bfseries}{\hspace{3cm}\thesection}{5pt}{}
\newcommand{\sectionbreak}{\clearpage} % New page on each section
 

\newlength{\mywidth}
\setlength{\mywidth}{14cm}

\newcommand{\cont}[1]{
\begin{tcolorbox}[center, boxrule=0.0 mm, width=\mywidth,arc=0mm,enhanced jigsaw,,colback=white,breakable]
#1	
\end{tcolorbox}
}

\newcommand{\info}[5]{
\begin{tcolorbox}[center, boxrule=0.1 mm, width=\mywidth,arc=0mm,enhanced jigsaw,breakable,colback=yellow!5]	
	
	\footnotesize
	\textbf{Øvingsområde}\\[5pt] #1 
	
	\textbf{Utstyr}\\ #2  \\
	
	\begin{tabular}{@{} p{4cm} p{4cm} l} 
		\textbf{Tid} & \textbf{Elevinndeling} & \textbf{Læringsarena} \\
		#3  & #4 & #5
	\end{tabular} 
\end{tcolorbox}	
}

\newcommand{\gjen}[1]{\begin{tcolorbox}[center,boxrule=0.1 mm, width=\mywidth,arc=0mm,colback=blue!3] {\large \textbf{Gjennomføring} \vspace{5 pt}}\newline #1  \end{tcolorbox}\vspace{-5pt}}
\newcommand{\eks}[1]{\begin{tcolorbox}[center,boxrule=0.1 mm, width=\mywidth,arc=0mm,colback=green!3] {\large \textbf{Eksempel} \vspace{5 pt}}\newline #1  \end{tcolorbox}\vspace{-5pt}}

\newcounter{opl}
%\numberwithin{opl}{article}


\newcommand{\opl}[1]{
\newpage
{\refstepcounter{opl} %\phantomsection 
\large \textbf{\theopl \;#1} \vsk}
}

% Headlines
\newcommand{\fork}{\textbf{Forkunnskapar}\\}
\newcommand{\forb}{\textbf{Forberedelsar}\\}
\newcommand{\opgvr}{\textbf{Oppgaver}}



%colors
\newcommand{\colr}[1]{{\color{red} #1}}
\newcommand{\colb}[1]{{\color{blue} #1}}
\newcommand{\colo}[1]{{\color{orange} #1}}
\newcommand{\colc}[1]{{\color{cyan} #1}}
\definecolor{projectgreen}{cmyk}{100,0,100,0}
\newcommand{\colg}[1]{{\color{projectgreen} #1}}

% Lister med bokstavar
\usepackage[inline]{enumitem}
% Opg
\newcommand{\abc}[1]{
	\begin{enumerate}[label=\alph*),leftmargin=18pt]
		#1
	\end{enumerate}
}

\usepackage[]{hyperref}

\begin{document}
\eks[2]{
	Finn lengda verdien til $ x $ i trekanten under:
	\fig{tri26i} \vs \vs
	
	\sv
	
	Vi veit at:
	\[ c^2=a^2+x^2 \]
	hvor $ c $ er lengden til den lengste siden og $ a $ lengden til den andre kortsiden. Derfor få vi at:
	\alg{
		17^2 &= 13^2 + x^2 \\
		289-225&=x^2  \\
		64&=x^2
	}
	Da $ \sqrt{64}=8 $, må lengden til $ x $ vere.
}
\newpage
Med denne trekanten som utgangspunkt dannar vi eit rektangel:
%\fig{tri20a_old}
Vi bruker no dei same namna på areala til dei farga trekantane som vi brukte på side \pageref{arnames}. Da er
\alg{
	R=4\cdot 5=20\quad,\quad
	O=\frac{4\cdot 3}{2}=6\quad,\quad
	G=\frac{2\cdot4}{2}=4
}
No har vi at
\algv{
	B &= R-O-G \\
	&=20-6-4 \\
	&=10
}

\fork{\ref{360} \firsum}{
	Inni firkanten teiknar vi eit punkt $ E $, som er slik at vi kan teikne ein strek fra kvart hjørne og til $ E $, utan å krysse sidene i firkanten. No har vi dei fire trekantane $ \triangle ABE $, $ \triangle BCE $, $ \triangle CDE $ og $ \triangle DEA $.
	\fig{kant6a_old}
	Av \rref{180} veit vi at summen av vinkelverdiane til desse trekantane er $ 4\cdot 180^\circ = 720^\circ $. I tillegg er
	\[  \angle AEB+\angle BEC+\angle CED+\angle DEA=360^\circ \]
	Altså utgjer dei andre vinklane i dei fire trekantane $ {720^\circ-360^\circ =360^\circ} $, og desse vinklane er $ \angle BAD $, $ \angle CBA $, $ \angle DCB $ og $ \angle ADC $.}
\end{document}

