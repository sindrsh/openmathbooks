\documentclass[english,hidelinks,pdftex, 11 pt, class=report,crop=false]{standalone}
\usepackage[T1]{fontenc}
\usepackage[utf8]{luainputenc}
\usepackage{lmodern} % load a font with all the characters
\usepackage{geometry}
\geometry{verbose,a4paper, inner=0cm, outer=0 cm, bmargin=2cm, tmargin=1cm}
%\textwidth=12cm
\setlength{\parindent}{0bp}
\usepackage{import}
\usepackage[subpreambles=false]{standalone}
\usepackage{amsmath}
\usepackage{amssymb}
\usepackage{esint}
\usepackage{babel}
\usepackage{tabu}
\usepackage[dvipsnames, table]{xcolor}
\usepackage{cancel}
\makeatother
\makeatletter
\usepackage{datetime2}
\usepackage{titlesec}
\usepackage[many]{tcolorbox}

% Eheter
\newcommand{\enh}[1]{\,\textrm{#1}}
%referances
\newcommand{\net}[2]{{\color{blue}\href{#1}{#2}}}

%Spaces
\newcommand{\vsk}{\\[12pt]}
\newcommand{\vs}{\vspace{-12pt}}

% Tabell for opplegg

\newcommand{\ovlist}[1]{
\vspace{-16pt}
\begin{itemize}
	#1
\end{itemize}
}

% Chapters and sections
\titleformat{\section}[block]{\bfseries}{\hspace{3cm}\thesection}{5pt}{}
\titleformat{\subsection}[block]{\bfseries}{\hspace{3cm}\thesection}{5pt}{}
\newcommand{\sectionbreak}{\clearpage} % New page on each section
 

\newlength{\mywidth}
\setlength{\mywidth}{14cm}

\newcommand{\cont}[1]{
\begin{tcolorbox}[center, boxrule=0.0 mm, width=\mywidth,arc=0mm,enhanced jigsaw,,colback=white,breakable]
#1	
\end{tcolorbox}
}

\newcommand{\info}[5]{
\begin{tcolorbox}[center, boxrule=0.1 mm, width=\mywidth,arc=0mm,enhanced jigsaw,breakable,colback=yellow!5]	
	
	\footnotesize
	\textbf{Øvingsområde}\\[5pt] #1 
	
	\textbf{Utstyr}\\ #2  \\
	
	\begin{tabular}{@{} p{4cm} p{4cm} l} 
		\textbf{Tid} & \textbf{Elevinndeling} & \textbf{Læringsarena} \\
		#3  & #4 & #5
	\end{tabular} 
\end{tcolorbox}	
}

\newcommand{\gjen}[1]{\begin{tcolorbox}[center,boxrule=0.1 mm, width=\mywidth,arc=0mm,colback=blue!3] {\large \textbf{Gjennomføring} \vspace{5 pt}}\newline #1  \end{tcolorbox}\vspace{-5pt}}
\newcommand{\eks}[1]{\begin{tcolorbox}[center,boxrule=0.1 mm, width=\mywidth,arc=0mm,colback=green!3] {\large \textbf{Eksempel} \vspace{5 pt}}\newline #1  \end{tcolorbox}\vspace{-5pt}}

\newcounter{opl}
%\numberwithin{opl}{article}


\newcommand{\opl}[1]{
\newpage
{\refstepcounter{opl} %\phantomsection 
\large \textbf{\theopl \;#1} \vsk}
}

% Headlines
\newcommand{\fork}{\textbf{Forkunnskapar}\\}
\newcommand{\forb}{\textbf{Forberedelsar}\\}
\newcommand{\opgvr}{\textbf{Oppgaver}}



%colors
\newcommand{\colr}[1]{{\color{red} #1}}
\newcommand{\colb}[1]{{\color{blue} #1}}
\newcommand{\colo}[1]{{\color{orange} #1}}
\newcommand{\colc}[1]{{\color{cyan} #1}}
\definecolor{projectgreen}{cmyk}{100,0,100,0}
\newcommand{\colg}[1]{{\color{projectgreen} #1}}

% Lister med bokstavar
\usepackage[inline]{enumitem}
% Opg
\newcommand{\abc}[1]{
	\begin{enumerate}[label=\alph*),leftmargin=18pt]
		#1
	\end{enumerate}
}

\usepackage[]{hyperref}

\begin{document}

\fork{}{
	Vi tar utgangspunkt i trekantene $ \triangle ABC $ og $ \triangle DEF $ der
	\begin{equation}
	\dfrac{AC}{DF}=\dfrac{BC}{EF} \label{vilkforma}
	\end{equation}
	og $ {\angle C =\angle F} $. Altså kan vi plassere $ \triangle DEF $ inn i $ \triangle ABC $, slik at $ C $ og $ F $ blir sammenfallende punkt. Av
	\begin{figure}
		\centering
		\includegraphics[]{geo12}
	\end{figure}
	Vi setter $ a=BC $, $ a'=CE $, $ b=AC $ og $ b'=DC $. Av \eqref{vilkforma} har vi no at
	\begin{equation}
	\frac{a}{b}=\frac{a'}{b'} \label{vilkformb}
	\end{equation}
	$ \triangle AGD\sim\triangle AHC $, altså er
	\alg{
		\frac{CH}{AC}=\frac{GD}{AD} \br
		CH&=\frac{GD\cdot AC}{AD}\br
		&=\frac{GD\cdot b}{b-b'}
	}
	$ \triangle HBC\sim\triangle JBE $, altså er
	\alg{
		\frac{CH}{BC}&=\frac{EJ}{BE}\br
		CH&=\frac{JE\cdot BC}{BE} \br
		&=\frac{JE\cdot a}{a-a'}
	}
	Vi sett dei to uttrykkene for $ CH $ lik hverandre:
	\alg{
		\frac{GD\cdot b}{b-b'}=\frac{JE\cdot a}{a-a'}
	}	
	Av \eqref{vilkformb} har vi at $ a'=\frac{ab'}{b} $ og $ b'=\frac{a'b}{a} $. Altså er
	\alg{
		\frac{GD\cdot b}{b-\frac{a'b}{a}}=\frac{JE\cdot a}{a-\frac{ab'}{b}} \br
		\frac{GD}{1-\frac{a'}{a}}=\frac{JE}{1-\frac{b'}{b}} 
	}
	Av \eqref{vilkformb} har vi også at $ {\frac{a'}{a}=\frac{b'}{b}}$, altså er $ GD $ og $ JE $ dividert med den samme verdien i likninga over. Dette betyr at $ {GD=JE} $, dermed er $ DE\parallel AB $, og da er $ \triangle ABC \sim \triangle DEF$.
} \vsk

\fork{}{\textbf{Vilkår iii} \os
	Gitt to lengder $ b $ og $ c $ og ein vinkel $ u $. Vi startar med følgande:
	\begin{enumerate}
		\item Vi lagar eit linjestykke $ AB $ med lengde $ c $.
		\item I $ A $ teiknar vi ein halvsirkel med radius $ b $. 
	\end{enumerate}
	Ved å la $ C $ vere plassert kor som helst på denne sirkelbua, har vi alle moglege variantar av ein trekant $ \triangle ABC $ med sidelengdene $ {AB=c} $ og $ {AC=b} $. Å plassere $ C $ langs boga til halvsirkelen er det same som å gi $ \angle A $ ein bestemt verdi. Det gjenstår no å vise at kvar plassering av $ C $ gir ei unik lengde av $ BC $.
	\fig{geo13b}
	Vi let $ C_1 $ og $ C_2 $ vere to potensielle plasseringar av $ C $, der $ C_2 $ langs halvsirkelen ligg nærare $ E $ (sjå figur over)  enn $ C_1 $. I $ C_1 $ stiplar vi ei sirkelboge med radius $ BC_1 $ og sentrum i $ B $. Da den stipla sirkelboga og halvsirkelen berre kan skjære kvarandre i $ C_1 $, vil alle andre punkt på halvsirkelen ligge enten innanfor eller utanfor den stipla sirkelboga. Slik vi har definert $ C_2 $, må dette punktet ligge utanfor den stipla sirkelboga, og dermed er $ BC_2 $ større enn $ BC_1 $. Av dette kan vi konkludere med at $ BC $ blir lengre dess nærare $ C $ beveger seg mot $ E $ langs halvsirkelen. Å sette $ {\angle A=u} $ vil altså gi ein unik verdi for $ BC $, og da ein unik trekant $ \triangle ABC $ der  $ AC=b $, $ c=AB $ og $ \angle BAC=u $.
	
}
\end{document}

