\documentclass[english,hidelinks,pdftex, 11 pt, class=report,crop=false]{standalone}
\usepackage[T1]{fontenc}
\usepackage[utf8]{luainputenc}
\usepackage{lmodern} % load a font with all the characters
\usepackage{geometry}
\geometry{verbose,a4paper, inner=0cm, outer=0 cm, bmargin=2cm, tmargin=1cm}
%\textwidth=12cm
\setlength{\parindent}{0bp}
\usepackage{import}
\usepackage[subpreambles=false]{standalone}
\usepackage{amsmath}
\usepackage{amssymb}
\usepackage{esint}
\usepackage{babel}
\usepackage{tabu}
\usepackage[dvipsnames, table]{xcolor}
\usepackage{cancel}
\makeatother
\makeatletter
\usepackage{datetime2}
\usepackage{titlesec}
\usepackage[many]{tcolorbox}

% Eheter
\newcommand{\enh}[1]{\,\textrm{#1}}
%referances
\newcommand{\net}[2]{{\color{blue}\href{#1}{#2}}}

%Spaces
\newcommand{\vsk}{\\[12pt]}
\newcommand{\vs}{\vspace{-12pt}}

% Tabell for opplegg

\newcommand{\ovlist}[1]{
\vspace{-16pt}
\begin{itemize}
	#1
\end{itemize}
}

% Chapters and sections
\titleformat{\section}[block]{\bfseries}{\hspace{3cm}\thesection}{5pt}{}
\titleformat{\subsection}[block]{\bfseries}{\hspace{3cm}\thesection}{5pt}{}
\newcommand{\sectionbreak}{\clearpage} % New page on each section
 

\newlength{\mywidth}
\setlength{\mywidth}{14cm}

\newcommand{\cont}[1]{
\begin{tcolorbox}[center, boxrule=0.0 mm, width=\mywidth,arc=0mm,enhanced jigsaw,,colback=white,breakable]
#1	
\end{tcolorbox}
}

\newcommand{\info}[5]{
\begin{tcolorbox}[center, boxrule=0.1 mm, width=\mywidth,arc=0mm,enhanced jigsaw,breakable,colback=yellow!5]	
	
	\footnotesize
	\textbf{Øvingsområde}\\[5pt] #1 
	
	\textbf{Utstyr}\\ #2  \\
	
	\begin{tabular}{@{} p{4cm} p{4cm} l} 
		\textbf{Tid} & \textbf{Elevinndeling} & \textbf{Læringsarena} \\
		#3  & #4 & #5
	\end{tabular} 
\end{tcolorbox}	
}

\newcommand{\gjen}[1]{\begin{tcolorbox}[center,boxrule=0.1 mm, width=\mywidth,arc=0mm,colback=blue!3] {\large \textbf{Gjennomføring} \vspace{5 pt}}\newline #1  \end{tcolorbox}\vspace{-5pt}}
\newcommand{\eks}[1]{\begin{tcolorbox}[center,boxrule=0.1 mm, width=\mywidth,arc=0mm,colback=green!3] {\large \textbf{Eksempel} \vspace{5 pt}}\newline #1  \end{tcolorbox}\vspace{-5pt}}

\newcounter{opl}
%\numberwithin{opl}{article}


\newcommand{\opl}[1]{
\newpage
{\refstepcounter{opl} %\phantomsection 
\large \textbf{\theopl \;#1} \vsk}
}

% Headlines
\newcommand{\fork}{\textbf{Forkunnskapar}\\}
\newcommand{\forb}{\textbf{Forberedelsar}\\}
\newcommand{\opgvr}{\textbf{Oppgaver}}



%colors
\newcommand{\colr}[1]{{\color{red} #1}}
\newcommand{\colb}[1]{{\color{blue} #1}}
\newcommand{\colo}[1]{{\color{orange} #1}}
\newcommand{\colc}[1]{{\color{cyan} #1}}
\definecolor{projectgreen}{cmyk}{100,0,100,0}
\newcommand{\colg}[1]{{\color{projectgreen} #1}}

% Lister med bokstavar
\usepackage[inline]{enumitem}
% Opg
\newcommand{\abc}[1]{
	\begin{enumerate}[label=\alph*),leftmargin=18pt]
		#1
	\end{enumerate}
}

\usepackage[]{hyperref}

\newcommand{\note}{Merk}
\newcommand{\notesm}[1]{{\footnotesize \textsl{\note:} #1}}
\newcommand{\ekstitle}{Eksempel }
\newcommand{\sprtitle}{Språkboksen}
\newcommand{\expl}{forklaring}
\newcommand{\pyt}{Pytagoras' setning}
\newcommand\sv{\vsk \textbf{Svar} \vspace{4 pt}\\}

%references
\newcommand{\reftab}[1]{\hrs{#1}{tabell}}
\newcommand{\rref}[1]{\hrs{#1}{regel}}
\newcommand{\dref}[1]{\hrs{#1}{definisjon}}
\newcommand{\refkap}[1]{\hrs{#1}{kapittel}}
\newcommand{\refsec}[1]{\hrs{#1}{seksjon}}
\newcommand{\refdsec}[1]{\hrs{#1}{delseksjon}}
\newcommand{\refved}[1]{\hrs{#1}{vedlegg}}
\newcommand{\eksref}[1]{\textsl{#1}}
\newcommand\fref[2][]{\hyperref[#2]{\textsl{figur \ref*{#2}#1}}}
\newcommand{\refop}[1]{{\color{blue}Oppgave \ref{#1}}}
\newcommand{\refops}[1]{{\color{blue}oppgave \ref{#1}}}


%Algebra
\newcommand{\kvadset}{Kvadratsetningene}
\newcommand{\aenato}{Sum-produkt-metoden}

% Geometry
\newcommand{\hlikb}{Midtnormalen i en likebeint trekant}
\newcommand{\arealsetn}{Arealsetningen}
\newcommand{\trkmedian}{Median}
\newcommand{\midtrk}{Midtnormal (i trekant)}
\newcommand{\innskrsirk}{Innskrevet sirkel}
\newcommand{\cossetn}{Cosinussetningen}
\newcommand{\perfvink}{Sentral- og periferivinkel}
\newcommand{\tang}{Tangent}

% Derivative
\newcommand{\derel}{Den deriverte av elementære funksjoner}
\newcommand{\divder}{Divisjonsregelen}
\newcommand{\kjernereg}{Kjerneregelen}
\newcommand{\prodregder}{Produktregelen}
\newcommand{\lhop}{L'Hopitals regel}

% Funksjonsdrofting
\newcommand{\monder}{Monotoniegenskaper og den deriverte}
\newcommand{\fderekstr}{$ \bm{f'=0} $ for lokale ektstremalpunkt}
\newcommand{\andredertest}{Andrederiverttesten}

% Vectors
\newcommand{\detar}{Arealformler med determinanter}
\newcommand{\avstpunktlin}{Avstand mellom punkt og linje}

%Appendix
\newcommand{\rolle}{Rolles teorem}
\newcommand{\meanval}{Middelverdisetningen}

% Solutions manual
\newcommand{\selos}{Se løsningsforslag.}


\begin{document}
\footnotesize
\opgt 	

\grubr{opglikbmidtn}
\fig{opglikbmidtnlos}
Vi lar $ D $ være punktet der halveringslinja til $ \angle ACB $ skjærer $ AB $. $ \triangle DAC\cong \triangle DBC $ fordi de har $ CD $ felles og $ AC=BC $ (trekantene oppfyller altså vilkår iii for formlikhet, og må da være kongruente). Følgelig er $ \angle BDA=\angle ADC $, og da er $ 2\angle DBA=180^\circ $. Altså er $ \angle DBA=90^\circ $, og da $ AD=BD $, ligger $ DC $ på midtnormalen til $ AB $.

\grubr{opggeoqrst}
\fig{opggeoqrst_lf}	
$ {\triangle EFC \sim \triangle DFB} $ fordi begge er rettvinklede, og $ {\angle CFE = \angle BFD}$ (de er toppvinkler). Dermed har vi at
\begin{equation}\label{opggeoqrsteq1}
	\frac{EF}{CE}=\frac{FD}{BD} 
\end{equation}
Videre er
\begin{equation}\label{opggeoqrsteq2}
	EF+FD= AD-AE
\end{equation}
Ved å løse likningssettet vi får av \eqref{opggeoqrsteq1} og \eqref{opggeoqrsteq2}, med hensyn på $ EF $ og $ ED $, får vi at
\[ 
EF = \frac{AD-AE}{CE+BD}CE\qquad,\qquad  FD=\frac{AD-AE}{CE+BD}BD
\]
Det doble arealet til $ \triangle ABC $ er gitt som
\begin{multline*}
(AE+EF)CE+(AD-FD)BD \\=\left(AE+\frac{AD-AE}{CE+BD}CE\right)CE+\left(AD-\frac{AD-AE}{CE+BD}BC\right)BD	
\end{multline*}
\alg{
&=\frac{1}{CE+BD}\left[\left(AE\cdot BD+AD\cdot CE\right)CE+\left(AD\cdot CE+AE\cdot BD\right)BD\right]\br
&=AD\cdot CE+ AE\cdot BD
}

\newpage
\grubr{opggeolikar}
\fig{opggeolikarlos}
Av å legge merke til trekanter med grunnlinje og høgde av lik lengde, finner vi at
\alg{
A_{\triangle AFE}=A_{\triangle FBE} &&
A_{\triangle AIE}=A_{\triangle EDI}\vn
A_{\triangle BCE}=A_{\triangle GCE}&&
A_{\triangle HDE}=A_{\triangle HCE}
}
Følgelig er
\alg{
(A_{\triangle AFE} + A_{\triangle AIE})+(A_{\triangle BCE}+A_{\triangle HDE}) &= (A_{\triangle FBE}+A_{\triangle EDI})+ (A_{\triangle GCE}+A_{\triangle HCE}) \\
A_{\square AFEI} + A_{\square GCHE} &= A_{\square FBGE} + A_{\square DIEH}
}
Altså er arealet til det blåfargede området er det samme som arealet til det grønnfargede området.
\newpage

\grubr{opggeo1530}
\abc{ 
	\item \mbox{}
	\fig{opggeosin15invlos}
	Med hensyn på vinkelsummen i $ \triangle ABC $ har vi at $ \angle ACB= 90-15^\circ=75^\circ$. Vi lar $ D $ være punktet på $ AB $ slik at $ \angle ACD=15^\circ $. Da er $ \angle CDA =75^\circ$ og $ \angle DCE=60^\circ $. Videre lar vi $ E $ være punktet på $ BC $ slik at $ CD=CE $, da er $ \triangle CDE $ likesidet. Vi setter $ s=CD $, $ a= BC $, $ b=AC $ og $ c=AB $. $ \triangle ABC\sim \triangle ACD $ fordi begge er rettvinklede, og $ \angle ACD=\angle ABC$. Dermed er
	\alg{
		AD=AC\frac{AC}{AB}=\frac{b^2}{c} \vn
		s=BC\frac{AC}{AB}=\frac{ab}{c}
	}
	Med hensyn på vinkelsummen i $ \triangle CBD $ er $ \angle BDC=180^\circ-15^\circ-60^\circ=105^\circ $, og da er $ \angle FDE=45^\circ $. Altså er $ \triangle DFE $ rettvinklet og likebeint, som betyr at $ DF=FE=\frac{s}{\sqrt{2}} $. Da $ \triangle ABC\sim FBE$, er $ \triangle ACD\sim \triangle FBE $, og dermed er
	\begin{flalign*}
		&&	EF\cdot CD &= AD\cdot EB \\
		&&	\frac{1}{\sqrt{2}}\left(\frac{ab}{c}\right)^2&=\frac{b^2}{c}\left(a-\frac{ab}{c}\right) &&(a,b\neq0) \\
		&& a&=c\sqrt{2}-b\sqrt{2}
	\end{flalign*}
	Av \pyt\ med hensyn på $ \triangle ABC$ har vi at $ c^2=a^2-b^2 $, og følgelig er
	\alg{
		a&=\sqrt{2}\sqrt{a^2-b^2}-b\sqrt{2} \\
		a+b\sqrt{2}&=\sqrt{2}\sqrt{a^2-b^2} \\
		a^2+2ab\sqrt{2}+2b^2&= 2(a^2-b^2) \\
		-a^2+2ab\sqrt{2}+4b^2&=0
	}
	Av \textit{abc}-formelen har vi at
	\alg{
		a&=\frac{-2b\sqrt{2}\pm \sqrt{8b^2+16b^2}}{-2} \\
		&= \left(\sqrt{2}\mp \sqrt{6}\right)b
	}
	Vi forkaster den negative løsningen for $ a $, og får at
	\nn{
		\frac{a}{b}=\sqrt{2}+\sqrt{6}
	}
\newpage
	\item \mbox{}
	\fig{opggeo1530}
	$ A_{\triangle DBC}=A_{\triangle ADC} $ fordi med henholdsvis $ DB $ og $ AD $ som grunnlinje har de lik høgde, og $ DB=AD $. Altså er $ AF\cdot DC = EB\cdot DC $, og da er $ AF=EB $. Videre er $ \triangle DAF\cong \triangle DBE $ fordi begge er rettvinklede $ \angle ADF =\angle BDE$ (de er toppvinkler), og $ AD=DB $. Vi setter $ x=DE $, $ a=EB $ og $ b=AC $. Da $ \triangle BCE $ er en $ 30^\circ $, $ 60^\circ $, $ 90^\circ $ trekant, er $ EC=\sqrt{3}a $ og $ BC=2a $. Da $ \triangle BGC $ er en $ 45^\circ $, $ 45^\circ $ , $ 90^\circ $ trekant, er $ GB=\frac{2}{\sqrt{2}}a $. Da $ A_{\triangle ABC}=2A_{\triangle DBC} $, har vi at
	\alg{
		b\cdot \frac{2}{\sqrt{2}}a &= 2(\sqrt{3}a+x)\cdot a \\
		b &= \sqrt{2}(\sqrt{3}a+x)
	}
	Av løsningen i oppgave a) har vi at $ {AC=(\sqrt{2}+\sqrt{6})AF}$, og dermed er $ b=a\sqrt{2}(\sqrt{3}+1)$. Altså er $ x=a $, som betyr at $ \triangle AFD $ er en $ 45^\circ $, $ 45^\circ $, $ 90^\circ $ trekant. Ved å betrakte vinkelsummen i $ \triangle CAF $, finner vi da at
	\alg{
		\angle DAC &=180^\circ-15^\circ-90^\circ-45^\circ \\
		&=30^\circ
	}
	\textbf{Alternativ metode for å vise at \boldmath $ x=a $} \os
	Av Pytagoras' setning på $ \triangle ACD $ har vi at
	\alg{
		AC^2 &= FC^2 + AF \\
		2(\sqrt{3}a+x)^2 &= (\sqrt{3}a+2x)^2 + a^2 \\
		x^2 &= a^2
	}
	
}

\end{document}

