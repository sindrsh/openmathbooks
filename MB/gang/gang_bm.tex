\documentclass[english,hidelinks,pdftex, 11 pt, class=report,crop=false]{standalone}
\usepackage[T1]{fontenc}
\usepackage[utf8]{luainputenc}
\usepackage{lmodern} % load a font with all the characters
\usepackage{geometry}
\geometry{verbose,a4paper, inner=0cm, outer=0 cm, bmargin=2cm, tmargin=1cm}
%\textwidth=12cm
\setlength{\parindent}{0bp}
\usepackage{import}
\usepackage[subpreambles=false]{standalone}
\usepackage{amsmath}
\usepackage{amssymb}
\usepackage{esint}
\usepackage{babel}
\usepackage{tabu}
\usepackage[dvipsnames, table]{xcolor}
\usepackage{cancel}
\makeatother
\makeatletter
\usepackage{datetime2}
\usepackage{titlesec}
\usepackage[many]{tcolorbox}

% Eheter
\newcommand{\enh}[1]{\,\textrm{#1}}
%referances
\newcommand{\net}[2]{{\color{blue}\href{#1}{#2}}}

%Spaces
\newcommand{\vsk}{\\[12pt]}
\newcommand{\vs}{\vspace{-12pt}}

% Tabell for opplegg

\newcommand{\ovlist}[1]{
\vspace{-16pt}
\begin{itemize}
	#1
\end{itemize}
}

% Chapters and sections
\titleformat{\section}[block]{\bfseries}{\hspace{3cm}\thesection}{5pt}{}
\titleformat{\subsection}[block]{\bfseries}{\hspace{3cm}\thesection}{5pt}{}
\newcommand{\sectionbreak}{\clearpage} % New page on each section
 

\newlength{\mywidth}
\setlength{\mywidth}{14cm}

\newcommand{\cont}[1]{
\begin{tcolorbox}[center, boxrule=0.0 mm, width=\mywidth,arc=0mm,enhanced jigsaw,,colback=white,breakable]
#1	
\end{tcolorbox}
}

\newcommand{\info}[5]{
\begin{tcolorbox}[center, boxrule=0.1 mm, width=\mywidth,arc=0mm,enhanced jigsaw,breakable,colback=yellow!5]	
	
	\footnotesize
	\textbf{Øvingsområde}\\[5pt] #1 
	
	\textbf{Utstyr}\\ #2  \\
	
	\begin{tabular}{@{} p{4cm} p{4cm} l} 
		\textbf{Tid} & \textbf{Elevinndeling} & \textbf{Læringsarena} \\
		#3  & #4 & #5
	\end{tabular} 
\end{tcolorbox}	
}

\newcommand{\gjen}[1]{\begin{tcolorbox}[center,boxrule=0.1 mm, width=\mywidth,arc=0mm,colback=blue!3] {\large \textbf{Gjennomføring} \vspace{5 pt}}\newline #1  \end{tcolorbox}\vspace{-5pt}}
\newcommand{\eks}[1]{\begin{tcolorbox}[center,boxrule=0.1 mm, width=\mywidth,arc=0mm,colback=green!3] {\large \textbf{Eksempel} \vspace{5 pt}}\newline #1  \end{tcolorbox}\vspace{-5pt}}

\newcounter{opl}
%\numberwithin{opl}{article}


\newcommand{\opl}[1]{
\newpage
{\refstepcounter{opl} %\phantomsection 
\large \textbf{\theopl \;#1} \vsk}
}

% Headlines
\newcommand{\fork}{\textbf{Forkunnskapar}\\}
\newcommand{\forb}{\textbf{Forberedelsar}\\}
\newcommand{\opgvr}{\textbf{Oppgaver}}



%colors
\newcommand{\colr}[1]{{\color{red} #1}}
\newcommand{\colb}[1]{{\color{blue} #1}}
\newcommand{\colo}[1]{{\color{orange} #1}}
\newcommand{\colc}[1]{{\color{cyan} #1}}
\definecolor{projectgreen}{cmyk}{100,0,100,0}
\newcommand{\colg}[1]{{\color{projectgreen} #1}}

% Lister med bokstavar
\usepackage[inline]{enumitem}
% Opg
\newcommand{\abc}[1]{
	\begin{enumerate}[label=\alph*),leftmargin=18pt]
		#1
	\end{enumerate}
}

\usepackage[]{hyperref}

\newcommand{\note}{Merk}
\newcommand{\notesm}[1]{{\footnotesize \textsl{\note:} #1}}
\newcommand{\ekstitle}{Eksempel }
\newcommand{\sprtitle}{Språkboksen}
\newcommand{\expl}{forklaring}
\newcommand{\pyt}{Pytagoras' setning}
\newcommand\sv{\vsk \textbf{Svar} \vspace{4 pt}\\}

%references
\newcommand{\reftab}[1]{\hrs{#1}{tabell}}
\newcommand{\rref}[1]{\hrs{#1}{regel}}
\newcommand{\dref}[1]{\hrs{#1}{definisjon}}
\newcommand{\refkap}[1]{\hrs{#1}{kapittel}}
\newcommand{\refsec}[1]{\hrs{#1}{seksjon}}
\newcommand{\refdsec}[1]{\hrs{#1}{delseksjon}}
\newcommand{\refved}[1]{\hrs{#1}{vedlegg}}
\newcommand{\eksref}[1]{\textsl{#1}}
\newcommand\fref[2][]{\hyperref[#2]{\textsl{figur \ref*{#2}#1}}}
\newcommand{\refop}[1]{{\color{blue}Oppgave \ref{#1}}}
\newcommand{\refops}[1]{{\color{blue}oppgave \ref{#1}}}


%Algebra
\newcommand{\kvadset}{Kvadratsetningene}
\newcommand{\aenato}{Sum-produkt-metoden}

% Geometry
\newcommand{\hlikb}{Midtnormalen i en likebeint trekant}
\newcommand{\arealsetn}{Arealsetningen}
\newcommand{\trkmedian}{Median}
\newcommand{\midtrk}{Midtnormal (i trekant)}
\newcommand{\innskrsirk}{Innskrevet sirkel}
\newcommand{\cossetn}{Cosinussetningen}
\newcommand{\perfvink}{Sentral- og periferivinkel}
\newcommand{\tang}{Tangent}

% Derivative
\newcommand{\derel}{Den deriverte av elementære funksjoner}
\newcommand{\divder}{Divisjonsregelen}
\newcommand{\kjernereg}{Kjerneregelen}
\newcommand{\prodregder}{Produktregelen}
\newcommand{\lhop}{L'Hopitals regel}

% Funksjonsdrofting
\newcommand{\monder}{Monotoniegenskaper og den deriverte}
\newcommand{\fderekstr}{$ \bm{f'=0} $ for lokale ektstremalpunkt}
\newcommand{\andredertest}{Andrederiverttesten}

% Vectors
\newcommand{\detar}{Arealformler med determinanter}
\newcommand{\avstpunktlin}{Avstand mellom punkt og linje}

%Appendix
\newcommand{\rolle}{Rolles teorem}
\newcommand{\meanval}{Middelverdisetningen}

% Solutions manual
\newcommand{\selos}{Se løsningsforslag.}

\begin{document}
	
\section{\gong \label{Gonging} }

\subsection*{Ganging med heltall; innledende definisjon}
Når vi legger sammen like tall, kan vi bruke \outl{gangetegnet} \sym{$ \cdot $}\;for å skrive regnestykkene våre kortere: \regv
\eks[]{\vsb \vs
\alg{
4+4+4 &= 4\cdot 3 \vn
8+8 &=8\cdot 2 \vn
 1+1+1+1+1&= 1\cdot5 
}
} \regv
\spr{
	Et gangestykke består av to eller flere \outl{faktorer}\index{faktor} og ett \outl{produkt}\index{produkt}. I gangestykket
	\[ {4\cdot 3=12} \]
	er $ 4 $ og $ 3 $ faktorer, mens $ 12 $ er produktet. \vsk
	
	Vanlige måter å si $ 4\cdot3 $ på er
	\begin{itemize}
		\item ''4 ganger 3''  \\
		\item ''4 ganget med 3''\\
		\item ''4 multiplisert med 3''
	\end{itemize}
	
	Mange nettsteder og bøker på engelsk bruker symbolet \sym{$ \times $} i steden for \sym{$ \cdot $}. I de fleste programmeringsspråk er \sym{*} symbolet for multiplikasjon.
}
\subsection*{Ganging av mengder} \label{gangmengd}
La oss nå bruke en figur for å se for oss gangestykket $ 2\cdot3 $:
\fig{2t3}
Og så kan vi legge merke til produktet av $ 3\cdot 2 $:
\fig{3t2}
\reg[\gangkom \label{gangkom}]{
Produktet er det samme uansett rekkefølge på faktorene.
}
\eks[]{\vsb \vs
\alg{
3\cdot 4 &=12= 4\cdot 3 \vn
6\cdot 7 &=42= 7\cdot6 \vn
8\cdot 9 &=72= 9\cdot8
}
}

\subsection*{Ganging på tallinja}
Vi kan også bruke tallinja for å regne ut gangestykker. For eksempel kan vi finne hva $ 2\cdot4 $ er ved å tenke slik:
\[\text{''} 2\cdot 4 \text{ betyr å vandre 2 plasser \textsl{mot høyre}, 4 ganger.}\text{''} \]
\[ 2\cdot4=8 \]
\fig{2t4l}
Også tallinja kan vi bruke for å overbevise oss om at rekkefølgen i et gangestykke ikke har noe å si:
\[\text{''} 4\cdot 2 \text{ betyr å vandre 4 plasser \textsl{mot høyre}, 2 ganger.}\text{''} \]
\[ 4\cdot2=8 \]
\fig{4t2l}

\newpage
\subsection*{Endelig definisjon av ganging med positive heltall}
Det ligger kanskje nærmest å tolke ''2 ganger 3'' som ''3, 2 ganger''. Da er
\[ \text{''2 ganger 3''}=3+3 \] 
På side \pageref{gangmengd} presenterete vi $ {2\cdot3} $, altså ''2 ganger 3'', som $ {2+2+2} $. Med denne tolkningen vil $ {3+3} $ svare til $ {3\cdot2} $, men nettopp det at multiplikasjon er en kommutativ operasjon (\rref{gangkom}) gjør at den ene tolkningen ikke utelukker den andre; $ {2\cdot3 =2+2+2} $ og $ {2\cdot3=3+3} $ er to uttrykk med samme verdi.\regv

\reg[Ganging som gjentatt addisjon \label{ganggjad2}]{
Ganging med et positivt heltal kan uttrykkes som gjentatt addisjon.
}
\eks[1]{\vsb \vs
	\alg{
		4+4+4 &= 4\cdot 3=3+3+3+3 \vn
		8+8 &=8\cdot 2=2+2+2+2+2+2+2 \vn
		1+1+1+1+1&= 1\cdot5 =5
	}
}
\info{Merk}{
At ganging med positive heltal kan uttrykkes som gjentatt addisjon, utelukker ikke andre uttrykk. Det er ikke feil å skrive at $ {2\cdot 3=1+5} $.
} \vsk \vsk

\end{document}

