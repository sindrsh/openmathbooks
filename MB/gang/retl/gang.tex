\documentclass[english,hidelinks,pdftex, 11 pt, class=article,crop=false]{standalone}
\usepackage[T1]{fontenc}
\usepackage[utf8]{luainputenc}
\usepackage{lmodern} % load a font with all the characters
\usepackage{geometry}
\geometry{verbose,a4paper, inner=0cm, outer=0 cm, bmargin=2cm, tmargin=1cm}
%\textwidth=12cm
\setlength{\parindent}{0bp}
\usepackage{import}
\usepackage[subpreambles=false]{standalone}
\usepackage{amsmath}
\usepackage{amssymb}
\usepackage{esint}
\usepackage{babel}
\usepackage{tabu}
\usepackage[dvipsnames, table]{xcolor}
\usepackage{cancel}
\makeatother
\makeatletter
\usepackage{datetime2}
\usepackage{titlesec}
\usepackage[many]{tcolorbox}

% Eheter
\newcommand{\enh}[1]{\,\textrm{#1}}
%referances
\newcommand{\net}[2]{{\color{blue}\href{#1}{#2}}}

%Spaces
\newcommand{\vsk}{\\[12pt]}
\newcommand{\vs}{\vspace{-12pt}}

% Tabell for opplegg

\newcommand{\ovlist}[1]{
\vspace{-16pt}
\begin{itemize}
	#1
\end{itemize}
}

% Chapters and sections
\titleformat{\section}[block]{\bfseries}{\hspace{3cm}\thesection}{5pt}{}
\titleformat{\subsection}[block]{\bfseries}{\hspace{3cm}\thesection}{5pt}{}
\newcommand{\sectionbreak}{\clearpage} % New page on each section
 

\newlength{\mywidth}
\setlength{\mywidth}{14cm}

\newcommand{\cont}[1]{
\begin{tcolorbox}[center, boxrule=0.0 mm, width=\mywidth,arc=0mm,enhanced jigsaw,,colback=white,breakable]
#1	
\end{tcolorbox}
}

\newcommand{\info}[5]{
\begin{tcolorbox}[center, boxrule=0.1 mm, width=\mywidth,arc=0mm,enhanced jigsaw,breakable,colback=yellow!5]	
	
	\footnotesize
	\textbf{Øvingsområde}\\[5pt] #1 
	
	\textbf{Utstyr}\\ #2  \\
	
	\begin{tabular}{@{} p{4cm} p{4cm} l} 
		\textbf{Tid} & \textbf{Elevinndeling} & \textbf{Læringsarena} \\
		#3  & #4 & #5
	\end{tabular} 
\end{tcolorbox}	
}

\newcommand{\gjen}[1]{\begin{tcolorbox}[center,boxrule=0.1 mm, width=\mywidth,arc=0mm,colback=blue!3] {\large \textbf{Gjennomføring} \vspace{5 pt}}\newline #1  \end{tcolorbox}\vspace{-5pt}}
\newcommand{\eks}[1]{\begin{tcolorbox}[center,boxrule=0.1 mm, width=\mywidth,arc=0mm,colback=green!3] {\large \textbf{Eksempel} \vspace{5 pt}}\newline #1  \end{tcolorbox}\vspace{-5pt}}

\newcounter{opl}
%\numberwithin{opl}{article}


\newcommand{\opl}[1]{
\newpage
{\refstepcounter{opl} %\phantomsection 
\large \textbf{\theopl \;#1} \vsk}
}

% Headlines
\newcommand{\fork}{\textbf{Forkunnskapar}\\}
\newcommand{\forb}{\textbf{Forberedelsar}\\}
\newcommand{\opgvr}{\textbf{Oppgaver}}



%colors
\newcommand{\colr}[1]{{\color{red} #1}}
\newcommand{\colb}[1]{{\color{blue} #1}}
\newcommand{\colo}[1]{{\color{orange} #1}}
\newcommand{\colc}[1]{{\color{cyan} #1}}
\definecolor{projectgreen}{cmyk}{100,0,100,0}
\newcommand{\colg}[1]{{\color{projectgreen} #1}}

% Lister med bokstavar
\usepackage[inline]{enumitem}
% Opg
\newcommand{\abc}[1]{
	\begin{enumerate}[label=\alph*),leftmargin=18pt]
		#1
	\end{enumerate}
}

\usepackage[]{hyperref}
\input{/home/sindre/G/preamb_opl}
\input{/home/sindre/G/delmal}



\begin{document}
\tbx{
\textbf{Essensen med ganging}\bs
Det er få tema som er så avgjerande å forstå som ganging, og dei to heilt sentrale aspekta ein elev må ha med seg er at

\begin{itemize}
	\item ganging med heiltal er ein forkorta skrivemåte\footnote{Eigentleg kan ein hevde at også ganging med desimaltal er gjentatt addisjon, men det krev ei noko invikla forklaring.} av gjentatt addisjon
	\item $ a\cdot b=b\cdot a $ (kommutativ lov)
\end{itemize}
\textbf{Gangetabellen}\bs
Å ha automatisert gangetabellen er noko elevar vil ha stor nytte av spesielt i vidare skulematematikk, men også i dagleg bruk etter endt skulegang. Samstundes kan eit sterkt fokus på innlæring av gangetabellen ta oppmerksemda vekk ifra dei to punkta gitt i forrige avsnitt. Slik kan elevar som har vanskar med å memorisere skape seg eit bilete av at hvis dei ikkje husker kva $ {6\cdot7} $ er, så er dette reknestykket uløyseleg. Det er derfor viktig at ein i arbeidet med ganging heile tida trenar på strategiar for å komme fram til gangstykker ein enda ikkje har automatisert. Til dømes
\begin{center}
	Kanskje veit eleven kva $ {6\cdot 5} $ er? Viss eleven da har oppnådd ei god forståing for gjentatt addisjon, kan hen finne at\\ $ 6\cdot7=6\cdot5+6+6=42 $
\end{center} 
Merk også at å lære den lille gangetabellen (ganging med tala 1 - 9) inneber å memorisere 100 tal:
\begin{figure}
	\centering
	\includegraphics[scale=0.2]{gngtb}
\end{figure}
Men, hvis eleven har ei god forståing av kommutativ lov, kokast dette ned til 55 tal:
\begin{figure}
	\centering
	\includegraphics[scale=0.2]{gngtb2}
\end{figure}
Korav 19 av desse kjem fra 1- eller 10-gongen.
}
\begin{comment}
	textbf{Ganging med fleirsifra tal}\bs
	Å kunne rekne ut fleirsifra tal ganga saman er ein ferdigheit som omlag utelukkande blir brukt innanfor skulematematikk\footnote{i det daglege vil ein løyse slike reknestykke enten ved å bruke kalkulator eller ved å gjere eit overslag.}, og viktigheita 
\end{comment}
\end{document}