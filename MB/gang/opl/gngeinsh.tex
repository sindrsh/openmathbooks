\documentclass[english,hidelinks,pdftex, 11 pt, class=article,crop=false]{standalone}
\usepackage[T1]{fontenc}
\usepackage[utf8]{luainputenc}
\usepackage{lmodern} % load a font with all the characters
\usepackage{geometry}
\geometry{verbose,a4paper, inner=0cm, outer=0 cm, bmargin=2cm, tmargin=1cm}
%\textwidth=12cm
\setlength{\parindent}{0bp}
\usepackage{import}
\usepackage[subpreambles=false]{standalone}
\usepackage{amsmath}
\usepackage{amssymb}
\usepackage{esint}
\usepackage{babel}
\usepackage{tabu}
\usepackage[dvipsnames, table]{xcolor}
\usepackage{cancel}
\makeatother
\makeatletter
\usepackage{datetime2}
\usepackage{titlesec}
\usepackage[many]{tcolorbox}

% Eheter
\newcommand{\enh}[1]{\,\textrm{#1}}
%referances
\newcommand{\net}[2]{{\color{blue}\href{#1}{#2}}}

%Spaces
\newcommand{\vsk}{\\[12pt]}
\newcommand{\vs}{\vspace{-12pt}}

% Tabell for opplegg

\newcommand{\ovlist}[1]{
\vspace{-16pt}
\begin{itemize}
	#1
\end{itemize}
}

% Chapters and sections
\titleformat{\section}[block]{\bfseries}{\hspace{3cm}\thesection}{5pt}{}
\titleformat{\subsection}[block]{\bfseries}{\hspace{3cm}\thesection}{5pt}{}
\newcommand{\sectionbreak}{\clearpage} % New page on each section
 

\newlength{\mywidth}
\setlength{\mywidth}{14cm}

\newcommand{\cont}[1]{
\begin{tcolorbox}[center, boxrule=0.0 mm, width=\mywidth,arc=0mm,enhanced jigsaw,,colback=white,breakable]
#1	
\end{tcolorbox}
}

\newcommand{\info}[5]{
\begin{tcolorbox}[center, boxrule=0.1 mm, width=\mywidth,arc=0mm,enhanced jigsaw,breakable,colback=yellow!5]	
	
	\footnotesize
	\textbf{Øvingsområde}\\[5pt] #1 
	
	\textbf{Utstyr}\\ #2  \\
	
	\begin{tabular}{@{} p{4cm} p{4cm} l} 
		\textbf{Tid} & \textbf{Elevinndeling} & \textbf{Læringsarena} \\
		#3  & #4 & #5
	\end{tabular} 
\end{tcolorbox}	
}

\newcommand{\gjen}[1]{\begin{tcolorbox}[center,boxrule=0.1 mm, width=\mywidth,arc=0mm,colback=blue!3] {\large \textbf{Gjennomføring} \vspace{5 pt}}\newline #1  \end{tcolorbox}\vspace{-5pt}}
\newcommand{\eks}[1]{\begin{tcolorbox}[center,boxrule=0.1 mm, width=\mywidth,arc=0mm,colback=green!3] {\large \textbf{Eksempel} \vspace{5 pt}}\newline #1  \end{tcolorbox}\vspace{-5pt}}

\newcounter{opl}
%\numberwithin{opl}{article}


\newcommand{\opl}[1]{
\newpage
{\refstepcounter{opl} %\phantomsection 
\large \textbf{\theopl \;#1} \vsk}
}

% Headlines
\newcommand{\fork}{\textbf{Forkunnskapar}\\}
\newcommand{\forb}{\textbf{Forberedelsar}\\}
\newcommand{\opgvr}{\textbf{Oppgaver}}



%colors
\newcommand{\colr}[1]{{\color{red} #1}}
\newcommand{\colb}[1]{{\color{blue} #1}}
\newcommand{\colo}[1]{{\color{orange} #1}}
\newcommand{\colc}[1]{{\color{cyan} #1}}
\definecolor{projectgreen}{cmyk}{100,0,100,0}
\newcommand{\colg}[1]{{\color{projectgreen} #1}}

% Lister med bokstavar
\usepackage[inline]{enumitem}
% Opg
\newcommand{\abc}[1]{
	\begin{enumerate}[label=\alph*),leftmargin=18pt]
		#1
	\end{enumerate}
}

\usepackage[]{hyperref}
\input{/home/sindre/G/preamb_opl}
\input{/home/sindre/G/delmal}



\begin{document}
\of{
\opl{Ganging}{Kast terning og fyll brettet}
\lstt{\textit{Med digitale hjelpemiddel:} \\ & OneNote/Showbie \\ & Explain Everything \\& 2-3 terningar per gruppe\\ & \\&
	\textit{Utan digitale hjelpemiddel:} \\&Utskrift av ruteark \\ & Fargeblyantar \\& 2-3 terningar per gruppe}
{45 min \,+}
{Grupper på 2-3}
{Inne}

\begin{enumerate}
	\item Kvar elev får utdelt eit $ 10\cdot10 $ ruteark. 
	\begin{center}
		\includegraphics[scale=0.075]{gong1}
	\end{center}
	Utskriftsversjon finn du \net{https://drive.google.com/open?id=1dV2pIzJ19Pkhm-y16nGIcZoXmIhfiHgY}{her}.\vsk
	
	\textit{Ved digitale hjelpemiddel til stades, del fila i OneNote/Showbie og få elevane til å laste den inn i Exlain Everything.}
	\item Elevane byttar på å kaste to terningar, desse dannar gangestykker som elevane skal teikne inn i sine ruteark.\vsk
	
	\textit{Eksempel:} I sin første runde får Ola 3 og 2 på sine to terningar. Han kan derfor teikne inn ein boks som er 2 ruter høg og 3 ruter brei, eller omvend, der han sjølv vil:
	\begin{center}
		\includegraphics[scale=0.075]{gong1b}
	\end{center}
	\item Elevane kan plassere boksane kor dei vil på brettet, men boksar kan ikkje overlappe.
	\item Spelet er ferdig når alle deltakarar i samme spelerunde ender opp med boksar dei ikkje har plass til. Den som da har dekt flest ruter har vunne.
\end{enumerate}
\textit{For å få med alle gangestykker fra 1-10 kan ein bruke 3 terningar i staden for berre to. Da kan elevane gonge summen av to terningar med den resterande terningen, så lenge summen blir mindre eller lik 10. Da bør ein kanskje innføre eit større ruteark.}
}

\of{
\opl{Ganging}{Pizza med pepperoni}
\lstt{Terningar}{20 min\,+}{Ingen}{Inne}

Sjå link under for gjennomføring av opplegg.\\
{\color{blue} \url{https://www.youcubed.org/tasks/pepperoni-pizza/}} 
}

\of{
\opl{Ganging}{Multiplikasjonskort}
\lstt{Utskrift av ark}{45 min\,+}{Valgfri}{Inne}

Sjå link under for gjennomføring av opplegg:\\
{\color{blue}\url{https://www.youcubed.org/tasks/math-cards/}}\vsk

\net{https://bhi61nm2cr3mkdgk1dtaov18-wpengine.netdna-ssl.com/wp-content/uploads/2017/03/Math-Card-Handout.pdf}{Ark for utskrift}
}

\of{
\opl{Ganging}{Gangekonkurranse}
\lstt{Ingen spesielle}
{20 min\,+}
{Individuelt}
{Inne eller ute}
\begin{enumerate}
	\item Alle elevar stiller på ei rekke. Lærar plukker ut to elevar som møter kvarandre i gangeduell.
	\item Lærar roper opp eit gangestykke. Den av dei to elevane som roper svaret først får gå eit skritt fram.
\end{enumerate}
\textit{Tips:  Ta med ei elevliste slik at det blir lettare å sørge for at elevane deltek i eit tilnærma likt antall duellar.}
}
\end{document}