\documentclass[english,hidelinks,pdftex, 11 pt, class=report,crop=false]{standalone}
\usepackage[T1]{fontenc}
\usepackage[utf8]{luainputenc}
\usepackage{lmodern} % load a font with all the characters
\usepackage{geometry}
\geometry{verbose,a4paper, inner=0cm, outer=0 cm, bmargin=2cm, tmargin=1cm}
%\textwidth=12cm
\setlength{\parindent}{0bp}
\usepackage{import}
\usepackage[subpreambles=false]{standalone}
\usepackage{amsmath}
\usepackage{amssymb}
\usepackage{esint}
\usepackage{babel}
\usepackage{tabu}
\usepackage[dvipsnames, table]{xcolor}
\usepackage{cancel}
\makeatother
\makeatletter
\usepackage{datetime2}
\usepackage{titlesec}
\usepackage[many]{tcolorbox}

% Eheter
\newcommand{\enh}[1]{\,\textrm{#1}}
%referances
\newcommand{\net}[2]{{\color{blue}\href{#1}{#2}}}

%Spaces
\newcommand{\vsk}{\\[12pt]}
\newcommand{\vs}{\vspace{-12pt}}

% Tabell for opplegg

\newcommand{\ovlist}[1]{
\vspace{-16pt}
\begin{itemize}
	#1
\end{itemize}
}

% Chapters and sections
\titleformat{\section}[block]{\bfseries}{\hspace{3cm}\thesection}{5pt}{}
\titleformat{\subsection}[block]{\bfseries}{\hspace{3cm}\thesection}{5pt}{}
\newcommand{\sectionbreak}{\clearpage} % New page on each section
 

\newlength{\mywidth}
\setlength{\mywidth}{14cm}

\newcommand{\cont}[1]{
\begin{tcolorbox}[center, boxrule=0.0 mm, width=\mywidth,arc=0mm,enhanced jigsaw,,colback=white,breakable]
#1	
\end{tcolorbox}
}

\newcommand{\info}[5]{
\begin{tcolorbox}[center, boxrule=0.1 mm, width=\mywidth,arc=0mm,enhanced jigsaw,breakable,colback=yellow!5]	
	
	\footnotesize
	\textbf{Øvingsområde}\\[5pt] #1 
	
	\textbf{Utstyr}\\ #2  \\
	
	\begin{tabular}{@{} p{4cm} p{4cm} l} 
		\textbf{Tid} & \textbf{Elevinndeling} & \textbf{Læringsarena} \\
		#3  & #4 & #5
	\end{tabular} 
\end{tcolorbox}	
}

\newcommand{\gjen}[1]{\begin{tcolorbox}[center,boxrule=0.1 mm, width=\mywidth,arc=0mm,colback=blue!3] {\large \textbf{Gjennomføring} \vspace{5 pt}}\newline #1  \end{tcolorbox}\vspace{-5pt}}
\newcommand{\eks}[1]{\begin{tcolorbox}[center,boxrule=0.1 mm, width=\mywidth,arc=0mm,colback=green!3] {\large \textbf{Eksempel} \vspace{5 pt}}\newline #1  \end{tcolorbox}\vspace{-5pt}}

\newcounter{opl}
%\numberwithin{opl}{article}


\newcommand{\opl}[1]{
\newpage
{\refstepcounter{opl} %\phantomsection 
\large \textbf{\theopl \;#1} \vsk}
}

% Headlines
\newcommand{\fork}{\textbf{Forkunnskapar}\\}
\newcommand{\forb}{\textbf{Forberedelsar}\\}
\newcommand{\opgvr}{\textbf{Oppgaver}}



%colors
\newcommand{\colr}[1]{{\color{red} #1}}
\newcommand{\colb}[1]{{\color{blue} #1}}
\newcommand{\colo}[1]{{\color{orange} #1}}
\newcommand{\colc}[1]{{\color{cyan} #1}}
\definecolor{projectgreen}{cmyk}{100,0,100,0}
\newcommand{\colg}[1]{{\color{projectgreen} #1}}

% Lister med bokstavar
\usepackage[inline]{enumitem}
% Opg
\newcommand{\abc}[1]{
	\begin{enumerate}[label=\alph*),leftmargin=18pt]
		#1
	\end{enumerate}
}

\usepackage[]{hyperref}

\newcommand{\note}{Merk}
\newcommand{\notesm}[1]{{\footnotesize \textsl{\note:} #1}}
\newcommand{\ekstitle}{Eksempel }
\newcommand{\sprtitle}{Språkboksen}
\newcommand{\expl}{forklaring}
\newcommand{\pyt}{Pytagoras' setning}
\newcommand\sv{\vsk \textbf{Svar} \vspace{4 pt}\\}

%references
\newcommand{\reftab}[1]{\hrs{#1}{tabell}}
\newcommand{\rref}[1]{\hrs{#1}{regel}}
\newcommand{\dref}[1]{\hrs{#1}{definisjon}}
\newcommand{\refkap}[1]{\hrs{#1}{kapittel}}
\newcommand{\refsec}[1]{\hrs{#1}{seksjon}}
\newcommand{\refdsec}[1]{\hrs{#1}{delseksjon}}
\newcommand{\refved}[1]{\hrs{#1}{vedlegg}}
\newcommand{\eksref}[1]{\textsl{#1}}
\newcommand\fref[2][]{\hyperref[#2]{\textsl{figur \ref*{#2}#1}}}
\newcommand{\refop}[1]{{\color{blue}Oppgave \ref{#1}}}
\newcommand{\refops}[1]{{\color{blue}oppgave \ref{#1}}}


%Algebra
\newcommand{\kvadset}{Kvadratsetningene}
\newcommand{\aenato}{Sum-produkt-metoden}

% Geometry
\newcommand{\hlikb}{Midtnormalen i en likebeint trekant}
\newcommand{\arealsetn}{Arealsetningen}
\newcommand{\trkmedian}{Median}
\newcommand{\midtrk}{Midtnormal (i trekant)}
\newcommand{\innskrsirk}{Innskrevet sirkel}
\newcommand{\cossetn}{Cosinussetningen}
\newcommand{\perfvink}{Sentral- og periferivinkel}
\newcommand{\tang}{Tangent}

% Derivative
\newcommand{\derel}{Den deriverte av elementære funksjoner}
\newcommand{\divder}{Divisjonsregelen}
\newcommand{\kjernereg}{Kjerneregelen}
\newcommand{\prodregder}{Produktregelen}
\newcommand{\lhop}{L'Hopitals regel}

% Funksjonsdrofting
\newcommand{\monder}{Monotoniegenskaper og den deriverte}
\newcommand{\fderekstr}{$ \bm{f'=0} $ for lokale ektstremalpunkt}
\newcommand{\andredertest}{Andrederiverttesten}

% Vectors
\newcommand{\detar}{Arealformler med determinanter}
\newcommand{\avstpunktlin}{Avstand mellom punkt og linje}

%Appendix
\newcommand{\rolle}{Rolles teorem}
\newcommand{\meanval}{Middelverdisetningen}

% Solutions manual
\newcommand{\selos}{Se løsningsforslag.}

\begin{document}

\newpage
\section{\fintro}
Variabler er verdier som forandrer seg. En verdi som forandrer seg i takt med at en variabel forandrer seg, kaller vi en \outl{funksjon}\index{funksjon}.\vsk
\fig{funk1}
I figurene over forandrer antallet ruter seg etter et bestemt mønster. Matematisk kan vi skildre dette mønsteret slik:
\alg{
\text{Antall ruter i \textsl{Figur \color{blue}1}}= 2\cdot {\color{blue}1}+1=3 \\ 
\text{Antall ruter i \textsl{Figur \color{blue}2}}= 2\cdot {\color{blue}2}+1=5 \\ 
\text{Antall ruter i \textsl{Figur \color{blue}3}}= 2\cdot {\color{blue}3}+1=7 \\ 
\text{Antall ruter i \textsl{Figur \color{blue}4}}= 2\cdot {\color{blue}4}+1=9
}
For en figur med et vilkårlig nummer $ x $ har vi at
\[ \text{Antall ruter i \textsl{Figur }}{\color{blue}x}=2{\color{blue}x}+1 \]
Antall ruter forandrer seg altså i takt med at $ x $ forandrer seg, og da sier vi at\regv
\st{\text{''Antall ruter i \textsl{Figur }$ x $'' er en funksjon av $ x $}\vsk

$ {2x+1} $ er \outl{funksjonsuttrykket}\index{funksjonsuttrykk} til funksjonen ''Antall ruter \\ i \textsl{Figur }$ x $''.
}

\newpage
\textbf{Generelle uttrykk} \\
Skulle vi jobbet videre med funskjonen vi akkurat har sett på, ville det blitt tungvint å hele tiden måtte skrive ''Antall ruter i \textsl{Figur }$ x $''. Det er vanlig å kalle også funksjoner bare for en bokstav, og i tillegg skrive variabelen funskjonen er avhengig av i parentes. La oss nå omdøpe funksjonen ''Antall ruter i \textsl{Figur} $ x $'' til $ a(x) $. Da har vi at
\[ \text{Antall ruter i \textsl{Figur }}x=a(x)=2x+1 \]
Hvis vi skriver $ a(x) $, men erstatter $ x $ med et bestemt tall, betyr det at vi skal erstatte $ x $ med dette tallet i funksjonsuttrykket vårt:
\alg{
a({\color{blue}1})&=2\cdot{\color{blue}1} +1=3 \\
a({\color{blue}2})&=2\cdot{\color{blue}2}+1=5 \\
a({\color{blue}3})&=2\cdot{\color{blue}3}+1=7\\
a({\color{blue}4})&=2\cdot{\color{blue}4}+1=9
}
\fig{funk1a}
\spr{
Som leseren kanskje har lagt merke til, er en funksjon med et tilhørende funksjonsuttrykk strengt tatt bare en likning med to ukjente. Men ordet \textit{funksjon} brukes istendefor \textit{likning} for å tydeliggjøre at vi har å gjøre med en likning hvor én variabel er isolert på den éne siden av likhetstegnet, og at vi har et uttrykk av en annen variabel\footnote{En funksjon kan også være uttrykt ved flere variabler.} på den andre siden. Et annet synonym for \textit{likning} og \textit{funksjon} er \outl{formel}\index{formel}. 
}
\newpage
\eks{
La antall ruter i mønsteret under være gitt av funksjonen $ a(x) $.
\fig{funk2}
\textbf{a)} Finn uttrykket for $ a(x) $.\bs
\textbf{b)} Hvor mange ruter er der når $ x=10 $? \bs
\textbf{c)} Hva er verdien til $ x $ når $ a(x)=628 $?

\sv
\abc{
\item Vi legger merke til at
\begin{itemize}
	\item når $ x=1 $, er det $ 1\cdot1+3=4 $ ruter.
	\item når $ x=2 $, er det $ 2\cdot2+3=7 $ ruter.
	\item når $ x=3 $, er det $ 3\cdot3+3=12 $ ruter.
	\item når $ x=4 $, er det $ 4\cdot4+3=17 $ ruter.
\end{itemize} 
Altså er
\[ a(x)=x\cdot x +3 =x^2+3 \]
\item \[ a(10)=10^2+3=100+3=103 \]
Når $ x=10 $, er det 103 ruter.
\item Vi har likningen
\algv{
	x^2+3&=628 \\
	x^2&=625
}	
Altså er
\[ x=15\qquad\vee\qquad x=-15 \]
Siden vi søker en positiv verdi for $ x $, er $ x=15 $.	
}
}
\section{\lingraf}
Når vi har en variabel $ x $ og en funksjon $ f(x) $, har vi hele tiden to verdier; verdien til $ x $ og den tilhørende verdien til $ f(x) $. Disse parene av verdier kan vi sette inn i et koordinatsystem\footnote{Se \hrs{Koord}{seksjon}.} for å lage \outl{grafen}\index{funksjon!grafen til} til $ f(x) $. \vsk

La oss bruke funksjonen 
\[ f(x)=2x-1 \]
som eksempel. Vi har at
\alg{
f(0)&=2\cdot0-1=-1 \vn
f(1)&=2\cdot1-1=1 \vn
f(2)&=2\cdot2-1=3 \vn
f(3)&=2\cdot3-1=5
} 
Disse parene av verdier kan vi sette opp i en tabell:
	\begin{center}
	\begin{tabular}{c | c |c |c|c}
		$ x $ & 0 & 1 & 2 & 3 \\ \hline
		$ f(x) $ &$  -1 $ & 1&3 &5
	\end{tabular}
\end{center}
Tabellen over gir punktene
\[ (0, -1)\quad\quad(1, 1)\quad\quad(2, 3)\quad\quad(3, 5) \]
Vi plasserer nå punktene i et koordinatsystem (se figur på side \pageref{funkfig}). I samband med funksjoner er det vanlig å kalle horisontalaksen og vertikalaksen for henholdsvis \outl{$ x $-aksen} og \outl{$ y $-aksen}. 
Grafen til $ f(x) $ er nå en tenkt strek som går gjennom alle de uendelig mange punktene vi kan lage av $ x$-verdier og de tilhørende $ f(x) $-verdiene. Vår funksjon er en \textit{lineær}\index{funksjon!lineær} funksjon, noe som betyr at grafen er en rett linje. Altså kan grafen tegnes ved å tegne linja som går gjennom punktene vi har funnet.\vsk

Som vi har vært inne på før, kan vi aldri tegne en hel linje, bare et utklipp av den. Dette gjelder som regel også for grafer. I figuren på side \pageref{funkfig} har vi tegnet grafen til $ f(x) $ for $ x $-verdier mellom $ -2 $ og $ 4 $. At $ x $ er i dette \outl{intervallet}\index{intervall} kan vi skrive som\footnote{Se symbolforklaringer på side \pageref{Symbol}.} $ -2\leq x\leq 4 $ eller $ x\in[-2, 4] $.
\fig{funk3} \label{funkfig}
\info{Merk}{
En lengde på $ x $-aksen trenger ikke å svar til samme verdi som en lengde på $ y $-aksen.
\fig{funkmerk}
}
\newpage
\reg[\funklin \label{funklin}]{
	En funksjon på formen \vs
	\[ f(x)=ax+b \]
	er en \outl{lineær} funksjon med \outl{stigningstall}\index{stigningstall} $ a $ og \outl{konstantledd}\index{konstantledd} $ b $. \\[12pt]
	
	Grafen til en lineær funksjon er en rett linje som går gjennom punktet $ (0, b) $. \vsk
	
	For to forskjellige $ x $-verdier, $ x_1 $ og $ x_2 $, er
	\[ a=\frac{f(x_2)-f(x_1)}{x_2-x_1} \]
	\fig{funk6}
}
\eks[1]{Finn stigningstallet og konstantleddet til funksjonene.
	\alg{
		f(x)&=2x+1 \vn
		g(x)&=-3+\frac{7}{2}	\vn
		h(x)&=\frac{1}{4}x-\frac{5}{6}\vn
		j(x)&=4-\frac{1}{2}x
	}
	\sv \vs \vs
	\begin{itemize}
		\item $ f(x) $ har stigningstall 2 og konstantledd 1.
		\item $ g(x) $ har stigningstall $ -3 $ og konstantledd $ \frac{7}{2} $.
		\item $ h(x) $ har stigningstall $ \frac{1}{4} $ og konstantledd $ -\frac{5}{6} $.
		\item $ j(x) $ har stigningstall $ -\frac{1}{2} $ og konstantledd 4.		
	\end{itemize}	
}
\newpage
\eks[2]{
	Tegn grafen til
	\[ f(x)=\dfrac{3}{4}x-2 \]
	 for $ x\in[-5, 6] $.
	
	\sv
	For å tegne grafen til en lineær funksjon trenger vi bare å finne to punkt som ligger på grafen. Hvilke to punkt dette er, er det fritt å velge, så for enklest mulig utregning starter vi med å finne punktet der $ x=0 $:
	\[ f(0)=\frac{3}{4}\cdot0-2=-2 \] 
	Videre velger vi $ {x=4} $, siden dette også gir oss en enkel utregning:
	\[ f(4)=\frac{3}{4}\cdot4-2=1 \]
	Nå har vi informasjonen vi trenger, og for ordens skyld setter vi den inn i en tabell:
	\begin{center}
		\begin{tabular}{c | c |c }
			$ x $ & 0 & 4 \\ \hline
			$ f(x) $ &$  -2 $ & 1
		\end{tabular}
	\end{center}
Vi tegner punktene og trekker en linje gjennom dem:
	\begin{figure}
		\centering
		\fig{funk4}
	\end{figure}
}
\eks[3]{
Finn funksjonsuttrykkene til $ f(x) $ og $ g(x) $.	
\fig{funk5} \vs
\sv
Vi starter med å finne funksjonsuttrykket til $ f(x) $. Punktet $ (0, 3) $ ligger på grafen til $ f(x) $ (se også figur på neste side). Da vet vi at $ {f(0)=3} $, og dette må bety at $ 3 $ er konstantleddet til $ f(x) $. Videre ser vi at punktet $ (1, 2) $ også ligger på grafen til $ f(x) $. Stigningstallet til $ f(x) $ er da gitt ved brøken
\[ \frac{2-3}{1-0}=-1 \] 
Altså er 
\[  f(x)=-x+3 \]
\newpage
\fig{funk5a}
Vi går så over til å finne uttrykket til $ g(x) $. Punktet $ (0, -1) $ ligger på grafen til $ g(x) $. Da vet vi at $ {f(0)=-1} $, og dette må bety at $ -1 $ er konstantleddet til $ g(x) $. Videre ser vi at punktet $ (5, 2) $ også ligger på grafen til $ g(x) $. Stigningstallet til $ g(x) $ er da gitt ved brøken
\[ \frac{2-(-1)}{5-0}=\frac{3}{5} \]
Altså er
\[ g(x)=\frac{3}{5}x+1 \]
}
\newpage
\eks[4]{
	Finn stigningstallet til $ f(x) $.
	\fig{funk7}
	\sv
	\fig{funk7b}
	Vi legger merke til at punktene $ (1, 2) $ og $ (4, 4) $ ligger på grafen til $ f(x) $. Altså er stigningstalelt til $ f(x) $ gitt ved brøken
	\alg{
		\frac{4-2}{4-1} =\frac{2}{3}
	}
}
\newpage
\fork{\ref{funklin} \funklin}{
\textbf{Uttrykk for $ \bm a$}\os	
Gitt en lineær funksjon
\[ f(x)=ax+b \]
For to forskjellige $ x $-verdier, $ x_1 $ og $ x_2 $, har vi at
\begin{equation}
f(x_1)=ax_1+b \label{funkfork}
\end{equation}
\begin{equation}
f(x_2)=ax_2+b \label{funkfork1}
\end{equation}
Vi trekker \eqref{funkfork} fra \eqref{funkfork1}, og får at 
\begin{align}
	f(x_2)-f(x_1)&=ax_2+b-(ax_1+b) \nonumber \br
	f(x_2)-f(x_1)&=ax_2-ax_1 \nonumber\\
	f(x_2)-f(x_1)&=a(x_2-x_1) \nonumber\\
	\frac{f(x_2)-f(x_1)}{x_2-x_1}&=a \label{funka}
\end{align}
\vsk

\textbf{Grafen til en lineær funksjon er ei rett linje}\os

Gitt en lineær funksjon $ f(x)=ax+b $ og to forskjellige $ x $-verdier $ x_1 $ og $ x_2 $. Vi setter $ {A=(x_1, b)} $, $ B=(x_2, b) $, $ C=(b, f(x_1)) $,\\ $ D=(0, f(x_2)) $ og $ E=(0, b) $.
\fig{funk6a}
Av \eqref{funka} har vi at
\begin{align}
	\frac{f(x_1)-f(0)}{x_1-0}&=a \nonumber\br
	\frac{ax_1+b-b}{x_1}&=a \nonumber \br
	\frac{ax_1}{x_1}=a \label{funkx1}
\end{align}
Tilsvarende er 
\begin{equation} \label{funkx2}
\frac{ax_2}{x_2}=a 
\end{equation}
Videre har vi at
\alg{
AC&=f(x_1)-b = ax_1 \vn
BD&=f(x_2)-b = ax_2 \vn
EA&= x_1 \vn
EB&= x_2
}
Av \eqref{funkx1} og \eqref{funkx2} har vi at
\[ \frac{ax_1}{x_1}=\frac{ax_2}{x_2} \]
Dette betyr at
\[ \frac{AC}{BD}=\frac{EA}{EB} \]
I tillegg er $ {\angle A=\angle B} $, altså oppfyller $ \triangle EAC $ og $ \triangle EBD $ vilkår (iii) fra \rref{vilkform}, og dermed er trekantane formlike. Dette betyr at $ C $ og $ D $ ligger på linje, og denne linja må vere grafen til $ f(x) $.
}
\section{Viktige punkt på grafer}

\reg[Skjæringspunkt til grafer]{
	Et punkt hvor to funksjoner har samme verdi kalles et \outl{skjæringspunkt} til funksjonene.
}

\eks[1]{
	Gitt de to funksjonene
	\alg{
		f(x) &= 2x+1 \vn 
		g(x) &= x+4
	}
	Finn skjæringspunktet til $ f(x) $ og $ g(x) $.
	
	\sv
	
	Vi kan finne skjæringspunktet både ved en \textsl{grafisk} og en \textsl{algebraisk} metode. \vsk
	
	\metode{Grafisk metode}{0.6\linewidth} \os
	
	Vi tegner grafene til funksjonene inn i det samme koordinatsystemet:
	\fig{funklig1}
	Vi leser av at funksjonene har samme verdi når $ {x=3} $, og da har begge funksjonene verdien 7. Altså er skjæringspunktet $ (3, 7) $. \newpage
	
	\metode{Algebraisk metode}{0.6\linewidth} \os
	At $ f(x) $ og $ g(x) $ har samme verdi gir likningen
	\alg{
		f(x)&=g(x) \\
		2x+1 &=x+4 \\
		x&=3
	}
	Videre har vi at
	\alg{
		f(3)&=2\cdot3+1=7 \vn 
		g(3)&=3+4=7
	}
	Altså er $ (3, 7) $ skjæringspunktet til grafene.\vsk
	
	{\footnotesize \mer Det hadde selvsagt holdt å bare finne én av $ f(3) $ og $ g(3) $.}
}
\newpage
\reg[Null-, bunn- og toppunkt]{
\outl{Nullpunkt} \\
En $ x $-verdi som gir funksjonsverdi 0.\vsk

\outl{Lokalt bunnpunkt} \\
Punkt der funksjonen (fra venstre) går fra å synke i verdi til å stige i verdi. \vsk

\outl{Lokalt toppunkt} \\
Punkt der funksjonen (fra venstre) går fra å stige i verdi til å synke i verdi \vsk

\outl{Globalt bunnpunkt} \\
Punkt der funksjonen har sin laveste verdi.\vsk

\outl{Globalt toppunkt} \\
Punkt der funksjonen har sin høyeste verdi.
}

\eks[]{
	\fig{funkdroft}
}
\newpage
\info{Hvorfor er nullpunkt en verdi?}{
	Det kan kanskje virke litt rart at vi kaller $ x $-verdier for nullpunkt, punkt har jo både en $ x $-verdi og en $ y $-verdi. Men når det er snakk om nullpunkt, er det underforstått at $ {y=0} $, og da er det tilstrekkelig å vite $ x $-verdien for å avgjøre hvilket punkt det er snakk om.  
}
\section{Navn på funksjoner}
\regdef[Potensfunksjoner]{
	Gitt konstantene $ k $ og $ b $, og en variabel $ x $. En funksjon på formen
	\begin{equation}\label{powrfunc}
		f(x)=k x^m	
	\end{equation}
	er da en \outl{potensfunksjon} med \outl{koeffisient} $ k $ og \outl{eksponent} $ m $.
}
\regdef[Polynomfunksjoner]{
	En \outl{polynomfunksjon} er én av følgende:
	\begin{itemize}
		\item en potensfunksjon med heltalls eksponent større eller lik 0.
		\item summen av flere potensfunksjoner med heltalls eksponent større eller lik 0.
	\end{itemize}
	Polynomfunksjoner kategoriseres etter den største eksponenten i funksjonsuttrykket. For konstantene $ a $, $ b $, $ c $ og $ d$, og en variabel $ x $, har vi at \vs
	\begin{center}
		\begin{tabular}{|r|l|} \hline
			\textbf{funksjonsuttyrykk} & \textbf{funksjonsnavn} \\ \hline
			$ ax+b $ &1. grads funksjon/polynom (lineær) \\
			$ ax^2+bx+c $& 2. grads funksjon/polynom (kvadratisk) \\
			$ ax^3+bx^2+cx+d $& 3. grads funksjon/polynom (kubisk) \\ \hline
		\end{tabular}
	\end{center}
}
\eks[1]{ 
	$ 4x^7-5x^2+4 $ er et 7. grads polynom. \vsk
	
	$ \frac{2}{7}x^5-3 $ er et et 5. grads polynom.
}
\newpage
\regdef[Rasjonale funksjoner]{
	Hvis uttrykket til en funksjon består av en brøk med et polynom i både teller og nevner, er funksjonen en \outl{rasjonal funksjon}.
}
\eks{
	$ f $, $ g $ og $ h $ er rasjonale funksjoner.
	\[ f(x)=\frac{1}{x} \qquad\qquad g(x)=\frac{x^2+3}{4x+2}\qquad\qquad h(x)=\frac{x^3-x+4}{-2x^7+9x^2}\]
}

\end{document}