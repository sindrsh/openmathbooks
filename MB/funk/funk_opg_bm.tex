\documentclass[english,hidelinks,pdftex, 11 pt, class=report,crop=false]{standalone}
\usepackage[T1]{fontenc}
\usepackage[utf8]{luainputenc}
\usepackage{lmodern} % load a font with all the characters
\usepackage{geometry}
\geometry{verbose,a4paper, inner=0cm, outer=0 cm, bmargin=2cm, tmargin=1cm}
%\textwidth=12cm
\setlength{\parindent}{0bp}
\usepackage{import}
\usepackage[subpreambles=false]{standalone}
\usepackage{amsmath}
\usepackage{amssymb}
\usepackage{esint}
\usepackage{babel}
\usepackage{tabu}
\usepackage[dvipsnames, table]{xcolor}
\usepackage{cancel}
\makeatother
\makeatletter
\usepackage{datetime2}
\usepackage{titlesec}
\usepackage[many]{tcolorbox}

% Eheter
\newcommand{\enh}[1]{\,\textrm{#1}}
%referances
\newcommand{\net}[2]{{\color{blue}\href{#1}{#2}}}

%Spaces
\newcommand{\vsk}{\\[12pt]}
\newcommand{\vs}{\vspace{-12pt}}

% Tabell for opplegg

\newcommand{\ovlist}[1]{
\vspace{-16pt}
\begin{itemize}
	#1
\end{itemize}
}

% Chapters and sections
\titleformat{\section}[block]{\bfseries}{\hspace{3cm}\thesection}{5pt}{}
\titleformat{\subsection}[block]{\bfseries}{\hspace{3cm}\thesection}{5pt}{}
\newcommand{\sectionbreak}{\clearpage} % New page on each section
 

\newlength{\mywidth}
\setlength{\mywidth}{14cm}

\newcommand{\cont}[1]{
\begin{tcolorbox}[center, boxrule=0.0 mm, width=\mywidth,arc=0mm,enhanced jigsaw,,colback=white,breakable]
#1	
\end{tcolorbox}
}

\newcommand{\info}[5]{
\begin{tcolorbox}[center, boxrule=0.1 mm, width=\mywidth,arc=0mm,enhanced jigsaw,breakable,colback=yellow!5]	
	
	\footnotesize
	\textbf{Øvingsområde}\\[5pt] #1 
	
	\textbf{Utstyr}\\ #2  \\
	
	\begin{tabular}{@{} p{4cm} p{4cm} l} 
		\textbf{Tid} & \textbf{Elevinndeling} & \textbf{Læringsarena} \\
		#3  & #4 & #5
	\end{tabular} 
\end{tcolorbox}	
}

\newcommand{\gjen}[1]{\begin{tcolorbox}[center,boxrule=0.1 mm, width=\mywidth,arc=0mm,colback=blue!3] {\large \textbf{Gjennomføring} \vspace{5 pt}}\newline #1  \end{tcolorbox}\vspace{-5pt}}
\newcommand{\eks}[1]{\begin{tcolorbox}[center,boxrule=0.1 mm, width=\mywidth,arc=0mm,colback=green!3] {\large \textbf{Eksempel} \vspace{5 pt}}\newline #1  \end{tcolorbox}\vspace{-5pt}}

\newcounter{opl}
%\numberwithin{opl}{article}


\newcommand{\opl}[1]{
\newpage
{\refstepcounter{opl} %\phantomsection 
\large \textbf{\theopl \;#1} \vsk}
}

% Headlines
\newcommand{\fork}{\textbf{Forkunnskapar}\\}
\newcommand{\forb}{\textbf{Forberedelsar}\\}
\newcommand{\opgvr}{\textbf{Oppgaver}}



%colors
\newcommand{\colr}[1]{{\color{red} #1}}
\newcommand{\colb}[1]{{\color{blue} #1}}
\newcommand{\colo}[1]{{\color{orange} #1}}
\newcommand{\colc}[1]{{\color{cyan} #1}}
\definecolor{projectgreen}{cmyk}{100,0,100,0}
\newcommand{\colg}[1]{{\color{projectgreen} #1}}

% Lister med bokstavar
\usepackage[inline]{enumitem}
% Opg
\newcommand{\abc}[1]{
	\begin{enumerate}[label=\alph*),leftmargin=18pt]
		#1
	\end{enumerate}
}

\usepackage[]{hyperref}

\newcommand{\note}{Merk}
\newcommand{\notesm}[1]{{\footnotesize \textsl{\note:} #1}}
\newcommand{\ekstitle}{Eksempel }
\newcommand{\sprtitle}{Språkboksen}
\newcommand{\expl}{forklaring}
\newcommand{\pyt}{Pytagoras' setning}
\newcommand\sv{\vsk \textbf{Svar} \vspace{4 pt}\\}

%references
\newcommand{\reftab}[1]{\hrs{#1}{tabell}}
\newcommand{\rref}[1]{\hrs{#1}{regel}}
\newcommand{\dref}[1]{\hrs{#1}{definisjon}}
\newcommand{\refkap}[1]{\hrs{#1}{kapittel}}
\newcommand{\refsec}[1]{\hrs{#1}{seksjon}}
\newcommand{\refdsec}[1]{\hrs{#1}{delseksjon}}
\newcommand{\refved}[1]{\hrs{#1}{vedlegg}}
\newcommand{\eksref}[1]{\textsl{#1}}
\newcommand\fref[2][]{\hyperref[#2]{\textsl{figur \ref*{#2}#1}}}
\newcommand{\refop}[1]{{\color{blue}Oppgave \ref{#1}}}
\newcommand{\refops}[1]{{\color{blue}oppgave \ref{#1}}}


%Algebra
\newcommand{\kvadset}{Kvadratsetningene}
\newcommand{\aenato}{Sum-produkt-metoden}

% Geometry
\newcommand{\hlikb}{Midtnormalen i en likebeint trekant}
\newcommand{\arealsetn}{Arealsetningen}
\newcommand{\trkmedian}{Median}
\newcommand{\midtrk}{Midtnormal (i trekant)}
\newcommand{\innskrsirk}{Innskrevet sirkel}
\newcommand{\cossetn}{Cosinussetningen}
\newcommand{\perfvink}{Sentral- og periferivinkel}
\newcommand{\tang}{Tangent}

% Derivative
\newcommand{\derel}{Den deriverte av elementære funksjoner}
\newcommand{\divder}{Divisjonsregelen}
\newcommand{\kjernereg}{Kjerneregelen}
\newcommand{\prodregder}{Produktregelen}
\newcommand{\lhop}{L'Hopitals regel}

% Funksjonsdrofting
\newcommand{\monder}{Monotoniegenskaper og den deriverte}
\newcommand{\fderekstr}{$ \bm{f'=0} $ for lokale ektstremalpunkt}
\newcommand{\andredertest}{Andrederiverttesten}

% Vectors
\newcommand{\detar}{Arealformler med determinanter}
\newcommand{\avstpunktlin}{Avstand mellom punkt og linje}

%Appendix
\newcommand{\rolle}{Rolles teorem}
\newcommand{\meanval}{Middelverdisetningen}

% Solutions manual
\newcommand{\selos}{Se løsningsforslag.}

\begin{document}
\opgt	

\op{opgfunfinnlin}
La antall ruter i figuren under være gitt ved $ f(x) $.
\fig{funkopg3}
\abc{
	\item Finn et uttrykk for $ f(x) $.
	\item Hvor mange ruter er der når $ x = 100 $?
	\item Hva er $ x $ når $ f(x)=24 $.
}

\op{opgfunkfem}
La antall ruter i figuren under være gitt ved $ a(x) $.
\fig{funkopg1}
\abc{
\item Finn et uttrykk for $ a(x) $.
\item Hvor mange ruter er der når $ x = 20 $?
\item Hva er $ x $ når $ a(x) = 405 $?
}
\newpage
\op{opgfunkbluegreen}
La antall ruter i figuren under være gitt ved $ b(x) $.
\fig{funkopg2}
\abc{
	\item Finn et uttrykk for $ b(x) $.
	\item Hvor mange ruter er der når $ x = 20 $?
	\item Hva er $ x $ når $ b(x) = 80 $?
}

\eksop{EGV22D1}{EG22D1opg1}
Under vises de tre første figurene i et mønster. Figurene er satt sammen av trekanter og kvadrater.
\fig{eksu22opg1}
Hvor mange trekanter og hvor mange firkanter vil det være i figur nummer 10?

\op{opgfunkparodd}
La $ x $ være et positivt heltall.
\abc{
	\item Lag en funskjon $ p(x) $ som gir verdien til positivt partall\\ nr. $ x $.
	\item Lag en funksjon $ o(x) $ som gir verdien til positivt oddetall\\ nr. $ x $.
}

\newpage
\nes

\op{opgfunkstigogkonst}
Finn stigningstallet og konstantleddet til funksjonene.\os

\abch{
	\item $ f(x) = 5x + 10  $
	\ \item $ g(x) = 3x - 12  $
}  \vsk

\abchs{3}{
	\item  $\displaystyle h(x) = -\frac{1}{7}x- 9 $ 
	\item $\displaystyle i(x)=\frac{3}{2}x-\frac{1}{4} $
}

\op{funkopgtegn}
Tegn grafen til disse funksjonene på intervallet $ x\in[-5, 5] $: \os

\abch{
\item $ f(x) = 2x - 1 $ 
\item $g(x) = -3x + 5 $
}

\nes

\op{funkopgligset}
Gitt likningsettet\vs
\alg{
	x-y&=5 \tag{I} \label{eks3a}\vn
	x+y&=9 \tag{II} \label{eks3b}
}
\abc{
\item Forklar hvorfor løsningen av likningssettet er skjæringspunktet til funksjonene
\algv{
f(x)&=x-5 \vn
g(x)&=9-x
}
\item Løs liknigssettet.
}

\newpage
\op{funkopgfinngraf1}
Finn funksjonsuttrykket til $ f(x) $
\begin{figure}
	\centering
	\includegraphics[scale=1]{\figp{funkeks4}}
\end{figure}

\op{funkopgfinngraf2}
Finn funksjonsuttrykket til $ f(x) $
\begin{figure}
	\centering
	\includegraphics[scale=0.5]{\figp{funkeks3}}
\end{figure}

\newpage
\op{funkopgfinngraf3}
Finn funksjonsuttrykkene til $ f(x) $ og $ g(x) $.
\begin{figure}
	\centering
	\includegraphics[scale=0.6]{\figp{funkeks2}}
\end{figure}

\newpage

\grubop{opgaddifparodd}
Bruk formlene fra \refops{opgfunkparodd} til å vise at
\abc{	
\item  summen/differansen mellom to partall er et partall.
\item summen/differansen mellom to oddetall er et partall.
\item summen/differansen mellom et partall og et oddetall er et oddetall.
}

\grubop{opgfunk3ukjente}
Funksjonen $ f(x)=ax^2+b x+ c $ går gjennom punktene $ (-3, 49) $, $ (0, 4) $ og $ (10, 149) $. Finn verdiene til $ a $, $ b $ og $ c $. 

\grubop{opgettpunkt} \vs
\abc{
\item Gitt at en lineær funksjon $ f(x) $ har stigningstall 3, og at punktet $ (2, 1) $ ligger på grafen til $ f(x) $. Finn funksjonsuttrykket til $ f(x) $.
\item Gitt en lineær funksjon $ f(x) $ med stigningstall $ a $, og punktet $ (x_1, y_1) $, som ligger på grafen til $ f(x) $. Vis at\footnote{Denne formelen kalles \textit{ettpunktsformelen}.}
\[ f(x) = a(x-x_1)+y_1\]
(Denne formelen kalles \outl{ettpunktsformelen} \index{ettpunktsformelen}.)
}

\grubop{opgfming}
Gitt funksjonene $ f(x) $ og $ g(x) $, hvor grafen til $ g $ er linja som går gjennom $ A=(a,f(a)) $ og $ B=(b, f(b)) $. Vis at
\[ f-g =f(x)-\frac{f(b)-f(a)}{b-a}(x-a)+f(a) \]




\end{document}