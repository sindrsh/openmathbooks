\documentclass[english,hidelinks,pdftex, 11 pt, class=report,crop=false]{standalone}

% note
\newcommand{\note}{Note}
\newcommand{\notesm}[1]{{\footnotesize \textsl{\note:} #1}}
\newcommand{\selos}{See the solutions manual.}

\newcommand{\texandasy}{The text is written in \LaTeX\ and the figures are made with the aid of Asymptote.}

\newcommand{\rknut}{Calculate.}
\newcommand\sv{\vsk \textbf{Answer} \vspace{4 pt}\\}
\newcommand{\ekstitle}{Example }
\newcommand{\sprtitle}{The language box}
\newcommand{\expl}{explanation}

% answers
\newcommand{\mulansw}{\notesm{Multiple possible answers.}}	
\newcommand{\faskap}{Chapter}

% exercises
\newcommand{\opgt}{\newpage \phantomsection \addcontentsline{toc}{section}{Exercises} \section*{Exercises for Chapter \thechapter}\vs \setcounter{section}{1}}

% references
\newcommand{\reftab}[1]{\hrs{#1}{Table}}
\newcommand{\rref}[1]{\hrs{#1}{Rule}}
\newcommand{\dref}[1]{\hrs{#1}{Definition}}
\newcommand{\refkap}[1]{\hrs{#1}{Chapter}}
\newcommand{\refsec}[1]{\hrs{#1}{Section}}
\newcommand{\refdsec}[1]{\hrs{#1}{Subsection}}
\newcommand{\refved}[1]{\hrs{#1}{Appendix}}
\newcommand{\eksref}[1]{\textsl{#1}}
\newcommand\fref[2][]{\hyperref[#2]{\textsl{Figure \ref*{#2}#1}}}
\newcommand{\refop}[1]{{\color{blue}Exercise \ref{#1}}}
\newcommand{\refops}[1]{{\color{blue}Exercise \ref{#1}}}

%%% SECTION HEADLINES %%%

% Our numbers
\newcommand{\likteikn}{The equal sign}
\newcommand{\talsifverd}{Numbers, digits and values}
\newcommand{\koordsys}{Coordinate systems}

% Calculations
\newcommand{\adi}{Addition}
\newcommand{\sub}{Subtraction}
\newcommand{\gong}{Multiplication}
\newcommand{\del}{Division}

%Factorization and order of operations
\newcommand{\fak}{Factorization}
\newcommand{\rrek}{Order of operations}

%Fractions
\newcommand{\brgrpr}{Introduction}
\newcommand{\brvu}{Values, expanding and simplifying}
\newcommand{\bradsub}{Addition and subtraction}
\newcommand{\brgngheil}{Fractions multiplied by integers}
\newcommand{\brdelheil}{Fractions divided by integers}
\newcommand{\brgngbr}{Fractions multiplied by fractions}
\newcommand{\brkans}{Cancelation of fractions}
\newcommand{\brdelmbr}{Division by fractions}
\newcommand{\Rasjtal}{Rational numbers}

%Negative numbers
\newcommand{\negintro}{Introduction}
\newcommand{\negrekn}{The elementary operations}
\newcommand{\negmeng}{Negative numbers as amounts}

%Calculation methods
\newcommand{\delmedtihundre}{Deling med 10, 100, 1\,000 osv.}

% Geometry 1
\newcommand{\omgr}{Terms}
\newcommand{\eignsk}{Attributes of triangles and quadrilaterals}
\newcommand{\omkr}{Perimeter}
\newcommand{\area}{Area}

%Algebra 
\newcommand{\algintro}{Introduction}
\newcommand{\pot}{Powers}
\newcommand{\irrasj}{Irrational numbers}

%Equations
\newcommand{\ligintro}{Introduction}
\newcommand{\liglos}{Solving with the elementary operations}
\newcommand{\ligloso}{Solving with elementary operations summarized}

%Functions
\newcommand{\fintro}{Introduction}
\newcommand{\lingraf}{Linear functions and graphs}

%Geometry 2
\newcommand{\geoform}{Formulas of area and perimeter}
\newcommand{\kongogsim}{Congruent and similar triangles}
\newcommand{\geofork}{Explanations}

% Names of rules
\newcommand{\adkom}{Addition is commutative}
\newcommand{\gangkom}{Multiplication is commutative}
\newcommand{\brdef}{Fractions as rewriting of division}
\newcommand{\brtbr}{Fractions multiplied by fractions}
\newcommand{\delmbr}{Fractions divided by fractions}
\newcommand{\gangpar}{Distributive law}
\newcommand{\gangparsam}{Paranthesis multiplied together}
\newcommand{\gangmnegto}{Multiplication by negative numbers I}
\newcommand{\gangmnegtre}{Multiplication by negative numbers II}
\newcommand{\konsttre}{Unique construction of triangles}
\newcommand{\kongtre}{Congruent triangles}
\newcommand{\topv}{Vertical angles}
\newcommand{\trisum}{The sum of angles in a triangle}
\newcommand{\firsum}{The sum of angles in a quadrilateral}
\newcommand{\potgang}{Multiplication by powers}
\newcommand{\potdivpot}{Division by powers}
\newcommand{\potanull}{The special case of \boldmath $a^0$}
\newcommand{\potneg}{Powers with negative exponents}
\newcommand{\potbr}{Fractions as base}
\newcommand{\faktgr}{Factors as base}
\newcommand{\potsomgrunn}{Powers as base}
\newcommand{\arsirk}{The area of a circle}
\newcommand{\artrap}{The area of a trapezoid}
\newcommand{\arpar}{The area of a parallelogram}
\newcommand{\pyt}{Pythagoras's theorem}
\newcommand{\forform}{Ratios in similar triangles}
\newcommand{\vilkform}{Terms of similar triangles}
\newcommand{\omkrsirk}{The perimeter of a circle (and the value of $ \bm \pi $)}
\newcommand{\artri}{The area of a triangle}
\newcommand{\arrekt}{The area of a rectangle}
\newcommand{\liknflyt}{Moving terms across the equal sign}
\newcommand{\funklin}{Linear functions}


\usepackage[T1]{fontenc}
\usepackage[utf8]{luainputenc}
\usepackage{lmodern} % load a font with all the characters
\usepackage{geometry}
\geometry{verbose,a4paper, inner=0cm, outer=0 cm, bmargin=2cm, tmargin=1cm}
%\textwidth=12cm
\setlength{\parindent}{0bp}
\usepackage{import}
\usepackage[subpreambles=false]{standalone}
\usepackage{amsmath}
\usepackage{amssymb}
\usepackage{esint}
\usepackage{babel}
\usepackage{tabu}
\usepackage[dvipsnames, table]{xcolor}
\usepackage{cancel}
\makeatother
\makeatletter
\usepackage{datetime2}
\usepackage{titlesec}
\usepackage[many]{tcolorbox}

% Eheter
\newcommand{\enh}[1]{\,\textrm{#1}}
%referances
\newcommand{\net}[2]{{\color{blue}\href{#1}{#2}}}

%Spaces
\newcommand{\vsk}{\\[12pt]}
\newcommand{\vs}{\vspace{-12pt}}

% Tabell for opplegg

\newcommand{\ovlist}[1]{
\vspace{-16pt}
\begin{itemize}
	#1
\end{itemize}
}

% Chapters and sections
\titleformat{\section}[block]{\bfseries}{\hspace{3cm}\thesection}{5pt}{}
\titleformat{\subsection}[block]{\bfseries}{\hspace{3cm}\thesection}{5pt}{}
\newcommand{\sectionbreak}{\clearpage} % New page on each section
 

\newlength{\mywidth}
\setlength{\mywidth}{14cm}

\newcommand{\cont}[1]{
\begin{tcolorbox}[center, boxrule=0.0 mm, width=\mywidth,arc=0mm,enhanced jigsaw,,colback=white,breakable]
#1	
\end{tcolorbox}
}

\newcommand{\info}[5]{
\begin{tcolorbox}[center, boxrule=0.1 mm, width=\mywidth,arc=0mm,enhanced jigsaw,breakable,colback=yellow!5]	
	
	\footnotesize
	\textbf{Øvingsområde}\\[5pt] #1 
	
	\textbf{Utstyr}\\ #2  \\
	
	\begin{tabular}{@{} p{4cm} p{4cm} l} 
		\textbf{Tid} & \textbf{Elevinndeling} & \textbf{Læringsarena} \\
		#3  & #4 & #5
	\end{tabular} 
\end{tcolorbox}	
}

\newcommand{\gjen}[1]{\begin{tcolorbox}[center,boxrule=0.1 mm, width=\mywidth,arc=0mm,colback=blue!3] {\large \textbf{Gjennomføring} \vspace{5 pt}}\newline #1  \end{tcolorbox}\vspace{-5pt}}
\newcommand{\eks}[1]{\begin{tcolorbox}[center,boxrule=0.1 mm, width=\mywidth,arc=0mm,colback=green!3] {\large \textbf{Eksempel} \vspace{5 pt}}\newline #1  \end{tcolorbox}\vspace{-5pt}}

\newcounter{opl}
%\numberwithin{opl}{article}


\newcommand{\opl}[1]{
\newpage
{\refstepcounter{opl} %\phantomsection 
\large \textbf{\theopl \;#1} \vsk}
}

% Headlines
\newcommand{\fork}{\textbf{Forkunnskapar}\\}
\newcommand{\forb}{\textbf{Forberedelsar}\\}
\newcommand{\opgvr}{\textbf{Oppgaver}}



%colors
\newcommand{\colr}[1]{{\color{red} #1}}
\newcommand{\colb}[1]{{\color{blue} #1}}
\newcommand{\colo}[1]{{\color{orange} #1}}
\newcommand{\colc}[1]{{\color{cyan} #1}}
\definecolor{projectgreen}{cmyk}{100,0,100,0}
\newcommand{\colg}[1]{{\color{projectgreen} #1}}

% Lister med bokstavar
\usepackage[inline]{enumitem}
% Opg
\newcommand{\abc}[1]{
	\begin{enumerate}[label=\alph*),leftmargin=18pt]
		#1
	\end{enumerate}
}

\usepackage[]{hyperref}

\begin{document}
\section{Addition}

\subsubsection{Column addition}
This method is founded on the base-10 positional notation, in turn adding the ones, the tens, the hundreds and so on.
\begin{center}
	\parbox{0.3\linewidth}{
\eks[1]{
	\begin{figure}
		\centering
		\includegraphics[]{rekfig/plus1}
	\end{figure}
}
}\qquad
\parbox{0.3\linewidth}{
\eks[2]{
	\begin{figure}
		\centering
		\includegraphics[]{rekfig/plus2}
	\end{figure}
}
}\\[12pt]
\parbox{0.3\linewidth}{
\eks[3]{
	\begin{figure}
		\centering
		\includegraphics[]{rekfig/plus3}
	\end{figure}
}}\qquad
\parbox{0.3\linewidth}{
\eks[4]{
	\begin{figure}
		\centering
		\includegraphics[]{rekfig/plus4}
	\end{figure}
}}
\end{center}
\fork{Example 1}{
\begin{figure}
	\centering
	\subfloat[]{\includegraphics{rekfig/plus1a}}\qquad
	\subfloat[]{\includegraphics{rekfig/plus1b}}\qquad
	\subfloat[]{\includegraphics{rekfig/plus1c}}
\end{figure}

\begin{enumerate}[label=\alph*)]
	\item We add the ones: $ 4+2=6 $
	\item We add the tens: $ 3+1=4 $
	\item We add the hundreds: $ 2+6=8 $
\end{enumerate}
} \newpage
\fork{Example 2}{
	\begin{figure}
		\centering
		\subfloat[]{\includegraphics{rekfig/plus2a}}\qquad
		\subfloat[]{\includegraphics{rekfig/plus2b}}\qquad
		\subfloat[]{\includegraphics{rekfig/plus2c}}
	\end{figure}
	
	\begin{enumerate}[label=\alph*)]
		\item We add the ones: $ 3+6=9 $
		\item We add the tens: $ {7+8=15} $. Since 10 tens equals 100, we add 1 to the hundreds position, and write the remaining 5 tens at the tens position.
		\item We add the hundreds: $ 1+2=3 $.
	\end{enumerate}
} \vsk

\spr{
Writing 1 on a place value to the left is calle ''carrying 1 over''.
}
\begin{comment}
	\subsubsection{The table method}
	This method starts with one of the terms and add numbers until the other term is reached. You are free to chose which numbers to add as long as you don't exceed the term you are supposed to reach.
	\begin{center}
		\parbox{0.3\linewidth}{
			\eks[1]{
				$ \colb{273}+\colc{86} = \colo{359} $ \vsk
				
				\begin{tabular}{r|r|r}
					&&\colb{273} \\ \hline
					6 & 6 & 279 \\
					30& 36 & 309 \\
					50& \colc{86} & \colo{359}
				\end{tabular}
			}
		} \qquad
		\parbox{0.3\linewidth}{
			\eks[2]{
				$ \colb{85}+\colc{79}=\colo{164} $  \vsk
				
				\begin{tabular}{r|r|r}
					& & \colb{85} \\ \hline 
					5 & 5 & 90 \\
					10 & 15 &100 \\
					64 & \colc{79} & \colo{164} \\
				\end{tabular} \vsk
			}
		}
	\end{center}
	\newpage
	\fork{Example 1}{
		\begin{figure}
			\centering
			\subfloat[]{
				\begin{tabular}{r|r|r}
					&&\colb{273} \\ \hline
					&  &  \\
					\phantom{30}& \phantom{36} &  \\
					&  & 
				\end{tabular}
			} \qquad
			\subfloat[]{
				\begin{tabular}{r|r|r}
					&&\colb{273} \\ \hline
					6& 6 & 279 \\
					\phantom{30}& \phantom{36} &  \\
					&  & 
				\end{tabular}
			}\vsk 
			
			\subfloat[]{
				\begin{tabular}{r|r|r}
					&&\colb{273} \\ \hline
					6& 6 & 279 \\
					30& 36 & 309  \\
					&  & 
				\end{tabular}
			}
			\qquad
			\subfloat[]{
				\begin{tabular}{r|r|r}
					&&\colb{273} \\ \hline
					6& 6 & 279 \\
					30& 36 & 309  \\
					50& \colc{86} & \colo{359}
				\end{tabular}
			}
		\end{figure}
		\begin{enumerate}[label=(\alph*)]
			\item We start with one of the terms. Often, it's wise choose the larger term.
			\item We add $ 6 $. In total we have added $ 6 $, and $ {273+6=279} $.
			\item We add 30. In total we have added 36, and $ 279+30=309 $.
			\item We add 50. In total we have added 86, which is our other term. Moreover, $ 309+50=359 $.
		\end{enumerate}
	} \vsk
	
	\info{Column addition versus the table method}{
		At first sight, the table method may seem like a somewhat cumbersome way of calculating addition. However, the flexibility it showcases when it comes to choosing numbers can be utilized when performing mental arithmetic. But the real strengths of the table method are most apparent when it comes to performing subtraction and division.
	}
\end{comment}

\section{Subtraction}
\subsubsection{Column subtraction}
This method is founded on the base-10 positional notation, in turn subtracting the ones, the tens, the hundreds and so on. It is also based on the perspective of numbers as amounts, so it does not allow negative differences. (see the explanation of \textsl{Example 2}).
\begin{center}
	\parbox{0.3\linewidth}{
\eks[1]{
	\begin{figure}
		\centering
		\includegraphics[]{rekfig/min1}
	\end{figure}
}} \qquad
\parbox{0.3\linewidth}{
\eks[2]{
	\begin{figure}
		\centering
		\includegraphics[]{rekfig/min2}
	\end{figure}
}} \\[12pt]
\parbox{0.3\linewidth}{
\eks[3]{
	\begin{figure}
		\centering
		\includegraphics[]{rekfig/min3}
	\end{figure}
}}\qquad
\parbox{0.3\linewidth}{
\eks[4]{
	\begin{figure}
		\centering
		\includegraphics[]{rekfig/min4}
	\end{figure}
}}

\end{center}
\fork{Example 1}{ \vs
\begin{figure}
	\centering
	\subfloat[]{\includegraphics{rekfig/min1a}}\qquad
	\subfloat[]{\includegraphics{rekfig/min1b}}\qquad
	\subfloat[]{\includegraphics{rekfig/min1c}}
\end{figure}

\begin{enumerate}[label=(\alph*)]
	\item We find the difference between the ones: $ {9-4=5} $
	\item We find the difference between the tens: $ {8-2=6} $. 
	\item We find the difference between the hundreds: $ {7-3=4} $.
\end{enumerate}
}
\newpage
\fork{Example 2}{ \vs
	\begin{figure}
		\centering
		\subfloat[]{\includegraphics{rekfig/min2a}}\qquad
		\subfloat[]{\includegraphics{rekfig/min2b}}
	\end{figure}
	
	\begin{enumerate}[label=(\alph*)]
		\item We notice that 7 is larger than 3, and thus we take 1 ten from the 9 at the tens position. This is marked by drawing a line across 9. Then we find the difference between the ones: $ {13-7=6} $
		\item Since we took 1 from the 9 tens, it is now only 8 tens. We find the difference between the tens: $ {8-5=3} $.
	\end{enumerate}
}
\subsubsection{The table method}
The table method takes advantage of subtraction being the opposite operation of addition. For example, the answer to the question ''What is $ 789-324 $?'' is the same as the answer to the question ''How much must i add to 324 in order to get 789?''. With the table method you can freely chose which numbers to chose as long as you end up with the targeted number.\\
\begin{center}
\parbox{0.35\linewidth}{
\eks[1]{
$ \colb{789}-\colr{324}=\colc{465} $	 \vsk

\begin{tabular}{r|r}
	& \colr{324} \\ \hline
	6&330 \\
	70&400 \\
	389&\colb{789} \\ \hline
	\colc{465}
\end{tabular}
}} \qquad
\parbox{0.35\linewidth}{
	\eks[2]{
		$ \colb{83}-\colr{67}=\colc{16} $	 \vsk
		
		\begin{tabular}{r|r}
			& \colr{67} \\ \hline
			3&70 \\
			13&\colb{83} \\ \hline
			\colc{16}
		\end{tabular} \vspace{14pt}
}} \\[12pt]
\parbox{0.35\linewidth}{
	\eks[3]{
		$ 564-478=86 $\vsk
		
		\begin{tabular}{r|r}
			& 478 \\ \hline
			2&480 \\
			20&500 \\ 
			64&564\\ \hline
			86
		\end{tabular}
}} \qquad 
\parbox{0.4\linewidth}{
	\eks[4]{
		$ {206,1-31,7=174,4} $\vsk
		
		\begin{tabular}{r|r}
			& 31,7 \\ \hline
			0,3& 32\phantom{,0} \\
			70\phantom{,0}&102\phantom{,0} \\ 
			104,1&206,1\\ \hline
			174,4
		\end{tabular}
}}
\end{center}
\fork{Example 1}{
	\[ \colb{789}-\colr{324}=\colo{465} \]
\begin{figure}
	\centering
	\subfloat[]{
	\begin{tabular}{r|r}
		& \colr{324} \\ \hline
		& \\
		& \\
		& \\ \hline
		&
	\end{tabular}	
} \qquad
	\subfloat[]{
	\begin{tabular}{r|r}
		& \colb{324} \\ \hline
	   \colb{6}& \colc{330}\\
		& \\
		& \\ \hline
		&
	\end{tabular}
}\qquad
	\subfloat[]{
	\begin{tabular}{r|r}
		& 324 \\ \hline
		6& \colb{330}\\
		\colb{70}& \colc{400} \\
		& \\ \hline
		&
	\end{tabular}	
}  \\[12pt]
\subfloat[]{
	\begin{tabular}{r|r}
		& 324 \\ \hline
		6& 330\\
		70& \colb{400} \\
		\colb{389}& \colc{789}\\ \hline
		&
	\end{tabular}	
}\qquad
\subfloat[]{
	\begin{tabular}{r|r}
		& 324 \\ \hline
		\colb{6}& 330\\
		\colb{70}& 400 \\
		\colb{389}& 789\\ \hline
		\colo{465}&
	\end{tabular}	
}
\end{figure}
\begin{enumerate}[label=(\alph*)]
	\item We start at 324.
	\item We add 6, and get $ {324+6=330} $
	\item We add 70, and get $ {70+330=400} $
	\item We add 389, and get $ {389+400=789} $. Now we have reached 789.
	\item We find the sum of the numbers we added: $ {6+70+389=465} $
\end{enumerate}
}
\section{Multiplication} \label{rekGanging}
\subsubsection{Multiplying by 10, 100, 1\,000 etc.}
\reg[Å gange heltall med 10, 100 osv. \label{gangheltallmed10100}]{
	\vs
	\begin{itemize}
		\item When multiplying an integer by 10, the product can be found by adding the digit 0 behind the integer.
		\item When multiplying an integer by 100, the product can be found by adding the digits 00 behind the integer.
		\item The same pattern applies for the numbers 1\,000, 10\,000 etc.
	\end{itemize}
}
\eks[1]{\vsb \vsb
	\alg{
		6\cdot \colb{10} &= 6\colb{0}\vn
		79\cdot \colb{10} &= 79\colb{0} \vn
		802\cdot\colb{10}&=802\colb{0}
	}
}
\eks[2]{ \vsb \vsb
\alg{ 
6\cdot\colb{100} &= 6\colb{00} \vn
79\cdot\colb{100} &= 7\,9\colb{00} \vn
802\cdot\colb{100} &=80\,2\colb{00}
}
}
\eks[3]{ \vsb \vsb
\alg{ 
	6\cdot\colb{1\,000} &= 6\,\colb{000} \vn
	79\cdot\colb{10\,000} &= 79\colb{0\,000} \vn
	802\cdot\colb{100\,000} &=80\,2\colb{00\,000}
}
}
\newpage
\reg[Multiplying decimal numbers by 10, 100, etc. \label{gangdesmed10100}]{
	\vs
	\begin{itemize}
		\item When multiplying an integer by 10, the product is found by moving the dot one position to the right. 
		\item When multiplying an integer by 100, the product is found by moving the dot one position to the right.
		\item The same pattern applies for the numbers 1\,000, 10\,000 etc.
	\end{itemize}
}
\eks[1]{\vsb \vsb
	\alg{
		7\colr{.}9\cdot 10 &= 79\colr{.}=79 \vn
		38\colr{.}02\cdot10&=380\colr{.}2 \vn
		0\colr{.}57\cdot 10 &=05\colr{.}7=5\colr{.}7 \vn
		0\colr{.}194\cdot 10&= 01\colr{.}94=1\colr{.}94
	}
}
\eks[2]{ \vsb \vsb
	\alg{
		7\colr{.}9\cdot 100 &= 790\colr{.}=790 \vn
		38\colr{.}02\cdot100&=3802\colr{.}=3\,802 \vn
		0\colr{.}57\cdot 100 &=057\colr{.}=57 \vn
		0\colr{.}194\cdot 100&= 019\colr{.}4=19\colr{.}4
	}
}
\eks[3]{ \vsb \vsb
	\alg{
	7\colr{.}9\cdot 1\,000 &= 7900\colr{.}=7\,900 \vn
	38\colr{.}02\cdot10\,000&=380020\colr{.}=380\,200 \vn
	0\colr{.}57\cdot 100\,000 &=57000\colr{.}=57\,000
}
}
\info{\note}{
\hrs{gangheltallmed10100}{Rule} is just a special case of \rref{gangdesmed10100}. For example, applying \rref{gangheltallmed10100} when calculating $ {7\cdot 10} $ yields the same answer as when applying \rref{gangdesmed10100} when calculating $ {7,0\cdot 10} $. 
}
\newpage
\fork{Multiplying by 10, 100 etc.}{
The Base-10 positional notation is founded on groups of tens, hundreds, thousands etc., and tenths, hundredths, thousandths etc. (see \rref{titalsys}). When multiplying a number by 10, all the ones in the number will form a group of tens, all the tens will form a group of hundreds an so on. Hence, every digit is moved one position to the left. Similarly, every digit is moved one position to the left when multiplying by 100, three places when multiplying by 1\,000 etc.
}
\subsubsection{Expanded form}
Multiplication with expanded form can be applied on multi digit numbers. The method is based on the distributive law (\rref{gangpar}). \regv
\eks[1]{ \vs
\begin{figure}
	\centering
	\includegraphics[]{rekfig/gang1}
\end{figure}
}
\eks[2]{ \vs
\begin{figure}
	\centering
	\includegraphics[]{rekfig/gfleirsif}
\end{figure}
}
\fork{Example 1}{
24 can be written as $ 20+4 $, so
\[ 24\cdot3 =(20+4)\cdot3 \]
Moreover, by \rref{gangpar},
\alg{
(20+4)\cdot 3 &=20\cdot 3 + 4\cdot 3 \\
&= 60+12 \\
&= 72
}
}
\newpage
\fork{Example 2}{
We have
\algv{
279&=200+70+9 \\
34 &=30+4 	
}
Thus
\[ 279\cdot34= (200+70+9)\cdot (30+4) \]
Moreover,
{
\footnotesize
\alg{
(200+70+9)\cdot (30+4) &=200\cdot 30+70\cdot30+9\cdot30+200\cdot4+70\cdot4+9\cdot4
\\
&=9486}
} \vs
}
\subsubsection{The compact method}
The compact method is based on the same principles as the expanded form method, only with a shorter way of writing. \regv

\eks[1]{
	\begin{figure}
		\centering
		\includegraphics[]{rekfig/gfleirsifa}
	\end{figure}
}
\newpage
\fork{Example 1}{
First we multiply the digits of 279 by 4:
\begin{itemize}
	\item $ 9\cdot 6=36 $, so we write 6 at the ones position and carry\\ over 3.
	\item $ 7\cdot4 =28$, so we write 8 at the tens position and carry\\ over 2.
	\item $ 2\cdot 4=8 $, so we write 8 at the hundreds position.
\end{itemize}
Then we multiply the digits of 279 by 30. This in turn can be simplified to multiplying by 3, as long as we shift the digits one position to the left, relative to when we multiplied by 4:
\begin{itemize}
	\item $ 9\cdot 3=27 $, so we write 7 at the tens position and carry \\ over 2. 
	\item $ 7\cdot3=21 $, so we write 1 at the hundreds position and carry \\ over 2.
	\item $ 2\cdot3=6 $, so we write 6 at the thousands position.
\end{itemize} 
}
\section{Division} \label{rekDivisjon}
\subsubsection{Division by 10, 100, 1\,000 etc.}
\reg[Deling med 10, 100, 1\,000 osv. \label{deledesmed10100}]{
When dividing a decimal number by 10, the quotient is found by moving the dot one position to the left.\vsk

When dividing a decimal number by 100, the quotient is found by moving the dot two positions to the left.\vsk

The same pattern applies for the numbers 1\,000, 10\,000 etc.
}
\eks[1]{ \vsb \vsb
	\alg{
200:10&=200\colr{.}0:10 \\&=20\colr{.}00\\&=20	\vn
45:10&=45\colr{.}0:10 \\&= 4\colr{.}50 \\&=4\colr{.}5
}
}
\eks[2]{ \vsb \vsb
	\alg{
		200:100&=200\colr{.}0:100 \\&=2\colr{.}000\\&=2	\vn
		45:100&=45\colr{.}0:100 \\&= 0\colr{.}450 \\&=0\colr{.}45
	}
}
\newpage
\eks[3]{ \vsb \vsb
\alg{
143\colr{.}7 :10 &= 14\colr{.}37 \vn
143\colr{.}7 :100 &= 1\colr{.}437 \vn
143\colr{.}7 :1\,000 &= 0\colr{.}1437 
}
}
\eks[4]{ \vsb \vsb
\alg{
93\colr{.}6:10 &= 9\colr{.}36 \vn
93\colr{.}6:100 &= 0\colr{.}936 \vn
93\colr{.}6:1\,000 &= 0\colr{.}0936
}
}
\fork{Division by 10, 100, 1\,000 osv.}{
The Base-10 positional notation is founded on groups of tens, hundreds, thousands etc., and tenths, hundredths, thousandths etc. (see \rref{titalsys}). When dividing a number by 10, all the ones in the number will form a group of tens, all the tens will form a group of ones an so on. Hence, every digit is moved one position to the right. Similarly, every digit is moved two positions to the right when multiplying by 100, three places when multiplying by 1\,000 etc.
}

\subsubsection{Long division}
Long division is based on the perspective of numbers as amount (see page \pageref{Divisjon}).

\begin{center}
	\parbox{0.3\linewidth}{
	\eks[1]{ \vsb
		\begin{figure}
			\centering
			\includegraphics[]{rekfig/del1}
		\end{figure} \vspace{18pt}
	}
}\qquad
\parbox{0.45\linewidth}{
	\eks[1]{ \vspace{-5pt}
		\begin{figure}
			\centering
			\includegraphics[]{rekfig/del2}
		\end{figure}
	}
}
\end{center}
\newpage
\fork{Example 1}{ \vs \vs
	\begin{figure}
		\centering
		\subfloat{\includegraphics{\figp{delalg}}}
		\qquad \subfloat{\includegraphics[angle=90]{\figp{delalg0}}}
	\end{figure}
	The above figure illustrate the amount 92, which we shall divide into 4 equal groups. 
	\begin{itemize}
		\item We start by distributing as many tens as possible. Of the 9 tens, each group can get 2. In total we have distributed $ 2\cdot 4=8 $ tens.
		\begin{figure}
			\centering
			\subfloat{\includegraphics{\figp{delalgaa}}}
			\qquad \subfloat{\includegraphics[angle=90]{\figp{delalga}}}
		\end{figure}
		\item Now we are left with 1 ten and 2 ones, which equals 12 ones. Of the 12 ones, each group can get 3. In total we have distributed $ 3\cdot 4= 12 $ ones.
		\begin{figure}
			\centering
			\includegraphics[angle=90]{\figp{delalgb}}
		\end{figure}
		\item The amount we started with, 92, is now equally distributed, and our calculation is done. In each group we got 23.
	\end{itemize}
}
\newpage
\subsubsection{The table method}
Tabellmetoden baserer seg på divisjon som omvendt operasjon av ganging. For Example er svaret på spørsmålet ''Hva er $ {76:4} $?'' det samme som svaret på spørsmålet ''Hvilket tall må jeg gange 4 med for å få 76?''. På samme vis som for tabellmetoden ved subtraksjon er det opp til en selv å velge passende tall for å nå målet.
\begin{center}
	\label{ekstbldiv}
	\parbox{0.35\linewidth}{
		\eks[1]{
			$ \colg{76}:\colb{4}=\colo{19} $	 \vsk
			
			\begin{tabular}{r|r|r}
				$ \cdot\, \colb{4} $&\\ \hline
				10&40&40 \\
				9& 36 &\colg{76} \\ \hline
				\colo{19}&
			\end{tabular}
		\vspace{42pt}
	}} \qquad
\parbox{0.35\linewidth}{
	\eks[2]{
		$ \colg{894}:\colb{3}=\colo{298} $	 \vsk
		
		\begin{tabular}{r|r|r}
			$ \cdot\, \colb{3} $&\\ \hline
			200& 600 &600 \\
			30&90 &690 \\
			30&90 &780 \\
			30& 90&870 \\
			8&24 &\colg{894} \\ \hline
			\colo{298} &
		\end{tabular}
}} \vsk

\parbox{0.415\linewidth}{
	\eks[3]{		
		$ 894:3=298 $	 \vsk
		
		\begin{tabular}{r|r|r}
			$ \cdot\, 3 $&\\ \hline
			300& 900&900 \\
			$ -2 $& $ -6 $ &894 \\ \hline
			298&
		\end{tabular} \vsk
	
\footnotesize	
\mer Samme regnestykke som i \textsl{Example 2}, men en annen utregning.
}
}
\end{center}
\newpage
\fork{Example 1}{
	Siden vi skal dele \colg{92} med \colb{4}, ganger vi med \colb{4} fram til vi når \colg{92}.
	\begin{figure}
		\centering
		\subfloat[]{
			\begin{tabular}{r|r|r}
				$ \cdot\, \colb{4} $&\\ \hline
				10&40&40 \\
				&& \\
				& &\\ \hline
				&
			\end{tabular}	
		} \qquad
		\subfloat[]{
			\begin{tabular}{r|r|r}
				$ \cdot\, \colb{4} $&\\ \hline
				10&40&40 \\
				10&40&80 \\
				&  & \\ \hline
				&
			\end{tabular}	
		} \\
		\subfloat[]{
			\begin{tabular}{r|r|r}
				$ \cdot\, \colb{4} $&\\ \hline
				10&40&40 \\
				10&40&80 \\
				3& 12 &\colg{92} \\ \hline
				&
			\end{tabular}	
		} \qquad
		\subfloat[]{
			\begin{tabular}{r|r|r}
				$ \cdot\, \colb{4} $&\\ \hline
				10&40&40 \\
				10&40&80 \\
				3& 12 &\colg{92} \\ \hline
				\colo{23}&
			\end{tabular}	
		}
	\end{figure}
	\begin{enumerate}[label=(\alph*)]
		\item Vi ganger 10 med 4, som er lik 40. Da har vi så langt kommet til 40.
		\item Vi ganger 10 med 4, som er lik 40. Da har vi så langt kommet til $ {40+40=80} $.
		\item Vi ganger 3 med 4, som er lik 12. Da har vi kommet til $ {80+12=92} $, som var målet.
		\item Vi legger sammen tallene vi ganget med, og får $ {10+10+3=23} $.
	\end{enumerate}
}
\info{Tips}{
	Det kan være lurt å se tilbake på utregninger gjort med tabellmetoden for å tenke over om man kunne valgt tall på en annen måte. I \textsl{Example 1} på side \pageref{ekstbldiv} kunne vi startet med å gange med 20. Dette er omtrent like enkelt som å gange med 10, og det ville ha brakt oss nærmere målet.
}
\newpage
\subsubsection{Divisjon med rest}
Det er langt ifra alltid at svaret ved divisjon blir et heltall. En måte å uttrykke slike svar på, er å ved å bruke begrepet \outl{rest}\index{rest}. Begrepet er best forklart ved Example: \regv
\eks[1]{
	Regn ut $ 11:4 $ med rest.
	
	\sv
	Det største heltallet vi kan gange med 4 uten at produktet blir større enn
	11, er 2. $ {2\cdot 4 = 8} $, så da har vi $ 11-8=3 $ i rest.
	\[ 11= \colb{2}\cdot 4 +\colc{3} \]	
	\fig{mod1}
	Dette betyr at
	\[ 11:4 =\text{\colb{2} og \colc{3} i rest}\]
}
\eks[2]{
	Regn ut $ 19:3 $ med rest.
	
	\sv
	Det største heltallet vi kan gange med 3 uten at produktet blir større enn
	19, er 6. $ {6\cdot 3 = 18} $, så da har vi $ 19-8=1 $ i rest.
	\[ 19= \colb{6}\cdot 3 +\colc{1} \]	
	\fig{mod2}
	Dette betyr at
	\[ 19:3 =\text{\colb{6} og \colc{1} i rest}\]
}
\newpage
\eks[3]{
	Regn ut $ 94:4 $ med rest.
	
	\sv
	\metode{Med oppstilling}{3.75cm}
	\[ 94:4 = \text{23 og 2 i rest} \]
	\begin{figure}
		\centering
		\includegraphics[]{rekfig/mod3}
	\end{figure}
	\mers{Da det blir feil å bruke \sym{$ = $} i figuren over, har vi valgt å bruke \sym{$  \rightarrow$}. 
	}\vsk 
	
	\metode{Med tabellmetoden}{3.75cm}
	\[ 94:4 = \text{23 og 2 i rest} \]
	\begin{center}
		\begin{tabular}{r|r|r}
			$ \cdot\, 4$&\\ \hline
			20&80&80 \\
			3& 12 &92 \\ \hline
			23&
		\end{tabular} \qquad $ 94-92=2 $
	\end{center}
}
\spr{
	Hvis vi utfører en \outl{modulo-operasjon}, finner vi resten i et delestykke. Dette blir ofte vist ved forkortingen \sym{mod}. For Example er
	\[ 11\text{ mod } 4= 3\qquad , \qquad 19\text{ mod 3} =1 \]
	I tillegg til \sym{\texttt{mod}}, blir også \sym{\texttt{\%}} og \sym{{\texttt{\string \\}}} brukt som symbol for denne operasjonen i programmeringsspråk.
}
\newpage
\subsubsection{Divisjon med blanda tall som svar}
\eks[1]{
	Regn ut $ 11:4 $. Skriv svaret som et blandet tall.
	
	\sv \vsb
	\[ 11:4 = \text{2 og 3 i rest}=2+\frac{3}{4} \]
}
\eks[2]{
	Regn ut $ 19:3 $. Skriv svaret som et blandet tall.
	
	\sv \vsb
	\[ 19:3=\text{6 og 1 i rest}=6+\frac{1}{3} \]
}
\fork{Example 1}{
	Vi starter med å legge merke til at $ 4=\frac{4}{1} $. Dette betyr at
	\[ 11:4 = 11:\frac{4}{1} \]
	Av \rref{delmbr} har vi at
	\[ 11:\frac{4}{1} = 11\cdot \frac{1}{4} \]
	Videre er $ 11=2\cdot 4+ 3 $, og da er
	\[ 11\cdot \frac{1}{4}=(2\cdot 4+3)\cdot\frac{1}{4} \]
	Av \rref{gangpar} har vi at
	\alg{
		(2\cdot 4+3)\cdot\frac{1}{4}&=2\cdot4\cdot\frac{1}{4}+3\cdot \frac{1}{4} \\
		&= 2+\frac{3}{4}
	}
}
\newpage
\subsection{Divisjon med desimaltall som svar}
\eks[1]{
	Regn ut $ 11:4 $. Oppgi svaret som desimaltal.
	
	\sv
	
	\metode{Med oppstilling}{3.75cm} \vs
	\[ 11:4=2,75 \]
	\begin{figure}
		\centering
		\includegraphics{rekfig/deldes1}
	\end{figure}
	\metode{Med tabellmetoden}{3.75cm} \vs
	\[ 11:4 = 2,75\]
	\begin{center}
		\begin{tabular}{r|r|r}
			$ \cdot\, 4$&\\ \hline
			2&8&8 \\
			0,5& 2 &10 \\
			0,25& 1 &11\\ \hline
			2,75&
		\end{tabular}
	\end{center}
}
\fork{Example 1; oppstilling}{
	Siden vi deler med 4, er det snakk om å fordele 11 likt i 4 grupper.
	\begin{itemize}
		\item 8 av de 11 enerne kan vi fordele likt i 4 grupper. Da har vi igjen 3 enere. Dette er det samme som 30 tideler.
		\item 28 av de 30 tidelene kan vi fordele likt i 4 grupper. Da har vi igjen 2 tideler. Dette er det samme som 20 hundredeler.
		\item 20 av de 20 hundredelene kan vi fordele likt i 4 grupper.
		\item Hele mengden 11 er nå fordelt, og da er vi ferdige med utegningen.
	\end{itemize}
}
\newpage
\section{Regning med tid \label{regningmedtid}}
Sekunder, minutter og timer er organisert i grupper på 60:
\alg{
	1\text{ minutt} &= 60\text{ sekund} \\
	1\text{ time} &= 60\text{ minutt} 
}
Dette betyr at \textsl{overganger} oppstår i utregninger når vi når 60.\regv

\eks[1]{
	$ \text{2\enh{t} 25\enh{min} } + \text{10\enh{t} 45\enh{min}}= \text{13\enh{t} 10\enh{min} } $\vsk
	
	\metode{Utrekningsmetode 1}{0.35\linewidth}
	\os
	\begin{tabular}{r|r|r}
		& &10\enh{t} 45\enh{min}  \\ \hline
		15\enh{min}  &15\enh{min} & 11\enh{t} 00\enh{min}  \\
		10\enh{min} &25\enh{min} & 11\enh{t} 10\enh{min} \\
		2\enh{t} & 2\enh{t} 25\enh{min}  & 13\enh{t} 10\enh{min}
	\end{tabular} \vsk \vsk
	
	\metode{Utrekningsmetode 2}{0.35\linewidth}\os
	\begin{tabular}{r|r|r}
		& & 10:45 \\ \hline 
		00:15 & 00:15 & 11:00 \\
		00:10 & 00:25 & 11:10 \\
		02:00 & 02:25 & 13:10
	\end{tabular}
} \regv

\eks[2]{
	$ \text{14\enh{t} 18\enh{min} } - \text{9\enh{t} 34\enh{min}}= \text{4\enh{t} 44\enh{min} } $\vsk
	
	\begin{center}
		\parbox{0.4\linewidth}{
			\metode{Utrekningsmetode 1}{0.9\linewidth} \os
			\begin{tabular}{r|r}
				&  9\enh{t} 34\enh{min} \\ \hline 
				26\enh{min} & 10\enh{t} 00\enh{min} \\
				18\enh{min} & 10\enh{t} 18\enh{min} \\
				4\enh{t} & 14\enh{t} 00\enh{min} \\ \hline
				4\enh{t} 44\enh{min}
			\end{tabular}
		} \qquad \qquad
		\parbox{0.4\linewidth}{
			\metode{Utrekningsmetode 2}{0.9\linewidth} \os
			\begin{tabular}{r|r}
				& 09:34 \\ \hline 
				00:26 & 10:00 \\
				00:18& 10:18 \\
				04:00 & 14:18 \\ \hline
				04:44 
			\end{tabular}
		}
	\end{center}
}


\section{Avrunding og overslagsregning}

\subsubsection{Avrunding}
Ved \outl{avrunding} av et tall minker vi antall siffer forskjellige fra 0 i tallet. Videre kan man runde av til \textsl{nærmeste ener}, \textsl{nærmeste tier} og lignende.\regv
\eks[1]{
	Ved avrunding til \textsl{nærmeste tier} avrundes
	\begin{itemize}
		\item 1, 2, 3 og 4 \textsl{ned} til 0 fordi de er nærmere 0 enn 10.
		\item 6, 7, 8 og 9 \textsl{opp} til 10 fordi de er nærmere 10 \\enn 0.
	\end{itemize}	
	5 avrundes også opp til 10.
	\fig{avrnd0}
}

\eks[2]{ \vs
	\begin{itemize}
		\item $\boldmath \textbf{63 avrundet til nærmeste tier} = 60 $ \\
		Dette fordi 63 er nærmere 60 enn 70.
		\fig{avrnda}
		\item $\boldmath \textbf{78 avrundet til nærmeste tier} = 80 $ \\
		Dette fordi 78 er nærmere 80 enn 70.
		\fig{avrndb}
		\item $\boldmath \textbf{359 avrundet til nærmeste hundrer} = 400 $\\
		Dette fordi 359 er nærmere 400 enn 300.
		\fig{avrndc}
		\item $ \boldmath \textbf{11,8 avrundet til nærmeste ener} = 12$ \\
		Dette fordi 11,8 er nærmere 12 enn 11.
		\fig{avrndd}
	\end{itemize}
}

\subsection{Overslagsregning}
Det er ikke alltid vi trenger å vite svaret på regnestykker helt nøyaktig, ofte er det viktigere at vi fort kan avgjøre hva svaret \textsl{omtrent} er det samme som, aller helst ved hoderegning. Når vi finner svar som omtrent er riktige, sier vi at vi gjør et \outl{overslag}. Et overslag innebærer at vi avrunder\footnote{\textit{Obs!} Avrunding ved overslag trenger ikke å innebære avrunding til nærmeste tier og lignende.} tallene som inngår i et regnestykke slik at utregningen blir enklere. \vsk


\spr{
	At noe er ''omtrent det samme som'' skriver vi ofte som ''cirka'' (''ca.''). Tegnet for ''cirka'' er \sym{$ \approx $}.
} 

\subsubsection{Overslag ved addisjon og ganging}
La oss gjøre et overslag på regnestykket
\[ 98,2+24,6 \]
Vi ser at $ 98,2 \approx 100 $. Skriver vi 100 istedenfor 98,2 i regnestykket vårt, får vi noe som er litt mer enn det nøyaktige svaret. Skal vi endre på 24,6 bør vi derfor gjøre det til et tall som er litt mindre. 24,6 er ganske nærme 20, så vi kan skrive
\[ 98,2+24,6 \approx 100 + 20 = 120 \]
Når vi gjør overslag på tall som legges sammen, bør vi altså prøve å gjøre det ene tallet større (runde opp) og et tall mindre (runde ned).\\

\linje
Det samme gjelder også hvis vi har ganging, for Example
\[ 1\,689\cdot12 \]
Her avrunder vi 12 til 10. For å ''veie opp'' for at svaret da blir litt mindre enn det egentlige, avrunder vi 1\,689 opp til 1\,700. Da får vi
\[ 1\,689\cdot12\approx 1\,700\cdot 10 =17\,000 \]
\subsubsection{Overslag ved subtraskjon og deling}
Skal et tall trekkes fra et annet, blir det litt annerledes. La oss gjøre et overslag på
\[ 186,4-28,9 \]
Hvis vi runder 186,4 opp til 190 får vi et svar som er større enn det egentlige, derfor bør vi også trekke ifra noe. Det kan vi gjøre ved også å runde 28,9 oppover (til 30):
\alg{
	186,4-28,9&\approx 190-30 \\&=160
}
Samme prinsippet gjelder for deling: 
\[ 145:17 \]
Vi avrunder 17 opp til 20. Deler vi noe med 20 istedenfor 17, blir svaret mindre. Derfor bør vi også runde 145 oppover (til 150):
\[ 145:17 \approx 150:20 = 75 \]

\subsubsection{Overslagsregning oppsummert}
\reg[Overslagsregning \label{tipsoverslag}]{ \vs
	\begin{itemize}
		\item Ved addisjon eller multiplikasjon mellom to tall, avrund gjerne et tall opp og et tall ned.
		\item Ved subtraksjon eller deling mellom to tall, avrund gjerne begge tall ned eller begge tall opp.
	\end{itemize}	
}
\eks[]{
	Rund av og finn omtrentlig svar for regnestykkene.\os
	
	\abch{
		\item $ {23,1+174,7} $ 
		\item $ {11,8\cdot107,2} $ 		
	} \os
	\abchs{3}{
		\item $ {37,4-18,9} $  \ \ 
		\item $ {1054:209} $
	}
	\vspace{-2pt}
	
	\sv  \vspace{-7pt}
	\abc{
		\item $ 32,1 + 174,7 \approx 30+170 = 200 $
		\item $ 11,8 \cdot 107,2 \approx 10\cdot110 = 1\,100 $
		\item $ 37,4 - 18,9 \approx 40-20 = 20 $
		\item $ 1\,054:209 \approx 1\,000:200 = 5 $
	}
} \vsk

\info{Kommentar
}{
	Det finnes ingen konkrete regler for hva man \textsl{kan} eller ikke \textsl{kan} tillate seg av forenklinger når man gjør et overslag, det som er kalt \rref{tipsoverslag} er strengt tatt ikke en regel, men et nyttig tips.\vsk
	
	Man kan også spørre seg hvor langt unna det faktiske svaret man kan tillate seg å være ved overslagsregning. Heller ikke dette er det noe fasitsvar på, men en grei føring er at overslaget og det faktiske svaret skal være av samme \outl{størrelsesorden}. Litt enkelt sagt betyr dette at hvis det faktiske svaret har med tusener å gjøre, bør også overslaget ha med tusener å gjøre. Mer nøyaktig sagt betyr det av det faktiske svaret og ditt overslag bør ha samme tierpotens når de er skrevet på standardform\footnote{Se \refsec{Standardform}}.
}
\newpage
\section{Standardform} \label{Standardform}
\textit{Obs! Denne seksjonen tar utgangspunkt i at leseren er kjent med potenser, som vi ser på i \refsec{Potensar}.}\vsk

Vi kan utnytte \rref{gangdesmed10100} og \rref{deledesmed10100}, og det vi kan om potenser, til å skrive tall på \outl{standardform}. \vsk

La oss se på tallet 6\,700. Av \rref{gangdesmed10100} vet vi at
\[ 6\,700=6,7\cdot1\,000 \]
Og siden $ 1000=10^3 $, er
\[ 6\,700=6,7\cdot1\,000=6,7\cdot 10^3 \]
\st{
	$ 6,7\cdot10^3 $ er 6\,700 skrevet på standardform fordi
	\begin{itemize}
		\item 6,7 er større enn 0 og mindre enn 10.
		\item $ 10^3 $ er en potens med grunntall 10 og eksponent 3, som er et heltall.
		\item 6,7 og $ 10^3 $ er ganget sammen.
	\end{itemize}
}
\linje \\[12pt]

La oss også se på tallet  0,093. Av \rref{deledesmed10100} har vi at
\[ 0,093=9,3: 100 \]
Men å dele med 100 er det samme som å gange med $ 10^{-2} $, altså er
\[ 0,093=9,3: 100=9,3\cdot10^{-2} \]
\st{
	$ 9,3\cdot10^{-2} $ er 0,093 skrevet på standardform fordi	
	\begin{itemize}
		\item 9,3 er større enn 0 og mindre enn 10.
		\item $ 10^{-2} $ er en potens med grunntall 10 og eksponent $ -2 $, som er et heltall.
		\item $ 9,3 $ og $ 10^{-2} $ er ganget sammen.
	\end{itemize} 
}
\reg[Standardform]{
	Et tall skrevet som
	\[ a\cdot 10^n \]
	hvor $ {0<a<10} $ og $ n $ er et heltall, er et tall skrevet på standardform.
}
\eks[1]{
	Skriv 980 på standardform.
	
	\sv \vsb
	\[ 980 = 9,8\cdot 10^2 \]
}
\eks[2]{
	Skriv 0,00671 på standardform.
	
	\sv \vsb
	\[ 0,00671 = 6,71\cdot 10^{-3} \]
}
\info{Tips}{
	For å skrive om tall på standardform kan du gjøre følgende:
	\begin{enumerate}
		\item Flytt komma slik at du får et tall som ligger mellom 0 og 10.
		\item Gang dette tallet med en tierpotens som har eksponent med tallverdi lik antallet plasser du flyttet komma. \qquad  Flyttet du komma mot venstre/høgre, er eksponenten positiv/negativ. 
	\end{enumerate}
}
\eks[3]{
	Skriv 9\,761\,432 på standardform.
	
	\sv \vs
	\begin{enumerate}
		\item 	Vi flytter komma 6 plasser til venstre, og får $ 9\colr{.}761432 $
		\item Vi ganger dette tallet med $ 10^6 $, og får at 
		\[ 9\,761\,432=9,761432\cdot 10^6 \] 
	\end{enumerate}
}
\newpage
\eks[4]{
	Skriv 0,00039 på standardform.
	
	\sv \vs
	\begin{enumerate}
		\item Vi flytter komma 4 plasser til høgre, og får $ 3,9 $.
		\item Vi ganger dette tallet med $ 10^{-4} $, og får at
		\[ 0,00039=3,9\cdot10^{-4} \]
	\end{enumerate}
}
\end{document}

