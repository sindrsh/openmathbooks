\documentclass[english,hidelinks,xetex, 11 pt, class=report,crop=false]{standalone}
\input{../../preamb_cyrillic}
% note

\newcommand{\note}{Примітка}
\newcommand{\notesm}[1]{{\footnotesize \textsl{\note:} #1}}
\newcommand{\selos}{Дивіться розв'язання.}
\newcommand{\ekstitle}{Приклад }
\newcommand{\sprtitle}{Мовна коробка}
\newcommand{\expl}{пояснення}

\newcommand{\texandasy}{Текст написано у \LaTeX, а фігури створено за допомогою Asymptote.}

\newcommand\sv{\vsk \textbf{Відповідь} \vspace{4 pt}\\}
	
	% Answers
	\newcommand{\mulansw}{\notesm{Декілька можливих відповідей.}}
	\newcommand{\faskap}{Розділ}
	
	% Exercises
	\newcommand{\opgt}{\newpage \phantomsection \addcontentsline{toc}{section}{Вправи} \section*{Вправи до розділу \thechapter}\vs \setcounter{section}{1}}
	
	%references
	\newcommand{\reftab}[1]{\hrs{#1}{таблиця}}
	\newcommand{\rref}[1]{\hrs{#1}{правило}}
	\newcommand{\dref}[1]{\hrs{#1}{визначення}}
	\newcommand{\refkap}[1]{\hrs{#1}{розділ}}
	\newcommand{\refsec}[1]{\hrs{#1}{секція}}
	\newcommand{\refdsec}[1]{\hrs{#1}{підрозділ}}
	\newcommand{\refved}[1]{\hrs{#1}{додаток}}
	\newcommand{\eksref}[1]{\textsl{#1}}
	\newcommand\fref[2][]{\hyperref[#2]{\textsl{рисунок \ref*{#2}#1}}}
	\newcommand{\refop}[1]{{\color{blue}Вправа \ref{#1}}}
	\newcommand{\refops}[1]{{\color{blue}вправа \ref{#1}}}
	
	%%% SECTION HEADLINES %%%
	
	% Our numbers
	\newcommand{\likteikn}{Знак рівності}
	\newcommand{\talsifverd}{Числа, цифри і значення}
	\newcommand{\koordsys}{Координатна система}
	
	% Calculations
	\newcommand{\adi}{Додавання}
	\newcommand{\sub}{Віднімання}
	\newcommand{\gong}{Множення (помноження)}
	\newcommand{\del}{Ділення (поділ)}
	
	%Factorization and order of operations
	\newcommand{\fak}{Розкладання на множники}
	\newcommand{\rrek}{Порядок дій}
	
	%Fractions
	\newcommand{\brgrpr}{Вступ}
	\newcommand{\brvu}{Значення, розширення та скорочення дробів}
	\newcommand{\bradsub}{Додавання і віднімання}
	\newcommand{\brgngheil}{Добуток дробу на ціле число}
	\newcommand{\brdelheil}{Частка дробу на ціле число}
	\newcommand{\brgngbr}{Добуток дробу на дріб}
	\newcommand{\brkans}{Скасування множників}
	\newcommand{\brdelmbr}{Частка на дріб}
	\newcommand{\Rasjtal}{Раціональні числа}
	
	%Negative numbers
	\newcommand{\negintro}{Вступ}
	\newcommand{\negrekn}{Чотири арифметичні операції з від'ємними числами}
	\newcommand{\negmeng}{Від'ємні числа як множина}
	
	% Geometry 1
	\newcommand{\omgr}{Поняття}
	\newcommand{\eignsk}{Властивості трикутників та чотирикутників}
	\newcommand{\omkr}{Периметр}
	\newcommand{\area}{Площа}
	
	%Algebra
	\newcommand{\algintro}{Вступ}
	\newcommand{\pot}{Показники}
	\newcommand{\irrasj}{Ірраціональні числа}
	
	%Equations
	\newcommand{\ligintro}{Вступ}
	\newcommand{\liglos}{Розв'язання за чотирма арифметичними операціями}
	\newcommand{\ligloso}{Підсумок методів розв'язання}
	
	%Functions
	\newcommand{\fintro}{Вступ}
	\newcommand{\lingraf}{Лінійні функції та графіки}
	
	%Geometry 2
	\newcommand{\geoform}{Формули для площі та периметру}
	\newcommand{\kongogsim}{Подібні та конгруентні трикутники}
	\newcommand{\geofork}{Пояснення}
	\newcommand{\pytomv}{\pyt, (зворотна версія)}
	
	% Names of rules
	\newcommand{\gangdestihundre}{Множення десяткових чисел на 10, 100 тощо}
	\newcommand{\delmedtihundre}{Ділення на 10, 100, 1,000 тощо}
	\newcommand{\adkom}{Комутативність додавання}
	\newcommand{\gangkom}{Комутативність множення}
	\newcommand{\brdef}{Дріб як перетворення ділення}
	\newcommand{\brtbr}{Добуток дробу на дріб}
	\newcommand{\delmbr}{Ділення дробу на дріб}
	\newcommand{\gangpar}{Множення з дужками (дистрибутивний закон)}
	\newcommand{\gangparsam}{Множен}




\begin{document}
\pagestyle{fancy}
\fancyhead{}
\fancyhead[C]{
\footnotesize Переклад з англійської на українську за допомогою ChatGPT. Можуть з'являтися дивні слова та речення.
}

\section{Додавання}

\subsubsection{Порядок дій}
Цей метод ґрунтується на системі розрядів, де ми почергово обчислюємо суму одиниць, десятків, сотень і так далі.
\begin{center}
	\parbox{0.3\linewidth}{
		\eks[1]{
			\begin{figure}
				\centering
				\includegraphics[]{rekfig/plus1}
			\end{figure}
		}
	}\qquad
	\parbox{0.3\linewidth}{
		\eks[2]{
			\begin{figure}
				\centering
				\includegraphics[]{rekfig/plus2}
			\end{figure}
		}
	}\\[12pt]
	\parbox{0.3\linewidth}{
		\eks[3]{
			\begin{figure}
				\centering
				\includegraphics[]{rekfig/plus3}
			\end{figure}
	}}\qquad
	\parbox{0.3\linewidth}{
		\eks[4]{
			\begin{figure}
				\centering
				\includegraphics[]{rekfig/plus4}
			\end{figure}
	}}
\end{center}
\fork{Приклад 1}{
	\begin{figure}
		\centering
		\subfloat[]{\includegraphics{rekfig/plus1a}}\qquad
		\subfloat[]{\includegraphics{rekfig/plus1b}}\qquad
		\subfloat[]{\includegraphics{rekfig/plus1c}}
	\end{figure}
	
	\begin{enumerate}[label=\alph*)]
		\item Ми додаємо одиниці: $ 4+2=6 $
		\item Ми додаємо десятки: $ 3+1=4 $
		\item Ми додаємо сотні: $ 2+6=8 $
	\end{enumerate}
} \newpage
\fork{Приклад 2}{
	\begin{figure}
		\centering
		\subfloat[]{\includegraphics{rekfig/plus2a}}\qquad
		\subfloat[]{\includegraphics{rekfig/plus2b}}\qquad
		\subfloat[]{\includegraphics{rekfig/plus2c}}
	\end{figure}
	
	\begin{enumerate}[label=\alph*)]
		\item Ми додаємо одиниці: $ 3+6=9 $
		\item Ми додаємо десятки: $ {7+8=15} $. Оскільки 10 десятків дорівнює 100, ми додаємо 1 на місце сотень і записуємо залишених 5 десятків на місце десятків.
		\item Ми додаємо сотні: $ 1+2=3 $.
	\end{enumerate}
} \vsk

\end{document}