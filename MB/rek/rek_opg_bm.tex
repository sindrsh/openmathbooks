\documentclass[english,hidelinks,pdftex, 11 pt, class=report,crop=false]{standalone}
\usepackage[T1]{fontenc}
\usepackage[utf8]{luainputenc}
\usepackage{lmodern} % load a font with all the characters
\usepackage{geometry}
\geometry{verbose,a4paper, inner=0cm, outer=0 cm, bmargin=2cm, tmargin=1cm}
%\textwidth=12cm
\setlength{\parindent}{0bp}
\usepackage{import}
\usepackage[subpreambles=false]{standalone}
\usepackage{amsmath}
\usepackage{amssymb}
\usepackage{esint}
\usepackage{babel}
\usepackage{tabu}
\usepackage[dvipsnames, table]{xcolor}
\usepackage{cancel}
\makeatother
\makeatletter
\usepackage{datetime2}
\usepackage{titlesec}
\usepackage[many]{tcolorbox}

% Eheter
\newcommand{\enh}[1]{\,\textrm{#1}}
%referances
\newcommand{\net}[2]{{\color{blue}\href{#1}{#2}}}

%Spaces
\newcommand{\vsk}{\\[12pt]}
\newcommand{\vs}{\vspace{-12pt}}

% Tabell for opplegg

\newcommand{\ovlist}[1]{
\vspace{-16pt}
\begin{itemize}
	#1
\end{itemize}
}

% Chapters and sections
\titleformat{\section}[block]{\bfseries}{\hspace{3cm}\thesection}{5pt}{}
\titleformat{\subsection}[block]{\bfseries}{\hspace{3cm}\thesection}{5pt}{}
\newcommand{\sectionbreak}{\clearpage} % New page on each section
 

\newlength{\mywidth}
\setlength{\mywidth}{14cm}

\newcommand{\cont}[1]{
\begin{tcolorbox}[center, boxrule=0.0 mm, width=\mywidth,arc=0mm,enhanced jigsaw,,colback=white,breakable]
#1	
\end{tcolorbox}
}

\newcommand{\info}[5]{
\begin{tcolorbox}[center, boxrule=0.1 mm, width=\mywidth,arc=0mm,enhanced jigsaw,breakable,colback=yellow!5]	
	
	\footnotesize
	\textbf{Øvingsområde}\\[5pt] #1 
	
	\textbf{Utstyr}\\ #2  \\
	
	\begin{tabular}{@{} p{4cm} p{4cm} l} 
		\textbf{Tid} & \textbf{Elevinndeling} & \textbf{Læringsarena} \\
		#3  & #4 & #5
	\end{tabular} 
\end{tcolorbox}	
}

\newcommand{\gjen}[1]{\begin{tcolorbox}[center,boxrule=0.1 mm, width=\mywidth,arc=0mm,colback=blue!3] {\large \textbf{Gjennomføring} \vspace{5 pt}}\newline #1  \end{tcolorbox}\vspace{-5pt}}
\newcommand{\eks}[1]{\begin{tcolorbox}[center,boxrule=0.1 mm, width=\mywidth,arc=0mm,colback=green!3] {\large \textbf{Eksempel} \vspace{5 pt}}\newline #1  \end{tcolorbox}\vspace{-5pt}}

\newcounter{opl}
%\numberwithin{opl}{article}


\newcommand{\opl}[1]{
\newpage
{\refstepcounter{opl} %\phantomsection 
\large \textbf{\theopl \;#1} \vsk}
}

% Headlines
\newcommand{\fork}{\textbf{Forkunnskapar}\\}
\newcommand{\forb}{\textbf{Forberedelsar}\\}
\newcommand{\opgvr}{\textbf{Oppgaver}}



%colors
\newcommand{\colr}[1]{{\color{red} #1}}
\newcommand{\colb}[1]{{\color{blue} #1}}
\newcommand{\colo}[1]{{\color{orange} #1}}
\newcommand{\colc}[1]{{\color{cyan} #1}}
\definecolor{projectgreen}{cmyk}{100,0,100,0}
\newcommand{\colg}[1]{{\color{projectgreen} #1}}

% Lister med bokstavar
\usepackage[inline]{enumitem}
% Opg
\newcommand{\abc}[1]{
	\begin{enumerate}[label=\alph*),leftmargin=18pt]
		#1
	\end{enumerate}
}

\usepackage[]{hyperref}

\newcommand{\note}{Merk}
\newcommand{\notesm}[1]{{\footnotesize \textsl{\note:} #1}}
\newcommand{\ekstitle}{Eksempel }
\newcommand{\sprtitle}{Språkboksen}
\newcommand{\expl}{forklaring}
\newcommand{\pyt}{Pytagoras' setning}
\newcommand\sv{\vsk \textbf{Svar} \vspace{4 pt}\\}

%references
\newcommand{\reftab}[1]{\hrs{#1}{tabell}}
\newcommand{\rref}[1]{\hrs{#1}{regel}}
\newcommand{\dref}[1]{\hrs{#1}{definisjon}}
\newcommand{\refkap}[1]{\hrs{#1}{kapittel}}
\newcommand{\refsec}[1]{\hrs{#1}{seksjon}}
\newcommand{\refdsec}[1]{\hrs{#1}{delseksjon}}
\newcommand{\refved}[1]{\hrs{#1}{vedlegg}}
\newcommand{\eksref}[1]{\textsl{#1}}
\newcommand\fref[2][]{\hyperref[#2]{\textsl{figur \ref*{#2}#1}}}
\newcommand{\refop}[1]{{\color{blue}Oppgave \ref{#1}}}
\newcommand{\refops}[1]{{\color{blue}oppgave \ref{#1}}}


%Algebra
\newcommand{\kvadset}{Kvadratsetningene}
\newcommand{\aenato}{Sum-produkt-metoden}

% Geometry
\newcommand{\hlikb}{Midtnormalen i en likebeint trekant}
\newcommand{\arealsetn}{Arealsetningen}
\newcommand{\trkmedian}{Median}
\newcommand{\midtrk}{Midtnormal (i trekant)}
\newcommand{\innskrsirk}{Innskrevet sirkel}
\newcommand{\cossetn}{Cosinussetningen}
\newcommand{\perfvink}{Sentral- og periferivinkel}
\newcommand{\tang}{Tangent}

% Derivative
\newcommand{\derel}{Den deriverte av elementære funksjoner}
\newcommand{\divder}{Divisjonsregelen}
\newcommand{\kjernereg}{Kjerneregelen}
\newcommand{\prodregder}{Produktregelen}
\newcommand{\lhop}{L'Hopitals regel}

% Funksjonsdrofting
\newcommand{\monder}{Monotoniegenskaper og den deriverte}
\newcommand{\fderekstr}{$ \bm{f'=0} $ for lokale ektstremalpunkt}
\newcommand{\andredertest}{Andrederiverttesten}

% Vectors
\newcommand{\detar}{Arealformler med determinanter}
\newcommand{\avstpunktlin}{Avstand mellom punkt og linje}

%Appendix
\newcommand{\rolle}{Rolles teorem}
\newcommand{\meanval}{Middelverdisetningen}

% Solutions manual
\newcommand{\selos}{Se løsningsforslag.}

\begin{document}
\opgt
\op{sumuovg}
Regn ut. \os
\abch{
\item $ 12+84 $
\item $ 36+51 $	
\item $ 328+571 $
\item $ 242+56 $
}

\op{summovg}
Regn ut. \os
\abch{
	\item $ 19+84 $
	\item $ 86+57 $	
	\item $ 529+471 $
	\item $ 202+808 $
}
\nes

\op{difuovg}
Regn ut. \os
\abch{
	\item $ 84-23 $
	\item $ 286-52 $	
	\item $ 529-401 $
	\item $ 782-131 $
}

\op{difmovg}
Regn ut. \os
\abch{
	\item $ 78-19 $
	\item $ 824-499 $	
	\item $ 731-208 $
	\item $ 1078-991 $
}
\nes

\op{gangetsif}
Regn ut. \os

\abch{
\item $ 12\cdot 3 $
\item $ 28\cdot 4 $	
\item $ 76\cdot 5 $
\item $ 43\cdot 6 $
} \os

\abchs{5}{
\item $ 109\cdot7 $
\item $ 98\cdot 8 $
\item $ 213\cdot 9 $
}

\op{gangtosif}
Regn ut. \os
\abch{
\item $ 29\cdot12 $
\item $ 83\cdot31 $
\item $ 91\cdot76 $
\item $ 14\cdot 83 $
}

\op{gangtretosif}
Regn ut. \os
\abch{
	\item $ 531\cdot56 $
	\item $ 83\cdot701 $
	\item $ 91\cdot673 $
	\item $ 731\cdot 67 $
}
\newpage
\op{gangdes1}
\abc{
\item Bruk kalkulator til å regne ut $ 27\cdot 5 $ og $ 2,7\cdot 5 $.
\item Bruk kalkulator til å regne ut $ 247\cdot 192 $ og $ 24,7\cdot 19,2 $.
\item Bruk kalkulator til å regne ut $ 928\cdot 74 $ og $ 9,28\cdot 7,4 $.
\item Bruk kalkulator til å regne ut $ 134\cdot4\,249 $ og $ 1,34\cdot42,49 $.
\item Sammenlign parene av svar fra oppgave a)\,-\,c), og lag en regel for hvordan du kan regne ut ganging med desimaltall.
}

\op{gangdes2}
Regn ut \os
\abch{
\item $ 82,3\cdot 5 $ \item $ 9,51\cdot 7 $ \item $ 22,4\cdot 1,7 $
}

\nes

\op{divtoensif}
Regn ut. \os
\abch{
	\item $ 98:2$
	\item $ 87:3 $
	\item $ 92:4 $
	\item $ 85:5 $
	\item $ 72:6 $
} 

\op{divtrensif}
Regn ut. \os
\abch{
	\item $ 378:2 $
	\item $ 224:4 $
	\item $ 495:5 $
}  \os

\abchs{5}{
\item $ 133:7$	
\item $ 208:8 $ \
\item $ 873:9 $
}

\nes


\op{stndform1}
Skriv tallet på standardform.\os
\abch{
	\item $ 98\,000 $ \ \ 
	\item $ 167\,000\,000 $
	\item $ 4\,819 $ \ \ \ \
	\item $ 21 $
} \os

\abchs{5}{
	\item $ 9\,132,27 $ 
	\item $ 893,7 $ \ \ \ \ \ \ \ \ 
	\item $ 18\,002,1 $
	\item $ 302,4 $
}

\newpage
\op{stndform2}
Skriv tallet på standardform.\os
\abch{
	\item $ 0,027 $
	\item $ 0,0001901 $
	\item $ 0,32 $
	\item $ 0,00000020032 $
} 

\op{stnddroft}
Gitt regnestykket
\[ 900\,000\,000\cdot 0,00007  \]
\abc{
	\item Forklar hvorfor regnestykket kan skrives som
	\[ 9\cdot 10^8 \cdot 7\cdot10^{-5} \]
	\item Bruk potensregler (se \refsec{Potensar}) og finn svaret på regnestykket fra a).
}

\eksop{T1H21D1}{T1H21D1opg1}
Rekn ut og skriv svaret på standardform
\[ \frac{6,2\cdot10 7 + 2,5\cdot10 8}{0,000002} \]

\newpage
\grubop{opgrekgangmed25}
Et tall kan ganges med 25 ved å
\begin{itemize}
\item dele tallet med 4
\item gange kvotienten med 100  
\end{itemize}
Metoden virker (selvsagt) best hvis tallet er delelig med 4.
\abc{
\item Forklar hvorfor denne metoden fungerer.
\item Forklar hvordan metoden kan brukes til å rekne ut $ 24^2 $.
}

\end{document}

