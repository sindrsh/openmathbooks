\documentclass[english,hidelinks,pdftex, 11 pt, class=report,crop=false]{standalone}
\usepackage[T1]{fontenc}
\usepackage[utf8]{luainputenc}
\usepackage{lmodern} % load a font with all the characters
\usepackage{geometry}
\geometry{verbose,a4paper, inner=0cm, outer=0 cm, bmargin=2cm, tmargin=1cm}
%\textwidth=12cm
\setlength{\parindent}{0bp}
\usepackage{import}
\usepackage[subpreambles=false]{standalone}
\usepackage{amsmath}
\usepackage{amssymb}
\usepackage{esint}
\usepackage{babel}
\usepackage{tabu}
\usepackage[dvipsnames, table]{xcolor}
\usepackage{cancel}
\makeatother
\makeatletter
\usepackage{datetime2}
\usepackage{titlesec}
\usepackage[many]{tcolorbox}

% Eheter
\newcommand{\enh}[1]{\,\textrm{#1}}
%referances
\newcommand{\net}[2]{{\color{blue}\href{#1}{#2}}}

%Spaces
\newcommand{\vsk}{\\[12pt]}
\newcommand{\vs}{\vspace{-12pt}}

% Tabell for opplegg

\newcommand{\ovlist}[1]{
\vspace{-16pt}
\begin{itemize}
	#1
\end{itemize}
}

% Chapters and sections
\titleformat{\section}[block]{\bfseries}{\hspace{3cm}\thesection}{5pt}{}
\titleformat{\subsection}[block]{\bfseries}{\hspace{3cm}\thesection}{5pt}{}
\newcommand{\sectionbreak}{\clearpage} % New page on each section
 

\newlength{\mywidth}
\setlength{\mywidth}{14cm}

\newcommand{\cont}[1]{
\begin{tcolorbox}[center, boxrule=0.0 mm, width=\mywidth,arc=0mm,enhanced jigsaw,,colback=white,breakable]
#1	
\end{tcolorbox}
}

\newcommand{\info}[5]{
\begin{tcolorbox}[center, boxrule=0.1 mm, width=\mywidth,arc=0mm,enhanced jigsaw,breakable,colback=yellow!5]	
	
	\footnotesize
	\textbf{Øvingsområde}\\[5pt] #1 
	
	\textbf{Utstyr}\\ #2  \\
	
	\begin{tabular}{@{} p{4cm} p{4cm} l} 
		\textbf{Tid} & \textbf{Elevinndeling} & \textbf{Læringsarena} \\
		#3  & #4 & #5
	\end{tabular} 
\end{tcolorbox}	
}

\newcommand{\gjen}[1]{\begin{tcolorbox}[center,boxrule=0.1 mm, width=\mywidth,arc=0mm,colback=blue!3] {\large \textbf{Gjennomføring} \vspace{5 pt}}\newline #1  \end{tcolorbox}\vspace{-5pt}}
\newcommand{\eks}[1]{\begin{tcolorbox}[center,boxrule=0.1 mm, width=\mywidth,arc=0mm,colback=green!3] {\large \textbf{Eksempel} \vspace{5 pt}}\newline #1  \end{tcolorbox}\vspace{-5pt}}

\newcounter{opl}
%\numberwithin{opl}{article}


\newcommand{\opl}[1]{
\newpage
{\refstepcounter{opl} %\phantomsection 
\large \textbf{\theopl \;#1} \vsk}
}

% Headlines
\newcommand{\fork}{\textbf{Forkunnskapar}\\}
\newcommand{\forb}{\textbf{Forberedelsar}\\}
\newcommand{\opgvr}{\textbf{Oppgaver}}



%colors
\newcommand{\colr}[1]{{\color{red} #1}}
\newcommand{\colb}[1]{{\color{blue} #1}}
\newcommand{\colo}[1]{{\color{orange} #1}}
\newcommand{\colc}[1]{{\color{cyan} #1}}
\definecolor{projectgreen}{cmyk}{100,0,100,0}
\newcommand{\colg}[1]{{\color{projectgreen} #1}}

% Lister med bokstavar
\usepackage[inline]{enumitem}
% Opg
\newcommand{\abc}[1]{
	\begin{enumerate}[label=\alph*),leftmargin=18pt]
		#1
	\end{enumerate}
}

\usepackage[]{hyperref}

% note
\newcommand{\note}{Note}
\newcommand{\notesm}[1]{{\footnotesize \textsl{\note:} #1}}
\newcommand{\selos}{See the solutions manual.}

\newcommand{\texandasy}{The text is written in \LaTeX\ and the figures are made with the aid of Asymptote.}

\newcommand{\rknut}{Calculate.}
\newcommand\sv{\vsk \textbf{Answer} \vspace{4 pt}\\}
\newcommand{\ekstitle}{Example }
\newcommand{\sprtitle}{The language box}
\newcommand{\expl}{explanation}

% answers
\newcommand{\mulansw}{\notesm{Multiple possible answers.}}	
\newcommand{\faskap}{Chapter}

% exercises
\newcommand{\opgt}{\newpage \phantomsection \addcontentsline{toc}{section}{Exercises} \section*{Exercises for Chapter \thechapter}\vs \setcounter{section}{1}}

% references
\newcommand{\reftab}[1]{\hrs{#1}{Table}}
\newcommand{\rref}[1]{\hrs{#1}{Rule}}
\newcommand{\dref}[1]{\hrs{#1}{Definition}}
\newcommand{\refkap}[1]{\hrs{#1}{Chapter}}
\newcommand{\refsec}[1]{\hrs{#1}{Section}}
\newcommand{\refdsec}[1]{\hrs{#1}{Subsection}}
\newcommand{\refved}[1]{\hrs{#1}{Appendix}}
\newcommand{\eksref}[1]{\textsl{#1}}
\newcommand\fref[2][]{\hyperref[#2]{\textsl{Figure \ref*{#2}#1}}}
\newcommand{\refop}[1]{{\color{blue}Exercise \ref{#1}}}
\newcommand{\refops}[1]{{\color{blue}Exercise \ref{#1}}}

%%% SECTION HEADLINES %%%

% Our numbers
\newcommand{\likteikn}{The equal sign}
\newcommand{\talsifverd}{Numbers, digits and values}
\newcommand{\koordsys}{Coordinate systems}

% Calculations
\newcommand{\adi}{Addition}
\newcommand{\sub}{Subtraction}
\newcommand{\gong}{Multiplication}
\newcommand{\del}{Division}

%Factorization and order of operations
\newcommand{\fak}{Factorization}
\newcommand{\rrek}{Order of operations}

%Fractions
\newcommand{\brgrpr}{Introduction}
\newcommand{\brvu}{Values, expanding and simplifying}
\newcommand{\bradsub}{Addition and subtraction}
\newcommand{\brgngheil}{Fractions multiplied by integers}
\newcommand{\brdelheil}{Fractions divided by integers}
\newcommand{\brgngbr}{Fractions multiplied by fractions}
\newcommand{\brkans}{Cancelation of fractions}
\newcommand{\brdelmbr}{Division by fractions}
\newcommand{\Rasjtal}{Rational numbers}

%Negative numbers
\newcommand{\negintro}{Introduction}
\newcommand{\negrekn}{The elementary operations}
\newcommand{\negmeng}{Negative numbers as amounts}

%Calculation methods
\newcommand{\delmedtihundre}{Deling med 10, 100, 1\,000 osv.}

% Geometry 1
\newcommand{\omgr}{Terms}
\newcommand{\eignsk}{Attributes of triangles and quadrilaterals}
\newcommand{\omkr}{Perimeter}
\newcommand{\area}{Area}

%Algebra 
\newcommand{\algintro}{Introduction}
\newcommand{\pot}{Powers}
\newcommand{\irrasj}{Irrational numbers}

%Equations
\newcommand{\ligintro}{Introduction}
\newcommand{\liglos}{Solving with the elementary operations}
\newcommand{\ligloso}{Solving with elementary operations summarized}

%Functions
\newcommand{\fintro}{Introduction}
\newcommand{\lingraf}{Linear functions and graphs}

%Geometry 2
\newcommand{\geoform}{Formulas of area and perimeter}
\newcommand{\kongogsim}{Congruent and similar triangles}
\newcommand{\geofork}{Explanations}

% Names of rules
\newcommand{\adkom}{Addition is commutative}
\newcommand{\gangkom}{Multiplication is commutative}
\newcommand{\brdef}{Fractions as rewriting of division}
\newcommand{\brtbr}{Fractions multiplied by fractions}
\newcommand{\delmbr}{Fractions divided by fractions}
\newcommand{\gangpar}{Distributive law}
\newcommand{\gangparsam}{Paranthesis multiplied together}
\newcommand{\gangmnegto}{Multiplication by negative numbers I}
\newcommand{\gangmnegtre}{Multiplication by negative numbers II}
\newcommand{\konsttre}{Unique construction of triangles}
\newcommand{\kongtre}{Congruent triangles}
\newcommand{\topv}{Vertical angles}
\newcommand{\trisum}{The sum of angles in a triangle}
\newcommand{\firsum}{The sum of angles in a quadrilateral}
\newcommand{\potgang}{Multiplication by powers}
\newcommand{\potdivpot}{Division by powers}
\newcommand{\potanull}{The special case of \boldmath $a^0$}
\newcommand{\potneg}{Powers with negative exponents}
\newcommand{\potbr}{Fractions as base}
\newcommand{\faktgr}{Factors as base}
\newcommand{\potsomgrunn}{Powers as base}
\newcommand{\arsirk}{The area of a circle}
\newcommand{\artrap}{The area of a trapezoid}
\newcommand{\arpar}{The area of a parallelogram}
\newcommand{\pyt}{Pythagoras's theorem}
\newcommand{\forform}{Ratios in similar triangles}
\newcommand{\vilkform}{Terms of similar triangles}
\newcommand{\omkrsirk}{The perimeter of a circle (and the value of $ \bm \pi $)}
\newcommand{\artri}{The area of a triangle}
\newcommand{\arrekt}{The area of a rectangle}
\newcommand{\liknflyt}{Moving terms across the equal sign}
\newcommand{\funklin}{Linear functions}


\begin{document}

\opgt 

\op{opgalggjentad}
Exploit the relation between repeated addition and multiplication (see \rref{ganggjad2} and \rref{gongneg}) to simplify the expressions. \os
\abch{
\item $ a+a+a $ \item $ a+a+a+a $ \item $ a+a+a+a+a+a+a $
} \vsk

\abchs{4}{
\item $ -b-b $
\item $ -b-b-b-b-b $
\item $ -k-k-k $
}

\op{opgalgadogsub}
Simplify the expressions. \os
\abch{
\item $ 2a+b-a $
\item $ -4a+2b+3a $
\item $ 7b-3a+2b $
} \vsk

\op{opgalgadogsub2}
Simplify the expressions. \os
\abch{
	\item $ 4c+2b-5a-3c $
	\item $ -9a-3c+3b+3c $
	\item $ 9b-3a+2b $
}

\op{opgalggong}
Use \rref{gangpar} to write the expression without parentheses.\os
\abch{
	\item $ 7(a+2) $
	\item $ 9(b+3) $
	\item $ 8(b-3c) $
	\item $ (-2)(3a+5b) $
} \\[12pt]

\abchs{5}{
	\item $ (9a+2) $
	\item $ (3b+8)a $
	\item $ (b-3c)(-a) $
	
} \\[12pt]
\abchs{8}{
	\item $ 2(a+3b+4c) $
	\item $ 9(3b-c+7a) $
	\item $ (3b-c+7a)(-2) $
}

\op{opgalgfaktoriser}
Use \rref{gangpar} to factorize the expression.\os
\abch{
\item $ 2a+2b $
\item $ 4ab+5b $
\item $ 9bc-c $
\item $ 4ac-2a $
}

\newpage
\op{opgalgkvad}
Prove that
\abc{
\item $ (a+b)^2=a^2+2ab+b^2 $
\item $ (a-b)^2=a^2-2ab+b^2 $
\item $ (a+b)(a-b)=a^2-b^2 $
}



\eksop{GV21D1}{GV21D1opg10}
\abc{
\item Simplify the expression.
\[ \frac{a+a+a+a}{4a} \]
\item What value will the expression $ \dfrac{y^2-2y}{y^2} $ attain if ${ x = 4} $ and $ {y = -2}$? 
}

\eksop{E22}{eksu22opg2}
Given the expression $ {(a+b)^2=16} $. Decide whether the below alternatives makes the expression valid.
\begin{itemize}
	\item $ a=2 $ and $ b=2 $
	\item $ a=8 $ and $ b=4 $
	\item $ a=8 $ and $ b=-4 $
\end{itemize}

\nes

\op{opgpot}
Write as a power. \os
\abch{
	\item $ 3\cdot3\cdot3\cdot3 $
	\item $ 5\cdot 5$
	\item $ 7\cdot7\cdot7\cdot7\cdot7\cdot7 $
} \vsk

\abchs{4}{
	\item $ a\cdot a\cdot a $
	\item $ b\cdot b $
	\item $ (-c)(-c)(-c)(-c) $
}

\op{opgpotverdi}
Find the value of the power.\os
\abch{
	\item $ 8^2 $
	\item $ 2^5 $
	\item $ 4^3 $
	\item $ (-2)^3$
	\item $ (-3)^5 $
	\item $ (-4)^4 $
} 

\op{opgpotskrivtilpot}
Write the expression as a power. \os
\abch{
	\item $ 2^7\cdot 2^9 $
	\item $ 3^4\cdot 3^7 $
	\item $ 9\cdot 9^5 $
	\item $ 6^8\cdot 6^{-3} $
	\item $ 5^3\cdot 5^{-7} $
} \\[12pt]

\abchs{6}{
	\item $ 10^8\cdot 10^{-3}\cdot 10^6 $
	\item $ a^9\cdot a^7 $
	\item $ k^5\cdot k^2 $
	\item $ x^5\cdot x^{-2} $
} \\[12pt]

\abchs{11}{
	\item $ x^{-4}\cdot x^{5}$
	\item $ a^{-5}\cdot a\cdot a^4 $
	\item $ a^{3}\cdot b^5\cdot a^2\cdot b^{-8} $
}

\op{opgrotter}
\rknut.\os
\abch{
	\item $ \sqrt{25} $
	\item $ \sqrt{100} $
	\item $ \sqrt{144} $
} \\[10pt]
\abchs{4}{
	\item $ \sqrt[3]{27} $
	\item $ \sqrt[3]{729} $
	\item $ \sqrt[5]{100000} $
}
\newpage
\grubop{1TH21D1opg6} 
(1TH21D1) \\%opg 6
Simplify the expression
\[ \frac{9^{\frac{1}{2}}\cdot 3^{-1}+9^0}{8^{\frac{3}{4}}} \]

\grubop{opgalgtverrsum}
By adding the digits of a number, we find the \outl{digit sum} of the number. For example is the digit sum of $ 14 $ equal to $ 1+4=5 $, and the digit sum of $ 918 $ equals $ 9+1+8=18 $. Prove that if the digit sum of a 3-digit integer is divisible by 3, then the number is also divisible by 3.\vsk

\mers{Generalizing the 3-digit case is quite easy, thus proving that the rule is valid for an integer with any number of digits.}
\end{document}


