\documentclass[english,hidelinks,pdftex, 11 pt, class=report,crop=false]{standalone}
\usepackage[T1]{fontenc}
\usepackage[utf8]{luainputenc}
\usepackage{lmodern} % load a font with all the characters
\usepackage{geometry}
\geometry{verbose,a4paper, inner=0cm, outer=0 cm, bmargin=2cm, tmargin=1cm}
%\textwidth=12cm
\setlength{\parindent}{0bp}
\usepackage{import}
\usepackage[subpreambles=false]{standalone}
\usepackage{amsmath}
\usepackage{amssymb}
\usepackage{esint}
\usepackage{babel}
\usepackage{tabu}
\usepackage[dvipsnames, table]{xcolor}
\usepackage{cancel}
\makeatother
\makeatletter
\usepackage{datetime2}
\usepackage{titlesec}
\usepackage[many]{tcolorbox}

% Eheter
\newcommand{\enh}[1]{\,\textrm{#1}}
%referances
\newcommand{\net}[2]{{\color{blue}\href{#1}{#2}}}

%Spaces
\newcommand{\vsk}{\\[12pt]}
\newcommand{\vs}{\vspace{-12pt}}

% Tabell for opplegg

\newcommand{\ovlist}[1]{
\vspace{-16pt}
\begin{itemize}
	#1
\end{itemize}
}

% Chapters and sections
\titleformat{\section}[block]{\bfseries}{\hspace{3cm}\thesection}{5pt}{}
\titleformat{\subsection}[block]{\bfseries}{\hspace{3cm}\thesection}{5pt}{}
\newcommand{\sectionbreak}{\clearpage} % New page on each section
 

\newlength{\mywidth}
\setlength{\mywidth}{14cm}

\newcommand{\cont}[1]{
\begin{tcolorbox}[center, boxrule=0.0 mm, width=\mywidth,arc=0mm,enhanced jigsaw,,colback=white,breakable]
#1	
\end{tcolorbox}
}

\newcommand{\info}[5]{
\begin{tcolorbox}[center, boxrule=0.1 mm, width=\mywidth,arc=0mm,enhanced jigsaw,breakable,colback=yellow!5]	
	
	\footnotesize
	\textbf{Øvingsområde}\\[5pt] #1 
	
	\textbf{Utstyr}\\ #2  \\
	
	\begin{tabular}{@{} p{4cm} p{4cm} l} 
		\textbf{Tid} & \textbf{Elevinndeling} & \textbf{Læringsarena} \\
		#3  & #4 & #5
	\end{tabular} 
\end{tcolorbox}	
}

\newcommand{\gjen}[1]{\begin{tcolorbox}[center,boxrule=0.1 mm, width=\mywidth,arc=0mm,colback=blue!3] {\large \textbf{Gjennomføring} \vspace{5 pt}}\newline #1  \end{tcolorbox}\vspace{-5pt}}
\newcommand{\eks}[1]{\begin{tcolorbox}[center,boxrule=0.1 mm, width=\mywidth,arc=0mm,colback=green!3] {\large \textbf{Eksempel} \vspace{5 pt}}\newline #1  \end{tcolorbox}\vspace{-5pt}}

\newcounter{opl}
%\numberwithin{opl}{article}


\newcommand{\opl}[1]{
\newpage
{\refstepcounter{opl} %\phantomsection 
\large \textbf{\theopl \;#1} \vsk}
}

% Headlines
\newcommand{\fork}{\textbf{Forkunnskapar}\\}
\newcommand{\forb}{\textbf{Forberedelsar}\\}
\newcommand{\opgvr}{\textbf{Oppgaver}}



%colors
\newcommand{\colr}[1]{{\color{red} #1}}
\newcommand{\colb}[1]{{\color{blue} #1}}
\newcommand{\colo}[1]{{\color{orange} #1}}
\newcommand{\colc}[1]{{\color{cyan} #1}}
\definecolor{projectgreen}{cmyk}{100,0,100,0}
\newcommand{\colg}[1]{{\color{projectgreen} #1}}

% Lister med bokstavar
\usepackage[inline]{enumitem}
% Opg
\newcommand{\abc}[1]{
	\begin{enumerate}[label=\alph*),leftmargin=18pt]
		#1
	\end{enumerate}
}

\usepackage[]{hyperref}

\newcommand{\note}{Merk}
\newcommand{\notesm}[1]{{\footnotesize \textsl{\note:} #1}}
\newcommand{\ekstitle}{Eksempel }
\newcommand{\sprtitle}{Språkboksen}
\newcommand{\expl}{forklaring}
\newcommand{\pyt}{Pytagoras' setning}
\newcommand\sv{\vsk \textbf{Svar} \vspace{4 pt}\\}

%references
\newcommand{\reftab}[1]{\hrs{#1}{tabell}}
\newcommand{\rref}[1]{\hrs{#1}{regel}}
\newcommand{\dref}[1]{\hrs{#1}{definisjon}}
\newcommand{\refkap}[1]{\hrs{#1}{kapittel}}
\newcommand{\refsec}[1]{\hrs{#1}{seksjon}}
\newcommand{\refdsec}[1]{\hrs{#1}{delseksjon}}
\newcommand{\refved}[1]{\hrs{#1}{vedlegg}}
\newcommand{\eksref}[1]{\textsl{#1}}
\newcommand\fref[2][]{\hyperref[#2]{\textsl{figur \ref*{#2}#1}}}
\newcommand{\refop}[1]{{\color{blue}Oppgave \ref{#1}}}
\newcommand{\refops}[1]{{\color{blue}oppgave \ref{#1}}}


%Algebra
\newcommand{\kvadset}{Kvadratsetningene}
\newcommand{\aenato}{Sum-produkt-metoden}

% Geometry
\newcommand{\hlikb}{Midtnormalen i en likebeint trekant}
\newcommand{\arealsetn}{Arealsetningen}
\newcommand{\trkmedian}{Median}
\newcommand{\midtrk}{Midtnormal (i trekant)}
\newcommand{\innskrsirk}{Innskrevet sirkel}
\newcommand{\cossetn}{Cosinussetningen}
\newcommand{\perfvink}{Sentral- og periferivinkel}
\newcommand{\tang}{Tangent}

% Derivative
\newcommand{\derel}{Den deriverte av elementære funksjoner}
\newcommand{\divder}{Divisjonsregelen}
\newcommand{\kjernereg}{Kjerneregelen}
\newcommand{\prodregder}{Produktregelen}
\newcommand{\lhop}{L'Hopitals regel}

% Funksjonsdrofting
\newcommand{\monder}{Monotoniegenskaper og den deriverte}
\newcommand{\fderekstr}{$ \bm{f'=0} $ for lokale ektstremalpunkt}
\newcommand{\andredertest}{Andrederiverttesten}

% Vectors
\newcommand{\detar}{Arealformler med determinanter}
\newcommand{\avstpunktlin}{Avstand mellom punkt og linje}

%Appendix
\newcommand{\rolle}{Rolles teorem}
\newcommand{\meanval}{Middelverdisetningen}

% Solutions manual
\newcommand{\selos}{Se løsningsforslag.}
\begin{document}

\opgt 

\op{opgalggjentad}
Utnytt koblingen mellom gjentatt addisjon og multiplikasjon (se \rref{ganggjad2} og \rref{gongneg}) til å skrive uttrykkene mer kompakt. \os
\abch{
\item $ a+a+a $ \item $ a+a+a+a $ \item $ a+a+a+a+a+a+a $
} \vsk

\abchs{4}{
\item $ -b-b $
\item $ -b-b-b-b-b $
\item $ -k-k-k $
}

\op{opgalgadogsub}
Skriv uttrykkene så kompakt som mulig \os
\abch{
\item $ 2a+b-a $
\item $ -4a+2b+3a $
\item $ 7b-3a+2b $
} \vsk

\op{opgalgadogsub2}
Skriv uttrykkene så kompakt som mulig \os
\abch{
	\item $ 4c+2b-5a-3c $
	\item $ -9a-3c+3b+3c $
	\item $ 9b-3a+2b $
}

\op{opgalggong}
Bruk \rref{gangpar} til å skrive om uttrykket til et uttrykk uten \\paranteser.\os
\abch{
	\item $ 7(a+2) $
	\item $ 9(b+3) $
	\item $ 8(b-3c) $
	\item $ (-2)(3a+5b) $
} \\[12pt]

\abchs{5}{
	\item $ (9a+2) $
	\item $ (3b+8)a $
	\item $ (b-3c)(-a) $
	
} \\[12pt]
\abchs{8}{
	\item $ 2(a+3b+4c) $
	\item $ 9(3b-c+7a) $
	\item $ (3b-c+7a)(-2) $
}

\op{opgalgfaktoriser}
Bruk \rref{gangpar} til å faktorisere uttrykket.\os
\abch{
\item $ 2a+2b $
\item $ 4ab+5b $
\item $ 9bc-c $
\item $ 4ac-2a $
}
\newpage
\eksop{GV21D1}{GV21D1opg10}
\abc{
\item Skriv så enkelt som mulig.
\[ \frac{a+a+a+a}{4a} \]
\item Hvilken verdi har uttrykket $ \dfrac{y^2-2y}{y^2} $ dersom ${ x = 4} $ og $ {y = -2}$? 
}

\eksop{E22}{eksu22opg2}
Gitt uttrykket $ {(a+b)^2=16} $.
Vurder om alternativene nedenfor gjør at uttrykket stemmer.
\begin{itemize}
	\item $ a=2 $ og $ b=2 $
	\item $ a=8 $ og $ b=4 $
	\item $ a=8 $ og $ b=-4 $
\end{itemize}

\nes

\op{opgpot}
Skriv som potenstall \os
\abch{
	\item $ 3\cdot3\cdot3\cdot3 $
	\item $ 5\cdot 5$
	\item $ 7\cdot7\cdot7\cdot7\cdot7\cdot7 $
} \vsk

\abchs{4}{
	\item $ a\cdot a\cdot a $
	\item $ b\cdot b $
	\item $ (-c)(-c)(-c)(-c) $
}

\op{opgpotverdi}
Finn verdien til potenstallet.\os
\abch{
	\item $ 8^2 $
	\item $ 2^5 $
	\item $ 4^3 $
	\item $ (-2)^3$
	\item $ (-3)^5 $
	\item $ (-4)^4 $
} 

\op{opgpotskrivtilpot}
Skriv om uttrykket til et potenstall. \os
\abch{
	\item $ 2^7\cdot 2^9 $
	\item $ 3^4\cdot 3^7 $
	\item $ 9\cdot 9^5 $
	\item $ 6^8\cdot 6^{-3} $
	\item $ 5^3\cdot 5^{-7} $
} \\[12pt]

\abchs{6}{
	\item $ 10^8\cdot 10^{-3}\cdot 10^6 $
	\item $ a^9\cdot a^7 $
	\item $ k^5\cdot k^2 $
	\item $ x^5\cdot x^{-2} $
} \\[12pt]

\abchs{11}{
	\item $ x^{-4}\cdot x^{5}$
	\item $ a^{-5}\cdot a\cdot a^4 $
	\item $ a^{3}\cdot b^5\cdot a^2\cdot b^{-8} $
}

\newpage
\op{opgrotter}
Regn ut.\os
\abch{
	\item $ \sqrt{25} $
	\item $ \sqrt{100} $
	\item $ \sqrt{144} $
} \\[10pt]
\abchs{4}{
	\item $ \sqrt[3]{27} $
	\item $ \sqrt[3]{729} $
	\item $ \sqrt[5]{100000} $
}
\newpage
\grubop{1TH21D1opg6} 
(1TH21D1) \\%opg 6
Skriv så enkelt som mulig
\[ \frac{9^{\frac{1}{2}}\cdot 3^{-1}+9^0}{8^{\frac{3}{4}}} \]

\grubop{opgalgtverrsum}
Ved å addere sifrene i et tall, finner vi \outl{tverrsummen} til tallet. For eksempel er tverrsummen til $ 14 $ lik $ 1+4=5 $, og tverrsummen til $ 918 $ er lik $ 9+1+8=18 $. Vis at hvis tverrsummen i et tresifret heltall er delelig med 3, så er også tallet delelig med 3.\vsk

\mers{Det er ganske lett å generalisere dette tilfellet, og slik vise at det gjelder for et heltall med et hvilket som helst antall siffer.}
\end{document}


