\documentclass[english,hidelinks,pdftex, 11 pt, class=report,crop=false]{standalone}
\usepackage[T1]{fontenc}
\usepackage[utf8]{luainputenc}
\usepackage{lmodern} % load a font with all the characters
\usepackage{geometry}
\geometry{verbose,a4paper, inner=0cm, outer=0 cm, bmargin=2cm, tmargin=1cm}
%\textwidth=12cm
\setlength{\parindent}{0bp}
\usepackage{import}
\usepackage[subpreambles=false]{standalone}
\usepackage{amsmath}
\usepackage{amssymb}
\usepackage{esint}
\usepackage{babel}
\usepackage{tabu}
\usepackage[dvipsnames, table]{xcolor}
\usepackage{cancel}
\makeatother
\makeatletter
\usepackage{datetime2}
\usepackage{titlesec}
\usepackage[many]{tcolorbox}

% Eheter
\newcommand{\enh}[1]{\,\textrm{#1}}
%referances
\newcommand{\net}[2]{{\color{blue}\href{#1}{#2}}}

%Spaces
\newcommand{\vsk}{\\[12pt]}
\newcommand{\vs}{\vspace{-12pt}}

% Tabell for opplegg

\newcommand{\ovlist}[1]{
\vspace{-16pt}
\begin{itemize}
	#1
\end{itemize}
}

% Chapters and sections
\titleformat{\section}[block]{\bfseries}{\hspace{3cm}\thesection}{5pt}{}
\titleformat{\subsection}[block]{\bfseries}{\hspace{3cm}\thesection}{5pt}{}
\newcommand{\sectionbreak}{\clearpage} % New page on each section
 

\newlength{\mywidth}
\setlength{\mywidth}{14cm}

\newcommand{\cont}[1]{
\begin{tcolorbox}[center, boxrule=0.0 mm, width=\mywidth,arc=0mm,enhanced jigsaw,,colback=white,breakable]
#1	
\end{tcolorbox}
}

\newcommand{\info}[5]{
\begin{tcolorbox}[center, boxrule=0.1 mm, width=\mywidth,arc=0mm,enhanced jigsaw,breakable,colback=yellow!5]	
	
	\footnotesize
	\textbf{Øvingsområde}\\[5pt] #1 
	
	\textbf{Utstyr}\\ #2  \\
	
	\begin{tabular}{@{} p{4cm} p{4cm} l} 
		\textbf{Tid} & \textbf{Elevinndeling} & \textbf{Læringsarena} \\
		#3  & #4 & #5
	\end{tabular} 
\end{tcolorbox}	
}

\newcommand{\gjen}[1]{\begin{tcolorbox}[center,boxrule=0.1 mm, width=\mywidth,arc=0mm,colback=blue!3] {\large \textbf{Gjennomføring} \vspace{5 pt}}\newline #1  \end{tcolorbox}\vspace{-5pt}}
\newcommand{\eks}[1]{\begin{tcolorbox}[center,boxrule=0.1 mm, width=\mywidth,arc=0mm,colback=green!3] {\large \textbf{Eksempel} \vspace{5 pt}}\newline #1  \end{tcolorbox}\vspace{-5pt}}

\newcounter{opl}
%\numberwithin{opl}{article}


\newcommand{\opl}[1]{
\newpage
{\refstepcounter{opl} %\phantomsection 
\large \textbf{\theopl \;#1} \vsk}
}

% Headlines
\newcommand{\fork}{\textbf{Forkunnskapar}\\}
\newcommand{\forb}{\textbf{Forberedelsar}\\}
\newcommand{\opgvr}{\textbf{Oppgaver}}



%colors
\newcommand{\colr}[1]{{\color{red} #1}}
\newcommand{\colb}[1]{{\color{blue} #1}}
\newcommand{\colo}[1]{{\color{orange} #1}}
\newcommand{\colc}[1]{{\color{cyan} #1}}
\definecolor{projectgreen}{cmyk}{100,0,100,0}
\newcommand{\colg}[1]{{\color{projectgreen} #1}}

% Lister med bokstavar
\usepackage[inline]{enumitem}
% Opg
\newcommand{\abc}[1]{
	\begin{enumerate}[label=\alph*),leftmargin=18pt]
		#1
	\end{enumerate}
}

\usepackage[]{hyperref}
%%% SECTION HEADLINES %%%

% note
\newcommand{\note}{Merk}
\newcommand{\notesm}[1]{{\footnotesize \textsl{\note:} #1}}
\newcommand{\selos}{Se løsnigsforslag.}
\newcommand{\ekstitle}{Eksempel }
\newcommand{\sprtitle}{Språkboksen}
\newcommand{\expl}{forklaring}

\newcommand\sv{\vsk \textbf{Answer} \vspace{4 pt}\\}

%references
\newcommand{\reftab}[1]{\hrs{#1}{tabell}}
\newcommand{\rref}[1]{\hrs{#1}{regel}}
\newcommand{\dref}[1]{\hrs{#1}{definisjon}}
\newcommand{\refkap}[1]{\hrs{#1}{kapittel}}
\newcommand{\refsec}[1]{\hrs{#1}{seksjon}}
\newcommand{\refdsec}[1]{\hrs{#1}{delseksjon}}
\newcommand{\refved}[1]{\hrs{#1}{vedlegg}}
\newcommand{\eksref}[1]{\textsl{#1}}
\newcommand\fref[2][]{\hyperref[#2]{\textsl{figur \ref*{#2}#1}}}
\newcommand{\refop}[1]{{\color{blue}Oppgave \ref{#1}}}
\newcommand{\refops}[1]{{\color{blue}oppgave \ref{#1}}}


% Our numbers
\newcommand{\likteikn}{Likskapsteiknet}
\newcommand{\talsifverd}{Tal, siffer og verdi}
\newcommand{\koordsys}{Koordinatsystem}

% Calculations
\newcommand{\adi}{Addisjon}
\newcommand{\sub}{Subtraksjon}
\newcommand{\gong}{Multiplikasjon (Gonging)}
\newcommand{\del}{Divisjon (deling)}

%Factorization and order of operations
\newcommand{\fak}{Faktorisering}
\newcommand{\rrek}{Reknerekkefølge}

%Fractions
\newcommand{\brgrpr}{Introduksjon}
\newcommand{\brvu}{Verdi, utviding og forkorting av brøk}
\newcommand{\bradsub}{Addisjon og subtraksjon}
\newcommand{\brgngheil}{Brøk gonga med heiltal}
\newcommand{\brdelheil}{Brøk delt med heiltal}
\newcommand{\brgngbr}{Brøk gonga med brøk}
\newcommand{\brkans}{Kansellering av faktorar}
\newcommand{\brdelmbr}{Deling med brøk}
\newcommand{\Rasjtal}{Rasjonale og blanda tal}

%Negative numbers
\newcommand{\negintro}{Introduksjon}
\newcommand{\negrekn}{Dei fire rekneartane med negative tal}
\newcommand{\negmeng}{Negative tal som mengde}

% Geometry 1
\newcommand{\omgr}{Omgrep}
\newcommand{\eignsk}{Eigenskapar for trekantar og firkantar}
\newcommand{\omkr}{Omkrins}
\newcommand{\area}{Areal}
\newcommand{\pytomv}{\pyt\, (omvendt versjon)}

%Algebra 
\newcommand{\algintro}{Introduksjon}
\newcommand{\pot}{Potensar}
\newcommand{\irrasj}{Irrasjonale tal}

%Equations
\newcommand{\ligintro}{Introduksjon}
\newcommand{\liglos}{Løysing ved dei fire rekneartane}
\newcommand{\ligloso}{Løysingsmetodane oppsummert}

%Functions
\newcommand{\fintro}{Introduksjon}
\newcommand{\lingraf}{Lineære funksjonar og grafar}

%Geometry 2
\newcommand{\geoform}{Formlar for areal og omkrins}
\newcommand{\kongogsim}{Kongruente og formlike trekantar}
\newcommand{\geofork}{Forklaringar}

% Names of rules
\newcommand{\gangdestihundre}{Å gange desimaltall med 10, 100 osv.}
\newcommand{\delmedtihundre}{Deling med 10, 100, 1\,000 osv.}
\newcommand{\adkom}{Addisjon er kommutativ}
\newcommand{\gangkom}{Multiplikasjon er kommutativ}
\newcommand{\brdef}{Brøk som omskriving av delestykke}
\newcommand{\brtbr}{Brøk gonga med brøk}
\newcommand{\delmbr}{Brøk delt på brøk}
\newcommand{\gangpar}{Gonging med parentes (distributiv lov)}
\newcommand{\gangparsam}{Parantesar gonga saman}
\newcommand{\gangmnegto}{Gonging med negative tal I}
\newcommand{\gangmnegtre}{Gonging med negative tal II}
\newcommand{\konsttre}{Konstruksjon av trekantar}
\newcommand{\kongtre}{Kongruente trekantar}
\newcommand{\topv}{Toppvinklar}
\newcommand{\trisum}{Summen av vinklane i ein trekant}
\newcommand{\firsum}{Summen av vinklane i ein firkant}
\newcommand{\potgang}{Gonging med potensar}
\newcommand{\potdivpot}{Divisjon med potensar}
\newcommand{\potanull}{Spesialtilfellet \boldmath $a^0$}
\newcommand{\potneg}{Potens med negativ eksponent}
\newcommand{\potbr}{Brøk som grunntal}
\newcommand{\faktgr}{Faktorar som grunntal}
\newcommand{\potsomgrunn}{Potens som grunntal}
\newcommand{\arsirk}{Arealet til ein sirkel}
\newcommand{\artrap}{Arealet til eit trapes}
\newcommand{\arpar}{Arealet til eit parallellogram}
\newcommand{\pyt}{Pytagoras' setning}
\newcommand{\volforml}{Volumet til tredimensjonale former}
\newcommand{\volkule}{Volumet til ei kule}
\newcommand{\forform}{Forhold i formlike trekantar}
\newcommand{\vilkform}{Vilkår i formlike trekantar}
\newcommand{\omkrsirk}{Omkrinsen til ein sirkel (og $ \bm \pi $)}
\newcommand{\artri}{Arealet til ein trekant}
\newcommand{\arrekt}{Arealet til eit rektangel}
\newcommand{\liknflyt}{Flytting av ledd over likskapsteiknet}
\newcommand{\funklin}{Lineære funksjonar}

\begin{document}
\section{\algintro}
\textit{Algebra}\index{algebra} er kort og godt matematikk der bokstavar representerer tal. Dette gjer at vi lettare kan jobbe med \textsl{generelle} tilfelle. For eksempel er $ {3\cdot 2=2\cdot3} $ og $ 6\cdot7=7\cdot6 $, men desse er berre to av dei uendeleg mange eksempla på at multiplikasjon er kommutativ! Ei av hensiktene med algebra er at vi ønsker å gi \textsl{eitt} eksempel som forklarer \textsl{alle} tilfelle, og sidan sifra våre (0-9) er uløseleg knytta til bestemde tal, bruker vi bokstavar for å nå dette målet. \vsk

Verdien til tala som er representert ved bokstavar vil ofte variere ut ifrå ein samanheng, og da kallar vi desse bokstavtala for \textit{variablar}\index{variabel}. Viss bokstavtala derimot har ein bestemd verdi, kallar vi dei for \textit{konstantar}\index{konstant}.

\vsk

I \textsl{Del I} av boka har vi sett på rekning med konkrete tal, likevel er dei fleste reglane vi har utleda \textsl{generelle}; dei gjeld for alle tal. På side \pageref{regstart}\,-\,\pageref{regslutt} har vi gjengitt mange av desse reglane på ei meir generell form. Ein fin introduksjon til algebra er å samanlikne reglane du finn her med slik du finn dei\footnote{Reglane sine nummer i \textsl{Del I} står i parentes.} i \textsl{Del I}. \vsk

\regv
\label{regstart}
\reg[\adkom\;(\ref{adkom}) \label{adkomalg}]{\vs
\[ a+ b =b+a \]
}
\eks{ \vsb
\[ 7+ 5=5+7 \]
} \vsk \vsk

\reg[\gangkom\;(\ref{gangkom})]{\vs
	\[ a\cdot b =b\cdot a \]
}
\eks[1]{ \vsb
	\[ 9\cdot 8=8\cdot9 \]
}
\eks[2]{ \vsb
\[  8\cdot a= a\cdot 8  \]
}
\newpage
\info{Gonging med bokstavuttrykk}{Når ein gongar saman bokstavar, er det vanleg å utelate gongeteiknet. Og om ein gongar saman ein bokstav og eit konkret tal, skriv ein det konkrete talet først. Dette betyr for eksempel at
	\[ a\cdot b= ab \]
	og at
	\[ a\cdot 8 =8a \]
I tillegg skriv vi også
\[ 1\cdot a=a \]
Det er også vanleg å utelate gongeteikn der parentesuttrykk er ein faktor:\[ 
3\cdot(a+b)=3(a+b) \]
}
\vsk 

\reg[\brdef\;(\ref{brdef})]{
\[ a:b=\frac{a}{b} \]
}
\eks[]{ \vs
\[a:2= \frac{a}{2} \]
}
 \vsk 

\reg[\brtbr\; (\ref{brtbr})]{
\[ \frac{a}{b}\cdot\frac{c}{d}=\frac{a c}{b d} \]
}
\eks[1]{ \vs
\algv{
\frac{2}{11}\cdot \frac{13}{21}&=\frac{2\cdot 13}{11\cdot21} =\frac{26}{231}
}
}
\eks[2]{ \vs
	\[ \frac{3}{b}\cdot \frac{a}{7}=\frac{3 a}{7b} \]
}
\newpage
\reg[\brdelmbr\;(\ref{delmbr})]{ 
\[ \frac{a}{b}:\frac{c}{d}=\frac{a}{b}\cdot \frac{d}{c} \]
}
\eks[1]{ \vs
\[ \frac{1}{2}:\frac{5}{7}=\frac{1}{2}\cdot \frac{7}{5} \]
}
\eks[2]{ \vsb \vs
\alg{
\frac{a}{13}:\frac{b}{3}&=\frac{a}{13}\cdot \frac{3}{b} \br
&=\frac{3a}{13b}
}
} \vsk \vsk

\reg[\gangpar\;(\ref{gangpar}) \label{gangpara}]{ \vs
\[ (a+b)c = a c + b c \]
} 
\eks[1]{ \vs
\[ (2+a)b =2b+ab \]
}
\eks[2]{\vs
\[ a(5b-3)=5ab-3a \]
} 
\vsk \vsk
\reg[\gangmnegto\;(\ref{gangmnegto})]{
\[ a\cdot(-b)=-(a\cdot b) \]
}
\eks[1]{ \vsb \vs
\alg{
3\cdot(-4)&=-(3\cdot 4) \\
&= 	-12
}
}
\eks[2]{ \vsb
	\algv{
	(-a)\cdot7&=-(a\cdot 7)\\
	&=-7a 
}
} \vsk \vsk

\reg[\gangmnegtre\;(\ref{gangmnegtre}) \label{gangmnegtrea}]{
\[ (-a)\cdot(-b)=a\cdot b \]
}
\eks[1]{ \vs \vs
\alg{
(-2)\cdot(-8)&=2\cdot 8 \\
&= 	16
}	
}
\eks[2]{ \vs 
\[ (-a)\cdot(-15)=15a \]
}
\label{regslutt}
\vsk \vsk

\spr{
	Viss vi i eit uttrykk har éin variabel isolert på den eine sida av likskapsteiknet, og konstantar og variablar på den andre sida, seier vi at den isolerte variabelen er \outl{uttrykt ved} dei andre tala. For eksempel, om vi har uttrykket $ a=2b-4 $, seir vi at ''$ a $ er uttrykt ved $ b $''. Har vi uttrykket $ q = 9y-x $, seier vi at ''$ q $ er uttrykt ved $ x $ og $ y $''.
}

\newpage

\info{Utvidingar av reglane}{
Noko av styrken til algebra er at vi kan lage oss kompakte reglar som det er lett å utvide også til andre tilfelle. La oss som eit eksempel finne eit anna uttrykk for
	\[ (a+b+c)d \]
\rref{gangpara} fortel oss ikkje direkte korleis vi kan rekne mellom parentesuttrykket og $ d $, men det er ingenting som hindrar oss i å omdøpe $ a+b $ til $ k $:
\[ a+b=k \]
Da er
\[ (a+b+c)d=(k+c)d \]
Av \rref{gangpara} har vi no at
\[ (k+c)d = kd+cd \]
Om vi sett inn att uttrykket for $ k $, får vi 
\[ kd+cd=(a+b)d+cd \]
Ved å utnytte \rref{gangpara} enda ein gong kan vi skrive
\[ (a+b)d+cd=ad+bc+cd \]
Altså er
\[ (a+b+c)d=ad+bc+cd \]
\it Obs! Dette eksempelet er \textsl{ikkje} meint for å vise korleis ein skal gå fram når ein har uttrykk som ikkje direkte er omfatta av \textsl{Regel} \ref{adkomalg}\,-\,\ref{gangmnegtrea}, men for å vise kvifor det alltid er nok å skrive reglar med færrast moglege ledd, faktorar og liknande. Oftast vil ein bruke utvidingar av reglane utan eingong å tenke over det, og i alle fall langt ifrå så pertentleg som det vi gjorde her.
}


\end{document}


