\documentclass[english,hidelinks,pdftex, 11 pt, class=report,crop=false]{standalone}
\usepackage[T1]{fontenc}
\usepackage[utf8]{luainputenc}
\usepackage{lmodern} % load a font with all the characters
\usepackage{geometry}
\geometry{verbose,a4paper, inner=0cm, outer=0 cm, bmargin=2cm, tmargin=1cm}
%\textwidth=12cm
\setlength{\parindent}{0bp}
\usepackage{import}
\usepackage[subpreambles=false]{standalone}
\usepackage{amsmath}
\usepackage{amssymb}
\usepackage{esint}
\usepackage{babel}
\usepackage{tabu}
\usepackage[dvipsnames, table]{xcolor}
\usepackage{cancel}
\makeatother
\makeatletter
\usepackage{datetime2}
\usepackage{titlesec}
\usepackage[many]{tcolorbox}

% Eheter
\newcommand{\enh}[1]{\,\textrm{#1}}
%referances
\newcommand{\net}[2]{{\color{blue}\href{#1}{#2}}}

%Spaces
\newcommand{\vsk}{\\[12pt]}
\newcommand{\vs}{\vspace{-12pt}}

% Tabell for opplegg

\newcommand{\ovlist}[1]{
\vspace{-16pt}
\begin{itemize}
	#1
\end{itemize}
}

% Chapters and sections
\titleformat{\section}[block]{\bfseries}{\hspace{3cm}\thesection}{5pt}{}
\titleformat{\subsection}[block]{\bfseries}{\hspace{3cm}\thesection}{5pt}{}
\newcommand{\sectionbreak}{\clearpage} % New page on each section
 

\newlength{\mywidth}
\setlength{\mywidth}{14cm}

\newcommand{\cont}[1]{
\begin{tcolorbox}[center, boxrule=0.0 mm, width=\mywidth,arc=0mm,enhanced jigsaw,,colback=white,breakable]
#1	
\end{tcolorbox}
}

\newcommand{\info}[5]{
\begin{tcolorbox}[center, boxrule=0.1 mm, width=\mywidth,arc=0mm,enhanced jigsaw,breakable,colback=yellow!5]	
	
	\footnotesize
	\textbf{Øvingsområde}\\[5pt] #1 
	
	\textbf{Utstyr}\\ #2  \\
	
	\begin{tabular}{@{} p{4cm} p{4cm} l} 
		\textbf{Tid} & \textbf{Elevinndeling} & \textbf{Læringsarena} \\
		#3  & #4 & #5
	\end{tabular} 
\end{tcolorbox}	
}

\newcommand{\gjen}[1]{\begin{tcolorbox}[center,boxrule=0.1 mm, width=\mywidth,arc=0mm,colback=blue!3] {\large \textbf{Gjennomføring} \vspace{5 pt}}\newline #1  \end{tcolorbox}\vspace{-5pt}}
\newcommand{\eks}[1]{\begin{tcolorbox}[center,boxrule=0.1 mm, width=\mywidth,arc=0mm,colback=green!3] {\large \textbf{Eksempel} \vspace{5 pt}}\newline #1  \end{tcolorbox}\vspace{-5pt}}

\newcounter{opl}
%\numberwithin{opl}{article}


\newcommand{\opl}[1]{
\newpage
{\refstepcounter{opl} %\phantomsection 
\large \textbf{\theopl \;#1} \vsk}
}

% Headlines
\newcommand{\fork}{\textbf{Forkunnskapar}\\}
\newcommand{\forb}{\textbf{Forberedelsar}\\}
\newcommand{\opgvr}{\textbf{Oppgaver}}



%colors
\newcommand{\colr}[1]{{\color{red} #1}}
\newcommand{\colb}[1]{{\color{blue} #1}}
\newcommand{\colo}[1]{{\color{orange} #1}}
\newcommand{\colc}[1]{{\color{cyan} #1}}
\definecolor{projectgreen}{cmyk}{100,0,100,0}
\newcommand{\colg}[1]{{\color{projectgreen} #1}}

% Lister med bokstavar
\usepackage[inline]{enumitem}
% Opg
\newcommand{\abc}[1]{
	\begin{enumerate}[label=\alph*),leftmargin=18pt]
		#1
	\end{enumerate}
}

\usepackage[]{hyperref}

% note
\newcommand{\note}{Note}
\newcommand{\notesm}[1]{{\footnotesize \textsl{\note:} #1}}
\newcommand{\selos}{See the solutions manual.}

\newcommand{\texandasy}{The text is written in \LaTeX\ and the figures are made with the aid of Asymptote.}

\newcommand{\rknut}{Calculate.}
\newcommand\sv{\vsk \textbf{Answer} \vspace{4 pt}\\}
\newcommand{\ekstitle}{Example }
\newcommand{\sprtitle}{The language box}
\newcommand{\expl}{explanation}

% answers
\newcommand{\mulansw}{\notesm{Multiple possible answers.}}	
\newcommand{\faskap}{Chapter}

% exercises
\newcommand{\opgt}{\newpage \phantomsection \addcontentsline{toc}{section}{Exercises} \section*{Exercises for Chapter \thechapter}\vs \setcounter{section}{1}}

% references
\newcommand{\reftab}[1]{\hrs{#1}{Table}}
\newcommand{\rref}[1]{\hrs{#1}{Rule}}
\newcommand{\dref}[1]{\hrs{#1}{Definition}}
\newcommand{\refkap}[1]{\hrs{#1}{Chapter}}
\newcommand{\refsec}[1]{\hrs{#1}{Section}}
\newcommand{\refdsec}[1]{\hrs{#1}{Subsection}}
\newcommand{\refved}[1]{\hrs{#1}{Appendix}}
\newcommand{\eksref}[1]{\textsl{#1}}
\newcommand\fref[2][]{\hyperref[#2]{\textsl{Figure \ref*{#2}#1}}}
\newcommand{\refop}[1]{{\color{blue}Exercise \ref{#1}}}
\newcommand{\refops}[1]{{\color{blue}Exercise \ref{#1}}}

%%% SECTION HEADLINES %%%

% Our numbers
\newcommand{\likteikn}{The equal sign}
\newcommand{\talsifverd}{Numbers, digits and values}
\newcommand{\koordsys}{Coordinate systems}

% Calculations
\newcommand{\adi}{Addition}
\newcommand{\sub}{Subtraction}
\newcommand{\gong}{Multiplication}
\newcommand{\del}{Division}

%Factorization and order of operations
\newcommand{\fak}{Factorization}
\newcommand{\rrek}{Order of operations}

%Fractions
\newcommand{\brgrpr}{Introduction}
\newcommand{\brvu}{Values, expanding and simplifying}
\newcommand{\bradsub}{Addition and subtraction}
\newcommand{\brgngheil}{Fractions multiplied by integers}
\newcommand{\brdelheil}{Fractions divided by integers}
\newcommand{\brgngbr}{Fractions multiplied by fractions}
\newcommand{\brkans}{Cancelation of fractions}
\newcommand{\brdelmbr}{Division by fractions}
\newcommand{\Rasjtal}{Rational numbers}

%Negative numbers
\newcommand{\negintro}{Introduction}
\newcommand{\negrekn}{The elementary operations}
\newcommand{\negmeng}{Negative numbers as amounts}

%Calculation methods
\newcommand{\delmedtihundre}{Deling med 10, 100, 1\,000 osv.}

% Geometry 1
\newcommand{\omgr}{Terms}
\newcommand{\eignsk}{Attributes of triangles and quadrilaterals}
\newcommand{\omkr}{Perimeter}
\newcommand{\area}{Area}

%Algebra 
\newcommand{\algintro}{Introduction}
\newcommand{\pot}{Powers}
\newcommand{\irrasj}{Irrational numbers}

%Equations
\newcommand{\ligintro}{Introduction}
\newcommand{\liglos}{Solving with the elementary operations}
\newcommand{\ligloso}{Solving with elementary operations summarized}

%Functions
\newcommand{\fintro}{Introduction}
\newcommand{\lingraf}{Linear functions and graphs}

%Geometry 2
\newcommand{\geoform}{Formulas of area and perimeter}
\newcommand{\kongogsim}{Congruent and similar triangles}
\newcommand{\geofork}{Explanations}

% Names of rules
\newcommand{\adkom}{Addition is commutative}
\newcommand{\gangkom}{Multiplication is commutative}
\newcommand{\brdef}{Fractions as rewriting of division}
\newcommand{\brtbr}{Fractions multiplied by fractions}
\newcommand{\delmbr}{Fractions divided by fractions}
\newcommand{\gangpar}{Distributive law}
\newcommand{\gangparsam}{Paranthesis multiplied together}
\newcommand{\gangmnegto}{Multiplication by negative numbers I}
\newcommand{\gangmnegtre}{Multiplication by negative numbers II}
\newcommand{\konsttre}{Unique construction of triangles}
\newcommand{\kongtre}{Congruent triangles}
\newcommand{\topv}{Vertical angles}
\newcommand{\trisum}{The sum of angles in a triangle}
\newcommand{\firsum}{The sum of angles in a quadrilateral}
\newcommand{\potgang}{Multiplication by powers}
\newcommand{\potdivpot}{Division by powers}
\newcommand{\potanull}{The special case of \boldmath $a^0$}
\newcommand{\potneg}{Powers with negative exponents}
\newcommand{\potbr}{Fractions as base}
\newcommand{\faktgr}{Factors as base}
\newcommand{\potsomgrunn}{Powers as base}
\newcommand{\arsirk}{The area of a circle}
\newcommand{\artrap}{The area of a trapezoid}
\newcommand{\arpar}{The area of a parallelogram}
\newcommand{\pyt}{Pythagoras's theorem}
\newcommand{\forform}{Ratios in similar triangles}
\newcommand{\vilkform}{Terms of similar triangles}
\newcommand{\omkrsirk}{The perimeter of a circle (and the value of $ \bm \pi $)}
\newcommand{\artri}{The area of a triangle}
\newcommand{\arrekt}{The area of a rectangle}
\newcommand{\liknflyt}{Moving terms across the equal sign}
\newcommand{\funklin}{Linear functions}



\begin{document}
\opgt
\op{opgbrverdidelelig}
Find the value of the fraction.\os
\abch{
\item $ \dfrac{18}{3} $	
\item $ \dfrac{20}{4} $
\item $ \dfrac{10}{5} $
\item $ \dfrac{42}{6} $
\item $ \dfrac{63}{7} $
\item $ \dfrac{32}{8} $	
}

\op{opgbrverdiikkedelelig}
Find the value of the fraction. Use a calculator if necessary.\os
\abch{
\item $ \dfrac{1}{2} $ 	
\item $ \dfrac{1}{4} $ 	
\item $ \dfrac{1}{5} $
\item $ \dfrac{3}{4} $
\item $ \dfrac{2}{5} $
\item $ \dfrac{3}{5} $
\item $ \dfrac{4}{5} $
} \vsk

\abchs{6}{
\item $ \dfrac{3}{2} $
\ \item $ \dfrac{1}{3} $
\item $ \dfrac{5}{2} $
\item $ \dfrac{5}{6} $
\item $ \dfrac{7}{5} $
\item $ \dfrac{11}{4} $
\item $ \dfrac{7}{10} $
}

\op{opgbrtallin}
Write the fraction positioned at the red mark.
\abc{
	\item \includegraphics{\figp{bropg1}}
	\item \includegraphics{\figp{bropg2}}
	\item \includegraphics{\figp{bropg3}}
}

\op{opgbrtallin2}
Write the fraction positioned at the red mark.
\abc{
	\item \includegraphics{\figp{bropg4}}
	\item \includegraphics{\figp{bropg5}}
	\item \includegraphics{\figp{bropg6}}
}
\nes
\newpage
\op{opgbrutvid}
\opgeks[]{ \vs
\[ \frac{9}{8}\text{ expanded by \colb{3}}=\frac{9\cdot\colb{3}}{8\cdot\colb{3}}=\frac{27}{24} \] 
} \vsk

Expand\os
\abch{
\item $ \dfrac{10}{3} $ by 2.	
\item $ \dfrac{3}{4} $ by 3.
\item $ \dfrac{3}{7} $ by 4.
} \vsk

\abchs{3}{
\item $ \dfrac{9}{8} $ by 5.
\ \ \item $ \dfrac{9}{5} $ by 6.
\item $ \dfrac{11}{4} $ by 7.
}

\op{opgbrutvidtil}
Expand
\abc{
	\item $ \frac{7}{3} $ to a fraction with denominator 15.
	\item $ \frac{3}{4} $ to a fraction with denominator 32.
	\item $ \frac{10}{9} $ to a fraction with denominator 63.
}

\op{opgbrfork}
\opgeks[]{ \vs
	\[ \frac{10}{8}\text{ simplified by \colb{2}}=\frac{10:\colb{2}}{8:\colb{2}}=\frac{5}{4} \] 
} \vsk

Simplify\os
\abch{
	\item $ \dfrac{14}{26} $ by 2.	
	\item $ \dfrac{15}{12} $ by 3.
	\item $ \dfrac{20}{16} $ by 4.
} \vsk

\abchs{3}{
	\item $ \dfrac{35}{50} $ by 5.
	\item $ \dfrac{54}{18} $ by 6.
	\item $ \dfrac{49}{63} $ by 7.
}

\op{opgbrforktil}
Simplify
\abc{
	\item $ \frac{27}{12} $ to a fraction with denominator 4.
	\item $ \frac{36}{20} $ to a fraction with denominator 5.
	\item $ \frac{18}{63} $ to a fraction with denominator 7.
}
\nes
\newpage

\op{opgbrad}
\rknut \os
\abch{
\item $ \displaystyle \frac{4}{3}+\frac{6}{3}$
\item $ \displaystyle \frac{5}{4}+\frac{9}{4}$
\item $ \displaystyle \frac{1}{6}+\frac{10}{6}$
\item $ \displaystyle \frac{8}{7}+\frac{2}{7}$
\item $ \displaystyle \frac{1}{2}+\frac{1}{2}$
}

\op{opgbrad2}
\rknut \os
\abch{
	\item $ \displaystyle \frac{10}{3}+\frac{4}{3}+\frac{8}{3}$
	\item $ \displaystyle \frac{4}{5}+\frac{3}{5}+\frac{1}{5}$
	\item $ \displaystyle \frac{11}{7}+\frac{2}{7}+\frac{4}{7}$		
}

\op{opgbrsub}
\rknut \os
\abch{
	\item $ \displaystyle \frac{5}{3}-\frac{4}{3}$
	\item $ \displaystyle \frac{9}{4}-\frac{7}{4}$
	\item $ \displaystyle \frac{11}{6}-\frac{1}{6}$
	\item $ \displaystyle \frac{11}{7}-\frac{4}{7}$
	\item $ \displaystyle \frac{1}{2}-\frac{1}{2}$
}

\op{opgbradandsub}
\rknut \os
\abch{
	\item $ \displaystyle \frac{4}{5}+\frac{3}{5}-\frac{1}{5}$
	\item $ \displaystyle \frac{11}{7}-\frac{2}{7}-\frac{4}{7}$
	\item $ \displaystyle \frac{10}{3}-\frac{4}{3}+\frac{8}{3}$		
}

\op{opgbrad3}
\rknut \os
\abch{
\item $\displaystyle \frac{2}{5}+\frac{3}{6} $
\item $\displaystyle \frac{5}{7}+\frac{4}{9} $
\item $\displaystyle \frac{10}{3}+\frac{7}{8} $
\item $\displaystyle \frac{7}{5}+\frac{9}{4} $
\item $\displaystyle \frac{1}{3}+\frac{1}{2} $
}

\op{opgbrsub2}
\rknut \os
\abch{
	\item $\displaystyle \frac{2}{5}-\frac{3}{10} $
	\item $\displaystyle \frac{5}{4}-\frac{4}{9} $
	\item $\displaystyle \frac{10}{9}-\frac{1}{8} $
	\item $\displaystyle \frac{4}{5}-\frac{1}{4} $
	\item $\displaystyle \frac{5}{2}-\frac{5}{3} $
}

\op{opgbradandsubmix}
\rknut \os
\abch{
\item $\displaystyle \frac{2}{3}+\frac{1}{2}-\frac{3}{4} $	
\item $\displaystyle \frac{10}{2}-\frac{1}{6}+\frac{2}{5} $
\item $\displaystyle \frac{9}{2}-\frac{2}{7}-\frac{1}{8} $
}
\nes
\newpage

\op{opgbrgongheil}
\rknut \os
\abch{
	\item $ \displaystyle \frac{4}{3}\cdot5 $
	\item $ \displaystyle \frac{5}{7}\cdot8 $
	\item $ \displaystyle \frac{9}{10}\cdot6 $
	\item $ \displaystyle \frac{8}{7}\cdot10 $
	\item $ \displaystyle \frac{3}{2}\cdot7 $
} \vsk

\abchs{6}{
	\item $ \displaystyle 7\cdot\frac{4}{3} $
	\ \item $ \displaystyle 5\cdot\frac{7}{3} $
	\item $ \displaystyle 3\cdot\frac{10}{7} $
	\item $ \displaystyle 1\cdot\frac{5}{11} $
	\item $ \displaystyle 8\cdot\frac{9}{17} $
}


\nes
\op{opgbrdelheil}
\rknut\os
\abch{
	\item $ \displaystyle \frac{4}{3}:5 $
	\ \ \item $ \displaystyle \frac{5}{7}:8 $
\ 	\item $ \displaystyle \frac{9}{10}:6 $
	\item $ \displaystyle \frac{8}{7}:10 $
	\item $ \displaystyle \frac{3}{2}:7 $
} \\[8pt]

\abchs{6}{
	\item $ \displaystyle \frac{9}{10}:11 $
	\item $ \displaystyle \frac{1}{5}:12 $
	\item $ \displaystyle \frac{9}{10}:29 $
	\item $ \displaystyle \frac{8}{9}:51 $
	\item $ \displaystyle \frac{3}{2}:79 $
}

\nes
\op{opgbrgongbr}
\rknut \os
\abch{
	\item $\displaystyle \frac{4}{3}\cdot\frac{5}{9} $
	\item $\displaystyle \frac{7}{8}\cdot\frac{1}{4} $
	\item $\displaystyle \frac{2}{7}\cdot\frac{9}{3} $
	\item $\displaystyle \frac{10}{3}\cdot\frac{6}{5} $
	\item $\displaystyle \frac{3}{2}\cdot\frac{7}{5} $
} \vsk

\abchs{6}{
	\item $ \dfrac{2}{3}\cdot\dfrac{5}{7} $
	\item $ \dfrac{8}{9}\cdot\dfrac{2}{3} $
	\item $ \dfrac{10}{3}\cdot\dfrac{8}{3} $
	\item $ \dfrac{4}{5}\cdot\dfrac{9}{7} $
	\item $ \dfrac{7}{2}\cdot\dfrac{5}{6} $
}

\nes
\op{opgbrgongbr2}
\rknut \os
\abch{
\item $\displaystyle \frac{3}{10}\cdot\frac{5}{4} $
\item $\displaystyle \frac{17}{8}\cdot\frac{9}{4} $
\item $\displaystyle \frac{23}{8}\cdot\frac{2}{4} $
\item $\displaystyle \frac{7}{81}\cdot\frac{3}{8} $
\item $\displaystyle \frac{7}{8}\cdot\frac{29}{41} $
}

\nes

\op{kansfakt}
Cancel as many factors as possible.\os 
\abch{
	\item $ \dfrac{3\cdot11\cdot8}{4\cdot8\cdot3}$
	\item $ \dfrac{5\cdot12\cdot7\cdot2}{2\cdot8\cdot12}$
	\item $ \dfrac{6\cdot10}{6\cdot9\cdot10}$
	\item $ \dfrac{7\cdot4\cdot3}{7\cdot3}$
}
\newpage
\op{opgbrforkortmfaktor}
Simplify the fraction as much as possible.\os
\abch{
	\item $ \dfrac{28}{16} $
	\item $ \dfrac{18}{42} $
	\item $ \dfrac{24}{36} $
	\item $ \dfrac{56}{49} $
	\item $ \dfrac{25}{50} $
	\item $ \dfrac{21}{14} $
}

\op{opgbrgongbrfakt}
\opgeks[]{ \vs
\[ \frac{3}{\colb{4}}\cdot 20=\frac{3}{\cancel{\colb{4}}}\cdot \cancel{\colb{4}}\cdot5=3\cdot 5=15 \]
} \regv

Exploit that the numerator is a factor in the other factor, and calculate.\os
\abch{
\item $ \dfrac{7}{3}\cdot 21 $
\item $ \dfrac{9}{5}\cdot 30 $
\item $ \dfrac{10}{7}\cdot 49 $
\item $ \dfrac{8}{9}\cdot 18 $
\item $ \dfrac{5}{4}\cdot 24 $
} \os
\abchs{6}{
\item $ 8\cdot\dfrac{3}{2} $
\ \ \item $ 35\cdot\dfrac{5}{7} $
\item $ 63\cdot\dfrac{2}{9} $
\ \item $ 48\cdot\dfrac{1}{6} $
\item $ 27\cdot\dfrac{7}{3} $
}
\nes

\op{opgheildelbr}
\rknut \os
\abch{
	\item $ 4:\dfrac{9}{8} $
	\item $ 7:\dfrac{3}{5} $
	\item $ 10:\dfrac{7}{3} $
	\item $ 5:\dfrac{4}{5} $
	\item $ 2:\dfrac{5}{11} $
}

\op{opgheildelbrfork}
Calculate, and simplify the fraction as much as possible. \os
\abch{
	\item $ 4:\dfrac{8}{9} $
	\item $ 7:\dfrac{21}{5} $
	\item $ 10:\dfrac{5}{3} $
	\item $ 5:\dfrac{5}{4} $
	\item $ 2:\dfrac{8}{11} $
}

\op{opgbrdelbr}
\rknut \os
\abch{
	\item $ \dfrac{2}{3}:\dfrac{5}{7} $
	\item $ \dfrac{8}{9}:\dfrac{5}{3} $
	\item $ \dfrac{10}{3}:\dfrac{7}{3} $
	\item $ \dfrac{1}{5}:\dfrac{4}{7} $
	\item $ \dfrac{6}{5}:\dfrac{3}{11} $
}
\newpage
\op{opgbrdelbr2}
\opgeks{ \vs
\[ \colb{\frac{3}{4}}: \colg{\frac{15}{8}}=\colb{\frac{\cancel{3}}{\cancel{2}\cdot\cancel{2}}}\cdot \colg{\frac{\cancel{2}\cdot\cancel{2}\cdot2}{\cancel{3}\cdot5}}=\frac{2}{5} \]
\mers{Here, we have chosen to prime factorize the numbers, however, this is not necessary if you detect the common factors of the numerators and the denominators.}
} \regv

Exploit that the numerators and denominators have common factors, and calculate. \os
\abch{
\item $ \displaystyle \frac{7}{9}:\frac{21}{12} $
\item $ \displaystyle \frac{35}{24}:\frac{7}{18} $
\item $ \displaystyle \frac{84}{55}:\frac{42}{77} $
}

\newpage

\grubop{opgfracval} 
Apply \rref{brdeenk} and \rref{brtbr} to insert the missing integers in place of ''\_''.
\abc{
\item Multiplying by $ \frac{1}{2} $ is the same as dividing by \_\,.
\item Multiplying by $ \frac{1}{4} $ is the same as dividing by \_\,.
\item Multiplying by $ \frac{1}{5} $ is the same as dividing by \_\,.	
}
Look back at the answers of \refops{opgbrverdiikkedelelig}a)\,-\,g). Fill inn the integer missing in place of ''\_''.
\abcs{4}{
\item Multiplying by $ 0,5 $ is the same as dividing by \_\,.
\item Multiplying by $ 0,25 $ is the same as dividing by \_\,.
\item Multiplying by $ 0,2 $ is the same as dividing by \_\,.
\item Multiplying by $ 0,75 $ is the same as multiplying by \_\, and dividing by \_\,.
\item Multiplying by $ 0,4 $ is the same as multiplying by \_\, and dividing by \_\,.
\item Multiplying by $ 0,6 $ is the same as multiplying by \_\, and dividing by \_\,.
\item Multiplying by $ 0,8 $ is the same as multiplying by \_\, and dividing by \_\,.
}
\newpage
\grubop{opgfracval2}
Look up \rref{delmbr} and the answers of \refops{opgbrverdiikkedelelig}a)\,-\,g).
Fill in the missing integers in place of ''\_''.
{\renewcommand{\labelenumi}{(\alph{enumi})}
	\begin{enumerate}
		\item Dividing by $ 0,5 $ is the same as multiplying by \_\,.
		\item Dividing by $ 0,25 $ is the same as multiplying by \_\,.
		\item Dividing by $ 0,2 $ is the same as multiplying by \_\,.
		\item Dividing by $ 0,75 $ is the same as multiplying by \_\, and \\ dividing by \_\,.
		\item Dividing by $ 0,4 $ is the same as multiplying by \_\, and \\ dividing by \_\,..
		\item Dividing by $ 0,6 $ is the same as multiplying by \_\, and \\ dividing by \_\,..
		\item Dividing by $ 0,8 $ is the same as multiplying by \_\, and \\ dividing by \_\,..	
\end{enumerate} }

\grubop{opgfracprim}
Calculate.\os
\abch{
\item $ \displaystyle \frac{5}{204}+\frac{7}{198}$
\item $ \displaystyle \frac{11}{350}+\frac{17}{315} $
}
\newpage
\end{document}


