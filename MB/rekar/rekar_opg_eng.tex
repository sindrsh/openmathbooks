\documentclass[english,hidelinks,pdftex, 11 pt, class=report,crop=false]{standalone}
\usepackage[T1]{fontenc}
\usepackage[utf8]{luainputenc}
\usepackage{lmodern} % load a font with all the characters
\usepackage{geometry}
\geometry{verbose,a4paper, inner=0cm, outer=0 cm, bmargin=2cm, tmargin=1cm}
%\textwidth=12cm
\setlength{\parindent}{0bp}
\usepackage{import}
\usepackage[subpreambles=false]{standalone}
\usepackage{amsmath}
\usepackage{amssymb}
\usepackage{esint}
\usepackage{babel}
\usepackage{tabu}
\usepackage[dvipsnames, table]{xcolor}
\usepackage{cancel}
\makeatother
\makeatletter
\usepackage{datetime2}
\usepackage{titlesec}
\usepackage[many]{tcolorbox}

% Eheter
\newcommand{\enh}[1]{\,\textrm{#1}}
%referances
\newcommand{\net}[2]{{\color{blue}\href{#1}{#2}}}

%Spaces
\newcommand{\vsk}{\\[12pt]}
\newcommand{\vs}{\vspace{-12pt}}

% Tabell for opplegg

\newcommand{\ovlist}[1]{
\vspace{-16pt}
\begin{itemize}
	#1
\end{itemize}
}

% Chapters and sections
\titleformat{\section}[block]{\bfseries}{\hspace{3cm}\thesection}{5pt}{}
\titleformat{\subsection}[block]{\bfseries}{\hspace{3cm}\thesection}{5pt}{}
\newcommand{\sectionbreak}{\clearpage} % New page on each section
 

\newlength{\mywidth}
\setlength{\mywidth}{14cm}

\newcommand{\cont}[1]{
\begin{tcolorbox}[center, boxrule=0.0 mm, width=\mywidth,arc=0mm,enhanced jigsaw,,colback=white,breakable]
#1	
\end{tcolorbox}
}

\newcommand{\info}[5]{
\begin{tcolorbox}[center, boxrule=0.1 mm, width=\mywidth,arc=0mm,enhanced jigsaw,breakable,colback=yellow!5]	
	
	\footnotesize
	\textbf{Øvingsområde}\\[5pt] #1 
	
	\textbf{Utstyr}\\ #2  \\
	
	\begin{tabular}{@{} p{4cm} p{4cm} l} 
		\textbf{Tid} & \textbf{Elevinndeling} & \textbf{Læringsarena} \\
		#3  & #4 & #5
	\end{tabular} 
\end{tcolorbox}	
}

\newcommand{\gjen}[1]{\begin{tcolorbox}[center,boxrule=0.1 mm, width=\mywidth,arc=0mm,colback=blue!3] {\large \textbf{Gjennomføring} \vspace{5 pt}}\newline #1  \end{tcolorbox}\vspace{-5pt}}
\newcommand{\eks}[1]{\begin{tcolorbox}[center,boxrule=0.1 mm, width=\mywidth,arc=0mm,colback=green!3] {\large \textbf{Eksempel} \vspace{5 pt}}\newline #1  \end{tcolorbox}\vspace{-5pt}}

\newcounter{opl}
%\numberwithin{opl}{article}


\newcommand{\opl}[1]{
\newpage
{\refstepcounter{opl} %\phantomsection 
\large \textbf{\theopl \;#1} \vsk}
}

% Headlines
\newcommand{\fork}{\textbf{Forkunnskapar}\\}
\newcommand{\forb}{\textbf{Forberedelsar}\\}
\newcommand{\opgvr}{\textbf{Oppgaver}}



%colors
\newcommand{\colr}[1]{{\color{red} #1}}
\newcommand{\colb}[1]{{\color{blue} #1}}
\newcommand{\colo}[1]{{\color{orange} #1}}
\newcommand{\colc}[1]{{\color{cyan} #1}}
\definecolor{projectgreen}{cmyk}{100,0,100,0}
\newcommand{\colg}[1]{{\color{projectgreen} #1}}

% Lister med bokstavar
\usepackage[inline]{enumitem}
% Opg
\newcommand{\abc}[1]{
	\begin{enumerate}[label=\alph*),leftmargin=18pt]
		#1
	\end{enumerate}
}

\usepackage[]{hyperref}

% note
\newcommand{\note}{Note}
\newcommand{\notesm}[1]{{\footnotesize \textsl{\note:} #1}}
\newcommand{\selos}{See the solutions manual.}

\newcommand{\texandasy}{The text is written in \LaTeX\ and the figures are made with the aid of Asymptote.}

\newcommand{\rknut}{Calculate.}
\newcommand\sv{\vsk \textbf{Answer} \vspace{4 pt}\\}
\newcommand{\ekstitle}{Example }
\newcommand{\sprtitle}{The language box}
\newcommand{\expl}{explanation}

% answers
\newcommand{\mulansw}{\notesm{Multiple possible answers.}}	
\newcommand{\faskap}{Chapter}

% exercises
\newcommand{\opgt}{\newpage \phantomsection \addcontentsline{toc}{section}{Exercises} \section*{Exercises for Chapter \thechapter}\vs \setcounter{section}{1}}

% references
\newcommand{\reftab}[1]{\hrs{#1}{Table}}
\newcommand{\rref}[1]{\hrs{#1}{Rule}}
\newcommand{\dref}[1]{\hrs{#1}{Definition}}
\newcommand{\refkap}[1]{\hrs{#1}{Chapter}}
\newcommand{\refsec}[1]{\hrs{#1}{Section}}
\newcommand{\refdsec}[1]{\hrs{#1}{Subsection}}
\newcommand{\refved}[1]{\hrs{#1}{Appendix}}
\newcommand{\eksref}[1]{\textsl{#1}}
\newcommand\fref[2][]{\hyperref[#2]{\textsl{Figure \ref*{#2}#1}}}
\newcommand{\refop}[1]{{\color{blue}Exercise \ref{#1}}}
\newcommand{\refops}[1]{{\color{blue}Exercise \ref{#1}}}

%%% SECTION HEADLINES %%%

% Our numbers
\newcommand{\likteikn}{The equal sign}
\newcommand{\talsifverd}{Numbers, digits and values}
\newcommand{\koordsys}{Coordinate systems}

% Calculations
\newcommand{\adi}{Addition}
\newcommand{\sub}{Subtraction}
\newcommand{\gong}{Multiplication}
\newcommand{\del}{Division}

%Factorization and order of operations
\newcommand{\fak}{Factorization}
\newcommand{\rrek}{Order of operations}

%Fractions
\newcommand{\brgrpr}{Introduction}
\newcommand{\brvu}{Values, expanding and simplifying}
\newcommand{\bradsub}{Addition and subtraction}
\newcommand{\brgngheil}{Fractions multiplied by integers}
\newcommand{\brdelheil}{Fractions divided by integers}
\newcommand{\brgngbr}{Fractions multiplied by fractions}
\newcommand{\brkans}{Cancelation of fractions}
\newcommand{\brdelmbr}{Division by fractions}
\newcommand{\Rasjtal}{Rational numbers}

%Negative numbers
\newcommand{\negintro}{Introduction}
\newcommand{\negrekn}{The elementary operations}
\newcommand{\negmeng}{Negative numbers as amounts}

%Calculation methods
\newcommand{\delmedtihundre}{Deling med 10, 100, 1\,000 osv.}

% Geometry 1
\newcommand{\omgr}{Terms}
\newcommand{\eignsk}{Attributes of triangles and quadrilaterals}
\newcommand{\omkr}{Perimeter}
\newcommand{\area}{Area}

%Algebra 
\newcommand{\algintro}{Introduction}
\newcommand{\pot}{Powers}
\newcommand{\irrasj}{Irrational numbers}

%Equations
\newcommand{\ligintro}{Introduction}
\newcommand{\liglos}{Solving with the elementary operations}
\newcommand{\ligloso}{Solving with elementary operations summarized}

%Functions
\newcommand{\fintro}{Introduction}
\newcommand{\lingraf}{Linear functions and graphs}

%Geometry 2
\newcommand{\geoform}{Formulas of area and perimeter}
\newcommand{\kongogsim}{Congruent and similar triangles}
\newcommand{\geofork}{Explanations}

% Names of rules
\newcommand{\adkom}{Addition is commutative}
\newcommand{\gangkom}{Multiplication is commutative}
\newcommand{\brdef}{Fractions as rewriting of division}
\newcommand{\brtbr}{Fractions multiplied by fractions}
\newcommand{\delmbr}{Fractions divided by fractions}
\newcommand{\gangpar}{Distributive law}
\newcommand{\gangparsam}{Paranthesis multiplied together}
\newcommand{\gangmnegto}{Multiplication by negative numbers I}
\newcommand{\gangmnegtre}{Multiplication by negative numbers II}
\newcommand{\konsttre}{Unique construction of triangles}
\newcommand{\kongtre}{Congruent triangles}
\newcommand{\topv}{Vertical angles}
\newcommand{\trisum}{The sum of angles in a triangle}
\newcommand{\firsum}{The sum of angles in a quadrilateral}
\newcommand{\potgang}{Multiplication by powers}
\newcommand{\potdivpot}{Division by powers}
\newcommand{\potanull}{The special case of \boldmath $a^0$}
\newcommand{\potneg}{Powers with negative exponents}
\newcommand{\potbr}{Fractions as base}
\newcommand{\faktgr}{Factors as base}
\newcommand{\potsomgrunn}{Powers as base}
\newcommand{\arsirk}{The area of a circle}
\newcommand{\artrap}{The area of a trapezoid}
\newcommand{\arpar}{The area of a parallelogram}
\newcommand{\pyt}{Pythagoras's theorem}
\newcommand{\forform}{Ratios in similar triangles}
\newcommand{\vilkform}{Terms of similar triangles}
\newcommand{\omkrsirk}{The perimeter of a circle (and the value of $ \bm \pi $)}
\newcommand{\artri}{The area of a triangle}
\newcommand{\arrekt}{The area of a rectangle}
\newcommand{\liknflyt}{Moving terms across the equal sign}
\newcommand{\funklin}{Linear functions}


\begin{document}

\opgt

\op{opgsumtotall}
Write the numbers as the sum of two numbers. \os
\opgeks{
We can write 3 as $ 1+2 $
}\vspace{5pt}
\abch{
\item 4
\item 5 
\item 6
\item 7
\item 8
\item 9
}

\op{opgsumtretall}
Write the numbers as the sum of three numbers. \os
\opgeks{
	We can write 4 as $ 1+2+1 $
} \vspace{5pt}
\abch{
	\item 5 
	\item 6
	\item 7
	\item 8
	\item 9
	\item 10
}

\op{opgtiervenn}
When the sum of two numbers equals 10, these numbers are called \outl{friends of ten}. For example are 1 and 9 friends of ten because $ 1+9=10 $. \os

\abcn{
\item Find the ten-friend of \os
\abch{
	\item 2
	\item 3 
	\item 4
	\item 5
} \vsk

\item  When exercise 1) is completed, why is it ''unnecessary'' to find the ten-friends of 6, 7 og 8? 
}
\newpage
\op{opgsumparodd}
\mers{You can allow yourself answering the questions by simply checking a couple of examples. For proof, see \refops{opgaddifparodd}.}\os

Choose the correct alternative of 1), 2) and 3).
\abc{
	\item The sum of two even numbers is \vspace{-5pt}
	\abcn{
		\item an even number.\\
		\item an odd number.\\
		\item sometimes an even number and sometimes an odd number.
	}
	\item The sum of two odd numbers is \vspace{-5pt}
	\abcn{
		\item an even number.\\
		\item an odd number.\\
		\item sometimes an even number and sometimes an odd number.
	}
	\item The sum of an even number and an odd number is \vspace{-5pt}
	\abcn{
		\item an even number.\\
		\item an odd number.\\
		\item sometimes an even number and sometimes an odd number.
	}
}
\newpage
\nes
\op{opgdifto}
Write the number as the difference of two numbers.\os
\opgeks{We can write 1 as $ 8-7 $.}
\abch{
	\item 2
	\item 3
	\item 4
	\item 5
	\item 6
	\item 7
	\item 8
}

\op{opgdifparodd}
\mers{You can allow yourself answering the questions by simply checking a couple of examples. For proof, see \refops{opgaddifparodd}.}\os

Choose the correct alternative of 1), 2) and 3).
\abc{
	\item The difference of two even numbers is \vspace{-5pt}
	\abcn{
		\item an even number.\\
		\item an odd number.\\
		\item sometimes an even number and sometimes an odd number.
	}
	\item The difference of two odd numbers is \vspace{-5pt}
	\abcn{
		\item an even number.\\
		\item an odd number.\\
		\item sometimes an even number and sometimes an odd number.
	}
	\item The difference of an even number and an odd number is \vspace{-5pt}
	\abcn{
		\item an even number.\\
		\item an odd number.\\
		\item sometimes an even number and sometimes an odd number.
	}
}
\newpage
\nes
\op{opgadsomgang}
Write as a multiplication, and write the alternative sum. \os
\opgeks{$ 3+3+3+3+3=3\cdot 5=5+5+5 $}
\abc{
	\item $ 2+2+2 $
	\item $ 3+3+3+3+3+3 $
	\item $ 4+4 $ \\
	\item $ 5+5+5+5+5+5+5+5+5+5 $
	\item $ 6+6+6+6 $
	\item $ 7+7+7+7 $
}

\op{opggangrut}
Draw boxes to find the answer to the multiplications.
\opgeks{\vsb
	\begin{center}
		\includegraphics{\figp{opggangruteks1}}\hspace{3cm}
		\includegraphics{\figp{opggangruteks2}}
	\end{center}
} \vspace{5pt}

\abch{
	\item $ 4\cdot 5 $ 
	\item $ 8\cdot 3 $
	\item $ 2\cdot 9 $
	\item $ 5\cdot 6 $
	\item $ 7\cdot 8$
}

\op{opggangparod} \vs
\abc{
	\item Will an integer multiplied by 2 always result in an even number or an odd number?
	\item Will an even number multiplied by 5 always result in an even number or an odd number? What digit will always be positioned at the one-place?
	\item Will an odd number multiplied by 5 always result in an even number or an odd number? What digit will always be positioned at the one-place?
}


\end{document}