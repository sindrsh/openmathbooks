\documentclass[english,hidelinks,pdftex, 11 pt, class=report,crop=false]{standalone}
\usepackage[T1]{fontenc}
\usepackage[utf8]{luainputenc}
\usepackage{lmodern} % load a font with all the characters
\usepackage{geometry}
\geometry{verbose,paperwidth=16.1 cm, paperheight=24 cm, inner=2.3cm, outer=1.8 cm, bmargin=2cm, tmargin=1.8cm}
\setlength{\parindent}{0bp}
\usepackage{import}
\usepackage[subpreambles=false]{standalone}
\usepackage{amsmath}
\usepackage{amssymb}
\usepackage{esint}
\usepackage{babel}
\usepackage{tabu}
\makeatother
\makeatletter

\usepackage{titlesec}
\usepackage{ragged2e}
\RaggedRight
\raggedbottom
\frenchspacing

% Norwegian names of figures, chapters, parts and content
\addto\captionsenglish{\renewcommand{\figurename}{Figure}}
\makeatletter
\addto\captionsenglish{\renewcommand{\chaptername}{Chapter}}
%\addto\captionsenglish{\renewcommand{\partname}{Part}}

%\addto\captionsenglish{\renewcommand{\contentsname}{Content}}

\usepackage{graphicx}
\usepackage{float}
\usepackage{subfig}
\usepackage{placeins}
\usepackage{cancel}
\usepackage{framed}
\usepackage{wrapfig}
\usepackage[subfigure]{tocloft}
\usepackage[font=footnotesize,labelfont=sl]{caption} % Figure caption
\usepackage{bm}
\usepackage[dvipsnames, table]{xcolor}
\definecolor{shadecolor}{rgb}{0.105469, 0.613281, 1}
\colorlet{shadecolor}{Emerald!15} 
\usepackage{icomma}
\makeatother
\usepackage[many]{tcolorbox}
\usepackage{multicol}
\usepackage{stackengine}

% For tabular
\usepackage{array}
\usepackage{multirow}
\usepackage{longtable} %breakable table

% Ligningsreferanser
\usepackage{mathtools}
\mathtoolsset{showonlyrefs}

% index
\usepackage{imakeidx}
\makeindex[title=Index]

%Footnote:
\usepackage[bottom, hang, flushmargin]{footmisc}
\usepackage{perpage} 
\MakePerPage{footnote}
\addtolength{\footnotesep}{2mm}
\renewcommand{\thefootnote}{\arabic{footnote}}
\renewcommand\footnoterule{\rule{\linewidth}{0.4pt}}
\renewcommand{\thempfootnote}{\arabic{mpfootnote}}

%colors
\definecolor{c1}{cmyk}{0,0.5,1,0}
\definecolor{c2}{cmyk}{1,0.25,1,0}
\definecolor{n3}{cmyk}{1,0.,1,0}
\definecolor{neg}{cmyk}{1,0.,0.,0}

% Lister med bokstavar
\usepackage{enumitem}

\newcounter{rg}
\numberwithin{rg}{chapter}
\newcommand{\reg}[2][]{\begin{tcolorbox}[boxrule=0.3 mm,arc=0mm,colback=blue!3] {\refstepcounter{rg}\phantomsection \large \textbf{\therg \;#1} \vspace{5 pt}}\newline #2  \end{tcolorbox}\vspace{-5pt}}

\newcommand\alg[1]{\begin{align} #1 \end{align}}

\newcommand\eks[2][]{\begin{tcolorbox}[boxrule=0.3 mm,arc=0mm,enhanced jigsaw,breakable,colback=green!3] {\large \textbf{Example #1} \vspace{5 pt}\\} #2 \end{tcolorbox}\vspace{-5pt} }

\newcommand{\st}[1]{\begin{tcolorbox}[boxrule=0.0 mm,arc=0mm,enhanced jigsaw,breakable,colback=yellow!12]{ #1} \end{tcolorbox}}

\newcommand{\spr}[1]{\begin{tcolorbox}[boxrule=0.3 mm,arc=0mm,enhanced jigsaw,breakable,colback=yellow!7] {\large \textbf{The language box} \vspace{5 pt}\\} #1 \end{tcolorbox}\vspace{-5pt} }

\newcommand{\sym}[1]{\colorbox{blue!15}{#1}}

\newcommand{\info}[2]{\begin{tcolorbox}[boxrule=0.3 mm,arc=0mm,enhanced jigsaw,breakable,colback=cyan!6] {\large \textbf{#1} \vspace{5 pt}\\} #2 \end{tcolorbox}\vspace{-5pt} }

\newcommand\algv[1]{\vspace{-11 pt}\begin{align*} #1 \end{align*}}

\newcommand{\regv}{\vspace{5pt}}
\newcommand{\mer}{\textsl{Note}: }
\newcommand{\merk}{Note}
\newcommand\vsk{\vspace{11pt}}
\newcommand\vs{\vspace{-11pt}}
\newcommand\vsb{\vspace{-16pt}}
\newcommand\sv{\vsk \textbf{Answer} \vspace{4 pt}\\}
\newcommand\br{\\[5 pt]}
\newcommand{\asym}[1]{../fig/#1}
\newcommand\algvv[1]{\vs\vs\begin{align*} #1 \end{align*}}
\newcommand{\y}[1]{$ {#1} $}
\newcommand{\os}{\\[5 pt]}
\newcommand{\prbxl}[2]{
\parbox[l][][l]{#1\linewidth}{#2
	}}
\newcommand{\prbxr}[2]{\parbox[r][][l]{#1\linewidth}{
		\setlength{\abovedisplayskip}{5pt}
		\setlength{\belowdisplayskip}{5pt}	
		\setlength{\abovedisplayshortskip}{0pt}
		\setlength{\belowdisplayshortskip}{0pt} 
		\begin{shaded}
			\footnotesize	#2 \end{shaded}}}

\renewcommand{\cfttoctitlefont}{\Large\bfseries}
\setlength{\cftaftertoctitleskip}{0 pt}
\setlength{\cftbeforetoctitleskip}{0 pt}

\newcommand{\bs}{\\[3pt]}
\newcommand{\vn}{\\[6pt]}
\newcommand{\fig}[1]{\begin{figure}
		\centering
		\includegraphics[]{\asym{#1}}
\end{figure}}

\newcommand{\sectionbreak}{\clearpage} % New page on each section

% Equation comments
\newcommand{\cm}[1]{\llap{\color{blue} #1}}

\newcommand\fork[2]{\begin{tcolorbox}[boxrule=0.3 mm,arc=0mm,enhanced jigsaw,breakable,colback=yellow!7] {\large \textbf{#1 (explanation)} \vspace{5 pt}\\} #2 \end{tcolorbox}\vspace{-5pt} }

% Colors
\newcommand{\colr}[1]{{\color{red} #1}}
\newcommand{\colb}[1]{{\color{blue} #1}}
\newcommand{\colo}[1]{{\color{orange} #1}}
\newcommand{\colc}[1]{{\color{cyan} #1}}
\definecolor{projectgreen}{cmyk}{100,0,100,0}
\newcommand{\colg}[1]{{\color{projectgreen} #1}}

%%% SECTION HEADLINES %%%

% Our numbers
\newcommand{\likteikn}{The equal sign}
\newcommand{\talsifverd}{Numbers, digits and values}
\newcommand{\koordsys}{Coordinate systems}

% Calculations
\newcommand{\adi}{Addition}
\newcommand{\sub}{Subtraction}
\newcommand{\gong}{Multiplication}
\newcommand{\del}{Division}

%Factorization and order of operations
\newcommand{\fak}{Factorization}
\newcommand{\rrek}{Order of operations}

%Fractions
\newcommand{\brgrpr}{Introduction}
\newcommand{\brvu}{Values, expanding and simplifying}
\newcommand{\bradsub}{Addition and subtraction}
\newcommand{\brgngheil}{Fractions multiplied by integers}
\newcommand{\brdelheil}{Fractions divided by integers}
\newcommand{\brgngbr}{Fractions multiplied by fractions}
\newcommand{\brkans}{Cancelation of fractions}
\newcommand{\brdelmbr}{Division by fractions}
\newcommand{\Rasjtal}{Rational numbers}

%Negative numbers
\newcommand{\negintro}{Introduction}
\newcommand{\negrekn}{The elementary operations}
\newcommand{\negmeng}{Negative numbers as amounts}

% Geometry 1
\newcommand{\omgr}{Terms}
\newcommand{\eignsk}{Attributes of triangles and quadrilaterals}
\newcommand{\omkr}{Perimeter}
\newcommand{\area}{Area}

%Algebra 
\newcommand{\algintro}{Introduction}
\newcommand{\pot}{Powers}
\newcommand{\irrasj}{Irrational numbers}

%Equations
\newcommand{\ligintro}{Introduction}
\newcommand{\liglos}{Solving with the elementary operations}
\newcommand{\ligloso}{Solving with elementary operations summarized}

%Functions
\newcommand{\fintro}{Introduction}
\newcommand{\lingraf}{Linear functions and graphs}

%Geometry 2
\newcommand{\geoform}{Formulas of area and perimeter}
\newcommand{\kongogsim}{Congruent and similar triangles}
\newcommand{\geofork}{Explanations}

% Names of rules
\newcommand{\adkom}{Addition is commutative}
\newcommand{\gangkom}{Multiplication is commutative}
\newcommand{\brdef}{Fractions as rewriting of division}
\newcommand{\brtbr}{Fractions multiplied by fractions}
\newcommand{\delmbr}{Fractions divided by fractions}
\newcommand{\gangpar}{Distributive law}
\newcommand{\gangparsam}{Paranthesis multiplied together}
\newcommand{\gangmnegto}{Multiplication by negative numbers I}
\newcommand{\gangmnegtre}{Multiplication by negative numbers II}
\newcommand{\konsttre}{Unique construction of triangles}
\newcommand{\kongtre}{Congruent triangles}
\newcommand{\topv}{Vertical angles}
\newcommand{\trisum}{The sum of angles in a triangle}
\newcommand{\firsum}{The sum of angles in a quadrilateral}
\newcommand{\potgang}{Multiplication by powers}
\newcommand{\potdivpot}{Division by powers}
\newcommand{\potanull}{The special case of \boldmath $a^0$}
\newcommand{\potneg}{Powers with negative exponents}
\newcommand{\potbr}{Fractions as base}
\newcommand{\faktgr}{Factors as base}
\newcommand{\potsomgrunn}{Powers as base}
\newcommand{\arsirk}{The area of a circle}
\newcommand{\artrap}{The area of a trapezoid}
\newcommand{\arpar}{The area of a parallelogram}
\newcommand{\pyt}{Pythagoras's theorem}
\newcommand{\forform}{Ratios in similar triangles}
\newcommand{\vilkform}{Terms of similar triangles}
\newcommand{\omkrsirk}{The perimeter of a circle (and the value of $ \bm \pi $)}
\newcommand{\artri}{The area of a triangle}
\newcommand{\arrekt}{The area of a rectangle}
\newcommand{\liknflyt}{Moving terms across the equal sign}
\newcommand{\funklin}{Linear functions}

%License
\newcommand{\lic}{\textit{First Principles of Math by Sindre Sogge Heggen is licensed under CC BY-NC-SA 4.0. To view a copy of this license, visit\\ 
		\net{http://creativecommons.org/licenses/by-nc-sa/4.0/}{http://creativecommons.org/licenses/by-nc-sa/4.0/}}}

%referances
\newcommand{\net}[2]{{\color{blue}\href{#1}{#2}}}
\newcommand{\hrs}[2]{\hyperref[#1]{\color{blue}\textsl{#2 \ref*{#1}}}}
\newcommand{\rref}[1]{\hrs{#1}{Rule}}
\newcommand{\refkap}[1]{\hrs{#1}{Chapter}}
\newcommand{\refsec}[1]{\hrs{#1}{Section}}

\usepackage{datetime2}
\usepackage[]{hyperref}

\begin{document}
\newpage
\section{\adi \label{Addisjon}}	
\subsection*{Addition with amounts}
When we have an amount and wish to add more, we use the symbol \sym{$ + $}. If we have 2 and want to add 3, we write
\[ 2+3=5 \]
\fig{plusm1}
The order in which we add have no impact on the results; starting off with 2 and adding 3 is the same as starting off with 3 and adding 2:
\[ 3+2=5 \]
\fig{plusm1a}
\spr{
	A calculation involving addition includes two or more \textit{terms}\index{term} and one \textit{sum}\index{sum}. In the calculation
	\[ 2+3=5 \]
	both $ 2 $ and $ 3 $ are terms while $ 5 $ is the sum.\vsk
	
	Common ways of saying $ 2+3 $ include
	\begin{itemize}
		\item ''2 plus 3''
		\item ''2 added to 3''
	 	\item ''2 and 3 added''
	\end{itemize}

} \vsk
\newpage
\reg[Addition is commutative \label{adkom}]{
The order of the terms has no impact on the sum.
}

\eks{ \vs \vsb
\alg{
2+5 &= 7 =5+2  \vn
6+3 &=9=3+6
}
} 

\subsection*{Addition on the number line: moving to the right}
On a number line, addition with positive numbers involves moving \textsl{to \\the right}:\regv
\eks[1]{
\[ 2+7=9 \]
\fig{plus2}
}
\eks[2]{
	\[ 4+11=15 \]
\fig{plus3}
}
\info{Interpretation of \sym{$\bm=$}}{
\sym{+} brings the possibility of expressing numbers in different ways, for example is $ {5=2+3} $ and $ {5=1+4} $. In this context, \sym{=} means ''has the same value as''. This is also the case regarding subtraction, multiplication and division which we'll look at in the next three sections.
}

\section{\sub\label{Subtraksjon}}
\subsection*{Subtraction with amounts}
When removing a part of an amount, we use the symbol \sym{$ - $}:
\[ 5-{\color{red} 3}=2 \]
\fig{min1c}

\spr{
	A calculation involving subtraction includes one or more \textit{terms}\index{term} and one \textit{difference}\index{difference}. In the calculation
	\[  5-3=2 \] 
	both $ 5 $ and $ 3 $ are terms while $ 2 $ is the difference. \vsk
	
	Common ways of saying $ 5-3 $ include
	\begin{itemize}
		\item ''5 minus 3'' \\
		\item ''3 subtracted from 5''
	\end{itemize}
} \vsk \vsk

\info{A new interpretation of 0}{
	As mentioned earlier in this book, 0 can be interpreted as ''nothing''. However, subtraction brings the possibility of expressing 0 by other numbers, for example $ {7-7=0} $ and $ {19-19=0} $. In many practical situations, 0 indicates some form of equilibrium, like two equal but opposite forces. 
} \vsk

\newpage
\subsection*{Subtraction on the number line: Moving to the left}
In \refsec{Addisjon}, we have seen that \sym{$ + $} (with positive numbers) involves moving  \textsl{to the right} on the number line. With \sym{$ - $} it's the opposite, we move \textsl{to the left}\footnote{In figures with number lines, the red colored arrows indicates that you shall start at the arrowhead and move to the other end.}: \regv

\eks[1]{ \vs
	\[ 6-4=2 \]
	\fig{mint}
}
\eks[2]{ \vs
	\[ 12-7=5 \]
	\fig{mint2}
}
\info{Notice}{
	At first it may seem a bit odd moving in the opposite direction of the way in which the arrows point, as in \textsl{Example 1} and \textsl{2}. However, in \refkap{Negtal} this will turn out to be useful.
}


\section{\gong \label{Gonging} }

\subsection*{Multiplication by integers: initial definition }
When adding equal numbers, we can use the multiplication symbol \sym{$ \cdot $}\; to write our calculations more compact: \regv
\eks[]{\vsb \vs
	\alg{
		4+4+4 &= 4\cdot 3 \vn
		8+8 &=8\cdot 2 \vn
		1+1+1+1+1&= 1\cdot5 
	}
} \regv
\spr{
	A calculation involving multiplication includes several \textit{factors}\index{factor} and one \textit{product}\index{product}. In the calculation
	\[ {4\cdot 3=12} \]
	both $ 4 $ and $ 3 $ are factors, while $ 12 $ is the product. \vsk
	
	Common ways of saying $ 4\cdot3 $ include
	\begin{itemize}
		\item ''4 times 3''  \\
		\item ''4 multiplied by 3''
		\item ''4 and 3 multiplied together''
	\end{itemize}
	
	A lot of texts use \sym{$ \times $} instead of \sym{$ \cdot $}. In computer programming, \sym{*} is the most common symbol for multiplication.
}
\subsection*{Multiplication involving amounts}
Let us illustrate $ 2\cdot3 $:
\fig{2t3}
Now notice the product of $ 3\cdot 2 $:
\fig{3t2}
\reg[\gangkom \label{gangkom}]{
	The order of the factors has no impact on the product.
}
\eks[]{\vsb \vs
	\alg{
		3\cdot 4 &=12= 4\cdot 3 \vn
		6\cdot 7 &=42= 7\cdot6 \vn
		8\cdot 9 &=72= 9\cdot8
	}
}

\subsection*{Multiplication on the number line}
We can also use the number line to calculate multiplications. In the case of $ 2\cdot4 $ we can think like this:
\[\text{''} 2\cdot 4 \text{ means moving 2 places to the right, 4 times.}\text{''} \]
\[ 2\cdot4=8 \]
\fig{2t4l}
We can also use the number line to prove to ourselves that multiplication is commutative:
\[\text{''} 4\cdot 2 \text{ means moving 4 places to the right, 2 times.}\text{''} \]
\[ 4\cdot2=8 \]
\fig{4t2l}


\subsection*{Final definition of multiplication by positive integers}
It may be the most intuitive to interpret ''2 times 3'' as ''3, 2 times''. Then it follows:
\[ \text{''2 times 3''}=3+3 \]
In this section we introduced $ {2\cdot3} $, that is ''2 times 3'', as $ {2+2+2} $. With this interpretation, $ {3+3} $ corresponds to $ {3\cdot2} $, but the fact that multiplication is a commutative operation (\rref{gangkom}) ensures that the one interpretation does not exclude the other; $ {2\cdot3 =2+2+2} $ and $ {2\cdot3=3+3} $ are two expressions of same value.\regv

\reg[Multiplication as repeated addition  \label{ganggjad2}]{
	Multiplication involving a positive integer can be expressed as repeated addition.
}
\eks[1]{\vsb \vs
	\alg{
		4+4+4 &= 4\cdot 3=3+3+3+3 \vn
		8+8 &=8\cdot 2=2+2+2+2+2+2+2 \vn
		1+1+1+1+1&= 1\cdot5 =5
	}
}
\info{Notice}{
	The fact that multiplication with positive integers can be expressed as repeated addition does not exclude other expressions. There's nothing wrong with writing $ {2\cdot 3=1+5} $.
}

\section{\del \label{Divisjon}}
\sym{:} is the symbol for division.
Division has three different inter-\\pretations:\regv

\reg[The three interpretations of division]{ \vs
	\begin{itemize}
		\item \textbf{Distribution of amounts} \\
		$ 12:3 = \text{''The number in each group when evenly distributing}$\\
		\hspace{1.6cm}12 into 3 groups'' 
		\item \textbf{Number of equal terms} \\
		$ 12:3=\text{''The number of 3's added to make 12''} $
		\item \textbf{The inverse operation of multiplication}\\
		$ 12:3=\text{''The number which yields 12 when multiplied by 3''} $
	\end{itemize}
} \regv

\spr{ \label{sprakdiv}
	A calculation involving division includes a \textit{dividend}\index{dividend}, a \textit{divisor}\index{divisor} and a \textit{quotient}\index{quotient}.	
	In the calculation
	\[  {12:3=4} \]
	$ 12 $ is the dividend, $ 3 $ is the divisor and $ 4 $ is the quotient.\vsk
	
	Common ways of saying $ 12:3 $ include
	\begin{itemize}
		\item ''12 divided by 3''
		\item ''12 to 3''
	\end{itemize}
	In a lot of contexts, \sym{$ / $} is used instead of \sym{$:$}, especially in computer programming. \vsk
	
	Sometimes $ {12:3} $ is called ''the \textit{ratio}\index{ratio} of 12 to 3''. \vsk
}
\newpage
\subsection*{Distribution of amounts}
The calculation $ {12:3} $ tells that we shall distribute 12 into 3 equal groups:
\fig{del1}
We observe that each group contains 4 boxes, which means that
\[ 12:3=4 \]


\subsection*{Number of equal terms}
\fig{del2}
12 equals the sum of 4 instances of 3, that is $ 12:3=4 $.


\subsection*{The inverse operation of multiplication}
We have just seen that if we divide 12 into 3 equal groups, we get 4 in each group. Hence $ 12:3=4$. The sum of these groups makes 12: 
\fig{del1b}
However, this is the same as multiplying 4 by 3, in other words:
If we know that $ {4\cdot 3=12} $, we also know that $ {12:3=4} $. As well we know that $ {12:4=3} $. 
\fig{del1a}


\eks[1]{Since $ 6\cdot 3 = 18  $, \vs
	\alg{
		18:6= 3\vn
		18:3=6
	}
}
\eks[2]{
	Since $ 5\cdot 7 = 35  $, \vs
	\alg{
		35:5= 7\vn
		35:7=5
	}
}


\end{document}