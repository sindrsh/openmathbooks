\documentclass[english,hidelinks,pdftex, 11 pt, class=report,crop=false]{standalone}
\usepackage[T1]{fontenc}
\usepackage[utf8]{luainputenc}
\usepackage{lmodern} % load a font with all the characters
\usepackage{geometry}
\geometry{verbose,a4paper, inner=0cm, outer=0 cm, bmargin=2cm, tmargin=1cm}
%\textwidth=12cm
\setlength{\parindent}{0bp}
\usepackage{import}
\usepackage[subpreambles=false]{standalone}
\usepackage{amsmath}
\usepackage{amssymb}
\usepackage{esint}
\usepackage{babel}
\usepackage{tabu}
\usepackage[dvipsnames, table]{xcolor}
\usepackage{cancel}
\makeatother
\makeatletter
\usepackage{datetime2}
\usepackage{titlesec}
\usepackage[many]{tcolorbox}

% Eheter
\newcommand{\enh}[1]{\,\textrm{#1}}
%referances
\newcommand{\net}[2]{{\color{blue}\href{#1}{#2}}}

%Spaces
\newcommand{\vsk}{\\[12pt]}
\newcommand{\vs}{\vspace{-12pt}}

% Tabell for opplegg

\newcommand{\ovlist}[1]{
\vspace{-16pt}
\begin{itemize}
	#1
\end{itemize}
}

% Chapters and sections
\titleformat{\section}[block]{\bfseries}{\hspace{3cm}\thesection}{5pt}{}
\titleformat{\subsection}[block]{\bfseries}{\hspace{3cm}\thesection}{5pt}{}
\newcommand{\sectionbreak}{\clearpage} % New page on each section
 

\newlength{\mywidth}
\setlength{\mywidth}{14cm}

\newcommand{\cont}[1]{
\begin{tcolorbox}[center, boxrule=0.0 mm, width=\mywidth,arc=0mm,enhanced jigsaw,,colback=white,breakable]
#1	
\end{tcolorbox}
}

\newcommand{\info}[5]{
\begin{tcolorbox}[center, boxrule=0.1 mm, width=\mywidth,arc=0mm,enhanced jigsaw,breakable,colback=yellow!5]	
	
	\footnotesize
	\textbf{Øvingsområde}\\[5pt] #1 
	
	\textbf{Utstyr}\\ #2  \\
	
	\begin{tabular}{@{} p{4cm} p{4cm} l} 
		\textbf{Tid} & \textbf{Elevinndeling} & \textbf{Læringsarena} \\
		#3  & #4 & #5
	\end{tabular} 
\end{tcolorbox}	
}

\newcommand{\gjen}[1]{\begin{tcolorbox}[center,boxrule=0.1 mm, width=\mywidth,arc=0mm,colback=blue!3] {\large \textbf{Gjennomføring} \vspace{5 pt}}\newline #1  \end{tcolorbox}\vspace{-5pt}}
\newcommand{\eks}[1]{\begin{tcolorbox}[center,boxrule=0.1 mm, width=\mywidth,arc=0mm,colback=green!3] {\large \textbf{Eksempel} \vspace{5 pt}}\newline #1  \end{tcolorbox}\vspace{-5pt}}

\newcounter{opl}
%\numberwithin{opl}{article}


\newcommand{\opl}[1]{
\newpage
{\refstepcounter{opl} %\phantomsection 
\large \textbf{\theopl \;#1} \vsk}
}

% Headlines
\newcommand{\fork}{\textbf{Forkunnskapar}\\}
\newcommand{\forb}{\textbf{Forberedelsar}\\}
\newcommand{\opgvr}{\textbf{Oppgaver}}



%colors
\newcommand{\colr}[1]{{\color{red} #1}}
\newcommand{\colb}[1]{{\color{blue} #1}}
\newcommand{\colo}[1]{{\color{orange} #1}}
\newcommand{\colc}[1]{{\color{cyan} #1}}
\definecolor{projectgreen}{cmyk}{100,0,100,0}
\newcommand{\colg}[1]{{\color{projectgreen} #1}}

% Lister med bokstavar
\usepackage[inline]{enumitem}
% Opg
\newcommand{\abc}[1]{
	\begin{enumerate}[label=\alph*),leftmargin=18pt]
		#1
	\end{enumerate}
}

\usepackage[]{hyperref}
%%% SECTION HEADLINES %%%

% note
\newcommand{\note}{Merk}
\newcommand{\notesm}[1]{{\footnotesize \textsl{\note:} #1}}
\newcommand{\selos}{Se løsnigsforslag.}
\newcommand{\ekstitle}{Eksempel }
\newcommand{\sprtitle}{Språkboksen}
\newcommand{\expl}{forklaring}

\newcommand\sv{\vsk \textbf{Answer} \vspace{4 pt}\\}

%references
\newcommand{\reftab}[1]{\hrs{#1}{tabell}}
\newcommand{\rref}[1]{\hrs{#1}{regel}}
\newcommand{\dref}[1]{\hrs{#1}{definisjon}}
\newcommand{\refkap}[1]{\hrs{#1}{kapittel}}
\newcommand{\refsec}[1]{\hrs{#1}{seksjon}}
\newcommand{\refdsec}[1]{\hrs{#1}{delseksjon}}
\newcommand{\refved}[1]{\hrs{#1}{vedlegg}}
\newcommand{\eksref}[1]{\textsl{#1}}
\newcommand\fref[2][]{\hyperref[#2]{\textsl{figur \ref*{#2}#1}}}
\newcommand{\refop}[1]{{\color{blue}Oppgave \ref{#1}}}
\newcommand{\refops}[1]{{\color{blue}oppgave \ref{#1}}}


% Our numbers
\newcommand{\likteikn}{Likskapsteiknet}
\newcommand{\talsifverd}{Tal, siffer og verdi}
\newcommand{\koordsys}{Koordinatsystem}

% Calculations
\newcommand{\adi}{Addisjon}
\newcommand{\sub}{Subtraksjon}
\newcommand{\gong}{Multiplikasjon (Gonging)}
\newcommand{\del}{Divisjon (deling)}

%Factorization and order of operations
\newcommand{\fak}{Faktorisering}
\newcommand{\rrek}{Reknerekkefølge}

%Fractions
\newcommand{\brgrpr}{Introduksjon}
\newcommand{\brvu}{Verdi, utviding og forkorting av brøk}
\newcommand{\bradsub}{Addisjon og subtraksjon}
\newcommand{\brgngheil}{Brøk gonga med heiltal}
\newcommand{\brdelheil}{Brøk delt med heiltal}
\newcommand{\brgngbr}{Brøk gonga med brøk}
\newcommand{\brkans}{Kansellering av faktorar}
\newcommand{\brdelmbr}{Deling med brøk}
\newcommand{\Rasjtal}{Rasjonale og blanda tal}

%Negative numbers
\newcommand{\negintro}{Introduksjon}
\newcommand{\negrekn}{Dei fire rekneartane med negative tal}
\newcommand{\negmeng}{Negative tal som mengde}

% Geometry 1
\newcommand{\omgr}{Omgrep}
\newcommand{\eignsk}{Eigenskapar for trekantar og firkantar}
\newcommand{\omkr}{Omkrins}
\newcommand{\area}{Areal}
\newcommand{\pytomv}{\pyt\, (omvendt versjon)}

%Algebra 
\newcommand{\algintro}{Introduksjon}
\newcommand{\pot}{Potensar}
\newcommand{\irrasj}{Irrasjonale tal}

%Equations
\newcommand{\ligintro}{Introduksjon}
\newcommand{\liglos}{Løysing ved dei fire rekneartane}
\newcommand{\ligloso}{Løysingsmetodane oppsummert}

%Functions
\newcommand{\fintro}{Introduksjon}
\newcommand{\lingraf}{Lineære funksjonar og grafar}

%Geometry 2
\newcommand{\geoform}{Formlar for areal og omkrins}
\newcommand{\kongogsim}{Kongruente og formlike trekantar}
\newcommand{\geofork}{Forklaringar}

% Names of rules
\newcommand{\gangdestihundre}{Å gange desimaltall med 10, 100 osv.}
\newcommand{\delmedtihundre}{Deling med 10, 100, 1\,000 osv.}
\newcommand{\adkom}{Addisjon er kommutativ}
\newcommand{\gangkom}{Multiplikasjon er kommutativ}
\newcommand{\brdef}{Brøk som omskriving av delestykke}
\newcommand{\brtbr}{Brøk gonga med brøk}
\newcommand{\delmbr}{Brøk delt på brøk}
\newcommand{\gangpar}{Gonging med parentes (distributiv lov)}
\newcommand{\gangparsam}{Parantesar gonga saman}
\newcommand{\gangmnegto}{Gonging med negative tal I}
\newcommand{\gangmnegtre}{Gonging med negative tal II}
\newcommand{\konsttre}{Konstruksjon av trekantar}
\newcommand{\kongtre}{Kongruente trekantar}
\newcommand{\topv}{Toppvinklar}
\newcommand{\trisum}{Summen av vinklane i ein trekant}
\newcommand{\firsum}{Summen av vinklane i ein firkant}
\newcommand{\potgang}{Gonging med potensar}
\newcommand{\potdivpot}{Divisjon med potensar}
\newcommand{\potanull}{Spesialtilfellet \boldmath $a^0$}
\newcommand{\potneg}{Potens med negativ eksponent}
\newcommand{\potbr}{Brøk som grunntal}
\newcommand{\faktgr}{Faktorar som grunntal}
\newcommand{\potsomgrunn}{Potens som grunntal}
\newcommand{\arsirk}{Arealet til ein sirkel}
\newcommand{\artrap}{Arealet til eit trapes}
\newcommand{\arpar}{Arealet til eit parallellogram}
\newcommand{\pyt}{Pytagoras' setning}
\newcommand{\volforml}{Volumet til tredimensjonale former}
\newcommand{\volkule}{Volumet til ei kule}
\newcommand{\forform}{Forhold i formlike trekantar}
\newcommand{\vilkform}{Vilkår i formlike trekantar}
\newcommand{\omkrsirk}{Omkrinsen til ein sirkel (og $ \bm \pi $)}
\newcommand{\artri}{Arealet til ein trekant}
\newcommand{\arrekt}{Arealet til eit rektangel}
\newcommand{\liknflyt}{Flytting av ledd over likskapsteiknet}
\newcommand{\funklin}{Lineære funksjonar}


\begin{document}
	\newpage
	\section{\adi \label{Addisjon}}	
	\subsection*{Addisjon med mengder: Å legge til}
	Når vi har ei mengde og skal legge til meir, bruker vi symbolet \sym{$ + $}. Har vi 2 og skal legge til 3, skriv vi
	\[ 2+3=5 \]
	\fig{plusm1}
	Rekkefølga vi legg saman tala på har ikkje noko å seie; å starte med 2 og så legge til 3 er det same som å starte med 3 og så legge til 2:
	\[ 3+2=5 \]
	\fig{plusm1a}
	\spr{
		Eit addisjonsstykke består av to eller fleire \textit{ledd}\index{ledd} og éin \textit{sum}\index{sum}. I reknestykket
		\[ 2+3=5 \]
		er både $ 2 $ og $ 3 $ ledd, mens $ 5 $ er summen.\vsk
		
		Vanlege måtar å seie $ 2+3 $ på er
		\begin{itemize}
			\item ''2 pluss 3''
			\item ''2 addert med 3''
			\item ''2 og 3 lagt saman''
		\end{itemize}
		Det å legge saman tal kallast også \textit{å summere}.
		
	} \vsk
	\newpage
	\reg[Addisjon er kommutativ \label{adkom}]{
		Summen er den same uansett rekkefølge på ledda.
	}
	\eks{ \vs \vsb
		\alg{
			2+5 &= 7 =5+2  \vn
			6+3 &=9=3+6
		}
	}
	
	\subsection*{Addisjon på tallinja: Vandring mot høgre}
	På ei tallinje vil addisjon med positive tal  innebere vandring \textsl{mot \\høgre}:\regv
	\eks[1]{
		\[ 2+7=9 \]
		\fig{plus2}
	}
	\eks[2]{
		\[ 4+11=15 \]
		\fig{plus3}
	}
	\info{Tydinga av \sym{$\bm=$}}{
		\sym{+} gir oss moglegheiten til å uttrykke tal på mange forskjellige måtar, for eksempel er $ {5=2+3} $ og $ {5=1+4} $. I denne samanhengen vil \sym{=} bety ''har same verdi som''. Dette gjeld også ved subtraksjon, multiplikasjon og divisjon, som vi skal sjå på i dei neste tre seksjonane.
	}
	
\section{\sub\label{Subtraksjon}}
\subsection*{Subtraksjon med mengder: Å trekke ifrå}
Når vi har ei mengde og tar bort ein del av den, bruker vi symbolet \sym{$ - $}:
\[ 5-{\color{red} 3}=2 \]
\fig{min1c}

\spr{
	Eit subtraksjonsstykke består av to eller fleire \textit{ledd} \index{ledd} og éin\\ \textit{differanse}\index{differanse}. I subtraksjonsstykket
	\[  5-3=2 \] 
	er både $ 5 $ og $ 3 $ ledd og $ 2 $ er differansen. \vsk
	
	Vanlege måtar å seie $ 5-3 $ på er
	\begin{itemize}
		\item ''5 minus 3'' \\
		\item ''5 fratrekt 3''
		\item ''3 subtrahert fra 5''
	\end{itemize}
} \vsk \vsk

\info{Ei ny tolking av 0}{
	Innleiingsvis i denne boka nemnde vi at 0 kan tolkast som ''ingenting''. Subtraksjon gir oss moglegheiten til å uttrykke 0 via andre tal. For eksempel er $ {7-7=0} $ og $ {19-19=0} $. I praktiske samanhengar vil 0 ofte innebere ei form for likevekt, for eksempel som at ei kraft og ei motkraft er like store.
} \vsk

\newpage
\subsection*{Subtraksjon på tallinja: Vandring mot venstre}
I \hrs{Addisjon}{seksjon} har vi sett at \sym{$ + $} (med positive tal) inneber at vi skal gå \\[2pt] \textsl{mot høgre} langs tallinja. Med \sym{$ - $} gjer vi omvend, vi går \textsl{mot venstre}\footnote{I figurar med tallinjer vil raudfarga piler indikere at ein startar ved pilspissen og vandrar til andre enden.}: \regv

\eks[1]{ \vs
	\[ 6-4=2 \]
	\fig{mint}
}
\eks[2]{ \vs
	\[ 12-7=5 \]
	\fig{mint2}
}
\info{Merk}{
	Med det første kan det kanskje verke litt rart at ein i \textsl{Eksempel 1} og \textsl{2} over skal gå i motsatt veg av retninga pila peiker i, men spesielt i \hrs{Negtal}{Kapittel} vil det lønne seg å tenke slik.
}


\section{\gong \label{Gonging} }

\subsection*{Gonging med heiltal: Innleiiande definisjon }
Når vi legg saman like tall, kan vi bruke gonge-symbolet \sym{$ \cdot $}\;for å skrive reknestykka våre kortare: \regv
\eks[]{\vsb \vs
	\alg{
		4+4+4 &= 4\cdot 3 \vn
		8+8 &=8\cdot 2 \vn
		1+1+1+1+1&= 1\cdot5 
	}
} \regv
\spr{
	Eit gongestykke består av to eller fleire \outl{faktorar}\index{faktor} og eitt\\ \outl{produkt}\index{produkt}. I gongestykket
	\[ {4\cdot 3=12} \]
	er $ 4 $ og $ 3 $ faktorar, mens $ 12 $ er produktet. \vsk
	
	Vanlege måtar å seie $ 4\cdot3 $ på er
	\begin{itemize}
		\item ''4 gonger 3''  \\
		\item ''4 gonga med 3''\\
		\item ''4 multiplisert med 3''
	\end{itemize}
	
	Mange nettstader og bøker på engelsk brukar symbolet \sym{$ \times $} i staden for \sym{$ \cdot $}. I dei fleste programmeringsspråk er \sym{*} symbolet for multiplikasjon.
}
\newpage
\subsection*{Gonging av mengder}
La oss no bruke ein figur for å sjå for oss gongestykket $ 2\cdot3 $:
\fig{2t3}
Og så kan vi legge merke til produktet til $ 3\cdot 2 $:
\fig{3t2}
\reg[\gangkom \label{gangkom}]{
	Produktet er det same uansett rekkefølge på faktorane.
}
\eks[]{\vsb \vs
	\alg{
		3\cdot 4 &=12= 4\cdot 3 \vn
		6\cdot 7 &=42= 7\cdot6 \vn
		8\cdot 9 &=72= 9\cdot8
	}
}

\subsection*{Gonging på tallinja}
Vi kan også bruke tallinja for å rekne ut gongestykker. For eksempel kan vi finne kva $ 2\cdot4 $ er ved å tenke slik:
\[\text{''} 2\cdot 4 \text{ betyr å vandre 2 plassar mot høgre, 4 gonger.}\text{''} \]
\[ 2\cdot4=8 \]
\fig{2t4l}

Også tallinja kan vi bruke for å overbevise oss om at rekkefølga i eit gongestykke ikkje har noko å seie:
\[\text{''} 4\cdot 2 \text{ betyr å vandre 4 plassar mot høgre, 2 gonger.}\text{''} \]
\[ 4\cdot2=8 \]
\fig{4t2l}


\subsection*{Endeleg definisjon av gonging med positive heiltal}
Det ligg kanskje nærast å tolke ''2 gonger 3'' som ''3, 2 gonger''. Da er
\[ \text{''2 gonger 3''}=3+3 \]
Innleiingsvis presenterete vi $ {2\cdot3} $, altså ''2 gonger 3'', som $ {2+2+2} $. Med denne tolkinga vil $ {3+3} $ svare til $ {3\cdot2} $, men nettopp det at multiplikasjon er ein kommutativ operasjon (\rref{gangkom}) gjer at den eine tolkinga ikkje utelukkar den andre; $ {2\cdot3 =2+2+2} $ og $ {2\cdot3=3+3} $ er to uttrykk med same verdi.\regv

\reg[Gonging som gjentatt addisjon \label{ganggjad2}]{
	Gonging med eit positivt heiltal kan uttrykkast som gjentatt addisjon.
}
\eks[1]{\vsb \vs
	\alg{
		4+4+4 &= 4\cdot 3=3+3+3+3 \vn
		8+8 &=8\cdot 2=2+2+2+2+2+2+2 \vn
		1+1+1+1+1&= 1\cdot5 =5
	}
}
\info{Merk}{
	At gonging med positive heiltal kan uttrykkast som gjentatt addisjon, utelukkar ikkje andre uttrykk. Det er ikkje feil å skrive at $ {2\cdot 3=1+5} $.
} 


\section{\del \label{Divisjon}}
\sym{:} er \outl{divisjonstegnet}.
I praksis har divisjon tre forskjellige tydingar, her eksemplifisert ved regnestykket $ 12:3 $:\regv

\reg[Divisjon sine tre tydingar]{ \vs
	\begin{itemize}
		\item \textbf{Inndeling av mengder} \\
		$ 12:3 = \text{''Antallet i kvar gruppe når 12 delast inn i 3 like}$\\
		\hspace{1.6cm}store grupper'' 
		\item \textbf{Antall gongar} \\
		$ 12:3=\text{''Antall gongar 3 går på 12''} $
		\item \textbf{Omvendt operasjon av multiplikasjon}\\
		$ 12:3=\text{''Tallet ein må gonge 3 med for å få 12''} $
	\end{itemize}
} \regv

\spr{ \label{sprakdiv}
	Eit divisjonsstykke består av ein \outl{dividend}\index{dividend}, ein \outl{divisor}\index{divisor} og ein\\ \outl{kvotient}\index{kvotient}.	
	I divisjonstykket
	\[  {12:3=4} \]
	er $ 12 $ dividenden, $ 3 $ er divisoren og $ 4 $ er kvotienten.\vsk
	
	Vanlege måtar å uttale $ 12:3 $ på er
	\begin{itemize}
		\item ''12 delt med/på 3''
		\item ''12 dividert med/på 3''
		\item ''12 på 3''
	\end{itemize}
	I nokre samanhengar blir $ {12:3} $ kalt ''\outl{forholdet}\index{forhold} mellom 12 og 3''. Da er 4 \outl{forholdstallet}\index{forholdstal}. \vsk
	
	Ofte brukast \sym{$ / $} i staden for \sym{$:$}, spesielt i programmeringsspråk.
}
\newpage
\subsection*{Divisjon av mengder}
Reknestykket $ {12:3} $ fortell oss at vi skal dele 12 inn i 3 like store \\grupper:
\fig{del1}
Vi ser at kvar gruppe inneheld 4 ruter, dette betyr at
\[ 12:3=4 \]


\subsection*{Antal gongar}
\fig{del2}
3 går 4 ganger på 12, altså er $ 12:3=4 $.


\subsection*{Omvend operasjon av multiplikasjon}
Vi har sett at viss vi deler 12 inn i 3 like grupper, får vi 4 i kvar gruppe. Altså er $ 12:3=4$. Om vi legg saman igjen desse gruppene, får vi naturligvis 12: 
\fig{del1b}
Men dette er det samme som å gonge 4 med 3. Altså;
om vi veit at $ {4\cdot 3=12} $, så veit vi også at $ {12:3=4} $. I tillegg veit vi da at $ {12:4=3} $. 
\fig{del1a}


\eks[1]{Sidan $ 6\cdot 3 = 18  $, er\vs
	\alg{
		18:6= 3\vn
		18:3=6
	}
}
\eks[2]{
	Sidan $ 5\cdot 7 = 35  $, er\vs
	\alg{
		35:5= 7\vn
		35:7=5
	}
}

\end{document}