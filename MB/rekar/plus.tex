\documentclass[english,hidelinks,pdftex, 11 pt, class=report,crop=false]{standalone}
\usepackage[T1]{fontenc}
\usepackage[utf8]{luainputenc}
\usepackage{lmodern} % load a font with all the characters
\usepackage{geometry}
\geometry{verbose,a4paper, inner=0cm, outer=0 cm, bmargin=2cm, tmargin=1cm}
%\textwidth=12cm
\setlength{\parindent}{0bp}
\usepackage{import}
\usepackage[subpreambles=false]{standalone}
\usepackage{amsmath}
\usepackage{amssymb}
\usepackage{esint}
\usepackage{babel}
\usepackage{tabu}
\usepackage[dvipsnames, table]{xcolor}
\usepackage{cancel}
\makeatother
\makeatletter
\usepackage{datetime2}
\usepackage{titlesec}
\usepackage[many]{tcolorbox}

% Eheter
\newcommand{\enh}[1]{\,\textrm{#1}}
%referances
\newcommand{\net}[2]{{\color{blue}\href{#1}{#2}}}

%Spaces
\newcommand{\vsk}{\\[12pt]}
\newcommand{\vs}{\vspace{-12pt}}

% Tabell for opplegg

\newcommand{\ovlist}[1]{
\vspace{-16pt}
\begin{itemize}
	#1
\end{itemize}
}

% Chapters and sections
\titleformat{\section}[block]{\bfseries}{\hspace{3cm}\thesection}{5pt}{}
\titleformat{\subsection}[block]{\bfseries}{\hspace{3cm}\thesection}{5pt}{}
\newcommand{\sectionbreak}{\clearpage} % New page on each section
 

\newlength{\mywidth}
\setlength{\mywidth}{14cm}

\newcommand{\cont}[1]{
\begin{tcolorbox}[center, boxrule=0.0 mm, width=\mywidth,arc=0mm,enhanced jigsaw,,colback=white,breakable]
#1	
\end{tcolorbox}
}

\newcommand{\info}[5]{
\begin{tcolorbox}[center, boxrule=0.1 mm, width=\mywidth,arc=0mm,enhanced jigsaw,breakable,colback=yellow!5]	
	
	\footnotesize
	\textbf{Øvingsområde}\\[5pt] #1 
	
	\textbf{Utstyr}\\ #2  \\
	
	\begin{tabular}{@{} p{4cm} p{4cm} l} 
		\textbf{Tid} & \textbf{Elevinndeling} & \textbf{Læringsarena} \\
		#3  & #4 & #5
	\end{tabular} 
\end{tcolorbox}	
}

\newcommand{\gjen}[1]{\begin{tcolorbox}[center,boxrule=0.1 mm, width=\mywidth,arc=0mm,colback=blue!3] {\large \textbf{Gjennomføring} \vspace{5 pt}}\newline #1  \end{tcolorbox}\vspace{-5pt}}
\newcommand{\eks}[1]{\begin{tcolorbox}[center,boxrule=0.1 mm, width=\mywidth,arc=0mm,colback=green!3] {\large \textbf{Eksempel} \vspace{5 pt}}\newline #1  \end{tcolorbox}\vspace{-5pt}}

\newcounter{opl}
%\numberwithin{opl}{article}


\newcommand{\opl}[1]{
\newpage
{\refstepcounter{opl} %\phantomsection 
\large \textbf{\theopl \;#1} \vsk}
}

% Headlines
\newcommand{\fork}{\textbf{Forkunnskapar}\\}
\newcommand{\forb}{\textbf{Forberedelsar}\\}
\newcommand{\opgvr}{\textbf{Oppgaver}}



%colors
\newcommand{\colr}[1]{{\color{red} #1}}
\newcommand{\colb}[1]{{\color{blue} #1}}
\newcommand{\colo}[1]{{\color{orange} #1}}
\newcommand{\colc}[1]{{\color{cyan} #1}}
\definecolor{projectgreen}{cmyk}{100,0,100,0}
\newcommand{\colg}[1]{{\color{projectgreen} #1}}

% Lister med bokstavar
\usepackage[inline]{enumitem}
% Opg
\newcommand{\abc}[1]{
	\begin{enumerate}[label=\alph*),leftmargin=18pt]
		#1
	\end{enumerate}
}

\usepackage[]{hyperref}
\begin{document}
\newpage
\section{\adi \label{Addisjon}}	
\subsection*{Addisjon med mengder: Å legge til}
Når vi har ei mengde og skal legge til meir, bruker vi symbolet \sym{$ + $}. Har vi 2 og skal legge til 3, skriv vi
\[ 2+3=5 \]
\fig{plusm1}
Rekkefølga vi legg saman tala på har ikkje noko å seie; å starte med 2 og så legge til 3 er det same som å starte med 3 og så legge til 2:
\[ 3+2=5 \]
\fig{plusm1a}
\spr{
	Eit addisjonsstykke består av to eller fleire \textit{ledd}\index{ledd} og éin \textit{sum}\index{sum}. I reknestykket
	\[ 2+3=5 \]
	er både $ 2 $ og $ 3 $ ledd, mens $ 5 $ er summen.\vsk
	
	Vanlege måtar å seie $ 2+3 $ på er
	\begin{itemize}
		\item ''2 pluss 3''
		\item ''2 addert med 3''
	 	\item ''2 og 3 lagt saman''
	\end{itemize}
Det å legge saman tal kallast også \textit{å summere}.

} \vsk
\newpage
\reg[Addisjon er kommutativ \label{adkom}]{
Summen er den same uansett rekkefølge på ledda.
}
\eks{ \vs \vsb
\alg{
2+5 &= 7 =5+2  \vn
6+3 &=9=3+6
}
}

\subsection*{Addisjon på tallinja: Vandring mot høgre}
På ei tallinje vil addisjon med positive tal  innebere vandring \textsl{mot \\høgre}:\regv
\eks[1]{
\[ 2+7=9 \]
\fig{plus2}
}
\eks[2]{
	\[ 4+11=15 \]
\fig{plus3}
}
\info{Tydinga av \sym{$\bm=$}}{
\sym{+} gir oss moglegheiten til å uttrykke tal på mange forskjellige måtar, for eksempel er $ {5=2+3} $ og $ {5=1+4} $. I denne samanhengen vil \sym{=} bety ''har same verdi som''. Dette gjeld også ved subtraksjon, multiplikasjon og divisjon, som vi skal sjå på i dei neste tre seksjonane.
}


\end{document}