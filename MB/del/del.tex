\documentclass[english,hidelinks,pdftex, 11 pt, class=report,crop=false]{standalone}
\usepackage[T1]{fontenc}
\usepackage[utf8]{luainputenc}
\usepackage{lmodern} % load a font with all the characters
\usepackage{geometry}
\geometry{verbose,a4paper, inner=0cm, outer=0 cm, bmargin=2cm, tmargin=1cm}
%\textwidth=12cm
\setlength{\parindent}{0bp}
\usepackage{import}
\usepackage[subpreambles=false]{standalone}
\usepackage{amsmath}
\usepackage{amssymb}
\usepackage{esint}
\usepackage{babel}
\usepackage{tabu}
\usepackage[dvipsnames, table]{xcolor}
\usepackage{cancel}
\makeatother
\makeatletter
\usepackage{datetime2}
\usepackage{titlesec}
\usepackage[many]{tcolorbox}

% Eheter
\newcommand{\enh}[1]{\,\textrm{#1}}
%referances
\newcommand{\net}[2]{{\color{blue}\href{#1}{#2}}}

%Spaces
\newcommand{\vsk}{\\[12pt]}
\newcommand{\vs}{\vspace{-12pt}}

% Tabell for opplegg

\newcommand{\ovlist}[1]{
\vspace{-16pt}
\begin{itemize}
	#1
\end{itemize}
}

% Chapters and sections
\titleformat{\section}[block]{\bfseries}{\hspace{3cm}\thesection}{5pt}{}
\titleformat{\subsection}[block]{\bfseries}{\hspace{3cm}\thesection}{5pt}{}
\newcommand{\sectionbreak}{\clearpage} % New page on each section
 

\newlength{\mywidth}
\setlength{\mywidth}{14cm}

\newcommand{\cont}[1]{
\begin{tcolorbox}[center, boxrule=0.0 mm, width=\mywidth,arc=0mm,enhanced jigsaw,,colback=white,breakable]
#1	
\end{tcolorbox}
}

\newcommand{\info}[5]{
\begin{tcolorbox}[center, boxrule=0.1 mm, width=\mywidth,arc=0mm,enhanced jigsaw,breakable,colback=yellow!5]	
	
	\footnotesize
	\textbf{Øvingsområde}\\[5pt] #1 
	
	\textbf{Utstyr}\\ #2  \\
	
	\begin{tabular}{@{} p{4cm} p{4cm} l} 
		\textbf{Tid} & \textbf{Elevinndeling} & \textbf{Læringsarena} \\
		#3  & #4 & #5
	\end{tabular} 
\end{tcolorbox}	
}

\newcommand{\gjen}[1]{\begin{tcolorbox}[center,boxrule=0.1 mm, width=\mywidth,arc=0mm,colback=blue!3] {\large \textbf{Gjennomføring} \vspace{5 pt}}\newline #1  \end{tcolorbox}\vspace{-5pt}}
\newcommand{\eks}[1]{\begin{tcolorbox}[center,boxrule=0.1 mm, width=\mywidth,arc=0mm,colback=green!3] {\large \textbf{Eksempel} \vspace{5 pt}}\newline #1  \end{tcolorbox}\vspace{-5pt}}

\newcounter{opl}
%\numberwithin{opl}{article}


\newcommand{\opl}[1]{
\newpage
{\refstepcounter{opl} %\phantomsection 
\large \textbf{\theopl \;#1} \vsk}
}

% Headlines
\newcommand{\fork}{\textbf{Forkunnskapar}\\}
\newcommand{\forb}{\textbf{Forberedelsar}\\}
\newcommand{\opgvr}{\textbf{Oppgaver}}



%colors
\newcommand{\colr}[1]{{\color{red} #1}}
\newcommand{\colb}[1]{{\color{blue} #1}}
\newcommand{\colo}[1]{{\color{orange} #1}}
\newcommand{\colc}[1]{{\color{cyan} #1}}
\definecolor{projectgreen}{cmyk}{100,0,100,0}
\newcommand{\colg}[1]{{\color{projectgreen} #1}}

% Lister med bokstavar
\usepackage[inline]{enumitem}
% Opg
\newcommand{\abc}[1]{
	\begin{enumerate}[label=\alph*),leftmargin=18pt]
		#1
	\end{enumerate}
}

\usepackage[]{hyperref}


\begin{document}
\newpage
\section{\del \label{Divisjon}}
\sym{:} er teiknet for divisjon.
I praksis har divisjon tre forskjellige tydingar, her eksemplifisert ved reknestykket $ 12:3 $:\regv

\reg[Divisjon sine tre tydingar]{ \vs
\begin{itemize}
	\item \textbf{Inndeling av mengder} \\
	$ 12:3 = \text{''Antalet i kvar gruppe når 12 delast inn i 3 like}$\\
	\hspace{1.6cm}store grupper'' 
	\item \textbf{Antal gonger} \\
	$ 12:3=\text{''Antal gonger 3 går på 12''} $
	\item \textbf{Omvend operasjon av multiplikasjon}\\
	$ 12:3=\text{''Talet ein må gonge 3 med for å få 12''} $
\end{itemize}
} \regv

\spr{ \label{sprakdiv}
Eit divisjonsstykke består av ein \textit{dividend}\index{dividend}, ein \textit{divisor}\index{divisor} og ein \textit{kvotient}\index{kvotient}.	
I divisjonstykket
\[  {12:3=4} \]
er $ 12 $ dividenden, $ 3 $ er divisoren og $ 4 $ er kvotienten.\vsk

Vanlege måtar å seie $ 12:3 $ på er
\begin{itemize}
	\item ''12 delt med 3''
	\item ''12 dividert med 3''
	\item ''12 på 3''
\end{itemize}
I nokre samanhengar blir $ 12:3 $ kalla ''\textit{forholdet}\index{forhold} mellom 12 og 3''. Da er 4 \textit{forholdstalet}\index{forholdstal}. \vsk

Ofte brukast \sym{$ / $} i staden for \sym{$:$}, spesielt i programmeringsspråk.
}
\newpage
\subsection*{Divisjon av mengder}
Reknestykket $ 12:3 $ fortel oss at vi skal dele 12 inn i 3 like store \\grupper:
\fig{del1}
Vi ser at kvar gruppe inneheld 4 ruter, dette betyr at
\[ 12:3=4 \]


\subsection*{Antal gonger}
	\fig{del2}
3 går 4 gonger på 12, altså er $ 12:3=4 $.


\subsection*{Omvend operasjon av multiplikasjon}
Vi har nyleg sett at viss vi deler 12 inn i 3 like grupper, får vi 4 i kvar gruppe. Altså er $ 12:3=4$. Og om vi legg saman igjen desse gruppene, får vi 12: 
\fig{del1b}
Men dette er det same som å gonge 4 med 3, med andre ord:
Om vi veit at $ {4\cdot 3=12} $, så veit vi at $ {12:3=4} $. I tillegg veit vi da at $ {12:4=3} $. 
\fig{del1a}


\eks[1]{Sidan $ 6\cdot 3 = 18  $, er\vs
	\alg{
		18:6= 3\vn
		18:3=6
	}
}
\eks[2]{
	Sidan $ 5\cdot 7 = 35  $, er\vs
	\alg{
		35:5= 7\vn
		35:7=5
	}
}

\end{document}

