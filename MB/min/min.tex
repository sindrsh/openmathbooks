\documentclass[english,hidelinks,pdftex, 11 pt, class=report,crop=false]{standalone}
\usepackage[T1]{fontenc}
\usepackage[utf8]{luainputenc}
\usepackage{lmodern} % load a font with all the characters
\usepackage{geometry}
\geometry{verbose,a4paper, inner=0cm, outer=0 cm, bmargin=2cm, tmargin=1cm}
%\textwidth=12cm
\setlength{\parindent}{0bp}
\usepackage{import}
\usepackage[subpreambles=false]{standalone}
\usepackage{amsmath}
\usepackage{amssymb}
\usepackage{esint}
\usepackage{babel}
\usepackage{tabu}
\usepackage[dvipsnames, table]{xcolor}
\usepackage{cancel}
\makeatother
\makeatletter
\usepackage{datetime2}
\usepackage{titlesec}
\usepackage[many]{tcolorbox}

% Eheter
\newcommand{\enh}[1]{\,\textrm{#1}}
%referances
\newcommand{\net}[2]{{\color{blue}\href{#1}{#2}}}

%Spaces
\newcommand{\vsk}{\\[12pt]}
\newcommand{\vs}{\vspace{-12pt}}

% Tabell for opplegg

\newcommand{\ovlist}[1]{
\vspace{-16pt}
\begin{itemize}
	#1
\end{itemize}
}

% Chapters and sections
\titleformat{\section}[block]{\bfseries}{\hspace{3cm}\thesection}{5pt}{}
\titleformat{\subsection}[block]{\bfseries}{\hspace{3cm}\thesection}{5pt}{}
\newcommand{\sectionbreak}{\clearpage} % New page on each section
 

\newlength{\mywidth}
\setlength{\mywidth}{14cm}

\newcommand{\cont}[1]{
\begin{tcolorbox}[center, boxrule=0.0 mm, width=\mywidth,arc=0mm,enhanced jigsaw,,colback=white,breakable]
#1	
\end{tcolorbox}
}

\newcommand{\info}[5]{
\begin{tcolorbox}[center, boxrule=0.1 mm, width=\mywidth,arc=0mm,enhanced jigsaw,breakable,colback=yellow!5]	
	
	\footnotesize
	\textbf{Øvingsområde}\\[5pt] #1 
	
	\textbf{Utstyr}\\ #2  \\
	
	\begin{tabular}{@{} p{4cm} p{4cm} l} 
		\textbf{Tid} & \textbf{Elevinndeling} & \textbf{Læringsarena} \\
		#3  & #4 & #5
	\end{tabular} 
\end{tcolorbox}	
}

\newcommand{\gjen}[1]{\begin{tcolorbox}[center,boxrule=0.1 mm, width=\mywidth,arc=0mm,colback=blue!3] {\large \textbf{Gjennomføring} \vspace{5 pt}}\newline #1  \end{tcolorbox}\vspace{-5pt}}
\newcommand{\eks}[1]{\begin{tcolorbox}[center,boxrule=0.1 mm, width=\mywidth,arc=0mm,colback=green!3] {\large \textbf{Eksempel} \vspace{5 pt}}\newline #1  \end{tcolorbox}\vspace{-5pt}}

\newcounter{opl}
%\numberwithin{opl}{article}


\newcommand{\opl}[1]{
\newpage
{\refstepcounter{opl} %\phantomsection 
\large \textbf{\theopl \;#1} \vsk}
}

% Headlines
\newcommand{\fork}{\textbf{Forkunnskapar}\\}
\newcommand{\forb}{\textbf{Forberedelsar}\\}
\newcommand{\opgvr}{\textbf{Oppgaver}}



%colors
\newcommand{\colr}[1]{{\color{red} #1}}
\newcommand{\colb}[1]{{\color{blue} #1}}
\newcommand{\colo}[1]{{\color{orange} #1}}
\newcommand{\colc}[1]{{\color{cyan} #1}}
\definecolor{projectgreen}{cmyk}{100,0,100,0}
\newcommand{\colg}[1]{{\color{projectgreen} #1}}

% Lister med bokstavar
\usepackage[inline]{enumitem}
% Opg
\newcommand{\abc}[1]{
	\begin{enumerate}[label=\alph*),leftmargin=18pt]
		#1
	\end{enumerate}
}

\usepackage[]{hyperref}
\begin{document}

\section{\sub\label{Subtraksjon}}
\subsection*{Subtraksjon med mengder: Å trekke ifrå}
Når vi har ei mengde og tar bort ein del av den, bruker vi symbolet \sym{$ - $}:
\[ 5-{\color{red} 3}=2 \]
\fig{min1c}

\spr{
Eit subtraksjonsstykke består av to eller fleire \textit{ledd} \index{ledd} og éin\\ \textit{differanse}\index{differanse}. I subtraksjonsstykket
\[  5-3=2 \] 
er både $ 5 $ og $ 3 $ ledd og $ 2 $ er differansen. \vsk

Vanlege måtar å seie $ 5-3 $ på er
\begin{itemize}
	\item ''5 minus 3'' \\
	\item ''5 fratrekt 3''
	\item ''3 subtrahert fra 5''
\end{itemize}
} \vsk \vsk

\info{Ei ny tolking av 0}{
Innleiingsvis i denne boka nemnde vi at 0 kan tolkast som ''ingenting''. Subtraksjon gir oss moglegheiten til å uttrykke 0 via andre tal. For eksempel er $ {7-7=0} $ og $ {19-19=0} $. I praktiske samanhengar vil 0 ofte innebere ei form for likevekt, for eksempel som at ei kraft og ei motkraft er like store.
} \vsk

\newpage
\subsection*{Subtraksjon på tallinja: Vandring mot venstre}
I \hrs{Addisjon}{seksjon} har vi sett at \sym{$ + $} (med positive tal) inneber at vi skal gå \\[2pt] \textsl{mot høgre} langs tallinja. Med \sym{$ - $} gjer vi omvend, vi går \textsl{mot venstre}\footnote{I figurar med tallinjer vil raudfarga piler indikere at ein startar ved pilspissen og vandrar til andre enden.}: \regv

\eks[1]{ \vs
\[ 6-4=2 \]
\fig{mint}
}
\eks[2]{ \vs
	\[ 12-7=5 \]
	\fig{mint2}
}
\info{Merk}{
Med det første kan det kanskje verke litt rart at ein i \textsl{Eksempel 1} og \textsl{2} over skal gå i motsatt veg av retninga pila peiker i, men spesielt i \hrs{Negtal}{Kapittel} vil det lønne seg å tenke slik.
}


\end{document}