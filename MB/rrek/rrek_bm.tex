\documentclass[english,hidelinks,pdftex, 11 pt, class=report,crop=false]{standalone}
\usepackage[T1]{fontenc}
\usepackage[utf8]{luainputenc}
\usepackage{lmodern} % load a font with all the characters
\usepackage{geometry}
\geometry{verbose,a4paper, inner=0cm, outer=0 cm, bmargin=2cm, tmargin=1cm}
%\textwidth=12cm
\setlength{\parindent}{0bp}
\usepackage{import}
\usepackage[subpreambles=false]{standalone}
\usepackage{amsmath}
\usepackage{amssymb}
\usepackage{esint}
\usepackage{babel}
\usepackage{tabu}
\usepackage[dvipsnames, table]{xcolor}
\usepackage{cancel}
\makeatother
\makeatletter
\usepackage{datetime2}
\usepackage{titlesec}
\usepackage[many]{tcolorbox}

% Eheter
\newcommand{\enh}[1]{\,\textrm{#1}}
%referances
\newcommand{\net}[2]{{\color{blue}\href{#1}{#2}}}

%Spaces
\newcommand{\vsk}{\\[12pt]}
\newcommand{\vs}{\vspace{-12pt}}

% Tabell for opplegg

\newcommand{\ovlist}[1]{
\vspace{-16pt}
\begin{itemize}
	#1
\end{itemize}
}

% Chapters and sections
\titleformat{\section}[block]{\bfseries}{\hspace{3cm}\thesection}{5pt}{}
\titleformat{\subsection}[block]{\bfseries}{\hspace{3cm}\thesection}{5pt}{}
\newcommand{\sectionbreak}{\clearpage} % New page on each section
 

\newlength{\mywidth}
\setlength{\mywidth}{14cm}

\newcommand{\cont}[1]{
\begin{tcolorbox}[center, boxrule=0.0 mm, width=\mywidth,arc=0mm,enhanced jigsaw,,colback=white,breakable]
#1	
\end{tcolorbox}
}

\newcommand{\info}[5]{
\begin{tcolorbox}[center, boxrule=0.1 mm, width=\mywidth,arc=0mm,enhanced jigsaw,breakable,colback=yellow!5]	
	
	\footnotesize
	\textbf{Øvingsområde}\\[5pt] #1 
	
	\textbf{Utstyr}\\ #2  \\
	
	\begin{tabular}{@{} p{4cm} p{4cm} l} 
		\textbf{Tid} & \textbf{Elevinndeling} & \textbf{Læringsarena} \\
		#3  & #4 & #5
	\end{tabular} 
\end{tcolorbox}	
}

\newcommand{\gjen}[1]{\begin{tcolorbox}[center,boxrule=0.1 mm, width=\mywidth,arc=0mm,colback=blue!3] {\large \textbf{Gjennomføring} \vspace{5 pt}}\newline #1  \end{tcolorbox}\vspace{-5pt}}
\newcommand{\eks}[1]{\begin{tcolorbox}[center,boxrule=0.1 mm, width=\mywidth,arc=0mm,colback=green!3] {\large \textbf{Eksempel} \vspace{5 pt}}\newline #1  \end{tcolorbox}\vspace{-5pt}}

\newcounter{opl}
%\numberwithin{opl}{article}


\newcommand{\opl}[1]{
\newpage
{\refstepcounter{opl} %\phantomsection 
\large \textbf{\theopl \;#1} \vsk}
}

% Headlines
\newcommand{\fork}{\textbf{Forkunnskapar}\\}
\newcommand{\forb}{\textbf{Forberedelsar}\\}
\newcommand{\opgvr}{\textbf{Oppgaver}}



%colors
\newcommand{\colr}[1]{{\color{red} #1}}
\newcommand{\colb}[1]{{\color{blue} #1}}
\newcommand{\colo}[1]{{\color{orange} #1}}
\newcommand{\colc}[1]{{\color{cyan} #1}}
\definecolor{projectgreen}{cmyk}{100,0,100,0}
\newcommand{\colg}[1]{{\color{projectgreen} #1}}

% Lister med bokstavar
\usepackage[inline]{enumitem}
% Opg
\newcommand{\abc}[1]{
	\begin{enumerate}[label=\alph*),leftmargin=18pt]
		#1
	\end{enumerate}
}

\usepackage[]{hyperref}

\newcommand{\note}{Merk}
\newcommand{\notesm}[1]{{\footnotesize \textsl{\note:} #1}}
\newcommand{\ekstitle}{Eksempel }
\newcommand{\sprtitle}{Språkboksen}
\newcommand{\expl}{forklaring}
\newcommand{\pyt}{Pytagoras' setning}
\newcommand\sv{\vsk \textbf{Svar} \vspace{4 pt}\\}

%references
\newcommand{\reftab}[1]{\hrs{#1}{tabell}}
\newcommand{\rref}[1]{\hrs{#1}{regel}}
\newcommand{\dref}[1]{\hrs{#1}{definisjon}}
\newcommand{\refkap}[1]{\hrs{#1}{kapittel}}
\newcommand{\refsec}[1]{\hrs{#1}{seksjon}}
\newcommand{\refdsec}[1]{\hrs{#1}{delseksjon}}
\newcommand{\refved}[1]{\hrs{#1}{vedlegg}}
\newcommand{\eksref}[1]{\textsl{#1}}
\newcommand\fref[2][]{\hyperref[#2]{\textsl{figur \ref*{#2}#1}}}
\newcommand{\refop}[1]{{\color{blue}Oppgave \ref{#1}}}
\newcommand{\refops}[1]{{\color{blue}oppgave \ref{#1}}}


%Algebra
\newcommand{\kvadset}{Kvadratsetningene}
\newcommand{\aenato}{Sum-produkt-metoden}

% Geometry
\newcommand{\hlikb}{Midtnormalen i en likebeint trekant}
\newcommand{\arealsetn}{Arealsetningen}
\newcommand{\trkmedian}{Median}
\newcommand{\midtrk}{Midtnormal (i trekant)}
\newcommand{\innskrsirk}{Innskrevet sirkel}
\newcommand{\cossetn}{Cosinussetningen}
\newcommand{\perfvink}{Sentral- og periferivinkel}
\newcommand{\tang}{Tangent}

% Derivative
\newcommand{\derel}{Den deriverte av elementære funksjoner}
\newcommand{\divder}{Divisjonsregelen}
\newcommand{\kjernereg}{Kjerneregelen}
\newcommand{\prodregder}{Produktregelen}
\newcommand{\lhop}{L'Hopitals regel}

% Funksjonsdrofting
\newcommand{\monder}{Monotoniegenskaper og den deriverte}
\newcommand{\fderekstr}{$ \bm{f'=0} $ for lokale ektstremalpunkt}
\newcommand{\andredertest}{Andrederiverttesten}

% Vectors
\newcommand{\detar}{Arealformler med determinanter}
\newcommand{\avstpunktlin}{Avstand mellom punkt og linje}

%Appendix
\newcommand{\rolle}{Rolles teorem}
\newcommand{\meanval}{Middelverdisetningen}

% Solutions manual
\newcommand{\selos}{Se løsningsforslag.}

\begin{document}
%\newpage
\section{\rrek}
\subsection*{Prioriteringen av regneartene}
Se på følgende regnestykke:
\[ 2+3\cdot4 \]
Et slikt regnestykke \textsl{kunne} man tolket på to måter:
\begin{enumerate}[label=(\roman*)]
	\item ''2 pluss 3 er 5. 5 ganget med 4 er 20. Svaret er 20.''
	\item ''3 ganget med 4 er 12. 2 pluss 12 er 14. Svaret er 14.''
\end{enumerate}
Men svarene blir ikke like! Det er altså behov for å ha noen regler om hva vi skal regne ut først. Den ene regelen er at vi må regne ut ganging eller deling \textsl{før} vi legger sammen eller trekker ifra, dette betyr at \regv
\st{ \vs
\alg{
	2+3\cdot 4&=\text{''Regn ut }3\cdot4\text{, og legg sammen med 2''}  \\
	&= 2+12 \\
	&= 14
}
}
Men hva om vi ønsket å legge sammen $ 2 $ og $ 3 $ først, og så gange summen med 4? Å fortelle at noe skal regnes ut først gjør vi ved hjelp av parenteser: \regv
\st{\vs
\alg{
(2+3)\cdot4&=\text{''Legg sammen 2 og 3, og gang med 4 etterpå''} \\
&= 5\cdot 4 \\
&= 20
}
}\regv

\reg[Regnerekkefølge \label{rrek}]{ \vspace{-5pt}
\begin{enumerate}
	\item Uttrykk med parentes
	\item Multiplikasjon eller divisjon
	\item Addisjon eller subtraksjon
\end{enumerate}
} 
\newpage
\eks[1]{
Regn ut
\[ 23-(3+9)+4\cdot 7 \]
\sv \vs \vs
\algv{
&& 23-(3+9)+4\cdot 7&=23-12+4\cdot7 &&\text{Parentes} \\
&&&=23-12+28 &&\text{Ganging} \\
&&&=39 &&\text{Addisjon og subtraksjon}
}
}
\eks[2]{
	Regn ut
	\[ 18:(7-5)-3 \]
	\sv \vs \vs
	\algv{
		&& 18:(7-5)-3&=18:2-3 &&\text{Parentes} \\
		&&&=9-3 &&\text{Deling} \\
		&&&=6 &&\text{Addisjon og subtraksjon}
	}
}
\subsection*{Ganging med parentes}
Hvor mange ruter ser vi i figuren under?
\fig{gang}
To måter man kan tenke på er disse:
\begin{enumerate}[label=(\roman*)]
	\item Det er $ 2\cdot4 =8 $ lilla ruter og $ 3\cdot4=12 $ grønne ruter. Til sammen er det $ 8+12 =20 $ ruter. Dette kan vi skrive som
\[ 2\cdot 4 + 3\cdot 4 = 20  \]
	\item Det er $ 2+3=5 $ ruter bortover og 4 ruter oppover. Altså er det $ 5\cdot4 =20 $ ruter totalt. Dette kan vi skrive som
	\[ (2+3)\cdot 4 = 20 \]
\end{enumerate}
Av disse to utregningene har vi at
\[ (2+3)\cdot4 = 2\cdot 4+ 3\cdot4 \]
\reg[\gangpar \label{gangpar}]{
Når et parentesuttrykk er en faktor, kan vi gange de andre faktorene med hvert enkelt ledd i parentesuttrykket.	 
%\fig{gang1}
}
\eks[1]{
\vs
\[ ({\color{orange}4}+{\color{ForestGreen}7})\cdot {\color{blue}8}={\color{orange}4}\cdot{\color{blue}8}+{\color{ForestGreen}7}\cdot{\color{blue}8} \]	
}
\eks[2]{ \vsb \vs
\alg{
(10-7)\cdot2 &= 10\cdot 2-7\cdot2\\
&=20-14 \\
&=6
}	
\mer Her vil det selvsagt være raskere å regne slik:
\[ (10-7)\cdot 2=3\cdot 2 =6 \]
}
\eks[2]{
Regn ut $ 12\cdot 3 $.

\sv
\vsb \vsb
\alg{
12\cdot 3&= (10+2)\cdot 3 \\
&=10\cdot 3 +2\cdot 3 \\
&=30 +6 \\
&=36
}	
}
\info{\note}{
Vi introduserte parenteser som en indikator på hva som skulle regnes ut først, men \rref{gangpar} gir en alternativ og likeverdig betydning av parenteser.
}
\newpage
\subsubsection{Å gange med 0}
Vi har tidligere sett at 0 kan skrives som en differanse mellom to tall, og dette kan vi nå utnytte til å finne produktet når vi ganger med 0. La oss se på regnestykket
\[ (2-2)\cdot3 \]
Av \rref{gangpar} har vi at
\alg{
	(2-2)\cdot3 &= 2\cdot3-2\cdot3\\&=6-6\\&=0
}
Siden $ 0=2-2 $, må dette bety at
\[ 0\cdot3=0 \]

\reg[Gonging med 0]{
	Viss 0 er en faktor, er produktet lik 0.
}
\eks[1]{ \vsb \vs
	\alg{
		7\cdot0&=0\vn
		0\cdot219 &=0
	}
}
\subsection*{Assosiative lover}
\reg[Assosiativ lov ved addisjon]{
Plasseringen av parenteser mellom ledd har ingen påvirkning på summen.
}
\eks[]{ \vsb \vs
\alg{
(2+3)+4&=5+3=9 \vn
2+(3+4)&=2+7=9
}
\fig{asso0}
}
\vsk \vsk

\reg[Assosiativ lov ved multiplikasjon]{
Plasseringen av parenteser mellom faktorer har ingen påvirkning på produktet.
}
\eks[]{ \vsb \vs
\alg{
(2\cdot3)\cdot4 &=6\cdot 4=24 \vn
2\cdot(3\cdot4)&=2\cdot 12 =24
}
\fig{asso1}
}  \vsk

I motsetning til addisjon og multiplikasjon, er hverken subtraksjon eller divisjon assosiative:
\alg{
(12-5)-4&=7-4=3 \vn
12-(5-4)&=12-1=11
}
\alg{
	(80:10):2&=8:2=4 \vn
	80:(10:2)&=80:5=16
}
Vi har sett at parentesene hjelper oss med å si noe om \textsl{prioriteringen} av regneartene, men det at subtraksjon og divisjon ikke er assosiative fører til at vi også må ha en regel for hvilken \textsl{retning} vi skal regne i. \regv

\reg[Retning på utregninger \label{rret}]{
Regnearter som ut ifra \rref{rrek} har lik prioritet, skal regnes fra venstre mot høyre.
}
\eks[1]{ \vsb \vsb
	\alg{
		12-5-4&=(12-5)-4 \\
		&=7-4\\
		&=3
	}
}
\eks[2]{ \vsb \vsb
	\alg{
		80:10:2&=(80:10):2 \\
		&=8:2 \\
		&=4
	}
}
\eks[3]{ \vsb \vs
	\alg{
		6: 3\cdot 4 &= (6:3)\cdot4\\ 
		&=2\cdot4 \\
		&= 8
	}
}

\section{\fak \label{Faktorisering}}
Når en heltalls dividend og en heltalls divisor resulterer i en heltalls \\kvotient, sier vi at dividenden er \outl{delelig} med divisoren. For eksempel er $ 6 $ delelig med $ 3 $ fordi $ {6:3=2} $, og $ 40 $ er delelig med $ 10 $ fordi $ {40:10=4} $. Begrepet delelig er med på å definere \outl{primtall}\index{primtall}:\regv

\regdef[Primtall]{
	Et naturlig tall som er større enn 1, og som bare er delelig med seg selv og 1, er et primtall.
}
\eks[]{
	De fem første primtallene er $ 2, 3, 5 , 7  $ og $ 11 $.
} \vsk \vsk

\regdef[Faktorisering]{\index{faktorisering}
	Faktorisering innebærer å skrive et tall som et produkt av andre tall.
}
\eks{
	Faktoriser 24 på tre forskjellige måter.
	
	\sv  \vsb \vs 
	\alg{
		24&=2\cdot 12 \br
		24&=3\cdot 8 \br
		24&=2\cdot 3 \cdot 4
	}
} 
\spr{
	Da 12 er delelig med 4, sier vi at 4 er en faktor i 12.
}
\newpage
\reg[Primtallsfaktorisering]{\index{primtallsfaktorisering}
	Faktorisering med bare primtall som faktorer kalles \\\outl{primtallsfaktorisering}.
}
\eks[]{
	Skriv 12 på primtallsfaktorisert form.
	
	\sv \vs \vs
	\[ 12= 2 \cdot2\cdot3 \]
}
\info{Primtallene mellom 1-100}{
	\label{primtalfig}
	\fig{primn100}
}

\end{document}

