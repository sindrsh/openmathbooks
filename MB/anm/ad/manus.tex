\documentclass[english,hidelinks,pdftex, 11 pt, class=report,crop=false]{standalone}
\usepackage[T1]{fontenc}
\usepackage[utf8]{luainputenc}
\usepackage{geometry}
\geometry{verbose,paperwidth=16.4 cm, paperheight=29cm, inner=2.05cm, outer=2.05 cm, bmargin=2cm, tmargin=1.8cm}
\usepackage{amsmath}
\usepackage{amssymb}
\usepackage{esint}
\usepackage{babel}
\usepackage{tabu}


\begin{document}
\begin{itemize}
\item La oss no rekne ut $ 279+46 $.
\item Det første me gjer da er å skrive 279 og 46 under kvarandre.
\item Så rekne me ut 9+6.
\item 9+6 e 15. Dermed he me 5 på einarplassen og 1 i mente.
\item Så går me over te å rekne ut 1+7+4.
\item 1+7+4 e 12. Dermed he me 2 på tiarplassen og 1 i mente.
\end{itemize}

\end{document}