\documentclass[english,hidelinks,pdftex, 11 pt, class=report,crop=false]{standalone}
\usepackage[T1]{fontenc}
\usepackage[utf8]{luainputenc}
\usepackage{geometry}
\geometry{verbose,paperwidth=16.4 cm, paperheight=29cm, inner=2.05cm, outer=2.05 cm, bmargin=2cm, tmargin=1.8cm}
\usepackage{amsmath}
\usepackage{amssymb}
\usepackage{esint}
\usepackage{babel}
\usepackage{tabu}


\begin{document}
\begin{itemize}
\item \textbf{Eksempel 1}\\
La oss no rekne ut $ 132-53 $. 
\item Det første me gjer, e å skrive 53. Det me skal gjer videre, e å legge te tal, heilt til me kjem fram te 132. Ka tall me legg te bestemme me sjøl, og da e det lurt å velge tall som gjer det lett for oss å rekne.
\item Det første talet me velg å legge te, e 7.\\ 3 og 7 e jo tiarvenna, så $ 53+7 $ e lik 60.
\item Det neste talet me legg te, e 40.\\
$ 60+40 $ e lik 100.
\item Så legg me te 32. \\
$ 100+32 $ e lik 132
\item No he me komme fram te 132, og da kan me summere alle tala me he lagt te.
\item $ 7+40+32 $ e lik $ 79 $, og det betyr at $ 132-53 $ e lik $ 79 $.



\end{itemize}
\end{document}