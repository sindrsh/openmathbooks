\documentclass[english,hidelinks,pdftex, 11 pt, class=report,crop=false]{standalone}
\usepackage[T1]{fontenc}
\usepackage[utf8]{luainputenc}
\usepackage{geometry}
\geometry{verbose,paperwidth=16.4 cm, paperheight=29cm, inner=2.05cm, outer=2.05 cm, bmargin=2cm, tmargin=1.8cm}
\usepackage{amsmath}
\usepackage{amssymb}
\usepackage{esint}
\usepackage{babel}
\usepackage{tabu}
\usepackage{lmodern}


\begin{document}
\begin{itemize}
	\item Likninge; å flytte ledd. Del 1.

	\item Eksempel 1
	
	Gitt likninga $ x-4=3 $. Sjøl om detta e ei likning mange vil sjå svaret på direkte, skal vi bruke ein metode som kjem til nytte når likninge blir meir kompliserte.
	
	$ = $-tegnet spelle ei utrulig stor rolle her. Tegnet fortell oss at det som står på venstre side he samme verdi som det som står på høgre side. Detta gjer at me kan utføre ein \textit{matematiske operasjonan} på ei side, så lenge me gjer det samme på den andre sida. \textit{Ein matematisk operasjon} e eit samleord for å bruke blant anna pluss, minus, ganging og deling.
	\item Svaret på ei likning he me når me he fått ein $ x $ til å stå aleine på ei side av likhetstegnet. For å få $ x $ til å stå aleine på venstre side, kan me legge til 4. Da må me også legge til 4 på høgre side.
	\item Sia $ -4+4=0 $, og $ 3+4 = 7 $, he me no $ x $ på venstre side og $ 7 $ på høgre side. Altså e svaret på likninga vår at $ x=7 $
	\item Sjøl om me no he funne svaret, skal me sjå litt tilbake på det me gjor. Heile poenget med å legge til 4, va å få $ x $ aleine på venstre side, og derfor va det egentlig litt unødvendig å skrive $ -4+4 $ på venstre side.
	\item Hvis vi tek vekk $ -4+4 $ fra venstre side kjem det til syne eit prinsipp som vi kan bruke i likninge: Vi kan flytte tall fra den eine sida til den andre, så lenge vi skifte fortegn på tallet. Her betyr det at $ -4 $ på venstre side blir til $ +4 $ når det flyttast over til høgre side.
\end{itemize}

\end{document}