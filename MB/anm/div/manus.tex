\documentclass[english,hidelinks,pdftex, 11 pt, class=report,crop=false]{standalone}
\usepackage[T1]{fontenc}
\usepackage[utf8]{luainputenc}
\usepackage{geometry}
\geometry{verbose,paperwidth=16.4 cm, paperheight=29cm, inner=2.05cm, outer=2.05 cm, bmargin=2cm, tmargin=1.8cm}
\usepackage{amsmath}
\usepackage{amssymb}
\usepackage{esint}
\usepackage{babel}
\usepackage{tabu}
\usepackage{lmodern}


\begin{document}
\begin{itemize}
	\item Deling
	\item Eksempel 1
	\item Me skal no rekne ut $ 68:4 $. Sia me dele med 4, skal me komme fram til 68 ved å gange tall med 4. Legg merke til at me sjøl velg ka tall me vil gange med.
	\item Me starte med å gange 4 med 10, som e 40. Så langt he me altså komme oss te 40. 
	\item Så gange me 4 med 7, som e 28. 
	\item Da he me komme te $40+28 $, som e 68, og det e talet me ønska å komme fram te.
	\item Da gjenstår det berre å legge sammen tale som me ganga med 4. Da he me $ 10+7 $, som e 17. Altså e $ 68:4 $ lik 17.
\end{itemize}
\begin{itemize}
	\item \textbf{Eksempel 2}
	\item Me ska no rekne ut $ 126:3 $. Me starte da med å gange 3 med 30, som e 90. Så langt he me altså 90.
	\item Så gange me 3 med 10, som e 30. Da he me $ 90+30 $, som e 120
	\item Så gange me 3 med 2, som e 6. Da he me $ 120+6 $, som e 126. 
	\item Da e me altså framme, og det gjenstår å legge sammen tala me he ganga med. Da fer me at $ 30+10+2 $, som e 42.
	\item 
	\item Som nevnt så e det, med metoden me ser på no, fritt for oss å velge ka tal me vil gange med for å komme fram te svar. La oss derfor sjå kordan me kunne løst reknestykket $ 126:3 $ på ein kortare måte.
	\item Me starte da med å gange 3 med 40, som e 120. Da he me komme te 120.
	\item Så gange me 3 med 2, som e 6. Da he me komme te 126, altså e me framme. Og 40+2 e 42. Så $ 126:3 $ e 42.
\end{itemize}
\end{document}