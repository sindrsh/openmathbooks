\documentclass[english,hidelinks,pdftex, 11 pt, class=report,crop=false]{standalone}
\usepackage[T1]{fontenc}
\usepackage[utf8]{luainputenc}
\usepackage{geometry}
\geometry{verbose,paperwidth=16.4 cm, paperheight=29cm, inner=2.05cm, outer=2.05 cm, bmargin=2cm, tmargin=1.8cm}
\usepackage{amsmath}
\usepackage{amssymb}
\usepackage{esint}
\usepackage{babel}
\usepackage{tabu}
\usepackage{lmodern}


\begin{document}
\begin{itemize}
	\item Likninge; å flytte ledd. Del 2.
	Me he tidligare sett på prinsippet om tall som bytte side i likninge. Her skal vi sjå meir på kordan vi bruke detta prinsippet til å løse likninge.
	\item Eksempel 1. Gitt likninga $ 4x-8=3x $. Formålet vårt no e altså å få ein $ x $ til å stå aleine på ei side av likhetstegnet.\\
	
	\item Me starte med å skifte side på $ 3x $. $ 3x $ på høgre side blir $ -3x $ på venstre side. Så skifte me side på $ -8 $. $ -8 $ på venstre side blir til $ 8 $ på høgre side.
	\item På venstre side he me no $ 4x-3x $, som e éin $ x $. Legg merke til at éin $ x $ skriv vi uten eit 1-tall foran. På høgre side he me $ 8 $. ALtså e svaret på likninga at $ x=8 $.
	
	\item Eksempel 2. Gitt likninga $ 8x-4=7x+6 $.
	\item Me skifte side på $ 7x $. $ 7x $ på høgre side blir $ -7x $ på venstre side. Me skifte side på $ -4 $. $ -4 $ på venstre side blir $ 4 $ på høgre side.
	\item På venstre side he me no $ 8x-7x $, som e $ x $. På høgre side he me 6+4, som e 10. Altså e svaret at $ x=10 $.

\end{itemize}

\end{document}