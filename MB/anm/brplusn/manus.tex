\documentclass[english,hidelinks,pdftex, 11 pt, class=report,crop=false]{standalone}
\usepackage[T1]{fontenc}
\usepackage[utf8]{luainputenc}
\usepackage{geometry}
\geometry{verbose,paperwidth=16.4 cm, paperheight=29cm, inner=2.05cm, outer=2.05 cm, bmargin=2cm, tmargin=1.8cm}
\usepackage{amsmath}
\usepackage{amssymb}
\usepackage{esint}
\usepackage{babel}
\usepackage{tabu}


\begin{document}
\begin{itemize}
\item La oss no rekne ut $ \frac{3}{5}-\frac{1}{2} $. Skal me legge sammen dessa brøkan, må me utvide dei slik at dei he samme nevnar.
\item Begge brøkan kan utvidast til å ha ti i nevnar. \\$ \frac{3}{5} $ e det samme som $ \frac{6}{10} $ og $ \frac{1}{2} $ e det samme som $ \frac{5}{10} $
\item Og $ \frac{6}{10}-\frac{5}{10} $ e det samme som $ \frac{1}{10} $
\item La oss no sjå ka me gjor reint matematisk.
\item Me starta me altså med å skulle rekne ut $ \frac{3}{5}-\frac{1}{2} $.
\item Så utvida me brøkan til å ha ti i nevnar. For å få det, må $ \frac{3}{5} $ utvidast med 2 og $ \frac{1}{2} $ utvidast med 5.
\item $ \frac{3}{5} $ utvida med 2 e $\frac{6}{10}$, og $ \frac{1}{2} $ utvida med 5 e $ \frac{5}{10} $.
\item Og $ \frac{6}{10}-\frac{5}{10} $ e $ \frac{1}{10} $
\end{itemize}

\end{document}