\documentclass[english,hidelinks,pdftex, 11 pt, class=report,crop=false]{standalone}
\usepackage[T1]{fontenc}
\usepackage[utf8]{luainputenc}
\usepackage{geometry}
\geometry{verbose,paperwidth=16.4 cm, paperheight=29cm, inner=2.05cm, outer=2.05 cm, bmargin=2cm, tmargin=1.8cm}
\usepackage{amsmath}
\usepackage{amssymb}
\usepackage{esint}
\usepackage{babel}
\usepackage{tabu}


\begin{document}
\textbf{Eksempel 3}
\begin{itemize}
\item La oss rekne ut $ 1218:6 $.
\item Det første me gjer, er å skrive opp 6-gangen.
\item Videre legg me merke te at 1 e mindre enn 6. Me tek derfor med oss også neste siffer, så no ønske e å komme så nærme som mulig 12 i 6-gangen.
\item Det nærmaste me kjem 12 i 6-gangen e nettopp 12, som tilsvare $ 2\cdot 6 $. Derfor skriv me 12 under 12 og 2 bak $ = $-tegnet.
\item Så rekne me ut 12-12, som e 0.
\item Videre trekke me ned 1 tallet. I den lille gangetabellen for 6 he me ingen tall som e mindre enn 1. Men me veit at $ 0\cdot6=0 $. Derfor skriv me 0 under 1 og 0 bak $ = $-teiknet.
\item Så rekne me ut 1-0, som e 1.
\item Videre trekk me ned 8-tallet. No ønske me å finne tallet i 6-gangen som e nærmast 18. Og det e 18, som tilsvare $ 3\cdot6 $. Derfor skriv me 18 under 18 og 3 bak $ = $-tegnet.
\end{itemize}

\end{document}