\documentclass[english,hidelinks,pdftex, 11 pt, class=report,crop=false]{standalone}
\usepackage[T1]{fontenc}
\usepackage[utf8]{luainputenc}
\usepackage{lmodern} % load a font with all the characters
\usepackage{geometry}
\geometry{verbose,a4paper, inner=0cm, outer=0 cm, bmargin=2cm, tmargin=1cm}
%\textwidth=12cm
\setlength{\parindent}{0bp}
\usepackage{import}
\usepackage[subpreambles=false]{standalone}
\usepackage{amsmath}
\usepackage{amssymb}
\usepackage{esint}
\usepackage{babel}
\usepackage{tabu}
\usepackage[dvipsnames, table]{xcolor}
\usepackage{cancel}
\makeatother
\makeatletter
\usepackage{datetime2}
\usepackage{titlesec}
\usepackage[many]{tcolorbox}

% Eheter
\newcommand{\enh}[1]{\,\textrm{#1}}
%referances
\newcommand{\net}[2]{{\color{blue}\href{#1}{#2}}}

%Spaces
\newcommand{\vsk}{\\[12pt]}
\newcommand{\vs}{\vspace{-12pt}}

% Tabell for opplegg

\newcommand{\ovlist}[1]{
\vspace{-16pt}
\begin{itemize}
	#1
\end{itemize}
}

% Chapters and sections
\titleformat{\section}[block]{\bfseries}{\hspace{3cm}\thesection}{5pt}{}
\titleformat{\subsection}[block]{\bfseries}{\hspace{3cm}\thesection}{5pt}{}
\newcommand{\sectionbreak}{\clearpage} % New page on each section
 

\newlength{\mywidth}
\setlength{\mywidth}{14cm}

\newcommand{\cont}[1]{
\begin{tcolorbox}[center, boxrule=0.0 mm, width=\mywidth,arc=0mm,enhanced jigsaw,,colback=white,breakable]
#1	
\end{tcolorbox}
}

\newcommand{\info}[5]{
\begin{tcolorbox}[center, boxrule=0.1 mm, width=\mywidth,arc=0mm,enhanced jigsaw,breakable,colback=yellow!5]	
	
	\footnotesize
	\textbf{Øvingsområde}\\[5pt] #1 
	
	\textbf{Utstyr}\\ #2  \\
	
	\begin{tabular}{@{} p{4cm} p{4cm} l} 
		\textbf{Tid} & \textbf{Elevinndeling} & \textbf{Læringsarena} \\
		#3  & #4 & #5
	\end{tabular} 
\end{tcolorbox}	
}

\newcommand{\gjen}[1]{\begin{tcolorbox}[center,boxrule=0.1 mm, width=\mywidth,arc=0mm,colback=blue!3] {\large \textbf{Gjennomføring} \vspace{5 pt}}\newline #1  \end{tcolorbox}\vspace{-5pt}}
\newcommand{\eks}[1]{\begin{tcolorbox}[center,boxrule=0.1 mm, width=\mywidth,arc=0mm,colback=green!3] {\large \textbf{Eksempel} \vspace{5 pt}}\newline #1  \end{tcolorbox}\vspace{-5pt}}

\newcounter{opl}
%\numberwithin{opl}{article}


\newcommand{\opl}[1]{
\newpage
{\refstepcounter{opl} %\phantomsection 
\large \textbf{\theopl \;#1} \vsk}
}

% Headlines
\newcommand{\fork}{\textbf{Forkunnskapar}\\}
\newcommand{\forb}{\textbf{Forberedelsar}\\}
\newcommand{\opgvr}{\textbf{Oppgaver}}



%colors
\newcommand{\colr}[1]{{\color{red} #1}}
\newcommand{\colb}[1]{{\color{blue} #1}}
\newcommand{\colo}[1]{{\color{orange} #1}}
\newcommand{\colc}[1]{{\color{cyan} #1}}
\definecolor{projectgreen}{cmyk}{100,0,100,0}
\newcommand{\colg}[1]{{\color{projectgreen} #1}}

% Lister med bokstavar
\usepackage[inline]{enumitem}
% Opg
\newcommand{\abc}[1]{
	\begin{enumerate}[label=\alph*),leftmargin=18pt]
		#1
	\end{enumerate}
}

\usepackage[]{hyperref}

\newcommand{\note}{Merk}
\newcommand{\notesm}[1]{{\footnotesize \textsl{\note:} #1}}
\newcommand{\ekstitle}{Eksempel }
\newcommand{\sprtitle}{Språkboksen}
\newcommand{\expl}{forklaring}
\newcommand{\pyt}{Pytagoras' setning}
\newcommand\sv{\vsk \textbf{Svar} \vspace{4 pt}\\}

%references
\newcommand{\reftab}[1]{\hrs{#1}{tabell}}
\newcommand{\rref}[1]{\hrs{#1}{regel}}
\newcommand{\dref}[1]{\hrs{#1}{definisjon}}
\newcommand{\refkap}[1]{\hrs{#1}{kapittel}}
\newcommand{\refsec}[1]{\hrs{#1}{seksjon}}
\newcommand{\refdsec}[1]{\hrs{#1}{delseksjon}}
\newcommand{\refved}[1]{\hrs{#1}{vedlegg}}
\newcommand{\eksref}[1]{\textsl{#1}}
\newcommand\fref[2][]{\hyperref[#2]{\textsl{figur \ref*{#2}#1}}}
\newcommand{\refop}[1]{{\color{blue}Oppgave \ref{#1}}}
\newcommand{\refops}[1]{{\color{blue}oppgave \ref{#1}}}


%Algebra
\newcommand{\kvadset}{Kvadratsetningene}
\newcommand{\aenato}{Sum-produkt-metoden}

% Geometry
\newcommand{\hlikb}{Midtnormalen i en likebeint trekant}
\newcommand{\arealsetn}{Arealsetningen}
\newcommand{\trkmedian}{Median}
\newcommand{\midtrk}{Midtnormal (i trekant)}
\newcommand{\innskrsirk}{Innskrevet sirkel}
\newcommand{\cossetn}{Cosinussetningen}
\newcommand{\perfvink}{Sentral- og periferivinkel}
\newcommand{\tang}{Tangent}

% Derivative
\newcommand{\derel}{Den deriverte av elementære funksjoner}
\newcommand{\divder}{Divisjonsregelen}
\newcommand{\kjernereg}{Kjerneregelen}
\newcommand{\prodregder}{Produktregelen}
\newcommand{\lhop}{L'Hopitals regel}

% Funksjonsdrofting
\newcommand{\monder}{Monotoniegenskaper og den deriverte}
\newcommand{\fderekstr}{$ \bm{f'=0} $ for lokale ektstremalpunkt}
\newcommand{\andredertest}{Andrederiverttesten}

% Vectors
\newcommand{\detar}{Arealformler med determinanter}
\newcommand{\avstpunktlin}{Avstand mellom punkt og linje}

%Appendix
\newcommand{\rolle}{Rolles teorem}
\newcommand{\meanval}{Middelverdisetningen}

% Solutions manual
\newcommand{\selos}{Se løsningsforslag.}
\begin{document}

\begin{picture}(100,80)
\put(100,0){\begin{minipage}[l]{0.8\columnwidth}
	\textit{ ''Wahrlich es ist nicht das Wissen, sondern das Lernen, \\ nicht das Besitzen, sondern das Erwerben, \\ nicht das Da-Seyn, sondern das Hinkommen, \\ was den grössten Genuss gewährt'' }
	\vsk  
	
	\textit{ ''Det er ikke å vite, men å lære, \\ ikke å eie, men å erverve,  \\ ikke å være til stede, men å komme dit, \\ som gir den største gleden.''}
	\vsk
	
	{\hfill --- Carl Friedrich Gauss}
	\end{minipage}}
\end{picture}
\vfill       

\

\lic
\begin{center}
	\DTMsetdatestyle{ddmmyyyy}	
	\DTMsetup{datesep={.},showseconds=false}
	\Today
\end{center}	
\newpage	
\section*{Forord}
Matematikk har et enormt omfang av forgreninger og anvendelser, men det aller meste bygger på en overkommeleg mengde med grunn-\\prinsipper. Det er disse jeg ønsker å presentere i denne boka. Et prinsipp i oppsummert form har jeg valgt å kalle en \textit{definisjon} eller en \textit{regel}. Regler/definisjoner finner du i blå tekstbokser, som oftest etterfulgt av eksempler på bruken av dem. Ett av hovudmålene til denne boka er å gi leseren en forståelse av hvorfor reglene er som de er. I kapittel 1\,-\,5 vil du finne forklaringer\footnote{Å \textsl{forklare} reglene i steden for å \textsl{bevise} dem er et bevisst valg. Et bevis stiller sterke matematiske krav som ofte må defineres både på forhand og underveis i en utledning av en regel, noe som kan føre til at forståelsen av hovedpoenget drukner i smådetaljer. Mange av forklaringene vil likevel være gyldige som bevis.} i forkant av hver regel, mens i kapittel 6 finner du forklaringer enten i forkant av eller direkte etter en regel (og eventuelle eksempel). Fra og med kapittel 7 er noen forklaringer lagt til den avsluttende seksjonen \textsl{Forklaringer}. Dette indikerer at de kan vere noe krevende å forstå og/eller at regelen er så intuitiv at mange vil oppleve det som overflødig å få den forklart. \vsk

\textbf{Boka si oppbygging}\\
Boka er delt inn i en \textsl{Del I} og en \textsl{Del II}. \textsl{Del I} handler i stor grad om å bygge en grunnleggende forståelse av tallene våre, og hvordan vi regner med dem. \textsl{Del II} introduserer konseptet algebra og de nært beslektede temaene potenser, likninger og funskjoner. I tillegg har både \textsl{Del I} og \textsl{Del II} avsluttende kapittel som handler om geometri. \vsk

\textbf{Kommende oppdateringer}\\
Det er tenkt at hvert kapittel skal avsluttes med en oppgavedel. Dette vil være på plass innen 23. august.
\vfill
\newpage
\st{
Kjære leser. \vsk

Denne boka er i utgangspunktet gratis å bruke, men jeg håper du forstår hvor mye tid og ressurser jeg har brukt på å lage den. Hvis du ender opp med å like boka, håper jeg derfor du kan donere 50\,kr via Vipps til 90559730 eller via \net{https://www.paypal.com/donate?hosted_button_id=M5PNVC6J7PBYY}{PayPal}. Vær vennlig å markere donasjonen med ''Mattebok'' ved bruk av Vipps. Pengene vil bli brukt til å fortsette arbeidet med å lage lærebøker som er med på å gjøre matematikk tilgjengeleg for alle.
På forhand takk! \vsk

Boka blir oppdatert så snart som råd når skrivefeil og lignende blir oppdaget, jeg vil derfor råde alle til å laste ned en ny versjon i ny og ne ved å følge \net{https://drive.google.com/file/d/1WiS51PH0V7FKyO-XZSedae_IfhTOfCaH/view?usp=sharing}{denne linken}.\vsk

Nynorskversjonen av boka finner du \net{https://github.com/sindrsh/FirstPrinciplesOfMath/blob/master/MB.pdf}{her}.\vsk

For spørsmål, ta kontakt på mail: \tt{sindre.heggen@gmail.com} 
}

\subsection*{Takk til}
Anne Jordal Myrset \os
Charlotte Merete Dahl\vsk

For mange gode innspill og kommentarer.
\end{document}

