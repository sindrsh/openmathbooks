\documentclass[english,hidelinks,pdftex, 11 pt, class=report,crop=false]{standalone}
\usepackage[T1]{fontenc}
\usepackage[utf8]{luainputenc}
\usepackage{lmodern} % load a font with all the characters
\usepackage{geometry}
\geometry{verbose,a4paper, inner=0cm, outer=0 cm, bmargin=2cm, tmargin=1cm}
%\textwidth=12cm
\setlength{\parindent}{0bp}
\usepackage{import}
\usepackage[subpreambles=false]{standalone}
\usepackage{amsmath}
\usepackage{amssymb}
\usepackage{esint}
\usepackage{babel}
\usepackage{tabu}
\usepackage[dvipsnames, table]{xcolor}
\usepackage{cancel}
\makeatother
\makeatletter
\usepackage{datetime2}
\usepackage{titlesec}
\usepackage[many]{tcolorbox}

% Eheter
\newcommand{\enh}[1]{\,\textrm{#1}}
%referances
\newcommand{\net}[2]{{\color{blue}\href{#1}{#2}}}

%Spaces
\newcommand{\vsk}{\\[12pt]}
\newcommand{\vs}{\vspace{-12pt}}

% Tabell for opplegg

\newcommand{\ovlist}[1]{
\vspace{-16pt}
\begin{itemize}
	#1
\end{itemize}
}

% Chapters and sections
\titleformat{\section}[block]{\bfseries}{\hspace{3cm}\thesection}{5pt}{}
\titleformat{\subsection}[block]{\bfseries}{\hspace{3cm}\thesection}{5pt}{}
\newcommand{\sectionbreak}{\clearpage} % New page on each section
 

\newlength{\mywidth}
\setlength{\mywidth}{14cm}

\newcommand{\cont}[1]{
\begin{tcolorbox}[center, boxrule=0.0 mm, width=\mywidth,arc=0mm,enhanced jigsaw,,colback=white,breakable]
#1	
\end{tcolorbox}
}

\newcommand{\info}[5]{
\begin{tcolorbox}[center, boxrule=0.1 mm, width=\mywidth,arc=0mm,enhanced jigsaw,breakable,colback=yellow!5]	
	
	\footnotesize
	\textbf{Øvingsområde}\\[5pt] #1 
	
	\textbf{Utstyr}\\ #2  \\
	
	\begin{tabular}{@{} p{4cm} p{4cm} l} 
		\textbf{Tid} & \textbf{Elevinndeling} & \textbf{Læringsarena} \\
		#3  & #4 & #5
	\end{tabular} 
\end{tcolorbox}	
}

\newcommand{\gjen}[1]{\begin{tcolorbox}[center,boxrule=0.1 mm, width=\mywidth,arc=0mm,colback=blue!3] {\large \textbf{Gjennomføring} \vspace{5 pt}}\newline #1  \end{tcolorbox}\vspace{-5pt}}
\newcommand{\eks}[1]{\begin{tcolorbox}[center,boxrule=0.1 mm, width=\mywidth,arc=0mm,colback=green!3] {\large \textbf{Eksempel} \vspace{5 pt}}\newline #1  \end{tcolorbox}\vspace{-5pt}}

\newcounter{opl}
%\numberwithin{opl}{article}


\newcommand{\opl}[1]{
\newpage
{\refstepcounter{opl} %\phantomsection 
\large \textbf{\theopl \;#1} \vsk}
}

% Headlines
\newcommand{\fork}{\textbf{Forkunnskapar}\\}
\newcommand{\forb}{\textbf{Forberedelsar}\\}
\newcommand{\opgvr}{\textbf{Oppgaver}}



%colors
\newcommand{\colr}[1]{{\color{red} #1}}
\newcommand{\colb}[1]{{\color{blue} #1}}
\newcommand{\colo}[1]{{\color{orange} #1}}
\newcommand{\colc}[1]{{\color{cyan} #1}}
\definecolor{projectgreen}{cmyk}{100,0,100,0}
\newcommand{\colg}[1]{{\color{projectgreen} #1}}

% Lister med bokstavar
\usepackage[inline]{enumitem}
% Opg
\newcommand{\abc}[1]{
	\begin{enumerate}[label=\alph*),leftmargin=18pt]
		#1
	\end{enumerate}
}

\usepackage[]{hyperref}
\begin{document}

\section{\pot \label{Potensar}}
\fig{pot}
Ein potens består av eit \textit{grunntal}\index{grunntal} og ein \textit{eksponent}\index{eksponent}. For eksempel er $2^{3}$ ein potens med grunntal 2 og
eksponent 3. Ein positiv, heiltals eksponent seier kor mange eksemplar
av grunntalet som skal gongast saman, altså er
\[ 2^3 =2\cdot2\cdot2 \]

\reg[Potenstall]{
$ {a^n} $ er eit potenstal med grunntal $ a $ og eksponent $ n $. 
\vsk

Viss $ n $ er eit naturleg tal, vil $ a^n $ svare til $ n $ eksemplar av $ a $ multiplisert med kvarandre. \vsk

\textit{Merk: } $ a^1=a $
}
\eks[1]{\vs \vs
\algv{
5^3 &= 5\cdot5\cdot5 \\
&= 125
}
}
\eks[2]{\vs \vs
	\[ c^4 = c\cdot c \cdot c \cdot c \]
}
\eks[3]{ \vs \vs
\algv{
(-7)^2 &= (-7)\cdot(-7) \\
&= 49
}
}\vsk \vsk

\spr{
Vanlege måtar å seie $ 2^3 $ på er
\begin{itemize}
	\item ''2 i tredje''
	\item ''2 opphøgd i 3''
\end{itemize}
I programmeringsspråk brukast gjerne symbolet \sym{\^{}} eller symbola \sym{**} mellom grunntall og eksponent.
}
\newpage
\info{Merk}{
Dei komande sidene vil innehalde reglar for potensar med tilhøyrande forklaringar. Sjølv om det er ønskeleg at dei har ei så generell form som mogleg, har vi i forklaringane valgt å bruke eksempel der eksponentane ikkje er variablar. Å bruke variablar som eksponentar ville gitt mykje mindre leservenlege uttrykk, og vi vil påstå at dei generelle tilfella kjem godt til synes også ved å studere konkrete tilfelle. 
} \vsk \vsk

\reg[\potgang \label{potgang}]{
\begin{equation}
a^{m}\cdot a^{n}=a^{m+n}	
\end{equation}
}
\eks[1]{\vs \vs
\algv{3^{5}\cdot3^{2}&=3^{5+2}\\&=3^{7}}
}
\eks[2]{\vs \vs
\algv{
b^4\cdot b^{11}&= b^{3+11}\\
&=b^{14}
}
}
\eks[3]{ \vs \vs
\algv{
a^5\cdot a^{-7} &= a^{5+(-7)} \\
&=a^{5-7} \\
&= a^{-2} 
}
(Sjå \rref{potneg} for korleis potens med negativ eksponent kan tolkast.)	
} 
\newpage
\fork{\ref{potgang} \potgang}{
	La  oss sjå på tilfellet 
	\[ a^{2}\cdot a^{3} \]
	Vi har at
	\algv{
		a^{2} & =2\cdot2\vn
		a^{3} & =2\cdot2\cdot2
	}
	
	Med andre ord kan vi skrive 
	\begin{align*}
	a^{2}\cdot a^{3} & =\overbrace{a \cdot a}^{a^{2}}\cdot\overbrace{a\cdot a\cdot a}^{a^{3}}\\
	& =a^{5}
	\end{align*}
} \vsk \vsk

\reg[\potdivpot \label{potdivpot}]{\vs
\[ \frac{a^{m}}{a^{n}}=a^{m-n} \] }

\eks[1]{\vspace{-20 pt}
\[
\frac{3^{5}}{3^{2}}=3^{5-2}=3^{3}
\]
} 
\eks[2]{ \vs \vsb
	\alg{
		\frac{2^{4}\cdot a^{7}}{a^{6}\cdot2^{2}}&=2^{4-2}\cdot a^{7-6}\\
		&=2^{2}a \\
		&=4a
	}
}
\newpage
\fork{\ref{potdivpot} \potdivpot}{
	La
	oss undersøke brøken
	\[ \frac{a^{5}}{a^{2}} \]
	Vi skriv
	ut potensane i tellar og nemnar: 
	\begin{align*}
	\frac{a^{5}}{a^{2}} & =\frac{a\cdot a\cdot a\cdot a\cdot a}{a\cdot a}\br
	& =\frac{\bcancel{a}\cdot\bcancel{a}\cdot a\cdot a\cdot a}{\bcancel{a}\cdot\bcancel{a}}\\
	& =a\cdot a\cdot a\\
	& =a^{3}
	\end{align*}
	Dette kunne vi ha skrive som
	\begin{align*}
	\frac{a^{5}}{a^{2}} & =a^{5-2}\\
	& =a^{3}
	\end{align*}
} \vsk \vsk

\reg[\potanull \label{pota0}]{\vs \vs
\[
a^{0}=1
\]
}
\eks[1]{\vs \vs\[
1000^{0}=1
\]}
\eks[2]{\vs \vs\[
(-b)^{0}=1
\]}
\fork{\ref{pota0} \potanull}{
	Eit tal delt på seg sjølv er alltid lik 1, derfor er 
	\[
	\frac{a^{n}}{a^{n}}=1
	\]
	Av dette, og \rref{potdivpot}, har vi at
	\algv{
		1&=\frac{a^{n}}{a^{n}}
		\\& =a^{n-n}\\
		& =a^{0}
	}
} \vsk \vsk

\reg[\potneg \label{potneg}]{
	\[ a^{-n}=\frac{1}{a^n} \]
}
\eks[1]{ \vs \vs
	\alg{
		a^{-8}&=\frac{1}{a^8}  
	}	
}
\eks[2]{ \vs \vs
\alg{
(-4)^{-3}&=\frac{1}{(-4)^3} 
=-\frac{1}{64}
}
}
\fork{\ref{potneg} \potneg}{
	Av \rref{pota0} har vi at $ a^0=1 $. Altså er
	\alg{
		\frac{1}{a^n}=\frac{a^0}{a^n}
	}
	Av \rref{potdivpot}  er
	\algv{
		\frac{a^0}{a^n}&=a^{0-n} \\
		&=a^{-n}
	}
} \vsk \vsk



\reg[\potbr \label{potbr}]{\vs
\begin{equation}\label{pbrg}
\left(\frac{a}{b}\right)^{m}=\frac{a^{m}}{b^{m}}
\end{equation}} 
\eks[1]{ \vs \vs
\alg{
\left(\frac{3}{4}\right)^2=\frac{3^2}{4^2} 
=\frac{9}{16}
}
}
\eks[2]{ \vs \vs
	\alg{
		\left(\frac{a}{7}\right)^3=\frac{a^3}{7^3} 
		=\frac{a^3}{343}
	}
}
\newpage
\fork{\ref{potbr} \potbr}{
	La  oss studere
	\[ \left(\frac{a}{b}\right)^3 \]
	Vi har at
	\begin{align*}
	\left(\frac{a}{b}\right)^3 	&=\frac{a}{b}\cdot \frac{a}{b}\cdot \frac{a}{b}\br
	& =\frac{a\cdot a\cdot a}{b\cdot b\cdot b}\br
	& =\frac{a^{3}}{b^{3}}
	\end{align*}
}\vsk \vsk

\reg[\faktgr \label{faktgr}]{
\begin{equation}\label{key}
\left(ab\right)^{m}=a^{m}b^{m}
\end{equation}
}
\eks[1]{ \vs \vs \vs
\alg{
(3a)^5&=3^5a^5 \\
&=243a^5 
}	
}
\eks[2]{\vs\vs
\[
(ab)^{4}=a^{4}b^{4}
\]
}
\fork{\ref{faktgr} \faktgr}{
	La  oss
	bruke ${(a\cdot b)^{3}}$ som eksempel. Vi har at
\alg{
	(a\cdot b)^{3}&=(a\cdot b)\cdot(a\cdot b)\cdot(a\cdot b) \\
	&=a\cdot a\cdot a \cdot b \cdot b \cdot b \\
	&=a^3b^3
}
}\vsk \vsk

\newpage
\reg[\potsomgrunn \label{potsomgrunn}]{\vs
\begin{equation}
\left(a^{m}\right)^{n}=a^{m\cdot n}
\end{equation}}
\eks[1]{ \vs \vs
\alg{
\left(c^4\right)^5&=c^{4\cdot5}\\
&=c^{20}	
}	
}
\eks[2]{ \vs \vs 
\alg{
\left(3^\frac{5}{4}\right)^8&=3^{\frac{5}{4}\cdot8} \\
&=3^{10}
}	
}
\fork{\ref{potsomgrunn} \potsomgrunn}{
	La  oss bruke $\left(a^{3}\right)^{4}$ som eksempel. Vi har at
	\begin{align*}
	\left(a^{3}\right)^{4} & =a^{3}\cdot a^{3}\cdot a^{3}\cdot a^{3}
	\end{align*}
	
	
	Av \rref{potgang} er
	\algv{
		a^{3}\cdot a^{3}\cdot a^{3}\cdot a^{3} & =a^{3+3+3+3}\\
		& =a^{3\cdot4}\\
		&=a^{12}
	}	
}

\newpage
\reg[\textit{n}-rot]{ \vs
\[ a^\frac{1}{n}=\sqrt[n]{a} \]
Symbolet \sym{$ \sqrt{\phantom{a}} $} kallast eit \textit{rotteikn}\index{rotteikn}. For eksponenten $ \frac{1}{2} $ er det vanleg å utelate 2 i rotteiknet:
\[ a^\frac{1}{2}=\sqrt{a} \]
}
\eks{
Av \rref{potsomgrunn} har vi at
\alg{
\left(a^b\right)^\frac{1}{b}&=a^{b\cdot \frac{1}{b}} \\
&=a	
}
For eksempel er	
\algv{
9^\frac{1}{2}=\sqrt{9}=3 &\text{, sidan } 3^2 =9 \vn
125^\frac{1}{3}=\sqrt[3]{125}=5 &\text{, sidan } 5^3 =125 \vn	
16^\frac{1}{4}=\sqrt[4]{16}=2 &\text{, sidan } 2^4 =16
}	
}
\spr{
$\sqrt{9} $ kallast ''kvadratrota til 9'' \vsk

$ \sqrt[5]{9} $ kallast ''femterota til 9''.
}
\newpage
\section{\irrasj}
\reg[Irrasjonale tal]{
Eit tal som \textsl{ikkje} er eit rasjonalt tal, er eit irrasjonalt tal\index{tal!irrasjonalt}\footnotemark.\vsk

Verdien til eit irrasjonalt tal har uendeleg mange desimalar med eit ikkje-repeterande mønster.
}
\footnotetext{Strengt tatt er irrasjonale tal alle \textit{reelle} tal som ikkje er rasjonale tal. Men for å forklare kva \textit{reelle} tal er, må vi forklare kva \textit{imaginære} tal er, og det har vi valgt å ikkje gjere i denne boka. }
\eks[1]{
$ \sqrt{2} $ er eit irrasjonalt tal.
\[ \sqrt{2}=1.414213562373... \]
}












\end{document}
 
