\documentclass[english,hidelinks,pdftex, 11 pt, class=report,crop=false]{standalone}
\usepackage[T1]{fontenc}
\usepackage[utf8]{luainputenc}
\usepackage{lmodern} % load a font with all the characters
\usepackage{geometry}
\geometry{verbose,a4paper, inner=0cm, outer=0 cm, bmargin=2cm, tmargin=1cm}
%\textwidth=12cm
\setlength{\parindent}{0bp}
\usepackage{import}
\usepackage[subpreambles=false]{standalone}
\usepackage{amsmath}
\usepackage{amssymb}
\usepackage{esint}
\usepackage{babel}
\usepackage{tabu}
\usepackage[dvipsnames, table]{xcolor}
\usepackage{cancel}
\makeatother
\makeatletter
\usepackage{datetime2}
\usepackage{titlesec}
\usepackage[many]{tcolorbox}

% Eheter
\newcommand{\enh}[1]{\,\textrm{#1}}
%referances
\newcommand{\net}[2]{{\color{blue}\href{#1}{#2}}}

%Spaces
\newcommand{\vsk}{\\[12pt]}
\newcommand{\vs}{\vspace{-12pt}}

% Tabell for opplegg

\newcommand{\ovlist}[1]{
\vspace{-16pt}
\begin{itemize}
	#1
\end{itemize}
}

% Chapters and sections
\titleformat{\section}[block]{\bfseries}{\hspace{3cm}\thesection}{5pt}{}
\titleformat{\subsection}[block]{\bfseries}{\hspace{3cm}\thesection}{5pt}{}
\newcommand{\sectionbreak}{\clearpage} % New page on each section
 

\newlength{\mywidth}
\setlength{\mywidth}{14cm}

\newcommand{\cont}[1]{
\begin{tcolorbox}[center, boxrule=0.0 mm, width=\mywidth,arc=0mm,enhanced jigsaw,,colback=white,breakable]
#1	
\end{tcolorbox}
}

\newcommand{\info}[5]{
\begin{tcolorbox}[center, boxrule=0.1 mm, width=\mywidth,arc=0mm,enhanced jigsaw,breakable,colback=yellow!5]	
	
	\footnotesize
	\textbf{Øvingsområde}\\[5pt] #1 
	
	\textbf{Utstyr}\\ #2  \\
	
	\begin{tabular}{@{} p{4cm} p{4cm} l} 
		\textbf{Tid} & \textbf{Elevinndeling} & \textbf{Læringsarena} \\
		#3  & #4 & #5
	\end{tabular} 
\end{tcolorbox}	
}

\newcommand{\gjen}[1]{\begin{tcolorbox}[center,boxrule=0.1 mm, width=\mywidth,arc=0mm,colback=blue!3] {\large \textbf{Gjennomføring} \vspace{5 pt}}\newline #1  \end{tcolorbox}\vspace{-5pt}}
\newcommand{\eks}[1]{\begin{tcolorbox}[center,boxrule=0.1 mm, width=\mywidth,arc=0mm,colback=green!3] {\large \textbf{Eksempel} \vspace{5 pt}}\newline #1  \end{tcolorbox}\vspace{-5pt}}

\newcounter{opl}
%\numberwithin{opl}{article}


\newcommand{\opl}[1]{
\newpage
{\refstepcounter{opl} %\phantomsection 
\large \textbf{\theopl \;#1} \vsk}
}

% Headlines
\newcommand{\fork}{\textbf{Forkunnskapar}\\}
\newcommand{\forb}{\textbf{Forberedelsar}\\}
\newcommand{\opgvr}{\textbf{Oppgaver}}



%colors
\newcommand{\colr}[1]{{\color{red} #1}}
\newcommand{\colb}[1]{{\color{blue} #1}}
\newcommand{\colo}[1]{{\color{orange} #1}}
\newcommand{\colc}[1]{{\color{cyan} #1}}
\definecolor{projectgreen}{cmyk}{100,0,100,0}
\newcommand{\colg}[1]{{\color{projectgreen} #1}}

% Lister med bokstavar
\usepackage[inline]{enumitem}
% Opg
\newcommand{\abc}[1]{
	\begin{enumerate}[label=\alph*),leftmargin=18pt]
		#1
	\end{enumerate}
}

\usepackage[]{hyperref}

\begin{document}
\section*{Comment (for the particularly interested)}
Also in geometry, axioms (see comment on page \pageref{Kommentar1}) lays the ground of the mathematical system we create, but the axiomatic structure of geometry is quite extensive and intricate. In addition, some theorems are such intuitively true that it, at least in a book like this, would be more confusing than clarifying to explain them all in detail.\vsk

However, it is worth noticing that \rref{konsttre} states three terms regarding the unique construction in a triangle, and \rref{kongtre} states a term regarding congruence. In more advanced texts on geometry, chances are that you will recognize the content of these rules as axioms and theorems of congruence: \regv

\begin{tcolorbox}[boxrule=0.3 mm,arc=0mm,colback=blue!5] {\large \textbf{Congruence} \vspace{5 pt}}\newline
	Two triangles $ \triangle ABC $ and $ \triangle DEF $ are congruent if one of the following terms are satisfied:
	\begin{enumerate}[label=\roman*)]
		\item $ AB=DE $, $ BC=EF $ and $ \angle A=\angle D $.
		\item $ \angle A=\angle D $, $ \angle B=\angle E $ and $ AB=DE $.
		\item $ AB=DE $, $ BC=EF $ and $ AC=FD $.
		\item $ {\angle A=\angle D} $ and $ {\angle B=\angle E} $ and, in addition, $ {AB=DE} $ or $ BC=EF $ or $ AC=FD $.
	\end{enumerate}
	\fig{geo14k}
\rule{1\linewidth}{0.75bp}
\begin{center}
	\begin{tabular}{rl}
		i) &The Side-angle-side (SAS) axiom\\
		ii) &The Angle-side-angle (ASA) theorem\\
		iii) & The Side-side-side (SSS) theorem \\
		iv) & The Side-angle-angle (SAA) theorem 
	\end{tabular}
\end{center}
\end{tcolorbox}
\newpage
In the text-box on the previous side, term i)\,-\,iii) brings sufficient information regarding the unique construction of a triangle. However, in this book we have chosen to separate the concepts of congruence and unique construction. This is done under the presumption that most people will have a good intuition about congruent triangles, while having more difficulties stating terms of unique construction $ - $  and it is not necessarily easy to observe this directly from the terms of congruence.\vsk

Also, observe that iv) is just term ii) in a wider sense, but it cannot be used as a term of unique construction. Therefore, this term is not found in either \rref{konsttre} or \rref{kongtre}.
\end{document}

