\documentclass[english,hidelinks,pdftex, 11 pt, class=report,crop=false]{standalone}
\usepackage[T1]{fontenc}
\usepackage[utf8]{luainputenc}
\usepackage{lmodern} % load a font with all the characters
\usepackage{geometry}
\geometry{verbose,a4paper, inner=0cm, outer=0 cm, bmargin=2cm, tmargin=1cm}
%\textwidth=12cm
\setlength{\parindent}{0bp}
\usepackage{import}
\usepackage[subpreambles=false]{standalone}
\usepackage{amsmath}
\usepackage{amssymb}
\usepackage{esint}
\usepackage{babel}
\usepackage{tabu}
\usepackage[dvipsnames, table]{xcolor}
\usepackage{cancel}
\makeatother
\makeatletter
\usepackage{datetime2}
\usepackage{titlesec}
\usepackage[many]{tcolorbox}

% Eheter
\newcommand{\enh}[1]{\,\textrm{#1}}
%referances
\newcommand{\net}[2]{{\color{blue}\href{#1}{#2}}}

%Spaces
\newcommand{\vsk}{\\[12pt]}
\newcommand{\vs}{\vspace{-12pt}}

% Tabell for opplegg

\newcommand{\ovlist}[1]{
\vspace{-16pt}
\begin{itemize}
	#1
\end{itemize}
}

% Chapters and sections
\titleformat{\section}[block]{\bfseries}{\hspace{3cm}\thesection}{5pt}{}
\titleformat{\subsection}[block]{\bfseries}{\hspace{3cm}\thesection}{5pt}{}
\newcommand{\sectionbreak}{\clearpage} % New page on each section
 

\newlength{\mywidth}
\setlength{\mywidth}{14cm}

\newcommand{\cont}[1]{
\begin{tcolorbox}[center, boxrule=0.0 mm, width=\mywidth,arc=0mm,enhanced jigsaw,,colback=white,breakable]
#1	
\end{tcolorbox}
}

\newcommand{\info}[5]{
\begin{tcolorbox}[center, boxrule=0.1 mm, width=\mywidth,arc=0mm,enhanced jigsaw,breakable,colback=yellow!5]	
	
	\footnotesize
	\textbf{Øvingsområde}\\[5pt] #1 
	
	\textbf{Utstyr}\\ #2  \\
	
	\begin{tabular}{@{} p{4cm} p{4cm} l} 
		\textbf{Tid} & \textbf{Elevinndeling} & \textbf{Læringsarena} \\
		#3  & #4 & #5
	\end{tabular} 
\end{tcolorbox}	
}

\newcommand{\gjen}[1]{\begin{tcolorbox}[center,boxrule=0.1 mm, width=\mywidth,arc=0mm,colback=blue!3] {\large \textbf{Gjennomføring} \vspace{5 pt}}\newline #1  \end{tcolorbox}\vspace{-5pt}}
\newcommand{\eks}[1]{\begin{tcolorbox}[center,boxrule=0.1 mm, width=\mywidth,arc=0mm,colback=green!3] {\large \textbf{Eksempel} \vspace{5 pt}}\newline #1  \end{tcolorbox}\vspace{-5pt}}

\newcounter{opl}
%\numberwithin{opl}{article}


\newcommand{\opl}[1]{
\newpage
{\refstepcounter{opl} %\phantomsection 
\large \textbf{\theopl \;#1} \vsk}
}

% Headlines
\newcommand{\fork}{\textbf{Forkunnskapar}\\}
\newcommand{\forb}{\textbf{Forberedelsar}\\}
\newcommand{\opgvr}{\textbf{Oppgaver}}



%colors
\newcommand{\colr}[1]{{\color{red} #1}}
\newcommand{\colb}[1]{{\color{blue} #1}}
\newcommand{\colo}[1]{{\color{orange} #1}}
\newcommand{\colc}[1]{{\color{cyan} #1}}
\definecolor{projectgreen}{cmyk}{100,0,100,0}
\newcommand{\colg}[1]{{\color{projectgreen} #1}}

% Lister med bokstavar
\usepackage[inline]{enumitem}
% Opg
\newcommand{\abc}[1]{
	\begin{enumerate}[label=\alph*),leftmargin=18pt]
		#1
	\end{enumerate}
}

\usepackage[]{hyperref}

\begin{document}
\section*{Kommentar (for den spesielt interesserte)}
Også i geometri har vi aksiom (se kommentar på side \pageref{Kommentar1}) som legger grunnlaget for det matematiske systemet vi skaper, men den aksiomatiske oppbyggingen av geometri er mye mer omstendelig og uoversiktlig enn den vi har innenfor regning. I tillegg er noen teorem i geometri så intuitivt sanne, at det i ei bok som dette ville blitt mer forvirrende enn oppklarende å skulle forklart alt i detalj.\vsk

Det som likevel bør nevnes, er at vi i \rref{konsttre} opplyser om tre vilkår for å unikt konstruere en trekant, og i \rref{kongtre} gir et vilkår for kongruens. I mer avanserte geometritekster vil man helst finne igjen innholdet i disse to reglene som aksiom og teorem for kongruens: \regv

\begin{tcolorbox}[boxrule=0.3 mm,arc=0mm,colback=blue!5] {\large \textbf{Kongruens} \vspace{5 pt}}\newline
	To trekanterr $ \triangle ABC $ og $ \triangle DEF $ er kongruente hvis ett av følgende vilkår er oppfylt:
	\begin{enumerate}[label=\roman*)]
		\item $ AB=DE $, $ BC=EF $ og $ \angle A=\angle D $.
		\item $ \angle A=\angle D $, $ \angle B=\angle E $ og $ AB=DE $.
		\item $ AB=DE $, $ BC=EF $ og $ AC=FD $.
		\item $ {\angle A=\angle D} $ og $ {\angle B=\angle E} $, i tillegg er $ {AB=DE} $ eller $ BC=EF $ eller $ AC=FD $.
	\end{enumerate}
	\fig{geo14k}
\rule{1\linewidth}{0.75bp}
\begin{center}
	\begin{tabular}{rl}
		i) &Side-vinkel-side (SAS) aksiomet\\
		ii) &Vinkel-side-vinkel (ASA) teoremet\\
		iii) & Side-side-side (SSS) teoremet \\
		iv) & Side-vinkel-vinkel (SAA) teoremet 
	\end{tabular}
\end{center}
\end{tcolorbox}
\spr{
Forkortelsene SAS, ASA, SSS og SAA kommer av de engelske navnene for henholdsvis side og vinkel; \textit{side} og \textit{angle}.
}
\newpage
I tekstboksen på forrige side gir også vilkår (i)\,-\,(iii) tilstrekkeleg informasjon om når en trekant kan bli unikt konstruert, men i denne boka har vi valgt å skille unik konstruksjon og kongruens fra hverandre. Dette er gjort i den tru om at de fleste vil ha en intuitiv tanke om hvilke trekantar som er kongruente eller ikke, men ha større problem med å svare på hva som må til for å unikt konstruere en trekant $ - $ \\og det er ikke nødvendigvis så lett å se dette direkte ut ifra kongruensvilkårene.\vsk

Legg også merke til at vilkår (iv) bare er en mer generell form av vilkår (ii), men altså ikke kan brukes som et vilkår for unik konstruksjon. Dette vilkåret finner man derfor ikke igjen i hverken \rref{konsttre} eller \rref{kongtre}.
\end{document}

