\documentclass[english,hidelinks,pdftex, 11 pt, class=report,crop=false]{standalone}
\usepackage[T1]{fontenc}
\usepackage[utf8]{luainputenc}
\usepackage{lmodern} % load a font with all the characters
\usepackage{geometry}
\geometry{verbose,a4paper, inner=0cm, outer=0 cm, bmargin=2cm, tmargin=1cm}
%\textwidth=12cm
\setlength{\parindent}{0bp}
\usepackage{import}
\usepackage[subpreambles=false]{standalone}
\usepackage{amsmath}
\usepackage{amssymb}
\usepackage{esint}
\usepackage{babel}
\usepackage{tabu}
\usepackage[dvipsnames, table]{xcolor}
\usepackage{cancel}
\makeatother
\makeatletter
\usepackage{datetime2}
\usepackage{titlesec}
\usepackage[many]{tcolorbox}

% Eheter
\newcommand{\enh}[1]{\,\textrm{#1}}
%referances
\newcommand{\net}[2]{{\color{blue}\href{#1}{#2}}}

%Spaces
\newcommand{\vsk}{\\[12pt]}
\newcommand{\vs}{\vspace{-12pt}}

% Tabell for opplegg

\newcommand{\ovlist}[1]{
\vspace{-16pt}
\begin{itemize}
	#1
\end{itemize}
}

% Chapters and sections
\titleformat{\section}[block]{\bfseries}{\hspace{3cm}\thesection}{5pt}{}
\titleformat{\subsection}[block]{\bfseries}{\hspace{3cm}\thesection}{5pt}{}
\newcommand{\sectionbreak}{\clearpage} % New page on each section
 

\newlength{\mywidth}
\setlength{\mywidth}{14cm}

\newcommand{\cont}[1]{
\begin{tcolorbox}[center, boxrule=0.0 mm, width=\mywidth,arc=0mm,enhanced jigsaw,,colback=white,breakable]
#1	
\end{tcolorbox}
}

\newcommand{\info}[5]{
\begin{tcolorbox}[center, boxrule=0.1 mm, width=\mywidth,arc=0mm,enhanced jigsaw,breakable,colback=yellow!5]	
	
	\footnotesize
	\textbf{Øvingsområde}\\[5pt] #1 
	
	\textbf{Utstyr}\\ #2  \\
	
	\begin{tabular}{@{} p{4cm} p{4cm} l} 
		\textbf{Tid} & \textbf{Elevinndeling} & \textbf{Læringsarena} \\
		#3  & #4 & #5
	\end{tabular} 
\end{tcolorbox}	
}

\newcommand{\gjen}[1]{\begin{tcolorbox}[center,boxrule=0.1 mm, width=\mywidth,arc=0mm,colback=blue!3] {\large \textbf{Gjennomføring} \vspace{5 pt}}\newline #1  \end{tcolorbox}\vspace{-5pt}}
\newcommand{\eks}[1]{\begin{tcolorbox}[center,boxrule=0.1 mm, width=\mywidth,arc=0mm,colback=green!3] {\large \textbf{Eksempel} \vspace{5 pt}}\newline #1  \end{tcolorbox}\vspace{-5pt}}

\newcounter{opl}
%\numberwithin{opl}{article}


\newcommand{\opl}[1]{
\newpage
{\refstepcounter{opl} %\phantomsection 
\large \textbf{\theopl \;#1} \vsk}
}

% Headlines
\newcommand{\fork}{\textbf{Forkunnskapar}\\}
\newcommand{\forb}{\textbf{Forberedelsar}\\}
\newcommand{\opgvr}{\textbf{Oppgaver}}



%colors
\newcommand{\colr}[1]{{\color{red} #1}}
\newcommand{\colb}[1]{{\color{blue} #1}}
\newcommand{\colo}[1]{{\color{orange} #1}}
\newcommand{\colc}[1]{{\color{cyan} #1}}
\definecolor{projectgreen}{cmyk}{100,0,100,0}
\newcommand{\colg}[1]{{\color{projectgreen} #1}}

% Lister med bokstavar
\usepackage[inline]{enumitem}
% Opg
\newcommand{\abc}[1]{
	\begin{enumerate}[label=\alph*),leftmargin=18pt]
		#1
	\end{enumerate}
}

\usepackage[]{hyperref}

\begin{document}
	
\section*{Comment (for the particularly interested) \label{Kommentar1}}
Mathematics is \textit{axiomatically} founded. This means we declare\footnote{Preferably, as few as possible.} some propositions to be true, and these are called \textit{axioms} or \textit{postulates}. For the subject of calculations we have 12 axioms\footnote{The number can slightly vary, depending on how the axioms are expressed.}, but in this book we have confined ourselves to explicitly mention the following 6:
\regv 

\begin{tcolorbox}[boxrule=0.3 mm,arc=0mm,colback=blue!5] {\large \textbf{Axioms} \vspace{5 pt}}\newline
For the numbers $ a $, $ b $ and $ c $ we have
\alg{
a+(b+c)&=(a+b)+c \tag{A1} \label{a1}\\
a+b&=b+a \tag{A2}\label{a2}\\
a(bc)&=(ab)c \tag{A3}\label{a3}\\
ab&=ba \tag{A4}\label{a4}\\
a(b+c)&=ab+ac \tag{A5} \label{a5}\\
a\cdot\frac{1}{a}&=1 &&(a\neq0) \tag{A6} \label{a6}
} 
\rule{1\linewidth}{0.75bp}
\begin{center}
	\begin{tabular}{rl}
		\eqref{a1} &Associative law for addition\\
		\eqref{a2} & Commutative law for addition \\	
		\eqref{a3} & Associative law for multiplication \\
		\eqref{a4} & Commutative law for multiplication \\		
		\eqref{a5} & Distributive law\\	
		\eqref{a6} & Existence of a multiplicative identity
	\end{tabular}
\end{center}
\end{tcolorbox}
\vsk
By applying axioms, we can derive more complex contexts which we call \textit{theorems}. In this book we chose to let \textsl{rules} be the collective name for definitions, theorems and axioms. This is because alle three, in practice, draws up guidelines (rules) inside the mathematical system in which we wish to operate.\vsk
\newpage
In \hrs{Del1}{Part} we have tried to present the \textsl{motivation} behind the axioms, because (obviously) they are not randomly selected. The train of thoughts that leads us to them is the following:
\begin{enumerate}
	\item Vi define positive numbers as representations of either an amount or a placement on a number line.
	\item We define what addition, subtraction, multiplication and division entail for positive integers (and 0).
	\item From the marks above, it's as good as self-evident that \eqref{a1}\,-\,\eqref{a6} is valid for all positive integers.
	\item We define also fractions as representations of either an amount or a placement on a number line. What the elementary operations entail for fractions rests upon what is valid for the positive integers.
	\item From the remarks above, we conclude that \eqref{a1}\,-\,\eqref{a6} is valid for all rational numbers.
	\item We introduce negative numbers and an extended interpretation of addition and subtraction. This in turn leads to the interpretations of multiplication and division involving negative numbers.
	\item \eqref{a1}\,-\,\eqref{a6} is still valid after the introduction of negative integers. Deriving that they are also valid for negative rational numbers is a formality (omitted in the book).
	\item We can never write the value of an irrational number exact, but it can be approximated by a rational number\footnote{For example, we can write $ \sqrt{2}=1.414213562373...\approx\frac{1414213562373}{1000000000000} $}. Therefore, all calculations involving irrational umbers is, in practice, calculations involving rational numbers, and in this way we can conclude that\footnote{\textsl{Attention!} This explanation is good enough for the aim of this book but is a rather extreme simplification. Irrational numbers are a very complex subject, in fact, many books presenting advanced mathematics utilize several chapters to cover the subject in full depth .} \eqref{a1}\,-\,\eqref{a6} is also valid for irrational numbers.
\end{enumerate}
A similar train of thoughts can be applied concerning the power-rules found in \refsec{Potensar}. 
\end{document}

