\documentclass[english,hidelinks,pdftex, 11 pt, class=report,crop=false]{standalone}
\usepackage[T1]{fontenc}
\usepackage[utf8]{luainputenc}
\usepackage{lmodern} % load a font with all the characters
\usepackage{geometry}
\geometry{verbose,a4paper, inner=0cm, outer=0 cm, bmargin=2cm, tmargin=1cm}
%\textwidth=12cm
\setlength{\parindent}{0bp}
\usepackage{import}
\usepackage[subpreambles=false]{standalone}
\usepackage{amsmath}
\usepackage{amssymb}
\usepackage{esint}
\usepackage{babel}
\usepackage{tabu}
\usepackage[dvipsnames, table]{xcolor}
\usepackage{cancel}
\makeatother
\makeatletter
\usepackage{datetime2}
\usepackage{titlesec}
\usepackage[many]{tcolorbox}

% Eheter
\newcommand{\enh}[1]{\,\textrm{#1}}
%referances
\newcommand{\net}[2]{{\color{blue}\href{#1}{#2}}}

%Spaces
\newcommand{\vsk}{\\[12pt]}
\newcommand{\vs}{\vspace{-12pt}}

% Tabell for opplegg

\newcommand{\ovlist}[1]{
\vspace{-16pt}
\begin{itemize}
	#1
\end{itemize}
}

% Chapters and sections
\titleformat{\section}[block]{\bfseries}{\hspace{3cm}\thesection}{5pt}{}
\titleformat{\subsection}[block]{\bfseries}{\hspace{3cm}\thesection}{5pt}{}
\newcommand{\sectionbreak}{\clearpage} % New page on each section
 

\newlength{\mywidth}
\setlength{\mywidth}{14cm}

\newcommand{\cont}[1]{
\begin{tcolorbox}[center, boxrule=0.0 mm, width=\mywidth,arc=0mm,enhanced jigsaw,,colback=white,breakable]
#1	
\end{tcolorbox}
}

\newcommand{\info}[5]{
\begin{tcolorbox}[center, boxrule=0.1 mm, width=\mywidth,arc=0mm,enhanced jigsaw,breakable,colback=yellow!5]	
	
	\footnotesize
	\textbf{Øvingsområde}\\[5pt] #1 
	
	\textbf{Utstyr}\\ #2  \\
	
	\begin{tabular}{@{} p{4cm} p{4cm} l} 
		\textbf{Tid} & \textbf{Elevinndeling} & \textbf{Læringsarena} \\
		#3  & #4 & #5
	\end{tabular} 
\end{tcolorbox}	
}

\newcommand{\gjen}[1]{\begin{tcolorbox}[center,boxrule=0.1 mm, width=\mywidth,arc=0mm,colback=blue!3] {\large \textbf{Gjennomføring} \vspace{5 pt}}\newline #1  \end{tcolorbox}\vspace{-5pt}}
\newcommand{\eks}[1]{\begin{tcolorbox}[center,boxrule=0.1 mm, width=\mywidth,arc=0mm,colback=green!3] {\large \textbf{Eksempel} \vspace{5 pt}}\newline #1  \end{tcolorbox}\vspace{-5pt}}

\newcounter{opl}
%\numberwithin{opl}{article}


\newcommand{\opl}[1]{
\newpage
{\refstepcounter{opl} %\phantomsection 
\large \textbf{\theopl \;#1} \vsk}
}

% Headlines
\newcommand{\fork}{\textbf{Forkunnskapar}\\}
\newcommand{\forb}{\textbf{Forberedelsar}\\}
\newcommand{\opgvr}{\textbf{Oppgaver}}



%colors
\newcommand{\colr}[1]{{\color{red} #1}}
\newcommand{\colb}[1]{{\color{blue} #1}}
\newcommand{\colo}[1]{{\color{orange} #1}}
\newcommand{\colc}[1]{{\color{cyan} #1}}
\definecolor{projectgreen}{cmyk}{100,0,100,0}
\newcommand{\colg}[1]{{\color{projectgreen} #1}}

% Lister med bokstavar
\usepackage[inline]{enumitem}
% Opg
\newcommand{\abc}[1]{
	\begin{enumerate}[label=\alph*),leftmargin=18pt]
		#1
	\end{enumerate}
}

\usepackage[]{hyperref}

\begin{document}
	
\section*{Kommentar (for den spesielt interesserte) \label{Kommentar1}}
Matematikk er såkalla \textit{aksiomatisk} oppbygd. Dette betyr at vi erklærer nokre\footnote{Helst så få som mogleg.} påstandar for å vere sanne, og desse kallar vi for \textit{aksiom} eller \textit{postulat}. I rekning har ein om lag 12 aksiom\footnote{Talet avheng litt av korleis ein formulerer påstandane.}, men i denne boka har vi holdt oss til å nemne desse 6:
\regv 

\begin{tcolorbox}[boxrule=0.3 mm,arc=0mm,colback=blue!5] {\large \textbf{Aksiom} \vspace{5 pt}}\newline
For tala $ a $, $ b $ og $ c $ har vi at
\alg{
a+(b+c)&=(a+b)+c \tag{A1} \label{a1}\\
a+b&=b+a \tag{A2}\label{a2}\\
a(bc)&=(ab)c \tag{A3}\label{a3}\\
ab&=ba \tag{A4}\label{a4}\\
a(b+c)&=ab+ac \tag{A5} \label{a5}\\
a\cdot\frac{1}{a}&=1 &&(a\neq0) \tag{A6} \label{a6}
} 
\rule{1\linewidth}{0.75bp}
\begin{center}
	\begin{tabular}{rl}
		\eqref{a1} &Assosiativ lov ved addisjon\\
		\eqref{a2} & Kommutativ lov ved addisjon \\	
		\eqref{a3} & Assosiativ lov ved multiplikasjon \\
		\eqref{a4} & Kommutativ lov ved multiplikasjon \\		
		\eqref{a5} & Distributiv lov\\	
		\eqref{a6} & Eksistens av multiplikativ identitet
	\end{tabular}
\end{center}
\end{tcolorbox}
\vsk
Aksioma legg sjølve fundamentet i eit matematisk system. Ved hjelp av dei finn vi fleire og meir komplekse sanningar som vi kallar \textit{teorem}. I denne boka har vi valgt å kalle både aksiom, definisjonar og teorem for \textsl{reglar}. Dette fordi aksiom, definisjonar og teorem alle i praksis gir føringar (reglar) for handlingsrommet vi har innanfor det matematiske systemet vi opererer i.\vsk
\newpage
I \textsl{Del I} har vi forsøkt å presentere \textsl{motivasjonen} bak aksioma, for dei er sjølvsagt ikkje tilfeldig utvalde. Tankerekka som leder oss fram til dei nemnde aksioma kan  oppsummerast slik:
\begin{enumerate}
	\item Vi definerer positive tal som representasjonar av enten ei mengde eller ei plassering på ei tallinje.
	\item Vi definerer kva addisjon, subtraksjon, multiplikasjon og divisjon inneber for positive heiltal (og 0).
	\item Ut ifrå punkta over tilseier all fornuft at \eqref{a1}\,-\,\eqref{a6} må gjelde for alle positive heiltal.
	\item Vi definerer også brøk som representasjonar av ei mengde eller som ei plassering på ei tallinje. Kva dei fire rekneartane inneber for brøkar bygger vi på det som gjeld for positive heiltal.
	\item Ut ifrå punkta over finn vi at \eqref{a1}\,-\,\eqref{a6} gjeld for alle positive, rasjonale tal.
	\item Vi innfører negative heiltal, og utvider tolkinga av addisjon og subtraksjon. Dette gir så ei tolking av multiplikasjon og divisjon med negative heiltal.
	\item \eqref{a1}\,-\,\eqref{a6} gjeld også etter innføringa av negative heiltal. Å vise at dei også gjeld for negative, rasjonale tal er da ein rein formalitet.
	\item Vi kan aldri skrive verdien til eit irrasjonalt tal heilt eksakt, men verdien kan tilnærmast ved eit rasjonalt tal\footnote{For eksempel kan ein skrive $ \sqrt{2}=1.414213562373...\approx\frac{1414213562373}{1000000000000} $}. Alle utrekningar som inneber irrasjonale tal er derfor i \textsl{praksis} utrekningar som inneber rasjonale tal, og slik kan vi seie at\footnote{\textit{Obs!} Denne forklaringa er god nok for boka sitt formål, men er ei ekstrem\qquad forenkling. Irrasjonale tal er eit komplisert tema som mange bøker for avansert matematikk bruker mange kapittel for å forklare i full dybde.} \eqref{a1}\,-\,\eqref{a6} gjeld også for irrasjonale tal.
\end{enumerate}
Ei liknande tankerekke kan nyttast for å argumentere for potensreglane vi fann i \hrs{Potensar}{seksjon}. 
\end{document}

