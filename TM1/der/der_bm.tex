\documentclass[english,hidelinks,pdftex, 11 pt, class=report,crop=false]{standalone}

\newcommand{\note}{Merk}

% Geometry
\newcommand{\hlikb}{Midtnormalen i en likebeint trekant}
\newcommand{\arealsetn}{Arealsetningen}
\newcommand{\trkmedian}{Medianer i trekanter}
\newcommand{\midtrk}{Midtnormaler i trekanter}
\newcommand{\innskrsirk}{Halveringslinjer og innskrevet sirkel i trekanter}
\newcommand{\cossetn}{Cosinussetningen}
\newcommand{\perfvink}{Sentral- og periferivinkel}
\newcommand{\tang}{Tangent}

\usepackage[T1]{fontenc}
\usepackage[utf8]{luainputenc}
\usepackage{lmodern} % load a font with all the characters
\usepackage{geometry}
\geometry{verbose,a4paper, inner=0cm, outer=0 cm, bmargin=2cm, tmargin=1cm}
%\textwidth=12cm
\setlength{\parindent}{0bp}
\usepackage{import}
\usepackage[subpreambles=false]{standalone}
\usepackage{amsmath}
\usepackage{amssymb}
\usepackage{esint}
\usepackage{babel}
\usepackage{tabu}
\usepackage[dvipsnames, table]{xcolor}
\usepackage{cancel}
\makeatother
\makeatletter
\usepackage{datetime2}
\usepackage{titlesec}
\usepackage[many]{tcolorbox}

% Eheter
\newcommand{\enh}[1]{\,\textrm{#1}}
%referances
\newcommand{\net}[2]{{\color{blue}\href{#1}{#2}}}

%Spaces
\newcommand{\vsk}{\\[12pt]}
\newcommand{\vs}{\vspace{-12pt}}

% Tabell for opplegg

\newcommand{\ovlist}[1]{
\vspace{-16pt}
\begin{itemize}
	#1
\end{itemize}
}

% Chapters and sections
\titleformat{\section}[block]{\bfseries}{\hspace{3cm}\thesection}{5pt}{}
\titleformat{\subsection}[block]{\bfseries}{\hspace{3cm}\thesection}{5pt}{}
\newcommand{\sectionbreak}{\clearpage} % New page on each section
 

\newlength{\mywidth}
\setlength{\mywidth}{14cm}

\newcommand{\cont}[1]{
\begin{tcolorbox}[center, boxrule=0.0 mm, width=\mywidth,arc=0mm,enhanced jigsaw,,colback=white,breakable]
#1	
\end{tcolorbox}
}

\newcommand{\info}[5]{
\begin{tcolorbox}[center, boxrule=0.1 mm, width=\mywidth,arc=0mm,enhanced jigsaw,breakable,colback=yellow!5]	
	
	\footnotesize
	\textbf{Øvingsområde}\\[5pt] #1 
	
	\textbf{Utstyr}\\ #2  \\
	
	\begin{tabular}{@{} p{4cm} p{4cm} l} 
		\textbf{Tid} & \textbf{Elevinndeling} & \textbf{Læringsarena} \\
		#3  & #4 & #5
	\end{tabular} 
\end{tcolorbox}	
}

\newcommand{\gjen}[1]{\begin{tcolorbox}[center,boxrule=0.1 mm, width=\mywidth,arc=0mm,colback=blue!3] {\large \textbf{Gjennomføring} \vspace{5 pt}}\newline #1  \end{tcolorbox}\vspace{-5pt}}
\newcommand{\eks}[1]{\begin{tcolorbox}[center,boxrule=0.1 mm, width=\mywidth,arc=0mm,colback=green!3] {\large \textbf{Eksempel} \vspace{5 pt}}\newline #1  \end{tcolorbox}\vspace{-5pt}}

\newcounter{opl}
%\numberwithin{opl}{article}


\newcommand{\opl}[1]{
\newpage
{\refstepcounter{opl} %\phantomsection 
\large \textbf{\theopl \;#1} \vsk}
}

% Headlines
\newcommand{\fork}{\textbf{Forkunnskapar}\\}
\newcommand{\forb}{\textbf{Forberedelsar}\\}
\newcommand{\opgvr}{\textbf{Oppgaver}}



%colors
\newcommand{\colr}[1]{{\color{red} #1}}
\newcommand{\colb}[1]{{\color{blue} #1}}
\newcommand{\colo}[1]{{\color{orange} #1}}
\newcommand{\colc}[1]{{\color{cyan} #1}}
\definecolor{projectgreen}{cmyk}{100,0,100,0}
\newcommand{\colg}[1]{{\color{projectgreen} #1}}

% Lister med bokstavar
\usepackage[inline]{enumitem}
% Opg
\newcommand{\abc}[1]{
	\begin{enumerate}[label=\alph*),leftmargin=18pt]
		#1
	\end{enumerate}
}

\usepackage[]{hyperref}


\begin{document}

\section{Definisjoner \label{grense}}
Gitt en funksjon $ f(x) $ og to verdier $ x $-verdier $ x_1 $ og $ x_2 $, hvor $ {x_1<x_2} $. Den gjennomsnittlige endringen til $ f $ fra $ x_1 $ til $ x_2 $ er da gitt som
\[ \frac{f(x_2)-f(x_1)}{x_2-x_1} \]
Uttrykket over forteller hvor mye funksjonsverdien endrer seg i forhold til hvor mye $ x $-verdien endrer seg, og gir stigningstallet til linja som går gjennom punktene $ (x_1, f(x_1)) $ og $ (x_2, f(x_2)) $.
\[ figur \]
La oss finne den gjennomsnittlige endringen til $ f(x)=x^2 $ når $ x=2 $ og $ x=3 $.  
\nn{
\frac{f(3)-f(2)}{3-2}= \frac{3^2-2^2}{1}
= 5
}


Men vi kan jo så mye bedre enn dette. Det er ingenting som hindrer oss i å gjøre intervallet vi studerer mye mindre, og med dèt komme mye nærmere punktet vi er ute etter. Faktisk kan vi tenke oss en avstand mellom de to \textit{x}-verdiene som er så nære 0 som overhodet mulig. Betegner vi denne avstanden som $ \Delta x $ så skriver vi $ \lim\limits_{\Delta x \to 0} $\,, som indikerer at vi studerer tilfeller i grensen hvor $ \Delta x $ går mot 0. 

Så om vi nå ser på gjennomsnittsstigningen til \textit{f} mellom $ x=2 $ og \textit{x} i umiddelbar nærhet av 2, gir dette oss en uendelig god tilnærming til stignigstallet vi er ute etter. Resultatet kaller vi da \textit{den deriverte av f med hensyn på x for} $ x=2 $, som vi skriver som $ f'(2) $:
\[f'(2) =\lim\limits_{\Delta x \to 0} \frac{f(2+\Delta x)-f(2)}{\Delta x} \]



Så la oss nå prøve å regne ut $ f'(2) $:
\alg{
f'(2) &=\lim\limits_{\Delta x \to 0} \frac{f(2+\Delta x)-f(2)}{\Delta x} \\
&=\lim\limits_{\Delta x \to 0} \frac{(2+\Delta x)^2-2^2}{\Delta x} \\
&= \lim\limits_{\Delta x \to 0} \frac{2^2+4\Delta x + (\Delta x)^2-2^2}{\Delta x} \\
&= 4
}
Metoden vi har brukt over kan brukes for en hvilken som helst kontinuerlig funksjon av $ x $ for et hvilket som helst valg av $ x $.

\reg[Definisjon av den deriverte \label{derdefa}]{
Gitt en funksjon $ f(x) $. Den deriverte av $ f $ i $ x=a $ er da gitt som
\[ f'(a) =\lim\limits_{h \to 0} \frac{f(a+h)-f(x)}{h} \]
Linja som har stigingstall $ f'(a) $, og som går gjennom punktet $ (a, f(a)) $, kalles \textit{tangeringslinja} til $ f $ for $ x=a $.
\fig{der1}
}

\eks{
Gitt $ f(x)=x^3 $. Finn $ f'(a) $ .\vsk

\sv
Vi har at
\alg{
f'(a) &=\lim\limits_{h \to 0} \frac{(a+h)^3-a^3}{h} \br
&= \lim\limits_{h \to 0}\frac{a^3+3a^2h +3 ah^2+ h^3-a^3}{h} \br
&= \lim\limits_{h \to 0}\left(3a^2 +3ah+h^2\right) \\
&= 3a^2
}
Altså er $ f'(a)=3a^2 $.
}
\reg[Den deriverte som funksjon]{
Gitt en funksjon $ f $. Den deriverte av $ f $ med hensyn på $ x $ er da\footnote{Gitt at grenseverdien eksisterer.} definert som 
\begin{equation}
f'(x)=\lim\limits_{h\to0}\frac{f(x+h)-f(x)}{h}	
\end{equation} 
}

\subsubsection{Linearisering av en funksjon}
Gitt en funksjon $ f(x) $ og en variabel $ k $. Siden $ f'(a) $ angir stigningstallet til $ f(a) $ for $ x=a $, vil en tilnærming til $ f(a+k) $ være (se figur ???)
\[ f(a+k)\approx f(a)+f'(a)k \]
Det er ofte nyttig å vite differansen mellom en tilnærming og den faktiske verdien:
\begin{equation}\label{vareps}
	\varepsilon_f = f(a+k)-\left[f(a)+f'(a)k\right]
\end{equation}
Vi legger merket til at\footnote{Dette overlates til leseren å vise.} $ \lim\limits_{h\to0}\frac{\varepsilon_f}{k} =0 $, og skriver om \eqref{vareps} til en formel for $ f(x+k) $: \regv
\reg[Linearisering av en funskjon]{
Gitt en funskjon $ f(x) $ og en variabel $ k $. Da finnes en funksjon $ \varepsilon $ slik at
\begin{equation} \label{fa_k}
f(x+k)=f(x)+f'(a)k+\varepsilon_f	
\end{equation}
hvor $ \lim\limits_{h\to0}\frac{\varepsilon_f}{k}=0 $. \vsk

Tilnærmingen
\[ f(x+k)\approx f(x)+f'(x)k \]
kalles da lineæarapproksimasjonen av $ f(x+k) $.
}
\section{Derivasjonsregler}

\reg[Den deriverte av utvalgte funksjoner]{
\vspace{-11 pt}
\begin{align*}
&  (x^r)' = r x^{r-1}  \\
& (\ln x)'=  \frac{1}{x} \\
& (\sin x)' = \cos (x) \\
& (\cos x)' = -\sin (x)  \\
& (e^x)' =e^{x}  \\
& (\tan x)' =  \frac{1}{\cos^{2}x} =  1+\tan^{2} x
\end{align*}
}	
\vspace{11 pt}
\section{Kjerneregelen}
\textbf{Bevis for kjerneregelen}

La oss se på tre funksjoner \textit{f} og \textit{g} som oppfyller likheten $f(x)= g(u(x)) $. \textit{f} beskrives direkte av \textit{x}, mens \textit{g} beskrives av indirekte av \textit{x} som en funksjon av $ u(x) $. \vsk

La oss bruke $f(x) =e^{x^2} $ som eksempel. Kjenner vi verdien til \textit{x}, kan vi fort regne ut hva verdien til $ f(x) $ er. For eksempel er: 
\[ f(2)=e^4 \]
Men vi kan også skrive $ g(u(x))=e^{u(x)} $, hvor $ u(x)=x^2 $. Denne skrivemåten impliserer at når vi kjenner verdien til \textit{x}, så regner vi først ut verdien til \textit{u}, før vi til slutt finnner verdien av $g$: 
\[ u(2) =4\qquad,\qquad 
g(u(2))=e^{u(2)}=e^4 \]
Så det vi har nå er fire unike størrelser: en varierende \textit{x}, \textit{f} som funksjon av \textit{x}, \textit{u} som funksjon av \textit{x} og \textit{g} som funksjon av \textit{u}.
\[ fig \]
Av derdef?? har vi at
\alg{
f(x)&=\lim\limits_{h \to 0}\frac{f(x+h)-f(x)}{h} \br
&= \lim\limits_{h \to 0}\frac{g\left[u(x+h)\right]-g\left[u(x)\right]}{h}
}
Vi setter $ k = u(x+h)-u(x) $. Da er
\alg{
\lim\limits_{h \to 0}\frac{g\left[u(x+h)\right]-g\left[u(x)\right]}{h}&= \lim\limits_{h \to 0}\frac{g\left[u+k\right]-g\left[u\right]}{h}
}
Av \eqref{fa_h} har vi at
\[ g(u)-g(u+k)= g'(u)k+\varepsilon_g  \]
Altså er
\alg{
\lim\limits_{h \to 0}\frac{g(u+k)-g(u)}{h}& = \lim\limits_{h\to 0}\frac{g'(u)k+\varepsilon_g}{h}\br
&=\lim\limits_{h\to 0}\left(g'(u)+\frac{\varepsilon_g}{k}\right)\frac{k}{h}
}
Da $ \lim\limits_{h\to0}k=0 $, er $ \lim\limits_{h\to0}\frac{\varepsilon_g}{k}=0 $. Videre har vi at $ \lim\limits_{h\to0}\frac{k}{h}=u'(x) $. Altså har vi at
\nn{
\lim\limits_{h\to 0}\left(g'(u)+\frac{\varepsilon_g}{k}\right)\frac{k}{h}=g'(u)u'(x)
}
\reg[Kjerneregelen \label{kjerne}]{
For en funksjon $ f(x)=g(u(x)) $ kan vi finne \textit{f} derivert med hensyn på \textit{x} som:
\[ f'(x) = g'(u)u'(x) \]
}
\vspace{11 pt}

\eks{
Finn $ f'(x) $ når  $ f(x)=e^{x^2+x+1} $
\\

\textbf{Svar:} Vi setter $ u=x^2+x+1 $, og får:
\begin{align*}
& g(u)=e^u \\
& g'(u)=e^u \\
& u'(x) = 2x + 1
\end{align*}
Altså blir:
\alg{
 f'(x) &= g'(u)u'(x) \\
 &= e^{(x^2+x+1)}(2x+1) 
}
}
\section{Produktregelen}
\textbf{Bevis for produktregelen}

Si at vi har en funksjon \textit{f} som består av to funksjoner \textit{u} og \textit{v}, som begge er avhengige av \textit{x}:  
\[ f(x)=u(x)v(x) \] 

For enhver kontinuerlig funksjon \textit{g} er $ g'(x) $ er definert som:
\[ g'= \lim\limits_{\Delta x\to 0}\frac{g(x+\Delta x)-g(x)}{\Delta x} \]

$ f'(x) $ kan vi derfor skrive som:
\[ f'=\lim\limits_{\Delta x\to 0}\frac{u(x+\Delta x)v(x+\Delta x)-u(x)v(x)}{\Delta x} \]

La oss nå skrive $ u(x) $ og $ v(x) $ som \textit{u} og \textit{v} og $ u(x+\Delta x) $ og $ v(x+\Delta x) $ som $ \tilde{u} $ og $ \tilde{v} $:
\alg{
f'(x) &= \lim\limits_{\Delta x\to 0} \frac{\tilde{u}\tilde{v}- u v}{\Delta x}
}
Vi kan alltids legge til 0 i form av $ \dfrac{u\tilde{v}}{\Delta x}- \dfrac{u\tilde{v}}{\Delta x} $:
\alg{f'&= \lim\limits_{\Delta x\to 0} \left[\frac{\tilde{u}\tilde{v}- u v}{\Delta x} + \frac{u\tilde{v}}{\Delta x}- \frac{u\tilde{v}}{\Delta x}\right] \\
&= \lim\limits_{\Delta x\to 0}\left[\frac{(\tilde{u}- u)\tilde{v}}{\Delta x} +\frac{u(\tilde{v}- v)}{\Delta x} \right]
	}
Siden vi for enhver kontinuerlig funksjon $ g $ har at $ \lim\limits_{\Delta x\to 0} \tilde{g}=g $ og at $ \lim\limits_{\Delta x\to 0}  \frac{\tilde{g}-g}{\Delta x}=g'$, får vi nå:
\[ f'=u'v+uv' \]	

\reg[Produktregelen ved derivasjon]{
Gitt $ {f(x)=u(x)v(x)} $ da er
\[ f'=u'v+uv' \]	
}
\section{Divisjonsregelen}
Dersom vi har uttrykket $ f(x)=\dfrac{u(x)}{v(x)} $ kan vi bruke produktregelen og kjerneregelen:
\alg{f'&=\left(\dfrac{u}{v} \right)'\\
	&= \left(u v^{-1}\right)' \\
	&= u' v^{-1} - u v^{-2}v' \\
	&= \frac{u'v - u v'}{v^2}}

\reg[Divisjonsregelen ved derivasjon]{
Dersom vi har funksjonen $ f(x)=\dfrac{u(x)}{v(x)} $, kan vi finne $ f'(x) $ ved:
\[ f'=\frac{u'v-uv'}{v^2} \]
}

\fork{L'hoptial}{
Siden $ f(a)=g(a)=0 $, er
\alg{
\lim\limits_{x\to a}\frac{f(x)}{g(x)}=\lim\limits_{x\to a} \frac{f(x)-f(a)}{g(x)-g(a)}
}
Vi setter $ k=a-x $, da har vi av linaprx?? at
\alg{
f(x)-f(a)&=f(x)-f(x+h)=-f'(x)k-\varepsilon_f \vn
g(x)-g(a)&=g(x)-g(x+h)=-g'(x)k-\varepsilon_g 
}
Altså er
\alg{
\lim\limits_{x\to a}\frac{f(x)}{g(x)}=\lim\limits_{x\to a}\frac{f'(x)+\frac{\varepsilon_f}{k}}{g'(x)+\frac{\varepsilon_g}{k}}
}
Da $ \lim\limits_{x\to a} k = 0$, har vi at $ {\lim\limits_{x\to a} \frac{\varepsilon_f}{k}= \lim\limits_{x\to a}  \frac{\varepsilon_g}{k}=0} $
Altså er 
\[ \lim\limits_{x\to a}\frac{f(x)}{g(x)}=\lim\limits_{x\to a}\frac{f'(x)}{g'(x)} \]}
\fork{L'hopital 2}{
Vi har at
\alg{
\lim\limits_{x\to a} \frac{g}{f}&=\lim\limits_{x\to a} \frac{\frac{1}{f}}{\frac{1}{g}} 
}
Da $ \lim\limits_{x\to a} f=\lim\limits_{x\to a} g=0 $, må $ \lim\limits_{x\to a} \frac{1}{f}=\lim\limits_{x\to a}\frac{1}{g}=0 $. Av Lhopital1?? har vi da at 
\alg{
\lim\limits_{x\to a} \frac{g}{f}&=\lim\limits_{x\to a} \frac{\frac{1}{f^2}f'}{\frac{1}{g^2}g'} 
}
Multipliserer vi begge sider med $ \lim\limits_{x\to a} \frac{f^2}{g^2} $, får vi at
\alg{
\lim\limits_{x\to a} \frac{f}{g}=\lim\limits_{x\to a}\frac{f'}{g'}
}	
}
\end{document}