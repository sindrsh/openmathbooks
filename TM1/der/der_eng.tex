\documentclass[english,hidelinks,pdftex, 11 pt, class=report,crop=false]{standalone}
\usepackage[T1]{fontenc}
\usepackage[utf8]{luainputenc}
\usepackage{lmodern} % load a font with all the characters
\usepackage{geometry}
\geometry{verbose,a4paper, inner=0cm, outer=0 cm, bmargin=2cm, tmargin=1cm}
%\textwidth=12cm
\setlength{\parindent}{0bp}
\usepackage{import}
\usepackage[subpreambles=false]{standalone}
\usepackage{amsmath}
\usepackage{amssymb}
\usepackage{esint}
\usepackage{babel}
\usepackage{tabu}
\usepackage[dvipsnames, table]{xcolor}
\usepackage{cancel}
\makeatother
\makeatletter
\usepackage{datetime2}
\usepackage{titlesec}
\usepackage[many]{tcolorbox}

% Eheter
\newcommand{\enh}[1]{\,\textrm{#1}}
%referances
\newcommand{\net}[2]{{\color{blue}\href{#1}{#2}}}

%Spaces
\newcommand{\vsk}{\\[12pt]}
\newcommand{\vs}{\vspace{-12pt}}

% Tabell for opplegg

\newcommand{\ovlist}[1]{
\vspace{-16pt}
\begin{itemize}
	#1
\end{itemize}
}

% Chapters and sections
\titleformat{\section}[block]{\bfseries}{\hspace{3cm}\thesection}{5pt}{}
\titleformat{\subsection}[block]{\bfseries}{\hspace{3cm}\thesection}{5pt}{}
\newcommand{\sectionbreak}{\clearpage} % New page on each section
 

\newlength{\mywidth}
\setlength{\mywidth}{14cm}

\newcommand{\cont}[1]{
\begin{tcolorbox}[center, boxrule=0.0 mm, width=\mywidth,arc=0mm,enhanced jigsaw,,colback=white,breakable]
#1	
\end{tcolorbox}
}

\newcommand{\info}[5]{
\begin{tcolorbox}[center, boxrule=0.1 mm, width=\mywidth,arc=0mm,enhanced jigsaw,breakable,colback=yellow!5]	
	
	\footnotesize
	\textbf{Øvingsområde}\\[5pt] #1 
	
	\textbf{Utstyr}\\ #2  \\
	
	\begin{tabular}{@{} p{4cm} p{4cm} l} 
		\textbf{Tid} & \textbf{Elevinndeling} & \textbf{Læringsarena} \\
		#3  & #4 & #5
	\end{tabular} 
\end{tcolorbox}	
}

\newcommand{\gjen}[1]{\begin{tcolorbox}[center,boxrule=0.1 mm, width=\mywidth,arc=0mm,colback=blue!3] {\large \textbf{Gjennomføring} \vspace{5 pt}}\newline #1  \end{tcolorbox}\vspace{-5pt}}
\newcommand{\eks}[1]{\begin{tcolorbox}[center,boxrule=0.1 mm, width=\mywidth,arc=0mm,colback=green!3] {\large \textbf{Eksempel} \vspace{5 pt}}\newline #1  \end{tcolorbox}\vspace{-5pt}}

\newcounter{opl}
%\numberwithin{opl}{article}


\newcommand{\opl}[1]{
\newpage
{\refstepcounter{opl} %\phantomsection 
\large \textbf{\theopl \;#1} \vsk}
}

% Headlines
\newcommand{\fork}{\textbf{Forkunnskapar}\\}
\newcommand{\forb}{\textbf{Forberedelsar}\\}
\newcommand{\opgvr}{\textbf{Oppgaver}}



%colors
\newcommand{\colr}[1]{{\color{red} #1}}
\newcommand{\colb}[1]{{\color{blue} #1}}
\newcommand{\colo}[1]{{\color{orange} #1}}
\newcommand{\colc}[1]{{\color{cyan} #1}}
\definecolor{projectgreen}{cmyk}{100,0,100,0}
\newcommand{\colg}[1]{{\color{projectgreen} #1}}

% Lister med bokstavar
\usepackage[inline]{enumitem}
% Opg
\newcommand{\abc}[1]{
	\begin{enumerate}[label=\alph*),leftmargin=18pt]
		#1
	\end{enumerate}
}

\usepackage[]{hyperref}

% note
\newcommand{\note}{Note}
\newcommand{\notesm}[1]{{\footnotesize \textsl{\note:} #1}}
\newcommand{\selos}{See the solutions manual.}

\newcommand{\texandasy}{The text is written in \LaTeX\ and the figures are made with the aid of Asymptote.}

\newcommand{\rknut}{Calculate.}
\newcommand\sv{\vsk \textbf{Answer} \vspace{4 pt}\\}
\newcommand{\ekstitle}{Example }
\newcommand{\sprtitle}{The language box}
\newcommand{\expl}{explanation}

% answers
\newcommand{\mulansw}{\notesm{Multiple possible answers.}}	
\newcommand{\faskap}{Chapter}

% exercises
\newcommand{\opgt}{\newpage \phantomsection \addcontentsline{toc}{section}{Exercises} \section*{Exercises for Chapter \thechapter}\vs \setcounter{section}{1}}

% references
\newcommand{\reftab}[1]{\hrs{#1}{Table}}
\newcommand{\rref}[1]{\hrs{#1}{Rule}}
\newcommand{\dref}[1]{\hrs{#1}{Definition}}
\newcommand{\refkap}[1]{\hrs{#1}{Chapter}}
\newcommand{\refsec}[1]{\hrs{#1}{Section}}
\newcommand{\refdsec}[1]{\hrs{#1}{Subsection}}
\newcommand{\refved}[1]{\hrs{#1}{Appendix}}
\newcommand{\eksref}[1]{\textsl{#1}}
\newcommand\fref[2][]{\hyperref[#2]{\textsl{Figure \ref*{#2}#1}}}
\newcommand{\refop}[1]{{\color{blue}Exercise \ref{#1}}}
\newcommand{\refops}[1]{{\color{blue}Exercise \ref{#1}}}

%%% SECTION HEADLINES %%%

% Our numbers
\newcommand{\likteikn}{The equal sign}
\newcommand{\talsifverd}{Numbers, digits and values}
\newcommand{\koordsys}{Coordinate systems}

% Calculations
\newcommand{\adi}{Addition}
\newcommand{\sub}{Subtraction}
\newcommand{\gong}{Multiplication}
\newcommand{\del}{Division}

%Factorization and order of operations
\newcommand{\fak}{Factorization}
\newcommand{\rrek}{Order of operations}

%Fractions
\newcommand{\brgrpr}{Introduction}
\newcommand{\brvu}{Values, expanding and simplifying}
\newcommand{\bradsub}{Addition and subtraction}
\newcommand{\brgngheil}{Fractions multiplied by integers}
\newcommand{\brdelheil}{Fractions divided by integers}
\newcommand{\brgngbr}{Fractions multiplied by fractions}
\newcommand{\brkans}{Cancelation of fractions}
\newcommand{\brdelmbr}{Division by fractions}
\newcommand{\Rasjtal}{Rational numbers}

%Negative numbers
\newcommand{\negintro}{Introduction}
\newcommand{\negrekn}{The elementary operations}
\newcommand{\negmeng}{Negative numbers as amounts}

%Calculation methods
\newcommand{\delmedtihundre}{Deling med 10, 100, 1\,000 osv.}

% Geometry 1
\newcommand{\omgr}{Terms}
\newcommand{\eignsk}{Attributes of triangles and quadrilaterals}
\newcommand{\omkr}{Perimeter}
\newcommand{\area}{Area}

%Algebra 
\newcommand{\algintro}{Introduction}
\newcommand{\pot}{Powers}
\newcommand{\irrasj}{Irrational numbers}

%Equations
\newcommand{\ligintro}{Introduction}
\newcommand{\liglos}{Solving with the elementary operations}
\newcommand{\ligloso}{Solving with elementary operations summarized}

%Functions
\newcommand{\fintro}{Introduction}
\newcommand{\lingraf}{Linear functions and graphs}

%Geometry 2
\newcommand{\geoform}{Formulas of area and perimeter}
\newcommand{\kongogsim}{Congruent and similar triangles}
\newcommand{\geofork}{Explanations}

% Names of rules
\newcommand{\adkom}{Addition is commutative}
\newcommand{\gangkom}{Multiplication is commutative}
\newcommand{\brdef}{Fractions as rewriting of division}
\newcommand{\brtbr}{Fractions multiplied by fractions}
\newcommand{\delmbr}{Fractions divided by fractions}
\newcommand{\gangpar}{Distributive law}
\newcommand{\gangparsam}{Paranthesis multiplied together}
\newcommand{\gangmnegto}{Multiplication by negative numbers I}
\newcommand{\gangmnegtre}{Multiplication by negative numbers II}
\newcommand{\konsttre}{Unique construction of triangles}
\newcommand{\kongtre}{Congruent triangles}
\newcommand{\topv}{Vertical angles}
\newcommand{\trisum}{The sum of angles in a triangle}
\newcommand{\firsum}{The sum of angles in a quadrilateral}
\newcommand{\potgang}{Multiplication by powers}
\newcommand{\potdivpot}{Division by powers}
\newcommand{\potanull}{The special case of \boldmath $a^0$}
\newcommand{\potneg}{Powers with negative exponents}
\newcommand{\potbr}{Fractions as base}
\newcommand{\faktgr}{Factors as base}
\newcommand{\potsomgrunn}{Powers as base}
\newcommand{\arsirk}{The area of a circle}
\newcommand{\artrap}{The area of a trapezoid}
\newcommand{\arpar}{The area of a parallelogram}
\newcommand{\pyt}{Pythagoras's theorem}
\newcommand{\forform}{Ratios in similar triangles}
\newcommand{\vilkform}{Terms of similar triangles}
\newcommand{\omkrsirk}{The perimeter of a circle (and the value of $ \bm \pi $)}
\newcommand{\artri}{The area of a triangle}
\newcommand{\arrekt}{The area of a rectangle}
\newcommand{\liknflyt}{Moving terms across the equal sign}
\newcommand{\funklin}{Linear functions}




\begin{document}

\section{Definitions \label{grense}}
Given a function $ f(x) $ and two $ x $-values $ a $ and $ b $. The change in $ f $ relative to the change in $ x $ for these values is given as
\begin{equation}\label{releq}
	\frac{f(b)-f(a)}{b-a}
\end{equation}
In \mb\ we have seen that the expression above gives the slope of the line that passes through the points $ (a, f(a)) $ and $ (b, f(b)) $. In a mathematical context, it is particularly interesting to examine \eqref{releq} when $ b $ approaches $ a $.
\begin{figure}
	\centering
	\subfloat[]{
		\includegraphics[scale=0.8]{\figp{der2a}}
	}\qquad
	\subfloat[]{
		\includegraphics[scale=0.8]{\figp{der2b}}
	}\\
	\subfloat[]{
		\includegraphics[scale=0.8]{\figp{der2c}}
	} \qquad
	\subfloat[]{
		\includegraphics[scale=0.8]{\figp{der2d}}
	}
\end{figure}
By introducing the number $ h $, and setting $ b=a+h $, we can write \eqref{releq} as
\[ \frac{f(a+h)-f(a)}{h} \]
To \outl{differentiate} involves examining the limit of this fraction as $ h $ approaches 0.
\info{\note}{
	In the text and figures above, we have assumed that $ {b>a} $, but this is not a prerequisite for the expressions to be valid.
}
\regdef[The derivative\label{derdefa}]{
	Given a function $ f(x) $. The \outl{derivative} of $ f $ at $ {x=a} $ is then given as
	\begin{equation}\label{derdefaeq}
		f'(a) =\lim\limits_{h \to 0} \frac{f(a+h)-f(a)}{h}
	\end{equation}
	The line that has the slope $ f'(a) $, and passes through the point $ (a, f(a)) $, is called the \outl{tangent line} to $ f $ for $ x=a $.
	\fig{der1}
}
\eks[1]{
	Given $ f(x)=x^2 $. Find $ f'(2) $.
	
	\sv
	We have that
	\algv{
		f'(2) &=\lim\limits_{h \to 0} \frac{f(2+h)-f(2)}{h} \\
		&=\lim\limits_{h \to 0} \frac{(2+h)^2-2^2}{h} \\
		&= \lim\limits_{h \to 0} \frac{2^2+4h + (h)^2-2^2}{h} \\
		&= 4
	}
}
\newpage
\eks[2]{ \label{dereks2}
	Given $ f(x)=x^3 $. Find $ f'(a) $ .\vsk
	
	\sv
	We have that
	\alg{
		f'(a) &=\lim\limits_{h \to 0} \frac{(a+h)^3-a^3}{h} \br
		&= \lim\limits_{h \to 0}\frac{a^3+3a^2h +3 ah^2+ h^3-a^3}{h} \br
		&= \lim\limits_{h \to 0}\left(3a^2 +3ah+h^2\right) \\
		&= 3a^2
	}
	Thus, $ f'(a)=3a^2 $.
}
\info{Alternative definition}{
	An equivalent version of \eqref{derdefaeq} is
	\begin{equation}\label{derdefalt}
		f'(a)=	\lim\limits_{b\to a} \frac{f(b)-f(a)}{b-a}
	\end{equation}
}

\subsubsection{Linearization of a Function}
Given a function $ f(x) $ and a variable $ k $. Since $ f'(a) $ represents the slope of $ f(a) $ for $ x=a $, an approximation to $ f(a+k) $ would be
\[ f(a+k)\approx f(a)+f'(a)k \]
It is often useful to know the difference $ \varepsilon $ between an approximation and the actual value:
\begin{equation}\label{vareps}
	\varepsilon = f(a+k)-\left[f(a)+f'(a)k\right]
\end{equation}
We note that\footnote{This is left as an exercise for the reader.} $ \lim\limits_{k\to0}\frac{\varepsilon}{k} =0 $, and reformulate \eqref{vareps} into a formula for $ f(a+k) $: \regv
\reg[Linearization of a function \label{linfunk}]{
	Given a function $ f(x) $ and a variable $ k $. Then there exists a function $ \varepsilon(k) $ such that
	\begin{equation} \label{linfunkeq}
		f(a+k)=f(a)+f'(a)k+\varepsilon
	\end{equation}
	where $ \lim\limits_{k\to0}\frac{\varepsilon}{k}=0 $. \vsk
	
	The approximation
	\begin{equation}\label{linaprx}
		f(a+k)\approx f(a)+f'(a)k
	\end{equation}
	is called the \outl{linear approximation} of $ f(a+k) $.
	\fig{flinear}
}

\section{Rules of Differentiation}
\subsubsection{The Derivative Function}
\textsl{Example 2} on page \pageref{dereks2} highlights something important; if the limit in \eqref{derdefaeq} exists, then $ f'(a) $ will be expressed in terms of $ a $. And although $ a $ is considered a constant along the way to this expression, there is nothing to prevent us from treating $ a $ as a variable afterwards. If $ f'(a) $ results from differentiating the function $ f(x) $, it is also convenient to rename $ a $ to $ x $:\regv
\reg[The derivative function]{
	Given a function $ f(x) $. The \outl{derivative function} of $ f $ is the function that results from replacing $ a $ in \eqref{derdefaeq} with $ x $. This function is written as $ f'(x) $.
}
\eks[]{
	Given $ {f(x)=x^3} $. Since\footnote{See \textsl{Example 2}, page \pageref{dereks2}.} $ {f'(a)=3a^2} $, $ f'(x)=3x^2 $. 
} \regv

\spr{
	Alternative notations for \sym{$f'$} are 
	\sym{$ (f)'$} and \sym{$\frac{\text{d}}{\text{d}x} f $}.
}
\newpage
\info{Derivative with respect to}{
	The differentiation we have seen so far has been a fraction with a difference of $ x $-values in the denominator and the associated difference of $ f $-values in the numerator. We say that $ f $ is differentiated with \outl{respect to $ x $}. In this book series, we will primarily look at functions that depend on only one variable. Given a function $ f(x) $, it is then understood that \sym{$ f' $} symbolizes $ f $ differentiated with respect to $ x $.\vsk
	
	At the same time, it is useful to be aware that a function can depend on several variables. For example, the function
	\[ f(x, y)= x^2+y^3 \]
	is a \outl{multivariable function}, dependent on both $ x $ and $ y $. In this case, we can use \sym{$ \frac{\text{d}}{\text{d}x} f$} to indicate differentiation with respect to $ x $, and \sym{$ \frac{\text{d}}{\text{d}x}f $} to indicate differentiation with respect to $ y $. The reader may like to explain for themselves why the following is true:
	\[ \frac{\text{d}}{\text{d}x}f = 2x \qquad,\qquad  \frac{\text{d}}{\text{d}y}f = 3y^2 ,\]
}

\newpage
\subsection{The Derivative of Elementary Functions}
\reg[\derel \label{derel}]{
	For a variable $ x $ and a constant $ r $, the following are true:
	\begin{align}
		& (e^x)' = e^{x} \label{derex} \vn	
		& (x^r)' = r x^{r-1} \label{derxr} \vn
		& (\ln x)'=  \frac{1}{x} \label{derlnx}
	\end{align}
}	
\reg[The Derivative of Composite Functions]{
	Given a constant $ a $ and the functions $ f(x) $ and $ g(x) $. Then,
	\begin{align}
		& \left(a\cdot f\right)'=a\cdot f' \vn
		& \left(f+g\right)' =f'+g' \vn
		& \left(f-g\right)'=f'-g'
	\end{align}
} \regv

\regdef[The Second Derivative]{
	Given a differentiable function $ f(x) $. Then, the \outl{second derivative} of $ f $ is given as
	\begin{equation}\label{andredereq}
		\left(f'\right)'=f''
	\end{equation}
} \regv

\regdef[The Derivative of a Vector Function \label{vektfunkder}]{
	Given the functions $ f(t) $, $ g(t) $, and $ v(t)=[f(t), g(t)] $. Then,
	\begin{equation}\label{vektfunkdereq}
		v'(t)=[f'(t), g'(t)]	
	\end{equation}
}
\newpage
\subsection{Chain, Product, and Quotient Rules in Differentiation}
\reg[\kjernereg \label{kjerne}]{
	For a function $ f(x)=g\left[u(x)\right] $, we have:
	\begin{equation}\label{kjerneeq}
		f'(x) = g'(u)u'(x)
	\end{equation}
}
\eks{
	Find $ f'(x) $ when  $ f(x)=e^{x^2+x+1} $.\\
	\sv
	
	We set $ u=x^2+x+1 $, and then
	\begin{align*}
		g(u)&=e^u &
		g'(u)&=e^u &
		u'(x) &= 2x + 1
	\end{align*}
	Thus,
	\algv{
		f'(x) &= g'(u)u'(x) \\
		&= e^u(2x+1)\\
		&= e^{x^2+x+1}(2x+1) 
	}
} \vsk

\reg[\prodregder \label{prodregder}]{
	Given the functions $ f(x) $, $ u(x) $, and $ v(x) $, where $ {f=uv} $, then
	\[ f'=u'v+uv' \]	
}
\eks[1]{
	Find the derivative of the function $ f(x)=x^2e^x $.
	
	\sv
	We set $ {u(x)=x^2} $ and $ v(x)=e^x $, then
	\alg{
		f &= uv
		&u'&=2x
		&v'&=e^x
	}
	Thus,
	\algv{
		f' &= 2xe^x +x^2 e^x \\
		&=xe^x(2+x)
	}
}
\vsk

\reg[\divder \label{divder}]{
	Given the functions $ f(x) $, $ u(x) $, and $ v(x) $, where $ f=\dfrac{u}{v} $. Then,
	\begin{equation}\label{divdereq}
		f'=\frac{u'v-uv'}{v^2} 
	\end{equation}
}
\eks{
	Find the derivative of the function $ f(x)=\frac{\ln x}{x^4} $.
	
	\sv
	We set $ u(x)=\ln x $ and $ v(x)=x^4 $, then
	\alg{
		f&=\frac{u}{v} & u'&=x^{-1} & v' &= 4x^3
	}
	Thus,
	\alg{
		f' &= \frac{x^{-1}\cdot x^4-\ln x\cdot 4x^3}{x^8} \br
		&= \frac{1-4 \ln x}{x^5}
	}
	\notesm{We could also find $ f' $ by setting $ {u(x)=\ln x} $ and $ {v(x)=x^{-4}} $, and then using the product rule.}
}
\newpage
\reg[\lhop \label{Lhop}]{
	Given two differentiable functions $ f(x) $ and $ g(x) $, where
	\[ f(a)=g(a)=0 \]
	Then,
	\nn{
		\lim\limits_{x\to a} \frac{f}{g}=\lim\limits_{x\to a}\frac{f'}{g'}\label{Lhopeq}
	}
}
\eks[]{
	Find the limit of $ \lim\limits_{x\to 0} \dfrac{e^x-1}{x}$.
	
	\sv
	We set $ {f(x)=e^x-1} $ and $ {g(x)=x} $, noting that $ {f(0)=g(0)=0} $. Therefore,
	\alg{
		\lim\limits_{x\to 0} \frac{e^x-1}{x}&=\lim\limits_{x\to0}\frac{f}{g}
		\br
		&=\lim\limits_{x\to0} \frac{f'}{g'}\br
		&=\lim\limits_{x\to0}\frac{e^x}{1} \\
		&= 1
	} 
}

\section*{Explanations}

\fork{\ref{kjerne} \kjernereg}{
	Let's consider three functions $ f $, $ g $, and $ u $, where\footnote{The square brackets $ \sym{$ [\ ] $} $ in this context have the same meaning as ordinary parentheses, they are just used to make the expressions cleaner.}
	\[ f(x)= g\left[u(x)\right] \]
	$f$ is directly described by $x$, while $g$ is indirectly described by $x$, via $u(x)$. \vsk
	
	Let's use $f(x) =e^{x^2} $ as an example. If we know the value of \textit{x}, we can easily calculate the value of $ f(x) $. For instance,
	\[ f(3)=e^{3^2}=e^9 \]
	But we can also write $ {g\left[u(x)\right]=e^{u(x)}} $, where $ u(x)=x^2 $. This notation implies that when we know the value of \textit{x}, we first calculate the value of \textit{u}, then find the value of $g$:
	\[ u(3) =3^2=9\qquad,\qquad 
	g\left[u(3)\right]=e^{u(3)}=e^9 \]
	From \eqref{derdefaeq}, we have:
	\alg{
		f'(x)&=\lim\limits_{h \to 0}\frac{f(x+h)-f(x)}{h} \br
		&= \lim\limits_{h \to 0}\frac{g\left[u(x+h)\right]-g\left[u(x)\right]}{h}
	}
	We set $ k = u(x+h)-u(x) $. Thus,
	\[ \lim\limits_{h \to 0}\frac{g\left[u(x+h)\right]-g\left[u(x)\right]}{h}= \lim\limits_{h to 0}\frac{g(u+k)-g(u)}{h} \]
	From \eqref{linfunkeq}, we have:
	\[ g(u+k)-g(u)= g'(u)k+\varepsilon_g  \]
	Thus,
	\alg{
		\lim\limits_{h \to 0}\frac{g(u+k)-g(u)}{h}& = \lim\limits_{h\to 0}\frac{g'(u)k+\varepsilon_g}{h}\br
		&=\lim\limits_{h\to 0}\left(g'(u)+\frac{\varepsilon_g}{k}\right)\frac{k}{h}
	}
	Since $ {\lim\limits_{h\to0}k=0} $, $ {\lim\limits_{h\to0}\frac{\varepsilon_g}{k}=0} $. Moreover, $ {\lim\limits_{h\to0}\frac{k}{h}=u'(x)} $. Thus,
	\nn{
		\lim\limits_{h\to 0}\left(g'(u)+\frac{\varepsilon_g}{k}\right)\frac{k}{h}=g'(u)u'(x)
	}
} \vsk

\fork{\ref{prodregder} \prodregder}{
	Given the functions $ f(x) $, $ u(x) $, and $ v(x) $, where
	\[ f=uv \] 
	From \eqref{derdefa}, then
	\[ f'=\lim\limits_{h\to 0}\frac{u(x+h)v(x+h)-uv}{h} \]
	Let's denote $ u(x+h) $ and $ v(x+h) $ as respectively $ \tilde{u} $ and $ \tilde{v} $:
	\alg{
		f' &= \lim\limits_{h\to 0} \frac{\tilde{u}\tilde{v}- u v}{h}
	}
	We can always add 0 in the form of $ \dfrac{u\tilde{v}}{h}- \dfrac{u\tilde{v}}{h} $:
	\alg{f'&= \lim\limits_{h\to 0} \left[\frac{\tilde{u}\tilde{v}- u v}{h} + \frac{u\tilde{v}}{h}- \frac{u\tilde{v}}{h}\right] \\
		&= \lim\limits_{h\to 0}\left[\frac{(\tilde{u}- u)\tilde{v}}{h} +\frac{u(\tilde{v}- v)}{h} \right]
	}
	Since for any continuous function $ g $, $ {\lim\limits_{h\to 0} \tilde{g}=g} $ and \\$ {\lim\limits_{h\to 0}  \frac{g(x+h)-g(x)}{h}=g'}$, it is:
	\[ f'=u'v+uv' \]
}

\newpage
\fork{\ref{derel} \derel}{
	\textbf{Equation (\ref{derxr})}\\
	Let's start by noting that
	\alg{(\ln x^r)'&=(r \ln x)' \\
		&= \frac{r}{x}
	}  
	We set $ u=x^r $. From the chain rule, we have:
	\alg{
		\frac{r}{x}&=	(\ln u)'\\
		&=\frac{1}{u} u' \\
		&=\frac{1}{x^r}(x^r)' 
	}
	Thus,
	\[ 	(x^r)'= \frac{r}{x}x^r = r x^{r-1} \]
	
	\textbf{Equation (\ref{derlnx})}\\	
	We have that $ {x=e^{\ln x}} $. We set $ {u=\ln x} $ and $ g(u)=e^u $. Then $ x=g(u) $, and
	\alg{
		g'(u) &= e^u=e^{\ln x}=x \br
		u'(x) &= \left(\ln x\right)'
	}
	From the chain rule, we have:
	\alg{
		\left(x\right)' &= g'(u)u'(x)\\
		&= x \left(\ln x\right)'
	}
	Since\footnote{See exercise ??.}  ${\left(x\right)'=1} $, we have:
	\[ 1 = x\left(\ln x \right)' \]
	Thus,
	\[ \left(\ln x\right)'= \frac{1}{x} \]
} 

\fork{\ref{divder} \divder}{
	We have that
	\[ f'=\left(\dfrac{u}{v} \right)'=\left(u v^{-1}\right)' \]
	From \refunnbr{prodregder}{product rule} and \refunnbr{kjerne}{chain rule}, then
	\alg{
		f'	&= u' v^{-1} - u v^{-2}v' \\
		&= \frac{u'v - u v'}{v^2}
	}
} \vsk

\fork{\ref{Lhop} \lhop}{
	Since $ f(a)=g(a)=0 $, it is:
	\alg{
		\lim\limits_{x\to a}\frac{f(x)}{g(x)}=\lim\limits_{x\to a} \frac{f(x)-f(a)}{g(x)-g(a)}=\frac{\dfrac{f(x)-f(a)}{x-a}}{\dfrac{g(x)-g(a)}{x-a}}
	}
	From \eqref{derdefalt}, then
	\[ \lim\limits_{x\to a}\frac{f(x)}{g(x)}=\lim\limits_{x\to a}\frac{f'(x)}{g'(x)} \]
}

\end{document}