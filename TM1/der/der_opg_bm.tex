\documentclass[english,hidelinks,pdftex, 11 pt, class=report,crop=false]{standalone}
\usepackage[T1]{fontenc}
\usepackage[utf8]{luainputenc}
\usepackage{lmodern} % load a font with all the characters
\usepackage{geometry}
\geometry{verbose,a4paper, inner=0cm, outer=0 cm, bmargin=2cm, tmargin=1cm}
%\textwidth=12cm
\setlength{\parindent}{0bp}
\usepackage{import}
\usepackage[subpreambles=false]{standalone}
\usepackage{amsmath}
\usepackage{amssymb}
\usepackage{esint}
\usepackage{babel}
\usepackage{tabu}
\usepackage[dvipsnames, table]{xcolor}
\usepackage{cancel}
\makeatother
\makeatletter
\usepackage{datetime2}
\usepackage{titlesec}
\usepackage[many]{tcolorbox}

% Eheter
\newcommand{\enh}[1]{\,\textrm{#1}}
%referances
\newcommand{\net}[2]{{\color{blue}\href{#1}{#2}}}

%Spaces
\newcommand{\vsk}{\\[12pt]}
\newcommand{\vs}{\vspace{-12pt}}

% Tabell for opplegg

\newcommand{\ovlist}[1]{
\vspace{-16pt}
\begin{itemize}
	#1
\end{itemize}
}

% Chapters and sections
\titleformat{\section}[block]{\bfseries}{\hspace{3cm}\thesection}{5pt}{}
\titleformat{\subsection}[block]{\bfseries}{\hspace{3cm}\thesection}{5pt}{}
\newcommand{\sectionbreak}{\clearpage} % New page on each section
 

\newlength{\mywidth}
\setlength{\mywidth}{14cm}

\newcommand{\cont}[1]{
\begin{tcolorbox}[center, boxrule=0.0 mm, width=\mywidth,arc=0mm,enhanced jigsaw,,colback=white,breakable]
#1	
\end{tcolorbox}
}

\newcommand{\info}[5]{
\begin{tcolorbox}[center, boxrule=0.1 mm, width=\mywidth,arc=0mm,enhanced jigsaw,breakable,colback=yellow!5]	
	
	\footnotesize
	\textbf{Øvingsområde}\\[5pt] #1 
	
	\textbf{Utstyr}\\ #2  \\
	
	\begin{tabular}{@{} p{4cm} p{4cm} l} 
		\textbf{Tid} & \textbf{Elevinndeling} & \textbf{Læringsarena} \\
		#3  & #4 & #5
	\end{tabular} 
\end{tcolorbox}	
}

\newcommand{\gjen}[1]{\begin{tcolorbox}[center,boxrule=0.1 mm, width=\mywidth,arc=0mm,colback=blue!3] {\large \textbf{Gjennomføring} \vspace{5 pt}}\newline #1  \end{tcolorbox}\vspace{-5pt}}
\newcommand{\eks}[1]{\begin{tcolorbox}[center,boxrule=0.1 mm, width=\mywidth,arc=0mm,colback=green!3] {\large \textbf{Eksempel} \vspace{5 pt}}\newline #1  \end{tcolorbox}\vspace{-5pt}}

\newcounter{opl}
%\numberwithin{opl}{article}


\newcommand{\opl}[1]{
\newpage
{\refstepcounter{opl} %\phantomsection 
\large \textbf{\theopl \;#1} \vsk}
}

% Headlines
\newcommand{\fork}{\textbf{Forkunnskapar}\\}
\newcommand{\forb}{\textbf{Forberedelsar}\\}
\newcommand{\opgvr}{\textbf{Oppgaver}}



%colors
\newcommand{\colr}[1]{{\color{red} #1}}
\newcommand{\colb}[1]{{\color{blue} #1}}
\newcommand{\colo}[1]{{\color{orange} #1}}
\newcommand{\colc}[1]{{\color{cyan} #1}}
\definecolor{projectgreen}{cmyk}{100,0,100,0}
\newcommand{\colg}[1]{{\color{projectgreen} #1}}

% Lister med bokstavar
\usepackage[inline]{enumitem}
% Opg
\newcommand{\abc}[1]{
	\begin{enumerate}[label=\alph*),leftmargin=18pt]
		#1
	\end{enumerate}
}

\usepackage[]{hyperref}

\newcommand{\note}{Merk}
\newcommand{\notesm}[1]{{\footnotesize \textsl{\note:} #1}}
\newcommand{\ekstitle}{Eksempel }
\newcommand{\sprtitle}{Språkboksen}
\newcommand{\expl}{forklaring}
\newcommand{\pyt}{Pytagoras' setning}
\newcommand\sv{\vsk \textbf{Svar} \vspace{4 pt}\\}

%references
\newcommand{\reftab}[1]{\hrs{#1}{tabell}}
\newcommand{\rref}[1]{\hrs{#1}{regel}}
\newcommand{\dref}[1]{\hrs{#1}{definisjon}}
\newcommand{\refkap}[1]{\hrs{#1}{kapittel}}
\newcommand{\refsec}[1]{\hrs{#1}{seksjon}}
\newcommand{\refdsec}[1]{\hrs{#1}{delseksjon}}
\newcommand{\refved}[1]{\hrs{#1}{vedlegg}}
\newcommand{\eksref}[1]{\textsl{#1}}
\newcommand\fref[2][]{\hyperref[#2]{\textsl{figur \ref*{#2}#1}}}
\newcommand{\refop}[1]{{\color{blue}Oppgave \ref{#1}}}
\newcommand{\refops}[1]{{\color{blue}oppgave \ref{#1}}}


%Algebra
\newcommand{\kvadset}{Kvadratsetningene}
\newcommand{\aenato}{Sum-produkt-metoden}

% Geometry
\newcommand{\hlikb}{Midtnormalen i en likebeint trekant}
\newcommand{\arealsetn}{Arealsetningen}
\newcommand{\trkmedian}{Median}
\newcommand{\midtrk}{Midtnormal (i trekant)}
\newcommand{\innskrsirk}{Innskrevet sirkel}
\newcommand{\cossetn}{Cosinussetningen}
\newcommand{\perfvink}{Sentral- og periferivinkel}
\newcommand{\tang}{Tangent}

% Derivative
\newcommand{\derel}{Den deriverte av elementære funksjoner}
\newcommand{\divder}{Divisjonsregelen}
\newcommand{\kjernereg}{Kjerneregelen}
\newcommand{\prodregder}{Produktregelen}
\newcommand{\lhop}{L'Hopitals regel}

% Funksjonsdrofting
\newcommand{\monder}{Monotoniegenskaper og den deriverte}
\newcommand{\fderekstr}{$ \bm{f'=0} $ for lokale ektstremalpunkt}
\newcommand{\andredertest}{Andrederiverttesten}

% Vectors
\newcommand{\detar}{Arealformler med determinanter}
\newcommand{\avstpunktlin}{Avstand mellom punkt og linje}

%Appendix
\newcommand{\rolle}{Rolles teorem}
\newcommand{\meanval}{Middelverdisetningen}

% Solutions manual
\newcommand{\selos}{Se løsningsforslag.}


\begin{document}
\opgt

\op{opgder1overx}
Bruk definisjonen av den deriverte til å vise at for funksjonen $f(x)=\frac{1}{x} $ er $ f'(x)=-\frac{1}{x^2} $.

\op{opgderxn} \vs
\abc{
	\item Bruk definisjonen av den deriverte til å finne den deriverte funksjonen til henholdsvis $ {f_1(x)=x} $, ${f_2(x)=x^2} $, ${f_3(x)= x^3} $.
	\item La $ f_n(x)=x^n $ for $ n\in \mathbb{N} $. Bruk det du fant i oppgave a) til å foreslå et uttrykk for $ f_n'(x)=x^n $.
}
\nes
\op{opgderpoly}
Deriver uttrykkene\os
\abch{
\item $ 5x^3 $
\item $ -8x^{6} $
\item $ \frac{3}{7}x^7 $
\item $ -x^{\frac{2}{3}} $
\item $ x^{\frac{9}{7}} $
}

\op{dereogln}
Deriver uttrykkene \os
\abch{
\item $ 2e^x $
\item $ -30e^x $
\item $ 8\ln x $
\item $ -4\ln x $
}

\op{opgderxr}
Forklar hvordan du kan omskrive uttrykk på formen $\frac{1}{x^k} $ slik at du kan anvende \eqref{derxr} til å derivere uttrykkene.

\op{opgderxr2}
Deriver uttrykkene (Hint: Se \refops{opgderxr})\os
\abch{
\item $ \dfrac{5}{x^2} $
\item $ \dfrac{7}{x^{10}} $
\item $ -\dfrac{2}{9x^{7}} $
\item $ \dfrac{3}{11x^{\frac{8}{5}}} $
}

\op{opgderruleselmnt}
Deriver funksjonene \os
\abch{
	\item $ g(x)= 3x^3-4x+\frac{1}{x} $
	\item $ f(x) = x^2 +\ln x $
	\item $ h(x)= \ln x+x^2+2 $
	\item $ a(x)=x^2+e^x $
	\item $ p(x)=e^x+\ln x $
}
\newpage
\op{opgdermhpx}
Deriver uttrykkene med hensyn på $ x $. \os

\abch{
\item $ ax^2+bx+c $ 
\item $ 7x^5-3ax+b $
\item $ -9qx^7+3px^3+b^3 $
}





\op{opgderrules}
Deriver funksjonene \os
\abch{
\item $ f(x)=x\sqrt{1-2x} $
\item $ p(x)= 3xe^{2x} $
\item $ h(x)=3x^2\ln x $
\item $ k(x)=\sqrt{4x^2-5} $
\item $ f(x)=x^3\sqrt{2x-1} $
\item $ q(x)=\frac{x^3}{x^2-2} $
\item $ f(x)=\left(x^2+2\right)^7 $
\item $ h(x)=\frac{x}{e^{x^2}} $
}

\op{opgderlhopeks}
Løs \grubr{R1V23D1opg2} ved hjelp av L'Hopitals regel.

\newpage
\grubop{R1V22D1opg1}
(R1V22D1)\\
En funskjon $ f $ er gitt ved
\begin{equation*}
f(x)= \left\lbrace{
		\begin{array}{rcr}
			x^2+1 &,&x<2 \br
			x-t   &,& x\geq 2
		\end{array}
	}\right. 
\end{equation*}
\abc{
\item Bestem tallet $ t $ slik at $ f $ blir en kontinuerlig funksjon. Husk å grunngi svaret. \item Avgjør om $ f $ er deriverbar i $ x=2 $ for den verdien av $ t $ du fant i oppgave a).
}

\grubop{opgdersirkrekt}
Et rektangel er innskrevet i en sirkel. Vis at rektangelets areal er størst hvis det er et kvadrat.

\grubop{t1h23d1opg3} (T1H23D1)\os
Funksjonen $ f $ er gitt ved
\[ f(x)=x^3-3x^2-x+4\]
Bestem ligningen for tangenten til $ f $ i punktet $ (1, f(1) $.

\grubop{opgbevisdivreg}
Bevis at \eqref{divdereq} er gyldig.

\grubop{opgbevisapow(x)der}
Bevis at $ \left(a^x\right)'= a^x\ln a $.

\newpage
\grubop{opgderkont}
\abc{
\item Vi at
\st{
	Hvis den deriverte funksjonen til $ f(x) $ er kontinuerlig for $ x\in[a, b] $, er $ f $ kontinuerlig for $ x\in(a, b) $.
}
Hint: Bruk \eqref{derdefalt}.
\item Bruk resultatet fra oppgave a) til å forklare at alle polynomfunskjoner er kontinuerlige for alle $ x $.
}

\grubop{opgder2dx}
Gitt at $ f'(x) $ er kontinuerlig, vis at
\[ \lim\limits_{h\to0}\frac{f(x+h)-f(x-h)}{h}=2f'(x) \]

\end{document}