\documentclass[english,hidelinks,pdftex, 11 pt, class=report,crop=false]{standalone}
\usepackage[T1]{fontenc}
\usepackage[utf8]{luainputenc}
\usepackage{lmodern} % load a font with all the characters
\usepackage{geometry}
\geometry{verbose,a4paper, inner=0cm, outer=0 cm, bmargin=2cm, tmargin=1cm}
%\textwidth=12cm
\setlength{\parindent}{0bp}
\usepackage{import}
\usepackage[subpreambles=false]{standalone}
\usepackage{amsmath}
\usepackage{amssymb}
\usepackage{esint}
\usepackage{babel}
\usepackage{tabu}
\usepackage[dvipsnames, table]{xcolor}
\usepackage{cancel}
\makeatother
\makeatletter
\usepackage{datetime2}
\usepackage{titlesec}
\usepackage[many]{tcolorbox}

% Eheter
\newcommand{\enh}[1]{\,\textrm{#1}}
%referances
\newcommand{\net}[2]{{\color{blue}\href{#1}{#2}}}

%Spaces
\newcommand{\vsk}{\\[12pt]}
\newcommand{\vs}{\vspace{-12pt}}

% Tabell for opplegg

\newcommand{\ovlist}[1]{
\vspace{-16pt}
\begin{itemize}
	#1
\end{itemize}
}

% Chapters and sections
\titleformat{\section}[block]{\bfseries}{\hspace{3cm}\thesection}{5pt}{}
\titleformat{\subsection}[block]{\bfseries}{\hspace{3cm}\thesection}{5pt}{}
\newcommand{\sectionbreak}{\clearpage} % New page on each section
 

\newlength{\mywidth}
\setlength{\mywidth}{14cm}

\newcommand{\cont}[1]{
\begin{tcolorbox}[center, boxrule=0.0 mm, width=\mywidth,arc=0mm,enhanced jigsaw,,colback=white,breakable]
#1	
\end{tcolorbox}
}

\newcommand{\info}[5]{
\begin{tcolorbox}[center, boxrule=0.1 mm, width=\mywidth,arc=0mm,enhanced jigsaw,breakable,colback=yellow!5]	
	
	\footnotesize
	\textbf{Øvingsområde}\\[5pt] #1 
	
	\textbf{Utstyr}\\ #2  \\
	
	\begin{tabular}{@{} p{4cm} p{4cm} l} 
		\textbf{Tid} & \textbf{Elevinndeling} & \textbf{Læringsarena} \\
		#3  & #4 & #5
	\end{tabular} 
\end{tcolorbox}	
}

\newcommand{\gjen}[1]{\begin{tcolorbox}[center,boxrule=0.1 mm, width=\mywidth,arc=0mm,colback=blue!3] {\large \textbf{Gjennomføring} \vspace{5 pt}}\newline #1  \end{tcolorbox}\vspace{-5pt}}
\newcommand{\eks}[1]{\begin{tcolorbox}[center,boxrule=0.1 mm, width=\mywidth,arc=0mm,colback=green!3] {\large \textbf{Eksempel} \vspace{5 pt}}\newline #1  \end{tcolorbox}\vspace{-5pt}}

\newcounter{opl}
%\numberwithin{opl}{article}


\newcommand{\opl}[1]{
\newpage
{\refstepcounter{opl} %\phantomsection 
\large \textbf{\theopl \;#1} \vsk}
}

% Headlines
\newcommand{\fork}{\textbf{Forkunnskapar}\\}
\newcommand{\forb}{\textbf{Forberedelsar}\\}
\newcommand{\opgvr}{\textbf{Oppgaver}}



%colors
\newcommand{\colr}[1]{{\color{red} #1}}
\newcommand{\colb}[1]{{\color{blue} #1}}
\newcommand{\colo}[1]{{\color{orange} #1}}
\newcommand{\colc}[1]{{\color{cyan} #1}}
\definecolor{projectgreen}{cmyk}{100,0,100,0}
\newcommand{\colg}[1]{{\color{projectgreen} #1}}

% Lister med bokstavar
\usepackage[inline]{enumitem}
% Opg
\newcommand{\abc}[1]{
	\begin{enumerate}[label=\alph*),leftmargin=18pt]
		#1
	\end{enumerate}
}

\usepackage[]{hyperref}

\newcommand{\note}{Merk}
\newcommand{\notesm}[1]{{\footnotesize \textsl{\note:} #1}}
\newcommand{\ekstitle}{Eksempel }
\newcommand{\sprtitle}{Språkboksen}
\newcommand{\expl}{forklaring}
\newcommand{\pyt}{Pytagoras' setning}
\newcommand\sv{\vsk \textbf{Svar} \vspace{4 pt}\\}

%references
\newcommand{\reftab}[1]{\hrs{#1}{tabell}}
\newcommand{\rref}[1]{\hrs{#1}{regel}}
\newcommand{\dref}[1]{\hrs{#1}{definisjon}}
\newcommand{\refkap}[1]{\hrs{#1}{kapittel}}
\newcommand{\refsec}[1]{\hrs{#1}{seksjon}}
\newcommand{\refdsec}[1]{\hrs{#1}{delseksjon}}
\newcommand{\refved}[1]{\hrs{#1}{vedlegg}}
\newcommand{\eksref}[1]{\textsl{#1}}
\newcommand\fref[2][]{\hyperref[#2]{\textsl{figur \ref*{#2}#1}}}
\newcommand{\refop}[1]{{\color{blue}Oppgave \ref{#1}}}
\newcommand{\refops}[1]{{\color{blue}oppgave \ref{#1}}}


%Algebra
\newcommand{\kvadset}{Kvadratsetningene}
\newcommand{\aenato}{Sum-produkt-metoden}

% Geometry
\newcommand{\hlikb}{Midtnormalen i en likebeint trekant}
\newcommand{\arealsetn}{Arealsetningen}
\newcommand{\trkmedian}{Median}
\newcommand{\midtrk}{Midtnormal (i trekant)}
\newcommand{\innskrsirk}{Innskrevet sirkel}
\newcommand{\cossetn}{Cosinussetningen}
\newcommand{\perfvink}{Sentral- og periferivinkel}
\newcommand{\tang}{Tangent}

% Derivative
\newcommand{\derel}{Den deriverte av elementære funksjoner}
\newcommand{\divder}{Divisjonsregelen}
\newcommand{\kjernereg}{Kjerneregelen}
\newcommand{\prodregder}{Produktregelen}
\newcommand{\lhop}{L'Hopitals regel}

% Funksjonsdrofting
\newcommand{\monder}{Monotoniegenskaper og den deriverte}
\newcommand{\fderekstr}{$ \bm{f'=0} $ for lokale ektstremalpunkt}
\newcommand{\andredertest}{Andrederiverttesten}

% Vectors
\newcommand{\detar}{Arealformler med determinanter}
\newcommand{\avstpunktlin}{Avstand mellom punkt og linje}

%Appendix
\newcommand{\rolle}{Rolles teorem}
\newcommand{\meanval}{Middelverdisetningen}

% Solutions manual
\newcommand{\selos}{Se løsningsforslag.}

\begin{document}
\opgt

\op{opggeosinvmin180}
Gitt $ v\in [0^\circ, 90^\circ] $.
\abc{
\item Vis at $ \sin v = \sin(180^\circ-v) $.
\item Vis at $ \cos v = -\cos(180^\circ-v) $
}

\op{opggeoarsetn}
Finn arealet til $ \triangle ABC $ når
\abc{
\item $ \angle A=60^\circ $, $ AB=5 $ og $ AC=7 $.
\item $ \angle B=18^\circ $, $ AB=4 $ og $ BC=3 $. $ \left(\sin 18^\circ = \frac{\sqrt{5}-1}{4}\right) $
\item $ \angle A= 75^\circ $, $ \angle B=60^\circ $, $ AC=\sqrt{6} $ og $ BC=\sqrt{3}+1 $
}

\op{opggeoarealsetn}
\abch{
\item Bevis arealsetningen.
\item Bevis sinussetningen.
}

\op{opggeoviscos}
\prbxl{0.5}{\abc{
		\item Vis at $ \cos 45^\circ=\frac{\sqrt{2}}{2} $.
		\item Vis at $ \sin 30^\circ=\frac{1}{2} $.
		\item Vis at $ \cos 30^\circ=\frac{\sqrt{3}}{2} $.
}}
\fgbxr{0.3}{\fig{opggeo1}}

\newpage
\eksop{1TH22D1}{1TH22D1opg1} 
\prbxl{0.5}{Gitt trekanten til høgre. Vis at
	\[ \frac{\sin u}{\cos u}=\tan u \]
\vspace{35pt}
}
\fgbxr{0.5}{\fig{1th22opg1}}



\op{opggeovistan}
Vis at $ \tan v = \dfrac{\sin v}{\cos v}  $.

\nes

\eksop{1TH21D1}{1TH21D1opg8}
\fig{eks1th21opg8}
Gitt trekanten over. Bestem lengden til siden $ BC $.

\op{opggeoabcr}
Gitt en trekant med sidelengder $ a $, $ b $ og $ c $ og innskrevet sirkel med radius $ r $. Forklar hvorfor arealet til trekanten er gitt som 
\[ \frac{1}{2}(a+b+c)r \]
\newpage
\op{opggeo2r}
La $ a=BC $, $ b=AC $, $ c=AB $ og $ DM=r $.
\abc{
	\item Vis at $ r= \frac{ac}{a+b+c}$.
	\item Vis at $ 2r=a+c-b $.
	\item Bruk uttrykkene fra oppgave a) og b) til å finne $ b^2 $ uttrykt ved $ a $ og $ c $. Hva kalles denne formelen?
}
\fig{geoopg2r}


\op{opggeotanchord}
Den røde linja tangerer sirkelen. Vis at $ \angle BAC=\angle EBC $.
\fig{geoopgtanchord}
\newpage

\grubop{1TV21D1opg2}
(1TV21D1)\\
Sorter verdiene i stigende rekkefølge.
\[ \sin 60^\circ\qquad\qquad \left(\frac{3}{4}\right)^{-1}\qquad\qquad\sin 160^\circ\qquad\qquad \lg 1\]

\grubop{1TH21D1opg9} 
(1TH21D1) \\
En trekant har omkrets 12, og den éne siden i trekanten har lengde 2. Bestem arealet til trekanten.

\grubop{1}
Bevis cosinussetningen.

\grubop{2}
Vis at $ \sin 18^\circ=\frac{1}{4}(\sqrt{5}-1) $. (Hint: Se figur.)
\fig{opggeo2}

\grubop{3}
Vis at
\[ \cos(u+v)= \cos u\cos v-\sin u \sin v \]
Det er tilstrekkelig å undersøke tilfellet hvor $ v,u \in [0^\circ, 90^\circ] $.
\end{document}


