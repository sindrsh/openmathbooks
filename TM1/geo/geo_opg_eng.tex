\documentclass[english,hidelinks,pdftex, 11 pt, class=report,crop=false]{standalone}
\usepackage[T1]{fontenc}
\usepackage[utf8]{luainputenc}
\usepackage{lmodern} % load a font with all the characters
\usepackage{geometry}
\geometry{verbose,a4paper, inner=0cm, outer=0 cm, bmargin=2cm, tmargin=1cm}
%\textwidth=12cm
\setlength{\parindent}{0bp}
\usepackage{import}
\usepackage[subpreambles=false]{standalone}
\usepackage{amsmath}
\usepackage{amssymb}
\usepackage{esint}
\usepackage{babel}
\usepackage{tabu}
\usepackage[dvipsnames, table]{xcolor}
\usepackage{cancel}
\makeatother
\makeatletter
\usepackage{datetime2}
\usepackage{titlesec}
\usepackage[many]{tcolorbox}

% Eheter
\newcommand{\enh}[1]{\,\textrm{#1}}
%referances
\newcommand{\net}[2]{{\color{blue}\href{#1}{#2}}}

%Spaces
\newcommand{\vsk}{\\[12pt]}
\newcommand{\vs}{\vspace{-12pt}}

% Tabell for opplegg

\newcommand{\ovlist}[1]{
\vspace{-16pt}
\begin{itemize}
	#1
\end{itemize}
}

% Chapters and sections
\titleformat{\section}[block]{\bfseries}{\hspace{3cm}\thesection}{5pt}{}
\titleformat{\subsection}[block]{\bfseries}{\hspace{3cm}\thesection}{5pt}{}
\newcommand{\sectionbreak}{\clearpage} % New page on each section
 

\newlength{\mywidth}
\setlength{\mywidth}{14cm}

\newcommand{\cont}[1]{
\begin{tcolorbox}[center, boxrule=0.0 mm, width=\mywidth,arc=0mm,enhanced jigsaw,,colback=white,breakable]
#1	
\end{tcolorbox}
}

\newcommand{\info}[5]{
\begin{tcolorbox}[center, boxrule=0.1 mm, width=\mywidth,arc=0mm,enhanced jigsaw,breakable,colback=yellow!5]	
	
	\footnotesize
	\textbf{Øvingsområde}\\[5pt] #1 
	
	\textbf{Utstyr}\\ #2  \\
	
	\begin{tabular}{@{} p{4cm} p{4cm} l} 
		\textbf{Tid} & \textbf{Elevinndeling} & \textbf{Læringsarena} \\
		#3  & #4 & #5
	\end{tabular} 
\end{tcolorbox}	
}

\newcommand{\gjen}[1]{\begin{tcolorbox}[center,boxrule=0.1 mm, width=\mywidth,arc=0mm,colback=blue!3] {\large \textbf{Gjennomføring} \vspace{5 pt}}\newline #1  \end{tcolorbox}\vspace{-5pt}}
\newcommand{\eks}[1]{\begin{tcolorbox}[center,boxrule=0.1 mm, width=\mywidth,arc=0mm,colback=green!3] {\large \textbf{Eksempel} \vspace{5 pt}}\newline #1  \end{tcolorbox}\vspace{-5pt}}

\newcounter{opl}
%\numberwithin{opl}{article}


\newcommand{\opl}[1]{
\newpage
{\refstepcounter{opl} %\phantomsection 
\large \textbf{\theopl \;#1} \vsk}
}

% Headlines
\newcommand{\fork}{\textbf{Forkunnskapar}\\}
\newcommand{\forb}{\textbf{Forberedelsar}\\}
\newcommand{\opgvr}{\textbf{Oppgaver}}



%colors
\newcommand{\colr}[1]{{\color{red} #1}}
\newcommand{\colb}[1]{{\color{blue} #1}}
\newcommand{\colo}[1]{{\color{orange} #1}}
\newcommand{\colc}[1]{{\color{cyan} #1}}
\definecolor{projectgreen}{cmyk}{100,0,100,0}
\newcommand{\colg}[1]{{\color{projectgreen} #1}}

% Lister med bokstavar
\usepackage[inline]{enumitem}
% Opg
\newcommand{\abc}[1]{
	\begin{enumerate}[label=\alph*),leftmargin=18pt]
		#1
	\end{enumerate}
}

\usepackage[]{hyperref}

% note
\newcommand{\note}{Note}
\newcommand{\notesm}[1]{{\footnotesize \textsl{\note:} #1}}
\newcommand{\selos}{See the solutions manual.}

\newcommand{\texandasy}{The text is written in \LaTeX\ and the figures are made with the aid of Asymptote.}

\newcommand{\rknut}{Calculate.}
\newcommand\sv{\vsk \textbf{Answer} \vspace{4 pt}\\}
\newcommand{\ekstitle}{Example }
\newcommand{\sprtitle}{The language box}
\newcommand{\expl}{explanation}

% answers
\newcommand{\mulansw}{\notesm{Multiple possible answers.}}	
\newcommand{\faskap}{Chapter}

% exercises
\newcommand{\opgt}{\newpage \phantomsection \addcontentsline{toc}{section}{Exercises} \section*{Exercises for Chapter \thechapter}\vs \setcounter{section}{1}}

% references
\newcommand{\reftab}[1]{\hrs{#1}{Table}}
\newcommand{\rref}[1]{\hrs{#1}{Rule}}
\newcommand{\dref}[1]{\hrs{#1}{Definition}}
\newcommand{\refkap}[1]{\hrs{#1}{Chapter}}
\newcommand{\refsec}[1]{\hrs{#1}{Section}}
\newcommand{\refdsec}[1]{\hrs{#1}{Subsection}}
\newcommand{\refved}[1]{\hrs{#1}{Appendix}}
\newcommand{\eksref}[1]{\textsl{#1}}
\newcommand\fref[2][]{\hyperref[#2]{\textsl{Figure \ref*{#2}#1}}}
\newcommand{\refop}[1]{{\color{blue}Exercise \ref{#1}}}
\newcommand{\refops}[1]{{\color{blue}Exercise \ref{#1}}}

%%% SECTION HEADLINES %%%

% Our numbers
\newcommand{\likteikn}{The equal sign}
\newcommand{\talsifverd}{Numbers, digits and values}
\newcommand{\koordsys}{Coordinate systems}

% Calculations
\newcommand{\adi}{Addition}
\newcommand{\sub}{Subtraction}
\newcommand{\gong}{Multiplication}
\newcommand{\del}{Division}

%Factorization and order of operations
\newcommand{\fak}{Factorization}
\newcommand{\rrek}{Order of operations}

%Fractions
\newcommand{\brgrpr}{Introduction}
\newcommand{\brvu}{Values, expanding and simplifying}
\newcommand{\bradsub}{Addition and subtraction}
\newcommand{\brgngheil}{Fractions multiplied by integers}
\newcommand{\brdelheil}{Fractions divided by integers}
\newcommand{\brgngbr}{Fractions multiplied by fractions}
\newcommand{\brkans}{Cancelation of fractions}
\newcommand{\brdelmbr}{Division by fractions}
\newcommand{\Rasjtal}{Rational numbers}

%Negative numbers
\newcommand{\negintro}{Introduction}
\newcommand{\negrekn}{The elementary operations}
\newcommand{\negmeng}{Negative numbers as amounts}

%Calculation methods
\newcommand{\delmedtihundre}{Deling med 10, 100, 1\,000 osv.}

% Geometry 1
\newcommand{\omgr}{Terms}
\newcommand{\eignsk}{Attributes of triangles and quadrilaterals}
\newcommand{\omkr}{Perimeter}
\newcommand{\area}{Area}

%Algebra 
\newcommand{\algintro}{Introduction}
\newcommand{\pot}{Powers}
\newcommand{\irrasj}{Irrational numbers}

%Equations
\newcommand{\ligintro}{Introduction}
\newcommand{\liglos}{Solving with the elementary operations}
\newcommand{\ligloso}{Solving with elementary operations summarized}

%Functions
\newcommand{\fintro}{Introduction}
\newcommand{\lingraf}{Linear functions and graphs}

%Geometry 2
\newcommand{\geoform}{Formulas of area and perimeter}
\newcommand{\kongogsim}{Congruent and similar triangles}
\newcommand{\geofork}{Explanations}

% Names of rules
\newcommand{\adkom}{Addition is commutative}
\newcommand{\gangkom}{Multiplication is commutative}
\newcommand{\brdef}{Fractions as rewriting of division}
\newcommand{\brtbr}{Fractions multiplied by fractions}
\newcommand{\delmbr}{Fractions divided by fractions}
\newcommand{\gangpar}{Distributive law}
\newcommand{\gangparsam}{Paranthesis multiplied together}
\newcommand{\gangmnegto}{Multiplication by negative numbers I}
\newcommand{\gangmnegtre}{Multiplication by negative numbers II}
\newcommand{\konsttre}{Unique construction of triangles}
\newcommand{\kongtre}{Congruent triangles}
\newcommand{\topv}{Vertical angles}
\newcommand{\trisum}{The sum of angles in a triangle}
\newcommand{\firsum}{The sum of angles in a quadrilateral}
\newcommand{\potgang}{Multiplication by powers}
\newcommand{\potdivpot}{Division by powers}
\newcommand{\potanull}{The special case of \boldmath $a^0$}
\newcommand{\potneg}{Powers with negative exponents}
\newcommand{\potbr}{Fractions as base}
\newcommand{\faktgr}{Factors as base}
\newcommand{\potsomgrunn}{Powers as base}
\newcommand{\arsirk}{The area of a circle}
\newcommand{\artrap}{The area of a trapezoid}
\newcommand{\arpar}{The area of a parallelogram}
\newcommand{\pyt}{Pythagoras's theorem}
\newcommand{\forform}{Ratios in similar triangles}
\newcommand{\vilkform}{Terms of similar triangles}
\newcommand{\omkrsirk}{The perimeter of a circle (and the value of $ \bm \pi $)}
\newcommand{\artri}{The area of a triangle}
\newcommand{\arrekt}{The area of a rectangle}
\newcommand{\liknflyt}{Moving terms across the equal sign}
\newcommand{\funklin}{Linear functions}



\begin{document}
\opgt

\op{opggeosinvmin180}
Given $ v\in [0^\circ, 90^\circ] $.
\abc{
	\item Show that $ \sin v = \sin(180^\circ-v) $.
	\item Show that $ \cos v = -\cos(180^\circ-v) $
}

\op{opggeoarsetn}
Find the area of $ \triangle ABC $ when
\abc{
	\item $ \angle A=60^\circ $, $ AB=5 $ and $ AC=7 $.
	\item $ \angle B=18^\circ $, $ AB=4 $ and $ BC=3 $. $ \left(\sin 18^\circ = \frac{\sqrt{5}-1}{4}\right) $
	\item $ \angle A= 75^\circ $, $ \angle B=60^\circ $, $ AC=\sqrt{6} $ and $ BC=\sqrt{3}+1 $
}

\op{opggeoarealsetn}
\abch{
	\item Prove the area theorem.
	\item Prove the sine theorem.
}

\op{opggeoviscos}
\prbxl{0.5}{\abc{
		\item Show that $ \cos 45^\circ=\frac{\sqrt{2}}{2} $.
		\item Show that $ \sin 30^\circ=\frac{1}{2} $.
		\item Show that $ \cos 30^\circ=\frac{\sqrt{3}}{2} $.
}}
\fgbxr{0.3}{\fig{opggeo1}}

\newpage
\eksop{1TV23D1}{1TV23D1opg1} \vs
\prbxl{0.5}{
	A right triangle has sides 6, 8, and 10. See the figure to the right. \os
	
	Show that
	\[ (\sin u)^2+(\cos u)^2 =1 \]
}
\fgbxr{0.5}{\fig{1tv23d1opg1}}

\eksop{1TH22D1}{1TH22D1opg1} 
\prbxl{0.5}{Given the triangle to the right. Show that
	\[ \frac{\sin u}{\cos u}=\tan u \]
	\vspace{35pt}
}
\fgbxr{0.5}{\fig{1th22opg1}}



\op{opggeovistan}
Show that $ \tan v = \dfrac{\sin v}{\cos v}  $.

\nes

\eksop{1TH21D1}{1TH21D1opg8}
\fig{eks1th21opg8}
Given the triangle above. Determine the length of side $ BC $.

\op{opggeoabcr}
Given a triangle with sides $ a $, $ b $, and $ c $ and an inscribed circle with radius $ r $. Explain why the area of the triangle is given as 
\[ \frac{1}{2}(a+b+c)r \]
\newpage
\op{opggeo2r}
Let $ a=BC $, $ b=AC $, $ c=AB $ and $ DM=r $.
\abc{
	\item Show that $ r= \frac{ac}{a+b+c}$.
	\item Show that $ 2r=a+c-b $.
	\item Use the expressions from tasks a) and b) to find $ b^2 $ expressed by $ a $ and $ c $. What is this formula called?
}
\fig{geoopg2r}

\op{opggeolikeperf}
Explain why, from \rref{perfvink}, it follows that two angles spanning the same arc are equal in size.
\fig{opggeolikeperf}

\op{opggeotalesTM} \vs
\abc{
	\item Show that Thales' theorem\footnote{See \mb.} follows from \rref{perfvink}.
	\item Given a right-angled triangle $ \triangle ABC $ with hypotenuse $ AB $.
	Which of  \sym{$ \Rightarrow $},  \sym{$ \Leftarrow $} and \sym{$ \iff $} should replace \sym{$ ??? $} below to describe the \textsl{inverse} case of Thales' theorem.
	{\small
		\[C=90^\circ \quad???\quad \text{$ AB $ is a diameter in the circumscribed circle of }\triangle ABC\]}
}

\newpage
\op{opggeotanchord}
The red line is tangent to the circle. Show that $ \angle BAC=\angle EBC $.
\fig{geoopgtanchord}

\newpage
\grubop{1TH21D1opg9} 
(1TH21D1) \\
A triangle has a perimeter of 12, and one side of the triangle is 2. Determine the area of the triangle.


\grubop{1TV21D1opg2}
(1TV21D1)\\
Sort the values in ascending order.
\[ \sin 60^\circ\qquad\qquad \left(\frac{3}{4}\right)^{-1}\qquad\qquad\sin 160^\circ\qquad\qquad \lg 1\]

\grubop{t1h23d1opg1}
An equilateral triangle has sides of length 2.
\fig{t1h23d1opg1}
Use the triangle to show that
\[ \cos 60^\circ = \frac{1}{2} \]

\grubop{t1h23d1opg4}
Which of the two triangles has the larger area? \os

Remember to argue why your answer is correct.
\fig{t1h23d1opg4}

\grubop{opggeolimsin0overx}
Show that 
\[\lim\limits_{x\to0} \frac{\sin x}{x}=1 \]


\grubop{opggeosin18}
Show that $ \sin 18^\circ=\frac{1}{4}(\sqrt{5}-1) $. (Hint: See figure.)
\fig{opggeosin18}


\grubop{opggeobeviscossetn}
Prove the cosine theorem.


\newpage
\grubop{opggeoviscosuv}
Show that
\[ \cos(u+v)= \cos u\cos v-\sin u \sin v \]
It is sufficient to examine the case where $ v,u \in [0^\circ, 90^\circ] $.

\grubop{opggeotalesTMconv}
Prove the converse case of Thales theorem (see \refops{opggeotalesTM} ).


\grubop{opggeo180minv}
Show that
\[ u=180^\circ-v \]
\fig{opggeo180minv}

\grubop{opggeokordteo}
Show that 
\[ AP\cdot PB = DP\cdot PC \]
\fig{opggeokordteo}
\mers{This result is often called \outl{the chord theorem}.}

\newpage
\grubop{opggeovinkar}
Let $ r $ be the radius of the semicircle. Express the area of the blue area in terms of $ v $ and $ r $.
\fig{opggeovinkar}

\grubop{opggeoR}
Let $ r $ be the radius of the circumscribed circle to $ \triangle ABC $. Show that
\[ r=\frac{abc}{4A_{\triangle ABC}} \]
\fig{opggeoR}

\grubop{opgeo45vink}
$ E $ is the midpoint of the square. Find the value of $ v $.
\fig{opggeo45vink}

\newpage
\grubop{opggeotangentsq}
Show that
\[ AB^2 = BC\cdot CD \]
\fig{opggeotangentsq}



\end{document}


