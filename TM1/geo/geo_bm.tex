\documentclass[english,hidelinks,pdftex, 11 pt, class=report,crop=false]{standalone}
\usepackage[T1]{fontenc}
\usepackage[utf8]{luainputenc}
\usepackage{lmodern} % load a font with all the characters
\usepackage{geometry}
\geometry{verbose,a4paper, inner=0cm, outer=0 cm, bmargin=2cm, tmargin=1cm}
%\textwidth=12cm
\setlength{\parindent}{0bp}
\usepackage{import}
\usepackage[subpreambles=false]{standalone}
\usepackage{amsmath}
\usepackage{amssymb}
\usepackage{esint}
\usepackage{babel}
\usepackage{tabu}
\usepackage[dvipsnames, table]{xcolor}
\usepackage{cancel}
\makeatother
\makeatletter
\usepackage{datetime2}
\usepackage{titlesec}
\usepackage[many]{tcolorbox}

% Eheter
\newcommand{\enh}[1]{\,\textrm{#1}}
%referances
\newcommand{\net}[2]{{\color{blue}\href{#1}{#2}}}

%Spaces
\newcommand{\vsk}{\\[12pt]}
\newcommand{\vs}{\vspace{-12pt}}

% Tabell for opplegg

\newcommand{\ovlist}[1]{
\vspace{-16pt}
\begin{itemize}
	#1
\end{itemize}
}

% Chapters and sections
\titleformat{\section}[block]{\bfseries}{\hspace{3cm}\thesection}{5pt}{}
\titleformat{\subsection}[block]{\bfseries}{\hspace{3cm}\thesection}{5pt}{}
\newcommand{\sectionbreak}{\clearpage} % New page on each section
 

\newlength{\mywidth}
\setlength{\mywidth}{14cm}

\newcommand{\cont}[1]{
\begin{tcolorbox}[center, boxrule=0.0 mm, width=\mywidth,arc=0mm,enhanced jigsaw,,colback=white,breakable]
#1	
\end{tcolorbox}
}

\newcommand{\info}[5]{
\begin{tcolorbox}[center, boxrule=0.1 mm, width=\mywidth,arc=0mm,enhanced jigsaw,breakable,colback=yellow!5]	
	
	\footnotesize
	\textbf{Øvingsområde}\\[5pt] #1 
	
	\textbf{Utstyr}\\ #2  \\
	
	\begin{tabular}{@{} p{4cm} p{4cm} l} 
		\textbf{Tid} & \textbf{Elevinndeling} & \textbf{Læringsarena} \\
		#3  & #4 & #5
	\end{tabular} 
\end{tcolorbox}	
}

\newcommand{\gjen}[1]{\begin{tcolorbox}[center,boxrule=0.1 mm, width=\mywidth,arc=0mm,colback=blue!3] {\large \textbf{Gjennomføring} \vspace{5 pt}}\newline #1  \end{tcolorbox}\vspace{-5pt}}
\newcommand{\eks}[1]{\begin{tcolorbox}[center,boxrule=0.1 mm, width=\mywidth,arc=0mm,colback=green!3] {\large \textbf{Eksempel} \vspace{5 pt}}\newline #1  \end{tcolorbox}\vspace{-5pt}}

\newcounter{opl}
%\numberwithin{opl}{article}


\newcommand{\opl}[1]{
\newpage
{\refstepcounter{opl} %\phantomsection 
\large \textbf{\theopl \;#1} \vsk}
}

% Headlines
\newcommand{\fork}{\textbf{Forkunnskapar}\\}
\newcommand{\forb}{\textbf{Forberedelsar}\\}
\newcommand{\opgvr}{\textbf{Oppgaver}}



%colors
\newcommand{\colr}[1]{{\color{red} #1}}
\newcommand{\colb}[1]{{\color{blue} #1}}
\newcommand{\colo}[1]{{\color{orange} #1}}
\newcommand{\colc}[1]{{\color{cyan} #1}}
\definecolor{projectgreen}{cmyk}{100,0,100,0}
\newcommand{\colg}[1]{{\color{projectgreen} #1}}

% Lister med bokstavar
\usepackage[inline]{enumitem}
% Opg
\newcommand{\abc}[1]{
	\begin{enumerate}[label=\alph*),leftmargin=18pt]
		#1
	\end{enumerate}
}

\usepackage[]{hyperref}

\newcommand{\note}{Merk}

% Geometry
\newcommand{\hlikb}{Midtnormalen i en likebeint trekant}
\newcommand{\arealsetn}{Arealsetningen}
\newcommand{\trkmedian}{Medianer i trekanter}
\newcommand{\midtrk}{Midtnormaler i trekanter}
\newcommand{\innskrsirk}{Halveringslinjer og innskrevet sirkel i trekanter}
\newcommand{\cossetn}{Cosinussetningen}
\newcommand{\perfvink}{Sentral- og periferivinkel}
\newcommand{\tang}{Tangent}


\begin{document}
\section{Definisjoner}
\reg[Halveringslinje]{
Gitt $ \angle BAC $. For et punkt $ P $ som ligger på \textit{halveringslinja} til vinkelen, er
\[ \angle BAP = PAC=\frac{1}{2}\angle BAC \] 
\fig{halfline}
}	
\reg[Midtpunkt]{
Midtpunktet $ C $ til $ AB $ er punktet på linjestykket som er slik at $ AC=CB $.
\fig{midtp} 
}
\reg[Midtnormal]{
Midtnormalen til $ AB $ står normalt på, og går gjennom midtpunktet til, $ AB $.
\fig{midnorm0}
}
\section{Egenskaper til trekanter}
\reg[Midtnormal i likebeint trekant \label{hlikb}]{
Gitt en likebeint trekant $ \triangle ABC $, hvor $ AC=BC $, som vist i figuren under. 
\fig{hlikb}
Høgda $ DC $ ligger da på midtnormalen til $ AB $.
}
\fork{\ref{hlikb}}{
Da både $ \triangle ADC $ og $ \triangle DBC $ er rettvinklede, har $ CD $ som korteste katet, og $ {AC=BC} $, følger det av Pytagoras' setning at $ {AD=BD} $.
}
\reg[Medianer i trekanter \label{trkmedian}]{
En \textit{median} er et linjestykke som går fra et hjørne i en trekant til midtpunktet på den motstående siden i trekanten. \vsk

De tre medianene i en trekant skjærer hverandre i ett og samme punkt.
\fig{median1a}
Gitt $ \triangle ABC $ med medianer $ CD $, $ BF $ og $ AE $, som skjærer hverandre i $ G $. Da er
\alg{	
\frac{CG}{GD}=\frac{BG}{GF}=\frac{AG}{GE} = 2
}
}
\fork{\ref{trkmedian}}{
\begin{figure}
	\centering
	\subfloat[]{
		\includegraphics{\figp{median1b}}
	}\;\;
	\subfloat[]{
		\includegraphics{\figp{median1c}}
	}
\end{figure}
Vi lar $ G $ være skjæringspunktet til $ BF $ og $ AE $, og tar det for gitt at dette ligger inne i $ \triangle ABC $. Da $ {AF=\frac{1}{2}AC} $ og $ {BE=\frac{1}{2}BC }$, er $ {ABF= BAE=\frac{1}{2}ABC} $. Dermed har $ F $ og $ E $ lik avstand til $ AB $, som betyr at $ {FE\parallel AB} $. Videre har vi også at
\alg{
ABG + AFG &= ABG + BGE \\
AFG &= BGE 
}
$ G $ har lik avstand til $ AF $ og $ FC $, og $ {AF=FC} $. Dermed er $ AFG=GFC $. Tilsvarende er $ BGE=GEC $. Altså har disse fire trekantene likt areal. Videre er
\alg{
AFG+GFC+GEC&= AEC \\
GEC &= \frac{1}{6}ABC
}
La $ H $ være skjæringspunktet til $ AE $ og $ CD $. Med samme framgangsmåte som over kan det vises at
\[HEC=\frac{1}{6}ABC \]
Da både $ \triangle GEC $ og $ \triangle HEC $ har $ CE $ som side, likt areal, og både $ G $ og $ H $ ligger på $ AE $, må $ G=H $. Altså skjærer medianene hverandre i ett og samme punkt. \vsk

$ \triangle ABC\sim\triangle FEC$ fordi de har parvis parallelle sider. Dermed er
\[ \frac{AB}{FE} = \frac{BC}{CE}=2  \]
$ {\triangle ABG\sim\triangle EFG }$ fordi $ \angle EGF $ og $ \angle AGB $ er toppvinkler og $ AB\parallel FE $. Dermed er
\[ \frac{GB}{FG}=\frac{AB}{FE}=2 \]
Tilsvarende kan det vises at
\[ \frac{CG}{GD}=\frac{AG}{GE}=2 \]
}
\reg[Midtnormaler i trekanter \label{midtrk}]{
Midtnormalene i en trekant møtes i ett og samme punkt. Dette punktet er sentrum i sirkelen som har hjørnene til trekanten på sin bue.	
\fig{midnorm1a}
}
\fork{\ref{midtrk}}{
\fig{midnorm1b}
Gitt $ \triangle ABC $ med midtpunktene $ D $, $ E $ og $ F $. Vi lar $ S $ være skjæringspunktet til de respektive midtnormalene til $ AC $ og $ AB $.
$ {\triangle AFS\sim\triangle CFS} $ fordi begge er rettvinklede, begge har $ FS $ som korteste katet, og $ AF=FC $. Tilsvarende er $ {\triangle ADS\sim\triangle BDS} $. Følgelig er $ {CS=AS=BS} $. Dette betyr at 
\begin{itemize}
	\item $ \triangle BSC $ er likebeint, og da går midtnormalen til $ BC $ gjennom $ S $.
	\item $ A $, $ B $ og $ C $ må nødvendigvis ligge på sirkelen med sentrum $ S $ og radius $ AS=BS=CS $ 
\end{itemize}
}
\reg[Halveringslinjer og innskrevet sirkel i trekanter \label{innskr1a}]{
Halveringslinjene til vinklene i en trekant møtes i ett og samme punkt. Dette punktet er sentrum i den \textit{innskrevne} sirkelen, som tangerer hver av sidene i trekanten.
\fig{innskr1a}
}
\fork{\ref{innskr1a}}{
\fig{innskr1b}
Gitt $ \triangle ABC $. Vi lar $ S $ være skjæringspunktet til de respective halveringslinjene til $ \angle BAC $ og $ \angle CBA $. Videre plasserer vi $ D $, $ E $ og $ F $ slik at $ {DS\perp AB} $, $ {ES\perp BC} $ og $ {FS\perp AC} $. $ {\triangle ASD\cong\triangle ASF} $ fordi begge er rettvinklede og har hypotenus $ AS $, og $ {\angle DAS=\angle SAF} $. Tilsvarende er $ {\triangle BSD \cong \triangle BSE} $. Dermed er $ {SE=SD=SF} $. Følgelig er $ F $, $ C $ og $ E $ de respektive tangeringspunktene til $ AB $, $ BC $ og $ AC $ og sirkelen med sentrum $ S $ og radius $ SE $. \vsk

Videre har vi at $ {\triangle CSE \cong  \triangle CSF} $, fordi begge er rettvinklede og har hypotenus $ CS $, og $ {SF=SE} $. Altså er $ {\angle FCS=\angle ECS}$, som betyr at $ CS $ ligger på halveringslinja til $ \angle ACB $.
}
\end{document}


