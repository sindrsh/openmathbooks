\documentclass[english,hidelinks,pdftex, 11 pt, class=report,crop=false]{standalone}
\usepackage[T1]{fontenc}
\usepackage[utf8]{luainputenc}
\usepackage{lmodern} % load a font with all the characters
\usepackage{geometry}
\geometry{verbose,a4paper, inner=0cm, outer=0 cm, bmargin=2cm, tmargin=1cm}
%\textwidth=12cm
\setlength{\parindent}{0bp}
\usepackage{import}
\usepackage[subpreambles=false]{standalone}
\usepackage{amsmath}
\usepackage{amssymb}
\usepackage{esint}
\usepackage{babel}
\usepackage{tabu}
\usepackage[dvipsnames, table]{xcolor}
\usepackage{cancel}
\makeatother
\makeatletter
\usepackage{datetime2}
\usepackage{titlesec}
\usepackage[many]{tcolorbox}

% Eheter
\newcommand{\enh}[1]{\,\textrm{#1}}
%referances
\newcommand{\net}[2]{{\color{blue}\href{#1}{#2}}}

%Spaces
\newcommand{\vsk}{\\[12pt]}
\newcommand{\vs}{\vspace{-12pt}}

% Tabell for opplegg

\newcommand{\ovlist}[1]{
\vspace{-16pt}
\begin{itemize}
	#1
\end{itemize}
}

% Chapters and sections
\titleformat{\section}[block]{\bfseries}{\hspace{3cm}\thesection}{5pt}{}
\titleformat{\subsection}[block]{\bfseries}{\hspace{3cm}\thesection}{5pt}{}
\newcommand{\sectionbreak}{\clearpage} % New page on each section
 

\newlength{\mywidth}
\setlength{\mywidth}{14cm}

\newcommand{\cont}[1]{
\begin{tcolorbox}[center, boxrule=0.0 mm, width=\mywidth,arc=0mm,enhanced jigsaw,,colback=white,breakable]
#1	
\end{tcolorbox}
}

\newcommand{\info}[5]{
\begin{tcolorbox}[center, boxrule=0.1 mm, width=\mywidth,arc=0mm,enhanced jigsaw,breakable,colback=yellow!5]	
	
	\footnotesize
	\textbf{Øvingsområde}\\[5pt] #1 
	
	\textbf{Utstyr}\\ #2  \\
	
	\begin{tabular}{@{} p{4cm} p{4cm} l} 
		\textbf{Tid} & \textbf{Elevinndeling} & \textbf{Læringsarena} \\
		#3  & #4 & #5
	\end{tabular} 
\end{tcolorbox}	
}

\newcommand{\gjen}[1]{\begin{tcolorbox}[center,boxrule=0.1 mm, width=\mywidth,arc=0mm,colback=blue!3] {\large \textbf{Gjennomføring} \vspace{5 pt}}\newline #1  \end{tcolorbox}\vspace{-5pt}}
\newcommand{\eks}[1]{\begin{tcolorbox}[center,boxrule=0.1 mm, width=\mywidth,arc=0mm,colback=green!3] {\large \textbf{Eksempel} \vspace{5 pt}}\newline #1  \end{tcolorbox}\vspace{-5pt}}

\newcounter{opl}
%\numberwithin{opl}{article}


\newcommand{\opl}[1]{
\newpage
{\refstepcounter{opl} %\phantomsection 
\large \textbf{\theopl \;#1} \vsk}
}

% Headlines
\newcommand{\fork}{\textbf{Forkunnskapar}\\}
\newcommand{\forb}{\textbf{Forberedelsar}\\}
\newcommand{\opgvr}{\textbf{Oppgaver}}



%colors
\newcommand{\colr}[1]{{\color{red} #1}}
\newcommand{\colb}[1]{{\color{blue} #1}}
\newcommand{\colo}[1]{{\color{orange} #1}}
\newcommand{\colc}[1]{{\color{cyan} #1}}
\definecolor{projectgreen}{cmyk}{100,0,100,0}
\newcommand{\colg}[1]{{\color{projectgreen} #1}}

% Lister med bokstavar
\usepackage[inline]{enumitem}
% Opg
\newcommand{\abc}[1]{
	\begin{enumerate}[label=\alph*),leftmargin=18pt]
		#1
	\end{enumerate}
}

\usepackage[]{hyperref}

\newcommand{\note}{Merk}
\newcommand{\notesm}[1]{{\footnotesize \textsl{\note:} #1}}
\newcommand{\ekstitle}{Eksempel }
\newcommand{\sprtitle}{Språkboksen}
\newcommand{\expl}{forklaring}
\newcommand{\pyt}{Pytagoras' setning}
\newcommand\sv{\vsk \textbf{Svar} \vspace{4 pt}\\}

%references
\newcommand{\reftab}[1]{\hrs{#1}{tabell}}
\newcommand{\rref}[1]{\hrs{#1}{regel}}
\newcommand{\dref}[1]{\hrs{#1}{definisjon}}
\newcommand{\refkap}[1]{\hrs{#1}{kapittel}}
\newcommand{\refsec}[1]{\hrs{#1}{seksjon}}
\newcommand{\refdsec}[1]{\hrs{#1}{delseksjon}}
\newcommand{\refved}[1]{\hrs{#1}{vedlegg}}
\newcommand{\eksref}[1]{\textsl{#1}}
\newcommand\fref[2][]{\hyperref[#2]{\textsl{figur \ref*{#2}#1}}}
\newcommand{\refop}[1]{{\color{blue}Oppgave \ref{#1}}}
\newcommand{\refops}[1]{{\color{blue}oppgave \ref{#1}}}


%Algebra
\newcommand{\kvadset}{Kvadratsetningene}
\newcommand{\aenato}{Sum-produkt-metoden}

% Geometry
\newcommand{\hlikb}{Midtnormalen i en likebeint trekant}
\newcommand{\arealsetn}{Arealsetningen}
\newcommand{\trkmedian}{Median}
\newcommand{\midtrk}{Midtnormal (i trekant)}
\newcommand{\innskrsirk}{Innskrevet sirkel}
\newcommand{\cossetn}{Cosinussetningen}
\newcommand{\perfvink}{Sentral- og periferivinkel}
\newcommand{\tang}{Tangent}

% Derivative
\newcommand{\derel}{Den deriverte av elementære funksjoner}
\newcommand{\divder}{Divisjonsregelen}
\newcommand{\kjernereg}{Kjerneregelen}
\newcommand{\prodregder}{Produktregelen}
\newcommand{\lhop}{L'Hopitals regel}

% Funksjonsdrofting
\newcommand{\monder}{Monotoniegenskaper og den deriverte}
\newcommand{\fderekstr}{$ \bm{f'=0} $ for lokale ektstremalpunkt}
\newcommand{\andredertest}{Andrederiverttesten}

% Vectors
\newcommand{\detar}{Arealformler med determinanter}
\newcommand{\avstpunktlin}{Avstand mellom punkt og linje}

%Appendix
\newcommand{\rolle}{Rolles teorem}
\newcommand{\meanval}{Middelverdisetningen}

% Solutions manual
\newcommand{\selos}{Se løsningsforslag.}

\begin{document}
\opr{opgalgligukonst2}	
\abc{
\item 
Vi har at
\algv{
	2x^2-4x&=0 \\
	x(2x-4)&=0
}
Altså er $ x=0 $, eller 
\algv{
	2x-4&=0 \\
	2x&=4 \\
	x&=2
}
}

\opr{opgalgsym}\\
La $ x_b $ være minimumspunktet til $ f $. Av symmetriegenskapene til $ f $ har vi at $ x_b=\frac{x_1+x_2}{2} $ hvis $ f(x_1)=f(x_2) $. Da $ {(-2)4=-8} $ og $ {4-2=2} $, er
\[ f(x)=x^2+2x-8=(x-2)(x+4) \]
Dette betyr at $ f(2)=f(-4)=0 $, og da er
\[ x_b=\frac{2+(-4)}{2}=-1 \]

\newpage
\grubr{opgealgsqrt27}\\
Da
\[ (\sqrt{x}+\sqrt{y})^2 = x^2+2\sqrt{xy}+y^2=27  \]
må $ \sqrt{xy} $ også være et heltall, og dermed må $ xy $ være et kvadrattall. %obs! Definer ordet 
Ved litt prøving og feiling finner vi at 
\[ x=3 \qquad, \qquad y = 12\]

\grubr{opgalgkvadsqrt}
\abc{
	\item 
	\alg{
	10-2\sqrt{21} &= 10-2\sqrt{7}\sqrt{3} \\
	&= \sqrt{7}^2+\sqrt{3}^2-2\sqrt{7}\sqrt{3} \\
	&= (\sqrt{7}-\sqrt{3})^2
	}
	\item
	\alg{
	13+2\sqrt{22} &= 13+2\sqrt{11}\sqrt{2} \\
	&= \sqrt{11}^2+\sqrt{2}^2+2\sqrt{11}\sqrt{2}\\
	&= (\sqrt{11}+\sqrt{2})^2
	}
	\item
	\alg{
	8+4\sqrt{3} &= 2(4+2\sqrt{3}) \\
	&= 2(\sqrt{3}^2+\sqrt{1}^2 +2\sqrt{1}\sqrt{3}) \\
	&= 2(\sqrt{3}+1)^2 \\
	&= (\sqrt{6}+\sqrt{2})^2
	}
	\item
	\alg{
	42-14\sqrt{5} &= 7(6-2\sqrt{5}) \\
	&= 7(\sqrt{5}^2+\sqrt{1}^2-\sqrt{5}\sqrt{1})\\
	&= 7(\sqrt{5}-\sqrt{1})^2 \\
	&= (\sqrt{35}-\sqrt{7})^2
	}
}

\grubr{opggeoher}
\fig{opggeoher}
Vi setter $ a=BC $, $ b=AC $, $ c=AB $, $ h=CD $ og $ d=AD $. Av \pyt\ med hensyn på $ \triangle ADC $ og $ \triangle DCB $ har vi at
\alg{
b^2-d^2 &= a^2-(c-d)^2 \\
d &= \frac{b^2+c^2-a^2}{2c}
}
Videre er
\alg{
h^2 &= b^2-d^2 \\
&= (b+d)(b-d) \\
&= \left(b+\frac{b^2+c^2-a^2}{2c}\right)\left(b-\frac{b^2+c^2-a^2}{2c}\right) \\
&= \frac{1}{4c^2}\left[(b+c)^2-a^2\right]\left[a^2-(b-c)^2\right] \\
&= \frac{1}{4c^2}(b+c+a)(b+c-a)(a+b-c)(a-b+c)
}
Da $ h>0 $, er
\[ h=\frac{1}{2c}\sqrt{(b+c+a)(b+c-a)(a+b-c)(a-b+c)} \]
Arealet $ T $ til $ \triangle ABC $ er nå gitt som
\alg{
T&= \frac{1}{2}hc\br
&= \frac{1}{4}\sqrt{(b+c+a)(b+c-a)(a+b-c)(a-b+c)}
}
\mers{Da en trekant må ha et areal med positiv verdi, kan Herons formel brukes for å vise \textit{trekantulikheten}, som vi utledet i \mb.}\vsk

\newpage
\grubr{opggeokvadsym} \\
\textbf{Alternativ 1}\\
Skal grafen til $ f $ være symmetrisk om linja $ {x=-\frac{b}{2a}} $, må vi for et tall $ k $ ha at
\begin{equation}\label{opggeokvadsymeq}
	f\left(k-\frac{b}{2a}\right)=f\left(-k-\frac{b}{2a}\right)
\end{equation}
For et tall $ d $ har vi at
\[ d-\frac{b}{2a}=\frac{2ad-b}{2a} \]
Videre er
\alg{
	f\left(d-\frac{b}{2a}\right)&=a\left(\frac{2ad-b}{2a} \right)^2+b\left(\frac{2ad-b}{2a} \right)+c \br
	&= \frac{4a^2d^2-4abd+b^2}{4a}+\frac{2abd-b^2}{2a}+c \br
	&=\frac{4a^2d^2-b^2}{4a}+c
}
Dette betyr at uansett om $ d=k $ eller om $ d=-k $, så vil uttrykket over være likt, og altså er \eqref{opggeokvadsymeq} gyldig. \vsk

\textbf{Alternativ 2}\\
Skal $ f $ være symmetrisk om en vertikallinje, må det bety at to tall $ t $ og $ s $ gir lik $ f $-verdi:
\begin{flalign*}
	&&f(s)&=f(t) \\
	&&a s^2+b s + c &= a t^2 + b t + c \\
	&&a(s^2-t^2)+b(s-t) &= 0 \\
	&&a(s-t)(s+t)+b(s-t)&=0 &&(s\neq t)\\
	&& a(s+t)+b&=0 \\
	&& t&= -\frac{b}{a}-s
\end{flalign*}
Vi lar $ x_s $ være $ x $-verdien til symmetrilinja til $ f $. $ x_s $ må ligge midt mellom $ s $ og $ t $. Vi lar $ t>s $, da er
\alg{
x_s &= s+\frac{1}{2}(t-s) \\
&= s+\frac{1}{2}\left(-\frac{b}{a}-s-s\right) \\
&=-\frac{b}{2a}
}
\newpage
\grubr{opgalgdrh}
\alg{
	d^2  r^2-(d+r)^2r^2&+4b d^2 (b-r)\\
	&=(-2dr-r^2) r^2+2b d (2bd-2dr)\\
	&=(2bd-2dr-r^2)r^2-2bdr^2+2bd(2bd-2dr-r^2)+2bdr^2\\
	&=(2bd-2dr-r^2)\left(r^2+2bd\right)
} 

\grubr{opgbevisabc} \\
Hvis $ f(x)=ax^2+bx+c=0 $, har vi for et tall $ k $ at
\[ f\left(\pm k-\frac{b}{2a}\right)=\frac{4a^2k^2-b^2}{4a}+c=0 \]
Løser vi denne ligningen med hensyn på $ k $, får vi at
\[ k=\pm \frac{\sqrt{b^2-4ac}}{2a} \]
Dette betyr at
\[ f\left(\pm \frac{\sqrt{b^2-4ac}}{2a}-\frac{b}{2a}\right)=0 \]
\end{document}