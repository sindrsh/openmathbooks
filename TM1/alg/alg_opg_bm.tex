\documentclass[english,hidelinks,pdftex, 11 pt, class=report,crop=false]{standalone}
\usepackage[T1]{fontenc}
\usepackage[utf8]{luainputenc}
\usepackage{lmodern} % load a font with all the characters
\usepackage{geometry}
\geometry{verbose,a4paper, inner=0cm, outer=0 cm, bmargin=2cm, tmargin=1cm}
%\textwidth=12cm
\setlength{\parindent}{0bp}
\usepackage{import}
\usepackage[subpreambles=false]{standalone}
\usepackage{amsmath}
\usepackage{amssymb}
\usepackage{esint}
\usepackage{babel}
\usepackage{tabu}
\usepackage[dvipsnames, table]{xcolor}
\usepackage{cancel}
\makeatother
\makeatletter
\usepackage{datetime2}
\usepackage{titlesec}
\usepackage[many]{tcolorbox}

% Eheter
\newcommand{\enh}[1]{\,\textrm{#1}}
%referances
\newcommand{\net}[2]{{\color{blue}\href{#1}{#2}}}

%Spaces
\newcommand{\vsk}{\\[12pt]}
\newcommand{\vs}{\vspace{-12pt}}

% Tabell for opplegg

\newcommand{\ovlist}[1]{
\vspace{-16pt}
\begin{itemize}
	#1
\end{itemize}
}

% Chapters and sections
\titleformat{\section}[block]{\bfseries}{\hspace{3cm}\thesection}{5pt}{}
\titleformat{\subsection}[block]{\bfseries}{\hspace{3cm}\thesection}{5pt}{}
\newcommand{\sectionbreak}{\clearpage} % New page on each section
 

\newlength{\mywidth}
\setlength{\mywidth}{14cm}

\newcommand{\cont}[1]{
\begin{tcolorbox}[center, boxrule=0.0 mm, width=\mywidth,arc=0mm,enhanced jigsaw,,colback=white,breakable]
#1	
\end{tcolorbox}
}

\newcommand{\info}[5]{
\begin{tcolorbox}[center, boxrule=0.1 mm, width=\mywidth,arc=0mm,enhanced jigsaw,breakable,colback=yellow!5]	
	
	\footnotesize
	\textbf{Øvingsområde}\\[5pt] #1 
	
	\textbf{Utstyr}\\ #2  \\
	
	\begin{tabular}{@{} p{4cm} p{4cm} l} 
		\textbf{Tid} & \textbf{Elevinndeling} & \textbf{Læringsarena} \\
		#3  & #4 & #5
	\end{tabular} 
\end{tcolorbox}	
}

\newcommand{\gjen}[1]{\begin{tcolorbox}[center,boxrule=0.1 mm, width=\mywidth,arc=0mm,colback=blue!3] {\large \textbf{Gjennomføring} \vspace{5 pt}}\newline #1  \end{tcolorbox}\vspace{-5pt}}
\newcommand{\eks}[1]{\begin{tcolorbox}[center,boxrule=0.1 mm, width=\mywidth,arc=0mm,colback=green!3] {\large \textbf{Eksempel} \vspace{5 pt}}\newline #1  \end{tcolorbox}\vspace{-5pt}}

\newcounter{opl}
%\numberwithin{opl}{article}


\newcommand{\opl}[1]{
\newpage
{\refstepcounter{opl} %\phantomsection 
\large \textbf{\theopl \;#1} \vsk}
}

% Headlines
\newcommand{\fork}{\textbf{Forkunnskapar}\\}
\newcommand{\forb}{\textbf{Forberedelsar}\\}
\newcommand{\opgvr}{\textbf{Oppgaver}}



%colors
\newcommand{\colr}[1]{{\color{red} #1}}
\newcommand{\colb}[1]{{\color{blue} #1}}
\newcommand{\colo}[1]{{\color{orange} #1}}
\newcommand{\colc}[1]{{\color{cyan} #1}}
\definecolor{projectgreen}{cmyk}{100,0,100,0}
\newcommand{\colg}[1]{{\color{projectgreen} #1}}

% Lister med bokstavar
\usepackage[inline]{enumitem}
% Opg
\newcommand{\abc}[1]{
	\begin{enumerate}[label=\alph*),leftmargin=18pt]
		#1
	\end{enumerate}
}

\usepackage[]{hyperref}

\newcommand{\note}{Merk}
\newcommand{\notesm}[1]{{\footnotesize \textsl{\note:} #1}}
\newcommand{\ekstitle}{Eksempel }
\newcommand{\sprtitle}{Språkboksen}
\newcommand{\expl}{forklaring}
\newcommand{\pyt}{Pytagoras' setning}
\newcommand\sv{\vsk \textbf{Svar} \vspace{4 pt}\\}

%references
\newcommand{\reftab}[1]{\hrs{#1}{tabell}}
\newcommand{\rref}[1]{\hrs{#1}{regel}}
\newcommand{\dref}[1]{\hrs{#1}{definisjon}}
\newcommand{\refkap}[1]{\hrs{#1}{kapittel}}
\newcommand{\refsec}[1]{\hrs{#1}{seksjon}}
\newcommand{\refdsec}[1]{\hrs{#1}{delseksjon}}
\newcommand{\refved}[1]{\hrs{#1}{vedlegg}}
\newcommand{\eksref}[1]{\textsl{#1}}
\newcommand\fref[2][]{\hyperref[#2]{\textsl{figur \ref*{#2}#1}}}
\newcommand{\refop}[1]{{\color{blue}Oppgave \ref{#1}}}
\newcommand{\refops}[1]{{\color{blue}oppgave \ref{#1}}}


%Algebra
\newcommand{\kvadset}{Kvadratsetningene}
\newcommand{\aenato}{Sum-produkt-metoden}

% Geometry
\newcommand{\hlikb}{Midtnormalen i en likebeint trekant}
\newcommand{\arealsetn}{Arealsetningen}
\newcommand{\trkmedian}{Median}
\newcommand{\midtrk}{Midtnormal (i trekant)}
\newcommand{\innskrsirk}{Innskrevet sirkel}
\newcommand{\cossetn}{Cosinussetningen}
\newcommand{\perfvink}{Sentral- og periferivinkel}
\newcommand{\tang}{Tangent}

% Derivative
\newcommand{\derel}{Den deriverte av elementære funksjoner}
\newcommand{\divder}{Divisjonsregelen}
\newcommand{\kjernereg}{Kjerneregelen}
\newcommand{\prodregder}{Produktregelen}
\newcommand{\lhop}{L'Hopitals regel}

% Funksjonsdrofting
\newcommand{\monder}{Monotoniegenskaper og den deriverte}
\newcommand{\fderekstr}{$ \bm{f'=0} $ for lokale ektstremalpunkt}
\newcommand{\andredertest}{Andrederiverttesten}

% Vectors
\newcommand{\detar}{Arealformler med determinanter}
\newcommand{\avstpunktlin}{Avstand mellom punkt og linje}

%Appendix
\newcommand{\rolle}{Rolles teorem}
\newcommand{\meanval}{Middelverdisetningen}

% Solutions manual
\newcommand{\selos}{Se løsningsforslag.}

\begin{document}

\opgt	

\op{opgalgfullkvad}
Skriv som fullstendige kvadrat.\os
\abch{
\item $ x^2+6x+9 $
\item $ b^2+14b+49 $
\item $ a^2-2a+1 $
} \os
\abchs{4}{
\item $ k^2-\frac{2}{3}k+\frac{1}{9} $
\item $ c^2-\frac{1}{2}c+\frac{1}{16} $
\,\item $ y^2+\frac{6}{7}y+\frac{9}{49} $
}

\op{opgalgfullkvad2}
Skriv som fullstendige kvadrat.\os
\abch{
	\item $ 25a^2+90a+81 $
	\item $ 9b^2+12a+4 $
	\item $ 64c^2-16c+1 $
} \os
\abchs{4}{
\item $ \frac{1}{4}d^2+\frac{3}{4}d+\frac{9}{16} $
\item $ \frac{1}{25}e^2+\frac{4}{35}e+\frac{4}{49} $
\item $ \frac{81}{64}f^2-\frac{15}{4}f+\frac{25}{9} $
}

\op{opgalgkongsetn}
Vis at
\[ (a-b)-b^2 =a(a-2b) \]

\op{opgalgbrokforenkl} \vs
\abc{
\item Gitt to heltall $ a $ og $ b $. Forklar hvorfor
$ (a+\sqrt{b})(a-\sqrt{b}) $ er et heltall.
\item Skriv om brøken $ \frac{5}{2-\sqrt{3}} $ til en brøk med heltalls nevner.
}

\op{opgalgledd}
Skriv om til et uttrykk der $ x $ er et ledd i et fullstendig kvadrat. \os
\abch{
\item $ x^2+6x-7 $
\item $ x^2-8x-20 $
\item $ x^2+12-45 $
}

\op{algopga1a2}
Hvorfor er det ved bruk av \refunnbr{a1a2}{sum-produkt-metoden} lurte å starte med å finne tall som oppfyller kravet $ a_1a_2=c $ (i motsetning til å finne tall som oppfyller kravet $ a_1+a_2=b $)?

\op{opgalgfakt}
Faktoriser uttrykkene fra \refops{opgalgledd}.

\newpage
\op{opgalgfullkvad3}
Faktoriser uttrykkene.\os
\abch{
	\item $ x^2-10kx+25k^2$
	\item $ y^2+8yz+16z^2 $
	\item $ a^2-20aq+100q^2 $
} \os
\abchs{4}{
	\item $ x^2 + x y - 20y^2 $
	\item $ a^2-9ab+14b^2 $
	\item $ y^2-9k^5y-k^2y+9k^7 $
}

\op{opgalgulikgraf}
Gitt ulikheten 
\[ x^2-9x+20>x-1 \]
\abc{
\item Bruk figuren under til å løse ulikheten.
\item Løs ulikheten ved hjelp av faktorisering.
}
\fig{opgalgulik}

\eksop{1TH21D1}{1TH21D1opg3}
Skriv så enkelt som mulig
\[ \frac{2x^2-2}{x2-2x +1} \]

\eksop{1TV21D1}{1TV21D1opg3}
Skriv så enkelt som mulig
\[ \frac{x}{x-3}+\frac{x-6}{x+3}-\frac{18}{x^2-9} \]

\eksop{1TH21D1}{1TH21D1opg2}
Løs ulikheten.
\[ x^2+2x-8 < 0 \]

\newpage
\op{opgalgulikbrok}
Gitt ulikheten
\[ \frac{10}{x+3}-\frac{2}{x+5}>0 \]
\abc{
\item Forklar hvorfor det er problematisk å gange begge sider av ulikheten med en fellesnevner.
\item Løs ulikheten.
}

\nes
\op{opgalgligukonst} 
Gitt likningen
\nn{
	ax^2+bx=0
}
Vis, uten å bruke \textit{abc}-formelen, at
\nn{x=0\qquad \vee \qquad x=-\frac{b}{a}}

\op{opgalgligukonst2}
Løs likningene.\os
\abch{
\item $ 2x^2-4x=0 $
\item $ 3x^2+27x=0 $ 
}\\[12pt]
\abchs{3}{
\item $ 7x^2+2x=0 $
\item $8x-9x^2=0 $
}

\op{opgalgabc}
Løs likningene.\os
\abch{
	\item $ x^2-4x-4=0 $ 	
	\item $ x^2+2x-15 $
	\item $ x^2+3x-70=0 $ 
	
}
\\[12pt]
\abchs{4}{
	\item $ x^2+5x-7=0 $
	\item $x^2-x-1=0 $
	\item $x^2-2x-9=0 $	
}
\\[12pt]
\abchs{7}{
	\item $ 5x^2+2x-7=0 $
	\item $8x^2-2x^2-9=0 $
	\item $3x^2-12x+1=0 $ 
}

\eksop{1TH21D1}{1TH21D1opg4}
Grafen til en andregradsfunksjon $ f $ går gjennom punktene $ (0, 12) $, $ (-3, 0) $ og $ (2, 0) $. Bestem $ f(x) $.

\newpage
\op{opgalgsym}
Grafen til $ {f(x)=x^2+2x-8} $ er symmetrisk om vertikallinja som går gjennom bunnpunktet. Finn $ x $-verdien til dette punktet.
\fig{opgalgsym}

\eksop{1TH21D1}{1TH21D1opg5} \vs \vs
\alg{
x^2+2x-y =-1 \tag{I}\vn
x+y = -2 \tag{II}
}
Vis at ligningssystemet ikke har løsning
\abc{
\item grafisk
\item ved regning
}
\nes
\op{algopgpoldiv}
Utfør polynomdivisjon på uttrykkene \os
\abch{
\item $ \frac{x^4-3x^2+5}{x^3+x} $
\item $ \frac{-7x^3-9x^2+x}{-4x^2+3} $
\item $ \frac{2x^3-6x^2+9x-27}{2x^2+9} $
}

\nes

\op{algopgpolfakt}
$ P(x)=0 $ for én av $ x\in\{-1, 2, 3\} $.
Faktoriser $ P $ når
\abc{
\item $ P=x^3-37x+84 $
\item $ P=x^3+10x^2+17x+18 $
\item $ P=2x^3+21x^2+61x+42 $
}

\nes

\op{opgalgpot1}
Løs likningen. \os
\abch{
\item $ 7\cdot5^x=14 $
\item $ 3\cdot8^x=27 $
\item $ 10\cdot2^x=19 $
}

\newpage
\op{opgalgpot2}
Vis at likningen 
\[ b\cdot a^x =c \]
har løsningen
\[ x=\log_a\frac{c}{b} \]

\op{opgalgsub}
Løs likningen. (Hint; se \refved{Bytvar})\os

\abch{
\item $ (\ln x)^2-5\ln x+ 6=0$
\item $ (\log x)^2-3\ln x- 70=0$
} \os
\abchs{3}{
\item $ e^{2x}-2x-3=0 $ \hspace{1cm}
\item $ e^{2x}+7x-18=0 $
}

\eksop{1TH21D1}{1TH21D1opg7}
Løs ligningene
\abc{
\item $ \lg(2x-6)=2 $
\item $ \dfrac{3^{2x}+3^{2x}+4}{2}=29 $
}


\newpage
\grubop{opgalgsqrt27} \vs
\[ \sqrt{27}=\sqrt{x}+\sqrt{y} \]
Finn de heltallige verdiene til $ x $ og $ y $.

\grubop{opgalgkvadsqrt}
Skriv uttrykkene på formen $ \left(\sqrt{a}+\sqrt{b}\right)^2 $, hvor $ a $ og $ b $ er heltall.
\abc{
\item $ 10-2\sqrt{21} $
\item $ 13+2\sqrt{22} $
\item $ 8+4\sqrt{3} $
\item $ 42-14\sqrt{5} $
}

\grubop{opggeoher}
For en trekant med sidelengder $ a $, $ b $ og $ c $ er arealet $ T $ gitt ved \outl{Herons formel}:
\[ T=\frac{1}{4}\sqrt{(a+b+c)(a+b-c)(a-b+c)(b+c-a)} \]
Bevis formelen.

\grubop{opggeokvadsym}
Gitt funksjonen  ${f(x)=a x^2+bx +c} $. Vis at grafen til $ f $ er symmetrisk om vertikallinja som går gjennom punktet $ \left(-\frac{b}{2a}, 0\right) $.

\grubop{opgaldrdh}
Vis at $ 2bd-2dr-r^2 $ er en faktor i uttrykket\\ $ d^2  r^2-(d+r)^2  r^2+4b d^2 (b-r) $.
\end{document}