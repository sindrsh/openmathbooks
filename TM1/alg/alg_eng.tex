\documentclass[english,hidelinks,pdftex, 11 pt, class=report,crop=false]{standalone}
\usepackage[T1]{fontenc}
\usepackage[utf8]{luainputenc}
\usepackage{lmodern} % load a font with all the characters
\usepackage{geometry}
\geometry{verbose,a4paper, inner=0cm, outer=0 cm, bmargin=2cm, tmargin=1cm}
%\textwidth=12cm
\setlength{\parindent}{0bp}
\usepackage{import}
\usepackage[subpreambles=false]{standalone}
\usepackage{amsmath}
\usepackage{amssymb}
\usepackage{esint}
\usepackage{babel}
\usepackage{tabu}
\usepackage[dvipsnames, table]{xcolor}
\usepackage{cancel}
\makeatother
\makeatletter
\usepackage{datetime2}
\usepackage{titlesec}
\usepackage[many]{tcolorbox}

% Eheter
\newcommand{\enh}[1]{\,\textrm{#1}}
%referances
\newcommand{\net}[2]{{\color{blue}\href{#1}{#2}}}

%Spaces
\newcommand{\vsk}{\\[12pt]}
\newcommand{\vs}{\vspace{-12pt}}

% Tabell for opplegg

\newcommand{\ovlist}[1]{
\vspace{-16pt}
\begin{itemize}
	#1
\end{itemize}
}

% Chapters and sections
\titleformat{\section}[block]{\bfseries}{\hspace{3cm}\thesection}{5pt}{}
\titleformat{\subsection}[block]{\bfseries}{\hspace{3cm}\thesection}{5pt}{}
\newcommand{\sectionbreak}{\clearpage} % New page on each section
 

\newlength{\mywidth}
\setlength{\mywidth}{14cm}

\newcommand{\cont}[1]{
\begin{tcolorbox}[center, boxrule=0.0 mm, width=\mywidth,arc=0mm,enhanced jigsaw,,colback=white,breakable]
#1	
\end{tcolorbox}
}

\newcommand{\info}[5]{
\begin{tcolorbox}[center, boxrule=0.1 mm, width=\mywidth,arc=0mm,enhanced jigsaw,breakable,colback=yellow!5]	
	
	\footnotesize
	\textbf{Øvingsområde}\\[5pt] #1 
	
	\textbf{Utstyr}\\ #2  \\
	
	\begin{tabular}{@{} p{4cm} p{4cm} l} 
		\textbf{Tid} & \textbf{Elevinndeling} & \textbf{Læringsarena} \\
		#3  & #4 & #5
	\end{tabular} 
\end{tcolorbox}	
}

\newcommand{\gjen}[1]{\begin{tcolorbox}[center,boxrule=0.1 mm, width=\mywidth,arc=0mm,colback=blue!3] {\large \textbf{Gjennomføring} \vspace{5 pt}}\newline #1  \end{tcolorbox}\vspace{-5pt}}
\newcommand{\eks}[1]{\begin{tcolorbox}[center,boxrule=0.1 mm, width=\mywidth,arc=0mm,colback=green!3] {\large \textbf{Eksempel} \vspace{5 pt}}\newline #1  \end{tcolorbox}\vspace{-5pt}}

\newcounter{opl}
%\numberwithin{opl}{article}


\newcommand{\opl}[1]{
\newpage
{\refstepcounter{opl} %\phantomsection 
\large \textbf{\theopl \;#1} \vsk}
}

% Headlines
\newcommand{\fork}{\textbf{Forkunnskapar}\\}
\newcommand{\forb}{\textbf{Forberedelsar}\\}
\newcommand{\opgvr}{\textbf{Oppgaver}}



%colors
\newcommand{\colr}[1]{{\color{red} #1}}
\newcommand{\colb}[1]{{\color{blue} #1}}
\newcommand{\colo}[1]{{\color{orange} #1}}
\newcommand{\colc}[1]{{\color{cyan} #1}}
\definecolor{projectgreen}{cmyk}{100,0,100,0}
\newcommand{\colg}[1]{{\color{projectgreen} #1}}

% Lister med bokstavar
\usepackage[inline]{enumitem}
% Opg
\newcommand{\abc}[1]{
	\begin{enumerate}[label=\alph*),leftmargin=18pt]
		#1
	\end{enumerate}
}

\usepackage[]{hyperref}

% note
\newcommand{\note}{Note}
\newcommand{\notesm}[1]{{\footnotesize \textsl{\note:} #1}}
\newcommand{\selos}{See the solutions manual.}

\newcommand{\texandasy}{The text is written in \LaTeX\ and the figures are made with the aid of Asymptote.}

\newcommand{\rknut}{Calculate.}
\newcommand\sv{\vsk \textbf{Answer} \vspace{4 pt}\\}
\newcommand{\ekstitle}{Example }
\newcommand{\sprtitle}{The language box}
\newcommand{\expl}{explanation}

% answers
\newcommand{\mulansw}{\notesm{Multiple possible answers.}}	
\newcommand{\faskap}{Chapter}

% exercises
\newcommand{\opgt}{\newpage \phantomsection \addcontentsline{toc}{section}{Exercises} \section*{Exercises for Chapter \thechapter}\vs \setcounter{section}{1}}

% references
\newcommand{\reftab}[1]{\hrs{#1}{Table}}
\newcommand{\rref}[1]{\hrs{#1}{Rule}}
\newcommand{\dref}[1]{\hrs{#1}{Definition}}
\newcommand{\refkap}[1]{\hrs{#1}{Chapter}}
\newcommand{\refsec}[1]{\hrs{#1}{Section}}
\newcommand{\refdsec}[1]{\hrs{#1}{Subsection}}
\newcommand{\refved}[1]{\hrs{#1}{Appendix}}
\newcommand{\eksref}[1]{\textsl{#1}}
\newcommand\fref[2][]{\hyperref[#2]{\textsl{Figure \ref*{#2}#1}}}
\newcommand{\refop}[1]{{\color{blue}Exercise \ref{#1}}}
\newcommand{\refops}[1]{{\color{blue}Exercise \ref{#1}}}

%%% SECTION HEADLINES %%%

% Our numbers
\newcommand{\likteikn}{The equal sign}
\newcommand{\talsifverd}{Numbers, digits and values}
\newcommand{\koordsys}{Coordinate systems}

% Calculations
\newcommand{\adi}{Addition}
\newcommand{\sub}{Subtraction}
\newcommand{\gong}{Multiplication}
\newcommand{\del}{Division}

%Factorization and order of operations
\newcommand{\fak}{Factorization}
\newcommand{\rrek}{Order of operations}

%Fractions
\newcommand{\brgrpr}{Introduction}
\newcommand{\brvu}{Values, expanding and simplifying}
\newcommand{\bradsub}{Addition and subtraction}
\newcommand{\brgngheil}{Fractions multiplied by integers}
\newcommand{\brdelheil}{Fractions divided by integers}
\newcommand{\brgngbr}{Fractions multiplied by fractions}
\newcommand{\brkans}{Cancelation of fractions}
\newcommand{\brdelmbr}{Division by fractions}
\newcommand{\Rasjtal}{Rational numbers}

%Negative numbers
\newcommand{\negintro}{Introduction}
\newcommand{\negrekn}{The elementary operations}
\newcommand{\negmeng}{Negative numbers as amounts}

%Calculation methods
\newcommand{\delmedtihundre}{Deling med 10, 100, 1\,000 osv.}

% Geometry 1
\newcommand{\omgr}{Terms}
\newcommand{\eignsk}{Attributes of triangles and quadrilaterals}
\newcommand{\omkr}{Perimeter}
\newcommand{\area}{Area}

%Algebra 
\newcommand{\algintro}{Introduction}
\newcommand{\pot}{Powers}
\newcommand{\irrasj}{Irrational numbers}

%Equations
\newcommand{\ligintro}{Introduction}
\newcommand{\liglos}{Solving with the elementary operations}
\newcommand{\ligloso}{Solving with elementary operations summarized}

%Functions
\newcommand{\fintro}{Introduction}
\newcommand{\lingraf}{Linear functions and graphs}

%Geometry 2
\newcommand{\geoform}{Formulas of area and perimeter}
\newcommand{\kongogsim}{Congruent and similar triangles}
\newcommand{\geofork}{Explanations}

% Names of rules
\newcommand{\adkom}{Addition is commutative}
\newcommand{\gangkom}{Multiplication is commutative}
\newcommand{\brdef}{Fractions as rewriting of division}
\newcommand{\brtbr}{Fractions multiplied by fractions}
\newcommand{\delmbr}{Fractions divided by fractions}
\newcommand{\gangpar}{Distributive law}
\newcommand{\gangparsam}{Paranthesis multiplied together}
\newcommand{\gangmnegto}{Multiplication by negative numbers I}
\newcommand{\gangmnegtre}{Multiplication by negative numbers II}
\newcommand{\konsttre}{Unique construction of triangles}
\newcommand{\kongtre}{Congruent triangles}
\newcommand{\topv}{Vertical angles}
\newcommand{\trisum}{The sum of angles in a triangle}
\newcommand{\firsum}{The sum of angles in a quadrilateral}
\newcommand{\potgang}{Multiplication by powers}
\newcommand{\potdivpot}{Division by powers}
\newcommand{\potanull}{The special case of \boldmath $a^0$}
\newcommand{\potneg}{Powers with negative exponents}
\newcommand{\potbr}{Fractions as base}
\newcommand{\faktgr}{Factors as base}
\newcommand{\potsomgrunn}{Powers as base}
\newcommand{\arsirk}{The area of a circle}
\newcommand{\artrap}{The area of a trapezoid}
\newcommand{\arpar}{The area of a parallelogram}
\newcommand{\pyt}{Pythagoras's theorem}
\newcommand{\forform}{Ratios in similar triangles}
\newcommand{\vilkform}{Terms of similar triangles}
\newcommand{\omkrsirk}{The perimeter of a circle (and the value of $ \bm \pi $)}
\newcommand{\artri}{The area of a triangle}
\newcommand{\arrekt}{The area of a rectangle}
\newcommand{\liknflyt}{Moving terms across the equal sign}
\newcommand{\funklin}{Linear functions}




\begin{document}
\section{Factorization}
\reg[\kvadset \label{kvadset}]{
	For two real numbers $ a $ and $ b $, we have
	\alg{
		&&(a+b)^2 &= a^2+2ab+b^2 && \text{(1st square formula)}\vn
		&&(a-b)^2 &= a^2-2ab+b^2 && \text{(2nd square formula)} \vn
		&&(a+b)(a-b)&= a^2-b^2 && \text{(3rd square formula)}
	}
}
\spr{
	$ (a+b)^2 $ and $ (a-b)^2 $ are called \outl{complete squares}.\vsk	
	
	The 3rd square formula is also called the \outl{conjugate formula}.\vsk
	
	All the square formulas are \textit{identities}. An \outl{identity} is an equation that is satisfied no matter which values are given to the variables in the equation. 
}
\eks[1]{
	Rewrite $ a^2+8a+16 $ to a complete square.
	
	\sv \vsb \vs
	\alg{
		a^2+8a+16 &= a^2+2\cdot4a+4^2 \\
		&= (a+4)^2
	}
}
\eks[2]{
	Rewrite $ {k^2+6k +7} $ to an expression where $ k $ is part of a complete square.
	
	\sv \vsb \vs
	\alg{
		k^2+6k +7 &= k^2+2\cdot3k+7  \\
		&= k^2+2\cdot3k+ 3^2-3^2+7 \\
		&= (k+3)^2-2
	}
}
\newpage
\eks[3]{
	Factorize $ x^2-10x+16 $.
	
	\sv
	We start by creating a complete square:
	\alg{
		x^2-10x+16 &= x^2-2\cdot5x+5^2-5^2+16 \\
		&= (x-5)^2-9
	}
	We note that $ 9=3^2 $, and use the 3rd square formula:
	\alg{
		(x-5)^2-3^2 &= (x-5+3)(x-5-3) \\
		&= (x-2)(x-8)
	}
	Thus, 
	\[ x^2-10x+16=(x-2)(x-8) \]
} 
\fork{\ref{kvadset} \kvadset}{
	The square formulas follow directly from the distributive law in multiplication (see \mb).
}
\vsk

\reg[\aenato \label{a1a2}]{
	Given $ x, b, c \in \mathbb{R} $. If $ a_1+a_2=b $ and $ a_1a_2=c$, then
	\begin{equation}
		x^2+bx+c=(x+a_1)(x+a_2)	\label{a1a2eq}
	\end{equation}
}
\eks[1]{
	Factorize the expression $ x^2-x-6 $.
	
	\sv
	Since $ 2(-3)=-6 $ and $ 2+(-3)=-1 $, we have
	\[ x^2-1x-6=(x+2)(x-3) \]
}
\newpage
\eks[2]{
	Factorize the expression $ b^2-5b+4 $.
	
	\sv
	Since ${(-4)( -1)=4} $ and ${ (-4)+(-1)=-5} $, we have
	\[ b^2-5b+4 =(b-4)(b-1) \] 
}
\eks[3]{
Solve the inequality
\[ x^2-8x-9\leq0 \] \vs
\sv
Since $ 1(-9)=-9 $ and $ 1+(-9)=-8 $, we have
\[ x^2-8x-9=(x+1)(x-9) \]
We set $ f=(x+1)(x-9) $, and make a \outl{sign table}:
\fig{fortgnskj2}
The sign table shows the following:
\begin{itemize}
	\item The expression $ {x+1} $ is negative when  $ {x<-1} $, equals 0 when $ x=-1 $, and is positive when $ {x>-1} $.
	\item The expression $ {x-9} $ is negative when $ {x<9} $, equals zero 0 when $ x=9 $, and is positive when $ {x>9} $.
	\item Since $ f=(x+1)(x-9) $, \vspace{-5pt}
	\begin{center}
		\begin{tabular}{l}
			$ f>0 $ when $ {{x\in[-\infty, -1)} \cup (9, \infty]} $ \br
			$ f=0 $ when $ {x\in\{-1, 9\}} $ \br
			$ f<0 $ when $ x\in(-1, 9)$
		\end{tabular}
	\end{center}
\end{itemize}
Therefore, $ x^2-8x-9\leq0 $ when $ x\in[-1, 9] $.
}
\fork{\ref{a1a2} \aenato}{
We have
\algv{
(x+a_1)(x+a_2)&=x^2+xa_2+a_1x+a_1a_2 \\
&= x^2+(a_1+a_2)x+a_1a_2
}
If $ a_1+a_2=b $ og $ {a_1a_2=c}$, then
\nn{
(x+a_1)(x+a_2) = x^2+bx+c
}
}
\newpage
\section{Quadratic Equations}
\reg[Quadratic equation with constant term]{
	Given the equation
	\begin{equation}\label{abc}
		ax^2+bx+c=0
	\end{equation}
	where $a, b, c \in \mathbb{R}$. Then 
	\begin{flalign*}
		&&\qquad\qquad\qquad    x&= \frac{-b\pm\sqrt{b^2-4ac}}{2a} &&\qquad(\textit{abc}\text{\,-formula})
	\end{flalign*}
	If $x=x_1$ and $x=x_2$ are the solutions given by the \textit{abc}-formula, is
	\begin{equation}\label{abcfakt}
		ax^2+bx+c=a(x-x_1)(x-x_2)
	\end{equation}
}
\eks[1]{
	\abc{
		\item Solve the equation    $ 2x^2-7x+5=0 $.
		\item Factorize the expression on the left side in task a). 
	} \vs
	\sv \vs
	\abc{
		\item We use the \textit{abc}-formula. Then $a=2$, $b=-7$ and $c=5$. Now we get that
		\alg{
			x &=\frac{-(-7)\pm\sqrt{(-7)^2-4\cdot2\cdot5}}{2\cdot2} \br
			&=\frac{7\pm\sqrt{49-40}}{4} \br
			&=\frac{7\pm\sqrt{9}}{4}\br
			&=\frac{7\pm3}{4}    
		}
		Either is
		\[ x=\frac{7+3}{4}=\frac{10}{4}=\frac{5}{2} \]    
		or
		\[ x=\frac{7-3}{4}=1 \]    
		
		\item $ 2x^2-7x+5=2(x-1)\left(x-\frac{5}{2}\right) $     
	}
}
\eks[2]{
	Solve the equation 
	\[ x^2+3x-10=0 \]
	\sv
	We use the \textit{abc}-formula. Then $a=1$, $b=3$, and $c=-10$. Now we get that
	\alg{
		x&=\frac{-3\pm\sqrt{3^2-4\cdot1\cdot(-10)}}{2\cdot1}\br
		&=\frac{-3\pm\sqrt{9+40}}{2} \br
		&=\frac{-3\pm\sqrt{49}}{2} \br
		&=\frac{-3\pm7}{2}
	}
	Thus
	\[ x=-5\qquad\vee\qquad x=2 \]
}
\eks[3]{
	Solve the equation    
	\[ 4x^2-8x+1=0 \]
	\sv
	By the \textit{abc}-formula, we have that
	\alg{
		x&=\frac{8\pm\sqrt{(-8)^2-4\cdot4\cdot1}}{2\cdot4} \\
		&=\frac{8\pm\sqrt{64-16}}{8} \br
		&=\frac{8\pm\sqrt{48}}{8} \br
		&=\frac{8\pm4\sqrt{3}}{8} \br
		&=\frac{2\pm\sqrt{3}}{2}
	}
	Thus
	\[ x=\frac{2+\sqrt{3}}{2}\quad\vee\quad x=\frac{2-\sqrt{3}}{2} \]
}

\newpage
\fork{Quadratic Equations}{
	Given the equation
	\[ ax^2+bx+c=0 \]
	We start by rewriting the equation:
	\[ x^2+\frac{b}{a}x+\frac{c}{a}=0 \]
	Then we make a perfect square and use the conjugate root theorem to factorize the expression:
	\alg{
		x^2+\frac{b}{a}x+\frac{c}{a}&=x^2+2\cdot\frac{b}{2a}x+\frac{c}{a}\\
		&=\left(x+\frac{b}{2a}\right)^2-\frac{b^2}{4a^2}+\frac{c}{a} \\
		&=\left(x+\frac{b}{2a}\right)^2-\frac{b^2-4ac}{4a^2} \\
		&=\left(x+\frac{b}{2a}\right)^2-\left(\sqrt{\frac{b^2-4ac}{4a^2}}\,\right)^2 \br
		&=\left(x+\frac{b}{2a}+\frac{\sqrt{b^2-4ac}}{2a}\,\right)\left(x+\frac{b}{2a}-\frac{\sqrt{b^2-4ac}}{2a}\,\right)
	}    
	The expression above equals 0 when
	\[ x=\frac{-b+\sqrt{b^2-4ac}}{2a}\qquad\vee\qquad x=\frac{-b-\sqrt{b^2-4ac}}{2a} \]
}

\newpage
\section{Polynomial Division}
\subsection{Methods}
When two given numbers are not divisible by each other, we can use fractions to express the quotient. For example,
\begin{equation}\label{17div3}
	\frac{17}{3}=5+\frac{2}{3}
\end{equation}
The idea behind \eqref{17div3} is that we rewrite the numerator so that we bring out the part of 17 that is divisible by 3:
\[ \frac{17}{3}=\frac{5\cdot3+2}{3}=5+\frac{2}{3} 
\]
The same reasoning can be applied to fractions with polynomials, and then it's called \outl{polynomial division}.
\regv
\eks[1]{ \label{polydiveks1}
	Perform polynomial division on the expression
	\[ \frac{2x^2+3x-4}{x+5} \] 
	\sv
	\metode{Method 1}{0.4\linewidth} \\
	We do the following steps; starting with the highest power of $x$ in the numerator, we create expressions that are divisible by the denominator.
	\alg{
		\frac{2x^2+3x-4}{x+5}&=\frac{2x(x+5)-10x+3x-4}{x+5} \br
		&= 2x+\frac{-7x-4}{x+5} \br
		&= 2x+\frac{-7(x+5)+35-4}{x+5} \br
		&= 2x-7 +\frac{31}{x+5}
	}
	\newpage
	\metode{Method 2}{0.4\linewidth}
	(See the calculation under the points)
	\begin{enumerate}[label=\roman*)]
		\item  We observe that the term with the highest order of $x$ in the dividend is $2x^2$. This expression can be obtained by multiplying the dividend by $2x$. We write $2x$ to the right of the equals sign, and subtract $2x(x+5)=2x^2+10x$.
		\item The difference from point ii) is ${-7x-4}$. We can bring out the term with the highest order of $x$ by multiplying the dividend by $-7$. We write $-7$ to the right of the equals sign, and subtract $-7(x+5)=-7x-35$.
		\item The difference from point iii) is $31$. This is an expression that has a lower order of $x$ than the dividend, and thus we write $\frac{31}{x+5}$ to the right of the equals sign.
	\end{enumerate}
	\alg{
		\phantom{-}&(2x^2+3x-4):(x+5) =2x-7+\frac{31}{x+5} \\ 
		-&\underline{(2x^2+10x)} \\
		&\phantom{02x}-7x-4  \\
		&\phantom{xx}-\underline{(-7x-35)}\\
		&\phantom{aaaaaaaaaa''\,}31 
	}
}
\newpage
\eks[2]{
	Perform polynomial division on the expression
	\[ \frac{x^3-4x^2+9}{x^2-2} \]
	\sv
	\metode{Method 1}{0.4\linewidth}\\ \vs
	\alg{
		\frac{x^3-4x^2+9}{x^2-2}&= \frac{x(x^2-2)+2x-4x^2+9}{x^2-2} \br
		&=x+\frac{-4x^2+2x+9}{x^2-2}\br
		&=x+\frac{-4(x^2-2)-8+2x+9}{x^2-2}\br
		&= x-4+\frac{2x+1}{x^2-2}
	}
	\metode{Method 2}{0.4\linewidth} \\ \vs
	\alg{
		\phantom{-}&(x^3-4x^2+9):(x^2-2) =x-4+\frac{2x+1}{x^2-2} \\ 
		-&\underline{(x^3-2x)} \\
		&\phantom{02x}-4x^2+2x+9  \\
		&\phantom{'}-\underline{(-\,4x^2+8)}\\
		&\phantom{aaaaaaaaaa'}2x+1 
	}
}\newpage

\eks[3]{ \label{polydiveks3}
	Perform polynomial division on the expression
	\[ \frac{x^3-3x^2-6x+8 }{x-4}\]
	
	\sv
	\metode{Method 1}{0.4\linewidth} \\ \vs
	\alg{
		\frac{x^3-3x^2-6x+8 }{x-4} &= \frac{x^2(x-4)+4x^2-3x^2-6x+8}{x-4} \br
		&= x^2 +\frac{x^2-6x+8}{x-4} \br
		&= x^2+\frac{x(x-4)+4x-6x+8}{x-4}\br
		&=x^2+x+\frac{-2x+8}{x-4} \br
		&= x^2+x-2
	}
	\metode{Method 2}{0.4\linewidth} \\ \vs
	\alg{
		\phantom{-}&(x^3-3x^2-6x+8):(x-4) =x^2+x-2 \\ 
		-&\underline{(x^3-4x^2)} \\
		&\phantom{02xx''',}x^2-6x+8  \\
		&\phantom{''}-\underline{(-\,x^2+4x)}\\
		&\phantom{aaaaaa''''}-2x+8 \\
		&\phantom{aaaa,,,}-\underline{(-2x+8)}\\
		&\phantom{aaaaaaaaaaaaaa,}0
	}
}

\newpage
\subsection{Divisibility and Factors}
The examples on pages \pageref{polydiveks1}\,-\,\pageref{polydiveks3} point to some important relationships that apply to general cases:\regv
\reg[Polynomial Division \label{polydiv}]{
	Let $ A_k $ denote a polynomial $ A $ of degree $ k $. Given the polynomial $ P_m $, then there exist polynomials $ Q_n $, $ S_{m-n} $, and $ R_{n-1} $, where $ m\geq n>0 $, such that
	\begin{equation}\label{polydiveq}
		\frac{P_m}{Q_n}=S_{m-n}+\frac{R_{n-1}}{Q_n}    
	\end{equation}
}
\spr{
	If $ R_{n-1}=0 $, we say that $ P_m $ is \outl{divisible} by $ Q_n $.
} 
\eks[1]{
	Investigate whether the polynomials are divisible by $ {x-3} $.
	\abc{
		\item $P(x)= x^3+5x^2-22x-56 $
		\item $K(x)= x^3+6x^2-13x-42 $
	}
	\sv \vs
	\abc{
		\item By polynomial division, we find that
		\[ \frac{P}{x-3}=x^2+8x+2-\frac{50}{x-3} \]
		Thus, $ P $ is \textsl{not} divisible by $ x-3 $.
		\item By polynomial division, we find that
		\[ \frac{K}{x-3}=x^2+9x+14 \]
		Thus, $ K $ is divisible by $ x-3 $.
	}
}


\newpage
\reg[Factors in Polynomials \label{polyfakt}]{
	Given a polynomial $ P(x) $ and a constant $ a $. Then we have that
	\begin{equation}
		P \text{ is divisible by } x-a  \Longleftrightarrow P(a)=0 
	\end{equation}
	If this is true, there exists a polynomial $ S(x) $ such that
	\begin{equation}\label{PfaktS}
		P=(a-x)S
	\end{equation}
}
\eks[1]{
	Given the polynomial
	\[ P(x)= x^3-3x^2-6x+8\]
	\abc{
		\item Show that $ x=1 $ solves the equation $ P=0 $.
		\item Factorize $ P $. 
	}
	\sv
	
	\abc{
		\item We investigate $ P(1) $:
		\alg{
			P(1)&=1^3-3\cdot1^2-6\cdot1+8 \\
			&= 0
		}
		Thus, $ P=0 $ when $ x=1 $.
		\item Since $ P(1)=0 $, $ x-1 $ is a factor in $ P $. By polynomial division, we find that
		\[ P=(x-1)(x^2-2x-8) \]
		Since $ 2(-4)=-8 $ and $ -4+2=-2 $, we have
		\[ x^2-2x-8=(x+2)(x-4) \]
		This means that
		\[ P=(x-1)(x+2)(x-4) \]
	}
}

\section{Euler's Number}
\outl{Euler's Number}\index{Euler's number}\index{e} is a constant of such significant importance in mathematics that it has been given its own letter; \sym{$ e $}. The number is irrational\footnote{And \net{https://en.wikipedia.org/wiki/Transcendental_number}{transcendental}.}, and the first ten digits are
\st{\[ e=2.718281828... \]}
The most fascinating properties of this number become apparent when investigating the function ${f(x)= e^x}$. This is an exponential function of such importance that it is simply known as \\\outl{the exponential function}. This function will be examined more closely in \refved{eulersnumberprop} and \refkap{Derivation}.
\fig{ekspfunk}

\section{Logarithms}
In \mb, we looked at powers, which consist of a base and an exponent. A \outl{logarithm} is a mathematical operation relative to a number. If a logarithm is relative to the base of a power, the operation will result in the exponent.\vsk

The logarithm relative to 10 is written $ \log_{10} $. For example,
\[ \log_{10} 10^2 = 2 \]
Furthermore, for example,
\[ \log_{10} 1000= \log_{10} 10^3 = 3\]
Consequently, we can write
\[ 1000 = 10^{\log_{10} 1000} \]
With the power rules as a starting point (see \mb), many rules for logarithms can be derived.\regv
\regdef[Logarithms]{
	Let $ \log_a $ denote the logarithm relative to $ {a>0}$. For $ m\in\mathbb{R} $, then
	\begin{equation}
		\log_a a^m = m
	\end{equation}
	Alternatively, we can write \vs
	\begin{equation}
		m=a^{\log_{a} m}
	\end{equation}
}
\eks[1]{ \vsb
	\[ \log_5 5^9 = 9 \]
}
\eks[2]{ \vsb
	\[ 3 = 8^{\log_8 3} \]
}
\spr{
	\sym{$ \log_{10}$} is often written as \sym{$ \log $}, while \sym{$ \log_e $} is often written as \sym{$ \ln $} or (!) \sym{log}. When using digital aids to find logarithm values, it is therefore important to check what the base is. In this book, we shall write \sym{$ \log_e $} as \sym{$ \ln $}.    \\[5pt]
	
	The logarithm with $ e $ as the base is called the \outl{natural logarithm}.
}
\regv
\eks[3]{ \vsb
	\[ \log 10^7 = 7 \]
}
\eks[4]{ \vsb
	\[ \ln e^{-3} = -3 \]
} \vsk

\reg[Logarithm Rules]{
	\notesm{The logarithm rules are here given by the natural logarithm. The same rules will apply by replacing $ \ln $ with $ \log_a $, and $ e $ with $ a $, for $ a>0 $.
	} \vsk
	
	For $ {x, y>0} $, we have that \vs
	\begin{align}
		\ln e&= 1\label{loga}\vn    
		\ln 1 &= 0 \label{log1}\vn
		\ln (xy)&=\ln x + \ln y \label{logxy} \vn
		\ln \left(\frac{x}{y}\right)&= \ln x - \ln y \label{logxdivy} 
	\end{align}
	For a number $ y $ and $ x>0 $, is
	\begin{equation}
		\ln x^y = y \ln x \label{logxexpy}
	\end{equation}
}
\eks[1]{\vsb
	\[ \ln\left(ex^5\right) =\ln e+\ln x^5=1+5\ln x\]
}
\eks[2]{\vsb
	\[ \ln \frac{1}{2} = \ln 1 - \ln 2 = -\ln 2\]
}
\newpage
\fork{Logarithm Rules}{
	\textbf{Equation} \textbf{(\ref{loga})}        
	\[ \ln e= \ln e^1=1  \]
	
	\textbf{Equation} \textbf{(\ref{log1})}
	\[ \ln 1 = \ln e^0 =0  \]
	
	\textbf{Equation} \textbf{(\ref{logxy})}\\
	For $ m, n\in \mathbb{R} $, we have that
	\alg{
		\ln e^{m+n}&=m+n \\
		&= \ln e^m + \ln e^n
	}
	We set\footnote{It is taken for granted here that all positive numbers different from 0 can be expressed as a power.} $ {x=e^m} $ and $ {y=e^n} $. Since $ {\ln e^{m+n}=\ln (e^m\cdot e^n)}$, then
	\[ \ln(x y)=\ln x+\ln y \]
	
	\textbf{Equation} \textbf{(\ref{logxdivy})}\\
	By examining $ \ln a^{m-n} $, and by setting $ {y=a^{-n}} $, the\\ explanation is analogous to that given for equation \eqref{logxy}.\vsk
	
	\textbf{Equation} \textbf{(\ref{logxexpy})} \os
	Since $ x=e^{\ln x} $ and\footnote{See power rules in \mb.} $ \left(e^{\ln x}\right)^y = e^{y \ln x} $, we have that
	\algv{
		\ln x^y &= \ln e^{y\ln x}  \\
		&= y\ln x
	}
}
\newpage
\section{Explanations}
\fork{\ref{polydiv} Polynomial Division}{
	Given the polynomials
	\alg{
		&P_m\text{, where }ax^m\text{ is the term with the highest degree}\vn
		&Q_{n}\text{, where } bx^n\text{ is the term with the highest degree}    
	}
	Then we can write
	\begin{equation*}
		P_m=\frac{a}{b}x^{m-n}Q_n-\frac{a}{b}x^{m-n}Q_n+P_m
	\end{equation*}
	We set $ {U=-\frac{a}{b}x^{m-n}Q_n+P_m} $, and note that $ U $ \\necessarily has a degree lower or equal to $ m-1 $. Further, we have that
	\begin{equation} \label{polydivu}
		\frac{P_m}{Q_n}=\frac{a}{b}x^{m-n}+\frac{U}{Q_n}
	\end{equation}
	Let's call the first and the second term on the right side in \eqref{polydivu} respectively a "power term" and a "remaining fraction". By following the procedure that led us to \eqref{polydivu}, we can also express $ \frac{U}{Q_n} $ by a "power term" and a "remaining term". This "power term" will have a degree less or equal to $ {m-1} $, while the numerator in the "remaining term" will have a degree less or equal to $ {m-2} $. By applying \eqref{polydivu} we can continually create new "power terms" and "remaining terms" until we have a "remaining term" with a degree of $ {n-1} $.
}
\newpage
\fork{\ref{polyfakt} Factorization of Polynomials}{
	\textbf{\boldmath $ P $ is divisible by $ x-a \Rightarrow P(a)=0 $.}\os
	For a polynomial $ S $, we have from \eqref{polydiveq} that
	\alg{
		\frac{P}{x-a}&= S \\
		P &= (x-a)S
	}
	Then obviously $ x=a $ is a solution for the equation $ P=0 $.\vsk
	
	\textbf{\boldmath $ P $ is divisible by $ x-a \Leftarrow P(a)=0 $.}\os
	From \eqref{polydiveq}, there exists a polynomial $ S $ and a constant $ R $ such that
	\alg{
		\frac{P}{x-a}&=S+\frac{R}{x-a} \br
		P &= (x-a)S+R
	}
	Since $ P(a)=0 $, $ 0=R $, and then $ P $ is divisible by $ x-a $.
}

\end{document}