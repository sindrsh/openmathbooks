\documentclass[english,hidelinks,pdftex, 11 pt, class=report,crop=false]{standalone}
\usepackage[T1]{fontenc}
\usepackage[utf8]{luainputenc}
\usepackage{lmodern} % load a font with all the characters
\usepackage{geometry}
\geometry{verbose,a4paper, inner=0cm, outer=0 cm, bmargin=2cm, tmargin=1cm}
%\textwidth=12cm
\setlength{\parindent}{0bp}
\usepackage{import}
\usepackage[subpreambles=false]{standalone}
\usepackage{amsmath}
\usepackage{amssymb}
\usepackage{esint}
\usepackage{babel}
\usepackage{tabu}
\usepackage[dvipsnames, table]{xcolor}
\usepackage{cancel}
\makeatother
\makeatletter
\usepackage{datetime2}
\usepackage{titlesec}
\usepackage[many]{tcolorbox}

% Eheter
\newcommand{\enh}[1]{\,\textrm{#1}}
%referances
\newcommand{\net}[2]{{\color{blue}\href{#1}{#2}}}

%Spaces
\newcommand{\vsk}{\\[12pt]}
\newcommand{\vs}{\vspace{-12pt}}

% Tabell for opplegg

\newcommand{\ovlist}[1]{
\vspace{-16pt}
\begin{itemize}
	#1
\end{itemize}
}

% Chapters and sections
\titleformat{\section}[block]{\bfseries}{\hspace{3cm}\thesection}{5pt}{}
\titleformat{\subsection}[block]{\bfseries}{\hspace{3cm}\thesection}{5pt}{}
\newcommand{\sectionbreak}{\clearpage} % New page on each section
 

\newlength{\mywidth}
\setlength{\mywidth}{14cm}

\newcommand{\cont}[1]{
\begin{tcolorbox}[center, boxrule=0.0 mm, width=\mywidth,arc=0mm,enhanced jigsaw,,colback=white,breakable]
#1	
\end{tcolorbox}
}

\newcommand{\info}[5]{
\begin{tcolorbox}[center, boxrule=0.1 mm, width=\mywidth,arc=0mm,enhanced jigsaw,breakable,colback=yellow!5]	
	
	\footnotesize
	\textbf{Øvingsområde}\\[5pt] #1 
	
	\textbf{Utstyr}\\ #2  \\
	
	\begin{tabular}{@{} p{4cm} p{4cm} l} 
		\textbf{Tid} & \textbf{Elevinndeling} & \textbf{Læringsarena} \\
		#3  & #4 & #5
	\end{tabular} 
\end{tcolorbox}	
}

\newcommand{\gjen}[1]{\begin{tcolorbox}[center,boxrule=0.1 mm, width=\mywidth,arc=0mm,colback=blue!3] {\large \textbf{Gjennomføring} \vspace{5 pt}}\newline #1  \end{tcolorbox}\vspace{-5pt}}
\newcommand{\eks}[1]{\begin{tcolorbox}[center,boxrule=0.1 mm, width=\mywidth,arc=0mm,colback=green!3] {\large \textbf{Eksempel} \vspace{5 pt}}\newline #1  \end{tcolorbox}\vspace{-5pt}}

\newcounter{opl}
%\numberwithin{opl}{article}


\newcommand{\opl}[1]{
\newpage
{\refstepcounter{opl} %\phantomsection 
\large \textbf{\theopl \;#1} \vsk}
}

% Headlines
\newcommand{\fork}{\textbf{Forkunnskapar}\\}
\newcommand{\forb}{\textbf{Forberedelsar}\\}
\newcommand{\opgvr}{\textbf{Oppgaver}}



%colors
\newcommand{\colr}[1]{{\color{red} #1}}
\newcommand{\colb}[1]{{\color{blue} #1}}
\newcommand{\colo}[1]{{\color{orange} #1}}
\newcommand{\colc}[1]{{\color{cyan} #1}}
\definecolor{projectgreen}{cmyk}{100,0,100,0}
\newcommand{\colg}[1]{{\color{projectgreen} #1}}

% Lister med bokstavar
\usepackage[inline]{enumitem}
% Opg
\newcommand{\abc}[1]{
	\begin{enumerate}[label=\alph*),leftmargin=18pt]
		#1
	\end{enumerate}
}

\usepackage[]{hyperref}

\newcommand{\note}{Merk}
\newcommand{\notesm}[1]{{\footnotesize \textsl{\note:} #1}}
\newcommand{\ekstitle}{Eksempel }
\newcommand{\sprtitle}{Språkboksen}
\newcommand{\expl}{forklaring}
\newcommand{\pyt}{Pytagoras' setning}
\newcommand\sv{\vsk \textbf{Svar} \vspace{4 pt}\\}

%references
\newcommand{\reftab}[1]{\hrs{#1}{tabell}}
\newcommand{\rref}[1]{\hrs{#1}{regel}}
\newcommand{\dref}[1]{\hrs{#1}{definisjon}}
\newcommand{\refkap}[1]{\hrs{#1}{kapittel}}
\newcommand{\refsec}[1]{\hrs{#1}{seksjon}}
\newcommand{\refdsec}[1]{\hrs{#1}{delseksjon}}
\newcommand{\refved}[1]{\hrs{#1}{vedlegg}}
\newcommand{\eksref}[1]{\textsl{#1}}
\newcommand\fref[2][]{\hyperref[#2]{\textsl{figur \ref*{#2}#1}}}
\newcommand{\refop}[1]{{\color{blue}Oppgave \ref{#1}}}
\newcommand{\refops}[1]{{\color{blue}oppgave \ref{#1}}}


%Algebra
\newcommand{\kvadset}{Kvadratsetningene}
\newcommand{\aenato}{Sum-produkt-metoden}

% Geometry
\newcommand{\hlikb}{Midtnormalen i en likebeint trekant}
\newcommand{\arealsetn}{Arealsetningen}
\newcommand{\trkmedian}{Median}
\newcommand{\midtrk}{Midtnormal (i trekant)}
\newcommand{\innskrsirk}{Innskrevet sirkel}
\newcommand{\cossetn}{Cosinussetningen}
\newcommand{\perfvink}{Sentral- og periferivinkel}
\newcommand{\tang}{Tangent}

% Derivative
\newcommand{\derel}{Den deriverte av elementære funksjoner}
\newcommand{\divder}{Divisjonsregelen}
\newcommand{\kjernereg}{Kjerneregelen}
\newcommand{\prodregder}{Produktregelen}
\newcommand{\lhop}{L'Hopitals regel}

% Funksjonsdrofting
\newcommand{\monder}{Monotoniegenskaper og den deriverte}
\newcommand{\fderekstr}{$ \bm{f'=0} $ for lokale ektstremalpunkt}
\newcommand{\andredertest}{Andrederiverttesten}

% Vectors
\newcommand{\detar}{Arealformler med determinanter}
\newcommand{\avstpunktlin}{Avstand mellom punkt og linje}

%Appendix
\newcommand{\rolle}{Rolles teorem}
\newcommand{\meanval}{Middelverdisetningen}

% Solutions manual
\newcommand{\selos}{Se løsningsforslag.}

\begin{document}
\section{Grenseverdier}
Si at vi starter med verdien 0.9, og deretter stadig legger til 9 som bakerste siffer. Da får vi verdiene 0.9, 0.99, 0.999 og så videre. Ved å legge til 9 som bakerste siffer på denne måten, \textsl{kan vi komme så nærme vi måtte ønske $ - $ men aldri nå eksakt $ - $ verdien 1}. Det å \\''komme så nærme vi måtte ønske $ - $ men aldri nå eksakt $ - $ en verdi'' vil vi heretter kalle å ''gå mot en verdi''. Metoden vi akkurat beskrev kan vi se på som en metode for å \textsl{gå mot} 1. Vi kan da si at \outl{grense-verdien} til denne metoden er 1. 
For å indikere en grenseverdi skriver vi \sym{$ \lim $}.\vsk

Det er viktig å tenke over at vi kan gå mot et tall fra to sider; fra venstre eller fra høgre på tallinjen. Med en metode som gir oss verdiene 0.9, 0.99, 0.999 og så videre, nærmer vi oss 1 fra venstre. Lager vi oss en metode som gir verdiene 1.1, 1.01, 1.001 og så videre, nærmer vi oss 1 fra høgre. Dette vises ved å markere \sym{$ + $} eller $ \sym{$ - $} $ over tallet vi går mot. 

\regv

\regdef[Grenseverdier]{ \vsb
\alg{
x\to a^+ &= x \text{ går mot } a \text{ fra høgre} \br
x\to a^- &= x \text{ går mot } a \text{ fra venstre} \br
x\to a &=  x \text{ går mot } a \text{ (fra både høgre og venstre)}\\[12pt]
\lim\limits_{x\to a} f(x) &= \text{grenseverdien til } f \text{ når } x \text{ går mot } a \\
&= \text{verdien }f\text{ går mot når }x \text{ går mot }a
} 
}

\spr{
Å gå mot en verdi fra høgre/venstre kalles også å gå mot en verdi ovenfra/nedenfra.
}
\info{\note}{
\sym{$ x\to a$} omfatter de to tilfellene \sym{$ x\to a^+ $} og \sym{$ x\to a^- $}. Ofte vil disse være så like at vi kan behandle \sym{$ x\to a $} som ett tilfelle.
}
\newpage

\info{En utvidelse av \sym{$ = $}}{
Det litt paradoksale med grenserverdier hvor $ x $ går mot $ a $, er at vi ofte ender opp med å erstatte $ x $ med $ a $, selv om vi per definisjon har at $ x\neq a $. For eksempel er
\begin{equation} \label{limxto2}
	\lim_{x\to 2}(x+1)=2+1=3 
\end{equation}
Det er verd å filosofere litt over likhetene i \eqref{limxto2}. Når $ x $ går mot 2, vil $ x $ aldri bli eksakt lik 2. Dette betyr at $ {x+1} $ aldri kan bli \textsl{eksakt lik} 3. Men \textsl{jo nærmere} $ x $ er lik 2, \textsl{jo nærmere} er $ {x+1} $ lik 3. Med andre ord går $ {x+1} $ mot 3 når $ x $ går mot 2. Likheten i \eqref{limxto2} viser altså ikke til et uttrykk som er \textsl{eksakt lik} en verdi, men et uttrykk som \textsl{går eksakt mot} en verdi. Dette gjør altså at grenseverdier bringer en noe utvidet forståelse av \sym{$ = $}.
}
\eks[1]{
Gitt $ f(x)=\frac{x^2+2x-3}{x-1} $. Finn $ \lim\limits_{x\to 1} f(x) $. 

\sv
Når $ x\neq1 $, har vi at
\alg{
f(x)&= \frac{(x-1)(x+3)}{x-1} \\
&= x+3
}
Dette betyr at 
\alg{
\lim\limits_{x\to 1} f(x) &= \lim\limits_{x\to 1} x+3 \\
&= 4
} 
}
\section{Kontinuitet}
\regdef[Kontinuitet]{
Gitt en funksjon $ f(x) $ og et tall $ c $. Hvis $ f(c) $ eksisterer, er $ f $ \outl{kontinuerlig} for $ x=c $ hvis
\begin{equation}\label{kont}
	\lim\limits_{x\to c} f(x)=f(c) 
\end{equation}
Hvis \eqref{kont} er ugyldig, er $ f $ \outl{diskontinuerlig} for $ x=c $.
}
\spr{Hvis en funksjon $ f(x) $ er kontinuerlig for alle $ x $, kalles $ f $ en \outl{kontinuerlig funksjon}.}
\eks[1 \label{konteks1}]{
Undersøk om funksjonene er kontinuerlige for $ x=2 $.
\abc{
\item \begin{equation}f(x)= \left\lbrace{
		\begin{array}{rcr}
			x+4 &,&x<2 \\
			-3x+12   &,& x\geq 2
		\end{array}
	}\right. 
\end{equation}	
\item \begin{equation}g(x)= \left\lbrace{
		\begin{array}{rcr}
			x+1 &,&x\leq2 \\
			-x+6   &,& x> 2
		\end{array}
	}\right. 
\end{equation}	
}

\sv \vs
\abc{
\item Vi har at
\algv{
\lim\limits_{x\to2^+} f(x)&=f(2)= -3\cdot2+12 = 6\br
\lim\limits_{x\to2^-} f(x)&= 2+4 = 6
}
Altså er $ f $ kontinuerlig for $ x=2 $.
\item Vi har at
\algv{
	\lim\limits_{x\to2^-} g(x)&=g(2)= 2+1 = 3\br
	\lim\limits_{x\to2^+} g(x)&= -2+6 = 4
}
Altså er $ g $ \textsl{ikke} kontinuerlig for $ x=2 $.
}
}
\newpage
\info{Visualisering av kontinuitet}{
Visuelt kan vi skille mellom kontinuerlige og diskontinuerlige funksjoner slik; kontinuerlige funksjoner har sammenhengende grafer, diskontinuerlige funksjoner har det ikke. Et utsnitt av grafene til funksjonene fra \eksref{Eksempel 1} på side \pageref{konteks1} ser slik ut:
\fig{konteks1}
Grafer fungerer utmerket til å avgjøre hvilke funskjoner vi \textsl{forventer} å være kontinuerlige eller ikke, men er aldri gyldige som et bevis for dette. 
}

\newpage
\end{document}