\documentclass[english,hidelinks,pdftex, 11 pt, class=report,crop=false]{standalone}
\usepackage[T1]{fontenc}
\usepackage[utf8]{luainputenc}
\usepackage{lmodern} % load a font with all the characters
\usepackage{geometry}
\geometry{verbose,a4paper, inner=0cm, outer=0 cm, bmargin=2cm, tmargin=1cm}
%\textwidth=12cm
\setlength{\parindent}{0bp}
\usepackage{import}
\usepackage[subpreambles=false]{standalone}
\usepackage{amsmath}
\usepackage{amssymb}
\usepackage{esint}
\usepackage{babel}
\usepackage{tabu}
\usepackage[dvipsnames, table]{xcolor}
\usepackage{cancel}
\makeatother
\makeatletter
\usepackage{datetime2}
\usepackage{titlesec}
\usepackage[many]{tcolorbox}

% Eheter
\newcommand{\enh}[1]{\,\textrm{#1}}
%referances
\newcommand{\net}[2]{{\color{blue}\href{#1}{#2}}}

%Spaces
\newcommand{\vsk}{\\[12pt]}
\newcommand{\vs}{\vspace{-12pt}}

% Tabell for opplegg

\newcommand{\ovlist}[1]{
\vspace{-16pt}
\begin{itemize}
	#1
\end{itemize}
}

% Chapters and sections
\titleformat{\section}[block]{\bfseries}{\hspace{3cm}\thesection}{5pt}{}
\titleformat{\subsection}[block]{\bfseries}{\hspace{3cm}\thesection}{5pt}{}
\newcommand{\sectionbreak}{\clearpage} % New page on each section
 

\newlength{\mywidth}
\setlength{\mywidth}{14cm}

\newcommand{\cont}[1]{
\begin{tcolorbox}[center, boxrule=0.0 mm, width=\mywidth,arc=0mm,enhanced jigsaw,,colback=white,breakable]
#1	
\end{tcolorbox}
}

\newcommand{\info}[5]{
\begin{tcolorbox}[center, boxrule=0.1 mm, width=\mywidth,arc=0mm,enhanced jigsaw,breakable,colback=yellow!5]	
	
	\footnotesize
	\textbf{Øvingsområde}\\[5pt] #1 
	
	\textbf{Utstyr}\\ #2  \\
	
	\begin{tabular}{@{} p{4cm} p{4cm} l} 
		\textbf{Tid} & \textbf{Elevinndeling} & \textbf{Læringsarena} \\
		#3  & #4 & #5
	\end{tabular} 
\end{tcolorbox}	
}

\newcommand{\gjen}[1]{\begin{tcolorbox}[center,boxrule=0.1 mm, width=\mywidth,arc=0mm,colback=blue!3] {\large \textbf{Gjennomføring} \vspace{5 pt}}\newline #1  \end{tcolorbox}\vspace{-5pt}}
\newcommand{\eks}[1]{\begin{tcolorbox}[center,boxrule=0.1 mm, width=\mywidth,arc=0mm,colback=green!3] {\large \textbf{Eksempel} \vspace{5 pt}}\newline #1  \end{tcolorbox}\vspace{-5pt}}

\newcounter{opl}
%\numberwithin{opl}{article}


\newcommand{\opl}[1]{
\newpage
{\refstepcounter{opl} %\phantomsection 
\large \textbf{\theopl \;#1} \vsk}
}

% Headlines
\newcommand{\fork}{\textbf{Forkunnskapar}\\}
\newcommand{\forb}{\textbf{Forberedelsar}\\}
\newcommand{\opgvr}{\textbf{Oppgaver}}



%colors
\newcommand{\colr}[1]{{\color{red} #1}}
\newcommand{\colb}[1]{{\color{blue} #1}}
\newcommand{\colo}[1]{{\color{orange} #1}}
\newcommand{\colc}[1]{{\color{cyan} #1}}
\definecolor{projectgreen}{cmyk}{100,0,100,0}
\newcommand{\colg}[1]{{\color{projectgreen} #1}}

% Lister med bokstavar
\usepackage[inline]{enumitem}
% Opg
\newcommand{\abc}[1]{
	\begin{enumerate}[label=\alph*),leftmargin=18pt]
		#1
	\end{enumerate}
}

\usepackage[]{hyperref}
\input{../sp_preamb_bm}

\begin{document}
\section{Grenseverdier}
Si at vi starter med verdien 0.9, og deretter stadig legger til 9 som bakerste siffer. Da får vi verdiene 0.9, 0.99, 0.999 og så videre. Ved å legge til 9 som bakerste siffer på denne måten, \textsl{kan vi komme så nærme vi måtte ønske $ - $ men aldri nå eksakt $ - $ verdien 1}. Det å ''komme så nærme vi måtte ønske $ - $ men aldri nå eksakt $ - $ en verdi'' vil vi heretter kalle å ''gå mot en verdi''. Metoden vi akkurat beskrev kan vi altså se på som en metode for å \textsl{gå mot} 1. Vi kan da si at \textit{grenseverdien} til denne metoden er 1. 
For å indikere en grenseverdi skriver vi \sym{$ \lim $}, og for å vise til at en variabel $ x $ går mot et tall $ a $ skriver vi \sym{$ x\to a $}:\regv

\reg[Grenseverdien til en funksjon]{ \vsb
\alg{
\lim\limits_{x\to a} f(x) &= \text{grenseverdien til } f \text{ når } x \text{ går mot } a \\
&= \text{verdien }f\text{ går mot når }x \text{ går mot }a
} 
}
\regv

\info{En utvidelse av \sym{$ = $}}{
Det litt paradoksale med grenserverdier hvor $ x $ går mot $ a $, er at vi ofte ender opp med å erstatte $ x $ med $ a $, selv om vi per definisjon har at $ x\neq a $. For eksempel skriver vi at
\begin{equation} \label{limxto2}
	\lim_{x\to 2}(x+1)=2+1=3 
\end{equation}
Det er verd å filosofere litt over likhetene i \eqref{limxto2}. Når $ x $ går mot 2, vil $ x $ aldri bli eksakt lik 2. Dette betyr at $ {x+1} $ aldri kan bli \textsl{eksakt lik} 3. Men \textsl{jo nærmere} $ x $ er lik 2, \textsl{jo nærmere} er $ {x+1} $ lik 3. Med andre ord går $ {x+1} $ mot 3 når $ x $ går mot 2. Likheten i \eqref{limxto2} viser altså ikke til et uttrykk som er \textsl{eksakt lik} en verdi, men et uttrykk som \textsl{går eksakt mot} en verdi. Dette gjør altså at grenseverdier bringer en noe utvidet forståelse av \sym{$ = $}.
}
\regv
\eks[1]{
Gitt $ f(x)=\frac{x^2+2x+1}{x-1} $. Finn $ \lim\limits_{x\to 1} f(x) $. 

\sv

Hensikten med denne oppgaven er å undersøke om $ f(x) $ går mot en bestemt verdi når $ x $ går mot 1, selv om $ f$ ikke
}


\end{document}