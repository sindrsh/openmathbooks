\documentclass[english,hidelinks,pdftex, 11 pt, class=report,crop=false]{standalone}
\usepackage[T1]{fontenc}
\usepackage[utf8]{luainputenc}
\usepackage{lmodern} % load a font with all the characters
\usepackage{geometry}
\geometry{verbose,a4paper, inner=0cm, outer=0 cm, bmargin=2cm, tmargin=1cm}
%\textwidth=12cm
\setlength{\parindent}{0bp}
\usepackage{import}
\usepackage[subpreambles=false]{standalone}
\usepackage{amsmath}
\usepackage{amssymb}
\usepackage{esint}
\usepackage{babel}
\usepackage{tabu}
\usepackage[dvipsnames, table]{xcolor}
\usepackage{cancel}
\makeatother
\makeatletter
\usepackage{datetime2}
\usepackage{titlesec}
\usepackage[many]{tcolorbox}

% Eheter
\newcommand{\enh}[1]{\,\textrm{#1}}
%referances
\newcommand{\net}[2]{{\color{blue}\href{#1}{#2}}}

%Spaces
\newcommand{\vsk}{\\[12pt]}
\newcommand{\vs}{\vspace{-12pt}}

% Tabell for opplegg

\newcommand{\ovlist}[1]{
\vspace{-16pt}
\begin{itemize}
	#1
\end{itemize}
}

% Chapters and sections
\titleformat{\section}[block]{\bfseries}{\hspace{3cm}\thesection}{5pt}{}
\titleformat{\subsection}[block]{\bfseries}{\hspace{3cm}\thesection}{5pt}{}
\newcommand{\sectionbreak}{\clearpage} % New page on each section
 

\newlength{\mywidth}
\setlength{\mywidth}{14cm}

\newcommand{\cont}[1]{
\begin{tcolorbox}[center, boxrule=0.0 mm, width=\mywidth,arc=0mm,enhanced jigsaw,,colback=white,breakable]
#1	
\end{tcolorbox}
}

\newcommand{\info}[5]{
\begin{tcolorbox}[center, boxrule=0.1 mm, width=\mywidth,arc=0mm,enhanced jigsaw,breakable,colback=yellow!5]	
	
	\footnotesize
	\textbf{Øvingsområde}\\[5pt] #1 
	
	\textbf{Utstyr}\\ #2  \\
	
	\begin{tabular}{@{} p{4cm} p{4cm} l} 
		\textbf{Tid} & \textbf{Elevinndeling} & \textbf{Læringsarena} \\
		#3  & #4 & #5
	\end{tabular} 
\end{tcolorbox}	
}

\newcommand{\gjen}[1]{\begin{tcolorbox}[center,boxrule=0.1 mm, width=\mywidth,arc=0mm,colback=blue!3] {\large \textbf{Gjennomføring} \vspace{5 pt}}\newline #1  \end{tcolorbox}\vspace{-5pt}}
\newcommand{\eks}[1]{\begin{tcolorbox}[center,boxrule=0.1 mm, width=\mywidth,arc=0mm,colback=green!3] {\large \textbf{Eksempel} \vspace{5 pt}}\newline #1  \end{tcolorbox}\vspace{-5pt}}

\newcounter{opl}
%\numberwithin{opl}{article}


\newcommand{\opl}[1]{
\newpage
{\refstepcounter{opl} %\phantomsection 
\large \textbf{\theopl \;#1} \vsk}
}

% Headlines
\newcommand{\fork}{\textbf{Forkunnskapar}\\}
\newcommand{\forb}{\textbf{Forberedelsar}\\}
\newcommand{\opgvr}{\textbf{Oppgaver}}



%colors
\newcommand{\colr}[1]{{\color{red} #1}}
\newcommand{\colb}[1]{{\color{blue} #1}}
\newcommand{\colo}[1]{{\color{orange} #1}}
\newcommand{\colc}[1]{{\color{cyan} #1}}
\definecolor{projectgreen}{cmyk}{100,0,100,0}
\newcommand{\colg}[1]{{\color{projectgreen} #1}}

% Lister med bokstavar
\usepackage[inline]{enumitem}
% Opg
\newcommand{\abc}[1]{
	\begin{enumerate}[label=\alph*),leftmargin=18pt]
		#1
	\end{enumerate}
}

\usepackage[]{hyperref}

% note
\newcommand{\note}{Note}
\newcommand{\notesm}[1]{{\footnotesize \textsl{\note:} #1}}
\newcommand{\selos}{See the solutions manual.}

\newcommand{\texandasy}{The text is written in \LaTeX\ and the figures are made with the aid of Asymptote.}

\newcommand{\rknut}{Calculate.}
\newcommand\sv{\vsk \textbf{Answer} \vspace{4 pt}\\}
\newcommand{\ekstitle}{Example }
\newcommand{\sprtitle}{The language box}
\newcommand{\expl}{explanation}

% answers
\newcommand{\mulansw}{\notesm{Multiple possible answers.}}	
\newcommand{\faskap}{Chapter}

% exercises
\newcommand{\opgt}{\newpage \phantomsection \addcontentsline{toc}{section}{Exercises} \section*{Exercises for Chapter \thechapter}\vs \setcounter{section}{1}}

% references
\newcommand{\reftab}[1]{\hrs{#1}{Table}}
\newcommand{\rref}[1]{\hrs{#1}{Rule}}
\newcommand{\dref}[1]{\hrs{#1}{Definition}}
\newcommand{\refkap}[1]{\hrs{#1}{Chapter}}
\newcommand{\refsec}[1]{\hrs{#1}{Section}}
\newcommand{\refdsec}[1]{\hrs{#1}{Subsection}}
\newcommand{\refved}[1]{\hrs{#1}{Appendix}}
\newcommand{\eksref}[1]{\textsl{#1}}
\newcommand\fref[2][]{\hyperref[#2]{\textsl{Figure \ref*{#2}#1}}}
\newcommand{\refop}[1]{{\color{blue}Exercise \ref{#1}}}
\newcommand{\refops}[1]{{\color{blue}Exercise \ref{#1}}}

%%% SECTION HEADLINES %%%

% Our numbers
\newcommand{\likteikn}{The equal sign}
\newcommand{\talsifverd}{Numbers, digits and values}
\newcommand{\koordsys}{Coordinate systems}

% Calculations
\newcommand{\adi}{Addition}
\newcommand{\sub}{Subtraction}
\newcommand{\gong}{Multiplication}
\newcommand{\del}{Division}

%Factorization and order of operations
\newcommand{\fak}{Factorization}
\newcommand{\rrek}{Order of operations}

%Fractions
\newcommand{\brgrpr}{Introduction}
\newcommand{\brvu}{Values, expanding and simplifying}
\newcommand{\bradsub}{Addition and subtraction}
\newcommand{\brgngheil}{Fractions multiplied by integers}
\newcommand{\brdelheil}{Fractions divided by integers}
\newcommand{\brgngbr}{Fractions multiplied by fractions}
\newcommand{\brkans}{Cancelation of fractions}
\newcommand{\brdelmbr}{Division by fractions}
\newcommand{\Rasjtal}{Rational numbers}

%Negative numbers
\newcommand{\negintro}{Introduction}
\newcommand{\negrekn}{The elementary operations}
\newcommand{\negmeng}{Negative numbers as amounts}

%Calculation methods
\newcommand{\delmedtihundre}{Deling med 10, 100, 1\,000 osv.}

% Geometry 1
\newcommand{\omgr}{Terms}
\newcommand{\eignsk}{Attributes of triangles and quadrilaterals}
\newcommand{\omkr}{Perimeter}
\newcommand{\area}{Area}

%Algebra 
\newcommand{\algintro}{Introduction}
\newcommand{\pot}{Powers}
\newcommand{\irrasj}{Irrational numbers}

%Equations
\newcommand{\ligintro}{Introduction}
\newcommand{\liglos}{Solving with the elementary operations}
\newcommand{\ligloso}{Solving with elementary operations summarized}

%Functions
\newcommand{\fintro}{Introduction}
\newcommand{\lingraf}{Linear functions and graphs}

%Geometry 2
\newcommand{\geoform}{Formulas of area and perimeter}
\newcommand{\kongogsim}{Congruent and similar triangles}
\newcommand{\geofork}{Explanations}

% Names of rules
\newcommand{\adkom}{Addition is commutative}
\newcommand{\gangkom}{Multiplication is commutative}
\newcommand{\brdef}{Fractions as rewriting of division}
\newcommand{\brtbr}{Fractions multiplied by fractions}
\newcommand{\delmbr}{Fractions divided by fractions}
\newcommand{\gangpar}{Distributive law}
\newcommand{\gangparsam}{Paranthesis multiplied together}
\newcommand{\gangmnegto}{Multiplication by negative numbers I}
\newcommand{\gangmnegtre}{Multiplication by negative numbers II}
\newcommand{\konsttre}{Unique construction of triangles}
\newcommand{\kongtre}{Congruent triangles}
\newcommand{\topv}{Vertical angles}
\newcommand{\trisum}{The sum of angles in a triangle}
\newcommand{\firsum}{The sum of angles in a quadrilateral}
\newcommand{\potgang}{Multiplication by powers}
\newcommand{\potdivpot}{Division by powers}
\newcommand{\potanull}{The special case of \boldmath $a^0$}
\newcommand{\potneg}{Powers with negative exponents}
\newcommand{\potbr}{Fractions as base}
\newcommand{\faktgr}{Factors as base}
\newcommand{\potsomgrunn}{Powers as base}
\newcommand{\arsirk}{The area of a circle}
\newcommand{\artrap}{The area of a trapezoid}
\newcommand{\arpar}{The area of a parallelogram}
\newcommand{\pyt}{Pythagoras's theorem}
\newcommand{\forform}{Ratios in similar triangles}
\newcommand{\vilkform}{Terms of similar triangles}
\newcommand{\omkrsirk}{The perimeter of a circle (and the value of $ \bm \pi $)}
\newcommand{\artri}{The area of a triangle}
\newcommand{\arrekt}{The area of a rectangle}
\newcommand{\liknflyt}{Moving terms across the equal sign}
\newcommand{\funklin}{Linear functions}



\begin{document}
\section{Limits}
Suppose we start with the value 0.9, and then continuously add 9 as the last digit. Then we get the values 0.9, 0.99, 0.999, and so on. By adding 9 as the last digit in this way, \textsl{we can get as close as we wish $-$ but never exactly reach $-$ the value 1}. The process of \textsl{"getting as close as we wish $-$ but never exactly reaching $-$ a value"} will henceforth be referred to as "approaching a value." The method we just described can be seen as a method to \textsl{approach} 1. We can then say that the \outl{limit} of this method is 1.
To indicate a limit, we write \sym{$ \lim $}.\vsk

It is important to consider that we can approach a number from two sides; from the left or from the right on the number line. With a method that gives us the values 0.9, 0.99, 0.999, etc., we are approaching 1 from the left. If we create a method that gives us the values 1.1, 1.01, 1.001, etc., we are approaching 1 from the right. This is shown by marking \sym{$ + $} or \sym{$ - $} above the number we are approaching.

\regv

\regdef[Limits]{ \vsb
	\alg{
		x\to a^+ &= x \text{ approaches } a \text{ from the right} \br
		x\to a^- &= x \text{ approaches } a \text{ from the left} \br
		x\to a &=  x \text{ approaches } a \text{ (from both the right and left)}\\[12pt]
		\lim\limits_{x\to a} f(x) &= \text{the limit of } f \text{ as } x \text{ approaches } a \\
		&= \text{the value }f\text{ approaches as }x \text{ approaches }a
	} 
}

\spr{
	Approaching a value from the right/left is also called approaching a value from above/below.
}
\info{\note}{
	\sym{$ x\to a$} covers the two cases \sym{$ x\to a^+ $} and \sym{$ x\to a^- $}. Often, these are so similar that we can treat \sym{$ x\to a $} as one case.
}
\newpage

\info{An Extension of \sym{$ = $}}{
	The somewhat paradoxical aspect of limit values where $ x $ approaches $ a $, is that we often end up substituting $ x $ with $ a $, even though we have by definition that $ x\neq a $. For example, the equation
	\begin{equation} \label{limxto2}
		\lim_{x\to 2}(x+1)=2+1=3 
	\end{equation}
	It's worth pondering the similarities in \eqref{limxto2}. When $ x $ approaches 2, $ x $ will never be exactly 2. This means that $ {x+1} $ can never be \textsl{exactly equal} to 3. But \textsl{the closer} $ x $ is to 2, \textsl{the closer} $ {x+1} $ is to 3. In other words, $ {x+1} $ approaches 3 as $ x $ approaches 2. The equality in \eqref{limxto2} thus does not refer to an expression that is \textsl{exactly equal} to a value, but to an expression that \textsl{exactly approaches} a value. This means that limit values bring a somewhat expanded understanding of \sym{$ = $}.
}
\eks[1]{
	Given $ f(x)=\frac{x^2+2x-3}{x-1} $. Find $ \lim\limits_{x\to 1} f(x) $. 
	
	\sv
	When $ x\neq1 $, we have
	\alg{
		f(x)&= \frac{(x-1)(x+3)}{x-1} \\
		&= x+3
	}
	This means that 
	\alg{
		\lim\limits_{x\to 1} f(x) &= \lim\limits_{x\to 1} x+3 \\
		&= 4
	} 
}
\section{Continuity}
\regdef[Continuity]{
	Given a function $ f(x) $ and a number $ c $. If $ f(c) $ exists, $ f $ is \outl{continuous} at $ x=c $ if
	\begin{equation}\label{kont}
		\lim\limits_{x\to c} f(x)=f(c) 
	\end{equation}
	If \eqref{kont} is not valid, $ f $ is \outl{discontinuous} at $ x=c $.
}
\spr{If a function $ f(x) $ is continuous for all $ x $, $ f $ is called a \outl{continuous function}.}
\eks[1 \label{konteks1}]{
	Examine whether the functions are continuous at $ x=2 $.
	\abc{
		\item \begin{equation}f(x)= \left\lbrace{
				\begin{array}{rcr}
					x+4 &,&x<2 \\
					-3x+12   &,& x\geq 2
				\end{array}
			}\right. 
		\end{equation}	
		\item \begin{equation}g(x)= \left\lbrace{
				\begin{array}{rcr}
					x+1 &,&x\leq2 \\
					-x+6   &,& x> 2
				\end{array}
			}\right. 
		\end{equation}	
	}
	
	\sv \vs
	\abc{
		\item We have that
		\algv{
			\lim\limits_{x\to2^+} f(x)&=f(2)= -3\cdot2+12 = 6\br
			\lim\limits_{x\to2^-} f(x)&= 2+4 = 6
		}
		Thus, $ f $ is continuous at $ x=2 $.
		\item We have that
		\algv{
			\lim\limits_{x\to2^-} g(x)&=g(2)= 2+1 = 3\br
			\lim\limits_{x\to2^+} g(x)&= -2+6 = 4
		}
		Thus, $ g $ is \textsl{not} continuous at $ x=2 $.
	}
}
\newpage
\info{Visualization of Continuity}{
	Visually, we can distinguish between continuous and discontinuous functions in this way; continuous functions have connected graphs, discontinuous functions do not. A snippet of the graphs of the functions from \eksref{Example 1} on page \pageref{konteks1} looks like this:
	\fig{konteks1}
	Graphs work excellently for determining which functions we \textsl{expect} to be continuous or not, but they are never valid as proof of this. 
}

\end{document}