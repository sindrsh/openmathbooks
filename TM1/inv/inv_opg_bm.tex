\documentclass[english,hidelinks,pdftex, 11 pt, class=report,crop=false]{standalone}

\newcommand{\note}{Merk}
\newcommand{\notesm}[1]{{\footnotesize \textsl{\note:} #1}}
\newcommand{\ekstitle}{Eksempel }
\newcommand{\sprtitle}{Språkboksen}
\newcommand{\expl}{forklaring}
\newcommand{\pyt}{Pytagoras' setning}
\newcommand\sv{\vsk \textbf{Svar} \vspace{4 pt}\\}

%references
\newcommand{\reftab}[1]{\hrs{#1}{tabell}}
\newcommand{\rref}[1]{\hrs{#1}{regel}}
\newcommand{\dref}[1]{\hrs{#1}{definisjon}}
\newcommand{\refkap}[1]{\hrs{#1}{kapittel}}
\newcommand{\refsec}[1]{\hrs{#1}{seksjon}}
\newcommand{\refdsec}[1]{\hrs{#1}{delseksjon}}
\newcommand{\refved}[1]{\hrs{#1}{vedlegg}}
\newcommand{\eksref}[1]{\textsl{#1}}
\newcommand\fref[2][]{\hyperref[#2]{\textsl{figur \ref*{#2}#1}}}
\newcommand{\refop}[1]{{\color{blue}Oppgave \ref{#1}}}
\newcommand{\refops}[1]{{\color{blue}oppgave \ref{#1}}}


%Algebra
\newcommand{\kvadset}{Kvadratsetningene}
\newcommand{\aenato}{Sum-produkt-metoden}

% Geometry
\newcommand{\hlikb}{Midtnormalen i en likebeint trekant}
\newcommand{\arealsetn}{Arealsetningen}
\newcommand{\trkmedian}{Median}
\newcommand{\midtrk}{Midtnormal (i trekant)}
\newcommand{\innskrsirk}{Innskrevet sirkel}
\newcommand{\cossetn}{Cosinussetningen}
\newcommand{\perfvink}{Sentral- og periferivinkel}
\newcommand{\tang}{Tangent}

% Derivative
\newcommand{\derel}{Den deriverte av elementære funksjoner}
\newcommand{\divder}{Divisjonsregelen}
\newcommand{\kjernereg}{Kjerneregelen}
\newcommand{\prodregder}{Produktregelen}
\newcommand{\lhop}{L'Hopitals regel}

% Funksjonsdrofting
\newcommand{\monder}{Monotoniegenskaper og den deriverte}
\newcommand{\fderekstr}{$ \bm{f'=0} $ for lokale ektstremalpunkt}
\newcommand{\andredertest}{Andrederiverttesten}

% Vectors
\newcommand{\detar}{Arealformler med determinanter}
\newcommand{\avstpunktlin}{Avstand mellom punkt og linje}

%Appendix
\newcommand{\rolle}{Rolles teorem}
\newcommand{\meanval}{Middelverdisetningen}

% Solutions manual
\newcommand{\selos}{Se løsningsforslag.}
\usepackage[T1]{fontenc}
\usepackage[utf8]{luainputenc}
\usepackage{lmodern} % load a font with all the characters
\usepackage{geometry}
\geometry{verbose,a4paper, inner=0cm, outer=0 cm, bmargin=2cm, tmargin=1cm}
%\textwidth=12cm
\setlength{\parindent}{0bp}
\usepackage{import}
\usepackage[subpreambles=false]{standalone}
\usepackage{amsmath}
\usepackage{amssymb}
\usepackage{esint}
\usepackage{babel}
\usepackage{tabu}
\usepackage[dvipsnames, table]{xcolor}
\usepackage{cancel}
\makeatother
\makeatletter
\usepackage{datetime2}
\usepackage{titlesec}
\usepackage[many]{tcolorbox}

% Eheter
\newcommand{\enh}[1]{\,\textrm{#1}}
%referances
\newcommand{\net}[2]{{\color{blue}\href{#1}{#2}}}

%Spaces
\newcommand{\vsk}{\\[12pt]}
\newcommand{\vs}{\vspace{-12pt}}

% Tabell for opplegg

\newcommand{\ovlist}[1]{
\vspace{-16pt}
\begin{itemize}
	#1
\end{itemize}
}

% Chapters and sections
\titleformat{\section}[block]{\bfseries}{\hspace{3cm}\thesection}{5pt}{}
\titleformat{\subsection}[block]{\bfseries}{\hspace{3cm}\thesection}{5pt}{}
\newcommand{\sectionbreak}{\clearpage} % New page on each section
 

\newlength{\mywidth}
\setlength{\mywidth}{14cm}

\newcommand{\cont}[1]{
\begin{tcolorbox}[center, boxrule=0.0 mm, width=\mywidth,arc=0mm,enhanced jigsaw,,colback=white,breakable]
#1	
\end{tcolorbox}
}

\newcommand{\info}[5]{
\begin{tcolorbox}[center, boxrule=0.1 mm, width=\mywidth,arc=0mm,enhanced jigsaw,breakable,colback=yellow!5]	
	
	\footnotesize
	\textbf{Øvingsområde}\\[5pt] #1 
	
	\textbf{Utstyr}\\ #2  \\
	
	\begin{tabular}{@{} p{4cm} p{4cm} l} 
		\textbf{Tid} & \textbf{Elevinndeling} & \textbf{Læringsarena} \\
		#3  & #4 & #5
	\end{tabular} 
\end{tcolorbox}	
}

\newcommand{\gjen}[1]{\begin{tcolorbox}[center,boxrule=0.1 mm, width=\mywidth,arc=0mm,colback=blue!3] {\large \textbf{Gjennomføring} \vspace{5 pt}}\newline #1  \end{tcolorbox}\vspace{-5pt}}
\newcommand{\eks}[1]{\begin{tcolorbox}[center,boxrule=0.1 mm, width=\mywidth,arc=0mm,colback=green!3] {\large \textbf{Eksempel} \vspace{5 pt}}\newline #1  \end{tcolorbox}\vspace{-5pt}}

\newcounter{opl}
%\numberwithin{opl}{article}


\newcommand{\opl}[1]{
\newpage
{\refstepcounter{opl} %\phantomsection 
\large \textbf{\theopl \;#1} \vsk}
}

% Headlines
\newcommand{\fork}{\textbf{Forkunnskapar}\\}
\newcommand{\forb}{\textbf{Forberedelsar}\\}
\newcommand{\opgvr}{\textbf{Oppgaver}}



%colors
\newcommand{\colr}[1]{{\color{red} #1}}
\newcommand{\colb}[1]{{\color{blue} #1}}
\newcommand{\colo}[1]{{\color{orange} #1}}
\newcommand{\colc}[1]{{\color{cyan} #1}}
\definecolor{projectgreen}{cmyk}{100,0,100,0}
\newcommand{\colg}[1]{{\color{projectgreen} #1}}

% Lister med bokstavar
\usepackage[inline]{enumitem}
% Opg
\newcommand{\abc}[1]{
	\begin{enumerate}[label=\alph*),leftmargin=18pt]
		#1
	\end{enumerate}
}

\usepackage[]{hyperref}


\begin{document}
\opgt

\nes
\op{opgfunkekstrm}
Gitt funksjonen 
$ f(x)=a(b-x)(c-x) $. Finn ekstremalpunktet til $ f $ uttrykt ved $ b $ og $ c $. 

\op{opgfunkfinnf}
Gitt en andregradsfunksjon $ f(x) $. Finn uttrykket til $ f $ når
\abc{
\item $ f $ har nullpunkt $ x=3 $ og $ x=-4 $, og ekstremalverdi 5.
\item $ f $ har nullpunkt $ x=-1 $ og $ x=10 $, og ekstremalverdi $ -100 $.
\item $ f $ har nullpunkt $ x=8 $, og toppunkt $ (10, 9) $.
}

\nes

\nes
\op{opgfunkomv}
Finn den omvendte funksjonen $ g $ til $ f $, og bekreft at $ g(f)=x $. \os
\abch{
\item $ f(x)=3x $ 
\item $ f(x)=-9x+2 $
\item $ f(x)=\frac{5}{2}x-7 $
} \os
\abchs{4}{
\item $ f(x)=\frac{3}{x-5} $ 
\item $ \sqrt{x} $
\item $ \sqrt[3]{x} $
\item $ \sqrt[4]{x+9} $
}

\op{opgfunkinvlogarithm}
Finn den omvendte funksjonen $ g $ til $ f $, og bekreft at $ g(f)=x $. \os
\abch{
\item $f(x) = e^x + 2$

\item $f(x) = \ln(x + 5)$

\item $f(x) = \frac{1}{\ln(x)}$
}



\op{opgfunkinvfinna}
Funksjonen $ f(x)=a(2-x-x^3) $ har en omvendt funksjon $ g(y) $, og $ g(490)= -4$. Finn verdien til $ a $.

\nes
\op{opgfunkdrtppktbnpkt}
Gitt en polynomfunksjon med ekstremalpunkt $ a $ og $ b $, som er de eneste ekstremalpunktene til funksjonen på intervallet $ [a, b] $. Forklar hvorfor funksjonen er injektiv på dette intervallet.

\newpage
\grubop{R1V23D2opg4}
(R1V23D2) \os

Nedanfor har vi tegnet grafane til tre funksjoner $ f, g, h $ og $ k $
\abc{
\item Avgjør og grunngi i hvert tilfelle om funksjonen har en omvendt funksjon.
\item Bestem definisjonsmengden til den omvendte funksjonen i de tilfellene hvor den finnnes.
}
\begin{figure}
\centering
\includegraphics[]{\figp{r1v23d2opg4_a}} \qquad
\includegraphics[]{\figp{r1v23d2opg4_b}} \vsk
\includegraphics[]{\figp{r1v23d2opg4_c}} \qquad
\includegraphics[]{\figp{r1v23d2opg4_d}}
\end{figure}

\grubop{opgfunkdrviskonv}
Vis at funksjonen $ {f(x)=a x^2+b x + c}  $ er konveks hvis $ {a>0} $ og konkav hvis $ {a<0} $.

\grubop{opggrubkvadfunk}
I figuren under har vi to parabler. Den grønne parabelen er tegnet ved å først speile den blå parabelen om horisontallinja gjennom bunnpunktet, for så å parallellforskyve parablene slik at de tangerer hverandre i et punkt $ B $. $ A $ og $ C $ ligger på horisontallinja gjennom $ B $, og $ D $ og $ E $ ligger langs samme horisontallinje.\os

Finn lengden til linjestykket $ AC $, uttrykt ved $ s $, når du vet at $ DE=2s $.
\fig{grubkvadfunk}
\end{document}