\documentclass[english,hidelinks,pdftex, 11 pt, class=report,crop=false]{standalone}
\usepackage[T1]{fontenc}
\usepackage[utf8]{luainputenc}
\usepackage{lmodern} % load a font with all the characters
\usepackage{geometry}
\geometry{verbose,a4paper, inner=0cm, outer=0 cm, bmargin=2cm, tmargin=1cm}
%\textwidth=12cm
\setlength{\parindent}{0bp}
\usepackage{import}
\usepackage[subpreambles=false]{standalone}
\usepackage{amsmath}
\usepackage{amssymb}
\usepackage{esint}
\usepackage{babel}
\usepackage{tabu}
\usepackage[dvipsnames, table]{xcolor}
\usepackage{cancel}
\makeatother
\makeatletter
\usepackage{datetime2}
\usepackage{titlesec}
\usepackage[many]{tcolorbox}

% Eheter
\newcommand{\enh}[1]{\,\textrm{#1}}
%referances
\newcommand{\net}[2]{{\color{blue}\href{#1}{#2}}}

%Spaces
\newcommand{\vsk}{\\[12pt]}
\newcommand{\vs}{\vspace{-12pt}}

% Tabell for opplegg

\newcommand{\ovlist}[1]{
\vspace{-16pt}
\begin{itemize}
	#1
\end{itemize}
}

% Chapters and sections
\titleformat{\section}[block]{\bfseries}{\hspace{3cm}\thesection}{5pt}{}
\titleformat{\subsection}[block]{\bfseries}{\hspace{3cm}\thesection}{5pt}{}
\newcommand{\sectionbreak}{\clearpage} % New page on each section
 

\newlength{\mywidth}
\setlength{\mywidth}{14cm}

\newcommand{\cont}[1]{
\begin{tcolorbox}[center, boxrule=0.0 mm, width=\mywidth,arc=0mm,enhanced jigsaw,,colback=white,breakable]
#1	
\end{tcolorbox}
}

\newcommand{\info}[5]{
\begin{tcolorbox}[center, boxrule=0.1 mm, width=\mywidth,arc=0mm,enhanced jigsaw,breakable,colback=yellow!5]	
	
	\footnotesize
	\textbf{Øvingsområde}\\[5pt] #1 
	
	\textbf{Utstyr}\\ #2  \\
	
	\begin{tabular}{@{} p{4cm} p{4cm} l} 
		\textbf{Tid} & \textbf{Elevinndeling} & \textbf{Læringsarena} \\
		#3  & #4 & #5
	\end{tabular} 
\end{tcolorbox}	
}

\newcommand{\gjen}[1]{\begin{tcolorbox}[center,boxrule=0.1 mm, width=\mywidth,arc=0mm,colback=blue!3] {\large \textbf{Gjennomføring} \vspace{5 pt}}\newline #1  \end{tcolorbox}\vspace{-5pt}}
\newcommand{\eks}[1]{\begin{tcolorbox}[center,boxrule=0.1 mm, width=\mywidth,arc=0mm,colback=green!3] {\large \textbf{Eksempel} \vspace{5 pt}}\newline #1  \end{tcolorbox}\vspace{-5pt}}

\newcounter{opl}
%\numberwithin{opl}{article}


\newcommand{\opl}[1]{
\newpage
{\refstepcounter{opl} %\phantomsection 
\large \textbf{\theopl \;#1} \vsk}
}

% Headlines
\newcommand{\fork}{\textbf{Forkunnskapar}\\}
\newcommand{\forb}{\textbf{Forberedelsar}\\}
\newcommand{\opgvr}{\textbf{Oppgaver}}



%colors
\newcommand{\colr}[1]{{\color{red} #1}}
\newcommand{\colb}[1]{{\color{blue} #1}}
\newcommand{\colo}[1]{{\color{orange} #1}}
\newcommand{\colc}[1]{{\color{cyan} #1}}
\definecolor{projectgreen}{cmyk}{100,0,100,0}
\newcommand{\colg}[1]{{\color{projectgreen} #1}}

% Lister med bokstavar
\usepackage[inline]{enumitem}
% Opg
\newcommand{\abc}[1]{
	\begin{enumerate}[label=\alph*),leftmargin=18pt]
		#1
	\end{enumerate}
}

\usepackage[]{hyperref}

% note
\newcommand{\note}{Note}
\newcommand{\notesm}[1]{{\footnotesize \textsl{\note:} #1}}
\newcommand{\selos}{See the solutions manual.}

\newcommand{\texandasy}{The text is written in \LaTeX\ and the figures are made with the aid of Asymptote.}

\newcommand{\rknut}{Calculate.}
\newcommand\sv{\vsk \textbf{Answer} \vspace{4 pt}\\}
\newcommand{\ekstitle}{Example }
\newcommand{\sprtitle}{The language box}
\newcommand{\expl}{explanation}

% answers
\newcommand{\mulansw}{\notesm{Multiple possible answers.}}	
\newcommand{\faskap}{Chapter}

% exercises
\newcommand{\opgt}{\newpage \phantomsection \addcontentsline{toc}{section}{Exercises} \section*{Exercises for Chapter \thechapter}\vs \setcounter{section}{1}}

% references
\newcommand{\reftab}[1]{\hrs{#1}{Table}}
\newcommand{\rref}[1]{\hrs{#1}{Rule}}
\newcommand{\dref}[1]{\hrs{#1}{Definition}}
\newcommand{\refkap}[1]{\hrs{#1}{Chapter}}
\newcommand{\refsec}[1]{\hrs{#1}{Section}}
\newcommand{\refdsec}[1]{\hrs{#1}{Subsection}}
\newcommand{\refved}[1]{\hrs{#1}{Appendix}}
\newcommand{\eksref}[1]{\textsl{#1}}
\newcommand\fref[2][]{\hyperref[#2]{\textsl{Figure \ref*{#2}#1}}}
\newcommand{\refop}[1]{{\color{blue}Exercise \ref{#1}}}
\newcommand{\refops}[1]{{\color{blue}Exercise \ref{#1}}}

%%% SECTION HEADLINES %%%

% Our numbers
\newcommand{\likteikn}{The equal sign}
\newcommand{\talsifverd}{Numbers, digits and values}
\newcommand{\koordsys}{Coordinate systems}

% Calculations
\newcommand{\adi}{Addition}
\newcommand{\sub}{Subtraction}
\newcommand{\gong}{Multiplication}
\newcommand{\del}{Division}

%Factorization and order of operations
\newcommand{\fak}{Factorization}
\newcommand{\rrek}{Order of operations}

%Fractions
\newcommand{\brgrpr}{Introduction}
\newcommand{\brvu}{Values, expanding and simplifying}
\newcommand{\bradsub}{Addition and subtraction}
\newcommand{\brgngheil}{Fractions multiplied by integers}
\newcommand{\brdelheil}{Fractions divided by integers}
\newcommand{\brgngbr}{Fractions multiplied by fractions}
\newcommand{\brkans}{Cancelation of fractions}
\newcommand{\brdelmbr}{Division by fractions}
\newcommand{\Rasjtal}{Rational numbers}

%Negative numbers
\newcommand{\negintro}{Introduction}
\newcommand{\negrekn}{The elementary operations}
\newcommand{\negmeng}{Negative numbers as amounts}

%Calculation methods
\newcommand{\delmedtihundre}{Deling med 10, 100, 1\,000 osv.}

% Geometry 1
\newcommand{\omgr}{Terms}
\newcommand{\eignsk}{Attributes of triangles and quadrilaterals}
\newcommand{\omkr}{Perimeter}
\newcommand{\area}{Area}

%Algebra 
\newcommand{\algintro}{Introduction}
\newcommand{\pot}{Powers}
\newcommand{\irrasj}{Irrational numbers}

%Equations
\newcommand{\ligintro}{Introduction}
\newcommand{\liglos}{Solving with the elementary operations}
\newcommand{\ligloso}{Solving with elementary operations summarized}

%Functions
\newcommand{\fintro}{Introduction}
\newcommand{\lingraf}{Linear functions and graphs}

%Geometry 2
\newcommand{\geoform}{Formulas of area and perimeter}
\newcommand{\kongogsim}{Congruent and similar triangles}
\newcommand{\geofork}{Explanations}

% Names of rules
\newcommand{\adkom}{Addition is commutative}
\newcommand{\gangkom}{Multiplication is commutative}
\newcommand{\brdef}{Fractions as rewriting of division}
\newcommand{\brtbr}{Fractions multiplied by fractions}
\newcommand{\delmbr}{Fractions divided by fractions}
\newcommand{\gangpar}{Distributive law}
\newcommand{\gangparsam}{Paranthesis multiplied together}
\newcommand{\gangmnegto}{Multiplication by negative numbers I}
\newcommand{\gangmnegtre}{Multiplication by negative numbers II}
\newcommand{\konsttre}{Unique construction of triangles}
\newcommand{\kongtre}{Congruent triangles}
\newcommand{\topv}{Vertical angles}
\newcommand{\trisum}{The sum of angles in a triangle}
\newcommand{\firsum}{The sum of angles in a quadrilateral}
\newcommand{\potgang}{Multiplication by powers}
\newcommand{\potdivpot}{Division by powers}
\newcommand{\potanull}{The special case of \boldmath $a^0$}
\newcommand{\potneg}{Powers with negative exponents}
\newcommand{\potbr}{Fractions as base}
\newcommand{\faktgr}{Factors as base}
\newcommand{\potsomgrunn}{Powers as base}
\newcommand{\arsirk}{The area of a circle}
\newcommand{\artrap}{The area of a trapezoid}
\newcommand{\arpar}{The area of a parallelogram}
\newcommand{\pyt}{Pythagoras's theorem}
\newcommand{\forform}{Ratios in similar triangles}
\newcommand{\vilkform}{Terms of similar triangles}
\newcommand{\omkrsirk}{The perimeter of a circle (and the value of $ \bm \pi $)}
\newcommand{\artri}{The area of a triangle}
\newcommand{\arrekt}{The area of a rectangle}
\newcommand{\liknflyt}{Moving terms across the equal sign}
\newcommand{\funklin}{Linear functions}



\begin{document}
\section{Monotonic Properties}
Most function values vary. Descriptions of how functions vary are called descriptions of the functions' \outl{monotonic properties}.\regv
\reg[Increasing and Decreasing Functions \label{incanddec}]{
	Given a function $ f(x) $.
	\begin{itemize}
		\item $ f $ is \outl{increasing} on the interval $ [a, b] $ if for all $ {x_1, x_2 \in[a, b]}$ we have that
		\begin{equation}\label{increasing}
			x_1<x_2 \Rightarrow f(x_1)\leq f(x_2)
		\end{equation}
		If $ {f(x_1)\leq f(x_2) }$ can be replaced with $ {f(x_1)< f(x_2)} $, then $ f $ is \outl{strictly increasing} on the interval.
		\fig{voksende}
		\item $ f $ is \outl{decreasing} on the interval $ [a, b] $ if for all $ {x_1, x_2 \in[a, b]}$ we have that
		\begin{equation}\label{decreasing}
			x_1<x_2 \Rightarrow f(x_1)\geq f(x_2)
		\end{equation}
		If $ {f(x_1)\geq f(x_2) }$ can be replaced with $ {f(x_1)> f(x_2)} $, then $ f $ is \outl{strictly decreasing} on the interval.
		\fig{avtagende}
	\end{itemize}
} \newpage

\reg[\monder \label{monder}]{
	Given $ f(x) $ differentiable on the interval $ [a, b] $.
	\begin{itemize}
		\item If $ {f'\geq0} $ for $ {x\in(a, b)} $, then $ f $ is increasing for $ x\in[a, b] $
		\item If $ {f'\leq0} $ for $ {x\in(a, b)} $, then $ f $ is decreasing for $ x\in[a, b] $
	\end{itemize}
	If respectively $ \geq $ and $ \leq $ can be replaced with $ > $ and $ < $, then $ f $ is strictly increasing/decreasing. 
} \regv

\eks[]{
	Determine on which intervals $ f $ is increasing/decreasing when
	\[ f(x)=\frac{1}{3} x^3 - 4x^2 + 12x\qquad,\qquad x\in[0, 8] \]
	\sv \vs
	We have that
	\[  f'(x)=x^2-8x+12 \]
	To clarify when $ f' $ is positive, negative, or equal to 0, we do two things; we factorize the expression of $ f' $ and draw a sign chart:
	\[ f'(x)=(x-2)(x-6) \]
	\fig{fortgnskj1}
	The sign chart illustrates the following:
	\begin{itemize}
		\item The expression $ x-2 $ is negative when  $ x\in[0, 2) $, equal to 0 when $ x=2 $, and positive when $ x\in(2, 8]  $.
		\item The expression $ x-6 $ is negative when $ x\in[0, 8) $, equal to 0 when $ x=6 $, and positive when $ x\in(6, 8]  $.
		\item Since $ f'=(x-2)(x-6) $, 
		\begin{center}
			\begin{tabular}{l}
				$ f'\geq0 $ when $ {{x\in(0, 2)} \cup (6, 8)} $ \br
				$ f'=0 $ when $ {x\in\{2, 6\}} $ \br
				$ f'\leq0 $ when $ x\in(2, 6)$
			\end{tabular}
		\end{center}
	\end{itemize}
	This means that
	\begin{center}
		\begin{tabular}{l}
			$ f $ is increasing when $ x\in[0, 2] \cup [6, 8] $ \br
			$ f $ is decreasing when $ x\in[2, 6]$
		\end{tabular}
	\end{center}
} \vsk

\reg[\verdmengdgittint \label{verdmengdgittint}]{
	Given a continuous function $ f(x) $ strictly increasing/decreasing for $ {x\in[a, b]} $. The domain of $ f $ on this interval is then $ [f(a), f(b))] $.
}

\section{Extreme Points, Turning Points, and Inflection Points}
\reg[Maximum and Minimum]{\label{max}\index{maximum}\index{minimum}	
	\notesm{A number $ c $ can be referred to as a point in function discussions, implying that it is the point $ (c, 0) $.} \br
	
	Given a function $ f(x) $ and a number $ c $.\bs
	\begin{itemize}
		\item $ f $ has an \outl{absolute maximum} $ f(c) $ if $ {f(c)\geq f(x)} $ for all $ x\in D_f $.
		
		\item $ f $ has an \outl{absolute minimum} $ f(c) $ if ${ f(c)\leq f(x) }$ for all $ x\in D_f $.
		\item $ f $ has a \outl{local maximum} $ f(c) $ if there exists an open interval $ I $ around $ c $ such that $ f(c)\geq f(x) $ for $ x\in I  $.
		
		\item $ f $ has a \outl{local minimum} $ f(c) $ if there exists an open interval $ I $ around $ c $ such that $ f(c)\leq f(x) $ for $ x\in I  $.
	\end{itemize}\vs
}
\spr{
	A maximum/minimum is also called a\\ \outl{maximum value/minimum value}.
}
\reg{Extreme Value and Extreme Point}{\index{extreme point}\index{extreme value}
	Given a function $ f(x) $ with maximum/minimum $ f(c) $. Then
	\begin{itemize}
		\item $ f(c) $ is an \outl{extreme value} for $ f $.
		\item $ c $ is an \outl{extreme point} for $ f $. Specifically, a maximum point/minimum point for $ f $.
		\item $ (c, f(c)) $ is a \outl{maximum point/minimum point} for $ f $.
	\end{itemize}
}
\reg{Critical Points}{
	A number $ c $ is a \outl{critical point} for a function $ f(x) $ if one of the following holds:
	\begin{itemize}
		\item $ f $ is not differentiable at $ c $
		\item $ f'(c)=0 $
	\end{itemize}
}\regv

\reg{First Derivative Test for Extrema \label{fdertest}}{
	Given a differentiable function $ f(x) $ and $ c\in[a, b] $.
	\begin{enumerate}[label=(\roman*)]
		\item If $ c $ is a local extremum point for $ f $, then $ {f'(c)= 0}$
		\item If $ {f'>0} $ for $ {x\in(a, c)} $ and $ {f'<0} $ for $ {x\in(c, b)} $, then $ c $ is a local maximum point for $ f $
		\item If $ {f'<0} $ for $ {x\in(a, c)} $ and $ {f'>0} $ for $ {x\in(c, b)} $, then $ c $ is a local minimum point for $ f $
	\end{enumerate}
}
\regv
\spr{
	What is described in points (ii) and (iii) is often referred to as ''$ f $ changes sign at $ c $''. 
}
\newpage
\eks[1]{
	Find the local minimum point and maximum point for 
	\[ f(x)=2x^3 + 9x^2 - 60x \]
	\sv
	We start by finding $ f' $:
	\alg{
		f'&=6x^2+18x-60 \\
		&= 6(x^2+3x-10)
	}
	Since $ 5(-2)=10 $ and $ 5-2=3 $, we have by \rref{a1a2} that
	\[ f'=6(x-2)(x+5) \]
	$ f'=0 $ for $ x=2 $ and $ x=-5 $. We have that
	\alg{
		f(-5)&=2^3+9\cdot2^2-60\cdot2 = -68	\br
		f(2)&=5^3+9\cdot5^2-60\cdot5 = 275 
	}
	Thus, $ (-5, 275) $ is the maximum point for $ f $ and $ (2, -68) $ is the minimum point for $ f $.
}



\newpage

\reg[Second Derivative Test \label{seconddertest}]{
	Given a differentiable function $ f(x) $ and a number $ c $.
	\begin{itemize}
		\item If $ {f'(c)=0} $ and $ {f''(c)<0} $, then $ f(c) $ is a local maximum.
		\item If $ {f'(c)=0 }$ and $ {f''(c)>0} $, then $ f(c) $ is a local minimum.
		\item If ${ f'(c)=f''(c)=0 }$, it cannot be determined from this information alone whether $ f(c) $ is a local maximum or minimum.
	\end{itemize}	
}
\fork{\ref{seconddertest} Second Derivative Test}{
	By the definition of the derivative, we have that
	\[ f''(c)=\lim\limits_{h \to 0}\frac{f'(c+h)-f'(c)}{h} \]
	When $ f'(c)=0$, we have
	\[ f''(c)=\lim\limits_{h \to 0}\frac{f'(c+h)}{h} \]
	When $ f''(c)<0 $, this means that
	\[\lim\limits_{h \to 0} \frac{f'(c+h)}{h}<0 \]
	So, $ f' $ must be positive when $ h $ approaches 0 from the left and negative when $ h $ approaches 0 from the right. Thus, $ f' $ changes sign at $ c $, which must then be a maximum point for $ f $. Similarly, $ c $ must be a minimum point for $ f $ if $ {f(c)=0} $ and $ {f''(c)<0} $.
}
\newpage
\reg{Inflection Point and Turning Point}{\index{inflection point}\index{turning point}
	For a continuous function $ f(x) $, we have that
	\begin{itemize}
		\item If $ {f''(c)=0} $ and $ f'' $ changes sign at $ c $, then $ c $ is an \outl{inflection point} for $ f $.
		\item If $ c $ is an inflection point for $ f $, then $ (c, f(c)) $ is a \outl{turning point}.
		\item $ f $ is convex on intervals where $ {f''>0} $, and concave on intervals where $ {f''<0} $. (See \rref{convconc} regarding convex and concave functions.)
	\end{itemize}\vs
}
\newpage
\eks[]{
	\[ f(x)=x^3-3x^2-144x-140\quad,\quad x\in[-14, 16] \]
	\fig{funkdroft}
	\begin{center}
		\small
		\begin{tabular}{c | c}
			\rowcolor{gray!25}		\textbf{point/value} & \textbf{type} \\ \hline
			\rowcolor{green!10}		$ A=(-14, -1456) $ & absolute minimum point \\ \hline
			\rowcolor{green!10}		$ -14 $ & extremum point; absolute minimum \\ \hline
			\rowcolor{green!10}		$ -1456 $& absolute minimum \\ \hline
			\rowcolor{cyan!10}		$ B=(-6, 400) $ & local maximum point \\ \hline	\rowcolor{cyan!10}		
			$ -6 $ & extremum point; local maximum point \\ \hline \rowcolor{cyan!10}	
			$ 400 $ & local maximum \\ \hline
			\rowcolor{green!10}		$ C=(-1, -286) $ & turning point  \\ \hline
			\rowcolor{green!10}		$ -1 $ & inflection point \\ \hline 
			\rowcolor{cyan!10}		$ D = (8, -972) $ & local minimum point \\ \hline
			\rowcolor{cyan!10}		$ 8 $ & extremum point; local minimum point \\ \hline
			\rowcolor{cyan!10}		$ -972 $ & local minimum \\ \hline
			\rowcolor{green!10}		$ E= (16, 884) $ & absolute maximum point \\ \hline
			\rowcolor{green!10}		$ 16 $ & extremum point; absolute maximum point \\ \hline
			\rowcolor{green!10}		$ 884 $ & absolute maximum \\ \hline
			\rowcolor{cyan!10}		$ -10 $, $ -1 $ and $ 14 $ & zero point \\ \hline
		\end{tabular}
	\end{center}
}
\newpage
\section{Asymptotes}
\reg[Vertical Asymptotes]{
	Given a function $ f(x) $ and a constant $ c $.
	\begin{itemize}
		\item If $ {\lim\limits_{x\to c^+} f(x)=\pm \infty}  $, then $ c $ is a \outl{vertical asymptote from above} for $ f $.
		\item If $ {\lim\limits_{x\to c^-} f(x)=\pm \infty}  $, then $ c $ is a \outl{vertical asymptote from below} for $ f $.
		\item If $ {\lim\limits_{x\to c} f(x)=\pm \infty}  $, then $ c $ is a \outl{vertical asymptote} for $ f $.
	\end{itemize}
}
\eks[]{
	Find the vertical asymptote of
	\[ f(x)=\frac{1}{x-3}+2 \]
	\sv
	We observe that
	\[ \lim\limits_{x\to 3}\left[\frac{1}{x-3}+2\right]=\pm \infty  \]
	Thus, $ x=3 $ is a vertical asymptote for $ f $.
} 
\newpage
\reg[Horizontal Asymptotes]{
	Given a function $ f(x) $. Then $ {y=c} $ is a \outl{horizontal asymptote} for $ f $ if 
	\[  {\lim\limits_{x\to |\infty|} f(x)=c} \]
}
\eks[]{
	Find the horizontal asymptote of
	\[ f(x)=\frac{1}{x-3}+2 \]
	\sv
	We observe that
	\[ \lim\limits_{x\to |\infty|}\left[\frac{1}{x-3}+2\right]=2 \]
	Thus, $ y=2 $ is a horizontal asymptote for $ f $.
}

\section{Convex and Concave Functions}
\reg{Convex and Concave Functions \label{convconc}}{
	Given a continuous function $ f(x) $.	\vsk
	
	If the entire line between $ (a, f(a)) $ and $ (b, f(b)) $ lies above the graph of $ f $ on the interval $ [a, b] $, then $ f $ is \outl{convex} for $ x\in[a, b] $.	
	\begin{figure}
		\centering
		\includegraphics[]{\figp{konv}}
	\end{figure}	
	If the entire line between $ (a, f(a)) $ and $ (b, f(b)) $ lies below the graph of $ f $ on the interval $ [a, b] $, then $ f $ is \outl{concave} for $ x\in[a, b] $.
	\begin{figure}
		\centering
		\includegraphics[]{\figp{konk}}
	\end{figure}
	\vs
}
\section{Injective and Inverse Functions}
\subsubsection{Injective Functions}
\reg{Injective Functions}{
	Given a function $ f(x) $. If all values of $ f $ are unique on the interval $ x\in[a, b] $, then $ f $ is \outl{injective} on this interval.
}
\spr{
	Another word for injective is \outl{one-to-one}.
}
\subsubsection{Inverse Functions}
Given the function $ f(x)=2x+1 $, which is obviously injective for all $ x\in \mathbb{R} $. This means that the equation $ f=2x+1 $ has only one solution, regardless of whether we solve with respect to $ x $ or $ f $. Solving with respect to $ x $, we have that
\[ x=\frac{f-1}{2} \]
Now, we have gone from having an expression for $ f $ to the ''reverse'', an expression for $ x $. Since both $ x $ and $ f $ are variables, $ x $ is a function of $ f $, and to clarify this, we could have written
\[ x(f)=\frac{f-1}{2} \]
This function is called the \outl{inverse function} of $ f $. If we substitute the expression for $ f $ into the expression for $ x(f) $, we necessarily get $ x $:
\alg{
	x\left(2x+1\right)&=\frac{2x+1-1}{2} \\
	&= x
} 
The equation above highlights a problem; it is very messy to treat $ x $ as both a function and a variable simultaneously. It is therefore common to rename both $ f $ and $ x $ so that the inverse function and the variable it depends on get new symbols. For example, we can set $ y=f$ and $ g=x $. The inverse function $ g $ of $ f $ is then that
\[ g(y)=\frac{y-1}{2} \]
\reg{Inverse Functions}{
	Given two injective functions $ f(x) $ and $ g(y) $. If
	\[ g(f)=x \]
	then $ f $ and $ g $ are \outl{inverse} functions.
} 
\eks[1]{Given the function $ f(x)=5x-3 $.
	\abc{
		\item Find the inverse function $ g $ of $ f $.
		\item Show that $ g(f)=x $. 
	} 
	
	\sv
	\abc{
		\item We set $ y=f $, and solve the equation with respect to $ x $:
		\alg{
			y&= 5x-3 \\
			x &= \frac{y+3}{5}
		}
		Then, $ g(y)=\frac{y+3}{5} $.
		\item When $ y = f $, we have that
		\alg{
			g(y)&=g(5x-3)\br 
			&= \frac{5x-3+3}{5} \\
			&= x
		}
	}
}

\newpage
\info{$ \bm {f^{-1}} $}{
	If $ f $ and $ g $ are inverse functions, $ g $ is often written as $ f^{-1} $. It is important to note that $ f^{-1} $ is not the same as $ (f)^{-1} $. For example, given $ f(x)= x+1 $. Then
	\[ f^{-1}=x-1 \qquad,\qquad (f)^{-1}=\frac{1}{x+1}\] 
	In all other cases except when $ {n=-1} $, it will be the case in this book that
	\[ f^n=(f)^n \] 
}
\section*{Explanations}
\fork{\ref{monder}\,\monder}{
	Given $ f(x) $, where $ {f'\geq0} $ for $ {x\in[a, b]} $. Let $ {x_1, x_2 \in(a, b)}$ and $ {x_2>x_1} $. By the \refunnbr{meanval}{mean value theorem}, there exists a number $ {c\in(x_1, x_2)} $ such that
	\alg{
		f'(c) &=\frac{f(x_2)-f(x_1)}{x_2-x_1}
	}
	Since $ c\in[a, b] $, $ {f'(x)\geq 0} $, and thus
	\alg{
		0\geq \frac{f(x_2)-f(x_1)}{x_2-x_1}
	}
	Consequently, $ {f(x_2)\geq f(x_1)} $, and from \dref{voksogav} then $ f $ is increasing on the interval $ (a, b) $. \vs
}\vsk

\fork{\ref{verdmengdgittint} \verdmengdgittint}{
	We aim to show that for any number $ {c\in(f(a), f(b))} $ there exists a number $x_c $ such that $ f(x_c)=c$.\os
	
	We let $ f(x) $ be a strictly increasing function and define $ P(c, n) $ given by the following procedure, described by a Python function (see \am for an introduction to Python):
	\python{verdmengdgittint.py}
	Since $ f $ is strictly increasing, we can always be sure that $ f(x_1)\leq f(x_3)\leq f(x_2)  $ and $ f(x_1)\leq c\leq f(x_2)  $ at the start of each iteration. As $ {n\to\infty} $, $ {x_2\to x_1} $, and since $ f $ is continuous, $ {\lim\limits_{n\to \infty} f(x_1)=f(x_2)} $. This means that $ {\lim\limits_{n\to \infty} P(c,n)\to c}$, and thus there must exist a number $ x_c\in(a, b) $ that makes $ f(x_c)=c$.
}


\fork{\ref{fderekstr} \fderekstr}{
	\textbf{Punkt (i)} \os
	
	La $ c $ være et lokalt maksimumspunkt for $ f $. For et tall $ h $ må vi da ha at $ c\geq x $ for $ x\in(c-|h|, c+|h|) $. Da er
	\[ f(c+h)-f(c)\leq 0 \] 
	Dette betyr at
	\alg{
		\lim\limits_{h\to 0^+}\frac{f(c+h)-f(c)}{h}\leq 0 
	}
	og at
	\alg{
		\lim\limits_{h\to 0^-}\frac{f(c+h)-f(c)}{h}\geq 0 
	}
	Følgelig er
	\[ \lim\limits_{h\to 0^-}\frac{f(c+h)-f(c)}{h}=\lim\limits_{h\to 0^+}\frac{f(c+h)-f(c)}{h} \]
	Altså er $ {f'(c)=0} $, og $ f' $ skifter fortegn fra positiv til negativ i $ c $. Med samme framgangsmåte kan det vises at dette også gjelder dersom $ c $ er et minimumspunkt, bare at da skifter $ f' $ fra negativ til positiv.\vsk
	
	\textbf{Punkt (ii)} \os
	Hvis $ {f'>0} $ på intervallet $ (a, c) $, har vi av \rref{monder} at $ f $ er sterkt voksende der. Hvis $ {f'<0} $ på $ (c, b) $, er $ f $ sterkt avtagende der. Dette må nødvendigvis bety at $ {f(c)\geq f(x)} $ for $ {x\in(a, b)} $, og da er $ c $ et maksimumspunkt.\vsk
	
	\textbf{Punkt (iii)} \os
	Tilsvarende resonnement som for punkt (ii).
}


\end{document}