\documentclass[english,hidelinks,pdftex, 11 pt, class=report,crop=false]{standalone}
\usepackage[T1]{fontenc}
\usepackage[utf8]{luainputenc}
\usepackage{lmodern} % load a font with all the characters
\usepackage{geometry}
\geometry{verbose,a4paper, inner=0cm, outer=0 cm, bmargin=2cm, tmargin=1cm}
%\textwidth=12cm
\setlength{\parindent}{0bp}
\usepackage{import}
\usepackage[subpreambles=false]{standalone}
\usepackage{amsmath}
\usepackage{amssymb}
\usepackage{esint}
\usepackage{babel}
\usepackage{tabu}
\usepackage[dvipsnames, table]{xcolor}
\usepackage{cancel}
\makeatother
\makeatletter
\usepackage{datetime2}
\usepackage{titlesec}
\usepackage[many]{tcolorbox}

% Eheter
\newcommand{\enh}[1]{\,\textrm{#1}}
%referances
\newcommand{\net}[2]{{\color{blue}\href{#1}{#2}}}

%Spaces
\newcommand{\vsk}{\\[12pt]}
\newcommand{\vs}{\vspace{-12pt}}

% Tabell for opplegg

\newcommand{\ovlist}[1]{
\vspace{-16pt}
\begin{itemize}
	#1
\end{itemize}
}

% Chapters and sections
\titleformat{\section}[block]{\bfseries}{\hspace{3cm}\thesection}{5pt}{}
\titleformat{\subsection}[block]{\bfseries}{\hspace{3cm}\thesection}{5pt}{}
\newcommand{\sectionbreak}{\clearpage} % New page on each section
 

\newlength{\mywidth}
\setlength{\mywidth}{14cm}

\newcommand{\cont}[1]{
\begin{tcolorbox}[center, boxrule=0.0 mm, width=\mywidth,arc=0mm,enhanced jigsaw,,colback=white,breakable]
#1	
\end{tcolorbox}
}

\newcommand{\info}[5]{
\begin{tcolorbox}[center, boxrule=0.1 mm, width=\mywidth,arc=0mm,enhanced jigsaw,breakable,colback=yellow!5]	
	
	\footnotesize
	\textbf{Øvingsområde}\\[5pt] #1 
	
	\textbf{Utstyr}\\ #2  \\
	
	\begin{tabular}{@{} p{4cm} p{4cm} l} 
		\textbf{Tid} & \textbf{Elevinndeling} & \textbf{Læringsarena} \\
		#3  & #4 & #5
	\end{tabular} 
\end{tcolorbox}	
}

\newcommand{\gjen}[1]{\begin{tcolorbox}[center,boxrule=0.1 mm, width=\mywidth,arc=0mm,colback=blue!3] {\large \textbf{Gjennomføring} \vspace{5 pt}}\newline #1  \end{tcolorbox}\vspace{-5pt}}
\newcommand{\eks}[1]{\begin{tcolorbox}[center,boxrule=0.1 mm, width=\mywidth,arc=0mm,colback=green!3] {\large \textbf{Eksempel} \vspace{5 pt}}\newline #1  \end{tcolorbox}\vspace{-5pt}}

\newcounter{opl}
%\numberwithin{opl}{article}


\newcommand{\opl}[1]{
\newpage
{\refstepcounter{opl} %\phantomsection 
\large \textbf{\theopl \;#1} \vsk}
}

% Headlines
\newcommand{\fork}{\textbf{Forkunnskapar}\\}
\newcommand{\forb}{\textbf{Forberedelsar}\\}
\newcommand{\opgvr}{\textbf{Oppgaver}}



%colors
\newcommand{\colr}[1]{{\color{red} #1}}
\newcommand{\colb}[1]{{\color{blue} #1}}
\newcommand{\colo}[1]{{\color{orange} #1}}
\newcommand{\colc}[1]{{\color{cyan} #1}}
\definecolor{projectgreen}{cmyk}{100,0,100,0}
\newcommand{\colg}[1]{{\color{projectgreen} #1}}

% Lister med bokstavar
\usepackage[inline]{enumitem}
% Opg
\newcommand{\abc}[1]{
	\begin{enumerate}[label=\alph*),leftmargin=18pt]
		#1
	\end{enumerate}
}

\usepackage[]{hyperref}

% note
\newcommand{\note}{Note}
\newcommand{\notesm}[1]{{\footnotesize \textsl{\note:} #1}}
\newcommand{\selos}{See the solutions manual.}

\newcommand{\texandasy}{The text is written in \LaTeX\ and the figures are made with the aid of Asymptote.}

\newcommand{\rknut}{Calculate.}
\newcommand\sv{\vsk \textbf{Answer} \vspace{4 pt}\\}
\newcommand{\ekstitle}{Example }
\newcommand{\sprtitle}{The language box}
\newcommand{\expl}{explanation}

% answers
\newcommand{\mulansw}{\notesm{Multiple possible answers.}}	
\newcommand{\faskap}{Chapter}

% exercises
\newcommand{\opgt}{\newpage \phantomsection \addcontentsline{toc}{section}{Exercises} \section*{Exercises for Chapter \thechapter}\vs \setcounter{section}{1}}

% references
\newcommand{\reftab}[1]{\hrs{#1}{Table}}
\newcommand{\rref}[1]{\hrs{#1}{Rule}}
\newcommand{\dref}[1]{\hrs{#1}{Definition}}
\newcommand{\refkap}[1]{\hrs{#1}{Chapter}}
\newcommand{\refsec}[1]{\hrs{#1}{Section}}
\newcommand{\refdsec}[1]{\hrs{#1}{Subsection}}
\newcommand{\refved}[1]{\hrs{#1}{Appendix}}
\newcommand{\eksref}[1]{\textsl{#1}}
\newcommand\fref[2][]{\hyperref[#2]{\textsl{Figure \ref*{#2}#1}}}
\newcommand{\refop}[1]{{\color{blue}Exercise \ref{#1}}}
\newcommand{\refops}[1]{{\color{blue}Exercise \ref{#1}}}

%%% SECTION HEADLINES %%%

% Our numbers
\newcommand{\likteikn}{The equal sign}
\newcommand{\talsifverd}{Numbers, digits and values}
\newcommand{\koordsys}{Coordinate systems}

% Calculations
\newcommand{\adi}{Addition}
\newcommand{\sub}{Subtraction}
\newcommand{\gong}{Multiplication}
\newcommand{\del}{Division}

%Factorization and order of operations
\newcommand{\fak}{Factorization}
\newcommand{\rrek}{Order of operations}

%Fractions
\newcommand{\brgrpr}{Introduction}
\newcommand{\brvu}{Values, expanding and simplifying}
\newcommand{\bradsub}{Addition and subtraction}
\newcommand{\brgngheil}{Fractions multiplied by integers}
\newcommand{\brdelheil}{Fractions divided by integers}
\newcommand{\brgngbr}{Fractions multiplied by fractions}
\newcommand{\brkans}{Cancelation of fractions}
\newcommand{\brdelmbr}{Division by fractions}
\newcommand{\Rasjtal}{Rational numbers}

%Negative numbers
\newcommand{\negintro}{Introduction}
\newcommand{\negrekn}{The elementary operations}
\newcommand{\negmeng}{Negative numbers as amounts}

%Calculation methods
\newcommand{\delmedtihundre}{Deling med 10, 100, 1\,000 osv.}

% Geometry 1
\newcommand{\omgr}{Terms}
\newcommand{\eignsk}{Attributes of triangles and quadrilaterals}
\newcommand{\omkr}{Perimeter}
\newcommand{\area}{Area}

%Algebra 
\newcommand{\algintro}{Introduction}
\newcommand{\pot}{Powers}
\newcommand{\irrasj}{Irrational numbers}

%Equations
\newcommand{\ligintro}{Introduction}
\newcommand{\liglos}{Solving with the elementary operations}
\newcommand{\ligloso}{Solving with elementary operations summarized}

%Functions
\newcommand{\fintro}{Introduction}
\newcommand{\lingraf}{Linear functions and graphs}

%Geometry 2
\newcommand{\geoform}{Formulas of area and perimeter}
\newcommand{\kongogsim}{Congruent and similar triangles}
\newcommand{\geofork}{Explanations}

% Names of rules
\newcommand{\adkom}{Addition is commutative}
\newcommand{\gangkom}{Multiplication is commutative}
\newcommand{\brdef}{Fractions as rewriting of division}
\newcommand{\brtbr}{Fractions multiplied by fractions}
\newcommand{\delmbr}{Fractions divided by fractions}
\newcommand{\gangpar}{Distributive law}
\newcommand{\gangparsam}{Paranthesis multiplied together}
\newcommand{\gangmnegto}{Multiplication by negative numbers I}
\newcommand{\gangmnegtre}{Multiplication by negative numbers II}
\newcommand{\konsttre}{Unique construction of triangles}
\newcommand{\kongtre}{Congruent triangles}
\newcommand{\topv}{Vertical angles}
\newcommand{\trisum}{The sum of angles in a triangle}
\newcommand{\firsum}{The sum of angles in a quadrilateral}
\newcommand{\potgang}{Multiplication by powers}
\newcommand{\potdivpot}{Division by powers}
\newcommand{\potanull}{The special case of \boldmath $a^0$}
\newcommand{\potneg}{Powers with negative exponents}
\newcommand{\potbr}{Fractions as base}
\newcommand{\faktgr}{Factors as base}
\newcommand{\potsomgrunn}{Powers as base}
\newcommand{\arsirk}{The area of a circle}
\newcommand{\artrap}{The area of a trapezoid}
\newcommand{\arpar}{The area of a parallelogram}
\newcommand{\pyt}{Pythagoras's theorem}
\newcommand{\forform}{Ratios in similar triangles}
\newcommand{\vilkform}{Terms of similar triangles}
\newcommand{\omkrsirk}{The perimeter of a circle (and the value of $ \bm \pi $)}
\newcommand{\artri}{The area of a triangle}
\newcommand{\arrekt}{The area of a rectangle}
\newcommand{\liknflyt}{Moving terms across the equal sign}
\newcommand{\funklin}{Linear functions}



\begin{document}	
\vedlegg{Euler's Number} \label{eulerstallfork}
\subsubsection{The Derivative as Motivation}
Given the function $ f(x)=a^x $. Then we have
\alg{
	\left(a^x\right)'&=\lim\limits_{h\to 0}\frac{a^{x+h}-a^x}{h}\br
	&= \lim\limits_{h\to 0}\frac{a^{x}a^h-a^x}{h}
}
Since $ x $ is independent of $ h $, we get
\alg{
	\left(a^x\right)'=a^x\lim\limits_{h\to 0}\frac{a^h-1}{h}
}
The equation above points towards something amazing; if there exists a number $ a $ such that $ {\lim\limits_{h\to 0}\frac{a^h-1}{h}=1} $, then the function $ a^x $ will be its own derivative function! That is, $ \left(a^x\right)'=a^x $. We now notice that if
\[ a=\lim\limits_{h\to 0}\left(1+h\right)^\frac{1}{h} \]
then
\algv{
	\lim\limits_{h\to 0}\frac{a^h-1}{h}&=\lim\limits_{h\to 0}\frac{\left(\left(1+h\right)^\frac{1}{h}\right)^h-1}{h}\br
	&=\frac{1+h-1}{h}\\
	&=1
}
If we can demonstrate that the limit $ \lim\limits_{h\to 0}\left(1+h\right)^\frac{1}{h} $ exists, we have thus found exactly the expression for $ a $ that we want.

\subsubsection{Investigation of the Limit}
We introduce the following two functions (the motivation to introduce $ g $ will appear later):
\[ f(h)=1+h \qquad,\qquad g(h)=2-\left(\frac{1}{4}\right)^{h}\]
Further, we investigate for which values $ f $ is less than $ g $. When $ f=g $, we have that
\begin{equation}\label{eforkleqh}
	1+h=2-\left(\frac{1}{4}\right)^h 
\end{equation}
We now make the following observation: Given two numbers $ c $ and $ k $, and the function $ p(h)=b^h $, where $ k>0 $ and $ 0<b<1 $. Then we have
\alg{
	p(c+k)-p(c)=b^{c+k}-b^{c}=b^c\left(b^{k}-1\right)
}
Similarly,
\alg{
	p(c+2k)-p(c+k)&=b^{c+k}\left(b^{k}-1\right)
}
Furthermore, $ b^{c+k}<b^c $ and $ b^k-1<1 $, which means that
\[ \frac{p(c+k)-p(c)}{k}<\frac{p(c+2k)-p(c+k)}{k} \]
Therefore, the line between $ (c, p(c)) $ and $ (c+k, p(c+k)) $ must be steeper than the line between $ (c+k, p(c+k)) $ and $ (c+2k, p(c+2k)) $, and thus $ (c+k, p(c+k)) $ must lie below the line between $ (c, p(c)) $ and $ (c+2k, p(c+2k)) $.
\fig{ekspfunk3}
Since $ p(h) $ is not a linear function, one of these three statements must be valid:
\begin{itemize}
	\item $ p $ is convex for all $ h$
	\item $ p $ is concave for all $ h $
	\item $ p $ alternates between concave and convex
\end{itemize}
But if $ p $ is concave, there must be an interval where $ (c+k, p(c+k)) $ lies above the line between $ (c, p(c)) $ and $ (c+2k, p(c+2k)) $, and this is contradictory. Thus, $ p $ must necessarily be convex for all $ h $.\vsk

\newpage
From what we just found, we can conclude that the function $2-\left(\frac{1}{4}\right)^h $ is concave for all $ h $, and since $ 1+h$ is a linear expression, \eqref{eforkleqh} has at most two solutions. It is easy to show that $ h=0 $ and $ h=\frac{1}{2} $ are the solutions to \eqref{eforkleqh}, and this means that
\begin{equation}\label{1plushleq}
	1+h\leq 2-\left(\frac{1}{4}\right)^h\qquad,\qquad x\in\left[0, \frac{1}{2}\right] 	
\end{equation}
\fig{ekspfunk2}
We set $z=\frac{1}{h} $ for $ h\neq0 $. Then
\[ \lim\limits_{h\to 0} (1+h)^h=\lim\limits_{z\to\infty}=\left(1+\frac{1}{z}\right)^\frac{1}{z} \]
Furthermore, \eqref{1plushleq} can be rewritten as
\begin{equation}\label{eforkleqz}
	1+\frac{1}{z}\leq 2-\left(\frac{1}{4}\right)^\frac{1}{z}\qquad,\qquad z\in[2, \infty]
\end{equation}

For $ z\to\infty $ we can therefore be sure that
\[ 1+\frac{1}{z}<1+1-\left(\frac{1}{4}\right)^\frac{1}{z}+\left(1-\left(\frac{1}{4}\right)^\frac{1}{z}\right)^2+\left(1-\left(\frac{1}{4}\right)^\frac{1}{z}\right)^3+... \]
The right side in the inequality above is recognized\footnote{See about geometric series in \tmto.} as an infinite geometric series where the sum is given as
\[ \frac{1}{1-\left(1-\left(\frac{1}{4}\right)^\frac{1}{z}\right)} =\frac{1}{\left(\frac{1}{4}\right)^\frac{1}{z}}=4^\frac{1}{z} \]
Thus, it is
\begin{equation}\label{eforkllim4}
	\lim\limits_{z\to\infty}\left(1+\frac{1}{z}\right)^z\leq \lim\limits_{z\to\infty}\left(4^\frac{1}{z}\right)^z=4
\end{equation}
\newpage
Furthermore, it is easy to show that the equation
\[ 1+h=2-\left(\frac{1}{2}\right)^h \]
has the solutions $ h=-1 $ and $ h=0 $, which means that
\[ 1+h\geq2-\left(\frac{1}{2}\right)^h \qquad,\qquad h\in[0, \infty]\]
In a similar manner as we came to an upper limit, we can use this to assert that
\[ \lim\limits_{z\to \infty}\left(1 + \frac{1}{z}\right)^z\geq 2 \] 
Thus, we know that $ \lim\limits_{z\to\infty }\left(1+\frac{1}{z}\right)^n $ lies somewhere between 2 and 4. Since the\\ expression contains only positive terms for $ {z\to\infty} $, we can also be sure that the limit is finite\footnote{As opposed to being indeterminate. For example, $ \lim\limits_{x\to \infty} \cos x $  will be indeterminate because $ \cos x $ oscillates between $ -1 $ and $ 1 $.}. It therefore makes sense to treat the limit value as a number, which we call $ e $:
\[ e=\lim\limits_{z\to\infty }\left(1+\frac{1}{z}\right)^z=\lim\limits_{h\to0}\left(1+h\right)^{\frac{1}{h}}  \]
\info{Note}{
	The most classic method for finding an upper and lower limit for $ \lim\limits_{z\to\infty }\left(1+\frac{1}{z}\right)^n $ is by using \net{https://en.wikipedia.org/wiki/Binomial\_theorem\#General\_case}{Binomial Theorem}. 
}
\subsubsection{A Look Back at the Derivative}
Differentiation of power functions was what motivated us to investigate the number $ e $. From what we have discussed in the preceding sections, it follows that
\[ \left(e^x\right)'=e^x \]
The equation above is simply one of the most important equations in\\
mathematics.

\end{document}