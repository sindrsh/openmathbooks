\documentclass[english,hidelinks,pdftex, 11 pt, class=report,crop=false]{standalone}
\usepackage[T1]{fontenc}
\usepackage[utf8]{luainputenc}
\usepackage{lmodern} % load a font with all the characters
\usepackage{geometry}
\geometry{verbose,a4paper, inner=0cm, outer=0 cm, bmargin=2cm, tmargin=1cm}
%\textwidth=12cm
\setlength{\parindent}{0bp}
\usepackage{import}
\usepackage[subpreambles=false]{standalone}
\usepackage{amsmath}
\usepackage{amssymb}
\usepackage{esint}
\usepackage{babel}
\usepackage{tabu}
\usepackage[dvipsnames, table]{xcolor}
\usepackage{cancel}
\makeatother
\makeatletter
\usepackage{datetime2}
\usepackage{titlesec}
\usepackage[many]{tcolorbox}

% Eheter
\newcommand{\enh}[1]{\,\textrm{#1}}
%referances
\newcommand{\net}[2]{{\color{blue}\href{#1}{#2}}}

%Spaces
\newcommand{\vsk}{\\[12pt]}
\newcommand{\vs}{\vspace{-12pt}}

% Tabell for opplegg

\newcommand{\ovlist}[1]{
\vspace{-16pt}
\begin{itemize}
	#1
\end{itemize}
}

% Chapters and sections
\titleformat{\section}[block]{\bfseries}{\hspace{3cm}\thesection}{5pt}{}
\titleformat{\subsection}[block]{\bfseries}{\hspace{3cm}\thesection}{5pt}{}
\newcommand{\sectionbreak}{\clearpage} % New page on each section
 

\newlength{\mywidth}
\setlength{\mywidth}{14cm}

\newcommand{\cont}[1]{
\begin{tcolorbox}[center, boxrule=0.0 mm, width=\mywidth,arc=0mm,enhanced jigsaw,,colback=white,breakable]
#1	
\end{tcolorbox}
}

\newcommand{\info}[5]{
\begin{tcolorbox}[center, boxrule=0.1 mm, width=\mywidth,arc=0mm,enhanced jigsaw,breakable,colback=yellow!5]	
	
	\footnotesize
	\textbf{Øvingsområde}\\[5pt] #1 
	
	\textbf{Utstyr}\\ #2  \\
	
	\begin{tabular}{@{} p{4cm} p{4cm} l} 
		\textbf{Tid} & \textbf{Elevinndeling} & \textbf{Læringsarena} \\
		#3  & #4 & #5
	\end{tabular} 
\end{tcolorbox}	
}

\newcommand{\gjen}[1]{\begin{tcolorbox}[center,boxrule=0.1 mm, width=\mywidth,arc=0mm,colback=blue!3] {\large \textbf{Gjennomføring} \vspace{5 pt}}\newline #1  \end{tcolorbox}\vspace{-5pt}}
\newcommand{\eks}[1]{\begin{tcolorbox}[center,boxrule=0.1 mm, width=\mywidth,arc=0mm,colback=green!3] {\large \textbf{Eksempel} \vspace{5 pt}}\newline #1  \end{tcolorbox}\vspace{-5pt}}

\newcounter{opl}
%\numberwithin{opl}{article}


\newcommand{\opl}[1]{
\newpage
{\refstepcounter{opl} %\phantomsection 
\large \textbf{\theopl \;#1} \vsk}
}

% Headlines
\newcommand{\fork}{\textbf{Forkunnskapar}\\}
\newcommand{\forb}{\textbf{Forberedelsar}\\}
\newcommand{\opgvr}{\textbf{Oppgaver}}



%colors
\newcommand{\colr}[1]{{\color{red} #1}}
\newcommand{\colb}[1]{{\color{blue} #1}}
\newcommand{\colo}[1]{{\color{orange} #1}}
\newcommand{\colc}[1]{{\color{cyan} #1}}
\definecolor{projectgreen}{cmyk}{100,0,100,0}
\newcommand{\colg}[1]{{\color{projectgreen} #1}}

% Lister med bokstavar
\usepackage[inline]{enumitem}
% Opg
\newcommand{\abc}[1]{
	\begin{enumerate}[label=\alph*),leftmargin=18pt]
		#1
	\end{enumerate}
}

\usepackage[]{hyperref}

\newcommand{\note}{Merk}
\newcommand{\notesm}[1]{{\footnotesize \textsl{\note:} #1}}
\newcommand{\ekstitle}{Eksempel }
\newcommand{\sprtitle}{Språkboksen}
\newcommand{\expl}{forklaring}
\newcommand{\pyt}{Pytagoras' setning}
\newcommand\sv{\vsk \textbf{Svar} \vspace{4 pt}\\}

%references
\newcommand{\reftab}[1]{\hrs{#1}{tabell}}
\newcommand{\rref}[1]{\hrs{#1}{regel}}
\newcommand{\dref}[1]{\hrs{#1}{definisjon}}
\newcommand{\refkap}[1]{\hrs{#1}{kapittel}}
\newcommand{\refsec}[1]{\hrs{#1}{seksjon}}
\newcommand{\refdsec}[1]{\hrs{#1}{delseksjon}}
\newcommand{\refved}[1]{\hrs{#1}{vedlegg}}
\newcommand{\eksref}[1]{\textsl{#1}}
\newcommand\fref[2][]{\hyperref[#2]{\textsl{figur \ref*{#2}#1}}}
\newcommand{\refop}[1]{{\color{blue}Oppgave \ref{#1}}}
\newcommand{\refops}[1]{{\color{blue}oppgave \ref{#1}}}


%Algebra
\newcommand{\kvadset}{Kvadratsetningene}
\newcommand{\aenato}{Sum-produkt-metoden}

% Geometry
\newcommand{\hlikb}{Midtnormalen i en likebeint trekant}
\newcommand{\arealsetn}{Arealsetningen}
\newcommand{\trkmedian}{Median}
\newcommand{\midtrk}{Midtnormal (i trekant)}
\newcommand{\innskrsirk}{Innskrevet sirkel}
\newcommand{\cossetn}{Cosinussetningen}
\newcommand{\perfvink}{Sentral- og periferivinkel}
\newcommand{\tang}{Tangent}

% Derivative
\newcommand{\derel}{Den deriverte av elementære funksjoner}
\newcommand{\divder}{Divisjonsregelen}
\newcommand{\kjernereg}{Kjerneregelen}
\newcommand{\prodregder}{Produktregelen}
\newcommand{\lhop}{L'Hopitals regel}

% Funksjonsdrofting
\newcommand{\monder}{Monotoniegenskaper og den deriverte}
\newcommand{\fderekstr}{$ \bm{f'=0} $ for lokale ektstremalpunkt}
\newcommand{\andredertest}{Andrederiverttesten}

% Vectors
\newcommand{\detar}{Arealformler med determinanter}
\newcommand{\avstpunktlin}{Avstand mellom punkt og linje}

%Appendix
\newcommand{\rolle}{Rolles teorem}
\newcommand{\meanval}{Middelverdisetningen}

% Solutions manual
\newcommand{\selos}{Se løsningsforslag.}

\begin{document}
\vedlegg{Tangeringslinja til en graf} \label{vdltangent}
\subsection*{Introduksjon}
Innen geometri er en \textit{tangeringslinje til en sirkel} definert som en linje som skjærer en sirkel i bare ett punkt (Moise, 1974). Av denne definisjonen kan det vises at 
\begin{itemize}
	\item en tangeringslinje står normalt på vektoren dannet av sentrum i sirkelen og skjæringspunktet
	\item enhver linje som har et skjæringspunkt med en sirkel, og hvor skjæringspunktet og sentrum i sirkelen danner en normalvektor til linja, er en tangeringslinje til sirkelen.
\end{itemize}
(Se \textsl{figur \ref{sirkoggraf}a}.) \vsk

Gitt en deriverbar funksjon $ f(x) $. Innen reell analyse defineres \textit{tangeringslinja til f i punktet $ (a, f(a)) $} som linja som går gjennom $ (a, f(a)) $ og har stigningstall $ f'(a) $ (Spivak, 1994). (Se \textsl{Figur \ref{sirkoggraf}b}.)
\begin{figure}[H]
\centering
\subfloat[]{\includegraphics{\figp{tang0a}}}\qquad \qquad
\subfloat[]{\includegraphics{\figp{tang0b}}}
\caption{\label{sirkoggraf}}
\end{figure}
Det er for mange ganske intuitivt at tangeringslinjer til sirkler og tangeringslinjer til grafer er nært beslektet, men formålet med denne teksten er å formalisere dette.

\subsection*{Senteret til krumningen}
Gitt en funksjon $ {\vec{r}(t)=[f(t), g(t)]} $ der $ f $ og $ g $ er kontinuerlige og to ganger deriverbare for alle $ t\in\mathbb{R} $, og hvor $ f''(t), g''(t)\neq0 $. For $ a, h\in\mathbb{R} $ setter vi $ b=a-h $ og $ c=a+h $. Videre innfører vi  punktene
\nn{
A=\vec{r}(a)\quad,\quad B= \vec{r}(b)\quad,\quad C=\vec{r}(c)
}
I tillegg introduserer vi skrivemåten $ k_d^{\hat{n}}(t) $, hvor $ \hat n $ erstattet med $ n $ eksemplar  av $ ' $ viser til den $ n $-te deriverte av funksjonen $ k(t) $ i punktet $ d $. \vsk

La $ S=(S_x, S_y) $ være sentrum i den omskrevne sirkelen til $ \triangle ABC $. På samme måte som vi finner den \textit{deriverte} i et punkt ved å la avstanden mellom to punkt på en graf gå mot 0, kan man finne \outl{krumningen} i et punkt ved å la avstanden mellom tre punkt gå mot 0. I vårt tilfelle er krumningen beskrevet av den omskrevne sirkelen til $ \triangle ABC $ når $ h $ går mot 0.
\begin{figure}[H]
\centering
\subfloat[]{
\includegraphics{\figp{tang1b}}
}
\subfloat[]{\includegraphics{\figp{tang1c}}}
\caption{\label{krumnfig}}
\end{figure}
\subsection*{Et likningssett for $\bm S $}
Vi har at
\alg{
	\vv{BA}&= [f_s-f_{b}, g_a-g_{b}] \br \vv{AC}&=[f_c-f_a, g_c-g_a]
}
La $ B_m $ og $ C_m $ være midptunktene til henholdsvis  (sekantene) $ AB $ og $ AC $. Da er
\nn{
	B_m = \frac{1}{2}(A+B)\qquad,\qquad C_m = \frac{1}{2}(A+C)
}
$ [g_a-g_b,f_b-f_a] $ er en normalvektor for $ \vv{BA} $, dette betyr at midtnormalen $\bm l_1 $ til $ AB $ kan parameterisere som
\nn{
{\bm l_1(p)}=B_m+[g_a-g_b,f_b-f_a] p \label{l1}
}
Tilsvarende er midtnormalen $\bm l_2 $ til $ AC $ parameterisert ved
\[ {\bm l_2(q)}=C_m+[g_a-g_c,f_c-f_a] q \]
$ S $ sammenfaller med skjæringspunktet til $\bm l_1 $ og $\bm l_2 $.
Ved å kreve at $\bm l_1= \bm l_2 $, får vi et lineært likningssett med $ p $ og $ q $ som ukjente. La $ q=q_s $ være løsningen av dette likningssettet, da vet vi at
\[ S=C_m+[g_a-g_c, f_c-f_a]q_s\]
Videre er
\[ \lim\limits_{h\to0}S=\lim\limits_{h\to0}\left(C_m+[g_a-g_c, f_c-g_a]\frac{h}{h}q_s\right)=A+[-g'_a, f'_a]\lim\limits_{h\to 0}h q_s \]
At grensen $ \lim\limits_{h\to 0}h q_s $ eksisterer viser vi avslutningsvis, og observerer nå dette: Når $ h\to0 $, blir $ \vv{AS} $ parallell med vektoren $ [-g'_a, f'_a] $. Vi har at $ \vec{r}'(a)=[f'_a, g'_a] $, og dermed er er 
\[ \vv{AS}\cdot \vec{r}\,'(a)=0 \]
Linja som går gjennom punktet $ \vec{r}(a) $, og som har $ \vec{r}\,'(a) $ som retningsvektor er altså en tangeringslinje til sirkelen som beskriver krumningen til $ \vec{r} $ i $ a $. \vsk

\subsubsection{Undersøkelse av grensenverdien}
Ved å løse det nevnte ligningssettet, finner vi at
\nn{
	q_s=\frac{1}{2}\frac{f_c(f_c-f_a)+f_b(f_a-f_c)+g_c(g_c-g_a)+g_b(g_a-g_c)}{f_b(g_c-g_a)+f_c(g_a-g_b)+f_a(g_b-g_c)}
}
Videre er
\[ \lim\limits_{h\to 0} q_s=\lim\limits_{h\to 0}\frac{h}{h}q_s=\lim\limits_{h\to 0}\frac{f_cf_a'-f_bf_a'+g_cg_a'-g_bg_a'}{f_bg_a'+f_cg_b'-2f_ag_a'} \]
Ved samme prosedyre har vi at
\[\lim\limits_{h\to 0}q_s= \lim\limits_{h\to 0} \frac{(f_a')^2+(g_a')^2}{f_a'g_b'-f_b'g_a'} \]
Videre har vi at {\footnotesize
\[ \lim\limits_{h\to 0}h q_s= h\frac{(f_a')^2+(g_a')^2}{f_a'g_b'-f_b'g_a'-f_b'g_b'+f_b'g_b'}=\lim\limits_{h\to 0}\frac{(f_a')^2+(g_a')^2}{f_b''g_b'+f_b'g_b''}=\frac{(f_a')^2+(g_a')^2}{f_a''g_a'+f_a'g_a''} \]}
\end{document}