\documentclass[english,hidelinks,pdftex, 11 pt, class=report,crop=false]{standalone}
\usepackage[T1]{fontenc}
\usepackage[utf8]{luainputenc}
\usepackage{lmodern} % load a font with all the characters
\usepackage{geometry}
\geometry{verbose,a4paper, inner=0cm, outer=0 cm, bmargin=2cm, tmargin=1cm}
%\textwidth=12cm
\setlength{\parindent}{0bp}
\usepackage{import}
\usepackage[subpreambles=false]{standalone}
\usepackage{amsmath}
\usepackage{amssymb}
\usepackage{esint}
\usepackage{babel}
\usepackage{tabu}
\usepackage[dvipsnames, table]{xcolor}
\usepackage{cancel}
\makeatother
\makeatletter
\usepackage{datetime2}
\usepackage{titlesec}
\usepackage[many]{tcolorbox}

% Eheter
\newcommand{\enh}[1]{\,\textrm{#1}}
%referances
\newcommand{\net}[2]{{\color{blue}\href{#1}{#2}}}

%Spaces
\newcommand{\vsk}{\\[12pt]}
\newcommand{\vs}{\vspace{-12pt}}

% Tabell for opplegg

\newcommand{\ovlist}[1]{
\vspace{-16pt}
\begin{itemize}
	#1
\end{itemize}
}

% Chapters and sections
\titleformat{\section}[block]{\bfseries}{\hspace{3cm}\thesection}{5pt}{}
\titleformat{\subsection}[block]{\bfseries}{\hspace{3cm}\thesection}{5pt}{}
\newcommand{\sectionbreak}{\clearpage} % New page on each section
 

\newlength{\mywidth}
\setlength{\mywidth}{14cm}

\newcommand{\cont}[1]{
\begin{tcolorbox}[center, boxrule=0.0 mm, width=\mywidth,arc=0mm,enhanced jigsaw,,colback=white,breakable]
#1	
\end{tcolorbox}
}

\newcommand{\info}[5]{
\begin{tcolorbox}[center, boxrule=0.1 mm, width=\mywidth,arc=0mm,enhanced jigsaw,breakable,colback=yellow!5]	
	
	\footnotesize
	\textbf{Øvingsområde}\\[5pt] #1 
	
	\textbf{Utstyr}\\ #2  \\
	
	\begin{tabular}{@{} p{4cm} p{4cm} l} 
		\textbf{Tid} & \textbf{Elevinndeling} & \textbf{Læringsarena} \\
		#3  & #4 & #5
	\end{tabular} 
\end{tcolorbox}	
}

\newcommand{\gjen}[1]{\begin{tcolorbox}[center,boxrule=0.1 mm, width=\mywidth,arc=0mm,colback=blue!3] {\large \textbf{Gjennomføring} \vspace{5 pt}}\newline #1  \end{tcolorbox}\vspace{-5pt}}
\newcommand{\eks}[1]{\begin{tcolorbox}[center,boxrule=0.1 mm, width=\mywidth,arc=0mm,colback=green!3] {\large \textbf{Eksempel} \vspace{5 pt}}\newline #1  \end{tcolorbox}\vspace{-5pt}}

\newcounter{opl}
%\numberwithin{opl}{article}


\newcommand{\opl}[1]{
\newpage
{\refstepcounter{opl} %\phantomsection 
\large \textbf{\theopl \;#1} \vsk}
}

% Headlines
\newcommand{\fork}{\textbf{Forkunnskapar}\\}
\newcommand{\forb}{\textbf{Forberedelsar}\\}
\newcommand{\opgvr}{\textbf{Oppgaver}}



%colors
\newcommand{\colr}[1]{{\color{red} #1}}
\newcommand{\colb}[1]{{\color{blue} #1}}
\newcommand{\colo}[1]{{\color{orange} #1}}
\newcommand{\colc}[1]{{\color{cyan} #1}}
\definecolor{projectgreen}{cmyk}{100,0,100,0}
\newcommand{\colg}[1]{{\color{projectgreen} #1}}

% Lister med bokstavar
\usepackage[inline]{enumitem}
% Opg
\newcommand{\abc}[1]{
	\begin{enumerate}[label=\alph*),leftmargin=18pt]
		#1
	\end{enumerate}
}

\usepackage[]{hyperref}

\newcommand{\note}{Merk}

% Geometry
\newcommand{\hlikb}{Midtnormalen i en likebeint trekant}
\newcommand{\arealsetn}{Arealsetningen}
\newcommand{\trkmedian}{Medianer i trekanter}
\newcommand{\midtrk}{Midtnormaler i trekanter}
\newcommand{\innskrsirk}{Halveringslinjer og innskrevet sirkel i trekanter}
\newcommand{\cossetn}{Cosinussetningen}
\newcommand{\perfvink}{Sentral- og periferivinkel}
\newcommand{\tang}{Tangent}


\begin{document}
\vedlegg{Tangeringslinja til en graf} \label{vdltangent}
\subsection*{Introduksjon}
Innen geometri er en \textit{tangeringslinje til en sirkel} definert som en linje som skjærer en sirkel i bare ett punkt (Moise, 1974). Av denne definisjonen kan det vises at 
\begin{itemize}
	\item en tangeringslinje står normalt på vektoren dannet av sentrum i sirkelen og skjæringspunktet
	\item enhver linje som har et skjæringspunkt med en sirkel, og hvor skjæringspunktet og sentrum i sirkelen danner en normalvektor til linja, er en tangeringslinje til sirkelen.
\end{itemize}
(Se \textsl{figur \ref{sirkoggraf}a}.) \vsk

Gitt en deriverbar funksjon $ f(x) $. Innen reell analyse defineres \textit{tangeringslinja til f i punktet $ (a, f(a)) $} som linja som går gjennom $ (a, f(a)) $ og har stigningstall $ f'(a) $ (Spivak, 1994). (Se \textsl{Figur \ref{sirkoggraf}b}.)
\begin{figure}[H]
\centering
\subfloat[]{\includegraphics{\figp{tang0a}}}\qquad \qquad
\subfloat[]{\includegraphics{\figp{tang0b}}}
\caption{\label{sirkoggraf}}
\end{figure}
Det er for mange ganske intuitivt at tangeringslinjer til sirkler og tangeringslinjer til grafer er nært beslektet, men formålet med denne teksten er å formalisere dette.
\begin{comment}
\begin{itemize}
	\item \textbf{Forkunnskaper}
	\item tangeringslinjer til sirkler
	\item omskrevne sirkler til trekanter
	\item vektorregning og skalarprodukt 
	\item parameterisering av linjer
	\item regneregler for grenseverdier
	\item definisjon av den deriverte og den andrederiverte 
	\item likningssett
	\item L'Hopitals regel	
	\item derivasjonsregler
\end{itemize}
\end{comment}
\subsection*{Senteret til krumningen}
Gitt en funksjon $ f(x) $ som er kontinuerlig og to ganger deriverbar for alle $ x\in\mathbb{R} $, og hvor $ f''(x)\neq0 $. For en gitt $ a $ lar vi $ {f_a=f(a)} $, og definerer funksjonene
\nn{
f_b(h)=f(a-h)\quad, \quad f_c(h)=f(a+h)
}
Vi innfører også punktene
\nn{
A=(a, f_a)\quad,\quad B=(a-h, f_b)\quad,\quad C=(a+h, f_c)
}
Videre lar vi $ S=(S_x, S_y) $ være sentrum i den omskrevne sirkelen til $ \triangle ABC $. På samme måte som vi finner den \textit{deriverte} i et punkt ved å la avstanden mellom to punkt på en graf gå mot 0, kan man finne \textit{krumningen} i et punkt ved å la avstanden mellom tre punkt gå mot 0. I vårt tilfelle er krumningen beskrevet av den omskrevne sirkelen til $ \triangle ABC $ når $ h $ går mot 0.
\begin{figure}[H]
\centering
\subfloat[]{
\includegraphics{\figp{tang1b}}
}
\subfloat[]{\includegraphics{\figp{tang1c}}}
\caption{\label{krumnfig}}
\end{figure}
\subsection*{Et likningssett for $\bm S $}
Vi har at
\nn{
	\vv{BA}= [h,f_a-f_b] \quad,\quad \vv{AC}=[h, f_c-f_a]
}
La $ B_m $ og $ C_m $ være midptunktene til henholdsvis  (sekantene) $ AB $ og $ AC $. Da er
\nn{
	B_m = B+\frac{1}{2} \vv{BA} \qquad,\qquad C_m = C+\frac{1}{2}\vv{AC} 
}
$ [f_a-f_b,-h] $ er en normalvektor for $ \vv{BA} $, dette betyr at midtnormalen $\bm l_1 $ til sekanten $ AB $ kan parameterisere som
\nn{
{\bm l_1(t)}=B_m+[f_a-f_b, -h]t \label{l1}
}
Tilsvarende er midtnormalen $\bm l_2 $ til sekanten $ AC $ parameterisert ved
\[ {\bm l_2(q)}=C_m+[f_c-f_a, -h]q \]
$ S $ sammenfaller med skjæringspunktet til $\bm l_1 $ og $\bm l_2 $.
Ved å kreve at $\bm l_1= \bm l_2 $, får vi et lineært likningssett med to ukjente som gir
\nn{
t=\frac{(f_a-f_c)(f_b-f_c)+2h^2}{2h(f_b+f_c-2f_a)} \label{t}
}
\subsection*{$\bm S$ når $\bm {h}$ går mot 0}
Vi definerer funkjonene $ \dot{f}_b$ , $ \dot{f}_c  $, $ \ddot{f}_b $ og $ \ddot{f}_c $ ut ifra de (respektive) deriverte og andrederiverte av $ f_b $ og $ f_c $ med hensyn på $ h $:
\alg{
	-\dot{f}_b&=\left(f_b\right)' =-f'(a-h) \vn
	\dot{f}_c&=\left(f_c\right)'=f'(a+h) \vn
	\ddot{f}_b&=\left(f_b\right)'' =f''(a-h) \vn
	\ddot{f}_c&=\left(f_c\right)''=f''(a+h)
}
Vi skal nå bruke dise funsjonene til å studere koordinatene til $ S $ når $ h $ går mot 0. Vi tar da med oss at
\alg{
&\lim\limits_{h\to0} \left\{h^2, h\right\}=0 &&
&\lim\limits_{h\to0} \left\{f_b, f_c\right\}=f_a \vn
&\lim\limits_{h\to0} \left\{\dot{f}_c,\dot{f}_b\right\}=f'_a&&
&\lim\limits_{h\to0} \left\{\ddot{f}_b, \ddot{f}_c\right\}=f_a''
}
hvor\footnote{Legg merke til at det her er snakk om $ f $ derivert med hensyn på $ x $, og evaluert i $ a $.} $ f'_a=f'(a) $ og $ f''_a=f''(a) $. \vsk

For $ t $ uttrykt ved \eqref{t} er (se \eqref{l1})
\nn{
	S_y=\frac{f_a+f_b+2ht}{2} = \frac{f_a+f_b}{2}+ht \label{sy}
}
Vi har at
\nn{
\lim\limits_{h\to 0} \frac{f_a+f_b}{2}=f_a
}
Videre er
\alg{
ht&=\frac{(f_c-f_a)(f_b-f_c)+2h^2}{2(f_b+f_c-2f_a)} \br
&=\frac{(f_c-f_a)(f_b-f_c)}{2(f_b+f_c-2f_a)}+\frac{h^2}{f_b+f_c-2f_a} \label{ht}
}
Når $ h $ går mot 0, er begge leddene i \eqref{ht} <<0 over 0>> uttrykk. Vi bruker L'Hopitals regel på det siste leddet: 
\begin{align}
	&&\lim\limits_{h\to0}\frac{h^2}{2(f_b+f_c-2f_a)}&=\lim\limits_{h\to0}\frac{\left(h^2\right)'}{(f_b+f_c-2f_a)'} \br
	&&&=\lim\limits_{h\to0} \frac{2h}{-\dot{f}_b+\dot{f}_c} &&\text{<<0 over 0>>}\br
	&&&=\lim\limits_{h\to0} \frac{2}{\ddot{f}_b+\ddot{f}_c}  \\
	&& &=\frac{1}{f_a''} \label{ht2}
\end{align}

Ved å bruke L'Hopitals regel på det første leddet i \eqref{ht} har vi at
\alg{
\lim\limits_{h\to 0}\frac{(f_c-f_a)(f_b-f_c)}{f_b+f_c-2f_a} &= \lim\limits_{h\to 0}\frac{\left((f_c-f_a)(f_b-f_c)\right)'}{(f_b+f_c-2f_a)'} 
}
Av produktregelen ved derivasjon er
\alg{
\lim\limits_{h\to 0}\frac{\left((f_a-f_c)(f_b-f_c)\right)'}{(f_b+f_c-2f_a)'} &=\lim\limits_{h\to 0} \left[\frac{\dot{f_c}(f_b-f_c)}{-\dot{f}_b+\dot{f}_c}+\frac{(f_c-f_a)(\dot{f}_b+\dot{f}_c)}{-\dot{f}_b+\dot{f}_c}\right]
}
Begge leddene over er <<0 over 0>> uttrykk. Vi undersøker dem hver for seg ved å anvende L'Hopitals regel:
\alg{
\lim\limits_{h\to 0}\frac{\dot{f_c}(f_b-f_c)}{-\dot{f}_b+\dot{f}_c} &= \lim\limits_{h\to 0} \left[\frac{\ddot{f}_c(f_b-f_c)}{\ddot{f}_b+\ddot{f}_c}+\frac{\dot{f}_c(\dot{f}_b+\dot{f}_c)}{\ddot{f}_b+\ddot{f}_c}\right] \br
&= 0+\frac{(f_a')^2}{2f_a''} \label{ht1a}
}
\begin{align}
	\lim\limits_{h\to 0}\frac{(f_c-f_a)(\dot{f}_b+\dot{f}_c)}{-\dot{f}_b+\dot{f}_c}&= \lim\limits_{h\to 0}\left[\frac{\dot{f}_c(\dot{f}_b+\dot{f}_c)}{\ddot{f}_b+\ddot{f}_c}+\frac{(f_c-f_a)(-\dot{f}_b+\dot{f}_c)}{\ddot{f}_b+\ddot{f}_c}\right]\br
	&=\frac{(f_a')^2}{2f_a''}+0 \label{ht1b}
\end{align}
Av \eqref{ht}, \eqref{ht2}, \eqref{ht1a} og \eqref{ht1b} har vi at
\[ \lim\limits_{h\to0} h t=\frac{1+(f_a')^2}{f_a''} \]
Dermed er 
\nn{
S_y=f_a+\frac{1+(f_a')^2}{f''_a}
}

Videre er (med $ t $ gitt av \eqref{t})
\[ S_x=(f_b-f_a)t+a-\frac{1}{2}h \]
Vi har at
\algv{
\lim\limits_{h\to0}(f_b-f_a)t&=\lim\limits_{h\to0} \frac{f_b-f_a}{h}\cdot ht \br
&= \lim\limits_{h\to0} \frac{f_b-f_a}{h}\cdot \lim\limits_{h\to0} ht  \br
&=-f_a'\frac{1+(f_a')^2}{f''_a}
}
Altså er
\nn{S_x=a-f'_a\frac{1+(f_a')^2}{f''_a}}
\subsection*{Avslutning}
Linja som har stigningstall $ f'(a) $, og som går gjennom $ (a, f(a)) $, er gitt ved funksjonen 
\[ g(x)=f_a'(x-a)+f_a \]
$ \vec{r}=[1, f_a] $ er en retningsvektoren til denne linja. 
Av uttrykkene vi har funnet for $ S_x $ og $ S_y $ har vi at
\[ S=\left(a-f'_a\frac{1+(f_a')^2}{f''_a}, f_a+\frac{1+(f_a')^2}{f''_a}\right) \]
Dermed er
\[ \vv{AS}= \frac{1}{f_a''}\left[-f_a(1+(f_a')^2), 1+(f_a')^2\right] \]
Siden $ {\vec{r}\cdot \vv{AS}=0 }$ og $ {g(a)=f(a)} $, er grafen til $ g $ tangeringslinja til sirkelen med sentrum $ S $ når $ h $ går mot 0. Altså er $ g $ tangeringslinja til sirkelen som beskriver krumningen til $ f $ når $ x=a $.


\end{document}