\documentclass[english,hidelinks,pdftex, 11 pt, class=report,crop=false]{standalone}
\usepackage[T1]{fontenc}
\usepackage[utf8]{luainputenc}
\usepackage{lmodern} % load a font with all the characters
\usepackage{geometry}
\geometry{verbose,a4paper, inner=0cm, outer=0 cm, bmargin=2cm, tmargin=1cm}
%\textwidth=12cm
\setlength{\parindent}{0bp}
\usepackage{import}
\usepackage[subpreambles=false]{standalone}
\usepackage{amsmath}
\usepackage{amssymb}
\usepackage{esint}
\usepackage{babel}
\usepackage{tabu}
\usepackage[dvipsnames, table]{xcolor}
\usepackage{cancel}
\makeatother
\makeatletter
\usepackage{datetime2}
\usepackage{titlesec}
\usepackage[many]{tcolorbox}

% Eheter
\newcommand{\enh}[1]{\,\textrm{#1}}
%referances
\newcommand{\net}[2]{{\color{blue}\href{#1}{#2}}}

%Spaces
\newcommand{\vsk}{\\[12pt]}
\newcommand{\vs}{\vspace{-12pt}}

% Tabell for opplegg

\newcommand{\ovlist}[1]{
\vspace{-16pt}
\begin{itemize}
	#1
\end{itemize}
}

% Chapters and sections
\titleformat{\section}[block]{\bfseries}{\hspace{3cm}\thesection}{5pt}{}
\titleformat{\subsection}[block]{\bfseries}{\hspace{3cm}\thesection}{5pt}{}
\newcommand{\sectionbreak}{\clearpage} % New page on each section
 

\newlength{\mywidth}
\setlength{\mywidth}{14cm}

\newcommand{\cont}[1]{
\begin{tcolorbox}[center, boxrule=0.0 mm, width=\mywidth,arc=0mm,enhanced jigsaw,,colback=white,breakable]
#1	
\end{tcolorbox}
}

\newcommand{\info}[5]{
\begin{tcolorbox}[center, boxrule=0.1 mm, width=\mywidth,arc=0mm,enhanced jigsaw,breakable,colback=yellow!5]	
	
	\footnotesize
	\textbf{Øvingsområde}\\[5pt] #1 
	
	\textbf{Utstyr}\\ #2  \\
	
	\begin{tabular}{@{} p{4cm} p{4cm} l} 
		\textbf{Tid} & \textbf{Elevinndeling} & \textbf{Læringsarena} \\
		#3  & #4 & #5
	\end{tabular} 
\end{tcolorbox}	
}

\newcommand{\gjen}[1]{\begin{tcolorbox}[center,boxrule=0.1 mm, width=\mywidth,arc=0mm,colback=blue!3] {\large \textbf{Gjennomføring} \vspace{5 pt}}\newline #1  \end{tcolorbox}\vspace{-5pt}}
\newcommand{\eks}[1]{\begin{tcolorbox}[center,boxrule=0.1 mm, width=\mywidth,arc=0mm,colback=green!3] {\large \textbf{Eksempel} \vspace{5 pt}}\newline #1  \end{tcolorbox}\vspace{-5pt}}

\newcounter{opl}
%\numberwithin{opl}{article}


\newcommand{\opl}[1]{
\newpage
{\refstepcounter{opl} %\phantomsection 
\large \textbf{\theopl \;#1} \vsk}
}

% Headlines
\newcommand{\fork}{\textbf{Forkunnskapar}\\}
\newcommand{\forb}{\textbf{Forberedelsar}\\}
\newcommand{\opgvr}{\textbf{Oppgaver}}



%colors
\newcommand{\colr}[1]{{\color{red} #1}}
\newcommand{\colb}[1]{{\color{blue} #1}}
\newcommand{\colo}[1]{{\color{orange} #1}}
\newcommand{\colc}[1]{{\color{cyan} #1}}
\definecolor{projectgreen}{cmyk}{100,0,100,0}
\newcommand{\colg}[1]{{\color{projectgreen} #1}}

% Lister med bokstavar
\usepackage[inline]{enumitem}
% Opg
\newcommand{\abc}[1]{
	\begin{enumerate}[label=\alph*),leftmargin=18pt]
		#1
	\end{enumerate}
}

\usepackage[]{hyperref}

\newcommand{\note}{Merk}
\newcommand{\notesm}[1]{{\footnotesize \textsl{\note:} #1}}
\newcommand{\ekstitle}{Eksempel }
\newcommand{\sprtitle}{Språkboksen}
\newcommand{\expl}{forklaring}
\newcommand{\pyt}{Pytagoras' setning}
\newcommand\sv{\vsk \textbf{Svar} \vspace{4 pt}\\}

%references
\newcommand{\reftab}[1]{\hrs{#1}{tabell}}
\newcommand{\rref}[1]{\hrs{#1}{regel}}
\newcommand{\dref}[1]{\hrs{#1}{definisjon}}
\newcommand{\refkap}[1]{\hrs{#1}{kapittel}}
\newcommand{\refsec}[1]{\hrs{#1}{seksjon}}
\newcommand{\refdsec}[1]{\hrs{#1}{delseksjon}}
\newcommand{\refved}[1]{\hrs{#1}{vedlegg}}
\newcommand{\eksref}[1]{\textsl{#1}}
\newcommand\fref[2][]{\hyperref[#2]{\textsl{figur \ref*{#2}#1}}}
\newcommand{\refop}[1]{{\color{blue}Oppgave \ref{#1}}}
\newcommand{\refops}[1]{{\color{blue}oppgave \ref{#1}}}


%Algebra
\newcommand{\kvadset}{Kvadratsetningene}
\newcommand{\aenato}{Sum-produkt-metoden}

% Geometry
\newcommand{\hlikb}{Midtnormalen i en likebeint trekant}
\newcommand{\arealsetn}{Arealsetningen}
\newcommand{\trkmedian}{Median}
\newcommand{\midtrk}{Midtnormal (i trekant)}
\newcommand{\innskrsirk}{Innskrevet sirkel}
\newcommand{\cossetn}{Cosinussetningen}
\newcommand{\perfvink}{Sentral- og periferivinkel}
\newcommand{\tang}{Tangent}

% Derivative
\newcommand{\derel}{Den deriverte av elementære funksjoner}
\newcommand{\divder}{Divisjonsregelen}
\newcommand{\kjernereg}{Kjerneregelen}
\newcommand{\prodregder}{Produktregelen}
\newcommand{\lhop}{L'Hopitals regel}

% Funksjonsdrofting
\newcommand{\monder}{Monotoniegenskaper og den deriverte}
\newcommand{\fderekstr}{$ \bm{f'=0} $ for lokale ektstremalpunkt}
\newcommand{\andredertest}{Andrederiverttesten}

% Vectors
\newcommand{\detar}{Arealformler med determinanter}
\newcommand{\avstpunktlin}{Avstand mellom punkt og linje}

%Appendix
\newcommand{\rolle}{Rolles teorem}
\newcommand{\meanval}{Middelverdisetningen}

% Solutions manual
\newcommand{\selos}{Se løsningsforslag.}

\begin{document}	
\subsubsection{Den deriverte som motivajon}
Gitt funksjonen $ f(x)=a^x $. Da har vi at
\alg{
\left(a^x\right)'&=\lim\limits_{h\to 0}\frac{a^{x+h}-a^x}{h}\br
&= \lim\limits_{h\to 0}\frac{a^{x}a^h-a^x}{h}
}
Da $ x $ er uavhengig av $ h $, får vi at
\alg{
\left(a^x\right)'=a^x\lim\limits_{h\to 0}\frac{a^h-1}{h}
}
Likningen over peker mot noe fantastisk; hvis det finnes et tall $ a $ som er slik at $ {\lim\limits_{h\to 0}\frac{a^h-1}{h}=1} $, så vil funksjonen $ a^x $ være sin egen deriverte funksjon! Altså er da $ \left(a^x\right)'=a^x $. Vi legger nå merke til at hvis
\[ a=\lim\limits_{h\to 0}\left(1+h\right)^\frac{1}{h} \]
så er 
\algv{
\lim\limits_{h\to 0}\frac{a^h-1}{h}&=\lim\limits_{h\to 0}\frac{\left(\left(1+h\right)^\frac{1}{h}\right)^h-1}{h}\br
&=\frac{1+h-1}{h}\\
&=1
}
Hvis vi kan vise at grenseverdien $ \lim\limits_{h\to 0}\left(1+h\right)^\frac{1}{h} $ eksisterer, har vi altså funnet akkurat det uttrykket for $ a $ som vi ønsket oss.

\subsubsection{Undersøking av grenseverdien}
Ved å sette $ z=\frac{1}{h} $, kan vi skrive
\[ \lim\limits_{h\to0}(1+h)^\frac{1}{h}=\lim\limits_{n\to \infty}\left(1+\frac{1}{z}\right)^z \]
Vi setter $ {\lim\limits_{z\to \infty}\left(1+\frac{1}{z}\right)^z=e}  $. Spørsmålet nå er om $ e $ har en endelig verdi eller ikke. Det første vi kan legge merke til, er at for alle $ {z\geq0 }$ er
\[ 1+\frac{1}{n}\geq1  \]
Dette må bety at $ e $ er voksende når $ {z\to\infty} $. Så lenge $ e $ har en øvre grense, kan vi da være sikre på at uttrykket går mot en endelig verdi\footnote{Hvis $ e $ hadde vært både voksende og avtagende når $ {z\to\infty} $, ville uttrykket svinget i verdi, og jobben med å finne en øvre og nedre grense ville fort blitt vanskeligere.}.\vsk

Vi setter $ {\lim\limits_{n\to \infty}\left(1+\frac{1}{n}\right)^n=e_n} $, hvor $ {n\in\mathbb{N}} $. Hvis $ e_n $ har en endelig verdi, er det åpenbart at $ e=e_n $.
Av binomialteoremet (se ??) har vi at
\begin{align}
	e_n&=1+\binom{n}{0}\frac{1}{n}+\binom{n}{2}\left(\frac{1}{n}\right)^2+\binom{n}{3}\left(\frac{1}{n}\right)^3+...\nonumber\\
	&= 1+1+\frac{n(n-1)}{2!n^2}+\frac{n(n-1)(n-2)}{3!n^3}+... \label{eforklns}
\end{align}
\begin{comment}
Videre legger vi merke til at
\alg{
\frac{n(n-1)}{2!n^2}&= \frac{1}{2!}\left(\frac{n}{n}-\frac{1}{n}\right)\br&=\frac{1}{2!}\left(1-\frac{1}{n}\right) \\[15pt]
\frac{n(n-1)(n-2)}{3!n^3}&=\frac{1}{3!}\left(1-\frac{1}{n}\right)\left(1-\frac{2}{n}\right)
}
Altså er
\begin{equation}\label{eforklvalpre}
\left(1+\frac{1}{n}\right)^n=1+1+\frac{1}{2!}\left(1-\frac{1}{n}\right)+\frac{1}{3!}\left(1-\frac{1}{n}\right)\left(1-\frac{2}{n}\right)+...
\end{equation}
\end{comment}
Gitt et heltall $ k $, hvor $ 1\leq k\leq n $. Da $\lim\limits_{n\to\infty} \frac{n-k}{n}\leq 1 $, følger det av \eqref{eforklns} at
\begin{equation}\label{eforkval}
2<e_n\leq1+1+\frac{1}{2!}+\frac{1}{3!}+...
\end{equation}
Siden  $ k!\geq 2^{k-1} $, er
\begin{equation}\label{eforklsup2n}
	1+\frac{1}{2!}+\frac{1}{3!}+...<1+\frac{1}{2}+\frac{1}{2^2}+...
\end{equation}
Uttrykket til høgre kjenner vi igjen som en uendelig geometrisk rekke med $ a_1=1 $ og $ k=\frac{1}{2} $. Summen $ S $ av denne rekka er (se AM2??) gitt som
\alg{
S&= \frac{a_1}{1-k} \\
&= \frac{1}{1-\frac{1}{2}} \\
&=2
}
Dette betyr at venstresiden i \eqref{eforklsup2n} er mindre enn 2, og da har vi av \eqref{eforkval} at
\[ e_n<3 \]
Altså er
\[ 2<e<3\]
Nå vet vi at $ e $ går mot en endelig verdi, og det gir derfor mening å behandle $ e $ som et gitt tall.

\subsubsection{Et tilbakeblikk på den deriverte}
Derivasjon av potensfunksjoner var det som motiverte oss til å undersøke tallet $ e $. Av det vi har drøftet i de foregående avsnittene, følger det at
\[ \left(e^x\right)'=e^x \]
Likningen over er rett og slett én av de viktigste likningene i matematikk.
\end{document}