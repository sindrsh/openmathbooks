\documentclass[english,hidelinks,pdftex, 11 pt, class=report,crop=false]{standalone}
\usepackage[T1]{fontenc}
\usepackage[utf8]{luainputenc}
\usepackage{lmodern} % load a font with all the characters
\usepackage{geometry}
\geometry{verbose,a4paper, inner=0cm, outer=0 cm, bmargin=2cm, tmargin=1cm}
%\textwidth=12cm
\setlength{\parindent}{0bp}
\usepackage{import}
\usepackage[subpreambles=false]{standalone}
\usepackage{amsmath}
\usepackage{amssymb}
\usepackage{esint}
\usepackage{babel}
\usepackage{tabu}
\usepackage[dvipsnames, table]{xcolor}
\usepackage{cancel}
\makeatother
\makeatletter
\usepackage{datetime2}
\usepackage{titlesec}
\usepackage[many]{tcolorbox}

% Eheter
\newcommand{\enh}[1]{\,\textrm{#1}}
%referances
\newcommand{\net}[2]{{\color{blue}\href{#1}{#2}}}

%Spaces
\newcommand{\vsk}{\\[12pt]}
\newcommand{\vs}{\vspace{-12pt}}

% Tabell for opplegg

\newcommand{\ovlist}[1]{
\vspace{-16pt}
\begin{itemize}
	#1
\end{itemize}
}

% Chapters and sections
\titleformat{\section}[block]{\bfseries}{\hspace{3cm}\thesection}{5pt}{}
\titleformat{\subsection}[block]{\bfseries}{\hspace{3cm}\thesection}{5pt}{}
\newcommand{\sectionbreak}{\clearpage} % New page on each section
 

\newlength{\mywidth}
\setlength{\mywidth}{14cm}

\newcommand{\cont}[1]{
\begin{tcolorbox}[center, boxrule=0.0 mm, width=\mywidth,arc=0mm,enhanced jigsaw,,colback=white,breakable]
#1	
\end{tcolorbox}
}

\newcommand{\info}[5]{
\begin{tcolorbox}[center, boxrule=0.1 mm, width=\mywidth,arc=0mm,enhanced jigsaw,breakable,colback=yellow!5]	
	
	\footnotesize
	\textbf{Øvingsområde}\\[5pt] #1 
	
	\textbf{Utstyr}\\ #2  \\
	
	\begin{tabular}{@{} p{4cm} p{4cm} l} 
		\textbf{Tid} & \textbf{Elevinndeling} & \textbf{Læringsarena} \\
		#3  & #4 & #5
	\end{tabular} 
\end{tcolorbox}	
}

\newcommand{\gjen}[1]{\begin{tcolorbox}[center,boxrule=0.1 mm, width=\mywidth,arc=0mm,colback=blue!3] {\large \textbf{Gjennomføring} \vspace{5 pt}}\newline #1  \end{tcolorbox}\vspace{-5pt}}
\newcommand{\eks}[1]{\begin{tcolorbox}[center,boxrule=0.1 mm, width=\mywidth,arc=0mm,colback=green!3] {\large \textbf{Eksempel} \vspace{5 pt}}\newline #1  \end{tcolorbox}\vspace{-5pt}}

\newcounter{opl}
%\numberwithin{opl}{article}


\newcommand{\opl}[1]{
\newpage
{\refstepcounter{opl} %\phantomsection 
\large \textbf{\theopl \;#1} \vsk}
}

% Headlines
\newcommand{\fork}{\textbf{Forkunnskapar}\\}
\newcommand{\forb}{\textbf{Forberedelsar}\\}
\newcommand{\opgvr}{\textbf{Oppgaver}}



%colors
\newcommand{\colr}[1]{{\color{red} #1}}
\newcommand{\colb}[1]{{\color{blue} #1}}
\newcommand{\colo}[1]{{\color{orange} #1}}
\newcommand{\colc}[1]{{\color{cyan} #1}}
\definecolor{projectgreen}{cmyk}{100,0,100,0}
\newcommand{\colg}[1]{{\color{projectgreen} #1}}

% Lister med bokstavar
\usepackage[inline]{enumitem}
% Opg
\newcommand{\abc}[1]{
	\begin{enumerate}[label=\alph*),leftmargin=18pt]
		#1
	\end{enumerate}
}

\usepackage[]{hyperref}

% note
\newcommand{\note}{Note}
\newcommand{\notesm}[1]{{\footnotesize \textsl{\note:} #1}}
\newcommand{\selos}{See the solutions manual.}

\newcommand{\texandasy}{The text is written in \LaTeX\ and the figures are made with the aid of Asymptote.}

\newcommand{\rknut}{Calculate.}
\newcommand\sv{\vsk \textbf{Answer} \vspace{4 pt}\\}
\newcommand{\ekstitle}{Example }
\newcommand{\sprtitle}{The language box}
\newcommand{\expl}{explanation}

% answers
\newcommand{\mulansw}{\notesm{Multiple possible answers.}}	
\newcommand{\faskap}{Chapter}

% exercises
\newcommand{\opgt}{\newpage \phantomsection \addcontentsline{toc}{section}{Exercises} \section*{Exercises for Chapter \thechapter}\vs \setcounter{section}{1}}

% references
\newcommand{\reftab}[1]{\hrs{#1}{Table}}
\newcommand{\rref}[1]{\hrs{#1}{Rule}}
\newcommand{\dref}[1]{\hrs{#1}{Definition}}
\newcommand{\refkap}[1]{\hrs{#1}{Chapter}}
\newcommand{\refsec}[1]{\hrs{#1}{Section}}
\newcommand{\refdsec}[1]{\hrs{#1}{Subsection}}
\newcommand{\refved}[1]{\hrs{#1}{Appendix}}
\newcommand{\eksref}[1]{\textsl{#1}}
\newcommand\fref[2][]{\hyperref[#2]{\textsl{Figure \ref*{#2}#1}}}
\newcommand{\refop}[1]{{\color{blue}Exercise \ref{#1}}}
\newcommand{\refops}[1]{{\color{blue}Exercise \ref{#1}}}

%%% SECTION HEADLINES %%%

% Our numbers
\newcommand{\likteikn}{The equal sign}
\newcommand{\talsifverd}{Numbers, digits and values}
\newcommand{\koordsys}{Coordinate systems}

% Calculations
\newcommand{\adi}{Addition}
\newcommand{\sub}{Subtraction}
\newcommand{\gong}{Multiplication}
\newcommand{\del}{Division}

%Factorization and order of operations
\newcommand{\fak}{Factorization}
\newcommand{\rrek}{Order of operations}

%Fractions
\newcommand{\brgrpr}{Introduction}
\newcommand{\brvu}{Values, expanding and simplifying}
\newcommand{\bradsub}{Addition and subtraction}
\newcommand{\brgngheil}{Fractions multiplied by integers}
\newcommand{\brdelheil}{Fractions divided by integers}
\newcommand{\brgngbr}{Fractions multiplied by fractions}
\newcommand{\brkans}{Cancelation of fractions}
\newcommand{\brdelmbr}{Division by fractions}
\newcommand{\Rasjtal}{Rational numbers}

%Negative numbers
\newcommand{\negintro}{Introduction}
\newcommand{\negrekn}{The elementary operations}
\newcommand{\negmeng}{Negative numbers as amounts}

%Calculation methods
\newcommand{\delmedtihundre}{Deling med 10, 100, 1\,000 osv.}

% Geometry 1
\newcommand{\omgr}{Terms}
\newcommand{\eignsk}{Attributes of triangles and quadrilaterals}
\newcommand{\omkr}{Perimeter}
\newcommand{\area}{Area}

%Algebra 
\newcommand{\algintro}{Introduction}
\newcommand{\pot}{Powers}
\newcommand{\irrasj}{Irrational numbers}

%Equations
\newcommand{\ligintro}{Introduction}
\newcommand{\liglos}{Solving with the elementary operations}
\newcommand{\ligloso}{Solving with elementary operations summarized}

%Functions
\newcommand{\fintro}{Introduction}
\newcommand{\lingraf}{Linear functions and graphs}

%Geometry 2
\newcommand{\geoform}{Formulas of area and perimeter}
\newcommand{\kongogsim}{Congruent and similar triangles}
\newcommand{\geofork}{Explanations}

% Names of rules
\newcommand{\adkom}{Addition is commutative}
\newcommand{\gangkom}{Multiplication is commutative}
\newcommand{\brdef}{Fractions as rewriting of division}
\newcommand{\brtbr}{Fractions multiplied by fractions}
\newcommand{\delmbr}{Fractions divided by fractions}
\newcommand{\gangpar}{Distributive law}
\newcommand{\gangparsam}{Paranthesis multiplied together}
\newcommand{\gangmnegto}{Multiplication by negative numbers I}
\newcommand{\gangmnegtre}{Multiplication by negative numbers II}
\newcommand{\konsttre}{Unique construction of triangles}
\newcommand{\kongtre}{Congruent triangles}
\newcommand{\topv}{Vertical angles}
\newcommand{\trisum}{The sum of angles in a triangle}
\newcommand{\firsum}{The sum of angles in a quadrilateral}
\newcommand{\potgang}{Multiplication by powers}
\newcommand{\potdivpot}{Division by powers}
\newcommand{\potanull}{The special case of \boldmath $a^0$}
\newcommand{\potneg}{Powers with negative exponents}
\newcommand{\potbr}{Fractions as base}
\newcommand{\faktgr}{Factors as base}
\newcommand{\potsomgrunn}{Powers as base}
\newcommand{\arsirk}{The area of a circle}
\newcommand{\artrap}{The area of a trapezoid}
\newcommand{\arpar}{The area of a parallelogram}
\newcommand{\pyt}{Pythagoras's theorem}
\newcommand{\forform}{Ratios in similar triangles}
\newcommand{\vilkform}{Terms of similar triangles}
\newcommand{\omkrsirk}{The perimeter of a circle (and the value of $ \bm \pi $)}
\newcommand{\artri}{The area of a triangle}
\newcommand{\arrekt}{The area of a rectangle}
\newcommand{\liknflyt}{Moving terms across the equal sign}
\newcommand{\funklin}{Linear functions}



\begin{document}

\vedlegg{The Tangent Line to a Graph} \label{vdltangent}
\subsection*{Introduction}
In geometry, a \textit{tangent line to a circle} is defined as a line that intersects a circle at exactly one point (Moise, 1974). From this definition, it can be shown that
\begin{itemize}
	\item a tangent line is perpendicular to the vector formed by the center of the circle and the point of intersection
	\item any line that has an intersection point with a circle, and where the intersection point and the center of the circle form a normal vector to the line, is a tangent line to the circle.
\end{itemize}
(See \textsl{Figure \ref{sirkoggraf}a}.) \vsk

Given a differentiable function $ f(x) $. In real analysis, the \textit{tangent line to $ f $ at the point $ (a, f(a)) $} is defined as the line passing through $ (a, f(a)) $ and having a slope of $ f'(a) $ (Spivak, 1994). (See \textsl{Figure \ref{sirkoggraf}b}.)
\begin{figure}[H]
	\centering
	\subfloat[]{\includegraphics{\figp{tang0a}}}\qquad \qquad
	\subfloat[]{\includegraphics{\figp{tang0b}}}
	\caption{\label{sirkoggraf}}
\end{figure}
It is quite intuitive for many that tangent lines to circles and tangent lines to graphs are closely related, but the purpose of this text is to formalize this.

\subsection*{The Center of Curvature}
Given a function $ {\vec{r}(t)=[f(t), g(t)]} $ where $ f $ and $ g $ are continuous and twice differentiable for all $ t\in\mathbb{R} $, and where $ f''(t), g''(t)\neq0 $. For $ a, h\in\mathbb{R} $ we set $ b=a-h $ and $ c=a+h $. Furthermore, we introduce the points
\nn{
	A=\vec{r}(a)\quad,\quad B= \vec{r}(b)\quad,\quad C=\vec{r}(c)
}
In addition, we introduce the notation $ k_d^{\hat{n}}(t) $, where $ \hat n $ replaced with $ n $ instances of $ ' $ denotes the $ n $-th derivative of the function $ k(t) $ at the point $ d $. \vsk

Let $ S=(S_x, S_y) $ be the center of the circumcircle of $ \triangle ABC $. In the same way that we find the \textit{derivative} at a point by letting the distance between two points on a graph approach 0, we can find the \outl{curvature} at a point by letting the distance between three points approach 0. In our case, the curvature is described by the circumcircle of $ \triangle ABC $ as $ h $ approaches 0.
\begin{figure}[H]
	\centering
	\subfloat[]{
		\includegraphics{\figp{tang1b}}
	}
	\subfloat[]{\includegraphics{\figp{tang1c}}}
	\caption{\label{krumnfig}}
\end{figure}
\subsection*{A System of Equations for finding $\bm S $}
We have that
\alg{
	\vv{BA}&= [f_s-f_{b}, g_a-g_{b}] \br \vv{AC}&=[f_c-f_a, g_c-g_a]
}
Let $ B_m $ and $ C_m $ be the midpoints of (the secants) $ AB $ and $ AC $, respectively. Then,
\nn{
	B_m = \frac{1}{2}(A+B)\qquad,\qquad C_m = \frac{1}{2}(A+C)
}
$ [g_a-g_b,f_b-f_a] $ is a normal vector for $ \vv{BA} $, which means that the perpendicular bisector $\bm l_1 $ of $ AB $ can be parameterized as
\nn{
	{\bm l_1(p)}=B_m+[g_a-g_b,f_b-f_a] p \label{l1}
}
Similarly, the perpendicular bisector $\bm l_2 $ of $ AC $ is parameterized by
\[ {\bm l_2(q)}=C_m+[g_a-g_c,f_c-f_a] q \]
$ S $ coincides with the intersection of $\bm l_1 $ and $\bm l_2 $.
By requiring that $\bm l_1= \bm l_2 $, we obtain a linear system of equations with $ p $ and $ q $ as unknowns. Let $ q=q_s $ be the solution to this system, then we know that
\[ S=C_m+[g_a-g_c, f_c-f_a]q_s\]
Furthermore,
\[ \lim\limits_{h\to0}S=\lim\limits_{h\to0}\left(C_m+[g_a-g_c, f_c-g_a]\frac{h}{h}q_s\right)=A+[g'_a, -f'_a]\lim\limits_{h\to 0}h q_s \]
We will show that the limit $ \lim\limits_{h\to 0}h q_s $ exists, and we observe this: As $ {h\to0} $, $ \vv{AS} $ becomes parallel to the vector $ [g'_a, -f'_a] $. We have that $ \vec{r}'(a)=[f'_a, g'_a] $, and thus,
\[ \vv{AS}\cdot \vec{r}\,'(a)=0 \]
The line passing through the point $ \vec{r}(a) $, and having $ \vec{r}\,'(a) $ as the direction vector, is therefore a tangent line to the circle describing the curvature of $ \vec{r} $ at $ a $.

\subsubsection{Examination of the limit}
By solving the mentioned system of equations, we find that
\nn{
	q_s=\frac{1}{2}\frac{f_c(f_c-f_a)+f_b(f_a-f_c)+g_c(g_c-g_a)+g_b(g_a-g_c)}{f_b(g_c-g_a)+f_c(g_a-g_b)+f_a(g_b-g_c)}
}
Furthermore,
\[ \lim\limits_{h\to 0} q_s=\lim\limits_{h\to 0}\frac{h}{h}q_s=\lim\limits_{h\to 0}\frac{f_cf_a'-f_bf_a'+g_cg_a'-g_bg_a'}{f_bg_a'+f_cg_b'-2f_ag_a'} \]
Using the same procedure, we have that
\[\lim\limits_{h\to 0}q_s= \lim\limits_{h\to 0} \frac{(f_a')^2+(g_a')^2}{f_a'g_b'-f_b'g_a'} \]
Furthermore,
\[ \lim\limits_{h\to 0}h q_s= h\frac{(f_a')^2+(g_a')^2}{f_a'g_b'-f_b'g_a'-f_b'g_b'+f_b'g_b'}= \frac{(f_a')^2+(g_a')^2}{f_b''g_b'+f_b'g_b''} \]



\end{document}