\documentclass[english,hidelinks,pdftex, 11 pt, class=report,crop=false]{standalone}
\usepackage[T1]{fontenc}
\usepackage[utf8]{luainputenc}
\usepackage{lmodern} % load a font with all the characters
\usepackage{geometry}
\geometry{verbose,a4paper, inner=0cm, outer=0 cm, bmargin=2cm, tmargin=1cm}
%\textwidth=12cm
\setlength{\parindent}{0bp}
\usepackage{import}
\usepackage[subpreambles=false]{standalone}
\usepackage{amsmath}
\usepackage{amssymb}
\usepackage{esint}
\usepackage{babel}
\usepackage{tabu}
\usepackage[dvipsnames, table]{xcolor}
\usepackage{cancel}
\makeatother
\makeatletter
\usepackage{datetime2}
\usepackage{titlesec}
\usepackage[many]{tcolorbox}

% Eheter
\newcommand{\enh}[1]{\,\textrm{#1}}
%referances
\newcommand{\net}[2]{{\color{blue}\href{#1}{#2}}}

%Spaces
\newcommand{\vsk}{\\[12pt]}
\newcommand{\vs}{\vspace{-12pt}}

% Tabell for opplegg

\newcommand{\ovlist}[1]{
\vspace{-16pt}
\begin{itemize}
	#1
\end{itemize}
}

% Chapters and sections
\titleformat{\section}[block]{\bfseries}{\hspace{3cm}\thesection}{5pt}{}
\titleformat{\subsection}[block]{\bfseries}{\hspace{3cm}\thesection}{5pt}{}
\newcommand{\sectionbreak}{\clearpage} % New page on each section
 

\newlength{\mywidth}
\setlength{\mywidth}{14cm}

\newcommand{\cont}[1]{
\begin{tcolorbox}[center, boxrule=0.0 mm, width=\mywidth,arc=0mm,enhanced jigsaw,,colback=white,breakable]
#1	
\end{tcolorbox}
}

\newcommand{\info}[5]{
\begin{tcolorbox}[center, boxrule=0.1 mm, width=\mywidth,arc=0mm,enhanced jigsaw,breakable,colback=yellow!5]	
	
	\footnotesize
	\textbf{Øvingsområde}\\[5pt] #1 
	
	\textbf{Utstyr}\\ #2  \\
	
	\begin{tabular}{@{} p{4cm} p{4cm} l} 
		\textbf{Tid} & \textbf{Elevinndeling} & \textbf{Læringsarena} \\
		#3  & #4 & #5
	\end{tabular} 
\end{tcolorbox}	
}

\newcommand{\gjen}[1]{\begin{tcolorbox}[center,boxrule=0.1 mm, width=\mywidth,arc=0mm,colback=blue!3] {\large \textbf{Gjennomføring} \vspace{5 pt}}\newline #1  \end{tcolorbox}\vspace{-5pt}}
\newcommand{\eks}[1]{\begin{tcolorbox}[center,boxrule=0.1 mm, width=\mywidth,arc=0mm,colback=green!3] {\large \textbf{Eksempel} \vspace{5 pt}}\newline #1  \end{tcolorbox}\vspace{-5pt}}

\newcounter{opl}
%\numberwithin{opl}{article}


\newcommand{\opl}[1]{
\newpage
{\refstepcounter{opl} %\phantomsection 
\large \textbf{\theopl \;#1} \vsk}
}

% Headlines
\newcommand{\fork}{\textbf{Forkunnskapar}\\}
\newcommand{\forb}{\textbf{Forberedelsar}\\}
\newcommand{\opgvr}{\textbf{Oppgaver}}



%colors
\newcommand{\colr}[1]{{\color{red} #1}}
\newcommand{\colb}[1]{{\color{blue} #1}}
\newcommand{\colo}[1]{{\color{orange} #1}}
\newcommand{\colc}[1]{{\color{cyan} #1}}
\definecolor{projectgreen}{cmyk}{100,0,100,0}
\newcommand{\colg}[1]{{\color{projectgreen} #1}}

% Lister med bokstavar
\usepackage[inline]{enumitem}
% Opg
\newcommand{\abc}[1]{
	\begin{enumerate}[label=\alph*),leftmargin=18pt]
		#1
	\end{enumerate}
}

\usepackage[]{hyperref}

% note
\newcommand{\note}{Note}
\newcommand{\notesm}[1]{{\footnotesize \textsl{\note:} #1}}
\newcommand{\selos}{See the solutions manual.}

\newcommand{\texandasy}{The text is written in \LaTeX\ and the figures are made with the aid of Asymptote.}

\newcommand{\rknut}{Calculate.}
\newcommand\sv{\vsk \textbf{Answer} \vspace{4 pt}\\}
\newcommand{\ekstitle}{Example }
\newcommand{\sprtitle}{The language box}
\newcommand{\expl}{explanation}

% answers
\newcommand{\mulansw}{\notesm{Multiple possible answers.}}	
\newcommand{\faskap}{Chapter}

% exercises
\newcommand{\opgt}{\newpage \phantomsection \addcontentsline{toc}{section}{Exercises} \section*{Exercises for Chapter \thechapter}\vs \setcounter{section}{1}}

% references
\newcommand{\reftab}[1]{\hrs{#1}{Table}}
\newcommand{\rref}[1]{\hrs{#1}{Rule}}
\newcommand{\dref}[1]{\hrs{#1}{Definition}}
\newcommand{\refkap}[1]{\hrs{#1}{Chapter}}
\newcommand{\refsec}[1]{\hrs{#1}{Section}}
\newcommand{\refdsec}[1]{\hrs{#1}{Subsection}}
\newcommand{\refved}[1]{\hrs{#1}{Appendix}}
\newcommand{\eksref}[1]{\textsl{#1}}
\newcommand\fref[2][]{\hyperref[#2]{\textsl{Figure \ref*{#2}#1}}}
\newcommand{\refop}[1]{{\color{blue}Exercise \ref{#1}}}
\newcommand{\refops}[1]{{\color{blue}Exercise \ref{#1}}}

%%% SECTION HEADLINES %%%

% Our numbers
\newcommand{\likteikn}{The equal sign}
\newcommand{\talsifverd}{Numbers, digits and values}
\newcommand{\koordsys}{Coordinate systems}

% Calculations
\newcommand{\adi}{Addition}
\newcommand{\sub}{Subtraction}
\newcommand{\gong}{Multiplication}
\newcommand{\del}{Division}

%Factorization and order of operations
\newcommand{\fak}{Factorization}
\newcommand{\rrek}{Order of operations}

%Fractions
\newcommand{\brgrpr}{Introduction}
\newcommand{\brvu}{Values, expanding and simplifying}
\newcommand{\bradsub}{Addition and subtraction}
\newcommand{\brgngheil}{Fractions multiplied by integers}
\newcommand{\brdelheil}{Fractions divided by integers}
\newcommand{\brgngbr}{Fractions multiplied by fractions}
\newcommand{\brkans}{Cancelation of fractions}
\newcommand{\brdelmbr}{Division by fractions}
\newcommand{\Rasjtal}{Rational numbers}

%Negative numbers
\newcommand{\negintro}{Introduction}
\newcommand{\negrekn}{The elementary operations}
\newcommand{\negmeng}{Negative numbers as amounts}

%Calculation methods
\newcommand{\delmedtihundre}{Deling med 10, 100, 1\,000 osv.}

% Geometry 1
\newcommand{\omgr}{Terms}
\newcommand{\eignsk}{Attributes of triangles and quadrilaterals}
\newcommand{\omkr}{Perimeter}
\newcommand{\area}{Area}

%Algebra 
\newcommand{\algintro}{Introduction}
\newcommand{\pot}{Powers}
\newcommand{\irrasj}{Irrational numbers}

%Equations
\newcommand{\ligintro}{Introduction}
\newcommand{\liglos}{Solving with the elementary operations}
\newcommand{\ligloso}{Solving with elementary operations summarized}

%Functions
\newcommand{\fintro}{Introduction}
\newcommand{\lingraf}{Linear functions and graphs}

%Geometry 2
\newcommand{\geoform}{Formulas of area and perimeter}
\newcommand{\kongogsim}{Congruent and similar triangles}
\newcommand{\geofork}{Explanations}

% Names of rules
\newcommand{\adkom}{Addition is commutative}
\newcommand{\gangkom}{Multiplication is commutative}
\newcommand{\brdef}{Fractions as rewriting of division}
\newcommand{\brtbr}{Fractions multiplied by fractions}
\newcommand{\delmbr}{Fractions divided by fractions}
\newcommand{\gangpar}{Distributive law}
\newcommand{\gangparsam}{Paranthesis multiplied together}
\newcommand{\gangmnegto}{Multiplication by negative numbers I}
\newcommand{\gangmnegtre}{Multiplication by negative numbers II}
\newcommand{\konsttre}{Unique construction of triangles}
\newcommand{\kongtre}{Congruent triangles}
\newcommand{\topv}{Vertical angles}
\newcommand{\trisum}{The sum of angles in a triangle}
\newcommand{\firsum}{The sum of angles in a quadrilateral}
\newcommand{\potgang}{Multiplication by powers}
\newcommand{\potdivpot}{Division by powers}
\newcommand{\potanull}{The special case of \boldmath $a^0$}
\newcommand{\potneg}{Powers with negative exponents}
\newcommand{\potbr}{Fractions as base}
\newcommand{\faktgr}{Factors as base}
\newcommand{\potsomgrunn}{Powers as base}
\newcommand{\arsirk}{The area of a circle}
\newcommand{\artrap}{The area of a trapezoid}
\newcommand{\arpar}{The area of a parallelogram}
\newcommand{\pyt}{Pythagoras's theorem}
\newcommand{\forform}{Ratios in similar triangles}
\newcommand{\vilkform}{Terms of similar triangles}
\newcommand{\omkrsirk}{The perimeter of a circle (and the value of $ \bm \pi $)}
\newcommand{\artri}{The area of a triangle}
\newcommand{\arrekt}{The area of a rectangle}
\newcommand{\liknflyt}{Moving terms across the equal sign}
\newcommand{\funklin}{Linear functions}



\begin{document}

\vedlegg{Tangent Line to a Graph} \label{vdltangent}
\subsection*{Introduction}
In geometry, a \textit{tangent line to a circle} is defined as a line that intersects a circle at only one point (Moise, 1974). From this definition, it can be shown that 
\begin{itemize}
	\item a tangent line is perpendicular to the vector formed by the center of the circle and the intersection point
	\item any line that intersects a circle, where the intersection point and the center of the circle form a normal vector to the line, is a tangent line to the circle.
\end{itemize}
(See \textsl{Figure \ref{sirkoggraf}a}.) \vsk

Given a differentiable function $ f(x) $. In real analysis, the \textit{tangent line to f at the point $ (a, f(a)) $} is defined as the line that passes through $ (a, f(a)) $ and has a slope $ f'(a) $ (Spivak, 1994). (See \textsl{Figure \ref{sirkoggraf}b}.)
\begin{figure}[H]
	\centering
	\subfloat[]{\includegraphics{\figp{tang0a}}}\qquad \qquad
	\subfloat[]{\includegraphics{\figp{tang0b}}}
	\caption{\label{sirkoggraf}}
\end{figure}
For many, it is quite intuitive that tangents to circles and tangents to graphs are closely related, but the purpose of this text is to formalize this.
\subsection*{Center of Curvature}
Given a function $ f(x) $ that is continuous and twice differentiable for all $ x\in\mathbb{R} $, and where $ f''(x)\neq0 $. For a given $ a $, we let $ {f_a=f(a)} $, and define the functions
\nn{
	f_b(h)=f(a-h)\quad, \quad f_c(h)=f(a+h)
}
We also introduce the points
\nn{
	A=(a, f_a)\quad,\quad B=(a-h, f_b)\quad,\quad C=(a+h, f_c)
}
Further, let $ S=(S_x, S_y) $ be the center of the circumscribed circle of $ \triangle ABC $. Just as we find the \textit{derivative} at a point by letting the distance between two points on a graph approach 0, one can find the \outl{curvature} at a point by letting the distance between three points approach 0. In our case, the curvature is described by the circumscribed circle to $ \triangle ABC $ as $ h $ approaches 0.
\begin{figure}[H]
	\centering
	\subfloat[]{
		\includegraphics{\figp{tang1b}}
	}
	\subfloat[]{\includegraphics{\figp{tang1c}}}
	\caption{\label{krumnfig}}
\end{figure}
\subsection*{A System of Equations for $\bm S $}
We have that
\nn{
	\vv{BA}= [h,f_a-f_b] \quad,\quad \vv{AC}=[h, f_c-f_a]
}
Let $ B_m $ and $ C_m $ be the midpoints of the secants $ AB $ and $ AC $, respectively. Then,
\nn{
	B_m = B+\frac{1}{2} \vv{BA} \qquad,\qquad C_m = C+\frac{1}{2}\vv{AC} 
}
$ [f_a-f_b,-h] $ is a normal vector for $ \vv{BA} $, meaning the mid-normal $\bm l_1 $ to the secant $ AB $ can be parameterized as
\nn{
	{\bm l_1(t)}=B_m+[f_a-f_b, -h]t \label{l1}
}
Similarly, the mid-normal $\bm l_2 $ to the secant $ AC $ is parameterized by
\[ {\bm l_2(q)}=C_m+[f_c-f_a, -h]q \]
$ S $ coincides with the intersection of $\bm l_1 $ and $\bm l_2 $.
By requiring that $\bm l_1= \bm l_2 $, we obtain a linear system of equations with two unknowns that gives
\nn{
	t=\frac{(f_a-f_c)(f_b-f_c)+2h^2}{2h(f_b+f_c-2f_a)} \label{t}
}
\subsection*{$\bm S$ as $\bm {h}$ approaches 0}
We define the functions $ \dot{f}_b$, $ \dot{f}_c  $, $ \ddot{f}_b $, and $ \ddot{f}_c $ based on the (respective) derivatives and second derivatives of $ f_b $ and $ f_c $ with respect to $ h $:
\alg{
	-\dot{f}_b&=\left(f_b\right)' =-f'(a-h) \vn
	\dot{f}_c&=\left(f_c\right)'=f'(a+h) \vn
	\ddot{f}_b&=\left(f_b\right)'' =f''(a-h) \vn
	\ddot{f}_c&=\left(f_c\right)''=f''(a+h)
}
We will now use these functions to study the coordinates of $ S $ as $ h $ approaches 0. We take into account that
\alg{
	&\lim\limits_{h\to0} \left\{h^2, h\right\}=0 &&
	&\lim\limits_{h\to0} \left\{f_b, f_c\right\}=f_a \vn
	&\lim\limits_{h\to0} \left\{\dot{f}_c,\dot{f}_b\right\}=f'_a&&
	&\lim\limits_{h\to0} \left\{\ddot{f}_b, \ddot{f}_c\right\}=f_a''
}
where\footnote{Note that we are talking about $ f $ derived with respect to $ x $, and evaluated at $ a $.} $ f'_a=f'(a) $ and $ f''_a=f''(a) $. \vsk

For $ t $ expressed by \eqref{t} is (see \eqref{l1})
\nn{
	S_y=\frac{f_a+f_b+2ht}{2} = \frac{f_a+f_b}{2}+ht \label{sy}
}
We have that
\nn{
	\lim\limits_{h\to 0} \frac{f_a+f_b}{2}=f_a
}
Further,
\alg{
	ht&=\frac{(f_c-f_a)(f_b-f_c)+2h^2}{2(f_b+f_c-2f_a)} \br
	&=\frac{(f_c-f_a)(f_b-f_c)}{2(f_b+f_c-2f_a)}+\frac{h^2}{f_b+f_c-2f_a} \label{ht}
}
As $ h $ approaches 0, both terms in \eqref{ht} are <<0 over 0>> expressions. We use L'Hopital's rule on the last term: 
\begin{align}
	&&\lim\limits_{h\to0}\frac{h^2}{2(f_b+f_c-2f_a)}&=\lim\limits_{h\to0}\frac{\left(h^2\right)'}{(f_b+f_c-2f_a)'} \br
	&&&=\lim\limits_{h\to0} \frac{2h}{-\dot{f}_b+\dot{f}_c} &&\text{<<0 over 0>>}\br
	&&&=\lim\limits_{h\to0} \frac{2}{\ddot{f}_b+\ddot{f}_c}  \\
	&& &=\frac{1}{f_a''} \label{ht2}
\end{align}

Using L'Hopital's rule on the first term in \eqref{ht} we have that
\alg{
	\lim\limits_{h\to 0}\frac{(f_c-f_a)(f_b-f_c)}{f_b+f_c-2f_a} &= \lim\limits_{h\to 0}\frac{\left((f_c-f_a)(f_b-f_c)\right)'}{(f_b+f_c-2f_a)'} 
}
By the product rule in differentiation,
\alg{
	\lim\limits_{h\to 0}\frac{\left((f_a-f_c)(f_b-f_c)\right)'}{(f_b+f_c-2f_a)'} &=\lim\limits_{h\to 0} \left[\frac{\dot{f_c}(f_b-f_c)}{-\dot{f}_b+\dot{f}_c}+\frac{(f_c-f_a)(\dot{f}_b+\dot{f}_c)}{-\dot{f}_b+\dot{f}_c}\right]
}
Both terms above are <<0 over 0>> expressions. We investigate them separately by applying L'Hopital's rule:
\alg{
	\lim\limits_{h\to 0}\frac{\dot{f_c}(f_b-f_c)}{-\dot{f}_b+\dot{f}_c} &= \lim\limits_{h\to 0} \left[\frac{\ddot{f}_c(f_b-f_c)}{\ddot{f}_b+\ddot{f}_c}+\frac{\dot{f}_c(\dot{f}_b+\dot{f}_c)}{\ddot{f}_b+\ddot{f}_c}\right] \br
	&= 0+\frac{(f_a')^2}{2f_a''} \label{ht1a}
}
\begin{align}
	\lim\limits_{h\to 0}\frac{(f_c-f_a)(\dot{f}_b+\dot{f}_c)}{-\dot{f}_b+\dot{f}_c}&= \lim\limits_{h\to 0}\left[\frac{\dot{f}_c(\dot{f}_b+\dot{f}_c)}{\ddot{f}_b+\ddot{f}_c}+\frac{(f_c-f_a)(-\dot{f}_b+\dot{f}_c)}{\ddot{f}_b+\ddot{f}_c}\right]\br
	&=\frac{(f_a')^2}{2f_a''}+0 \label{ht1b}
\end{align}
By \eqref{ht}, \eqref{ht2}, \eqref{ht1a} and \eqref{ht1b} we have that
\[ \lim\limits_{h\to0} h t=\frac{1+(f_a')^2}{f_a''} \]
Thus,
\nn{
	S_y=f_a+\frac{1+(f_a')^2}{f''_a}
}

Furthermore, with $ t $ given by \eqref{t}
\[ S_x=(f_b-f_a)t+a-\frac{1}{2}h \]
We have that
\algv{
	\lim\limits_{h\to0}(f_b-f_a)t&=\lim\limits_{h\to0} \frac{f_b-f_a}{h}\cdot ht \br
	&= \lim\limits_{h\to0} \frac{f_b-f_a}{h}\cdot \lim\limits_{h\to0} ht  \br
	&=-f_a'\frac{1+(f_a')^2}{f''_a}
}
Thus,
\nn{S_x=a-f'_a\frac{1+(f_a')^2}{f''_a}}
\subsection*{Conclusion}
The line that has a slope $ f'(a) $, and that passes through $ (a, f(a)) $, is given by the function 
\[ g(x)=f_a'(x-a)+f_a \]
$ \vec{r}=[1, f_a] $ is the direction vector of this line. 
From the expressions we have found for $ S_x $ and $ S_y $ we have that
\[ S=\left(a-f'_a\frac{1+(f_a')^2}{f''_a}, f_a+\frac{1+(f_a')^2}{f''_a}\right) \]
Thus,
\[ \vv{AS}= \frac{1}{f_a''}\left[-f_a(1+(f_a')^2), 1+(f_a')^2\right] \]
Since $ {\vec{r}\cdot \vv{AS}=0 }$ and $ {g(a)=f(a)} $, the graph of $ g $ is the tangent line to the circle with center $ S $ as $ h $ approaches 0. Thus, $ g $ is the tangent line to the circle that describes the curvature of $ f $ when $ x=a $.


\end{document}