\documentclass[english,hidelinks,pdftex, 11 pt, class=report,crop=false]{standalone}
\usepackage[T1]{fontenc}
\usepackage[utf8]{luainputenc}
\usepackage{lmodern} % load a font with all the characters
\usepackage{geometry}
\geometry{verbose,a4paper, inner=0cm, outer=0 cm, bmargin=2cm, tmargin=1cm}
%\textwidth=12cm
\setlength{\parindent}{0bp}
\usepackage{import}
\usepackage[subpreambles=false]{standalone}
\usepackage{amsmath}
\usepackage{amssymb}
\usepackage{esint}
\usepackage{babel}
\usepackage{tabu}
\usepackage[dvipsnames, table]{xcolor}
\usepackage{cancel}
\makeatother
\makeatletter
\usepackage{datetime2}
\usepackage{titlesec}
\usepackage[many]{tcolorbox}

% Eheter
\newcommand{\enh}[1]{\,\textrm{#1}}
%referances
\newcommand{\net}[2]{{\color{blue}\href{#1}{#2}}}

%Spaces
\newcommand{\vsk}{\\[12pt]}
\newcommand{\vs}{\vspace{-12pt}}

% Tabell for opplegg

\newcommand{\ovlist}[1]{
\vspace{-16pt}
\begin{itemize}
	#1
\end{itemize}
}

% Chapters and sections
\titleformat{\section}[block]{\bfseries}{\hspace{3cm}\thesection}{5pt}{}
\titleformat{\subsection}[block]{\bfseries}{\hspace{3cm}\thesection}{5pt}{}
\newcommand{\sectionbreak}{\clearpage} % New page on each section
 

\newlength{\mywidth}
\setlength{\mywidth}{14cm}

\newcommand{\cont}[1]{
\begin{tcolorbox}[center, boxrule=0.0 mm, width=\mywidth,arc=0mm,enhanced jigsaw,,colback=white,breakable]
#1	
\end{tcolorbox}
}

\newcommand{\info}[5]{
\begin{tcolorbox}[center, boxrule=0.1 mm, width=\mywidth,arc=0mm,enhanced jigsaw,breakable,colback=yellow!5]	
	
	\footnotesize
	\textbf{Øvingsområde}\\[5pt] #1 
	
	\textbf{Utstyr}\\ #2  \\
	
	\begin{tabular}{@{} p{4cm} p{4cm} l} 
		\textbf{Tid} & \textbf{Elevinndeling} & \textbf{Læringsarena} \\
		#3  & #4 & #5
	\end{tabular} 
\end{tcolorbox}	
}

\newcommand{\gjen}[1]{\begin{tcolorbox}[center,boxrule=0.1 mm, width=\mywidth,arc=0mm,colback=blue!3] {\large \textbf{Gjennomføring} \vspace{5 pt}}\newline #1  \end{tcolorbox}\vspace{-5pt}}
\newcommand{\eks}[1]{\begin{tcolorbox}[center,boxrule=0.1 mm, width=\mywidth,arc=0mm,colback=green!3] {\large \textbf{Eksempel} \vspace{5 pt}}\newline #1  \end{tcolorbox}\vspace{-5pt}}

\newcounter{opl}
%\numberwithin{opl}{article}


\newcommand{\opl}[1]{
\newpage
{\refstepcounter{opl} %\phantomsection 
\large \textbf{\theopl \;#1} \vsk}
}

% Headlines
\newcommand{\fork}{\textbf{Forkunnskapar}\\}
\newcommand{\forb}{\textbf{Forberedelsar}\\}
\newcommand{\opgvr}{\textbf{Oppgaver}}



%colors
\newcommand{\colr}[1]{{\color{red} #1}}
\newcommand{\colb}[1]{{\color{blue} #1}}
\newcommand{\colo}[1]{{\color{orange} #1}}
\newcommand{\colc}[1]{{\color{cyan} #1}}
\definecolor{projectgreen}{cmyk}{100,0,100,0}
\newcommand{\colg}[1]{{\color{projectgreen} #1}}

% Lister med bokstavar
\usepackage[inline]{enumitem}
% Opg
\newcommand{\abc}[1]{
	\begin{enumerate}[label=\alph*),leftmargin=18pt]
		#1
	\end{enumerate}
}

\usepackage[]{hyperref}

\newcommand{\note}{Merk}
\newcommand{\notesm}[1]{{\footnotesize \textsl{\note:} #1}}
\newcommand{\ekstitle}{Eksempel }
\newcommand{\sprtitle}{Språkboksen}
\newcommand{\expl}{forklaring}
\newcommand{\pyt}{Pytagoras' setning}
\newcommand\sv{\vsk \textbf{Svar} \vspace{4 pt}\\}

%references
\newcommand{\reftab}[1]{\hrs{#1}{tabell}}
\newcommand{\rref}[1]{\hrs{#1}{regel}}
\newcommand{\dref}[1]{\hrs{#1}{definisjon}}
\newcommand{\refkap}[1]{\hrs{#1}{kapittel}}
\newcommand{\refsec}[1]{\hrs{#1}{seksjon}}
\newcommand{\refdsec}[1]{\hrs{#1}{delseksjon}}
\newcommand{\refved}[1]{\hrs{#1}{vedlegg}}
\newcommand{\eksref}[1]{\textsl{#1}}
\newcommand\fref[2][]{\hyperref[#2]{\textsl{figur \ref*{#2}#1}}}
\newcommand{\refop}[1]{{\color{blue}Oppgave \ref{#1}}}
\newcommand{\refops}[1]{{\color{blue}oppgave \ref{#1}}}


%Algebra
\newcommand{\kvadset}{Kvadratsetningene}
\newcommand{\aenato}{Sum-produkt-metoden}

% Geometry
\newcommand{\hlikb}{Midtnormalen i en likebeint trekant}
\newcommand{\arealsetn}{Arealsetningen}
\newcommand{\trkmedian}{Median}
\newcommand{\midtrk}{Midtnormal (i trekant)}
\newcommand{\innskrsirk}{Innskrevet sirkel}
\newcommand{\cossetn}{Cosinussetningen}
\newcommand{\perfvink}{Sentral- og periferivinkel}
\newcommand{\tang}{Tangent}

% Derivative
\newcommand{\derel}{Den deriverte av elementære funksjoner}
\newcommand{\divder}{Divisjonsregelen}
\newcommand{\kjernereg}{Kjerneregelen}
\newcommand{\prodregder}{Produktregelen}
\newcommand{\lhop}{L'Hopitals regel}

% Funksjonsdrofting
\newcommand{\monder}{Monotoniegenskaper og den deriverte}
\newcommand{\fderekstr}{$ \bm{f'=0} $ for lokale ektstremalpunkt}
\newcommand{\andredertest}{Andrederiverttesten}

% Vectors
\newcommand{\detar}{Arealformler med determinanter}
\newcommand{\avstpunktlin}{Avstand mellom punkt og linje}

%Appendix
\newcommand{\rolle}{Rolles teorem}
\newcommand{\meanval}{Middelverdisetningen}

% Solutions manual
\newcommand{\selos}{Se løsningsforslag.}

\begin{document}
\section{Mengder}
\subsection{Definisjon}
En samling av tall kalles en \outl{mengde}\footnote{En mengde kan også være en samling av andre matematiske objekter, som for eksempel funksjoner, men i denne boka holder det å se på mengder av tall.
}, og et tall som er en del av en mengde kalles et \outl{element}. Mengder kan inneholde et endelig\\ antall elementer og de kan inneholde uendelig mange elementer. \regv
\regdef[Mengder]{
For to tall $ a $ og $ b $, hvor $ a\leq b $, har vi at
\begin{center}
	\begin{tabular}{c l}
		$ [a, b] $ & er mengden av alle tall større eller lik $ a $ \\
		&og mindre eller lik $ b $. \\
		$ (a, b] $ & er mengden av alle tall større enn $ a $ \\
		&og mindre eller lik $ b $.\\ 
		$ [a, b) $ & er mengden av alle reelle tall større eller lik $ a $ \\
		&og mindre enn $ b $.
	\end{tabular}
\end{center}
$ [a, b] $ kalles et \outl{lukket intervall}, $ (a, b) $ kalles et \outl{åpent \\intervall}, og både $ (a, b] $ og $ [a, b) $ kalles \outl{halvåpne intervall}.\vsk

Mengden som inneholder bare $ a $ og $ b $ skrives som $ \{a, b\} $. \vsk

At $ x $ er et element i en mengde $ M $ skrives som $ x\in M $.\vsk

At $ x $ \textsl{ikke} er et element i en mengde $ M $ skrives som $ x\not \in M $. \vsk

At mengden $ M $ består av mengdene $ M_1 $ og $ M_2 $ skrives som \\$ M = M_1 \cup M_2 $. \vsk

At $ x $ er utelatt av en mengde $ M $ skrives som $ M \setminus x $
}
\spr{
$ x \in M$ uttales ''$ x $ inneholdt i $ M $''.\vsk

Mange tekster bruker \sym{$ \langle $} istedenfor \sym{$ ( $} for å indikere åpne (eller halvåpne) intervall.
}
\info{\note}{
Når vi heretter i boka definerer et intervall beskrevet av $ a $ og $ b $, tar vi det for gitt at $ a $ og $ b $ er to tall, og at $ a\leq b $.
}

\eks[1]{
Mengden av alle heltall større enn 0 og mindre enn 10 kan vi skrive som
\[ \{1, 2, 3, 4, 5, 6, 7, 8, 9\} \]
Denne mengden inneholder 9 elementer. 3 er et element i denne mengden, og da kan vi skrive $ 3\in\{1, 2, 3, 4, 5, 6, 7, 8, 9\}  $\vsk 

10 er ikke et element i denne mengden, og da kan vi skrive \\$ 10 \not\in  \{1, 2, 3, 4, 5, 6, 7, 8, 9\} $.
} 
\eks[2]{
I uttrykket $ 0\ ?\  x\  ?\  1 $, erstatt \sym{?} med et ulikhetssymbol slik at uttrykket gjelder for alle $ {x\in M} $, og avgjør om 1 er inneholdt i $ M $.
\abc{
\item $ M = [0, 1] $
\item $ M = (0, 1] $
\item $ M = [0, 1) $
}
\sv \vs
\abc{
\item $ 0\leq x \leq 1 $. Videre er $ 1\in M $.
\item $ 0< x \leq 1 $. Videre er $ 1\in M $.
\item $ 0\leq x < 1 $. Videre er $ 1\not \in M $.
}
}\vsk

\regdef[Navn på mengder \label{mengder}]{\vs
\begin{center}
	\begin{tabular}{c l}
		$ \mathbb{N} $ & Mengden av alle positive heltall\footnote{Inneholder \textsl{ikke} 0.}\os
		$ \mathbb{Z} $ & Mengden av alle heltall\footnote{Inneholder 0.}\os
		$ \mathbb{Q} $ & Mengden av alle rasjonale tall\os
		$ \mathbb{R} $ & Mengden av alle reelle tall\os
		$ \mathbb{C} $ & Mengden av alle komplekse tall\\
	\end{tabular}
\end{center}
}
\subsection{Symbolet for uendelig}
Mengdene i \dref{mengder} inneholder uendelig mange elementer. Noen ganger ønsker vi å avgrense deler av en uendelig mengde, og da melder det seg et behov for et symbol som  er med på å symbolisere dette. \sym{$ \infty $} er symbolet for en uendelig stor, positiv verdi.\regv 

\eks[]{
Et vilkår om at $ x\geq2 $ kan vi skrive som $ x\in[2 ,\, \infty) $.\vsk

Et vilkår om at $ x<-7 $ kan vi skrive som $ x\in(-\infty ,-7) $.\vsk
}
\spr{
De to intervallene i eksempelet over kan også skrives som $ [2, \rightarrow) $ og $ (\leftarrow, -7) $.
}
\info{\note}{
\sym{$ \infty $} er ikke noe bestemt tall. Å bruke de fire grunnleggende regneartene alene med dette symbolet gir derfor ingen mening.
}
\subsection{Verdi- og definisjonsmengder}
\regdef[Verdi- og definisjonsmengder]{
Gitt en funksjon $ f(x) $. 
\begin{itemize}
	\item Mengden som utelukkende inneholder alle verdier $ x $ kan ha, er \outl{definisjonsmengden} til $ f $. Denne mengden skrives som $ D_f $.
	\item Mengden som utelukkende inneholder alle verdier $ f $ kan ha når $ {x\in D_f} $, er \outl{verdimengden} til $ f $. Denne mengden skrives som $ V_f $. 
\end{itemize}
}
\eks[1]
{ \label{mengeks1}
Figuren under viser $ {f(x)=2x+1} $, hvor $ {D_f=[1, 3]} $. Da er ${ V_f=[1, 5]} $. \vs
\fig{defmengeks1}
}
\eks[2]{
Figuren under viser $ f(x)={\frac{1}{x}} $, hvor $ {D_f}=[-3, -1] \cup [2, 5] $. Da er $ {V_f=\left[-1, -\frac{1}{3}\right]\cup \left[\frac{1}{5},\frac{1}{2}\right]} $. \vs
\fig{defmengeks2}
}
\newpage
\info{\note}{
Definisjonsmengden til en funksjon bestemmes av to ting; hvilken sammenheng funksjonen skal brukes i, og eventuelle verdier som gir et udefinert funksjonsuttrykk. I \textsl{Eksempel 1} på side \pageref{mengeks1} er definisjonsmengden helt vilkårlig valgt, siden funksjonen er definert for alle $ x $. I \textsl{Eksempel 2} derimot er ikke funksjonen definert for $ {x=0} $, så en definisjonsmengde som inneholdt denne verdien for $ x $ ville ikke gitt mening.
}
\section{Vilkår}
\subsection{Symboler for vilkår}
Symbolet \sym{$ \Rightarrow $} bruker vi for å vise til at hvis et vilkår er oppfylt, så er et annet (eller flere) vilkår også oppfylt. For eksempel; i \mb\;så vi at hvis en trekant er rettvinklet, er Pytagoras' setning gyldig. Dette kan vi skrive slik:
\[ \text{trekanten er rettvinklet} \Rightarrow \text{Pytagoras' setning er gyldig}\]
Men vi så også at det omvendte gjelder; hvis Pytagoras' setning er gyldig, må trekanten være rettvinklet. Da kan vi skrive
\[ \text{trekanten er rettvinklet} \iff \text{Pytagoras' setning er gyldig}\]
Det er veldig viktig å være bevisst forskjellen på \sym{$ \Rightarrow $} og \sym{$ \iff $}; at vilkår A oppfylt gir B oppfylt, trenger ikke å bety at vilkår B oppfylt gir vilkår A oppfylt! \regv

\eks[1]{ \vs \vs
\[ \text{firkanten er et kvadrat}\Rightarrow\text{firkanten har fire like lange sider} \]
}
\eks[2]{ \vs 
	\[ \text{tallet er et primtall større enn 2}\Rightarrow\text{tallet er et oddetall} \]
}
\eks[3]{\vs
\[ \text{tallet er et partall}\iff\text{tallet er delelig med 2} \]
}
\newpage
\subsection{Funksjoner med vilkår}
Funksjoner kan gjerne ha flere uttrykk som gjelder ved forskjellige vilkår. La oss for eksempel definere en funksjon $ f(x) $ slik:
\alg{
&\text{For }x<1 \text{ er funksjonsuttrykket } {-2x+1}\\
&\text{For }x\geq 1\text{ er funksjonsuttrykket } x^2-2x
}
\figc{funkvilk}{Grafen til $ f $ på intervallet $ [-1, 3] $.}
Dette kan vi skrive som
\begin{equation*}f(x)= \left\lbrace{
		\begin{array}{rcr}
			-2x+1 &,&x<1 \br
			x^2-2x   &,& x\geq 1
		\end{array}
	}\right. \label{fsplit}
\end{equation*}

\end{document}