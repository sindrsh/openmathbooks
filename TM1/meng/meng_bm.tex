\documentclass[english,hidelinks,pdftex, 11 pt, class=report,crop=false]{standalone}
\usepackage[T1]{fontenc}
\usepackage[utf8]{luainputenc}
\usepackage{lmodern} % load a font with all the characters
\usepackage{geometry}
\geometry{verbose,a4paper, inner=0cm, outer=0 cm, bmargin=2cm, tmargin=1cm}
%\textwidth=12cm
\setlength{\parindent}{0bp}
\usepackage{import}
\usepackage[subpreambles=false]{standalone}
\usepackage{amsmath}
\usepackage{amssymb}
\usepackage{esint}
\usepackage{babel}
\usepackage{tabu}
\usepackage[dvipsnames, table]{xcolor}
\usepackage{cancel}
\makeatother
\makeatletter
\usepackage{datetime2}
\usepackage{titlesec}
\usepackage[many]{tcolorbox}

% Eheter
\newcommand{\enh}[1]{\,\textrm{#1}}
%referances
\newcommand{\net}[2]{{\color{blue}\href{#1}{#2}}}

%Spaces
\newcommand{\vsk}{\\[12pt]}
\newcommand{\vs}{\vspace{-12pt}}

% Tabell for opplegg

\newcommand{\ovlist}[1]{
\vspace{-16pt}
\begin{itemize}
	#1
\end{itemize}
}

% Chapters and sections
\titleformat{\section}[block]{\bfseries}{\hspace{3cm}\thesection}{5pt}{}
\titleformat{\subsection}[block]{\bfseries}{\hspace{3cm}\thesection}{5pt}{}
\newcommand{\sectionbreak}{\clearpage} % New page on each section
 

\newlength{\mywidth}
\setlength{\mywidth}{14cm}

\newcommand{\cont}[1]{
\begin{tcolorbox}[center, boxrule=0.0 mm, width=\mywidth,arc=0mm,enhanced jigsaw,,colback=white,breakable]
#1	
\end{tcolorbox}
}

\newcommand{\info}[5]{
\begin{tcolorbox}[center, boxrule=0.1 mm, width=\mywidth,arc=0mm,enhanced jigsaw,breakable,colback=yellow!5]	
	
	\footnotesize
	\textbf{Øvingsområde}\\[5pt] #1 
	
	\textbf{Utstyr}\\ #2  \\
	
	\begin{tabular}{@{} p{4cm} p{4cm} l} 
		\textbf{Tid} & \textbf{Elevinndeling} & \textbf{Læringsarena} \\
		#3  & #4 & #5
	\end{tabular} 
\end{tcolorbox}	
}

\newcommand{\gjen}[1]{\begin{tcolorbox}[center,boxrule=0.1 mm, width=\mywidth,arc=0mm,colback=blue!3] {\large \textbf{Gjennomføring} \vspace{5 pt}}\newline #1  \end{tcolorbox}\vspace{-5pt}}
\newcommand{\eks}[1]{\begin{tcolorbox}[center,boxrule=0.1 mm, width=\mywidth,arc=0mm,colback=green!3] {\large \textbf{Eksempel} \vspace{5 pt}}\newline #1  \end{tcolorbox}\vspace{-5pt}}

\newcounter{opl}
%\numberwithin{opl}{article}


\newcommand{\opl}[1]{
\newpage
{\refstepcounter{opl} %\phantomsection 
\large \textbf{\theopl \;#1} \vsk}
}

% Headlines
\newcommand{\fork}{\textbf{Forkunnskapar}\\}
\newcommand{\forb}{\textbf{Forberedelsar}\\}
\newcommand{\opgvr}{\textbf{Oppgaver}}



%colors
\newcommand{\colr}[1]{{\color{red} #1}}
\newcommand{\colb}[1]{{\color{blue} #1}}
\newcommand{\colo}[1]{{\color{orange} #1}}
\newcommand{\colc}[1]{{\color{cyan} #1}}
\definecolor{projectgreen}{cmyk}{100,0,100,0}
\newcommand{\colg}[1]{{\color{projectgreen} #1}}

% Lister med bokstavar
\usepackage[inline]{enumitem}
% Opg
\newcommand{\abc}[1]{
	\begin{enumerate}[label=\alph*),leftmargin=18pt]
		#1
	\end{enumerate}
}

\usepackage[]{hyperref}

\newcommand{\note}{Merk}

% Geometry
\newcommand{\hlikb}{Midtnormalen i en likebeint trekant}
\newcommand{\arealsetn}{Arealsetningen}
\newcommand{\trkmedian}{Medianer i trekanter}
\newcommand{\midtrk}{Midtnormaler i trekanter}
\newcommand{\innskrsirk}{Halveringslinjer og innskrevet sirkel i trekanter}
\newcommand{\cossetn}{Cosinussetningen}
\newcommand{\perfvink}{Sentral- og periferivinkel}
\newcommand{\tang}{Tangent}


\begin{document}
\section{Mengder}
En samling av tall kalles en \textit{mengde}\footnote{En mengde kan også være en samling av andre matematiske objekter, som for eksempel funksjoner, men i denne boka holder det å se på mengder av tall.
}, og et tall som er en del av en mengde kalles et \textit{element} i denne mengden. Mengder kan inneholde et endelig antall elementer og de kan inneholde uendelig mange elementer. \regv
\reg[Mengder]{
For to reelle tall $ a $ og $ b $, hvor $ a<b $, har vi at
\begin{center}
	\begin{tabular}{c l}
		$ [a, b] $ & er mengden av alle reelle tall større eller lik $ a $ \\
		&og mindre eller lik $ b $. \\
		$ (a, b] $ & er mengden av alle reelle tall større enn $ a $ \\
		&og mindre eller lik $ b $.\\ 
		$ [a, b) $ & er mengden av alle reelle tall større eller lik $ a $ \\
		&og mindre enn $ b $.
	\end{tabular}
\end{center}
$ [a, b] $ kalles et lukket intervall, mens både $ (a, b] $ og $ [a, b) $ kalles halvåpne intervall.\vsk

Mengden av tre tall $ a $, $ b $ og $ c $ skrives som $ \{a, b, c\} $.\vsk

At $ x $ er et element i en mengde $ M $ skrives som $ x\in M $.\vsk

At $ x $ ikke er et element i en mengde $ M $ skrives som $ x\not \in M $. 
}
\spr{
$ x \in M$ uttales ''$ x $ inneholdt i $ M $'' eller ''$ x $ er et element i $ M $''.
}
\eks[1]{
Mengden av alle heltall større enn 0 og mindre enn 10 skriver vi som 
\[ \{1, 2, 3, 4, 5, 6, 7, 8, 9\} \]
Denne mengden inneholder 9 elementer. 3 er et element i denne mengden, og da kan vi skrive $ 3\in\{1, 2, 3, 4, 5, 6, 7, 8, 9\}  $\vsk 

10 er ikke et element i denne mengden, og da kan vi skrive \\$ 10 \not\in  \{1, 2, 3, 4, 5, 6, 7, 8, 9\} $.
} 
\eks[2]{
Skriv opp ulikhetene som gjelder for alle $ {x\in M} $, og om 1 er inneholdt i $ M $.
\abc{
\item $ M = [0, 1] $
\item $ M = (0, 1] $
\item $ M = [0, 1) $
}
\sv \vs
\abc{
\item $ 0\leq x \leq 1 $. Videre er $ 1\in M $.
\item $ 0< x \leq 1 $. Videre er $ 1\in M $.
\item $ 0\leq x < 1 $. Videre er $ 1\not \in M $.
}
}\vsk

\reg[Navn på mengder]{\vs
\begin{center}
	\begin{tabular}{c l}
		$ \mathbb{N} $ & Mengden av alle positive heltall\footnote{Inneholder \textit{ikke} 0.}\\
		$ \mathbb{Z} $ & Mengden av alle heltall\footnote{Inneholder 0.}\\
		$ \mathbb{Q} $ & Mengden av alle rasjonale tall\\
		$ \mathbb{R} $ & Mengden av alle reelle tall\\
		$ \mathbb{C} $ & Mengden av alle komplekse tall\\
	\end{tabular}
\end{center}
}
\section{Verdi- og definisjonsmengder}
Alle funksjoner har en definisjonsmengde og en verdimengde. For en funksjon $ f(x) $, er definisjonsmengden den mengden som utelukkende inneholder alle verdier $ x $ kan ha. Denne mengden skrives da som $ D_f $. Hvilke verdier $ x $ kan ha er bestemt av to ting:
\begin{itemize}
	\item Hvilken sammenheng $ x $ skal brukes i.
	\item Om $ f $ ikke er definert for visse $ x $-verdier.
\end{itemize}

La oss først bruke $ f(x)=2x+1 $ som et eksempel. Denne funksjonene er definert for alle $ x\in \mathbb{R} $. Vi kunne derfor latt $ \mathbb{R} $ være definisjonsmengden til $ f $, men for enhelhets skyld velger vi her $ D_f=[0, 1] $. Mengden som utelukkende inneholder alle verdier $ f $ kan ha når $ x\in D_f $, er verdimengden til $ f $. Denne mengden skrives som $ V_f $. I dette tilfellet er (forklar for deg selv hvorfor) $ f\in [1, 3] $, altså er $ V_f=[1, 3] $.\vsk

La oss videre se på funksjonen $ g(x)=\frac{1}{x} $. Denne funksjonen er ikke definert for $ x=0 $, noe som betyr at vi allerede har fått en restriksjon på definisjonsmengden til $ g $. Også her gjør vi det enkelt, og unngår\footnote{I \refsec{??} skal vi se nærmere på funksjoner som $ g $ når $ x $ nærmer seg 0.} 0 med god klaring ved å sette $ D_g=[1, 2] $. Da er (forklar for deg selv hvorfor) $ V_g=\left[\frac{1}{2}, 1\right] $.\regv
 
\reg[Verdi- og definisjonsmengder]{
Gitt en funksjon $ f(x) $. Mengden som utelukkende inneholder alle verdier $ x $ kan ha, er da definisjonsmengden til $ f $. Denne mengden skrives som $ D_f $.\vsk

Mengden som utelukkende inneholder alle verdier $ f $ kan ha når $ x\in D_f $, er verdimengden til $ f $. 
}

\end{document}