\newcommand{\cosf}{\rg[Cosinusfunksjonen]{
		En funksjon $ f(x) $ på formen
		\[f(x)= a \cos (kx+c) + d \]
		kalles en cosinusfunksjon med amplitude $ |a| $, bølgetall $ k $, fase $ c $ og likevektslinje $ d $.
		\subimport{fig/}{fcos2} 
		$ k $ er gitt ved relasjonen:
		\[ k = \frac{2\pi}{P} \]
		hvor $ P $ er perioden til $ f $. Videre kan $ c $ finnes ut ifra ligningen:
		\[ k x_1+ c = 0 \]
		hvor $ x_1 $ er $ x $-verdien til et toppunkt. \\
		
		Funksjonen har ekstremalpunkter der hvor: 
		\[ kx +c = 2\pi n \quad \vee\quad kx+c = \pi + 2\pi n \]
		for alle $ n\in \mathbb{Z} $.
	}}
\newcommand{\sinf}{\rg[Sinusfunksjonen]{
		En funksjon $ f(x) $ på formen
		\[f(x)= a \sin (kx+c) + d \]
		kalles en sinusfunksjon med amplitude $ |a| $, bølgetall $ b $, fase $ c $ og likevektslinje $ d $. \\
		
		$ k $ er gitt ved relasjonen:
		\[ k = \frac{2\pi}{P} \]
		hvor $ P $ er perioden til $ f $. Videre kan $ c $ finnes ut ifra ligningen:
		\[ k x_1+ c = \frac{\pi}{2} \]
		hvor $ x_1 $ er $ x $-verdien til et toppunkt. \\
		
		Funksjonen har ekstremalpunkter der hvor: 
		\[ kx +c = \pm \frac{\pi}{2}+2\pi n \]
		for alle $ n\in \mathbb{Z} $.
	}}
\newcommand{\rel}{\rg[Relasjonene mellom sinus- og cosinusfunksjoner]{
		\vspace{-11 pt}
		\begin{align}
		&\cos \left(kx+c -\frac{\pi}{2}\right)= \sin(kx+c)	\\[5 pt]
		& \sin\left(kx +c + \frac{\pi}{2}\right)= \cos(kx+c)
		\end{align}
	}}
\newcommand{\rele}{\eks{
		Skriv om funksjonen $ f(x)= 2 \cos (3x+\pi)+1 $ til en sinusfunksjon. \\
		
		\sv
		Det eneste vi må sørge for er å gjøre om cosinusuttrykket til et sinusuttrykk. Og vi vet at:
		\alg{
			\cos(3x+\pi)&= \cos\left(3x+\pi+\frac{\pi}{2}-\frac{\pi}{2}\right)\\
			&= \sin\left(3x+\pi+\frac{\pi}{2}\right)	\\
			&= \sin\left(3x+\frac{3\pi}{2}\right)
		}
		Dermed får vi:
		\[ f(x) = 2\sin\left(3x+\frac{3\pi}{2}\right)+1\]	
	}}
\newcommand{\komb}{\rg[Sinus og cosinus kombinert \label{rsin}]{
		Vi kan skrive:
		\algv{a \cos kx +b \sin kx= r\sin(kx+c) }
		der $ r=\sqrt{a^2+b^2} $ og hvor:
		\algv{
			\cos c &= \frac{b}{r}\\[5 pt]
			\sin c &= \frac{a}{r}
		}}}
\newcommand{\kombe}{	
	\eks{
		Skriv om $ \sqrt{3} \sin(\pi x)-\cos(\pi x) $ til et sinusuttrykk. \\
		
		\sv
		Vi starter med å finne $ r $:
		\begin{align*}
		R&=\sqrt{\sqrt{3}^{\,2}+(-1)^2}  \\
		&= \sqrt{4} \\
		&= 2
		\end{align*}
		Videre krever vi at:
		\algv{
			\cos c &= \frac{\sqrt{3}}{2}\\[5 pt]
			\sin c &= -\frac{1}{2}}
		Tallet $ c=-\frac{\pi}{6} $ oppfyller dette kravet, og dermed har vi funnet at: 
		\[ \sqrt{3} \sin(\pi x)-\cos(\pi x) = 2 \sin\left(\pi x-\frac{\pi}{6}\right) \]			
		}}