\documentclass[english,hidelinks,pdftex, 11 pt, class=report,crop=false]{standalone}
\usepackage[T1]{fontenc}
\usepackage[utf8]{luainputenc}
\usepackage{lmodern} % load a font with all the characters
\usepackage{geometry}
\geometry{verbose,a4paper, inner=0cm, outer=0 cm, bmargin=2cm, tmargin=1cm}
%\textwidth=12cm
\setlength{\parindent}{0bp}
\usepackage{import}
\usepackage[subpreambles=false]{standalone}
\usepackage{amsmath}
\usepackage{amssymb}
\usepackage{esint}
\usepackage{babel}
\usepackage{tabu}
\usepackage[dvipsnames, table]{xcolor}
\usepackage{cancel}
\makeatother
\makeatletter
\usepackage{datetime2}
\usepackage{titlesec}
\usepackage[many]{tcolorbox}

% Eheter
\newcommand{\enh}[1]{\,\textrm{#1}}
%referances
\newcommand{\net}[2]{{\color{blue}\href{#1}{#2}}}

%Spaces
\newcommand{\vsk}{\\[12pt]}
\newcommand{\vs}{\vspace{-12pt}}

% Tabell for opplegg

\newcommand{\ovlist}[1]{
\vspace{-16pt}
\begin{itemize}
	#1
\end{itemize}
}

% Chapters and sections
\titleformat{\section}[block]{\bfseries}{\hspace{3cm}\thesection}{5pt}{}
\titleformat{\subsection}[block]{\bfseries}{\hspace{3cm}\thesection}{5pt}{}
\newcommand{\sectionbreak}{\clearpage} % New page on each section
 

\newlength{\mywidth}
\setlength{\mywidth}{14cm}

\newcommand{\cont}[1]{
\begin{tcolorbox}[center, boxrule=0.0 mm, width=\mywidth,arc=0mm,enhanced jigsaw,,colback=white,breakable]
#1	
\end{tcolorbox}
}

\newcommand{\info}[5]{
\begin{tcolorbox}[center, boxrule=0.1 mm, width=\mywidth,arc=0mm,enhanced jigsaw,breakable,colback=yellow!5]	
	
	\footnotesize
	\textbf{Øvingsområde}\\[5pt] #1 
	
	\textbf{Utstyr}\\ #2  \\
	
	\begin{tabular}{@{} p{4cm} p{4cm} l} 
		\textbf{Tid} & \textbf{Elevinndeling} & \textbf{Læringsarena} \\
		#3  & #4 & #5
	\end{tabular} 
\end{tcolorbox}	
}

\newcommand{\gjen}[1]{\begin{tcolorbox}[center,boxrule=0.1 mm, width=\mywidth,arc=0mm,colback=blue!3] {\large \textbf{Gjennomføring} \vspace{5 pt}}\newline #1  \end{tcolorbox}\vspace{-5pt}}
\newcommand{\eks}[1]{\begin{tcolorbox}[center,boxrule=0.1 mm, width=\mywidth,arc=0mm,colback=green!3] {\large \textbf{Eksempel} \vspace{5 pt}}\newline #1  \end{tcolorbox}\vspace{-5pt}}

\newcounter{opl}
%\numberwithin{opl}{article}


\newcommand{\opl}[1]{
\newpage
{\refstepcounter{opl} %\phantomsection 
\large \textbf{\theopl \;#1} \vsk}
}

% Headlines
\newcommand{\fork}{\textbf{Forkunnskapar}\\}
\newcommand{\forb}{\textbf{Forberedelsar}\\}
\newcommand{\opgvr}{\textbf{Oppgaver}}



%colors
\newcommand{\colr}[1]{{\color{red} #1}}
\newcommand{\colb}[1]{{\color{blue} #1}}
\newcommand{\colo}[1]{{\color{orange} #1}}
\newcommand{\colc}[1]{{\color{cyan} #1}}
\definecolor{projectgreen}{cmyk}{100,0,100,0}
\newcommand{\colg}[1]{{\color{projectgreen} #1}}

% Lister med bokstavar
\usepackage[inline]{enumitem}
% Opg
\newcommand{\abc}[1]{
	\begin{enumerate}[label=\alph*),leftmargin=18pt]
		#1
	\end{enumerate}
}

\usepackage[]{hyperref}

\begin{document}
\subimport{}{rg}

Mål for opplæringen er at eleven skal kunne
\begin{itemize}
	\item forenkle og løse lineære og kvadratiske likninger i trigonometriske uttrykk ved å bruke sammenhenger mellom de trigonometriske funksjonene
	\item omforme trigonometriske uttrykk av typen $ a \sin kx + b \cos kx $ , og bruke dem til å modellere periodiske fenomener
\end{itemize}
\newpage
\section{Trigonometriske funksjoner}
\tssec{Cosinusfunksjoner}
La oss nå ta en titt på funksjonen
\nreq{
f(x) = a\cos (kx+c) + d 	
	}
hvor $ a $, $ k $, $ c $ og $ d $ er konstanter. Dette kaller vi en \textit{cosinusfunksjon}. Under ser vi grafen til $ f(x)= 2\cos\left(\frac{\pi}{2} x\right)+1 $:
\subimport{fig/}{fcos} 
Og vi noterer oss dette: \vs
\begin{itemize}
	\item grafen er symmetrisk om linja $ y=1 $, som vi derfor kaller for \textit{likevektslinja} til grafen. Verdien til likevektslinja samsvarer med konstantleddet til $ f $.
	\item vertikalavstanden fra likevektslinja til et toppunkt er $ 2 $, noe som samsvarer med faktoren foran cosinusuttrykket. Dette tallet kalles \textit{amplituden}.
	\item horisontalavstanden fra et toppunkt til et annet er 4, denne avstanden kalles \textit{perioden} (eventuelt \textit{bølgelengden}). 
\end{itemize}
Om vi ikke visste uttrykket til $ f $, kunne vi altså likevel ut ifra  \fref{fcos} og punktene over sett at $ a=2 $ og $ d=1 $. Men hva med $ b $ og $ c $? 

La oss starte med det enkleste: Når vi kjenner perioden $ P=4 $, kan vi finne $ k $ ut ifra følgende relasjon:
\algv{
k &= \frac{2\pi}{P} \\[5 pt]
&= \frac{2\pi}{4} \\[5 pt]
&= \frac{\pi}{2}	
	}

For å finne $ c $ gjør vi denne observasjonen: En cosinusfunksjon med positiv $ a $ må ha et toppunkt der hvor $ kx+c=0 $ (fordi $ \cos 0=1 $). Siden $ f $ har et toppunkt der hvor $ x=2 $, må vi ha at:
\alg{
	\frac{\pi}{2}\cdot 2 +c &= 0 \\
	c &= -\pi
	}

Før vi tar en liten oppsummering skal vi kort studere grafen til $ g = -2\cos(\frac{\pi}{2}x-\pi) $
\subimport{fig/}{gcos}
Den eneste forskjellen på uttrykkene til $ f $ og $ g $ er at $ g $ har faktoren $ -2 $ foran cosinusuttrykket. Vertikalavstanden fra likevektslinja til et toppunkt er likevel 2 også for $ g $, som derfor har 2 som amplitude. For en hvilken som helst cosinusfunksjon er altså $ |a| $ lik verdien til amplituden. Men fordi $ a $ er negativ har $ g $ et toppunkt når $ kx+c=\pi $ (siden $ \cos \pi=-1 $).
\cosf

\tssec{Sinusfunksjoner}
Om du har lest forrige delseksjon, har du kanskje allerede tenkt deg fram til at funksjoner på formen
\[ f(x) = a\sin (kx+c)+d\]
kalles \textit{sinusfunksjoner}. Amplituden, bølgetallet og likevektslinjen finner vi på akkurat samme måte som for cosinusfunksjoner. 

Fasen finner vi derimot ved å observere at en sinusfunksjon må ha en maksimalverdi der hvor $ kx+c=\frac{\pi}{2} $ hvis $ a $ er positiv \big(fordi $ \sin \frac{\pi}{2}=1 $\big), og der hvor $ kx+c = -\frac{\pi}{2} $ hvis $ a $ er negativ \big(fordi $ \sin -\frac{\pi}{2}=-1 $\big).

\sinf
\rele

\tssec{Sinus og cosinus kombinert}
Vi skal nå se på funksjoner av typen
\[ f(x) = a \cos kx + b \sin kx \]
og hvordan vi kan skrive om disse til én enkelt sinusfunksjon.

Under ser vi grafene til $ f(x) = \cos(\pi x) + \sqrt{3}\sin(\pi x)$
og\, $ g(x) = 2 \sin\left(\pi x +\frac{\pi}{6}	\right)$:
\subimport{fig/}{sincos} 
Disse grafene ser jo fullstendig identiske ut, og det er de også. Saken er at vi kan omskrive $ f $ til $ g $ eller omvendt:

\komb
\kombe

I kapittel ref!! så vi på mange forskjellige trigonometriske ligninger. Nå som vi har lært å kombinere sinus- og cosiunusuttrykk skal vi her se på en siste variant:
\rg[\boldmath $ a\sin kx + b\cos kx = d$]{
Ligningen 
\[ a\sin kx + b\cos kx = d \]	
hvor $ a $, $ b $ og $ c $ er konstanter kan løses ved å omforme venstresiden til et rent sinusuttrykk og deretter løse den resulterende sinusligningen.
	}
\eks{
Løs ligningen
\[ \sin 3x + \cos 3x = \sqrt 2 \]	
\sv
Vi starter med å finne det kombinerte sinusuttrykket for venstresiden av ligningen:
\alg{
R &= \sqrt{1^2 + 1^2}	\\
&= \sqrt{2}
	}
\alg{
	\cos c &= \frac{1}{\sqrt{2}}\br
	&= \frac{\sqrt{2}}{2}\br
	\sin c &= \frac{1}{\sqrt{2}}
	}
$c= \frac{\pi}{4} $ oppfyller kravene over, og vi kan derfor skrive:
\alg{
	\sqrt{2}\sin\left(3x+\frac{\pi}{4}\right)&=\sqrt{2} \br
	\sin\left(3x+\frac{\pi}{4}\right)&=1
	}
Altså har vi at:
\alg{
	3x + \frac{\pi}{4} &= \frac{\pi}{2} + 2\pi n\br
	x &= \frac{1}{3}\left(\frac{\pi}{4}+2\pi n\right)
	}
}


\newpage
\tsec{Forklaringer}
\subsection*{Tallet \boldmath$ k $}
Vi skal nå vise hvorfor vi har relasjonen $ k=\dfrac{2\pi}{P} $.

La oss tenke oss en cosinus- eller sinusfunksjon med $ kx+c $ som argument. Si videre at $ x_1 $ og $ x_2 $ er $ x $-verdien til to naboliggende toppunkt. 
\subimport{fig/}{fcos3} 
Siden et nytt toppunkt kommer for hver gang vi legger til $ 2\pi $ i argumentet, vet vi at:
\algv{
	kx_1+c + 2\pi &=kx_2+c \\
	k(x_2-x_1) &= 2\pi \\
	k &= \frac{2\pi}{x_2-x_1}
	}
Siden $ x_2-x_1 $ er det vi kaller for perioden $ P $, har vi vist det vi skulle.

\subsection*{Sinus og cosinus kombinert}
Vi har uttrykket:
\begin{equation}
a \cos kx + b \sin kx \label{sc}
\end{equation}
Videre vet vi at (se (\ref{suv}))
\begin{equation}
r \sin(kx+c)= r\sin c\cos kx + r\cos c \sin kx  \label{s}
\end{equation}
Uttrykkene fra ligning (\ref{sc}) og (\ref{s}) er like hvis:
\begin{align}
a &= r\sin c \label{a} \\
b &= r\cos c \label{b}
\end{align}
Kvadrerer vi ligning (\ref{a}) og (\ref{b}), får vi:
\begin{align}
a^2 &= r^2\sin^2 c \label{a2}\\
b^2 &= r^2\cos^2 c \label{b2}
\end{align}
Hvis vi nå legger sammen ligning (\ref{a2}) og (\ref{b2}), finner vi et uttrykk for $a$:
\alg{
r^2\sin^2 c+r^2\cos^2 c &= a^2 +b^2 \\
r^2(\sin^2 c+\cos^2 c) &= a^2 +b^2 \\
r^2 &= a^2+b^2 \\
r &= \pm \sqrt{a^2+b^2} 	
	}
Dersom vi velger den positive løsningen for $ r $, får vi at:
\alg{r &=\sqrt{a^2+b^2} \\
	\cos c &= \frac{b}{r} \\[5 pt]
	\sin c &= \frac{a}{r}}

\end{document}

\begin{comment}
\subsection*{Relasjoner}
Fra (\ref{suv}) har vi at:
\alg{\sin \left(kx + c +\frac{\pi}{2}\right) &= \sin (kx +c)\cos \left(\frac{\pi}{2}\right) +\cos (kx +c)\sin \left(\frac{\pi}{2}\right) \\
&= \cos(kx+c)
}
Fra (\ref{cuv}) har vi at:
\alg{\cos \left(kx + c -\frac{\pi}{2}\right) &= \cos (kx +c)\cos \left(-\frac{\pi}{2}\right) -\sin (kx +c)\sin \left(-\frac{\pi}{2}\right) \\
&= \sin(kx+c)
}
\end{comment}
