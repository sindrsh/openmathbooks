\documentclass[hidelinks, 11 pt, class=report,crop=false]{standalone}
\usepackage{geometry}
\geometry{verbose,paperwidth=16.1 cm, paperheight=24 cm, inner=2.3cm, outer=1.8 cm, bmargin=2cm, tmargin=1.8cm}
\usepackage[dvipsnames, xetex, table]{xcolor}
\setlength{\parindent}{0bp}
\usepackage{import}
\usepackage[subpreambles=false]{standalone}
\usepackage{amsmath}
\usepackage{amssymb}
\usepackage{esint}

\usepackage{imakeidx}
\makeindex[title=Indeks]

\usepackage[many]{tcolorbox}
\usepackage{mathtools} % for mathclap



% Lister med bokstavar
\usepackage[inline]{enumitem}
\newcounter{rg}
\numberwithin{rg}{chapter}

%referances
\newcommand{\net}[2]{{\color{blue}\href{#1}{#2}}}
\newcommand{\hrs}[2]{\hyperref[#1]{\color{blue}#2 \ref*{#1}}}
\newcommand{\refunnbr}[2]{\hyperref[#1]{\color{blue}#2}}

\newcommand\fork[2]{\begin{tcolorbox}[boxrule=0.3 mm,arc=0mm,enhanced jigsaw,breakable,colback=yellow!7] {\large \textbf{#1 (\expl)} \vspace{5 pt}\\} #2 \end{tcolorbox}\vspace{-5pt} }
\newcommand{\mb}{\net{https://sindrsh.github.io/FirstPrinciplesOfMath/}{MB}}
\newcommand{\reg}[2][]{\begin{tcolorbox}[boxrule=0.3 mm,arc=0mm,colback=blue!3] {\refstepcounter{rg}\phantomsection \large \textbf{\therg \;#1} \vspace{5 pt}}\newline #2  \end{tcolorbox}\vspace{-5pt}}
\newcommand{\regdef}[2][]{\begin{tcolorbox}[boxrule=0.3 mm,arc=0mm,colback=blue!3] {\refstepcounter{rg}\phantomsection \large \textbf{\therg \;#1} \vspace{5 pt}}\newline #2  \end{tcolorbox}\vspace{-5pt}}
\newcommand{\words}[1]{\begin{tcolorbox}[boxrule=0.3 mm,arc=0mm,colback=teal!3] #1  \end{tcolorbox}\vspace{-5pt}}


\newcommand\eks[2][]{\begin{tcolorbox}[boxrule=0.3 mm,arc=0mm,enhanced jigsaw,breakable,colback=green!3] {\large \textbf{\ekstitle #1} \vspace{5 pt}\\} #2 \end{tcolorbox}\vspace{-5pt} }

\newcommand{\st}[1]{\begin{tcolorbox}[boxrule=0.0 mm,arc=0mm,enhanced jigsaw,breakable,colback=yellow!12]{ #1} \end{tcolorbox}}

\newcommand{\spr}[1]{\begin{tcolorbox}[boxrule=0.3 mm,arc=0mm,enhanced jigsaw,breakable,colback=yellow!7] {\large \textbf{\sprtitle} \vspace{5 pt}\\} #1 \end{tcolorbox}\vspace{-5pt} }

\newcommand{\info}[2]{\begin{tcolorbox}[boxrule=0.3 mm,arc=0mm,enhanced jigsaw,breakable,colback=cyan!6] {\large \textbf{#1} \vspace{5 pt}\\} #2 \end{tcolorbox}\vspace{-5pt} }

\newcommand\algv[1]{\vspace{-11 pt}\begin{align*} #1 \end{align*}}


\newcommand\alg[1]{\begin{align*} #1 \end{align*}}
\newcommand{\sym}[1]{\colorbox{blue!15}{#1}}
\newcommand{\regv}{\vspace{5pt}}
\newcommand{\mer}{\textsl{\note}: }
\newcommand{\mers}[1]{{\footnotesize \mer #1}}
\newcommand\vsk{\vspace{11pt}}
\newcommand{\tbs}{\vspace{5pt}}
\newcommand\vs{\vspace{-11pt}}
\newcommand\vsb{\vspace{-16pt}}
\newcommand\br{\\[5 pt]}
\newcommand{\figp}[1]{../fig/#1}
\newcommand\algvv[1]{\vs\vs\begin{align*} #1 \end{align*}}
\newcommand{\y}[1]{$ {#1} $}
\newcommand{\os}{\\[5 pt]}
\newcommand{\prbxl}[2]{
	\parbox[l][][l]{#1\linewidth}{#2
}}
\newcommand{\prbxr}[2]{\parbox[r][][l]{#1\linewidth}{
		\setlength{\abovedisplayskip}{5pt}
		\setlength{\belowdisplayskip}{5pt}	
		\setlength{\abovedisplayshortskip}{0pt}
		\setlength{\belowdisplayshortskip}{0pt} 
		\begin{shaded}
			\footnotesize	#2 \end{shaded}}}
\newcommand{\fgbxr}[2]{
	\parbox[r][][l]{#1\linewidth}{#2
}}		


\usepackage{datetime2}

% outline word
\newcommand{\outl}[1]{{\boldmath \color{teal}\textbf{#1}}}
%line to seperate examples
\newcommand{\linje}{\rule{\linewidth}{1pt} }


\usepackage[]{hyperref}

% arabic language
\usepackage{polyglossia}
\setdefaultlanguage{arabic}
\setmainfont{Amiri}
\newcommand{\note}{ملاحظة}
\newcommand{\notesm}[1]{{\footnotesize \textsl{\note:} #1}}
\newcommand{\ekstitle}{مثال }
\newcommand{\sprtitle}{صندوق اللغة}
\newcommand{\expl}{توضيح}

\newcommand\sv{\vsk \textbf{الإجابة} \vspace{4 pt}\\}

%references
\newcommand{\reftab}[1]{\hrs{#1}{جدول}}
\newcommand{\rref}[1]{\hrs{#1}{قاعدة}}
\newcommand{\dref}[1]{\hrs{#1}{تعريف}}
\newcommand{\refkap}[1]{\hrs{#1}{فصل}}
\newcommand{\refsec}[1]{\hrs{#1}{قسم}}
\newcommand{\refdsec}[1]{\hrs{#1}{قسم فرعي}}
\newcommand{\refved}[1]{\hrs{#1}{ملحق}}
\newcommand{\eksref}[1]{\textsl{#1}}
\newcommand\fref[2][]{\hyperref[#2]{\textsl{شكل \ref*{#2}#1}}}
\newcommand{\refop}[1]{{\color{blue}تمرين \ref{#1}}}
\newcommand{\refops}[1]{{\color{blue}تمرين \ref{#1}}}
\newcommand{\refgrubs}[1]{{\color{blue}تأمل \ref{#1}}}

\newcommand{\openmathser}{\openmath\,-\,السلسلة}

% Exercises
\newcommand{\opgt}{\newpage \phantomsection \addcontentsline{toc}{section}{تمارين} \section*{تمارين للفصل \thechapter}\vs \setcounter{section}{1}}

% Sequences and series
\newcommand{\sumarrek}{مجموع سلسلة حسابية}
\newcommand{\sumgerek}{مجموع سلسلة هندسية}
\newcommand{\regnregsum}{قوانين جمع السلسلة}

% Trigonometry
\newcommand{\sincoskomb}{جمع السين و الكوس}
\newcommand{\cosfunk}{وظيفة الكوسين}
\newcommand{\trid}{هويات تريغونومترية}
\newcommand{\deravtri}{المشتقة للوظائف التريغونومترية}
% Solutions manual
\newcommand{\selos}{انظر إلى الحل المقترح.}
\newcommand{\se}[1]{انظر المثال في الصفحة \pageref{#1}}

%Vectors
\newcommand{\parvek}{متوازي الاتجاهات}
\newcommand{\vekpro}{منتج النواة}
\newcommand{\vekproarvol}{منتج النواة كمساحة وحجم}

% 3D geometries
\newcommand{\linrom}{خط في الفضاء}
\newcommand{\avstplnpkt}{المسافة بين نقطة ومستوى}

% Integral
\newcommand{\bestminten}{التكامل المحدد I}
\newcommand{\anfundteo}{نظرية أساس الأناليز}
\newcommand{\intuf}{تكامل وظائف مختارة}
\newcommand{\bytvar}{تغيير المتغير}
\newcommand{\intvol}{التكامل كحجم}
\newcommand{\andordlindif}{معادلات التفاضل الخطية من الدرجة الثانية}


\begin{document}
\setcounter{chapter}{1}
\section{سلاسل}\index{سلسلة}
السلاسل هي تسلسل للأرقام، عادة ما تكون مفصولة بفواصل. في السلسلة
\begin{equation}
	2, 4, 8, 16  \label{folg}
\end{equation}
نقول أن لدينا أربعة \outl{عناصر}\index{عنصر}. العنصر رقم 1 يكون قيمته 2، والعنصر رقم 2 يكون قيمته 4 وهكذا. يتم وصف كل عنصر في السلسلة عادة بحرف مؤشر. إذا اخترنا الحرف \( a \) للسلسلة أعلاه، يمكننا كتابة \( a_1 = 2 \)، \( a_2 = 4 \) وهكذا.\vsk

عندما نستخدم \( a_i \) للإشارة إلى العناصر في السلسلة، نستخدم \( {i\in \mathbb{N}} \). مثل المجموعات، يمكننا استخدام \sym{$ \lbrace\rbrace $} للدلالة على سلسلة، و \sym{$ \in $} للإشارة إلى أن عنصر معين موجود في السلسلة. مثلاً \( 8\in\lbrace2, 4, 8, 16\rbrace \).\vsk

غالباً ما يمكن ربط الأرقام في السلسلة ببعضها. إذا ضربنا عنصراً من السلسلة في \eqref{folg} بـ \( 2 \)، فسنحصل على العنصر التالي. الصيغة \outl{المتكررة}\index{صيغة متكررة} هي
\[ a_i = 2\cdot a_{i-1} \]
في الصيغة المتكررة نستخدم القيمة السابقة للعثور على القيمة التالية.\vsk

السلسلة المذكورة هي سلسلة \outl{محددة}\index{سلسلة!محددة} لأنها تحتوي على عدد محدد من العناصر. لو استخدمنا الصيغة المتكررة، يمكننا إضافة المزيد من العناصر ونحصل على السلسلة
\begin{equation}
	2, 4, 8, 16, 32, 64, ...  \label{folg1}
\end{equation}
\( ... \) يعني أن العناصر تستمر إلى ما لا نهاية، ويُطلق على السلسلة اسم سلسلة \outl{لا متناهية}\index{سلسلة!لا متناهية}.\vsk

ماذا لو أردنا البحث عن العنصر رقم 20 في هذه السلسلة، أي \( a_{20} \)؟ سيكون من المفيد البحث عن صيغة \outl{صريحة}. للقيام بذلك نكتب بعض العناصر ونرى إذا كنا نجد نمطًا:
\alg{
	& a_1 = 2 = 2^1 \\
	& a_2 = 4 = 2^2 \\
	& a_3 = 8 = 2^3
}
من المعادلات أعلاه ندرك أن للعنصر رقم \( i \) يمكن كتابته على النحو التالي:
\[ a_i=2^i \]
وبهذه الطريقة يمكننا بسهولة العثور على العنصر رقم 20:
\alg{
	a_{20}&=2^{20} \\
	&= 1048576
}
الصيغة الصريحة تعطينا تعبيرًا يتم حساب قيمة العنصر مباشرة من خلاله. عندما نمتلك تعبيرًا من هذا النوع، فإنه من المعتاد أيضًا كتابة هذا كآخر عنصر في السلسلة، وتصبح \eqref{folg1} كالتالي:
\[  2, 4, 8, 16, 32, 64, ..., 2^i \]
\subsection{التتابعات الحسابية} \index{تتابع!حسابي}

التتابع
\[ 2, 5, 8, 11, 14, 17 \]
يُطلق عليه \textit{تتابع حسابي}. في التتابع الحسابي, يكون لدينا فرق ثابت بين كل عنصرين متتاليين، وفي هذه الحالة الفرق هو 3. عندما نكتب العناصر الثلاثة الأولى نستطيع أن نجد النمط للصيغة المحددة \index{صيغة محددة}:
\alg{
	& a_1 = 2 = 2+3\cdot0\\
	& a_2 = 5 = 2+3\cdot1\\
	& a_3 = 8 = 2 +3\cdot2
}
من المعادلات أعلاه نلاحظ أن
\[ a_i=2+3\cdot(i-1) \]
\reg[التتابع الحسابي]{
	العنصر \( a_i \) في \outl{تتابع حسابي} معطى بالصيغة التكرارية
	\begin{equation}\label{arrekeq}
		a_i=a_{i-1}+d
	\end{equation}
	والصيغة المحددة
	\begin{equation}
		a_i=a_1+d(i-1) \label{ekspl}
	\end{equation}
	حيث \( d \) هو الفرق الثابت \( a_i-a_{i-1} \).
}
\newpage
\eks[]{ابحث عن الصيغة التكرارية والصيغة المحددة للتتابع
	\[ 7, 13, 19, 25, ... \]
	\vs
	\sv
	التتابع له فرق ثابت \( {d=6} \) والعنصر الأول هو \( {a_1=7} \). الصيغة التكرارية تصبح 
	\[ a_i = a_{i-1}+6 \]
	في حين تصبح الصيغة المحددة
	\[ a_i=7+6(i-1) \]
}

\subsection{التتابعات الهندسية} \index{تتابع!هندسي}

التتابع 
\[ 2, 6, 18, 54, 162 \]
يُطلق عليه \textit{تتابع هندسي}. في التتابع الهندسي, النسبة بين كل عنصرين متتاليين تكون ثابتة، وفي هذه الحالة النسبة هي 3. هنا أيضًا نستطيع أن نكتشف نمطًا ثابتًا:
\algv{
	& a_1 = 2 = 2\cdot3^0\\
	& a_2 = 6 = 2\cdot3^1\\
	& a_3 = 18 = 2\cdot3^2
}
إذاً الصيغة المحددة هي
\[ a_i = 2\cdot3^{i-1} \]
\reg[التتابع الهندسي]{
	العنصر \( a_i \) في \outl{تتابع هندسي} بنسبة \( k \) معطى بالصيغة التكرارية
	\begin{equation}\label{geofolgrek}
		a_i=a_{i-1}\cdot k
	\end{equation}
	والصيغة المحددة
	\begin{equation}\label{geofolgeks}
		a_i=a_1\cdot k^{i-1}
	\end{equation}
}
\eks[1]{
	ابحث عن الصيغة التكرارية والصيغة المحددة للتتابع
	\[ 5, 10, 20, 40, ... \]
	
	\sv
	التتابع له نسبة ثابتة \( {k=2} \)، والعنصر الأول هو \( {a_1=5} \). الصيغة التكرارية تصبح
	\[ a_i= a_{i-1}\cdot 2 \]
	في حين تصبح الصيغة المحددة
	\[ a_i=5\cdot2^{i-1} \]
}
\eks[2]{تتابع هندسي له \( {a_1 = 2} \) و \( {k=4} \). لأي قيمة من \( i \) يكون \( {a_i=128 } \)? \\
	
	\sv
	لدينا المعادلة
	\algv{
		2\cdot4^{i-1}&=128 \\
		4^{i-1}&= 64 \\
		4^{i-1} &= 4^3 \\
		i-1 &= 3\\
		i &= 4
	}
	إذا كان \( {a_4=128} \).
}
\section{سلاسل}\index{سلسلة}
\textit{السلسلة} هي في الواقع نفسها \textit{عبارة الجمع} (انظر \mb). على سبيل المثال، هي
\[ 2+6+18+54+162 \]
سلسلة. نستخدم المصطلح \textit{عنصر} بنفس الطريقة التي نستخدم فيها \textit{عنصر} لتتابع، في السلسلة أعلاه، العنصر رقم 3 له القيمة 18، وفي المجموع هناك خمسة عناصر.\vsk

بالنسبة لسلسلة، من الطبيعي أننا لا نرغب فقط في معرفة قيمة كل عنصر بشكل فردي، ولكن أيضًا ما هو مجموع جميع العناصر. طالما أن السلسلة ليست لانهائية، يمكن دائمًا جمع العناصر واحدًا تلو الآخر، ولكن بالنسبة لبعض السلاسل، هناك تعبيرات تعطينا المجموع بكثير أقل من العمل (وحتى في حالات السلاسل اللانهائية).

\subsection{سلاسل حسابية}
\reg[\sumarrek \label{sumarrek}]{
	
	\index{سلسلة!حسابية}
	إذا كانت العناصر في سلسلة يمكن وصفها بتتابع حسابي، تُسمى السلسلة بـ\outl{سلسلة حسابية}.\vsk
	
	المجموع \( S_n \) لأول \( n \) عنصر في سلسلة حسابية هو
	\begin{equation}\label{arrek}
		S_n = n\frac{a_1+a_n}{2}
	\end{equation}
	حيث \( a_1 \) هو العنصر الأول في السلسلة.
} \regv

\eks{ \label{arrekeks}
	نظرًا للسلسلة اللانهائية
	\[ 3+7+11+... \] 
	ابحث عن مجموع العناصر العشرة الأولى \os
	
	\sv
	العنصر \( i \) في السلسلة \(a_i\) معطى بالصيغة
	\[ a_i=3+4(i-1) \]
	لذا، هذه هي سلسلة حسابية، ومجموع العناصر \( n \) الأولى هو
	\alg{
		a_{10} &= 3+4(10-1) \\
		&= 39	
	} 
	العناصر العشرة الأولى هي
	\alg{
		S_{10} &= 10\cdot\frac{3+39}{2} \\
		&= 210
	}
} \vsk
\fork{\ref{sumarrek} \sumarrek}{
	باستخدام الصيغة الصريحة من \eqref{ekspl}، يمكننا كتابة مجموع سلسلة حسابية بها \( n \) عنصر كـ
	\begin{equation}
		S_n = a_1 + (a_1+d) + (a_1+ 2d)+...+ (a_1+d(n-1)) \label{a1}
	\end{equation}
	ويمكن أيضًا التعبير عن عنصر في السلسلة كـ:
	\[ a_i= a_n-(n-i)d \]
	لـ \( 1\leq i\leq n \).
	وبذلك، يمكننا كتابة المجموع (هنا العنصر الأخير أولاً، ثم الذي قبله وهكذا)
	\begin{equation}
		S_n = a_n + (a_n-d)+(a_n-2d)+...+(a_n-d(n-1)) \label{a2}
	\end{equation}
	عند جمع \eqref{a1} و \eqref{a2}، نحصل على \( 2S_n \) على الجانب الأيسر. على الجانب الأيمن، جميع \textit{d} تُلغى، ونحصل على
	\alg{
		2S_n &= na_1 + na_n \br
		S_n&=n\frac{a_1+a_n}{2} 
	}
}

\newpage
\subsection{التسلسلات الهندسية} 
\reg[\sumgerek \label{geom} \index{rekke!geometrisk}]{
	إذا كان يمكن وصف أحد أجزاء التسلسل بأنه تتابع هندسي، يُسمى التسلسل \outl{تسلسل هندسي}.	\vsk
	
	مجموع $ S_n $ للـ $ n $ جزء الأولى في التسلسل الهندسي بنسبة $ k $ والجزء الأول $ a_1 $ معطى كالتالي:
	\begin{equation}
		S_n=a_1\frac{1-k^n}{1-k}\quad, \quad k\neq 1 \label{sumg}
	\end{equation}
	إذا كان $ {k=1}, $ فإن:
	\begin{equation}
		S_n = na_1
	\end{equation}
}
\eks{ \label{gerekeks}
	نظرًا للتسلسل اللانهائي:
	\[ 3+6+12+24+... \]	
	ابحث عن مجموع الـ 15 جزء الأولى.
	
	\sv
	هذا هو التسلسل الهندسي بـ $ {a_1=3} $ و $ {k=2} $. مجموع الـ 15 جزء الأولى هو:
	\alg{
		S_{15}&= 3\cdot\frac{2^{15}-1}{1-2} \br
		&= 	3\cdot\frac{1-32768}{-1} \br
		&= 98301
	}
}
\newpage
\fork{\ref{geom} \sumgerek}{
	مجموع $ S_n $ للتسلسل الهندسي بـ $ n $ جزء هو:
	\begin{equation}
		S_n = a_1 + a_1k + a_1k^2+...+a_1 k^{n-2}+a_1 k^{n-1} \label{geo1}
	\end{equation}
	إذا ضربنا هذا المجموع بـ $ k $، نحصل على:
	\begin{equation}
		kS_n = a_1k + a_1k^2 + a_1k^3+...+a_1 k^{n-1}+a_1 k^{n} \label{geo2}
	\end{equation}
	التعبير الذي نبحث عنه يظهر عندما نخصم \eqref{geo2} من \eqref{geo1}:
	\alg{
		S_n-kS_n &= a_1-a_1k^n \\
		S_n(1-k) &= a_1(1-k^n) \\
		S_n &= a_1\frac{(1-k^n)}{1-k}
	}
}

\subsection{التسلسلات الهندسية اللانهائية}
عندما يكون للتسلسل الهندسي عناصر لانهائية، نلاحظ:

إذا كان $ {|k|<1} $، فإن:
\alg{
	\lim\limits_{n\to\infty} S_n &=\lim\limits_{n\to\infty} a_1\frac{1-k^n}{1-k}  \\[5pt]
	&= a_1 \frac{1}{1-k}
}
إذاً، مجموع عناصر لانهائية يتجه نحو قيمة محددة (واقعية)! عندما يكون هذا حقيقة نقول أن التسلسل \textit{يتقارب}\index{konvergere} وأن التسلسل هو متقارب\index{rekke!konvergent}. إذا كان من ناحية أخرى $ |k|\geq 1$، يتجه المجموع نحو $ \pm \infty$. في هذه الحالة نقول أن التسلسل \textit{يتباعد}\index{divergere} وأن التسلسل هو متباعد\index{rekke!divergent}.\regv
\reg[مجموع التسلسل الهندسي اللانهائي]{
	للتسلسل الهندسي اللانهائي بنسبة $ {k<|1|} $ والجزء الأول $ a_1 $، مجموع $ S_\infty $ للتسلسل معطى كالتالي:
	\begin{equation}
		S_\infty = \frac{a_1}{1-k}
	\end{equation}
	
	إذا كان $ |k|\geq 1 $، سيرتفع المجموع نحو $ \pm \infty $.
}
\newpage
\eks{ \label{gerekueneks}
	نظرًا للتسلسل اللانهائي: 
	\[ 1+\frac{1}{x}+\frac{1}{x^2}+.... \]
	\textbf{a)} ما هي قيمة $ x $ التي يكون فيها التسلسل متقاربًا؟ \os
	
	\textbf{b)} أظهر أن \y{S_\infty=\frac{x}{x-1}} عندما يتقارب التسلسل.\os
	
	\textbf{c)} ما هي قيمة $ x $ التي يكون فيها مجموع التسلسل مساويًا لـ $ \frac{3}{2} $؟\os
	
	\textbf{d)} ما هي قيمة $ x $ التي يكون فيها مجموع التسلسل مساويًا لـ $ -1 $?\\
	
	\sv
	\textbf{a)} هذا هو التسلسل الهندسي بـ $ {k=\frac{1}{x} }$ و ${ a_1 = 1} $. التسلسل يكون متقاربًا عندما $ {|k|<1} $، ولذلك نتطلب:
	\[ |x|>1  \] 
	
	\textbf{b)} عندما $ |x|>1 $، نحصل على:
	\alg{
		S_\infty &= \frac{a_1}{1-k} \\
		&= \frac{1}{1-\frac{1}{x}} \\
		&= \frac{1}{\frac{x-1}{x}} \\
		&= \frac{x}{x-1}
	}
	وهو ما كنا بحاجة لإثباته.\vsk
}

\newpage
\subsection{رمز المجموع}\index{رمز المجموع}
سننظر الآن في رمز يبسط طريقة كتابة السلاسل بشكل كبير. يصبح هذا الرمز مهمًا بشكل خاص في {التكامل}، حيث سندرس \textit{التكامل}.\vsk

في السابق كتبنا السلسلة بشكل مباشر أو أقله. على سبيل المثال، رأينا السلسلة
\[ 2+6+18+54+162 \]
بالصيغة الصريحة
\[ a_n = 2\cdot3^{n-1} \]
باستخدام رمز المجموع $ \sum $ يمكننا ضغط سلسلتنا بشكل كبير. عند كتابة $ \sum\limits_{i=1}^5 $ نشير إلى أن $ i $ هو متغير متزايد يبدأ من 1 ويزداد بمقدار 1 حتى 5. إذا كنا نعبر عن الصيغة الصريحة للسلسلة باستخدام $ i $، يمكننا كتابة السلسلة ك $ \sum\limits_{i=1}^5 2\cdot3^{i-1} $، مع فهم أنه يتعين وضع علامة جمع كلما زاد $ i $ بمقدار 1:
\[ 2+6+18+54+162=\sum\limits_{i=1}^5 2\cdot3^{i-1} \]
أما السلسلة اللانهائية 
$ 2+6+18+... $ يمكننا كتابتها كالتالي
\[ \sum\limits_{i=1}^\infty 2\cdot3^{i-1} \]

بالنسبة لرمز المجموع، لدينا أيضًا بعض القواعد الحسابية التي يجدر بنا ذكرها:\regv
\reg[قواعد الجمع \label{قواعد الجمع}]{
	لهاتين السلاسلتين $ \lbrace a_i\rbrace $ و $ \lbrace b_i\rbrace $ وثابت $ c $ لدينا:
	\begin{align}
		\sum_{i=j}^{n} \left(a_i+b_i\right) &= \sum_{i=j}^{n} a_i + \sum_{i=j}^{n} b_i \label{قاعدة1}\br
		\sum_{i=j}^{n} c a_i &= c\sum_{i=j}^{n} a_i \label{قاعدة2}
	\end{align}
	حيث $ {j, n \in \mathbb{N}} $ و $ {j<n} $.
}

\newpage
\fork{\ref{قواعد الجمع} \ قواعد الجمع}{
	عند كتابة المجموع وإعادة ترتيب الجمع، ندرك أن
	\alg{
		\sum_{i=1}^{n} \left(a_i+b_i\right) &= a_1+b_1 + a_2+b_2 + ... + a_n + b_n \\
		&= a_1+a_2+...+a_n + b_1+b_2+...+b_n \\
		&= \sum_{i=1}^{n} a_i + \sum_{i=1}^{n} b_i 
	}
	عند كتابة المجموع واستخراج العامل $ c $، ندرك أيضًا أن
	\alg{
		\sum_{i=1}^{n} c a_i &= ca_1 + ca_2 + ...+ ca_n \\
		&= c(a_1+a_2+...+a_n) \\
		&= c\sum_{i=1}^{n} a_i 
	}
}

\newpage
\section{الاستقراء} \index{الاستقراء}
في الرياضيات النظرية، هناك متطلبات صارمة لإثبات الصيغ. الطريقة التي يتم استخدامها خصوصاً للصيغ التي تحتوي على أعداد صحيحة هي \textit{الاستقراء}. والمبدأ هو كما يلي\footnote{سيتم استخدام كل من كلمة صيغة ومعادلة بالتبادل. الصيغة في الواقع هي مجرد معادلة حيث يمكننا العثور على الكمية المجهولة مباشرةً عن طريق إدخال الكميات المعروفة.}:\vsk

\textsl{إذا كان لدينا معادلة صحيحة لعدد صحيح $ n $. إذا كنا نستطيع أن نظهر أن المعادلة صحيحة أيضاً عندما نضيف العدد الصحيح بـ 1، فقد أظهرنا أن المعادلة صحيحة لجميع الأعداد الصحيحة التي تكون أكبر من أو تساوي} $ n $.\vsk

قد يكون من الصعب في البداية فهم مبدأ الاستقراء، لذا دعونا ننتقل مباشرة إلى مثال:\vsk

نريد أن نظهر أن مجموع الأعداد الزوجية الـ $ n $ الأولى يساوي $ n(n+1) $:
\begin{equation}
	2+4+6+...+2n=n(n+1) \label{induk}
\end{equation}
نبدأ بإظهار أن هذا صحيح لـ $ {n=1} $:
\alg{2 &= 1\cdot(1+1) \\
	2&= 2}
الآن نعرف عن عدد صحيح، وهو $ {n=1} $، أن الصيغة صحيحة له. بعد ذلك نفترض أن المعادلة صحيحة حتى العنصر رقم $ k $. نريد بعد ذلك التحقق من صحتها للعنصر التالي، أي عندما $ n=k+1 $. الجمع يصبح
\[ 2+4+6+...+\quad\mathclap{\overbrace{2k}^{\text{العنصر رقم }k} \;\;\;\,}+\quad\;\;\;\mathclap{\underbrace{2(k+1)}_{\text{العنصر رقم }k+1}}\qquad=(k+1)((k+1)+1) \]
ولكن حتى العنصر رقم $ k $، تم افتراض أن \eqref{induk} صحيحة، وبالتالي نحصل على\footnote{قد يبدو غريبًا قليلًا أن نكتب $ {2+4+6+...+2k }$، ونفترض أن صيغتنا صحيحة لهذا المجموع. يبدو أننا نفترض أنها صحيحة لـ $ {n=1 }$, $ {n=2 }$ وهكذا. ولكن هذه مجرد طريقة كتابة غير حقيقية تُستخدم للمجموع حتى العنصر رقم $ k $. لأننا بعد ذلك نقول أننا نعرف عن عدد $ k $ أن هذا الافتراض صحيح بالنسبة له، وهو $ {k=1} $، وبالتالي لدينا عنصر واحد فقط قبل العنصر رقم $ {k+1} $. 
	
	في الأمثلة التالية، سنترك العنصر رقم $ k $ ضمن الرمز ''$ ... $''. }
\algv{
	\underbrace{2+4+6+...+2k}_{k(k+1)}+2(k+1)&=(k+1)((k+1)+1) \\ 
	k(k+1)+2(k+1)&=(k+1)(k+2)\br
	(k+1)(k+2)&= (k+1)(k+2)
}
والآن نأتي إلى الاستنتاج المدهش: لقد أظهرنا أن \eqref{induk} صحيحة لـ ${ n=1 }$. بالإضافة إلى أننا أظهرنا إذا كانت المعادلة صحيحة لعدد صحيح ${n= k} $، فإنها صحيحة أيضاً لـ $ {n=k+1} $. وبسبب ذلك نعرف أن \eqref{induk} صحيحة لـ ${n= 1+1=2} $. ولكن عندما نعرف أنها صحيحة لـ $ {n=2} $، فإنها صحيحة أيضاً لـ $ {n=2+1=3} $ وهكذا، أي لجميع الأعداد الصحيحة!\regv
\reg[الاستقراء]{
	عندما نريد بالاستقراء أن نظهر أن المعادلة
	\begin{equation}\label{induksjon}
		A(n)=B(n)
	\end{equation}
	صحيحة لكل $ n\in\mathbb{N} $، نقوم بما يلي:
	\begin{enumerate}
		\item التحقق من أن \eqref{induksjon} صحيحة لـ $ {n=1} $.
		\item التحقق من أن \eqref{induksjon} صحيحة لـ $ {n=k+1} $، مع افتراض أنها صحيحة لـ $ {n=k} $.
	\end{enumerate}
}
\newpage
\eks[1]{
	أظهر باستخدام الاستقراء أن مجموع الأعداد الفردية الـ $ n $ الأولى معطى بالمعادلة
	\[ 1+3+5+...+(2n-1) = n^2 \]
	لكل $ n\in\mathbb{N} $.
	
	\sv
	نتحقق أن الفرض صحيح لـ $ {n=1} $:
	\alg{
		1 &= 1^2 \\
		1 &= 1
	}
	نفترض أن الفرض صحيح لـ $ {n=k} $ ونتحقق منه لـ $ {n=k+1 }$:
	\alg{
		\underbrace{1+3+5+...}_{k^2}+(2(k+1)-1) &=(k+1)^2 \\
		k^2 + 2k+1 &= (k+1)^2 \br
		(k+1)^2 &= (k+1)^2
	}
	وبذلك يكون الفرض قد أثبت لكل $ n\in\mathbb{N} $.\vsk
	
	\textsl{ملاحظة:} إذا واجهت مشكلة في عاملة الجانب الأيسر أثناء الاستقراء، يمكنك كحل بديل كتابة الجانب الأيمن بدلاً من ذلك، ولكن من الأفضل أن تتجنب ذلك. هذا من أجل الأناقة (حتى الرياضيات لا تستطيع التخلص من نوع من الغرور)، ولكن أيضًا لأن فرصة الخطأ في الحساب تصبح أقل.
}
\newpage
\eks[2]{
	أظهر باستخدام الاستقراء:
	\[ 1^3 + 2^3 + 3^3 + ... + n^3= \frac{n^2(n+1)^2}{4} \]
	لكل $ n\in\mathbb{N} $.
	
	\sv
	نبدأ بالتحقق لـ $ n=1 $:
	\alg{1 &= \frac{1^2\cdot(1+1)^2}{4} \\
		1^3&= \frac{2^2}{4} \\
		1 &= 1}
	المعادلة صحيحة بالتالي لـ $ {n=1 }$. نفترض أنها صحيحة أيضًا لـ $ {n=k} $ ونتحقق لـ $ {n=k+1} $:
	\alg{\underbrace{1^3+2^3+3^3+...}_{\frac{k^2(k+1)^2}{4}}+(k+1)^3 &= \frac{(k+1)^2(k+1+1)^2}{4} \\
		\frac{k^2(k+1)^2}{4} + (k+1)^3 &= \frac{(k+1)^2(k+2)^2}{4} \\
		\frac{k^2(k+1)^2+4(k+1)^3}{4} &= \\
		\frac{(k+1)^2(k^2+4(k+1))}{4} &= \\	
		\frac{(k+1)^2(k^2+4k+4))}{4} &= \\	
		\frac{(k+1)^2(k+2)^2}{4} &= \frac{(k+1)^2(k+2)^2}{4}
	}
	وبذلك الفرض قد أثبت لكل $ n\in\mathbb{N} $.		
}\newpage
\eks[3]{\label{prodind}
	أظهر باستخدام الاستقراء:
	\[ 3\cdot9\cdot27\cdot... \cdot 3^n= 3^{\frac{1}{2}n(n+1)}\]\vs
	\sv
	نتحقق أن الفرض صحيح لـ $ n=1 $:
	\alg{
		3 &= 3^{\frac{1}{2}\cdot1(1+1)} \\
		3 &= 3^1
	}
	نفترض أن الفرض صحيح أيضًا لـ $ {n=k}$ ونتحقق لـ $ n=k+1 $:
	\alg{
		\underbrace{3\cdot9\cdot27\cdot...}_{3^{\frac{1}{2}k(k+1)}} \cdot 3^{k+1} &= 3^{\frac{1}{2}(k+1)(k+1+1)} \br
		3^{\frac{1}{2}k(k+1)}\cdot  3^{k+1} &= 3^{\frac{1}{2}(k+1)(k+2)} \\
		3^{\frac{1}{2}k(k+1)+k+1} &= \\
		3^{\frac{1}{2}k(k+1)+\frac{2}{2}(k+1)} &= \\
		3^{\frac{1}{2}(k+1)(k+2)} &= 3^{\frac{1}{2}(k+1)(k+2)}
	}
	وبذلك الفرض قد أثبت لكل $ n\in \mathbb{N} $.
}


\end{document}