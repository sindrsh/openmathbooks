\documentclass[english,hidelinks,pdftex, 11 pt, class=report,crop=false]{standalone}
\usepackage[T1]{fontenc}
\usepackage[utf8]{luainputenc}
\usepackage{lmodern} % load a font with all the characters
\usepackage{geometry}
\geometry{verbose,a4paper, inner=0cm, outer=0 cm, bmargin=2cm, tmargin=1cm}
%\textwidth=12cm
\setlength{\parindent}{0bp}
\usepackage{import}
\usepackage[subpreambles=false]{standalone}
\usepackage{amsmath}
\usepackage{amssymb}
\usepackage{esint}
\usepackage{babel}
\usepackage{tabu}
\usepackage[dvipsnames, table]{xcolor}
\usepackage{cancel}
\makeatother
\makeatletter
\usepackage{datetime2}
\usepackage{titlesec}
\usepackage[many]{tcolorbox}

% Eheter
\newcommand{\enh}[1]{\,\textrm{#1}}
%referances
\newcommand{\net}[2]{{\color{blue}\href{#1}{#2}}}

%Spaces
\newcommand{\vsk}{\\[12pt]}
\newcommand{\vs}{\vspace{-12pt}}

% Tabell for opplegg

\newcommand{\ovlist}[1]{
\vspace{-16pt}
\begin{itemize}
	#1
\end{itemize}
}

% Chapters and sections
\titleformat{\section}[block]{\bfseries}{\hspace{3cm}\thesection}{5pt}{}
\titleformat{\subsection}[block]{\bfseries}{\hspace{3cm}\thesection}{5pt}{}
\newcommand{\sectionbreak}{\clearpage} % New page on each section
 

\newlength{\mywidth}
\setlength{\mywidth}{14cm}

\newcommand{\cont}[1]{
\begin{tcolorbox}[center, boxrule=0.0 mm, width=\mywidth,arc=0mm,enhanced jigsaw,,colback=white,breakable]
#1	
\end{tcolorbox}
}

\newcommand{\info}[5]{
\begin{tcolorbox}[center, boxrule=0.1 mm, width=\mywidth,arc=0mm,enhanced jigsaw,breakable,colback=yellow!5]	
	
	\footnotesize
	\textbf{Øvingsområde}\\[5pt] #1 
	
	\textbf{Utstyr}\\ #2  \\
	
	\begin{tabular}{@{} p{4cm} p{4cm} l} 
		\textbf{Tid} & \textbf{Elevinndeling} & \textbf{Læringsarena} \\
		#3  & #4 & #5
	\end{tabular} 
\end{tcolorbox}	
}

\newcommand{\gjen}[1]{\begin{tcolorbox}[center,boxrule=0.1 mm, width=\mywidth,arc=0mm,colback=blue!3] {\large \textbf{Gjennomføring} \vspace{5 pt}}\newline #1  \end{tcolorbox}\vspace{-5pt}}
\newcommand{\eks}[1]{\begin{tcolorbox}[center,boxrule=0.1 mm, width=\mywidth,arc=0mm,colback=green!3] {\large \textbf{Eksempel} \vspace{5 pt}}\newline #1  \end{tcolorbox}\vspace{-5pt}}

\newcounter{opl}
%\numberwithin{opl}{article}


\newcommand{\opl}[1]{
\newpage
{\refstepcounter{opl} %\phantomsection 
\large \textbf{\theopl \;#1} \vsk}
}

% Headlines
\newcommand{\fork}{\textbf{Forkunnskapar}\\}
\newcommand{\forb}{\textbf{Forberedelsar}\\}
\newcommand{\opgvr}{\textbf{Oppgaver}}



%colors
\newcommand{\colr}[1]{{\color{red} #1}}
\newcommand{\colb}[1]{{\color{blue} #1}}
\newcommand{\colo}[1]{{\color{orange} #1}}
\newcommand{\colc}[1]{{\color{cyan} #1}}
\definecolor{projectgreen}{cmyk}{100,0,100,0}
\newcommand{\colg}[1]{{\color{projectgreen} #1}}

% Lister med bokstavar
\usepackage[inline]{enumitem}
% Opg
\newcommand{\abc}[1]{
	\begin{enumerate}[label=\alph*),leftmargin=18pt]
		#1
	\end{enumerate}
}

\usepackage[]{hyperref}

\newcommand{\note}{Merk}
\newcommand{\notesm}[1]{{\footnotesize \textsl{\note:} #1}}
\newcommand{\ekstitle}{Eksempel }
\newcommand{\sprtitle}{Språkboksen}
\newcommand{\expl}{forklaring}
\newcommand{\pyt}{Pytagoras' setning}
\newcommand\sv{\vsk \textbf{Svar} \vspace{4 pt}\\}

%references
\newcommand{\reftab}[1]{\hrs{#1}{tabell}}
\newcommand{\rref}[1]{\hrs{#1}{regel}}
\newcommand{\dref}[1]{\hrs{#1}{definisjon}}
\newcommand{\refkap}[1]{\hrs{#1}{kapittel}}
\newcommand{\refsec}[1]{\hrs{#1}{seksjon}}
\newcommand{\refdsec}[1]{\hrs{#1}{delseksjon}}
\newcommand{\refved}[1]{\hrs{#1}{vedlegg}}
\newcommand{\eksref}[1]{\textsl{#1}}
\newcommand\fref[2][]{\hyperref[#2]{\textsl{figur \ref*{#2}#1}}}
\newcommand{\refop}[1]{{\color{blue}Oppgave \ref{#1}}}
\newcommand{\refops}[1]{{\color{blue}oppgave \ref{#1}}}


%Algebra
\newcommand{\kvadset}{Kvadratsetningene}
\newcommand{\aenato}{Sum-produkt-metoden}

% Geometry
\newcommand{\hlikb}{Midtnormalen i en likebeint trekant}
\newcommand{\arealsetn}{Arealsetningen}
\newcommand{\trkmedian}{Median}
\newcommand{\midtrk}{Midtnormal (i trekant)}
\newcommand{\innskrsirk}{Innskrevet sirkel}
\newcommand{\cossetn}{Cosinussetningen}
\newcommand{\perfvink}{Sentral- og periferivinkel}
\newcommand{\tang}{Tangent}

% Derivative
\newcommand{\derel}{Den deriverte av elementære funksjoner}
\newcommand{\divder}{Divisjonsregelen}
\newcommand{\kjernereg}{Kjerneregelen}
\newcommand{\prodregder}{Produktregelen}
\newcommand{\lhop}{L'Hopitals regel}

% Funksjonsdrofting
\newcommand{\monder}{Monotoniegenskaper og den deriverte}
\newcommand{\fderekstr}{$ \bm{f'=0} $ for lokale ektstremalpunkt}
\newcommand{\andredertest}{Andrederiverttesten}

% Vectors
\newcommand{\detar}{Arealformler med determinanter}
\newcommand{\avstpunktlin}{Avstand mellom punkt og linje}

%Appendix
\newcommand{\rolle}{Rolles teorem}
\newcommand{\meanval}{Middelverdisetningen}

% Solutions manual
\newcommand{\selos}{Se løsningsforslag.}

\begin{document}
	
\opgt

\op{parlinjeo}
Ei linje går gjennom punktene $ A=(-2, 3, -5) $ og $ B=(-1, 1, -4) $.\os

\textbf{a)}	Finn en parameterframstilling for linja.\os

\textbf{b)} Sjekk om punktene $ C=(-4, 7, -7) $ og $ D=(-3, 5, 4) $ ligger på linja.

\op{krysslinj}
To linjer $ l $ og $ m $ krysser hverandre i et punkt $ A $. Parameteriseringen til linjene er gitt som
\[l: \left\lbrace{
	\begin{array}{l}
	x=-3-2t  \\
	y= 3+ 2t   \\
	z= 1-t 
	\end{array}
}\right. \quad 
m: \left\lbrace{
	\begin{array}{l}
	x=-7 +3s  \\
	y= 5- 2s   \\
	z= s 
	\end{array}
}\right. \]
Finn koordinatene til $ A $.

\op{finnparplan}
Et plan inneholder punktene $ (1, 1,-1) $, $ (-2, -3, -1) $ og $ (5, 6, 1) $. \os

\textbf{a)} Finn en parameterisering for planet.\os

\textbf{b)} Sjekk om punktet $ (-9, 5, 3) $ ligger i planet.


\op{finnparplan2}
Et plan har retningsvektoren $ [2, 1, -5] $ og inneholder linja gitt ved parameteriseringen
\[ l: \left\lbrace{
	\begin{array}{l}
	x=2-4t  \\
	y= -3+ 2t   \\
	z= 5+t 
	\end{array}
}\right. \]
Finn en parameterisering for planet.


\nes
\op{finnplan}
Et plan er utspent av vektorene $ [-4, 2, 0] $ og $ [-3, 0, 3] $ og inneholder punket $ (-2, 2, 1) $. Finn en ligning for planet.
\newpage
\op{finnplan2}
Et plan $ \alpha $ er gitt ved parameteriseringen
\[\alpha: \left\lbrace{
	\begin{array}{l}
	x=-4 + 2s  \\
	y= 2+ 3s + 2t   \\
	z= 1-t 
	\end{array}
}\right. \]
\textbf{a)} Finn to retningsvektorer for planet.\os

\textbf{b)} Finn en ligning for planet.

\op{finnplan3}
Et plan er gitt ved ligningen
\[ 10x-3y-4z=0 \]
\textbf{a)} Sjekk om punktene $ (1, -2, 4) $ og $ (4, -2, 1) $ ligger i planet.\os

\textbf{b)} Finn en parameterframstilling for planet.

\op{finnplan4}
Et plan går gjennom origo og inneholder punktet $ A=(-2, 1, 1) $. For en gitt $ t $ er vektoren $ \vec{u}=[3t, 5, t] $ ortogonal med vektoren mellom origo og $ A $. For dette valget av $ t $ er $ \vec{u} $ også en normalvektor for planet. Finn en ligning for planet.

\op{kuleopg}
Ei kule er gitt ved ligningen
\[ x^2 + 2 x + y^2 - 4 y + z^2 - 12 z + 32 = 0  \]
\textbf{a)} Finn sentrum og radiusen til kula.\os

\textbf{b)} Vis at punktet $ A=(1, 3, 8) $ ligger på kuleflaten.\os

\textbf{c)} Bestem ligningen til tangentplanet til kuleflaten i punktet $ A $. 

\op{kuleopg2}
Ei kule er gitt ved ligningen 
\[ x^2 - 6x + y^2 + 2y + z^2 - 10z-14=0 \]
\textbf{a)} Finn sentrum $ S $ og radiusen $ r $ til kula.\os

\textbf{b)} Sjekk om punktene $ A= (4, 1, 6) $ og $ B=(-6, -4, 1) $ ligger innenfor, utenfor eller på kuleflaten.
\newpage
\nes
\op{avstplinopg} Ei linje $ l $ går gjennom punktene $ (1, 0, -2) $ og $ (2, -2,0) $. Finn avstanden mellom $ l $ og punktet $ (1, -3, 1) $. 

\op{avstplopg}
Et plan er gitt ved ligningen:
\[ -3x+4y+z-7 = 0 \]
Finn avstanden mellom planet og punktet $ (-3,2,3) $.

\op{toparllin} 
To parallelle plan $ \alpha $ og $ \beta $ er henholdsvis gitt ved ligningene 
\[ 3x -2y +z +12 = 0 \]
og 
\[ 3x -2y +z = 0  \]
\textbf{a)} Finn en normalvektor til planene.\os

\textbf{b)} Finn et punkt som ligger i ett av planene. \os

\textsl{Hint}: Velg fritt en verdi for $ x $ og $ y $, og løs resulterende ligning for $ z $.\os

\textbf{c)} Finn avstanden mellom planene.\os

\textbf{d)} Finn en parameterframstilling for ett av planene.
\newpage
\op{kuleopg3}
Når et plan $ \alpha $ skjærer en kuleflate med sentrum $ S $, kan vi alltids studere geometrien fra en slik vinkel at planet ligger rett horisontalt. Et snitt av figuren vil da se slik ut:
\fig{plkul}
Punktet $ A $ er sentrum i sirkelen hvor kuleflaten skjærer planet, og ved formlikhet kan vi vise (prøv selv!) at linjestykket $ AS $ står normalt på $ \alpha $.\os

La $ \alpha $ være gitt ved ligningen 
\[ 2x - y - 2z +1=0   \] 
Dette planet skjærer en kuleflate gitt ved ligningen
\[x^2 - 6x + y^2 + 4y + z^2 -23 = 0 \]
\textbf{a)} Hva er koordinatene til $ S $?\os

\textbf{b)} Finn en parameterframstilling for linja som går gjennom $ A $ og $ S $.\os

\textbf{c)} Finn koordiantene til de to punktene hvor kuleflaten og linja gjennom $ AS $ krysser.\os

\textbf{d)} Hva er koordinatene til $ A $?\os

\textbf{e)} Hvor stor er radiusen til sirkelen hvor $ A $ er sentrum?
\newpage
\grubop{r2h23d1opg4} 
(R2H23D1) \\
Et plan $ \alpha $ er gitt ved likningen
\[ x-2y+2z+1=0 \]
Vi har gitt punktet $ A=(4, 2, 2) $
\abc{
\item Bestem en parameterframstilling for linja gjennom $ A $ som står normalt på planet $ \alpha $.
\item Bestem avstanden fra $ A $ til $ \alpha $.
}

\end{document}