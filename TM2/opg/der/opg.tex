\documentclass[english,hidelinks,pdftex, 11 pt, class=report,crop=false]{standalone}
\usepackage[T1]{fontenc}
\usepackage[utf8]{luainputenc}
\usepackage{lmodern} % load a font with all the characters
\usepackage{geometry}
\geometry{verbose,a4paper, inner=0cm, outer=0 cm, bmargin=2cm, tmargin=1cm}
%\textwidth=12cm
\setlength{\parindent}{0bp}
\usepackage{import}
\usepackage[subpreambles=false]{standalone}
\usepackage{amsmath}
\usepackage{amssymb}
\usepackage{esint}
\usepackage{babel}
\usepackage{tabu}
\usepackage[dvipsnames, table]{xcolor}
\usepackage{cancel}
\makeatother
\makeatletter
\usepackage{datetime2}
\usepackage{titlesec}
\usepackage[many]{tcolorbox}

% Eheter
\newcommand{\enh}[1]{\,\textrm{#1}}
%referances
\newcommand{\net}[2]{{\color{blue}\href{#1}{#2}}}

%Spaces
\newcommand{\vsk}{\\[12pt]}
\newcommand{\vs}{\vspace{-12pt}}

% Tabell for opplegg

\newcommand{\ovlist}[1]{
\vspace{-16pt}
\begin{itemize}
	#1
\end{itemize}
}

% Chapters and sections
\titleformat{\section}[block]{\bfseries}{\hspace{3cm}\thesection}{5pt}{}
\titleformat{\subsection}[block]{\bfseries}{\hspace{3cm}\thesection}{5pt}{}
\newcommand{\sectionbreak}{\clearpage} % New page on each section
 

\newlength{\mywidth}
\setlength{\mywidth}{14cm}

\newcommand{\cont}[1]{
\begin{tcolorbox}[center, boxrule=0.0 mm, width=\mywidth,arc=0mm,enhanced jigsaw,,colback=white,breakable]
#1	
\end{tcolorbox}
}

\newcommand{\info}[5]{
\begin{tcolorbox}[center, boxrule=0.1 mm, width=\mywidth,arc=0mm,enhanced jigsaw,breakable,colback=yellow!5]	
	
	\footnotesize
	\textbf{Øvingsområde}\\[5pt] #1 
	
	\textbf{Utstyr}\\ #2  \\
	
	\begin{tabular}{@{} p{4cm} p{4cm} l} 
		\textbf{Tid} & \textbf{Elevinndeling} & \textbf{Læringsarena} \\
		#3  & #4 & #5
	\end{tabular} 
\end{tcolorbox}	
}

\newcommand{\gjen}[1]{\begin{tcolorbox}[center,boxrule=0.1 mm, width=\mywidth,arc=0mm,colback=blue!3] {\large \textbf{Gjennomføring} \vspace{5 pt}}\newline #1  \end{tcolorbox}\vspace{-5pt}}
\newcommand{\eks}[1]{\begin{tcolorbox}[center,boxrule=0.1 mm, width=\mywidth,arc=0mm,colback=green!3] {\large \textbf{Eksempel} \vspace{5 pt}}\newline #1  \end{tcolorbox}\vspace{-5pt}}

\newcounter{opl}
%\numberwithin{opl}{article}


\newcommand{\opl}[1]{
\newpage
{\refstepcounter{opl} %\phantomsection 
\large \textbf{\theopl \;#1} \vsk}
}

% Headlines
\newcommand{\fork}{\textbf{Forkunnskapar}\\}
\newcommand{\forb}{\textbf{Forberedelsar}\\}
\newcommand{\opgvr}{\textbf{Oppgaver}}



%colors
\newcommand{\colr}[1]{{\color{red} #1}}
\newcommand{\colb}[1]{{\color{blue} #1}}
\newcommand{\colo}[1]{{\color{orange} #1}}
\newcommand{\colc}[1]{{\color{cyan} #1}}
\definecolor{projectgreen}{cmyk}{100,0,100,0}
\newcommand{\colg}[1]{{\color{projectgreen} #1}}

% Lister med bokstavar
\usepackage[inline]{enumitem}
% Opg
\newcommand{\abc}[1]{
	\begin{enumerate}[label=\alph*),leftmargin=18pt]
		#1
	\end{enumerate}
}

\usepackage[]{hyperref}

\begin{document}
	
\eqlen	
\opgt
\setcounter{section}{1}	
\opl{dmhpx}
Deriver $ f $ mhp. $ x $: \br

\begin{tabular}{@{}l l}	
	\textbf{a)} $ f(x)= 3\tan (4x)  $&\quad\textbf{b)} $f(x)=e^{-4x}\cos x$  \vspace{9 pt}\\
	\textbf{c)} $f(x)=-2 \cos x \sin x$
	&\quad \textbf{d)} $f(x)= \sqrt{\tan x}$ \vspace{9 pt}\\
	\textbf{e)} $f(x)= e^{2x} \ln x $ &\quad\textbf{f)} $f(x)= \dfrac{\sin x}{\ln x} $ 
\end{tabular}

\opl{vislncosx}
\textbf{a)} Gitt funksjonen
\[ f(x)=\ln(\cos x) \]
Vis at $ {f'(x)=-\tan x} $.\os
\textbf{b)} Gitt funksjonen
\[ f(x)=-2\cos x \sin x \]
Vis at $ {f'(x)=-2\cos(2x)} $

\nes
\opl{xexpx}
Gitt funksjonen
\[  f(x)=xe^{-x} \quad,\quad x\in[-1, 2]\]
\textbf{a)} Finn ekstremalpunktene til $ f $.\os

\textbf{b)} Finn maksimal- og minimalverdien til $ f $.

\opl{vendepkt}
Gitt funksjonen
\[ f(x)=a\cos(kx+c)+d  \]
Alle punkt hvor $ {f''=0} $ er et infleksjonspunkt (se \hrv{pktpgrf}).\os

\textbf{a)} Forklar hvorfor alle vendepunktene til $ f $
ligger på likevektslinja ${ y= d }$. \os

\textbf{b)} Finn alle infleksjonspunktene til $ f $, uttrykt ved $ k $ og $ c $. \os


\opl{vendepkt2}
Finn infleksjonspunktene og vendepunkene til funksjonen
\[ f(x)=\frac{1}{a+x^2} \]
uttrykt ved $ a $. (Se \hrv{pktpgrf})
%\vspace{- \parskip}\vspace{- \parskip}

\nes
\newpage
\opl{antideropg}
Forklar hvorfor:\os

\textbf{a)} ${F(x)= e^{x^2}+4} $ er en antiderivert av ${f(x)= 2xe^{x^2}} $\os

\textbf{b)} ${F(x)= -\sin x} $ er en antiderivert av ${f(x)= -\cos x} $

\ekspop
Gitt funksjonen
\[ f(x)=e^x(\cos x+\sin x)\quad,\quad x\in\left[\pi, \infty\right]\]
Finn ekstremalpunktene til $ f $ og bruk dette til å forklare at ekstremalverdiene til $ f $ kan uttrykkes som en geometrisk følge.
\end{document}