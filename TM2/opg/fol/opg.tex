\documentclass[english,hidelinks,pdftex, 11 pt, class=report,crop=false]{standalone}
\usepackage[T1]{fontenc}
\usepackage[utf8]{luainputenc}
\usepackage{lmodern} % load a font with all the characters
\usepackage{geometry}
\geometry{verbose,a4paper, inner=0cm, outer=0 cm, bmargin=2cm, tmargin=1cm}
%\textwidth=12cm
\setlength{\parindent}{0bp}
\usepackage{import}
\usepackage[subpreambles=false]{standalone}
\usepackage{amsmath}
\usepackage{amssymb}
\usepackage{esint}
\usepackage{babel}
\usepackage{tabu}
\usepackage[dvipsnames, table]{xcolor}
\usepackage{cancel}
\makeatother
\makeatletter
\usepackage{datetime2}
\usepackage{titlesec}
\usepackage[many]{tcolorbox}

% Eheter
\newcommand{\enh}[1]{\,\textrm{#1}}
%referances
\newcommand{\net}[2]{{\color{blue}\href{#1}{#2}}}

%Spaces
\newcommand{\vsk}{\\[12pt]}
\newcommand{\vs}{\vspace{-12pt}}

% Tabell for opplegg

\newcommand{\ovlist}[1]{
\vspace{-16pt}
\begin{itemize}
	#1
\end{itemize}
}

% Chapters and sections
\titleformat{\section}[block]{\bfseries}{\hspace{3cm}\thesection}{5pt}{}
\titleformat{\subsection}[block]{\bfseries}{\hspace{3cm}\thesection}{5pt}{}
\newcommand{\sectionbreak}{\clearpage} % New page on each section
 

\newlength{\mywidth}
\setlength{\mywidth}{14cm}

\newcommand{\cont}[1]{
\begin{tcolorbox}[center, boxrule=0.0 mm, width=\mywidth,arc=0mm,enhanced jigsaw,,colback=white,breakable]
#1	
\end{tcolorbox}
}

\newcommand{\info}[5]{
\begin{tcolorbox}[center, boxrule=0.1 mm, width=\mywidth,arc=0mm,enhanced jigsaw,breakable,colback=yellow!5]	
	
	\footnotesize
	\textbf{Øvingsområde}\\[5pt] #1 
	
	\textbf{Utstyr}\\ #2  \\
	
	\begin{tabular}{@{} p{4cm} p{4cm} l} 
		\textbf{Tid} & \textbf{Elevinndeling} & \textbf{Læringsarena} \\
		#3  & #4 & #5
	\end{tabular} 
\end{tcolorbox}	
}

\newcommand{\gjen}[1]{\begin{tcolorbox}[center,boxrule=0.1 mm, width=\mywidth,arc=0mm,colback=blue!3] {\large \textbf{Gjennomføring} \vspace{5 pt}}\newline #1  \end{tcolorbox}\vspace{-5pt}}
\newcommand{\eks}[1]{\begin{tcolorbox}[center,boxrule=0.1 mm, width=\mywidth,arc=0mm,colback=green!3] {\large \textbf{Eksempel} \vspace{5 pt}}\newline #1  \end{tcolorbox}\vspace{-5pt}}

\newcounter{opl}
%\numberwithin{opl}{article}


\newcommand{\opl}[1]{
\newpage
{\refstepcounter{opl} %\phantomsection 
\large \textbf{\theopl \;#1} \vsk}
}

% Headlines
\newcommand{\fork}{\textbf{Forkunnskapar}\\}
\newcommand{\forb}{\textbf{Forberedelsar}\\}
\newcommand{\opgvr}{\textbf{Oppgaver}}



%colors
\newcommand{\colr}[1]{{\color{red} #1}}
\newcommand{\colb}[1]{{\color{blue} #1}}
\newcommand{\colo}[1]{{\color{orange} #1}}
\newcommand{\colc}[1]{{\color{cyan} #1}}
\definecolor{projectgreen}{cmyk}{100,0,100,0}
\newcommand{\colg}[1]{{\color{projectgreen} #1}}

% Lister med bokstavar
\usepackage[inline]{enumitem}
% Opg
\newcommand{\abc}[1]{
	\begin{enumerate}[label=\alph*),leftmargin=18pt]
		#1
	\end{enumerate}
}

\usepackage[]{hyperref}

\begin{document}
\opgt
\setcounter{section}{1}	

\op{odpar}
\textbf{a)} Skriv opp de tre første partallene. Lag en rekursiv og en eksplisitt formel for det $ i $-te partallet.\os
	
\textbf{b)} Skriv opp de tre første oddetallene. Lag en eksplisitt formel for det $ i $-te oddetallet. 

\op{eksar}
Finn det eksplisitte uttrykket til  den aritmetiske følgen når du vet at\os
\begin{tabular}{@{}l l}	
	\textbf{a)} $ a_1=3 $ og $ a_4 = 30 $ \os 
	\textbf{b)} $ a_1 = 5 $ og $ a_{11} = -25 $ \os
	\textbf{c)} $ a_3 =14 $ og $ a_5=26 $ 
\end{tabular}\os

\op{eksgeo}
Finn det eksplisitte uttrykket til den geometriske følgen når du vet at\os
\begin{tabular}{@{}l l}	
	\textbf{a)} $ a_1=\frac{1}{2} $ og $ a_2 = \frac{1}{6} $ \os 
	\textbf{b)} $ a_1 = 5 $ og $ a_4 = 40 $
\end{tabular} \os

\begin{comment}
	\op
Bruk formelen fra oppgave \textsl{\ref{sumkvad}a} og det eksplisitte uttrykket for en aritmetisk følge til å forklare at summen av en aritmetisk rekke kan skrives som:
\[na_1 +\frac{dn(n-1)}{2} \]


\op
a) Bruk (\ref{sumg}) og vis at summen $ S_\infty $ når $ n\to\infty $ blir:
\[ S_\infty=a_1\frac{1}{1-k} \]
for $ |k|<1 $.
\end{comment}
\nes

\op{parodd}
\textbf{a)} Bruk figuren under til å forklare at summen $ S_n $ av de $ n $ første naturlige tallene er gitt ved
\[S_n=\frac{n(n+1)}{2}  \]

\begin{figure}
	\centering
	\includegraphics[]{../../fig/sum}
\end{figure}\vs
\textbf{b)} Skriv opp summen av det første, de to første og de tre første  oddetallene. Bruk en lignende figur som i oppgave a) til å vise at summen $ S_n $ av de $ n $ første oddetallene er
\[ S_n = n^2 \] 

\op{opgnegsum}
$ -1-2-3 $ er en rekke. Skriv om rekka slik at den blir uttrykt ved ledd som adderes med hverandre. 

\op{sum10ar}
Finn $ S_{10} $ for rekkene:\os
\begin{tabular}{@{}l l}	
	\textbf{a)} $ 7+13+19+25+\ldots $\quad
	\textbf{b)} $ 1+9+17+25+\ldots $
\end{tabular} \os

\op{ar435}
Gitt rekken 
\[ 8+11+14+\ldots \]
For hvilken $ n $ er summen av rekken lik 435?

\op{opgarekfraeks}
Gitt den uendelige rekken
\[ 3+7+11+... \] 
For hvilken $ n $ er summen av rekken lik 903?

\op{opggerekfraeks}
Gitt den uendelige rekken
\[ 3+6+12+24+... \]	
For hvor mange element er summen av rekken lik 93?

\op{viseks3}
Bruk summen av en aritmetisk rekke til å vise at ligningen gitt i \hyperref[prodind]{\textsl{Eksempel 3}} på s. \pageref{prodind} er sann.

\op{geon}
Gitt rekken
\[ 3+12+48+\ldots+768 \]
Finn summen av rekken. 
\newpage
\op{geoa12}
En geometrisk rekke har $ {a_1 = 2} $ og $ {k=3} $.\os 

\textbf{a)} Vis at summen $ S_n $ kan skrives som:
\[ S_n = 3^n-1 \]
\textbf{b)} Regn ut summen for de tre første leddene.\os

\textbf{c)} For hvilken $ n $ er $ S_n=728 $?



\begin{comment}
\textbf{c)} Hvis du fortsetter å spare slik, og medregner innskudd samme måned, når vil du ha 24200 kr på konto? 
\end{comment}

\op{1over4}
Gitt den uendelige rekken
\[ 4+1+\frac{1}{4}+\ldots \]
\textbf{a)} Forklar hvorfor rekken er konvergent.\os

\textbf{b)} Finn summen av den uendelige rekken.

\op{opggerekuendeks}
Gitt den uendelige rekka 
\[ 1+\frac{1}{x}+\frac{1}{x^2}+.... \]
\abc{
\item  For hvilken $ x $ er summen av rekka lik $ \frac{3}{2} $?\os

\item For hvilken $ x $ er summen av rekka lik $ -1 $?
}

\begin{comment}
	\op{stav} 
	Tenk at uendelig mange personer skal sette sammen en stav. Første person legger på en meter, neste person legger på 0.1 m, neste legger på 0.01 m osv. Hvor lang blir staven?
\end{comment}
\newpage
\op{099er1}
\textbf{a)} Skriv det uendelige desimaltallet 0.999... som en uendelig geometrisk rekke.\os

\textbf{b)} Forklar hvorfor rekken er konvergent og bruk dette faktumet til å finne summen av rekken. 

\op{geokonv}
Gitt den uendelige rekken
\[\frac{1}{3} +\frac{1}{3}(x-2)+ \frac{1}{3}(x-2)^2+\ldots\]
\textbf{a)} For hvilke $ x $ er rekken konvergent? \os

\textbf{b)} For hvilken $ x $ er $ S_n = \frac{2}{9} $?\os

\textbf{c)} For hvilken $ x $ er $ S_n= \frac{1}{6}$? \os

\nes
\op{ind}
Vis ved induksjon at for alle $ n\in \mathbb{N} $ er\\[10pt]
\begin{tabular}{@{}l l}	
	\textbf{a)} $ 1+2+3+\ldots+n = \dfrac{n(n+1)}{2} $ \\[10pt]
	\textbf{b)} $ 1+2 +2^2 +\ldots+ 2^{n-1}= 2^n-1$\\[10pt]
	\textbf{c)} $ 4+4^2+4^3+\ldots+4^n = \dfrac{4}{3}(4^n-1) $ \\[10pt]
	\textbf{d)} $ 1^2 + 2^2+3^2+ \ldots+ n^2 = \dfrac{n(2n+1)(n+1)}{6} $ 
\end{tabular} 

\op{div3}
Vis ved induksjon at $ n(n^2+2) $ er delelig med 3 for alle $ n\in\mathbb{N} $.

\op{factorials}
\textbf{a)}
Vis ved induksjon at:
\[\frac{1\cdot2}{1}\cdot\frac{1\cdot2\cdot3\cdot4}{1\cdot2\cdot3}\cdot
\ldots\cdot\frac{(2n)!}{(2n-1)!}=2^n n! \]
\textsl{Hint}: \y{(2(k+1))!=(2k+1)!(2k+2)}.\os

\textbf{b)} Hvordan kan venstresiden i a) skrives enklere? Utfør induksjonsbeviset på nytt etter forenklingen.
\newpage
\grubop{opgsumnkvad}
Måletl med denne oppgaven er å, uten bruk av induksjon, vise at summen av $ n $ kvadrater er gitt ved følgende formel:
\[ \sum\limits_{i=1}^n i^2 = \frac{n(2n+1)(n+1)}{6} \tag{I}\label{sumkvad}\]
\textbf{a)} Forklar hvorfor vi kan skrive
\[ 1^2 + 2^2 + 3^2+\ldots = 1 + (1+3) + (1+3+5)+ \ldots  \]
\textsl{Hint}: se opg. \ref{parodd} b).\os

\textbf{b)} Ut ifra det du fant i a), forklar at
\[ \sum\limits_{i=1}^n i^2 = n+\sum\limits_{i=1}^n (n-i)(2i+1)  \]
\textbf{c)} Skriv ut alle kjente summer fra b) og løs ligningen med hensyn på $ \sum\limits_{i=1}^n i^2 $, du skal da komme fram til (\ref{sumkvad}).
\end{document}
\begin{comment}
\op{eksplar}
Finn det eksplisitte uttrykket til den aritmetiske rekken når du vet at:
\os
\begin{tabular}{@{}l l}	
\textbf{a)} $ a_1=-5 $ og $ a_8=44 $. \os 
\textbf{b)} $ a_7=39 $ og $ d=6 $. \os
\textbf{c)} $ a_5 = 16 $ og $ a_{10} = 31 $
\end{tabular}
\end{comment}