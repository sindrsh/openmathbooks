\documentclass[english,hidelinks,pdftex, 11 pt, class=report,crop=false]{standalone}
\usepackage[T1]{fontenc}
\usepackage[utf8]{luainputenc}
\usepackage{lmodern} % load a font with all the characters
\usepackage{geometry}
\geometry{verbose,a4paper, inner=0cm, outer=0 cm, bmargin=2cm, tmargin=1cm}
%\textwidth=12cm
\setlength{\parindent}{0bp}
\usepackage{import}
\usepackage[subpreambles=false]{standalone}
\usepackage{amsmath}
\usepackage{amssymb}
\usepackage{esint}
\usepackage{babel}
\usepackage{tabu}
\usepackage[dvipsnames, table]{xcolor}
\usepackage{cancel}
\makeatother
\makeatletter
\usepackage{datetime2}
\usepackage{titlesec}
\usepackage[many]{tcolorbox}

% Eheter
\newcommand{\enh}[1]{\,\textrm{#1}}
%referances
\newcommand{\net}[2]{{\color{blue}\href{#1}{#2}}}

%Spaces
\newcommand{\vsk}{\\[12pt]}
\newcommand{\vs}{\vspace{-12pt}}

% Tabell for opplegg

\newcommand{\ovlist}[1]{
\vspace{-16pt}
\begin{itemize}
	#1
\end{itemize}
}

% Chapters and sections
\titleformat{\section}[block]{\bfseries}{\hspace{3cm}\thesection}{5pt}{}
\titleformat{\subsection}[block]{\bfseries}{\hspace{3cm}\thesection}{5pt}{}
\newcommand{\sectionbreak}{\clearpage} % New page on each section
 

\newlength{\mywidth}
\setlength{\mywidth}{14cm}

\newcommand{\cont}[1]{
\begin{tcolorbox}[center, boxrule=0.0 mm, width=\mywidth,arc=0mm,enhanced jigsaw,,colback=white,breakable]
#1	
\end{tcolorbox}
}

\newcommand{\info}[5]{
\begin{tcolorbox}[center, boxrule=0.1 mm, width=\mywidth,arc=0mm,enhanced jigsaw,breakable,colback=yellow!5]	
	
	\footnotesize
	\textbf{Øvingsområde}\\[5pt] #1 
	
	\textbf{Utstyr}\\ #2  \\
	
	\begin{tabular}{@{} p{4cm} p{4cm} l} 
		\textbf{Tid} & \textbf{Elevinndeling} & \textbf{Læringsarena} \\
		#3  & #4 & #5
	\end{tabular} 
\end{tcolorbox}	
}

\newcommand{\gjen}[1]{\begin{tcolorbox}[center,boxrule=0.1 mm, width=\mywidth,arc=0mm,colback=blue!3] {\large \textbf{Gjennomføring} \vspace{5 pt}}\newline #1  \end{tcolorbox}\vspace{-5pt}}
\newcommand{\eks}[1]{\begin{tcolorbox}[center,boxrule=0.1 mm, width=\mywidth,arc=0mm,colback=green!3] {\large \textbf{Eksempel} \vspace{5 pt}}\newline #1  \end{tcolorbox}\vspace{-5pt}}

\newcounter{opl}
%\numberwithin{opl}{article}


\newcommand{\opl}[1]{
\newpage
{\refstepcounter{opl} %\phantomsection 
\large \textbf{\theopl \;#1} \vsk}
}

% Headlines
\newcommand{\fork}{\textbf{Forkunnskapar}\\}
\newcommand{\forb}{\textbf{Forberedelsar}\\}
\newcommand{\opgvr}{\textbf{Oppgaver}}



%colors
\newcommand{\colr}[1]{{\color{red} #1}}
\newcommand{\colb}[1]{{\color{blue} #1}}
\newcommand{\colo}[1]{{\color{orange} #1}}
\newcommand{\colc}[1]{{\color{cyan} #1}}
\definecolor{projectgreen}{cmyk}{100,0,100,0}
\newcommand{\colg}[1]{{\color{projectgreen} #1}}

% Lister med bokstavar
\usepackage[inline]{enumitem}
% Opg
\newcommand{\abc}[1]{
	\begin{enumerate}[label=\alph*),leftmargin=18pt]
		#1
	\end{enumerate}
}

\usepackage[]{hyperref}

\begin{document}
\opgt

\nes
	\opl{kjernederye}
	Vis at vi for enhver funksjon $ f(x) $ har at
	\[ \left(y(x) e^{F(x)}\right)' = y'(x)e^{F(x)}+f(x)y(x)e^{F(x)} \]
	hvor $ F $ er en antiderivert til $ f $.
	
	\opl{FODE}
	Løs ligningen:\os
	
	\begin{tabular}{@{}l l }
		\textbf{a)} $ y'+4y=8 $	\quad
		&\textbf{b)} $\displaystyle y' + \frac{1}{x}y - \cos x=0$  \\[11 pt]
		\textbf{c)} $\displaystyle y' + \frac{3}{x}y = 6x +2 $		
		&\textbf{d)} $\displaystyle y'+3x^2 y =(1+3x^2)e^x$ 
	\end{tabular}

\nes	
\opl{sepode}
Finn den generelle løsningen av ligningen:\os

\begin{tabular}{@{}l l l}	
	\textbf{a)} $\displaystyle y'=ye^x \cos x$\quad	
	\textbf{b)} $\displaystyle y'=\frac{1}{y}3x^2(y^2+1) $ 	
\end{tabular}

\opl{sepode2}
Løs ligningen:\os

\textbf{a)} $\displaystyle xy'-y=2x^2y\quad,\quad y(1)=1$\os
\textbf{b)} $2\sqrt{x}y'=\cos^2 y\quad,\quad y(4)=\frac{\pi}{4}$ 
\newpage
\nes
\opl{retnopg}
Figuren under viser et retningsdiagram for en differensialligning.
\begin{figure}
	\centering
	\includegraphics[scale=0.85]{retn}
\end{figure}
\textbf{a)} Skisser integralkurven som går gjennom punket $ (0, 4) $ og integralkurven som går gjennom punktet $ (0, -4) $.\os

\textbf{b)} Bruk figuren til å anslå stigningstallet til alle integralkurvene for $ x=0 $.\vsk

Integralkurvene er løsninger av ligningen $ y' + 4x y = 3x$. 
\os
\textbf{c)} Bruk ligningen til å verifisere anslaget fra oppgave b).\os

\textbf{b)} Finn stigningstallet i punktene $ (-3, 5) $ og $ (4, 2) $.

\nes
\opl{SODEopg}
Finn den generelle løsningen av ligningen:\os

\textbf{a)} $ y''-y'-2y =0 $\os

\textbf{b)} $ 2y''-12y'+18y=0 $\os

\textbf{c)} $ y''-4y'+13y=0 $

\opl{speslos}
Finn løsningen av differensialligningen:\os

\textbf{a)} $ y''-2y'-15y=0\quad,\quad y(0)=-1,\,y'(0)=2 $\os

\textbf{b)} $ y''+10y'+25y=0\quad,\quad y(0)=2,\,y'(0)=1  $\os

\textbf{c)} $ y''-2y'+5y=0\quad,\quad y(0)=1,\,y'(0)=1  $

\nes
\opl{folketall}
I året 2015 var tallet på en populasjon 100 millioner. Fra og med dette året er det forventet at folkeveksten vil være proporsjonal med folketallet.\os

\textbf{a)} Sett opp en differensialligning som beskriver situasjonen over.\os

\textbf{b)} I 2016 var folketallet 101 millioner. Bruk denne informasjonen til å finne uttrykket $ y(t) $ som gir folketallet (i millioner) $ t $ år etter 2015.

\opl{newtavkj}
En gjenstand med temperaturen $ T$ befinner seg i et rom med temperaturen $ T_r $. Det antas at temperaturen til gjenstanden er likt fordelt hele tiden og at romtemperaturen ikke blir påvirket av gjenstandens temperatur. Når $ T$ er høyere enn $ T_r $ kan vi bruke \textit{Newtons avkjølingslov} for å tilnærme hvordan $ T $ vil utvikle seg med tiden $ t $:
\[ T'=-k(T-T_r) \]
$ k $ er en konstant som må bestemmes ut ifra gjenstandens termodynamiske egenskaper. \os

\textbf{a)} Finn den generelle løsningen av ligningen over.

En gjenstand med temperaturen 95\,$ ^\circ $C blir plassert i et rom med temperaturen 15\,$ ^\circ $C. Vi bruker Newtons avkjølingslov til å anslå gjenstandens temperatur $ T $ etter $ t $ minutter. $ k $ har verdien $ \frac{\ln 2}{5} $. \os

\textbf{b)} Finn et uttrykk for $ T $.\os

\textbf{c)} Bestem $ T(15) $.\os

\textbf{d)} Hva skjer når vi lar tiden gå mot uendelig?

\opl{fjormassopg}
Gitt en funksjon $ y(t) $ på formen
\[y(t) =a\cos (\omega t) +b\sin (\omega t) \] 
hvor $ a $, $ b $ og $ \omega $ er konstanter (når $ t $ er en tidsvariabel, kaller vi gjerne $ \omega $ for \textit{vinkelfrekvensen}\index{vinkel!-frekvens}). Vis at vi for alle fjør-masse systemer uten demping har at
\[ \omega=\sqrt{\frac{k}{m}} \] 

\opl{demfjoropg}
En klosse med masse $ {m=1} $ henger vertikalt i en fjør med fjørkonstant $ {k=25} $. Klossen strekkes slik at den forflyttes en lengde $ 0.5 $ fra likevektspunktet $ {y=0} $, og blir etterpå sluppet. La $ y(t) $ være forflytningen relativt til likevektspunktet tiden $ t $ etter at bevegelsen har startet.\os

\textbf{a)} Finn et uttrykk for $ y $.\os

\textbf{b)} Finn perioden til $ y $.

Klossen og fjøren blir plassert i en omgivelse der dempingskonstanten er funnet å være $ {q=6} $. Ved et tidspunkt satt til $ {t=0} $ passerer klossen likevektspunktet med en fart $ y'(0)=4 $.\os

\textbf{c)} Finn det nye uttrykket for $ y $. 

\ekspop
\textbf{a)} Vis at vi kan omskrive et fjør-masse system med demping til ligningen
\[ y''+2\alpha y'+\omega^2y=0 \tag{I}\label{kritisk} \]
hvor $ 2\alpha=\frac{b}{m} $ og $ \omega=\sqrt{\frac{k}{m}} $.\os

\textbf{b)} Vis at løsningen av den karakteristiske ligningen av (\ref{kritisk}) kan skrives som
\[ r=-\alpha\pm\sqrt{\alpha^2-\omega^2} \]
\textbf{c)} For de tre tilfellene $ \alpha>\omega $, $ \alpha=\omega $ og $ \alpha<\omega $, hvilket uttrykk antar løsningen av (\ref{kritisk})? \os

\textsl{Hint}: Se (\ref{tore})-(\ref{kompleks}).\os%(Oppgaven fortsetter på neste side).

\textbf{d)} La $ y(t) $ være løsningen av (\ref{kritisk}). Forklar hvorfor $ \lim\limits_{t\to\infty}y(t) =0 $. (Ta det for gitt at $ \lim\limits_{t\to \infty} te^{-at}=0 $ når $ a>0 $).
\end{document}