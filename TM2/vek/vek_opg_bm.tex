\documentclass[english,hidelinks,pdftex, 11 pt, class=report,crop=false]{standalone}
\usepackage[T1]{fontenc}
\usepackage[utf8]{luainputenc}
\usepackage{lmodern} % load a font with all the characters
\usepackage{geometry}
\geometry{verbose,a4paper, inner=0cm, outer=0 cm, bmargin=2cm, tmargin=1cm}
%\textwidth=12cm
\setlength{\parindent}{0bp}
\usepackage{import}
\usepackage[subpreambles=false]{standalone}
\usepackage{amsmath}
\usepackage{amssymb}
\usepackage{esint}
\usepackage{babel}
\usepackage{tabu}
\usepackage[dvipsnames, table]{xcolor}
\usepackage{cancel}
\makeatother
\makeatletter
\usepackage{datetime2}
\usepackage{titlesec}
\usepackage[many]{tcolorbox}

% Eheter
\newcommand{\enh}[1]{\,\textrm{#1}}
%referances
\newcommand{\net}[2]{{\color{blue}\href{#1}{#2}}}

%Spaces
\newcommand{\vsk}{\\[12pt]}
\newcommand{\vs}{\vspace{-12pt}}

% Tabell for opplegg

\newcommand{\ovlist}[1]{
\vspace{-16pt}
\begin{itemize}
	#1
\end{itemize}
}

% Chapters and sections
\titleformat{\section}[block]{\bfseries}{\hspace{3cm}\thesection}{5pt}{}
\titleformat{\subsection}[block]{\bfseries}{\hspace{3cm}\thesection}{5pt}{}
\newcommand{\sectionbreak}{\clearpage} % New page on each section
 

\newlength{\mywidth}
\setlength{\mywidth}{14cm}

\newcommand{\cont}[1]{
\begin{tcolorbox}[center, boxrule=0.0 mm, width=\mywidth,arc=0mm,enhanced jigsaw,,colback=white,breakable]
#1	
\end{tcolorbox}
}

\newcommand{\info}[5]{
\begin{tcolorbox}[center, boxrule=0.1 mm, width=\mywidth,arc=0mm,enhanced jigsaw,breakable,colback=yellow!5]	
	
	\footnotesize
	\textbf{Øvingsområde}\\[5pt] #1 
	
	\textbf{Utstyr}\\ #2  \\
	
	\begin{tabular}{@{} p{4cm} p{4cm} l} 
		\textbf{Tid} & \textbf{Elevinndeling} & \textbf{Læringsarena} \\
		#3  & #4 & #5
	\end{tabular} 
\end{tcolorbox}	
}

\newcommand{\gjen}[1]{\begin{tcolorbox}[center,boxrule=0.1 mm, width=\mywidth,arc=0mm,colback=blue!3] {\large \textbf{Gjennomføring} \vspace{5 pt}}\newline #1  \end{tcolorbox}\vspace{-5pt}}
\newcommand{\eks}[1]{\begin{tcolorbox}[center,boxrule=0.1 mm, width=\mywidth,arc=0mm,colback=green!3] {\large \textbf{Eksempel} \vspace{5 pt}}\newline #1  \end{tcolorbox}\vspace{-5pt}}

\newcounter{opl}
%\numberwithin{opl}{article}


\newcommand{\opl}[1]{
\newpage
{\refstepcounter{opl} %\phantomsection 
\large \textbf{\theopl \;#1} \vsk}
}

% Headlines
\newcommand{\fork}{\textbf{Forkunnskapar}\\}
\newcommand{\forb}{\textbf{Forberedelsar}\\}
\newcommand{\opgvr}{\textbf{Oppgaver}}



%colors
\newcommand{\colr}[1]{{\color{red} #1}}
\newcommand{\colb}[1]{{\color{blue} #1}}
\newcommand{\colo}[1]{{\color{orange} #1}}
\newcommand{\colc}[1]{{\color{cyan} #1}}
\definecolor{projectgreen}{cmyk}{100,0,100,0}
\newcommand{\colg}[1]{{\color{projectgreen} #1}}

% Lister med bokstavar
\usepackage[inline]{enumitem}
% Opg
\newcommand{\abc}[1]{
	\begin{enumerate}[label=\alph*),leftmargin=18pt]
		#1
	\end{enumerate}
}

\usepackage[]{hyperref}

\newcommand{\note}{Merk}
\newcommand{\notesm}[1]{{\footnotesize \textsl{\note:} #1}}
\newcommand{\ekstitle}{Eksempel }
\newcommand{\sprtitle}{Språkboksen}
\newcommand{\expl}{forklaring}
\newcommand{\pyt}{Pytagoras' setning}
\newcommand\sv{\vsk \textbf{Svar} \vspace{4 pt}\\}

%references
\newcommand{\reftab}[1]{\hrs{#1}{tabell}}
\newcommand{\rref}[1]{\hrs{#1}{regel}}
\newcommand{\dref}[1]{\hrs{#1}{definisjon}}
\newcommand{\refkap}[1]{\hrs{#1}{kapittel}}
\newcommand{\refsec}[1]{\hrs{#1}{seksjon}}
\newcommand{\refdsec}[1]{\hrs{#1}{delseksjon}}
\newcommand{\refved}[1]{\hrs{#1}{vedlegg}}
\newcommand{\eksref}[1]{\textsl{#1}}
\newcommand\fref[2][]{\hyperref[#2]{\textsl{figur \ref*{#2}#1}}}
\newcommand{\refop}[1]{{\color{blue}Oppgave \ref{#1}}}
\newcommand{\refops}[1]{{\color{blue}oppgave \ref{#1}}}


%Algebra
\newcommand{\kvadset}{Kvadratsetningene}
\newcommand{\aenato}{Sum-produkt-metoden}

% Geometry
\newcommand{\hlikb}{Midtnormalen i en likebeint trekant}
\newcommand{\arealsetn}{Arealsetningen}
\newcommand{\trkmedian}{Median}
\newcommand{\midtrk}{Midtnormal (i trekant)}
\newcommand{\innskrsirk}{Innskrevet sirkel}
\newcommand{\cossetn}{Cosinussetningen}
\newcommand{\perfvink}{Sentral- og periferivinkel}
\newcommand{\tang}{Tangent}

% Derivative
\newcommand{\derel}{Den deriverte av elementære funksjoner}
\newcommand{\divder}{Divisjonsregelen}
\newcommand{\kjernereg}{Kjerneregelen}
\newcommand{\prodregder}{Produktregelen}
\newcommand{\lhop}{L'Hopitals regel}

% Funksjonsdrofting
\newcommand{\monder}{Monotoniegenskaper og den deriverte}
\newcommand{\fderekstr}{$ \bm{f'=0} $ for lokale ektstremalpunkt}
\newcommand{\andredertest}{Andrederiverttesten}

% Vectors
\newcommand{\detar}{Arealformler med determinanter}
\newcommand{\avstpunktlin}{Avstand mellom punkt og linje}

%Appendix
\newcommand{\rolle}{Rolles teorem}
\newcommand{\meanval}{Middelverdisetningen}

% Solutions manual
\newcommand{\selos}{Se løsningsforslag.}
\begin{document}

\opgt

\setcounter{section}{1}	
\op{leno}
Finn lengden av vektorene:\os
\begin{tabular}{@{}l l}
\textbf{a)} $ [-2, 1, 5] $ & \quad\textbf{b)} $ [\sqrt{3}, 2,  \sqrt{2}] $
\end{tabular}	
	
\op{closest}
Hvilket av punktene $ {B=(3, -2, 1)} $ og $ {C=(0, 5, 6) }$ ligger nærmest punktet $ {A=(1, -1, -2)} $?

\op{lenfakt}
Gitt vektoren
\[ \vec{u}=[ad, bd, cd]\]
\textbf{a)} Vis at
\[ |\vec{u}|=d\sqrt{a^2 + b^2 + c^2} \]
når $ d>0 $.\os

\textbf{b)} Forklar at
\[ |\vec{u}|=|d|\sqrt{a^2 + b^2 + c^2} \]
når $ d<0 $.

\nes
\op{skalfaktor}
Gitt vektorene
\[ \vec{u}=[ad, bd, cd] \text{ og } \vec{v}=[eh, fh, gh] \]
Vis at
\[ \vec{u}\cdot\vec{v}=dh(ae+bf+cg) \] \vs

\op{skalproo}
Finn skalarproduktet av vektorene:\os

\textbf{a)} $ \vec{a}=[2, 4, 6] $ og $ \vec{b}=[-5, 0, -1] $\os 

\textbf{b)} $ \vec{a}=[-9, 1, 5] $ og $\vec{b}= [-2, 1, -2] $\os

\textbf{c)} $ \vec{a}=\left[\frac{1}{5}, \frac{3}{5},- \frac{1}{5}\right] $ og $ \vec{b}=[512, -128, 64] $. \textsl{Tips:} Bruk resultatet fra opg. \ref{skalfaktor}.
\newpage
\op{skalpro2o}
Finn skalarproduktet av $ \vec{a} $ og $ \vec{b} $, som utspenner vinkelen $ \theta $, når du vet at\os

\textbf{a)} $ |\vec{a}|=5 $, $ |\vec{b}|= 2$ og $ \theta = 60^\circ $\os

\textbf{b)} $ |\vec{a}|=5 $, $ |\vec{b}|= 2$ og $ \theta = 150^\circ $

\op{finntheta} 
Finn vinkelen mellom $ \vec{a} $ og $ \vec{b} $ når\os

\textbf{a)} $ \vec{a}=[ 5 ,-5,  2]$ og $ \vec{b}=[ 3 ,-4 , 5] $\os

\textbf{b)} $ \vec{a}=[ 2 ,-1,  -3]$ og $ \vec{b}=[ -1 ,-3 , -2] $\os

\textbf{c)} $ \vec{a}=[ -1 ,-2,  2]$ og $ \vec{b}=[ -3 , 5 , -4] $

\op{forkort}
Forkort uttrykkene når du vet at $ |\vec{a}|=1 $, $ |\vec{b}|=2 $, $ |\vec{c}|=5 $, $ \vec{a}\cdot\vec{b}=0 $ og $ \vec{b}\cdot\vec{c}=0 $.\os

\textbf{a)} $ \vec{b}\cdot(\vec{a}+\vec{c}) + 3(\vec{a}+\vec{b})^2 $ \os

\textbf{b)} $ (\vec{a}+ \vec{b}+\vec{c})^2$

\nes
\op{orto}
Sjekk om $ \vec{a} $ og $ \vec{b} $ er ortogonale når\os

\textbf{a)} $ \vec{a}=[2, 4, -2] $ og $ \vec{b}=[3, 1, 1] $\os

\textbf{b)} $ \vec{a}=[-18, 12, 9] $ og $ \vec{b}=[1, -2, 1] $\os

\textbf{c)} $ \vec{a}=[5, 5, -1] $ og $ \vec{b}=[5, -4, 5] $

\op{torto}
Gitt vektoren
\[ \vec{u}=[-5, -1, 6] \]
Finn $ t $ slik at $ \vec{u}\perp \vec{v} $ når\os

\textbf{a)} $ \vec{v}=[t, 3t, 2] $\os

\textbf{b)} $ \vec{v}=[t, t^2, 1] $

\op{sjekkpar}
Sjekk om $ \vec{a}\parallel\vec{b} $ når\os

\textbf{a)} $ \vec{a}=[8, 4, -2] $ og $ \vec{b}=[4, 2, 4] $\os

\textbf{b)} $ \vec{a}=[-3, 5, 2] $ og $ \vec{b}=\left[-\frac{9}{7}, \frac{15}{7}, \frac{6}{7}\right] $ 
\newpage
\op{finntpar}
Gitt vektoren 
\[ \vec{a}=[-3, 1, 8] \]
Om mulig, finn $ t $ slik at $ \vec{a}\parallel\vec{b} $ når\os

\textbf{a)} $ \vec{b}=[t+3, 1-t, -16] $\os

\textbf{b)} $ \vec{b}=[t^2+2, t, -(5t^2+3)] $

\op{finnsogt}
Finn $ s $ og $ t $ slik at $\vec{u}=[4, 6+s, -(s+t)] $ og $ \vec{v}=\left[\frac{12}{7}, \frac{2t-9s}{7}, \frac{3s-t}{7}\right] $ er parallelle. \os

\nes
\op{vis22det}
Vis at
\[  \left|\begin{matrix}
ae & be \\
cf & df
\end{matrix}\right|=ef\left|\begin{matrix}
a & b \\
c & d
\end{matrix}\right| \]

\begin{comment}
\op{arparo}
Finn aralet til parallellogrammet utspent av (\textsl{Tips:} Bruk resultatet fra opg. \ref{vis22det}):

\textbf{a)} $ [-2, 7] $ og $ [12, 8] $

\textbf{b)} $ [-2, 4] $ og $ [24, -16] $\\
\end{comment}

\op{abparvekpro0}
Vis at hvis $ \vec{u}||\vec{v} $, så er $ \vec{u}\times\vec{v}=0 $

\op{lagrangesid}
For to vektorer $ \vec{u} $ og $ \vec{v} $ er \textit{Lagranges identitet} gitt som
\[ |\vec{u}\times\vec{v}|^2=|\vec{u}|^2|\vec{v}|^2-(\vec{u}\cdot\vec{v}\,)^2 \]
Bruk identiteten og definisjonen av skalarproduktet til å vise at
\[ |\vec{u}\times\vec{v}|=|\vec{u}||\vec{v}|\sin \angle(\vec{u}, \vec{v})  \]\vs

\op{tetraro}
Et tetraeted er utspent av vektorene $ \vec{a}=[2, -2, 1],\; \vec{b}=[3, -3, 1] $ og $ \vec{c}=[2, -3, 2] $, hvor $ \vec{a} $ og $ \vec{b} $ utspenner grunnflaten.\os

\textbf{a)} Vis at arealet av grunnflaten er $ \sqrt{2} $.\os

\textbf{b)} Vis at volumet av tetraetedet er $ \frac{1}{6} $.
\newpage

\op{opgvekgrub1a}
Løs \refgrubs{r2v23d1opg3} a) ved å
\abc{
\item bruke likning \eqref{volavpyr}
\item bruke den klassiske formelen for volumet til en pyramide (se \mb).
}

\op{parfinnh}
Et parallellepidet er utspent av vektorene $ \vec{a},\; \vec{b} $ og $ \vec{c} $. Vi har at ${|a|=3}$,  $|{\vec{b}|=4}$, $ {\vec{a}\cdot \vec{b}=0}  $, og at grunnflaten er utspent av $ \vec{a} $ og $ \vec{b} $. \os

\textbf{a)} Finn lengden av diagonalen til grunnflaten.\os

La $ \theta $ være vinkelen mellom $ {\vec{a}\times\vec{b}} $ og $ \vec{c} $ og la $ {\theta\in[0^\circ, 90^\circ]} $.\os

\textbf{b)} Lag en tegning og forklar hvorfor høyden $ h $ i parallellepipedet er gitt som
\[ h= |\vec{c}|\cos \theta \]
\textbf{d)} Forklar hvorfor volumet $ V $ av parallellepidetet kan skrives som
\[ V= |\vec{a}\times\vec{b}||c|\cos \theta\]\vs

\op{kryserlikdet}
Gitt vektorene $ \vec{u}=[a, b, c] $, $ \vec{v}=[d, e, f] $ og $ \vec{w}=[g, h, i] $. Vis at
\[ \vec{u}\times\vec{v}\cdot\vec{w}= \vec{w}\times\vec{u}\cdot\vec{v}\]
Tre pyramider er utspent av vektorene $ \vec{u}=[a, b, c] $, $ \vec{v}=[d, e, f] $ og $ \vec{w}=[g, h, i] $. Grunnflatene til pyramidene er henholdsvis utspent av $ \vec{u} $ og $ \vec{v} $, $ \vec{u} $ og $ \vec{w} $ og $ \vec{v} $ og $ \vec{w} $. Hva er uttrykket til volumet av pyramidene?

\newpage
\grubop{r2v23d1opg3}
(R2V23D1)\os
Punktene $ A=(0,0,0) $, $ B=(0,5,0) $, $ C=(4,2,0) $ og $ T=(0,0,5) $ danner en pyramide, slik figuren viser.
\fig{r2v23d1opg3}
\abc{
\item Regn ut volumet til pyramiden.
\item Regn ut arealet til $ \triangle BCT $.
\item Bestem avstanden fra punktet $ A $ til planet som går gjennom $ B $, $ C $ og $ T $.
}

\end{document}