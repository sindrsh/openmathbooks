\documentclass[english,hidelinks,pdftex, 11 pt, class=report,crop=false]{standalone}
\usepackage[T1]{fontenc}
\usepackage[utf8]{luainputenc}
\usepackage{lmodern} % load a font with all the characters
\usepackage{geometry}
\geometry{verbose,a4paper, inner=0cm, outer=0 cm, bmargin=2cm, tmargin=1cm}
%\textwidth=12cm
\setlength{\parindent}{0bp}
\usepackage{import}
\usepackage[subpreambles=false]{standalone}
\usepackage{amsmath}
\usepackage{amssymb}
\usepackage{esint}
\usepackage{babel}
\usepackage{tabu}
\usepackage[dvipsnames, table]{xcolor}
\usepackage{cancel}
\makeatother
\makeatletter
\usepackage{datetime2}
\usepackage{titlesec}
\usepackage[many]{tcolorbox}

% Eheter
\newcommand{\enh}[1]{\,\textrm{#1}}
%referances
\newcommand{\net}[2]{{\color{blue}\href{#1}{#2}}}

%Spaces
\newcommand{\vsk}{\\[12pt]}
\newcommand{\vs}{\vspace{-12pt}}

% Tabell for opplegg

\newcommand{\ovlist}[1]{
\vspace{-16pt}
\begin{itemize}
	#1
\end{itemize}
}

% Chapters and sections
\titleformat{\section}[block]{\bfseries}{\hspace{3cm}\thesection}{5pt}{}
\titleformat{\subsection}[block]{\bfseries}{\hspace{3cm}\thesection}{5pt}{}
\newcommand{\sectionbreak}{\clearpage} % New page on each section
 

\newlength{\mywidth}
\setlength{\mywidth}{14cm}

\newcommand{\cont}[1]{
\begin{tcolorbox}[center, boxrule=0.0 mm, width=\mywidth,arc=0mm,enhanced jigsaw,,colback=white,breakable]
#1	
\end{tcolorbox}
}

\newcommand{\info}[5]{
\begin{tcolorbox}[center, boxrule=0.1 mm, width=\mywidth,arc=0mm,enhanced jigsaw,breakable,colback=yellow!5]	
	
	\footnotesize
	\textbf{Øvingsområde}\\[5pt] #1 
	
	\textbf{Utstyr}\\ #2  \\
	
	\begin{tabular}{@{} p{4cm} p{4cm} l} 
		\textbf{Tid} & \textbf{Elevinndeling} & \textbf{Læringsarena} \\
		#3  & #4 & #5
	\end{tabular} 
\end{tcolorbox}	
}

\newcommand{\gjen}[1]{\begin{tcolorbox}[center,boxrule=0.1 mm, width=\mywidth,arc=0mm,colback=blue!3] {\large \textbf{Gjennomføring} \vspace{5 pt}}\newline #1  \end{tcolorbox}\vspace{-5pt}}
\newcommand{\eks}[1]{\begin{tcolorbox}[center,boxrule=0.1 mm, width=\mywidth,arc=0mm,colback=green!3] {\large \textbf{Eksempel} \vspace{5 pt}}\newline #1  \end{tcolorbox}\vspace{-5pt}}

\newcounter{opl}
%\numberwithin{opl}{article}


\newcommand{\opl}[1]{
\newpage
{\refstepcounter{opl} %\phantomsection 
\large \textbf{\theopl \;#1} \vsk}
}

% Headlines
\newcommand{\fork}{\textbf{Forkunnskapar}\\}
\newcommand{\forb}{\textbf{Forberedelsar}\\}
\newcommand{\opgvr}{\textbf{Oppgaver}}



%colors
\newcommand{\colr}[1]{{\color{red} #1}}
\newcommand{\colb}[1]{{\color{blue} #1}}
\newcommand{\colo}[1]{{\color{orange} #1}}
\newcommand{\colc}[1]{{\color{cyan} #1}}
\definecolor{projectgreen}{cmyk}{100,0,100,0}
\newcommand{\colg}[1]{{\color{projectgreen} #1}}

% Lister med bokstavar
\usepackage[inline]{enumitem}
% Opg
\newcommand{\abc}[1]{
	\begin{enumerate}[label=\alph*),leftmargin=18pt]
		#1
	\end{enumerate}
}

\usepackage[]{hyperref}

\newcommand{\note}{Merk}
\newcommand{\notesm}[1]{{\footnotesize \textsl{\note:} #1}}
\newcommand{\ekstitle}{Eksempel }
\newcommand{\sprtitle}{Språkboksen}
\newcommand{\expl}{forklaring}
\newcommand{\pyt}{Pytagoras' setning}
\newcommand\sv{\vsk \textbf{Svar} \vspace{4 pt}\\}

%references
\newcommand{\reftab}[1]{\hrs{#1}{tabell}}
\newcommand{\rref}[1]{\hrs{#1}{regel}}
\newcommand{\dref}[1]{\hrs{#1}{definisjon}}
\newcommand{\refkap}[1]{\hrs{#1}{kapittel}}
\newcommand{\refsec}[1]{\hrs{#1}{seksjon}}
\newcommand{\refdsec}[1]{\hrs{#1}{delseksjon}}
\newcommand{\refved}[1]{\hrs{#1}{vedlegg}}
\newcommand{\eksref}[1]{\textsl{#1}}
\newcommand\fref[2][]{\hyperref[#2]{\textsl{figur \ref*{#2}#1}}}
\newcommand{\refop}[1]{{\color{blue}Oppgave \ref{#1}}}
\newcommand{\refops}[1]{{\color{blue}oppgave \ref{#1}}}


%Algebra
\newcommand{\kvadset}{Kvadratsetningene}
\newcommand{\aenato}{Sum-produkt-metoden}

% Geometry
\newcommand{\hlikb}{Midtnormalen i en likebeint trekant}
\newcommand{\arealsetn}{Arealsetningen}
\newcommand{\trkmedian}{Median}
\newcommand{\midtrk}{Midtnormal (i trekant)}
\newcommand{\innskrsirk}{Innskrevet sirkel}
\newcommand{\cossetn}{Cosinussetningen}
\newcommand{\perfvink}{Sentral- og periferivinkel}
\newcommand{\tang}{Tangent}

% Derivative
\newcommand{\derel}{Den deriverte av elementære funksjoner}
\newcommand{\divder}{Divisjonsregelen}
\newcommand{\kjernereg}{Kjerneregelen}
\newcommand{\prodregder}{Produktregelen}
\newcommand{\lhop}{L'Hopitals regel}

% Funksjonsdrofting
\newcommand{\monder}{Monotoniegenskaper og den deriverte}
\newcommand{\fderekstr}{$ \bm{f'=0} $ for lokale ektstremalpunkt}
\newcommand{\andredertest}{Andrederiverttesten}

% Vectors
\newcommand{\detar}{Arealformler med determinanter}
\newcommand{\avstpunktlin}{Avstand mellom punkt og linje}

%Appendix
\newcommand{\rolle}{Rolles teorem}
\newcommand{\meanval}{Middelverdisetningen}

% Solutions manual
\newcommand{\selos}{Se løsningsforslag.}

\begin{document}


\opr{closest}
\algv{\vv{AB} &= [3-1 , -2-(-1) , 1-(-2)]\\
&= [2, 1, 3] \\
\left|\vv{AB}\right|&= \sqrt{14}\\
&\\
\vv{AC} &= [0-1 , 5-(-1) , 6-(-2)]\\
&= [-1, 6, 8] \\
\left|\vv{AC}\right|&= \sqrt{101}
}
Siden $ {\sqrt{101}>\sqrt{14}} $ er $ B $ nærmest $ A $.\vsk

\opr{lenfakt}\\
\textbf{a)}
\algv{
	|\vec{u}|&=\sqrt{(ad)^2 + (bd)^2 + (bd)^2}\\
	&= \sqrt{a^2 d^2 + b^2 d^2 + c^2 d^2}\\
	&= \sqrt{d^2(a^2+b^2+c^2)}\\
	&= d\sqrt{a^2 + b^2 + c^2}
}
\textbf{b)} Som i opg. a) kan vi også her skrive
\[ |\vec{u}|=\sqrt{d^2(a^2+b^2+c^2)}\]
men siden $ d^2 $ er et positivt tall, mens $ d $ er negativ, har vi at:
\[ d\neq \sqrt{d^2} \]
istedenfor er:
\[ |d|= \sqrt{d^2} \]
derfor kan vi skrive:
\[ |\vec{u}|=|d|\sqrt{a^2 + b^2 + c^2} \]
\opr{skalfaktor}
\algv{
\vec{u}\cdot\vec{v} &= [ad, bd, cd]\cdot[eh, fh, gh]\\
&= adeh+bdfh+cdgh \\
&= dh(ae+ bf+ cg)
}

\opr{skalproo}\\
\textbf{c)}
\algv{
\vec{a}\cdot\vec{b}&= \left[\dfrac{1}{5}, \dfrac{3}{5}, \dfrac{1}{5}\right]\cdot\vec{b}=[512, -128, 64] \br
&= \frac{1}{5}[1, 3, -1]\cdot 64[8, -2, 1]\br
&= \frac{64}{5}(8-6-1) \br
&= \frac{64}{5}
}


\opr{finntheta} 
Finn vinkelen mellom $ \vec{a} $ og $ \vec{b} $ når:

\textbf{a)} $ \vec{a}=[ 5 ,-5,  2]$ og $ \vec{b}=[ 3 ,-4 , 5] $
\algv{
|\vec{a}| &= \sqrt{5^2 +(-5)^2 +2^2} \\
&= \sqrt{54} \\
&= \sqrt{9\cdot6} \\
&= 3\sqrt 6\\
& \\
|\vec{b}| &= \sqrt{3^2+(-4)^2 + 5^2} \\
&= \sqrt{50}\\
&= \sqrt{25\cdot2}\\
&= 5\sqrt{2}
}
\algv{
	\vec{a}\cdot\vec{b} &= [ 5 ,-5,  2]\cdot\vec{b}=[ 3 ,-4 , 5] \\
	&=15+20+10\\
&= 45 }
\algv{\cos \theta &= \frac{\vec{a}\cdot\vec{b}}{|\vec{a}||\vec{b}|}\br
&= \frac{45}{3\sqrt{6}\cdot5\sqrt{2}} \br
&= \frac{3}{2\sqrt{3}}\br
&= \frac{3\sqrt{3}}{2\sqrt{3}\sqrt{3}} \br
&= \frac{\sqrt{3}}{2}
}
Dette betyr at $ \theta=30^\circ $.

\opr{forkort}
\abc{
\item 
\alg{
\vec{b}\cdot(\vec{a}+\vec{c}) + 3(\vec{a}+\vec{b})^2 &= \vec{b}\cdot \vec{a}+\vec{b}\cdot \vec{c}+3\vec{a}^2+6\vec{a}\cdot\vec{b}+3\vec{b}^2 \\
&= 0+0+3\vec{a}^2+0+3\vec{b}^2 \\
&= 3\cdot 1^2+3\cdot2^2 \\
&= 15
}
}

\textbf{b)} 
\algv{
(\vec{a}+ \vec{b}+\vec{c})^2 &= (\vec{a}+ \vec{b}+\vec{c})\cdot(\vec{a}+ \vec{b}+\vec{c}) \\
&= \vec{a}^{\,2}+\vec{a}\cdot\vec{b}+\vec{a}\cdot\vec{c}+\vec{b}\cdot\vec{a}+\vec{b}^{\,2}+\vec{b}\cdot\vec{c}+\vec{c}\cdot\vec{a}+\vec{c}\cdot\vec{b}+\vec{c}^{\,2} \\
&= 1^2+0+ \vec{a}\cdot\vec{c}+0+2^2+0+\vec{c}\cdot\vec{a}+0+5^2 \\
&= 2(15+\vec{a}\cdot\vec{c}) \\
&= 
}


\opr{torto}\\
\textbf{b)} Vi krever at
\alg{\vec{u}\cdot\vec{v} &= 0 \\ 
[-5, -1, 6] \cdot[t, t^2, 1]&= 0 \\
-5t -t^2 +6 &= 0 \\
t^2+5t-6 &= 0
}
Da $ (-1)\cdot6 = -6 $ og $ -1+6=5 $, kan vi skrive at
\[ (t-6)(t+1)=0 \]
Kravet er dermed oppfylt hvis $ t\in\lbrace-1, 6\rbrace $. 

\opr{sjekkpar}
\textbf{a)} Vi regner fort ut at forholdet mellom både førstekomponentene og andrekomponenten er 2, men at forholdet mellom tredjekomponentene er $ -\frac{1}{2} $. Vektorene er derfor ikke parallelle.

\textbf{b)} Vi observerer at:
\[ \vec{b}=\frac{3}{7}[-3, 5, 2] \]
dermed er $ \vec{b} $ et multiplum av $ \vec{a} $ og da er $ \vec{a}||\vec{b} $.

\opr{finntpar}\\
\textbf{a)} Vi bruker forholdet mellom første- og tredjekomponententene for å sette opp en ligning for $ t $:
\alg{-\frac{t+3}{3}&=-\frac{16}{8} \br
t+3 &= 6 \\
t &= 3
}
Forholdet mellom andrekomponentene blir da:
\alg{
\frac{1-3}{1}&= -2
}
Forholdet er $ -2 $ for alle komponentene når $ t=3 $ og da er $ \vec{a}||\vec{b} $.

\textbf{b)} Også her bruker vi første- og tredjekomponententene for å sette opp en ligning for $ t $, fordi vi da får isolert det kvadratiske leddet:
\alg{
	-\frac{t^2+2}{3}&= -\frac{(5t^2+3)}{8} \\
	8t^2 + 16 &= 15 t^2 + 9 \\
	7 x^2 &= 7 \\
	x &= \pm 1
}
Når $ t=1 $ er forholdet mellom både førstkomponentene og tredjekomponentene lik
\[ -\frac{1^2+2}{3}=-1 \]
Og forholdet mellom andrekomponentene er:
\[ \frac{1}{1}=1 \]
For $ t=1 $ er altså $ \vec{a} $ og $ \vec{b} $ ikke parallelle. Vi ser derimot fort at forholdet mellom hver av komponentene blir $ -1 $ når $ t=-1 $, for dette valget av $ t $ er derfor $ \vec{a}||\vec{b} $.

\opr{finnsogt}
$\vec{u}=[4, 6+s, -(s+t)] $ og $ \vec{v}=\left[\frac{12}{7}, \frac{2t-9s}{7}, \frac{3s-t}{7}\right] $
Vi starter med å observere at:
\[ \vec{v}=\frac{1}{7}[12, 2t-9s, 3s-t] \]
Vi definerer $ \vec{w}= [12, 2t-9s, 3s-t] $. Skal vi ha at $ \vec{u}||\vec{v} $, må vi også hat at $ \vec{u}||\vec{w} $. Siden forholdet mellom førstekomponentene til $ \vec{u} $ og $ \vec{w} $ er 3, krever vi at $ \vec{u}=3\vec{w} $. Da kan vi sette opp følgende ligningssystem:
\alg{
2t-9s &= 3(6+s) \tag{I}\label{3I}\br
3s-t &= -3(s+t) \tag{II}\label{3II}
}
Av \ref{3II} får vi at:
\alg{
3s-t &= -3s-3t \\
2t &= -6s \\
t &= -3s 
}
Setter vi $ t=-3s $ inn i (\ref{3II}) får vi:
\alg{
2(-3s) -9s &= 18+3s\\
-6s -9s &=  18 +3s \\
-18s &= 18 \\
s&=-1
}
Altså er $ \vec{u} $ og $ \vec{v} $ parallelle hvis $ s=-1 $ og $ t=-3s=3 $. 

\opr{vis22det}
\alg{
\left|\begin{matrix}
	ae & be \\
	cf & df
\end{matrix}\right|&= aedf-becf \\
&= ef(ad-bc) \\
&= ef\left|\begin{matrix}
	a & b \\
	c & d
\end{matrix}\right|
i}

\opr{arparo}

\textbf{b)} Arelet er gitt som tallverdien til $ \det(\vec{a}, \vec{b}) $:
\alg{
\det(\vec{a}, \vec{b}) &= \left|\begin{matrix}
	-2 & 4 \\
	24 & -16
\end{matrix}\right|\\	&= 2\cdot8 \left|\begin{matrix}
-1 & 2 \\
3 & -2
\end{matrix}\right| \\
&= 16((-1)\cdot(-2)-2\cdot3) \\
&= 16\cdot(-4) \\
&= -64
}
Arealet er altså 64.

\opr{abparvekpro0}\\
Hvis $ \vec{u}||\vec{v} $ betyr dette at hvis vi skriver $ \vec{u}=[a, b, c] $, så kan vi skrive $ \vec{v}=d[a, b, c] $. Vi får da at:
\alg{
\vec{u}\times\vec{v}&= \left|\begin{matrix}
	\vec{e}_x & \vec{e}_y & \vec{e}_z \\
	a & b & c \\
	da & db & dc
\end{matrix}\right| \\
&= d\vec{e}_x \left|\begin{matrix}
	b & c \\
	b & c
\end{matrix}\right|-d\vec{e}_y \left|\begin{matrix}
a & c \\
a & c
\end{matrix}\right|+d\vec{e}_x \left|\begin{matrix}
a & b \\
a & b
\end{matrix}\right| \\
&= 0
}
Resultatet fra \ref{vis22det} er her brukt i andre linje for å forenkle regningen av $ 2\times2 $ determinantene.

\opr{tetraro}\\
\textbf{a)} Arealet til grunnflaten tilsvarer $ \frac{1}{2}|\vec{a}\times\vec{b}| $. Vi har at
\alg{
	\vec{a}\times\vec{b}  &= 
\left|\begin{matrix}
	\vec{e}_x & \vec{e}_y & \vec{e}_z \\
	2 & -2 & 1 \\
	3 & -3 & 1
\end{matrix}\right| \\
&= [(-2)\cdot1-1\cdot(-3), -(2\cdot1-1\cdot3), 2\cdot(-3)-(-2)\cdot3] \\
&= [-2+3, -(2-3), -6+6] \\
&= [1, 1, 0]
}
Altså er 
\[ \frac{1}{2}|\vec{a}\times\vec{b} |= \frac{1}{2}\sqrt{1^2+1^2} = \frac{\sqrt{2}}{2} \]
\textbf{b)} Av (??) vet vi at volumet $ V $ er gitt som:
\[ V = \frac{1}{6}|\vec{a}\times\vec{b} \cdot \vec{c}| \]
Vi har at
\algv{
\vec{a}\times\vec{b} \cdot \vec{c} &= [1, 1, 0]\cdot[2, -3, 2] \\
&= 2-3\\
&= -1
}
Og dermed er $ V=\frac{1}{6} $.
\newpage
\opr{parfinnh}\\
\textbf{a)} Diagonalen til grunnflaten kan uttrykkes som vektoren $ \vec{a}+\vec{b} $, og lengden blir da (husk at $ |\vec{u}|^2 = \vec{u}^{\,2} $):
\alg{
|\vec{a}+\vec{b}| &= \sqrt{\left(\vec{a}+\vec{b}\right)^2} \\
&= \sqrt{\vec{a}^{\,2}+\vec{a}\cdot\vec{b}+\vec{b}^{\,2}} \\
&= \sqrt{3^2 +0 +4^2} \\
&= \sqrt{25} \\
&= 5
}
\end{document}