\newcommand{\bint}{\rg[Bestemt integral I]{Det bestemte integralet $ I $ av en funksjon $ f(x) $ over intervallet $ [a, b] $ er gitt som
		\nreq{I= \lim\limits_{n\to \infty}\sum\limits_{i=1}^{n} f(x_{i})\Delta x \label{bint}}
		hvor $ {x_i=a+(i-1)\Delta x}$ og $ {\Delta x=\frac{b-a}{n}} $.
	}}
\newcommand{\anfndto}{\rg[Analysens fundamentalteorem]{\index{analysens fundamentalteorem}
		Gitt en funksjon $ f(x) $ definert på intervallet $ [a, b] $. Hvis $ F $ er en antiderivert til $ f $, er
		\begin{equation}
		\int\limits_a^b f(x)\, dx = F(b)-F(a) \label{anfund2}
		\end{equation}\vs
	}}
\newcommand{\anfndtoe}{\eks{
Gitt funksjonen $ f(x)= e^{\sin x}  $. Finn $ \int\limits_0^\frac{\pi}{2} f'(x)\,dx $ .

\sv
Siden $ f $ er en antiderivert til $ f'(x) $, må vi ha at
\alg{\int\limits_0^\frac{\pi}{2} f'(x)\,dx &= f\left(\frac{\pi}{2}\right)-f(0)\\
	&= e^{\sin \frac{\pi}{2}}-e^{\sin 0} \\
	&= e-1
		}\vds
}}
\newcommand{\uint}{\rg[Ubestemt integral]{
		Det ubestemte integralet av $ f(x) $ er gitt som
		\begin{equation}
		\int f(x)\, dx  = F(x)+ C \label{uint}
		\end{equation}
		Hvor $ F $ er en antiderivert til $ f $ og $ C $ er en vilkårlig konstant.
	}}
\newcommand{\uinte}{}
\newcommand{\uinteto}{}
\newcommand{\anfnden}{\rg[Analysens fundamentalteorem (del 1)]{
		For enhver kontinuerlig funksjon $ f(x) $ har vi at
		\[ \left(\int f(x)\,dx\right)'=f(x)  \]	}
}
\newcommand{\anfndene}{	\eks{	
		Vi har at $ \int f(x) \, dx= e^{x^2}+C $. Hva er $ f(x) $? \\
		
		\sv 
		Fra analysens fundamentalteorem vet vi at vi kan finne $ f(x) $ ved å derivere det ubestemte integralet. Vi bruker da kjerneregelen, og finner at
		\[ \left(e^{x^2}\right)'=2xe^{x^2} \]
		Derfor er \[ f(x)=2xe^{x^2} \]
	}}
\newcommand{\uints}{}
\newcommand{\uintse}{}
\newcommand{\dint}{
	}
\newcommand{\dinte}{}
\newcommand{\dinteto}{}
\newcommand{\byt}{}
\newcommand{\byte}{}
\newcommand{\byteto}{}
\newcommand{\dbr}{}
\newcommand{\dbre}{}
\newcommand{\iar}{\rg[Integral som areal I]{\index{integral!som areal}Gitt en kontinuerlig funksjon $ f(x) $ og to tall $ a $ og $ b $ der $ {a<b }$. \vsk

	Hvis $ {f\geq0} $ for  ${x\in [a, b]} $, er arealet $ A $ avgrenset av $ f $ på dette intervallet gitt som
	\[ A=\int\limits_a^b f \,dx \]
	\begin{figure}[H]
	\centering
	\includegraphics[]{\asym{intpos}}
\end{figure}
	Hvis $ {f\leq0 }$ for ${x\in [a, b]} $, er arealet $ A $ avgrenset av $ f $ på dette intervallet gitt som
	\[ A=-\int\limits_a^b f \,dx \]
	\begin{figure}[H]
	\centering
	\includegraphics[]{\asym{intneg}}
\end{figure}\vs
	}}
\newcommand{\ivo}{}
\newcommand{\ivoe}{}
\newcommand{\omdr}{}}
\newcommand{\omdre}{}
\newcommand{\bintto}{\rg[Bestemt integral II]{
		Det bestemte integralet $ I $ av en funksjon $ f(x) $ over intervallet $ [a, b] $ skrives som
		\nreq{I = \int\limits_{a}^b f(x)\,dx \label{bintint}}\vs
}}
\newcommand{\bytb}{\rg[Bytte av variabel for bestemt integral]{
		Gitt funksjonene $ u(x) $ og $ g(u) $. Da har vi at
		\begin{equation}
		\int\limits_a^b g(u) u'\, dx=\int\limits_{u(a)}^{u(b)}  g(u) \, du \label{bytvarb}
		\end{equation}\vs
}}
\newcommand{\iarto}{}