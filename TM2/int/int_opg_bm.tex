\documentclass[english,hidelinks,pdftex, 11 pt, class=report,crop=false]{standalone}
\usepackage[T1]{fontenc}
\usepackage[utf8]{luainputenc}
\usepackage{lmodern} % load a font with all the characters
\usepackage{geometry}
\geometry{verbose,a4paper, inner=0cm, outer=0 cm, bmargin=2cm, tmargin=1cm}
%\textwidth=12cm
\setlength{\parindent}{0bp}
\usepackage{import}
\usepackage[subpreambles=false]{standalone}
\usepackage{amsmath}
\usepackage{amssymb}
\usepackage{esint}
\usepackage{babel}
\usepackage{tabu}
\usepackage[dvipsnames, table]{xcolor}
\usepackage{cancel}
\makeatother
\makeatletter
\usepackage{datetime2}
\usepackage{titlesec}
\usepackage[many]{tcolorbox}

% Eheter
\newcommand{\enh}[1]{\,\textrm{#1}}
%referances
\newcommand{\net}[2]{{\color{blue}\href{#1}{#2}}}

%Spaces
\newcommand{\vsk}{\\[12pt]}
\newcommand{\vs}{\vspace{-12pt}}

% Tabell for opplegg

\newcommand{\ovlist}[1]{
\vspace{-16pt}
\begin{itemize}
	#1
\end{itemize}
}

% Chapters and sections
\titleformat{\section}[block]{\bfseries}{\hspace{3cm}\thesection}{5pt}{}
\titleformat{\subsection}[block]{\bfseries}{\hspace{3cm}\thesection}{5pt}{}
\newcommand{\sectionbreak}{\clearpage} % New page on each section
 

\newlength{\mywidth}
\setlength{\mywidth}{14cm}

\newcommand{\cont}[1]{
\begin{tcolorbox}[center, boxrule=0.0 mm, width=\mywidth,arc=0mm,enhanced jigsaw,,colback=white,breakable]
#1	
\end{tcolorbox}
}

\newcommand{\info}[5]{
\begin{tcolorbox}[center, boxrule=0.1 mm, width=\mywidth,arc=0mm,enhanced jigsaw,breakable,colback=yellow!5]	
	
	\footnotesize
	\textbf{Øvingsområde}\\[5pt] #1 
	
	\textbf{Utstyr}\\ #2  \\
	
	\begin{tabular}{@{} p{4cm} p{4cm} l} 
		\textbf{Tid} & \textbf{Elevinndeling} & \textbf{Læringsarena} \\
		#3  & #4 & #5
	\end{tabular} 
\end{tcolorbox}	
}

\newcommand{\gjen}[1]{\begin{tcolorbox}[center,boxrule=0.1 mm, width=\mywidth,arc=0mm,colback=blue!3] {\large \textbf{Gjennomføring} \vspace{5 pt}}\newline #1  \end{tcolorbox}\vspace{-5pt}}
\newcommand{\eks}[1]{\begin{tcolorbox}[center,boxrule=0.1 mm, width=\mywidth,arc=0mm,colback=green!3] {\large \textbf{Eksempel} \vspace{5 pt}}\newline #1  \end{tcolorbox}\vspace{-5pt}}

\newcounter{opl}
%\numberwithin{opl}{article}


\newcommand{\opl}[1]{
\newpage
{\refstepcounter{opl} %\phantomsection 
\large \textbf{\theopl \;#1} \vsk}
}

% Headlines
\newcommand{\fork}{\textbf{Forkunnskapar}\\}
\newcommand{\forb}{\textbf{Forberedelsar}\\}
\newcommand{\opgvr}{\textbf{Oppgaver}}



%colors
\newcommand{\colr}[1]{{\color{red} #1}}
\newcommand{\colb}[1]{{\color{blue} #1}}
\newcommand{\colo}[1]{{\color{orange} #1}}
\newcommand{\colc}[1]{{\color{cyan} #1}}
\definecolor{projectgreen}{cmyk}{100,0,100,0}
\newcommand{\colg}[1]{{\color{projectgreen} #1}}

% Lister med bokstavar
\usepackage[inline]{enumitem}
% Opg
\newcommand{\abc}[1]{
	\begin{enumerate}[label=\alph*),leftmargin=18pt]
		#1
	\end{enumerate}
}

\usepackage[]{hyperref}

\newcommand{\note}{Merk}
\newcommand{\notesm}[1]{{\footnotesize \textsl{\note:} #1}}
\newcommand{\ekstitle}{Eksempel }
\newcommand{\sprtitle}{Språkboksen}
\newcommand{\expl}{forklaring}
\newcommand{\pyt}{Pytagoras' setning}
\newcommand\sv{\vsk \textbf{Svar} \vspace{4 pt}\\}

%references
\newcommand{\reftab}[1]{\hrs{#1}{tabell}}
\newcommand{\rref}[1]{\hrs{#1}{regel}}
\newcommand{\dref}[1]{\hrs{#1}{definisjon}}
\newcommand{\refkap}[1]{\hrs{#1}{kapittel}}
\newcommand{\refsec}[1]{\hrs{#1}{seksjon}}
\newcommand{\refdsec}[1]{\hrs{#1}{delseksjon}}
\newcommand{\refved}[1]{\hrs{#1}{vedlegg}}
\newcommand{\eksref}[1]{\textsl{#1}}
\newcommand\fref[2][]{\hyperref[#2]{\textsl{figur \ref*{#2}#1}}}
\newcommand{\refop}[1]{{\color{blue}Oppgave \ref{#1}}}
\newcommand{\refops}[1]{{\color{blue}oppgave \ref{#1}}}


%Algebra
\newcommand{\kvadset}{Kvadratsetningene}
\newcommand{\aenato}{Sum-produkt-metoden}

% Geometry
\newcommand{\hlikb}{Midtnormalen i en likebeint trekant}
\newcommand{\arealsetn}{Arealsetningen}
\newcommand{\trkmedian}{Median}
\newcommand{\midtrk}{Midtnormal (i trekant)}
\newcommand{\innskrsirk}{Innskrevet sirkel}
\newcommand{\cossetn}{Cosinussetningen}
\newcommand{\perfvink}{Sentral- og periferivinkel}
\newcommand{\tang}{Tangent}

% Derivative
\newcommand{\derel}{Den deriverte av elementære funksjoner}
\newcommand{\divder}{Divisjonsregelen}
\newcommand{\kjernereg}{Kjerneregelen}
\newcommand{\prodregder}{Produktregelen}
\newcommand{\lhop}{L'Hopitals regel}

% Funksjonsdrofting
\newcommand{\monder}{Monotoniegenskaper og den deriverte}
\newcommand{\fderekstr}{$ \bm{f'=0} $ for lokale ektstremalpunkt}
\newcommand{\andredertest}{Andrederiverttesten}

% Vectors
\newcommand{\detar}{Arealformler med determinanter}
\newcommand{\avstpunktlin}{Avstand mellom punkt og linje}

%Appendix
\newcommand{\rolle}{Rolles teorem}
\newcommand{\meanval}{Middelverdisetningen}

% Solutions manual
\newcommand{\selos}{Se løsningsforslag.}

\begin{document}
	
	
\opgt

\op{antiderint}
\textbf{a)} Deriver funksjonen $ f(x)=4x^5 $.\os

\textbf{b)} Finn det bestemte integralet $ \int\limits_0^2 20 x^4 \, dx $.

\op{Fogf}
Relasjonen mellom en funksjon $ F(x) $ og $f(x) $ er at $ F'(x)=f(x) $. Videre er $ F(1)=1 $ og $ F(4)=9 $.\os

Finn det bestemte integralet \y{\int\limits_1^4 f(x) \, dx}.

\op{ecos2x}
\textbf{a)} Deriver funksjonen $ f(x)=e^{\cos^2 x} $.\os

\textbf{b)} Finn det ubestemte integralet \[ \int -\sin (2x)\, e^{\cos^2 x}\,dx \]\vs\vs

\op{visantider}
Vis at\os
\textbf{a)} $\displaystyle \int x(x+2)e^x \,dx = x^2 e^x + C $ \os

\textbf{b)} $\displaystyle \int -e^{x^2+\cos x} (-2 x+\sin x)\,dx= e^{\cos x+x^2}+C  $

\nes

\op{intsopg}
Finn integalene:\os
\begin{tabular}{@{}l l l l}	
\textbf{a)} $ \displaystyle \int \frac{3}{4 x} \,dx$ &\quad \textbf{b)} $ \displaystyle \int-\frac{7}{\cos^2 t}\,dt $ &\quad \textbf{c)} $ -4x^5 $ \\[20 pt]
\textbf{d)}\ $\displaystyle \int \cos(\pi x) \,dx$ & \quad
\textbf{e)} $\displaystyle \int 4e^{-4t} \,dt$ &\quad\textbf{f)} $ \displaystyle \int \left(2x^4\,dx - \frac{3}{x^{\frac{3}{2}}}\right) \,dx$ \\[20pt]
\textbf{g)} $\displaystyle \int \sqrt{x^5}\,dx $
\end{tabular} 
\newpage
\op{opgintbestmt}
Regn ut de bestemte integralene.\os
\abch{
\item $\displaystyle \int_{-1}^{1} \left(4x^3-x\right)\,dx $
\item  $\displaystyle \int_{0}^{\ln 2} e^{2x}\,dx $
}

\op{avcos}
Gjennomsnittet av en funksjon $ f(x) $ over et intervall $ [a, b] $ kan vi skrive som
\[ \frac{1}{b-a}\int\limits_a^b f(x)\,dx \]
Gitt et vilkårlig tall $ c $, vis at gjennomsnittet av $f(x)=\cos x+d  $ over intervallet $ [c, c+2\pi] $ er lik $ d $. 

\op{opgintviselmnt}
Bevis \eqref{intsinkx}-\eqref{intexpkx} ved å bruke integrasjon ved substutisjon.

\op{bytvaropg}
Finn integralene:\os

\begin{tabular}{@{}l l l}	
\textbf{a)} $\displaystyle \int xe^{x^2} \, dx  $ &\;\textbf{b)} $\displaystyle \int\limits_1^2 8xe^{2x^2-3}\,dx $ &\;\textbf{c)} $\displaystyle \int \tan x \, dx $ \\ \vspace{3pt} 
\textbf{d)} $ \displaystyle \int\limits_0^\frac{\pi}{3}\frac{\sin x}{\cos^3 x} \, dx $ &\;\textbf{e)} $ \displaystyle \int \frac{4x+5}{2x^2 + 5x}\,dx $
&\;\textbf{f)} $ \displaystyle \int \frac{3x+2}{3x^2 + 4x+3}\,dx $
\end{tabular} 

\op{trigint}
Anvend to av de trigonometriske identitetene og bytte av variabel to ganger for å finne integralet
\[ \int \sin (2x) e^{1-\cos^2 x}\,dx \]\vs


\op{delvisintopg}
Finn det bestemte/ubestemte integralet:\os
\begin{tabular}{@{}l l l}	
	\textbf{a)} $\displaystyle \int (x-1)\cos x \, dx$&\quad	\textbf{b)} $\displaystyle \int \sqrt{x}\ln x\,dx $ &\quad\textbf{c)} $\displaystyle \int\limits_1^e  \frac{\ln x}{x^2} $ 
\end{tabular}

\op{delvisintopg2}
Vis at
\[ \int \sin^2 x\,dx=\frac{1}{2}(x-\sin x \cos x)+C \]\vs

\op{delbropsopg}
Finn det bestemte/ubestemte integralet:\os

\begin{tabular}{@{}l l l}	
	\textbf{a)} $ \displaystyle \int\limits_4^5 \frac{13-4x}{x^2-5x+6}\,dx $	&\quad \textbf{b)} $ \displaystyle \int \frac{41 - 4 x}{(x - 5) (x + 2)}\, dx $ \\ \\
	\textbf{c)}$\displaystyle \int\limits \frac{x^2+9x-16}{(x-2)(x^2-1)} dx$  &\quad \textbf{d)} $ \displaystyle \int\frac{3 x^2 - 14 x + 10}{x^3 - 3 x^2 + 2 x} $
\end{tabular}

\op{delbrogpoldiv}
Finn det ubestemte integralet:
\[\int \frac{3 x^3 - 2 x^2 - 20 x + 2}{x^2-x-6}\,dx \]
\textsl{Hint}: Bruk polynomdivisjon.

\newpage
\nes
\op{gerfminusdb}
Relasjonen mellom to funksjoner $ f(x) $ og $ g(x) $ og en konstant $ d $ er at
\[ g=f+d \]
\textbf{a)} Ta det for gitt at $ f $ og $ g $ er som vist på figuren under.
\begin{figure}
\centering
\includegraphics[scale=0.9]{../fig/int6a}\quad
\includegraphics[scale=0.9]{../fig/int6b}
\end{figure}
Forklar ut ifra en arealbetraktning hvorfor\[ \int\limits_a^b f \,dx=\int\limits_a^b g\,dx -(b-a)d  \]
\textbf{b)} Bekreft likheten i oppgave a) ved integrasjon.

\op{fogF}
Under vises grafen til $ F(x) $ og $ f(x) $. $ F $ er en antiderivert av $ f $.
\begin{figure}
	\centering
	\includegraphics[scale=0.9]{../fig/faropg}\\
	\raggedright
\end{figure}
Forklar hvorfor arealet av det oransje området er like stort som arealet av det grønne området.
\newpage
\nes
\op{kulevolopg}
La en kule med radius $ r $ være plassert i et koordinatsystem med variabelen $ x $ langs horisontalaksen. Kula er plassert slik at sentrum ligger i origo.\os

\textbf{a)} Lag en tegning og bestem kulas tverrsnitt $ A $ langs horisontalaksen, uttrykt ved $ r $ og $ x $.\os

\textbf{b)} Finn volumet $ V $ av kula.

\op{omdropg} 
Finn volumet av omdreiningslegemene til funksjonene på intervallet $ [0, 1] $:\os

\begin{tabular}{@{}l l l}	
	\textbf{a)} $ f(x)=e^x  $&\quad\textbf{b)} $\displaystyle f(x)= \frac{1}{\sqrt{2}}\sqrt{1-\cos(2 \pi x)}$ &\quad
\end{tabular}
\newpage
\grubop{r2v23d1opg2}
(R2V23D1)
\abc{
\item Vis at hvis $ f(x)=\tan x $, så er $ f'(x)=1+\tan^2 x $.
\item Regn ut
\[ \int \frac{1+\tan^2 x}{\tan x}\, dx \]
}

\grubop{r2h23d1opg1}
(R2H23D1)\os
Regn ut integralet
\[ \int_{-1}^{1}x^3+2x\,dx \]
Hva forteller svaret deg?


\grubop{r2h23d1opg2}
(R2H23D1)\os
Figuren viser grafene til funksjonene $ f(x)=\cos x$ og $ g(x)=\sin x $.
\fig{r2h23d1opg2}
Bestem arealet til det fargede området.

\grubop{opgintxkvadmdef}
Bruk definisjonen fra (\ref{bint}) til å vise at
\[ \int\limits_a^b x^2 \,dx = \frac{1}{3}(b^3-a^3) \]
\begin{comment}
\textsl{Hint}: Bruk summen av de naturlige tallene og \eqref{sumkvad} fra s. \pageref{sumkvad}.
\end{comment}

\newpage
\grubop{opgfunklen}
Gitt en funksjon $ f(x) $ integrerbar på intervallet $ [a, b] $. Vis at \\lengden $ l $ til grafen til $ f $ er
\[ l=\int_{a}^{b} \sqrt{1+g^2}\,dx \]
hvor $ g(x)=f'(x) $.
\fig{opgfunklen}

\begin{comment}
	\subsection*{GeoGebra-oppgaver}
	\op
	Hvis $ |x|\ll 1 $ kan man gjøre følgende tilnærming:
	\[f(x)= \frac{1}{\cos^2 {x}} \approx g(x)=1 + x^2 + \frac{2}{3}x^4 \]
	
	\textbf{a)} Tegn $ f(x) $ og $ g(x) $ inn i et koordinatsystem for $ x\in  [-0.4, 0.4]$
	
	\textbf{b)} Bruk det faktum at $  \int f(x)\, dx = \tan x $ og at $ \int f(x)\, dx \approx \int g(x)\, dx$ for å finne en tilnærming til $ \tan x $
\end{comment}
\end{document}