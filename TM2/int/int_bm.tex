\documentclass[english,hidelinks,pdftex, 11 pt, class=report,crop=false]{standalone}
\usepackage[T1]{fontenc}
\usepackage[utf8]{luainputenc}
\usepackage{lmodern} % load a font with all the characters
\usepackage{geometry}
\geometry{verbose,a4paper, inner=0cm, outer=0 cm, bmargin=2cm, tmargin=1cm}
%\textwidth=12cm
\setlength{\parindent}{0bp}
\usepackage{import}
\usepackage[subpreambles=false]{standalone}
\usepackage{amsmath}
\usepackage{amssymb}
\usepackage{esint}
\usepackage{babel}
\usepackage{tabu}
\usepackage[dvipsnames, table]{xcolor}
\usepackage{cancel}
\makeatother
\makeatletter
\usepackage{datetime2}
\usepackage{titlesec}
\usepackage[many]{tcolorbox}

% Eheter
\newcommand{\enh}[1]{\,\textrm{#1}}
%referances
\newcommand{\net}[2]{{\color{blue}\href{#1}{#2}}}

%Spaces
\newcommand{\vsk}{\\[12pt]}
\newcommand{\vs}{\vspace{-12pt}}

% Tabell for opplegg

\newcommand{\ovlist}[1]{
\vspace{-16pt}
\begin{itemize}
	#1
\end{itemize}
}

% Chapters and sections
\titleformat{\section}[block]{\bfseries}{\hspace{3cm}\thesection}{5pt}{}
\titleformat{\subsection}[block]{\bfseries}{\hspace{3cm}\thesection}{5pt}{}
\newcommand{\sectionbreak}{\clearpage} % New page on each section
 

\newlength{\mywidth}
\setlength{\mywidth}{14cm}

\newcommand{\cont}[1]{
\begin{tcolorbox}[center, boxrule=0.0 mm, width=\mywidth,arc=0mm,enhanced jigsaw,,colback=white,breakable]
#1	
\end{tcolorbox}
}

\newcommand{\info}[5]{
\begin{tcolorbox}[center, boxrule=0.1 mm, width=\mywidth,arc=0mm,enhanced jigsaw,breakable,colback=yellow!5]	
	
	\footnotesize
	\textbf{Øvingsområde}\\[5pt] #1 
	
	\textbf{Utstyr}\\ #2  \\
	
	\begin{tabular}{@{} p{4cm} p{4cm} l} 
		\textbf{Tid} & \textbf{Elevinndeling} & \textbf{Læringsarena} \\
		#3  & #4 & #5
	\end{tabular} 
\end{tcolorbox}	
}

\newcommand{\gjen}[1]{\begin{tcolorbox}[center,boxrule=0.1 mm, width=\mywidth,arc=0mm,colback=blue!3] {\large \textbf{Gjennomføring} \vspace{5 pt}}\newline #1  \end{tcolorbox}\vspace{-5pt}}
\newcommand{\eks}[1]{\begin{tcolorbox}[center,boxrule=0.1 mm, width=\mywidth,arc=0mm,colback=green!3] {\large \textbf{Eksempel} \vspace{5 pt}}\newline #1  \end{tcolorbox}\vspace{-5pt}}

\newcounter{opl}
%\numberwithin{opl}{article}


\newcommand{\opl}[1]{
\newpage
{\refstepcounter{opl} %\phantomsection 
\large \textbf{\theopl \;#1} \vsk}
}

% Headlines
\newcommand{\fork}{\textbf{Forkunnskapar}\\}
\newcommand{\forb}{\textbf{Forberedelsar}\\}
\newcommand{\opgvr}{\textbf{Oppgaver}}



%colors
\newcommand{\colr}[1]{{\color{red} #1}}
\newcommand{\colb}[1]{{\color{blue} #1}}
\newcommand{\colo}[1]{{\color{orange} #1}}
\newcommand{\colc}[1]{{\color{cyan} #1}}
\definecolor{projectgreen}{cmyk}{100,0,100,0}
\newcommand{\colg}[1]{{\color{projectgreen} #1}}

% Lister med bokstavar
\usepackage[inline]{enumitem}
% Opg
\newcommand{\abc}[1]{
	\begin{enumerate}[label=\alph*),leftmargin=18pt]
		#1
	\end{enumerate}
}

\usepackage[]{hyperref}

\newcommand{\note}{Merk}
\newcommand{\notesm}[1]{{\footnotesize \textsl{\note:} #1}}
\newcommand{\ekstitle}{Eksempel }
\newcommand{\sprtitle}{Språkboksen}
\newcommand{\expl}{forklaring}
\newcommand{\pyt}{Pytagoras' setning}
\newcommand\sv{\vsk \textbf{Svar} \vspace{4 pt}\\}

%references
\newcommand{\reftab}[1]{\hrs{#1}{tabell}}
\newcommand{\rref}[1]{\hrs{#1}{regel}}
\newcommand{\dref}[1]{\hrs{#1}{definisjon}}
\newcommand{\refkap}[1]{\hrs{#1}{kapittel}}
\newcommand{\refsec}[1]{\hrs{#1}{seksjon}}
\newcommand{\refdsec}[1]{\hrs{#1}{delseksjon}}
\newcommand{\refved}[1]{\hrs{#1}{vedlegg}}
\newcommand{\eksref}[1]{\textsl{#1}}
\newcommand\fref[2][]{\hyperref[#2]{\textsl{figur \ref*{#2}#1}}}
\newcommand{\refop}[1]{{\color{blue}Oppgave \ref{#1}}}
\newcommand{\refops}[1]{{\color{blue}oppgave \ref{#1}}}


%Algebra
\newcommand{\kvadset}{Kvadratsetningene}
\newcommand{\aenato}{Sum-produkt-metoden}

% Geometry
\newcommand{\hlikb}{Midtnormalen i en likebeint trekant}
\newcommand{\arealsetn}{Arealsetningen}
\newcommand{\trkmedian}{Median}
\newcommand{\midtrk}{Midtnormal (i trekant)}
\newcommand{\innskrsirk}{Innskrevet sirkel}
\newcommand{\cossetn}{Cosinussetningen}
\newcommand{\perfvink}{Sentral- og periferivinkel}
\newcommand{\tang}{Tangent}

% Derivative
\newcommand{\derel}{Den deriverte av elementære funksjoner}
\newcommand{\divder}{Divisjonsregelen}
\newcommand{\kjernereg}{Kjerneregelen}
\newcommand{\prodregder}{Produktregelen}
\newcommand{\lhop}{L'Hopitals regel}

% Funksjonsdrofting
\newcommand{\monder}{Monotoniegenskaper og den deriverte}
\newcommand{\fderekstr}{$ \bm{f'=0} $ for lokale ektstremalpunkt}
\newcommand{\andredertest}{Andrederiverttesten}

% Vectors
\newcommand{\detar}{Arealformler med determinanter}
\newcommand{\avstpunktlin}{Avstand mellom punkt og linje}

%Appendix
\newcommand{\rolle}{Rolles teorem}
\newcommand{\meanval}{Middelverdisetningen}

% Solutions manual
\newcommand{\selos}{Se løsningsforslag.}

\begin{document}
\section{Bestemt og ubestemt integral \label{BestemtogubestemtI}}
\regdef[Integral som areal]{
Gitt en funksjon $ f(x) $ som er positiv og kontinuerlig for alle $ {x\in[a, b] }$. \outl{Integralet} $ I $ til $ f $ på intervallet $ {x\in[a, b]} $  tilsvarer da arealet avgrenset av $ x $-aksen, linjene $ {x=a} $ og ${ x= b} $, og grafen til $ f $.
\fig{intpos}
}\vsk

Gitt en funksjon $ f(x) $, som vist i \fref{fartvar}. Vi ønsker nå å finne  en tilmærming for integralet $ I $ til $ f $ på intervallet $ [a, b] $.
\figc{int1a}{\label{fartvar}}
Vi starter med å dele $ [a, b] $ inn i tre delintervaller, som da får bredden $ \Delta x=\frac{b-a}{3} $. Videre bruker vi $ f(x) $ i starten av hvert delintervall som høgden i et rektangel. $ x $-verdiene dette gjelder kaller vi $ {x_1=a} $, $ x_2 $ og $ x_3 $. Vi kan nå tilnærme $ I $ som summen av tre areal, $ s_1 $, $ s_2 $ og $ s_3 $:
\alg{
	I &\approx s_1 + s_2 + s_3 \\
	&\approx f(a)\Delta x + f(x_2)\Delta x + f(x_3)\Delta x \\
	&\approx \left[f(a) +f(x_2) +f(x_3)\right]\Delta x
}
\figc{int1}{\label{binttreint}}

Intuitivt vil vi tenke at jo mindre intervaller vi bruker, jo bedre må tilnærmingen være.
\begin{figure}[H]
	\centering
	\subfloat[]{\includegraphics[scale=0.9]{\figp{int2}}}\quad
	\subfloat[]{\includegraphics[scale=0.9]{\figp{int3}}}	
	\captionof{figure}{\textsl{a)} 10 intervaller \textsl{b)} 20 intervaller \label{toints}}
\end{figure}
Lar vi $ n $ være antall delintervaller, og $ {n\to \infty} $, får vi at
\begin{align}
	I&\approx\lim\limits_{n\to \infty}\left[f(x_1)+fv(x_2)+...+f(x_{n})\right]\Delta x \nonumber \\
	&\approx \lim\limits_{n\to \infty}\sum\limits_{i=1}^{n} f(x_{i})\Delta x \label{intdefpreeq}
\end{align}
hvor $ x_i=a+(i-1)\Delta x$ og $ \Delta x=\frac{b-a}{n} $ (legg merke til at $ t_1=a $). Det kan vises at grenseverdien i \eqref{intdefpreeq} er lik $ I $, og det fører oss til følgende definisjon:
\regdef[\bestminten \label{bestminten}]{\outl{Det bestemte integralet} $ I $ av en funksjon $ f(x) $ over intervallet $ [a, b] $ er gitt som
	\begin{equation}
		I= \lim\limits_{n\to \infty}\sum\limits_{i=1}^{n} f(x_{i})\Delta x \label{bint}
	\end{equation}
	hvor $ {x_i=a+(i-1)\Delta x}$ og $ {\Delta x=\frac{b-a}{n}} $.
}
\eks{
Finn det bestemte integralet av ${ f(x)=x }$ på intervallet $ {x\in[0, 4]} $.

\sv
Vi har her at $ {f(x_i)=x_i=(i-1)\Delta x} $, hvor $ {\Delta x=\frac{4}{n}} $. Setter vi dette inn i \eqref{bint}, får vi at
\alg{
I&= \lim\limits_{n\to \infty}\sum\limits_{i=1}^{n}(i-1)\left(\frac{4}{n}\right)^2 \\
&=4^2 \lim\limits_{n\to \infty} \frac{1}{n^2}\left(\frac{n(n+1)}{2}-n \right) \\
&= 4^2 \lim\limits_{n\to \infty}\frac{1}{n^2} \left(\frac{n^2+n}{2}-n \right) \\
&= 16\cdot\frac{1}{2} \\
&= 8
}
\mers{I overgangen mellom første og andre linje i ligningen over har vi brukt summen av en aritmetisk rekke.}
}
\newpage
I kommende seksjoner skal vi finne integraler på en helt annen måte enn i eksempelet over. Læresetningen som sørger for dette er så viktig at den rett og slett kalles \outl{analysens fundamentalteorem}\footnote{Analyse i matematisk sammenheng kan, kort oppsummert, sies å være studien av funksjoner. Et teorem er en læresetning som kan bevises.}. Da teoremet gir oss en metode som omgår utregning av summer, lønner det seg å skrive integralet på en mer kompakt form:\regv
\reg[Bestemt integral II]{
	Det bestemte integralet $ I $ av en funksjon $ f(x) $ over intervallet $ [a, b] $ skrives som
\begin{equation}
	I = \int\limits_{a}^b f(x)\,dx \label{bintint}
\end{equation}
}
\info{Omskriving}{ \vspace{-5pt}
I overgangen mellom \eqref{bint} og \eqref{bintint} har vi erstattet $ \lim\limits_{n\to \infty}\sum\limits_{i=1}^{n} $ med $ \int\limits_{a}^b $, $ \Delta x $ med $ dx $, og i tillegg fjernet alle indekser.
}
\label{bestmintend}

\subsection{Den antideriverte}\index{antiderivert}
Vi skal nå se på en definisjon som kan virke veldig triviell, men som viser seg å være svært viktig i neste delseksjon. \vsk

La oss starte med å se på funksjonen $ {f(x)=x^2}  $.
Å derivere $ f $ mhp. $ x $ byr på få problemer:
\[ f'=2x \]
Hva nå med den deriverte av $ {g(x)=x^2+1} $? Svaret blir det samme som for $ f'$:
\[ g'=2x \]
Allerede nå innser vi at det finnes en haug av funksjoner, rett og slett uendelig mange, som har $ 2x $ som sin deriverte. Tiden er derfor inne for å lage en samlebetegnelse for alle funksjoner med samme deriverte:\regv
\reg[Den antideriverte]{Hvis $ F(x) $ er en deriverbar funksjon og $ {F'(x)=f(x) }$, da er $ F $ en antiderivert av $ f $.}
\eks{
	Undersøk om følgende funksjoner er en antiderivert til ${f(x)=2x+e^x } $:
	\alg{
		g(x)&=x^2 +e^x  \\[5 pt]
		h(x)&=x^2 +e^{2x}\\[5 pt]
		k(x)&=x^2+e^x+4 
	}\vs
	
	\sv
	Vi finner de deriverte av $ g $, $ h $ og $ k $:
	\alg{
		g'(x)&=2x+e^x  \os 
		h'(x)&=2x+2e^{2x} \os
		k(x)&=2x+e^x
	}
	Siden $ {g'(x)=k'(x)=f(x)} $, mens $ { h'(x)\neq f(x)} $, er bare $ g(x) $ og $ k(x) $ en antiderivert til $ f $.
}
\newpage
\subsection{\anfundteo \label{anfundteo}}	
\reg[Analysens fundamentalteorem]{\index{analysens fundamentalteorem}
		Gitt en funksjon $ f(x) $ definert på intervallet $ [a, b] $. Hvis $ F $ er en antiderivert til $ f $, er
		\begin{equation}
			\int\limits_a^b f(x)\, dx = F(b)-F(a) \label{anfund2}
		\end{equation}\vs
	}
\eks[]{
	Gitt funksjonen $ f(x)= e^{\sin x}  $. Finn $ \int\limits_0^\frac{\pi}{2} f'(x)\,dx $ .
	
	\sv
	Siden $ f $ er en antiderivert til $ f'(x) $, har vi at
	\alg{\int\limits_0^\frac{\pi}{2} f'(x)\,dx &= f\left(\frac{\pi}{2}\right)-f(0)\\
		&= e^{\sin \frac{\pi}{2}}-e^{\sin 0} \\
		&= e-1
	}
}
\subsection{Ubestemte integral}\index{integral!ubestemt}
Vi har hittil sett på det \outl{bestemte integralet}, som har sitt navn fordi integralet er over et intervall der start- og sluttverdien er gitt. Det \textit{ubestemte} integralet til en funksjon $ f(x) $ skriver vi derimot som
\[ \int\limits_c^x f(t)\, dt \]
Navnet ubestemt kommer av at $ c $ er en vilkårlig konstant og at $ x $ er en varierende verdi\footnote{Det kan kanskje se litt rart ut at vi har skrevet $ f(t) $ i integralet når vi snakker om $ f(x) $, men dette gjøres bare for å skille mellom de to varierende verdiene $ x $ og $ t $. $ x $ kan være en hvilken som helst verdi, men for det ubestemte integralet ser vi på $ f $ for verdiene $ {t\in [a, x]} $, altså $ f(t) $. Og da er det ikke $ x $ som varierer, men $ t $, derav $ dt $.}.\vsk

Hvis vi lar $ F$ være en antiderivert til $ f $, har vi fra (\ref{anfund2}) at:
\[ \int\limits_c^x f(t)\, dt = F(x)-F(c)  \]
Siden $ c $ er en konstant, må $ -F(c) $ også være det. Denne kalles \outl{integrasjonskonstanten}, og omdøpes gjerne til $ C $. Det er også vanlig å forenkle skrivemåten til det ubestemte integralet ved å fjerne grensene og bare skrive $ f(x)\,dx $ etter integraltegnet.\regv
\reg[Ubestemt integral]{
	Det ubestemte integralet av $ f(x) $ er gitt som
	\begin{equation}
		\int f(x)\, dx  = F(x)+ C \label{uint}
	\end{equation}
	Hvor $ F $ er en antiderivert til $ f $ og $ C $ er en vilkårlig konstant.
}
\info{\note}{
Når ikke annet er nevnt, tar vi det heretter for gitt at størrelser skrevet som store bokstaver er vilkårlige konstanter som resultat av integrasjon.
}
\eks[1]{
	Ved derivasjon vet vi at $ {(x^2)'=2x} $. Bruk dette til å å finne $ \int 2x \,dx$. \\ 	
	
	\sv
	Fra derivasjonen ser vi at $ x^2 $ er en antiderivert til $ 2x $. Vi kan dermed skrive
	\[ \int 2x \,dx =  x^2 + C  \]\vs
}
\newpage
\eks[2]{
	Ved derivasjon vet vi at $ {(x^2+3)'=2x} $. Bruk dette til å finne $ \int 2x \,dx$. \\ 	
	
	\sv
	Fra derivasjonen ser vi at $ {x^2+3 }$ er en antiderivert til $ 2x $. Vi kan dermed skrive
	\[ \int 2x \,dx =  x^2 +3 +C  \]
	Men siden $ C $ er en vilkårlig konstant, kan vi liksågodt lage oss en ny konstant $ D=C+3 $, og får da at
	\[ \int 2x \,dx =  x^2 + D  \]
	\textsl{Merk}: Siden integrasjonskonstanter er vilkårlige, kan vi tillate oss å komprimere flere konstanter til én. I utregningen over kunne vi skrevet $ C $ opp igen, underforstått at 3 var ''trekt inn'' i denne konstanten:
	\[ \int 2x\,dx=x^2+3+C=x^2+C \]	
}
\section{Integralregning}
Å finne bestemte og ubestemte integraler er et stort og viktig felt innenfor matematikken. Analysens fundamentalteorem forteller oss at nøkkelen er å finne en antiderivert til funksjonen vi ønsker å integrere.
\subsection{\intuf \label{intuf}}
Vi skal etterhvert se at å finne integraler ofte krever spesielle metoder, men noen grunnleggende relasjoner bør vi huske: \regv
\reg[Ubestemte integraler]{
	For to funksjoner $ f(x) $ og $ g(x) $, og to konstanter $ k $ og $ C $, er
	\begin{align}
		& \int \left[f(x)+g(x)\right]\,dx= \int f(x)\,dx + \int g(x)\,dx \label{intfplusg}\\
		& \int kf(x)\,dx = k\int f(x)\, dx \label{konstint} \\ 
		& \int x^k \, dx = \frac{1}{k+1}x^{k+1} \label{xhatr}  + C \qquad (k\neq -1) \\
		& \int \sin (kx) \, dx = -\frac{1}{k}\cos (kx) + C \\
		& \int \cos (kx) \, dx = \frac{1}{k}\sin (kx) + C \\
		& \int e^{kx} \, dx = \frac{1}{k}e^{kx} + C \\
		& \int \frac{1}{\cos^{2}x} \, dx  =\tan x +C \label{tanint} \\
		& \int \frac{1}{x+k} \, dx = \ln|x+k| + C \label{lnint} 
	\end{align}\vs
}
\newpage
\eks[1]{
	Finn det bestemte integralet $ \displaystyle \int\limits_{0}^{\frac{\pi}{4}} \frac{8}{1-\sin^2 x}\,dx $.
	
	\sv
	Vi starter med å observere at $ {1-\sin^2 x= \cos^2 x} $. I tillegg vet vi fra (\ref{konstint}) at konstanten 8 kan trekkes utenfor integralet. Vi kan derfor skrive integralet vårt som
	\[ 8\int\limits_{0}^{\frac{\pi}{4}} \frac{1}{\cos^2 x}\,dx \]
	Fra (\ref{tanint}) vet vi at $ \tan x $ er en antiderivert til $ \frac{1}{\cos^2 x} $. Når vi har funnet en antiderivert fører vi gjerne slik\footnote{Forklar for deg selv hvorfor vi ikke trenger å ta hensyn til konstanten når vi skal finne et bestemt integral.}:
	\alg{
		8\int\limits_{0}^{\frac{\pi}{4}} \frac{1}{\cos^2 x}\,dx &= 8\big[\tan x\big]_0^{\frac{\pi}{4}}	\\
		&= 8\left[\tan \frac{\pi}{4}-\tan 0\right] \br
		&= 8[1-0] \\
		&= 8
	}
	
	\textsl{Merk}: Bruken av klammeparantes er bare en annen måte å skrive (\ref{anfund2}) på.\vs
}
\newpage
\eks[2]{
Finn det ubestemte integralet $ \displaystyle \int \left(\frac{1}{x^4}+\sqrt[3]{x}\right)dx $.

\sv
Vi utnytter at $ {\frac{1}{x^4}=x^{-4}} $ og at $ {\sqrt[3]{x}=x^{\frac{1}{3}}} $. Ved (\ref{intfplusg}) og (\ref{xhatr}) kan vi skrive:
\alg{
\int \left(\frac{1}{x^4}+\sqrt[3]{x}\right)dx &= \int \left(x^{-4} +x^{\frac{1}{3}}\right)\,dx \\
&= \frac{1}{-4+1}x^{-4+1}+\frac{1}{\frac{1}{3}+1}x^{\frac{1}{3}+1}+C \br
&= -\frac{1}{3}x^{-3}+\frac{3}{4}x^{\frac{4}{3}}+C
}
\vs
}
\fork{\ref{intuf} \intuf}{
(\ref{intfplusg}) og (\ref{konstint}) følger direkte av (\ref{sumreg1}) og (\ref{sumreg2}).\vsk

Ut ifra definisjonen av det ubestemte integralet (se (\ref{uint})) har vi at
\[ \int f(x)\,dx = F(x)+C \]
hvis $ {F'=f}  $. For alle ubestemte integraler gitt i (\ref{xhatr})-(\ref{tanint}) kan dette sjekkes via enkle derivasjonoperasjoner og er derfor overlatt til leseren.

}


\newpage
\subsection{Bytte av variabel}
Vi skal nå se på en metode som kalles \textit{bytte av variabel}\footnote{Det er flere framgangsmåter for denne metoden. Den vi her presenterer er, etter forfatterens mening, den raskeste for integraler som er pensum i R2. For mer avanserte integraler bør man kjenne til framgangsmåten presentert i \refved{intleibn}.}\index{integrasjon!bytte av variabel} (også kalt \textit{substutisjon}). Med denne kan vi ofte forenkle integralregningen betraktelig.\regv
\reg[\bytvar \label{bytvar}]{
	Gitt funksjonene $ f(x) $, $ u(x) $ og $ g(u) $. Hvis $ \int f(x) \, dx $ kan skrives om til $ \int g(u) u'\, dx  $, kan integralet løses med $ u $ som variabel:  
	\begin{equation}
		\int g(u) u'\, dx=\int g(u) \, du \label{bytvareq}
	\end{equation}\vs
}
\eks[1]{
	Finn det ubestemte integralet 
	\[ \int 8x \sin \left(4x^2 \right) \, dx \] \vs
	\sv
	Vi setter $ {u(x)=4x^2 }$ og ${g(u)=\sin u }$. Dermed blir $ {u'=8x }$, og da er
	\alg{
		\int 8x \sin \left(4x^2 \right)\, dx &= \int u'g(u) \, dx \\
		&= \int g(u) \,du \\
		&= \int \sin u \, du \\
		&= -\cos u + C\\
		&= -\cos \left(4x^2\right) + C
	}
	\textsl{Merk}: Når integralet vi skal finne er mhp. $ x $, er det viktig at sluttuttrykket har $ x $ som eneste variabel.
}
\newpage
\eks[2]{
	Finn det bestemte integralet
	\[\int\limits_0^2 x^2 e^{2x^3}\, dx \] \vs
	\sv 
	Vi setter $ {u(x)=2x^3} $ og $ {g(u)=e^u} $, da blir $ {u'=6x^2 }$. I integralet vi skal løse mangler vi altså faktoren 6 for å kunne anvende oss av \eqref{bytvareq}. Men vi kan alltids gange integralet vårt med 1, skrevet som $ \frac{6}{6} $. Da kan vi trekke 6-tallet vi ønsker inn i integralet, og la resten av brøken forbli utenfor:
	\alg{
		\int\limits_0^2 x^2 e^{2x^3} \, dx &= \frac{6}{6}\int\limits_0^2 x^2 e^{2x^3} \, dx \\
		&= \frac{1}{6}\int\limits_0^2 6x^2 e^{2x^3} \, dx
	}
	Nå ligger alt til rette for å bytte variabel:
	\alg{
		\frac{1}{6}\int\limits_0^2 6x^2 e^{2x^3} \, dx&= \frac{1}{6}\int\limits_0^2 u'g(u) \,dx \\
		&= \frac{1}{6} \int\limits_0^2 g(u)\, du \\
		&= \frac{1}{6} \int\limits_0^2 e^u \, du \\
		&= \frac{1}{6}\big[e^u\big]_0^2 \\
		&= \frac{1}{6}\big[e^{2x^3}\big]_0^2 \\
		&= \frac{1}{6}\left(e{2\cdot2^3-e^{2\cdot0^2}}\right) \\
		&= \frac{1}{6}\left(e^{16}-1\right)
	}
	Det finnes også en alternativ måte for å regne ut bestemte integral ved bytte av variabel, se \refved{bestbyt} for denne.
}
\eks[3]{
	Buelengden til grafen til en funksjon $ f(x) $ på intervallet $ [a, b] $ er gitt som
	\[ \int\limits_a^b\sqrt{1+(f')^2}\,dx \tag{I}\label{Ieks3} \]
	Finn lengden til funksjonen
	\[ f(x)=\frac{1}{3}x^\frac{3}{2}\qquad,\qquad x\in[0, 5] \]
	
	\sv
	Vi har at
	\algv{
	f'= \frac{1}{2}x^{\frac{1}{2}}	
	}
Og videre at
\alg{
	(f')^2=\frac{1}{4}x
}
	Det ubestemte integralet i \eqref{Ieks3} blir da
	\alg{
	\int\limits\sqrt{1+\frac{1}{4}x}\,dx		}
Vi setter $ u=1+\frac{1}{4}x $ og $ g(u)=u^\frac{1}{2} $. Da er $ u'=\frac{1}{4} $. Nå har vi at
	\alg{
	\int\limits\sqrt{1+\frac{1}{4}x}\,dx
	&= 4\int  u^{\frac{1}{2}} u'\,dx \\
	&=4\int u^{\frac{1}{2}}\,du \br
	&=\frac{8}{3}u^{\frac{3}{2}}+C\br
	&=\frac{8}{3}\left(1+\frac{1}{4}x\right)^{\frac{3}{2}}+C
}
Altså er
\alg{
\int\limits_0^5 \sqrt{1+(f')^2}\,dx&=	\frac{8}{3}\left[\left(1+\frac{1}{4}x\right)^{\frac{3}{2}}\right]_0^{5}\br
&=\frac{8}{3}\left(\left(1+\frac{5}{4}\right)^{\frac{3}{2}}-1\right) \br
&=\frac{8}{3}\left(\left(\frac{9}{4}\right)^{\frac{3}{2}}-1\right) \br
&=\frac{8}{3}\left(\frac{27}{8}-1\right)\br
&=\frac{19}{3}
}
\textsl{Merk:} En litt lettere utrekning kunne vi fått ved å observere at
\[ \sqrt{1+\frac{1}{4}x}=
 \frac{1}{2}\sqrt{4+x} \]
Med denne omskrivingen kunne vi valgt substutisjonen \\$ u=4+x $, og dermed fått at $ u'=1 $.
}

\subsection{Delvis integrasjon}
Hvis vi ikke finner et passende bytte av variabel for å løse et integral, kan vi isteden prøve med \textit{delvis integrasjon}\index{integrasjon!delvis}. Vi starter med å utlede ligningen som legger grunnlaget for metoden.\vsk

Gitt produktet av to funksjoner $ u(x) $ og $ v(x) $, altså $ uv $. Av produktregelen ved derivasjon (se \tmen) har vi at
\[ (uv)'=u'v+uv' \]
Videre integrerer\footnote{Når vi har flere ubestemte itegraler, trenger vi bare ta med integrasjonskonstanten for én av dem. Derfor er ikke konstanten fra integrasjonen av $ (uv)' $ tatt med.} vi begge sider av ligningen over mhp. $ x $:
\alg{
	\int (uv)' \, dx &=\int \left(u'v+uv'\right) \, dx \\
	uv &= \int \left(u'v+uv'\right) \, dx \\
	uv -\int u'v \, dx &= \int uv' \, dx
}
\reg[Delvis integrasjon]{ For to funksjoner $ u(x) $ og $ v(x) $ har vi at
	\begin{equation}
		\int uv' \, dx = uv -\int u'v \, dx \label{delvint}
	\end{equation} \vs
}
\eks[1]{
	Integrer funksjonen ${f(x)= x \ln x }$. 
	
	\sv
	Vi observerer at $ f(x) $ er sammensatt av $ x $ og $ \ln x $. Trikset bak delvis integrasjon er å sette én av disse til å være funksjonen $ u(x) $ og den andre til å være den deriverte av $ v(x) $, altså $ v'(x) $. Da har vi en ligning som i (\ref{delvint}) og kan (forhåpentligvis) bruke denne til å finne integralet vi søker.\vsk
	
	Vi må integrere $ v' $ for å finne $ v $ og derivere $ u $ for å finne $ u' $. Siden $ \ln x $ er lett å derivere, men vanskelig å integrere, setter vi
	\alg{
		u &=\ln x \\
		v'&=x
	} 
	Da må vi ha at\footnote{Hvorfor ikke $ {v=\frac{1}{2}x^2 +C} $? Vi hadde jo da fått samme $ v' $. \newline
		
		Hvis vi lar $ V $ betegne en antiderivert til $ v' $, kan vi skrive $ {v=V+C }$. Av (\ref{delvint}) har vi da at
		\alg{
			\int uv'\, dx&=  u(V+C) -\int u'(V+C) \, dx \\
			&=  u(V+C) -\int u'V \, dx-\int Cu' \, dx \\
			&=  uV+Cu -\int u'V \, dx- Cu \\
			&= uV- \int u'V \, dx
		}
		Vi har endt opp med et uttrykk hvor $ C $ ikke lenger deltar. Vi får altså det samme svaret uansett hva verdien til $ C $ er, og da velger vi selvsagt fra starten av at $ {C=0} $.}
	\algv{
		u'&=\frac{1}{x} \br
		v&=\frac{1}{2}x^2 
	}
	Altså kan vi skrive (rekkefølgen på $ v' $ og $ u $ har selvsagt ingenting å si i (\ref{delvint}))
	\alg{
		\int x \ln x \, dx &= \int v'u \, dx \\
		& = uv -\int u'v \, dx \\
		&= \ln x \cdot \frac{1}{2}x^2 - \int \frac{1}{x}\cdot\frac{1}{2}x^2 \, dx \\
		&=  \frac{1}{2}x^2 \ln x- \int \frac{1}{2}x \, dx \\
		&= \frac{1}{2}x^2 \ln x - \frac{1}{4}x^2 + C
	}\vsb
}
\newpage
\eks[2]{
	Integrer funksjonen $f(x)=\ln x $. \\
	
	\sv 
	Vi starter med å skrive $ f(x)=\ln x\cdot 1  $, og setter
	\alg{
		u &= \ln x \\
		v' &= 1
	}
	Vi får da at
	\algv{
		u'&= \frac{1}{x} \\
		v &= x		
	}
	$\int f \,dx$ finner vi nå ved delvis integrasjon:
	\alg{
		\int \ln x\cdot1 \, dx &= \int u v' \,dx \\
		&= uv -\int u'v \, dx \\
		&= x\ln x  - \int x\cdot\frac{1}{x} \, dx \\
		&= x\ln x - x + C \\
		&= x(\ln x- 1)+C
	}\vs
}
\subsection{Delbrøksoppspaltning}
Gitt integralet
\[\int  \frac{4x+5}{(x+1)(x+2)} \,dx \]
Etter litt testing vil vi finne at både delvis integrasjon og bytte av variabel kommer til kort i vår søken etter en antiderivert. Hva vi heller kan gjøre, er å ta i bruk \textit{delbrøksoppspaltning}\index{integrasjon!delbrøksoppspaltning}. \vsk

Vi merker oss da at integranden\index{integrand}\footnote{For $ \int f(x)\,dx $ sier vi at $ f $ er \textit{integranden}.} er en brøk med nevneren $ (x+1)(x+2) $. Dette betyr at den kan skrives som to separate brøker med $ (x+1) $ og $ (x+2) $ som nevnere:
\begin{equation}
\frac{4x+5}{(x+1)(x+2)} = \frac{A}{x+1}+\frac{B}{x+2} \label{delbr}
\end{equation}
$ A $ og $ B $ er to konstanter, vår oppgave blir nå å bestemme verdien til disse.\vsk

Vi starter med å gange begge sider av (\ref{delbr}) med fellesnevneren:
\alg{
	\frac{4x+5}{(x+1)(x+2)}(x+1)(x+2) &= \left(\frac{A}{x+1}+\frac{B}{x+2}\right)(x+1)(x+2) \br
	4x+5 &= A(x+2)+B(x+1)
	}
For det rette valget av $ A $ og $ B $ er uttrykkene over like for alle verdier av $ x $. Når $ {x=-1} $, har vi bare $ A $ som ukjent:
\alg{
	4\cdot(-1)+5 &= A(-1+2)+B(-1+1)\\
	1 &= A	
	}
Og ved å sette $ x=-2 $, finner vi $ B $:
\alg{
	4\cdot(-2) +5&= A(-2+2)+B(-2+1) \\
	-3 &= -B \\
	3 &= B	
	}
Nå kan vi altså skrive
\[ \frac{4x+5}{(x+1)(x+2)} = \frac{1}{x+1}+\frac{3}{x+2}  \]
Dette er to brøker vi kan å integrere\footnote{\textsl{Obs!} I søken etter $ A $ og $ B $ valgte vi verdiene ${ x=-1 }$ og $ {x=-2} $. I ligningene hvor vi satte inn disse verdiene var dette helt uskyldig, men i integralet må vi være observante. Vi får nemlig 0 i nevner hvis én av disse verdiene ligger i intervallet vi skal integere over. Er det snakk om et bestemt integral må vi derfor passe på at dette ikke er tilfelle.} (se (\ref{lnint})):
\alg{
\int \frac{4x+5}{(x+1)(x+2)}\,dx &= \int \left(\frac{1}{x+1}+\frac{3}{x+2}\right)\,dx \\
&= \ln |x+1| + 3\ln|x+2| 	
	}
\newpage
\reg[Integrasjon ved delbrøksoppspaltning ]{
	For integraler på formen
	\[ \int \frac{a+bx+cx^2+...}{(x-d)(x-e)(x-f)...}\,dx\]
	hvor $ a, b, c, ... $ er konstanter, skriver vi om integranden til
	\[ \frac{A}{(x-d)}+\frac{B}{(x-e)}+\frac{C}{(x-f)}+... \]
	og finner så de ukjente konstantene $ A, B, C, ... $
}
\eks[1]{
	Finn det ubestemte integralet
	\[ \int \frac{3x^2+3x+2}{x^3-x}\,dx \]	\vs
	\sv
	Vi starter med å faktorisere nevneren i integranden, og får at
	\[ \frac{3x^2+3x+2}{x^3-x} = \frac{3x^2+3x+2}{x(x+1)(x-1)} \]
	Denne brøken ønsker vi å skrive som
	\[ \frac{3x^2+3x+2}{x(x+1)(x-1)} =  \frac{A}{x}+\frac{B}{x+1}+\frac{C}{x-1} \]
	For å finne $ A $, $ B $ og $ C $, omskriver vi ligningen ved å gange med fellesnevneren $ x(x+1)(x-1) $:
	\[ 	3x^2+3x+2 =  A(x+1)(x-1)+ Bx(x-1)+ Cx(x+1)	 \]
	Ligningen må holde for alle verdier av $ x $. Vi setter først $ {x=0 }$, og får at
	\alg{
		2 &= A\cdot(-1) \\
		-2&=A 
	}
	Videre setter vi $ {x=-1 }$:
	\alg{
		3\cdot(-1)^2+3(-1)+2 &= B\cdot(-1)(-1-1) 	\\
		1 &= B
	}
	Til slutt setter vi $ {x=1} $:
	\alg{
		3\cdot1^2+3\cdot1+2 &= C(1+1)	\\
		4&= C
	}
	Integralet vi skal finne kan vi derfor skrive som
	\small
	\[ \int \left(-\frac{2}{x}+\frac{1}{x+1}+\frac{4}{x-1}\right)\,dx = -2\ln |x|+\ln|x+1|+4\ln|x-1|+D \] \vs
}
\eks[2]{\label{delbre2}
Finn det ubestemte integralet
\[ \int \frac{x^3 + 5 x^2 + x - 4}{x^2+x-2}\,dx\] \vs
\sv

Hvis telleren har potenser av høyere orden\footnote{Her har telleren tre som høyeste orden, mens nevneren har to.} enn nevneren, må vi starte med en polynomdivisjon:

\begin{align*}
\phantom{-}&(x^3 + 5 x^2 + x - 4):(x^2+x-2) =x+4+\frac{-x+4}{x^2+x-2} \\ 
-&\underline{(x^3+x^2-2x)} \\
&\phantom{(x^3)}4x^2+3x-4 \\
&\phantom{\,}-\underline{(4x^2+4x-8)}\\
&\phantom{aaaaaaa\,}-x+4
\end{align*}
Vi observerer videre at nevneren i brøken kan omskrives til $ (x-1)(x+2) $, for to konstanter $ A $ og $ B $ har vi altså at
\alg{
	\frac{A}{x-1}+\frac{B}{x+2}&=\frac{-x+4}{x^2+x-2}  \br
	A(x+2)+B(x-1)&=-x+4
}
Når $ {x=-2} $, får vi at
\alg{
	B(-2-1)&=-(-2)+4 \\
	B &= -2
}
Og når ${ x=1} $, er
\alg{
	A(1+2) &= -1+4 \\
	A &= 1
}
Integralet blir derfor
\[ \int \left(x+4+\frac{1}{x-1}-\frac{2}{x+2}\right)\,dx
= \frac{1}{2}x^2+4x + \ln |x-1|-2\ln|x+2|+C \] \vs
}
\begin{comment}
\begin{multline*}
\int \left(x+4+\frac{1}{x-1}-\frac{2}{x+2}\right)\,dx
= \frac{1}{2}x^2+4x + \ln (x-1)\\-2\ln(x+2)+C
\end{multline*}
\end{comment}
\section{Areal og volum}
\subsection{Avgrenset areal}
Arealet avgrenset av grafen til $ f $, $ x $-aksen, og linjene $ {x=a} $ og $ {x=b} $ skal vi for enkelhetsskyld kalle \outl{arealet avgrenset av $ f $ for $x\in [a, b] $}. I \refsec{BestemtogubestemtI} har vi sett at det er en sterk sammenheng mellom dette arealet og det bestemte ingegralet av $ f $:\regv
 
\reg[Integral som areal I]{\index{integral!som areal}Gitt en kontinuerlig funksjon $ f(x) $ og to tall $ a $ og $ b $ der $ {a<b }$. \vsk
	
	Hvis $ {f\geq0} $ for  ${x\in [a, b]} $, er arealet $ A $ avgrenset av $ f $ på dette intervallet gitt som
	\[ A=\int\limits_a^b f \,dx \] \vsb
	\fig{intpos}
	Hvis $ {f\leq0 }$ for ${x\in [a, b]} $, er arealet $ A $ avgrenset av $ f $ på dette intervallet gitt som
	\[ A=-\int\limits_a^b f \,dx \] \vsb
	\fig{intneg}
}
\newpage
\textbf{Areal avgrenset av to funksjoner}\bs
Noen ganger ønsker vi også å finne arealet avgrenset av to funksjoner. Da må vi sørge for at vi har tilstrekkelig med informasjon om disse før vi utfører integrasjonen:\regv
\reg[Integral som areal II]{
	Gitt to kontinuerlige funksjoner $ f(x) $ og $ g(x) $ og tre tall $ a $, $ b $ og $ c $ der $ {a<c<b }$. \vsk
	
	Hvis $ {f>g} $ for ${x\in [a, b]} $, er arealet $ A $ avgrenset mellom $ f $ og $ g $ på dette intervallet gitt ved
	\begin{equation}\label{foggintr}
		A = \int\limits_{a}^{b} (f-g)\,dx
	\end{equation} 
	\begin{figure}[H]
		\centering
		\includegraphics[]{\figp{foggint}}	
	\end{figure}
	Hvis $ {f\geq g} $ for ${x\in [a, c]} $ og $ {g\geq f} $ for ${x\in [c, b]} $, er arealet $ A $ avgrenset mellom $ f $ og $ g $ for ${x\in [a, b]} $ gitt ved
	\begin{equation}\label{foggintr2}
		A = \int\limits_{a}^{c} (f-g)\,dx + \int\limits_{c}^{b} (g-f)\,dx
	\end{equation} 
	\begin{figure}[H]
		\centering
		\includegraphics[]{\figp{foggint2}}	
	\end{figure}
}
\newpage
\eks{
Gitt funksjonene $ {f(x)=\sin\left(\frac{\pi}{2}x\right)} $ og $ {g(x)=2x-1} $. Vi har da at $ {f\geq g} $ for ${ x\leq 1} $ og at $ {g\geq f} $ for $ {x\geq1} $. Finn arealet $ A $ avgrenset av $ f $ og $ g $ for $ x\in [0, 2] $.  \\

\sv
Ut ifra informasjonen over er arealet gitt ved ligningen
\alg{
A &= \int\limits_{0}^{1} (f-g)\,dx + \int\limits_{1}^{2} (g-f)\,dx 
}
Vi starter med å regne ut de to integralene hver for seg:
\alg{
\int\limits_{0}^{1} (f-g)\,dx&= \left[-\frac{2}{\pi}\cos\left(\frac{\pi}{2}x\right)-(x^2-x)\right]_0^1\\
&= -\left[\frac{2}{\pi}\cos\left(\frac{\pi}{2}x\right)+(x^2-x)\right]_0^1\br
&= \frac{2}{\pi}\cos\left(\frac{\pi}{2}\cdot1\right)+(1^2-1) \\ &\qquad\qquad-\left(\frac{2}{\pi}\cos\left(\frac{\pi}{2}\cdot0\right)+(0^2-0)\right)\\
&= \frac{2}{\pi}
} \vs
\alg{
\int\limits_{1}^{2} (g-f)\,dx&=\left[(x^2-x)+\frac{2}{\pi}\cos\left(\frac{\pi}{2}x\right)\right]_1^2\\
&=(2^2-2)+\frac{2}{\pi}\cos\left(\frac{\pi}{2}\cdot2\right) \\ &\qquad\qquad-\left(-\frac{2}{\pi}\cos\left(\frac{\pi}{2}\cdot1\right)-(1^2-1)\right) \\
&=2 -\frac{2}{\pi}
}
Summen av disse to integralene er $ 2 $, som altså er arealet.
}

\newpage
\subsection{Volumet av en figur}
Vi har sett hvordan integraler kan brukes til å finne arealer, men de kan også brukes til å finne volumer:\regv
\reg[\intvol \label{intvol}]{\index{integral!som volum}
	Gitt en tredimensjonal figur plassert i et koordinatsystem, med endepunktene satt til verdiene $ a $ og $ b $ langs $ x $-aksen.
	\begin{center}
		\centering
		\includegraphics[scale=0.7]{\figp{test222}}
	\end{center}
	La videre $ A(x) $ være tverrsnittsarealet av figuren for verdien $ x $. Volumet $ V $ av figuren er da gitt som
	\begin{equation}
		V = \int\limits_a^b A\,dx	\label{intvoleq}
	\end{equation}
}
\eks[]{
	Vis at volumet $ V $ av ei rett kjegle er gitt som
	\[ V=\frac{1}{3}\pi h r^2 \]
	hvor $ r $ er radiusen til grunnflata og $ h $ er høgden til kjegla.\\	
	
	\sv
	Vi plasserer kjegla inn i et koordinatsystem med høyden langs $ x $-aksen og spissen plassert i origo. 
	\begin{figure}[H]
		\centering
		\includegraphics[]{\figp{kjegle}}
	\end{figure}
	Radiusen $ r_t(x) $ kan beskrives som en rett linje med stigningstall $ \frac{r}{h} $: 
	\[ r_t(x) = \frac{r}{h}x \]
	Arealet $ A(x) $ av tverrsnittet blir da
	\alg{A(x) &= \pi r_t^2 \\
		&= \pi\left(\frac{r}{h}\right)^2x^2
	}
	Altså er volumet av kjegla gitt som
	\alg{
		\int\limits_0^h A\, dx &=\int\limits_0^h \pi \left(\frac{r}{h}\right)^2 x^2\, dx  \\
		&=\pi\frac{r^2}{h^2}\int\limits_0^h x^2\, dx \\
		&= \pi\frac{r^2}{h^2}\left[\frac{1}{3}x^3\right]_0^h	\\
		&= \frac{1}{3}\pi h r^2
	}\vs
}
\newpage
\subsection{Volum av omdreiningslegemer}
Si vi har en funksjon $ f(x) $ gitt på intervallet $ [a, b] $, med en graf som vist i \fref[a]{omdr}. Tenk nå at vi dreier linjestykket $ 360^\circ $ om $ x $-aksen. Formen vi da har ''skjært'' ut, vist i \fref[b]{omdr}, er det vi kaller \textit{omdreiningslegemet}\index{omdreiningsleme} til $ f(x) $ på intervallet $ [a, b] $.
\begin{figure}[H]
	\centering
\subfloat[a)]{
	\includegraphics[]{\figp{omdr2}}
} \\
\subfloat[b)]{
	\includegraphics[]{\figp{omdr3}}
	}
	\captionof{figure}{\textsl{a)} Grafen til $ f $. \textsl{b)} Omdreiningslegemet til $ f $. \label{omdr}}
\end{figure}

Tverrsnittet (langs $ x $-aksen) til en slik figur er alltid sirkelformet,\\ tverrsnittsarealet er derfor $ \pi r^2 $, hvor $ r(x) $ er radiusen til tverrsnittet. Men siden radiusen tilsvarer høyden fra $ x $-aksen opp til $ f $, er $ {r=f}$. Av \eqref{intvoleq} kan vi da skrive
\[ V = \int\limits_a^b A \,dx 
=\int\limits_a^b \pi f^2 \,dx =
\pi\int\limits_a^b f^2 \,dx \]
\reg[Volum av omdreiningslegemer]{\index{omdreiningslegeme!volumet av}Volumet $ V $ av omdreiningslegemet til $ f(x) $ på intervallet $ [a, b] $ er gitt som
	\begin{equation}
		V = \pi\int\limits_a^b f^2\,dx 
	\end{equation}	\vs
}
\newpage
\eks{
	Gitt funksjonen \[ f(x)=\sqrt{x} \]
	finn volumet av omdreiningsleget til $ f $ på intervallet $ [1, 3] $. 
	
	\sv
	Volumet vi søker er gitt som
	\alg{
		\pi\int\limits_1^3 f^2\,dx &= \pi\int\limits_1^3 \left(\sqrt{x}\right)^2\,dx \\
		&= \pi\int\limits_1^3 x\,dx \\
		&= \pi\left[\frac{1}{2}x^2\right]_1^3 \\
		&= \frac{\pi}{2} \left[9-1\right] \\
		&= 4\pi
	}\vs
}
\newpage
\tsec{Forklaringer}
\fork{\ref{bestminten} \bestminten}{
\label{bintforklaring}
\figc{int5}{\label{Ifig} Integralet $ I $ tilsvarer det avgrensede arealet i grønt.}
La oss ta utgangspunkt i funksjonen $ f(x) $, med en graf som vist i \fref{Ifig}. Vårt mål er nå å finne $ I $.\vsk

Vi starter med å splitte $ [a, b] $ inn i $ n $ mindre delintervaller, alle med bredden $ {\Delta x = \frac{b-a}{n}} $. I tillegg lar vi $ x_i $ for $ {i\in\lbrace 1, 2,\,...\,, n\rbrace} $ betegne den $ x $-verdien som er slik at $ f(x_i) $ er den laveste verdien til $ f$ på delintervall nr. $ i $. \vsk

Arealet avgrenset av delintervallet og $ f $ tilnærmer vi som $s_i= f(x_i)\Delta x $, da har vi at (se \fref{bintforkl})
\alg{
I &\geq s_1+s_2+...+s_i\\
I&\geq f(x_1)\Delta x + f(x_2)\Delta x +...+f(x_n)\Delta x  \\
I &\geq \sum\limits_{i=i}^{n} f(x_i)\Delta x	
}
\figc{int4}{Arealene av $ s_i $ markert som grønne søyler og arealene av $ c_i $ markert som blå søyler. Bredden til hver søyle er $ \Delta x = \frac{b-a}{n} $ (her er \y{n=4 }).\label{bintforkl}}
Videre må det finnes et tall $ {h_i\in [0, 1)} $ som er slik at $ f(x_i+h_i\Delta x) $ er den høyeste verdien til $ f $ på delintervallet. Vi lar $ c_i $ betegne arealet til søylen med $ \Delta x $ som bredde og  $ f(x_i+h_i\Delta x)-f(x_i)$ som høyde:
\[ c_i = (f(x_i+h_i\Delta x)-f(x_i))\Delta x \]
Hvis vi legger til alle $ c_i $ i det første estimatet vårt, får vi en tilnærming som må være større eller lik det egentlige arealet. Derfor kan vi skrive
\[\sum\limits_{i=1}^{n} f(x_i)\Delta x \leq I \leq \sum\limits_{i=1}^{n} f(x_i)\Delta x+ \sum\limits_{i=1}^{n} c_i \]
Én av $ c $-verdiene må være større eller lik alle andre $ c $-verdier. Vi lar $ m $ betegne indeksen til nettopp denne $ c $-verdien. Da må vi ha at
\[0\leq \sum\limits_{i=1}^{n} c_i \leq nc_{m} \]  
Men når \y{n\to\infty}, går summen $ nc_m $ mot 0:
\alg{
\lim\limits_{n\to\infty}  n c_{m}&=\lim\limits_{n\to\infty} n(f(x_{m}+h_{m}\Delta x)-f(x_{m}))\Delta x	\\
&= \lim\limits_{n\to\infty} n(f(x_{m}+h_{m}\Delta x)-f(x_{m}))\frac{b-a}{n} \\
&= \lim\limits_{n\to\infty} (f(x_{m}+h_{m}\Delta x)-f(x_{m}))(b-a)\\
&= \lim\limits_{n\to\infty} (f(x_{m})-f(x_{m}))(b-a) \\
&= 0	
}
\newpage
Følgelig er $ \lim\limits_{x\to\infty} \sum\limits_{i=1}^{n} c_i =0,$ og da er
\begin{align}
\lim\limits_{n\to\infty}\sum\limits_{i=1}^{n} f(x_i)\Delta x \leq &I \leq \lim\limits_{n\to\infty}\left(\,\sum\limits_{i=1}^{n} f(x_i)\Delta x+ \sum\limits_{i=1}^{n} c_i\right) \label{bintforkl1} \br
\lim\limits_{n\to\infty}\sum\limits_{i=1}^{n} f(x_i)\Delta x \leq &I\leq \lim\limits_{n\to\infty}\sum\limits_{i=1}^{n} f(x_i)\Delta x \\
&I= \lim\limits_{n\to\infty}\sum\limits_{i=1}^{n} f(x_i)\Delta x	\label{bintforkl2}
\end{align}
Det vi har kommet fram til nå er vel og bra, men skal vi regne ut et integral blir det slitsomt å inspisere $ f(x) $ på uendelig mange delintervaller for å finne de laveste funksjonsverdiene i hver av dem! Vi merker oss derfor at venstresiden i (\ref{bintforkl1}), i vårt tilfelle, representerer det kraftigste underestimatet av $ I $, mens høyresiden er det kraftigste overestimatet. I (\ref{bintforkl1})-(\ref{bintforkl2}) har vi vist at begge disse estimatene går mot $ I $ når $ {n\to\infty} $, dette betyr at vi for andre valg av $ x_i $ på hvert intervall også kommer fram til ønsket resultat. Regneteknisk vil det ofte være lurt å velge $ x_i=a+(i-1)\Delta x $ for $ {i\in\lbrace 1, 2,\,...\,, n\rbrace} $, slik som i (\ref{bint}).\vsk

\textbf{Integral som areal for negative funksjoner}\bs
Hva nå om vi isteden skulle finne arealet avgrenset av $ x $-aksen, linjene $ {x=a} $ og $ {x= b }$ og grafen til $ {g(x) = -f(x)} $? \vsk

Grafene til $ f $ og $ g $ er fullstendig symmetriske om $ x$-aksen, dette må bety at arealet $ A $ de avgrenser på et intervall må være helt likt. Og vi vet at\vs
\alg{
A &= \lim\limits_{n\to\infty}\sum\limits_{i=1}^{n} f(x_i)\Delta x \\
&= 	\lim\limits_{n\to\infty}\sum\limits_{i=1}^{n} -g(x_i)\Delta x \\
&= -\lim\limits_{n\to\infty}\sum\limits_{i=1}^{n} g(x_i)\Delta x
}
Av dette kan vi utvide den geometriske definisjonen av integralet:\vsk

\textsl{Gitt en funksjon $ f(x) $ som er negativ og kontiunerlig for alle $ {x\in[a, b] }$. Integralet $ I $ multiplisert med $ -1 $ tilsvarer arealet avgrenset av $ x $-aksen, linjene $ {x=a} $ og ${ x= b} $ og grafen til $ f $.}
\label{bintslutt}
} \vsk

\fork{\ref{anfundteo} \anfundteo}{
Vi ønsker å vise at integralet $ I $ av en funksjon $ f(x) $ på intervallet $ [a, b] $ er gitt som
\[ I = F(b)-F(a) \]
hvor $ F $ er en antiderivert til $ f $. For å vise dette skal vi anvende oss av \eqref{bint}. Spesielt verdt å merke seg er at $ x_1=a $ og at $ x_{n+1}=b $.\vsk

Fra tidligere vet vi at den deriverte av en funksjon $ f(x) $ er gitt som
\[ f'(x) = \lim\limits_{\Delta x\to 0}\frac{f(x+\Delta x)-f(x)}{\Delta x} \]
Med vår $ \Delta x=\frac{b-a}{n} $ kan vi omskrive grensen:
\[  f'(x)=\lim\limits_{n\to \infty}\frac{f(x+\Delta x)-f(x)}{\Delta x} \]
La $ F(x) $ være en antiderivert til $ f(x) $, da er
\[ F'(x)=f(x)=\lim\limits_{n\to \infty}\frac{F(x+\Delta x)-F(x)}{\Delta x}  \]
Vi erstatter $ f $ i \eqref{bint} med uttrykket over, og får at
\alg{I=
	\lim\limits_{n\to \infty}\sum\limits_{i=1}^{n} \frac{F(x_i+\Delta x)-F(x_i)}{\Delta x} \Delta x 
}
Fordi $ {x_{i+1}=x_i+\Delta x} $, har vi videre at
\alg{
	I&=	\lim\limits_{n\to \infty}\sum\limits_{i=1}^{n} \left(F(x_{i+1})-F(x_i)\right) \\
	&= \lim\limits_{n\to\infty} \left(F(x_2)-F(x_1)+F(x_3)-F(x_2)+...+F(x_{n+1})-F(x_{n})\right)
}
Av dette legger vi merke til at alle $ F(x_i) $ kansellerer hverandre, bortsett fra i endepunktene. Vi sitter altså igjen med summen
\alg{
	I &=\lim\limits_{n\to\infty}\left(-F(x_1)+ F(x_{n+1})\right) \\
	&= F(b)-F(a)
}
}


\fork{\ref{bytvar} \bytvar}{
Gitt en funksjon $ F(x) $ som vil anta samme verdier som $ G(u(x)) $:
\begin{equation}
	F(x)=G(u)  
\end{equation}
La oss nå skrive $ F'(x) $ som $ f(x) $ og $ G'(u) $ som $ g(u) $. For to konstanter $ C $ og $ D $ må vi ha at
\alg{
	\int f(x)\, dx &= F(x)+C \\
	&\text{og} \\
	\int g(u)\, du &= G(u)+D 
} Det må derfor finnes en konstant $ E $ som er slik at
\[ \int f(x)\, dx+E=\int g(u)\, du\]

Men av kjerneregelen (se \tmen) har vi følgende relasjon:
\[ f(x)= g(u)u'   \]
Vi kan derfor skrive
\[ \int g(u)u' \, dx+E =\int g(u)\, du\]
Når vi utfører integrasjonen på enten venstre eller høyre side, får vi en ny konstant som vi kan slå sammen med $ E $. I praksis kan vi derfor utelate $ E $, noe som er gjort i (\ref{bytvar}).
}
\newpage
\fork{\ref{intvol} \intvol}{
Vi setter geometrien vår inn i et koordinatsystem, og tar for gitt at vi har en funksjon $ A(x) $ som gir oss tverrsnittsarealet for alle gyldige $ x $.
\figc{test22}{Volumet av geometrien (gul) tilnærmes ved summen av hver $ A(x_i)\Delta x $ (blå).\label{volgjen}}
Vi deler $ [a, b] $ inn i $ n $ delintervaller, der hvert intervall har lengden $ {\Delta x=\frac{b-a}{n}} $ og startverdi $ {x_i=a+(i-1)\Delta x}$ for $ {i\in\lbrace 1, 2, ...\, , n\rbrace }$. Vi tilnærmer volumet til geometrien ved å legge sammen volumene på formen $ A(x_i)\Delta x  $. Når vi lar $ n $ gå mot uendelig vil summen gå mot volumet til gjenstanden\footnote{Argumentasjonen for denne påstanden blir identisk med den gitt i forklaringen for det bestemte integralet (se side \pageref{bintforklaring}).}, dette kan vi skrive som
\alg{
	V &=\lim\limits_{x\to \infty} \left(A(x_0)\Delta x + A(x_1)\Delta x +  ... +A(x_{n})\Delta x\right) \\
	&=\lim\limits_{x\to \infty} \left(A(x_0)+ A(x_1) +  ... +A(x_{n})\right)\Delta x \\
	&= \lim\limits_{x\to \infty} \sum\limits_{i=1}^{n} A(x_i) \Delta x
}
Uttrykket over er analogt til definisjonen av det bestemte integralet fra ligning (\ref{bint}).
}



\end{document}