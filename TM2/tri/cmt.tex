\documentclass[english,hidelinks,pdftex, 11 pt, class=report,crop=false]{standalone}
\usepackage[T1]{fontenc}
\usepackage[utf8]{luainputenc}
\usepackage{lmodern} % load a font with all the characters
\usepackage{geometry}
\geometry{verbose,a4paper, inner=0cm, outer=0 cm, bmargin=2cm, tmargin=1cm}
%\textwidth=12cm
\setlength{\parindent}{0bp}
\usepackage{import}
\usepackage[subpreambles=false]{standalone}
\usepackage{amsmath}
\usepackage{amssymb}
\usepackage{esint}
\usepackage{babel}
\usepackage{tabu}
\usepackage[dvipsnames, table]{xcolor}
\usepackage{cancel}
\makeatother
\makeatletter
\usepackage{datetime2}
\usepackage{titlesec}
\usepackage[many]{tcolorbox}

% Eheter
\newcommand{\enh}[1]{\,\textrm{#1}}
%referances
\newcommand{\net}[2]{{\color{blue}\href{#1}{#2}}}

%Spaces
\newcommand{\vsk}{\\[12pt]}
\newcommand{\vs}{\vspace{-12pt}}

% Tabell for opplegg

\newcommand{\ovlist}[1]{
\vspace{-16pt}
\begin{itemize}
	#1
\end{itemize}
}

% Chapters and sections
\titleformat{\section}[block]{\bfseries}{\hspace{3cm}\thesection}{5pt}{}
\titleformat{\subsection}[block]{\bfseries}{\hspace{3cm}\thesection}{5pt}{}
\newcommand{\sectionbreak}{\clearpage} % New page on each section
 

\newlength{\mywidth}
\setlength{\mywidth}{14cm}

\newcommand{\cont}[1]{
\begin{tcolorbox}[center, boxrule=0.0 mm, width=\mywidth,arc=0mm,enhanced jigsaw,,colback=white,breakable]
#1	
\end{tcolorbox}
}

\newcommand{\info}[5]{
\begin{tcolorbox}[center, boxrule=0.1 mm, width=\mywidth,arc=0mm,enhanced jigsaw,breakable,colback=yellow!5]	
	
	\footnotesize
	\textbf{Øvingsområde}\\[5pt] #1 
	
	\textbf{Utstyr}\\ #2  \\
	
	\begin{tabular}{@{} p{4cm} p{4cm} l} 
		\textbf{Tid} & \textbf{Elevinndeling} & \textbf{Læringsarena} \\
		#3  & #4 & #5
	\end{tabular} 
\end{tcolorbox}	
}

\newcommand{\gjen}[1]{\begin{tcolorbox}[center,boxrule=0.1 mm, width=\mywidth,arc=0mm,colback=blue!3] {\large \textbf{Gjennomføring} \vspace{5 pt}}\newline #1  \end{tcolorbox}\vspace{-5pt}}
\newcommand{\eks}[1]{\begin{tcolorbox}[center,boxrule=0.1 mm, width=\mywidth,arc=0mm,colback=green!3] {\large \textbf{Eksempel} \vspace{5 pt}}\newline #1  \end{tcolorbox}\vspace{-5pt}}

\newcounter{opl}
%\numberwithin{opl}{article}


\newcommand{\opl}[1]{
\newpage
{\refstepcounter{opl} %\phantomsection 
\large \textbf{\theopl \;#1} \vsk}
}

% Headlines
\newcommand{\fork}{\textbf{Forkunnskapar}\\}
\newcommand{\forb}{\textbf{Forberedelsar}\\}
\newcommand{\opgvr}{\textbf{Oppgaver}}



%colors
\newcommand{\colr}[1]{{\color{red} #1}}
\newcommand{\colb}[1]{{\color{blue} #1}}
\newcommand{\colo}[1]{{\color{orange} #1}}
\newcommand{\colc}[1]{{\color{cyan} #1}}
\definecolor{projectgreen}{cmyk}{100,0,100,0}
\newcommand{\colg}[1]{{\color{projectgreen} #1}}

% Lister med bokstavar
\usepackage[inline]{enumitem}
% Opg
\newcommand{\abc}[1]{
	\begin{enumerate}[label=\alph*),leftmargin=18pt]
		#1
	\end{enumerate}
}

\usepackage[]{hyperref}
\begin{document}
\section*{Kommentar}
I enkelte tekster som omtaler trigonometriske funksjoner finner man formuleringer som denne:
\alg{
&f(x)=\sin x\quad,\quad x\in[0, 2\pi] \tag{I}\label{I}\\
&g(x)=\sin x \quad,\quad x\in[0^\circ, 360^\circ] \tag{II} \label{II}.
}
Dette skaper det feilaktige bildet av at $ f $ og $ g $ er den samme funksjonen, med uttrykket $ \sin x $, men at det er opp til oss å velge om $ x $ er et tall eller vinkelmålet grader.\vsk

Det er viktig å innse at de trigonometriske funskjonene vi nå har introdusert, er funksjoner som bare kan ha \textit{tall} som argumenter $ - $ $ x $ kan ikke bære enheter som grader, meter o.l. Men det kan selvfølgelig være at man ønsker å la $ x $ \textsl{representere} grader, en korrekt måte å skrive $ g $ på er da
\[ g(x)=\sin^\circ x\quad,\quad x\in[0, 360]\]
hvor $ ^\circ $ indikerer at $ g $ er sinusverdien til $ x $ grader. Relasjonen mellom $ \sin^\circ x $ og $ \sin x $ er
\[ \sin^\circ x =\sin\left(\frac{\pi}{180}x\right) \]
Selv om vi enda ikke har studert den deriverte av sinusfunksjoner, bør du allerede nå (via kjerneregelen) ane at $ (\sin^\circ x )'\neq(\sin x)'$. Å presentere $ f $ og $ g $ med like uttrykk, som i (\ref{I})  og (\ref{II}), blir derfor helt feil.\vsk

Når det for eksempel skrives $ \sin 45^\circ $, menes det altså strengt tatt $ \sin^\circ 45 $. Likevel skal vi bruke denne skrivemåten i neste kapittel fordi den er så utbredt. For å ha alt på det tørre, definerer vi her og nå at symbolet $ ^\circ $ rett og slett ikke er noe annet enn brøken\footnote{Dette er i samsvar med (\ref{gradertilrad}).} $ \frac{\pi}{180} $. På denne måten blir:
\[ \sin x^\circ= \sin\left(\frac{\pi}{180}x\right)=\sin^\circ x \]
\small
\begin{comment}
	Før vi forlater pedanteriet må det også være nevnt at den utbredte introduksjonen av trigonometriske funksjoner via vinkelmål ikke må forlede en til å tro at funksjonene alltid har noe med konkrete vinkler å gjøre. For eksempel har trigonometriske funksjoner en enestående rolle i beskrivelsen av bølgefenomener og svingninger\footnote{Svingninger skal vi se på i kapittelref??}. Spenner du man genstand fast i en fjør og trekker i den slik at den svinger loddrett opp og ned, kan man modellere bevegelsen som en sinusfunksjon varierende med tiden. Det vil da ikke være noen slags vinkel med i denne beskrivelsen.
\end{comment}
\end{document}


