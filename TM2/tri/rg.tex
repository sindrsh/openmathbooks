\newcommand\rad{
	\rg[Relasjonen mellom grader og radianer]{\vspace{-5 pt}
		\begin{equation}\label{gradertilrad}
		1^\circ\ = \frac{\pi}{180}
		\end{equation}	\vs
	}	
}
\newcommand{\sincos}{}
\newcommand{\tang}{}
\newcommand{\arcus}{\rg[Arcusuttrykkene]{
		Uttrykket
		\nreq{\text{atri}\;x = d}
		hvor tri erstattes med $ \sin, \cos $ eller $ \tan $, betyr at
		\nreq{\text{tri}\;d=x}
		\vds
	}
}
\newcommand{\arcuse}{	}
\newcommand{\trien}{\rg[Trigonometriske identiteter]{ 
		\vds
		\begin{align} 
		\cos (u+v) &=\cos u \cos v - \sin u \sin v \label{cuv} \\
		\cos (u-v) &=\cos u \cos v + \sin u \sin v \label{cu-v} \\
		\sin (u+v) &= \sin u \cos v +\cos u \sin v \label{suv} \\
		\sin (u-v) &= \sin u \cos v - \cos u \sin v \\
		\cos(-x) &= \cos x \label{-cosxcosx}\\
		\sin(-x) &= -\sin x \\
		\cos(x\pm\pi)&=-\cos x \\
		\sin(x\pm\pi)&=-\sin x \\
		\sin (2x) &= 2\cos x \sin x \label{sin2x}\\			
		\cos^2 x + \sin^2 x &= 1 \label{1}\br
		\cos\left(u-\frac{\pi}{2}\right)&=\sin u \label{cossomsi}\br
		\sin\left(u+\frac{\pi}{2}\right)&=\cos u \label{sinsomcos}		
		\end{align}
		\vs
		}}
\newcommand{\triene}{}
\newcommand{\kvad}{}
\newcommand{\coslig}{
}
\newcommand{\coslige}{	
	}
\newcommand{\cosligeto}{}
\newcommand{\sinlig}{
}
\newcommand{\sinlige}{	}
\newcommand{\tanlig}{
}
\newcommand{\tanlige}{	}	
\newcommand{\kvaden}{
	}
\newcommand{\kvadene}{}
%Funksjoner		
\newcommand{\sinf}{}
\newcommand{\rel}{\rg[Relasjonene mellom sinus- og cosinusfunksjoner]{
		\vspace{-11 pt}
		\begin{align}
		&\cos \left(kx+c -\frac{\pi}{2}\right)= \sin(kx+c)	\\[5 pt]
		& \sin\left(kx +c + \frac{\pi}{2}\right)= \cos(kx+c)
		\end{align}
	}}
\newcommand{\komb}{}	
\newcommand{\kombe}{	
	}				
\newcommand{\asinbcos}{ 
}
\newcommand{\asinbcosd}{}

\newcommand{\tanf}{}