\documentclass[english,hidelinks,pdftex, 11 pt, class=report,crop=false]{standalone}
\usepackage[T1]{fontenc}
\usepackage[utf8]{luainputenc}
\usepackage{lmodern} % load a font with all the characters
\usepackage{geometry}
\geometry{verbose,a4paper, inner=0cm, outer=0 cm, bmargin=2cm, tmargin=1cm}
%\textwidth=12cm
\setlength{\parindent}{0bp}
\usepackage{import}
\usepackage[subpreambles=false]{standalone}
\usepackage{amsmath}
\usepackage{amssymb}
\usepackage{esint}
\usepackage{babel}
\usepackage{tabu}
\usepackage[dvipsnames, table]{xcolor}
\usepackage{cancel}
\makeatother
\makeatletter
\usepackage{datetime2}
\usepackage{titlesec}
\usepackage[many]{tcolorbox}

% Eheter
\newcommand{\enh}[1]{\,\textrm{#1}}
%referances
\newcommand{\net}[2]{{\color{blue}\href{#1}{#2}}}

%Spaces
\newcommand{\vsk}{\\[12pt]}
\newcommand{\vs}{\vspace{-12pt}}

% Tabell for opplegg

\newcommand{\ovlist}[1]{
\vspace{-16pt}
\begin{itemize}
	#1
\end{itemize}
}

% Chapters and sections
\titleformat{\section}[block]{\bfseries}{\hspace{3cm}\thesection}{5pt}{}
\titleformat{\subsection}[block]{\bfseries}{\hspace{3cm}\thesection}{5pt}{}
\newcommand{\sectionbreak}{\clearpage} % New page on each section
 

\newlength{\mywidth}
\setlength{\mywidth}{14cm}

\newcommand{\cont}[1]{
\begin{tcolorbox}[center, boxrule=0.0 mm, width=\mywidth,arc=0mm,enhanced jigsaw,,colback=white,breakable]
#1	
\end{tcolorbox}
}

\newcommand{\info}[5]{
\begin{tcolorbox}[center, boxrule=0.1 mm, width=\mywidth,arc=0mm,enhanced jigsaw,breakable,colback=yellow!5]	
	
	\footnotesize
	\textbf{Øvingsområde}\\[5pt] #1 
	
	\textbf{Utstyr}\\ #2  \\
	
	\begin{tabular}{@{} p{4cm} p{4cm} l} 
		\textbf{Tid} & \textbf{Elevinndeling} & \textbf{Læringsarena} \\
		#3  & #4 & #5
	\end{tabular} 
\end{tcolorbox}	
}

\newcommand{\gjen}[1]{\begin{tcolorbox}[center,boxrule=0.1 mm, width=\mywidth,arc=0mm,colback=blue!3] {\large \textbf{Gjennomføring} \vspace{5 pt}}\newline #1  \end{tcolorbox}\vspace{-5pt}}
\newcommand{\eks}[1]{\begin{tcolorbox}[center,boxrule=0.1 mm, width=\mywidth,arc=0mm,colback=green!3] {\large \textbf{Eksempel} \vspace{5 pt}}\newline #1  \end{tcolorbox}\vspace{-5pt}}

\newcounter{opl}
%\numberwithin{opl}{article}


\newcommand{\opl}[1]{
\newpage
{\refstepcounter{opl} %\phantomsection 
\large \textbf{\theopl \;#1} \vsk}
}

% Headlines
\newcommand{\fork}{\textbf{Forkunnskapar}\\}
\newcommand{\forb}{\textbf{Forberedelsar}\\}
\newcommand{\opgvr}{\textbf{Oppgaver}}



%colors
\newcommand{\colr}[1]{{\color{red} #1}}
\newcommand{\colb}[1]{{\color{blue} #1}}
\newcommand{\colo}[1]{{\color{orange} #1}}
\newcommand{\colc}[1]{{\color{cyan} #1}}
\definecolor{projectgreen}{cmyk}{100,0,100,0}
\newcommand{\colg}[1]{{\color{projectgreen} #1}}

% Lister med bokstavar
\usepackage[inline]{enumitem}
% Opg
\newcommand{\abc}[1]{
	\begin{enumerate}[label=\alph*),leftmargin=18pt]
		#1
	\end{enumerate}
}

\usepackage[]{hyperref}
\begin{document}
\newpage
\section*{Forord}
\textbf{Bokas bruksområde}\\
I lag med \net{arg1}{Matematikken sine byggesteinar} (MB) dekker denne boka matematikk for 5.-10. klassse og for VGS-faga 1P og 2P. Mens MB tek for seg dei teoretiske grunnprinsippa matte er bygd på, er denne boka menit for å vise korleis matte kan anvendast i det daglege. Det er likevel med ein viss ambivalens eg bruker ordet ''anvendt''. Eg er hellig overbevist om at dei aller fleste har behov å bruke matematikk i konkrete, praktiske situasjonar for å få opplevinga av at matematikk blir anvendt. Eg håper derfor desse gratis-bøkene kan frigi midlar for skular, som da kan investere i utstyr som gjer at elevar (og lærarar) får måle, estimere, kalkulere og vurdere ut i frå reelle situasjoner.\vsk

\textbf{Boka si disponering} \\
Da boka gapar over matematikk for 5. klasse og heilt til VGS, vil kanskje mange meine at språket er noko avansert, spesielt for dei yngste. Men forenklingar fører ofte til at ein stadig må vende tilbake til tema for å kommentere utvidingar og/eller unntak, og da dannast det fort eit unødig kronglete og innvikla bilde av matematikken si oppbygging. Eg trur ein i lengda er tent med å presentere temaa så utfyllande som mogleg, og heller bruke god tid på å forstå dei éin gong for alle.\vsk

Nokon vil kanskje også reagere på at eksempla er veldig enkle, at dei viser få samansatte problem. Éin av grunnane til dette er at slik vil det faktisk vere for dei aller fleste etter endt skulegong; det handlar om å bruke formlar direkte. Ein annen grunn er at eg meiner det å meistre likningar er den overlegent beste måten å løyse sammensette problem på, og derfor handlar nesten helie kapittel 6 om problemløysing.\vsk


\textbf{Tilbakemeldingar og eventuelle endringer} \\
Eg håper å høre frå deg med tilbakemeldinger om boka. Merk likevel at alle har sine tankar om korleis ei lærebok ideelt sett bør utformast, så ikkje tolk det som utakksemd viss tilbakemeldingar ikkje blir tatt til etterretning. Husk at kodekilden til både denne \net{https://sindrsh.github.io/AppliedMath/}{boka} og \mb\;ligg open for alle på GitHub; med litt kunnskapar om Git og \LaTeX\;kan du enkelt gjere endringar slik det passer deg og klassen din!


\newpage
\textbf{Gjøreliste} \\
Prosjektet som denne boka er ein viktig del av er under stadig utvikling. Her er ei liste med komande gjereremål, i prioritert rekkefølge:
\begin{itemize}
	\item Korrigere skrivefeil. Dette gjerast kontinuerleg, gir du beskjed om feil funne til {\tt sindre.heggen@gmail.com}, vil korrigering som oftast bli utført samme dag.
	\item Legge til fleire oppgåver både i denne boka og i \mb. 
	\item Legge til fasit 
	\item Legge til forklaring av delingsalgoritmen.
	\item Lage ei pensumoversikt for denne boka og \mb\,sett opp mot kompetansemålene f.o.m. 5. klasse og t.o.m. 2P.
	\item Vidareutvikle \net{https://hellandmatte.netlify.app/}{nettside} med læringsvideoar, undervisningsopplegg og meir. 
\end{itemize}¨
\newpage

\end{document}