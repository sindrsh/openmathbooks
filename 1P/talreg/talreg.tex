\documentclass[english,hidelinks,pdftex, 11 pt, class=report,crop=false]{standalone}
\usepackage[T1]{fontenc}
\usepackage[utf8]{luainputenc}
\usepackage{lmodern} % load a font with all the characters
\usepackage{geometry}
\geometry{verbose,a4paper, inner=0cm, outer=0 cm, bmargin=2cm, tmargin=1cm}
%\textwidth=12cm
\setlength{\parindent}{0bp}
\usepackage{import}
\usepackage[subpreambles=false]{standalone}
\usepackage{amsmath}
\usepackage{amssymb}
\usepackage{esint}
\usepackage{babel}
\usepackage{tabu}
\usepackage[dvipsnames, table]{xcolor}
\usepackage{cancel}
\makeatother
\makeatletter
\usepackage{datetime2}
\usepackage{titlesec}
\usepackage[many]{tcolorbox}

% Eheter
\newcommand{\enh}[1]{\,\textrm{#1}}
%referances
\newcommand{\net}[2]{{\color{blue}\href{#1}{#2}}}

%Spaces
\newcommand{\vsk}{\\[12pt]}
\newcommand{\vs}{\vspace{-12pt}}

% Tabell for opplegg

\newcommand{\ovlist}[1]{
\vspace{-16pt}
\begin{itemize}
	#1
\end{itemize}
}

% Chapters and sections
\titleformat{\section}[block]{\bfseries}{\hspace{3cm}\thesection}{5pt}{}
\titleformat{\subsection}[block]{\bfseries}{\hspace{3cm}\thesection}{5pt}{}
\newcommand{\sectionbreak}{\clearpage} % New page on each section
 

\newlength{\mywidth}
\setlength{\mywidth}{14cm}

\newcommand{\cont}[1]{
\begin{tcolorbox}[center, boxrule=0.0 mm, width=\mywidth,arc=0mm,enhanced jigsaw,,colback=white,breakable]
#1	
\end{tcolorbox}
}

\newcommand{\info}[5]{
\begin{tcolorbox}[center, boxrule=0.1 mm, width=\mywidth,arc=0mm,enhanced jigsaw,breakable,colback=yellow!5]	
	
	\footnotesize
	\textbf{Øvingsområde}\\[5pt] #1 
	
	\textbf{Utstyr}\\ #2  \\
	
	\begin{tabular}{@{} p{4cm} p{4cm} l} 
		\textbf{Tid} & \textbf{Elevinndeling} & \textbf{Læringsarena} \\
		#3  & #4 & #5
	\end{tabular} 
\end{tcolorbox}	
}

\newcommand{\gjen}[1]{\begin{tcolorbox}[center,boxrule=0.1 mm, width=\mywidth,arc=0mm,colback=blue!3] {\large \textbf{Gjennomføring} \vspace{5 pt}}\newline #1  \end{tcolorbox}\vspace{-5pt}}
\newcommand{\eks}[1]{\begin{tcolorbox}[center,boxrule=0.1 mm, width=\mywidth,arc=0mm,colback=green!3] {\large \textbf{Eksempel} \vspace{5 pt}}\newline #1  \end{tcolorbox}\vspace{-5pt}}

\newcounter{opl}
%\numberwithin{opl}{article}


\newcommand{\opl}[1]{
\newpage
{\refstepcounter{opl} %\phantomsection 
\large \textbf{\theopl \;#1} \vsk}
}

% Headlines
\newcommand{\fork}{\textbf{Forkunnskapar}\\}
\newcommand{\forb}{\textbf{Forberedelsar}\\}
\newcommand{\opgvr}{\textbf{Oppgaver}}



%colors
\newcommand{\colr}[1]{{\color{red} #1}}
\newcommand{\colb}[1]{{\color{blue} #1}}
\newcommand{\colo}[1]{{\color{orange} #1}}
\newcommand{\colc}[1]{{\color{cyan} #1}}
\definecolor{projectgreen}{cmyk}{100,0,100,0}
\newcommand{\colg}[1]{{\color{projectgreen} #1}}

% Lister med bokstavar
\usepackage[inline]{enumitem}
% Opg
\newcommand{\abc}[1]{
	\begin{enumerate}[label=\alph*),leftmargin=18pt]
		#1
	\end{enumerate}
}

\usepackage[]{hyperref}

\newcommand{\note}{Merk}
\newcommand{\notesm}[1]{{\footnotesize \textsl{\note:} #1}}
\newcommand{\ekstitle}{Eksempel }
\newcommand{\sprtitle}{Språkboksen}
\newcommand{\expl}{forklaring}
\newcommand{\pyt}{Pytagoras' setning}
\newcommand\sv{\vsk \textbf{Svar} \vspace{4 pt}\\}

%references
\newcommand{\reftab}[1]{\hrs{#1}{tabell}}
\newcommand{\rref}[1]{\hrs{#1}{regel}}
\newcommand{\dref}[1]{\hrs{#1}{definisjon}}
\newcommand{\refkap}[1]{\hrs{#1}{kapittel}}
\newcommand{\refsec}[1]{\hrs{#1}{seksjon}}
\newcommand{\refdsec}[1]{\hrs{#1}{delseksjon}}
\newcommand{\refved}[1]{\hrs{#1}{vedlegg}}
\newcommand{\eksref}[1]{\textsl{#1}}
\newcommand\fref[2][]{\hyperref[#2]{\textsl{figur \ref*{#2}#1}}}
\newcommand{\refop}[1]{{\color{blue}Oppgave \ref{#1}}}
\newcommand{\refops}[1]{{\color{blue}oppgave \ref{#1}}}


%Algebra
\newcommand{\kvadset}{Kvadratsetningene}
\newcommand{\aenato}{Sum-produkt-metoden}

% Geometry
\newcommand{\hlikb}{Midtnormalen i en likebeint trekant}
\newcommand{\arealsetn}{Arealsetningen}
\newcommand{\trkmedian}{Median}
\newcommand{\midtrk}{Midtnormal (i trekant)}
\newcommand{\innskrsirk}{Innskrevet sirkel}
\newcommand{\cossetn}{Cosinussetningen}
\newcommand{\perfvink}{Sentral- og periferivinkel}
\newcommand{\tang}{Tangent}

% Derivative
\newcommand{\derel}{Den deriverte av elementære funksjoner}
\newcommand{\divder}{Divisjonsregelen}
\newcommand{\kjernereg}{Kjerneregelen}
\newcommand{\prodregder}{Produktregelen}
\newcommand{\lhop}{L'Hopitals regel}

% Funksjonsdrofting
\newcommand{\monder}{Monotoniegenskaper og den deriverte}
\newcommand{\fderekstr}{$ \bm{f'=0} $ for lokale ektstremalpunkt}
\newcommand{\andredertest}{Andrederiverttesten}

% Vectors
\newcommand{\detar}{Arealformler med determinanter}
\newcommand{\avstpunktlin}{Avstand mellom punkt og linje}

%Appendix
\newcommand{\rolle}{Rolles teorem}
\newcommand{\meanval}{Middelverdisetningen}

% Solutions manual
\newcommand{\selos}{Se løsningsforslag.}

\begin{document}


\section{Tallregning}
\subsection*{Introduksjon til negative tall}
Vi har tidligere sett at (for eksempel) tallet 5 på ei tallinje ligger 5 enerlengder til høyre for 0. 
\fig{neg1}
Men hva om vi går andre veien, altså mot venstre? Dette spørsmålet svarer vi på ved å innføre \textit{negative tall}. \index{tall!negativt}\index{tall!positivt}.\regv

\reg[Positive og negative tal]{
På en tallinje gjelder følgende:
\begin{itemize}
	\item Tall plassert \textsl{til høyre} for 0 er \outl{positive tall}.
	\item Tall plassert \textsl{til venstre} for 0 er \outl{negative tal}.
\end{itemize}
\fig{neg2}
}\vsk 
Vi kan ikke hele tiden bruke en tallinje for å avgjøre om et tall er\\ negativt eller positivt, og derfor bruker vi et tegn for å vise at tall er negative. Dette tegnet er rett og slett \sym{$ - $}, altså det samme tegnet som vi bruker ved subtraksjon. $ 5 $ er med dét et positivt tall, mens $ -5 $ er et negativt tall. På tallinja er det slik at
\begin{itemize}
	\item 5 ligger 5 enerlengder \textsl{til høyre} for 0.
	\item $ -5 $ ligger 5 enerlengder \textsl{til venstre} for 0.
\end{itemize}
\fig{neg3}
Den store forskjellen på $ 5 $ og $ -5 $ er altså på hvilken side av 0 tallene ligger. Da $ 5 $ og $ -5 $ har samme avstand til 0, sier vi at $ 5 $ og $ -5 $ har samme \outl{lengde}\index{lengde}. \regv

\reg[Lengde \index{tallverdi}\index{absoluttverdi}]{
Lengden til et tall skrives ved symbolet \sym{| |}.\vsk	
	
Lengden til et positivt tall er verdien til tallet.\vsk
	
Lengden til et negativt tall er verdien til det positive tallet med samme siffer.
} 
\eks[1]{ \vs \vs
\[ |27|=27 \]
}
\eks[2]{ \vs \vs
\[\left|-27\right|=27 \]
}
\spr{
Andre ord for lengde er \outl{tallverdi} og \outl{absoluttverdi}.
}
\info{Fortegn}{
\outl{Fortegn}\index{fortegn} er en samlebetegnelse for \sym{$ + $} og \sym{$ - $}. $ 5 $ har \sym{$ + $} som fortegn og $ -5 $ har \sym{$ - $} som fortegn.
}
\newpage
\subsection*{Regning med negative tall}
Ved innføringen av negative tall får de fire regneartene nye sider som vi må se på trinnvis. Når vi adderer, subtraherer, multipliserer eller dividerer med negative tall vil vi ofte, for tydeligheten sin skyld, skrive negative tall med parentes rundt. Da skriver vi for eksempel $ -4 $ som $ (-4) $. 

\subsection*{Addisjon}
Når vi adderte i \hrs{Addisjon}{seksjon} så vi på \sym{$ + $} som vandring \textsl{mot høyre}. Negative tall gjør at vi må utvide begrepet for \sym{$ + $}\,: \regv
\st{\begin{center}
		\begin{tabular}{cl}
			$ + $&''Like langt og i \textsl{samme} retning som''	
		\end{tabular}
\end{center}}
La oss se på regnestykket
\[ 7+(-4) \]
Vår utvidede definisjon av \sym{$ + $} sier oss nå at
\[ 7+(-4)=\text{''}7 \text{ og like langt og i \textsl{samme} retning som } (-4)\text{''} \]
$ (-4) $ har lengde 4 og retning \textsl{mot venstre}. Vårt regnestykke sier altså at vi skal starte på 7, og deretter gå lengden 4 \textsl{mot venstre}.
\[ 7+(-4)=3 \]
\fig{neg6}

\reg[Addisjon med negative tall]{
Å addere et negativt tall er det samme som å subtrahere tallet med samme tallverdi.
}
\eks[1]{ \vs
	\[ 4+(-3)=4-3=1 \]
}
\eks[2]{ \vs
	\[ -8+(-3)=-8-3=-11 \]
}
\info{Addisjon er kommutativ}{
\hrs{adkom}{Regel} er gjeldende også etter innføringen av negative tall, for eksempel er	
\[ 7+(-3)=4=-3+7 \]	
}

\subsection*{Subtraksjon}
I \hrs{Subtraksjon}{seksjon} så vi på \sym{$ - $} som vandring \textsl{mot venstre}. Også betydningen av \sym{$ - $} må utvides når vi jobber med negative tall:\regv

\st{\begin{center}
	\begin{tabular}{cl}
		$ - $&''Like langt og i \textsl{motsatt} retning som''	
	\end{tabular}
\end{center}}
La oss se på regnestykket
\[ 2-(-6) \]
Med vår utvidede betydning av \sym{$-$}, kan vi skrive
\[ 2-(-6)=\text{''}2 \text{ og like langt og i \textsl{motsatt} retning som } (-6)\text{''} \]
$ -6 $ har lengde 6 og retning \textsl{mot venstre}. Når vi skal gå samme lengde, men i \textsl{motsatt} retning, må vi altså gå lengden 6 \textsl{mot høyre}\footnote{Vi minner enda en gang om at rødfargen på pila indikerer at man skal vandre fra pilspissen til andre enden.}. Dette er det samme som å addere 6:
\[ 2-(-6)=2+6=8 \] \vs
\fig{neg7}
\reg[Subtraksjon med negative tall]{
Å subtrahere et negativt tall er det samme som å addere det positive tallet med samme tallverdi.
}
\eks[1]{ \vs
\[ 11-(-9)=11+9=20 \]
}
\eks[2]{ \vs
	\[ -3-(-7)=-3+7=4 \]
}
\begin{comment}
	\info{Merk}{
	Med innføringen av negative tall kan subtraksjon bli sett på som addisjon med negative tal. For eksempel er $ {7-3=7+(-3)} $ og $ {2-7=2+(-7)} $.
	}
\end{comment}

\subsection*{Multiplikasjon}
I \hrs{Gonging}{seksjon} introduserte vi ganging med positive heltall som gjentatt addisjon. Med våre utvidede begrep av addisjon og subtraksjon, kan vi nå også utvide begrepet multiplikasjon: \regv

\reg[Multiplikasjon med positive og negative tall\label{gongneg}]{
Ganging med et positivt heltall er det samme som gjentatt addisjon.\vsk

Ganging med et negativt heltal er det samme som gjentatt subtraksjon.
}
\eks[1]{ \vs \vs \label{negeksempel}
\alg{
	2\cdot3 &=\text{''Like langt og i \textsl{samme} retning som 2, 3 ganger''}\\
	&=2+2+2 \\
	&=6
}
}
\eks[2]{ \vs \vs
\alg{
	(-2)\cdot3&=\text{''Like langt og i \textsl{samme} retning som }(-2) \text{, 3 ganger''} \\
	&=-2-2-2\\
	&=-6
}
}
\eks[3]{ \vs \vs
\alg{
	2\cdot(-3)&=\text{''Like langt og i \textsl{motsatt} retning som 2, 3 ganger''} \\
	&=-2-2-2 \\
	&=-6
}
}
\eks[4]{ \vs \vs
	\alg{
		(-3)\cdot(-4)&=\text{''Like langt og i \textsl{motsatt} retning som $ -3 $, 4 ganger''}\\
		&=3+3+3+3\\
		&=12
	}	 	 
}
\info{Multiplikasjon er kommutativ}{
\textsl{Eksempel 2} og \textsl{Eksempel 3} på side \pageref{negeksempel} illustrerer at \rref{gangkom} også er gjeldende etter innføringen av negative tall:
\[ (-2)\cdot3=3\cdot(-2) \]
}  \vsk
Det blir tungvint å regne ganging som gjentatt addisjon/subtraksjon hver gang vi har et negativt tall involvert, men som en direkte konsekvens av \rref{gongneg} kan vi lage oss følgende to regler:\regv

\reg[Multiplikasjon mellom et negativt og et positivt tall \label{gangmnegto}]{
Produktet av et negativt og et positivt tall er et negativt tall. \vsk

Tallverdien til faktorene ganget sammen gir tallverdien til\\ produktet.
}
\eks[1]{
Regn ut $ (-7)\cdot8 $

\sv
Siden $ 7\cdot8=56 $, er $ (-7)\cdot8=-56 $
}
\eks[2]{
	Regn ut $ 3\cdot(-9) $.
	
	\sv
	Siden $ 3\cdot9=27 $, er $ 3\cdot(-9)=-27 $
}
\vsk

\reg[Multiplikasjon mellom to negative tall\label{gangmnegtre}]{
Produktet av to negative tall er et positivt tall. \vsk

Tallverdien til faktorene ganget sammen gir verdien til produktet.
} 
\eks[1]{ \vsb
\[ (-5)\cdot(-10)=5\cdot10=50 \]
}
\eks[2]{
\vsb
\[ (-2)\cdot(-8)=2\cdot8=16 \]
}
\subsection*{Divisjon}
Definisjonen av divisjon (se \hrs{Divisjon}{seksjon}), kombinert med det vi vet om multiplikasjon med negative tall, gir oss nå dette:\regv

\st{\begin{center}
		\algv{
			-18:6=\text{''Tallet jeg må gange 6 med for å få $ -18 $''}
		}
		$ 6\cdot(-3)=-18 $, altså er $ -18:6=-3 $
\end{center}} \vsk
\st{
\begin{center}
	\algv{
		42:(-7)=\text{''Tallet jeg må gange $ -7 $ med for å få 42''}
	}
	$ (-7)\cdot(-8)=42 $, altså er $ 42:(-7)=-8 $
\end{center}
}\vsk
\st{
\begin{center}
	\algv{
		-45:(-5)=\text{''Tallet jeg må gange $ -5 $ med for å få $ -45 $''}
	}
	$ (-5)\cdot9=-45 $, altså er $ -45:(-5)=9 $
\end{center}
 }\vsk
\reg[Divisjon med negative tall]{
Divisjon mellom et positivt og et negativt tall gir et negativt tall.\vsk

Divisjon mellom to negative tall gir et positivt tall. \vsk

Tallverdien til dividenden delt med tallverdien til divisoren gir tallverdien til kvotienten. 
}
\eks[1]{ \vsb
	\[ -24:6=-4 \]
}
\eks[2]{ \vsb
	\[ 24:(-2)=-12 \]
}
\eks[3]{ \vsb
	\[ -24:(-3)=8 \]
}
\eks[4]{ \vsb
\[ \frac{2}{-3}=-\frac{2}{3} \]
}
\eks[5]{ \vsb
	\[ \frac{-10}{7}=-\frac{10}{7} \]
}
\newpage
\subsubsection{Ganging med 10, 100, 1\,000 osv.}
\reg[Å gange heltall med 10, 100 osv. \label{gangheltallmed10100}]{
	\vs
	\begin{itemize}
		\item Når man ganger et heltall med 10, får man svaret ved å legge til sifferet 0 bak heltallet.
		\item Når man ganger et heltall med 100, får man svaret ved å legge til sifrene 00 bak heltallet.
		\item Det samme mønsteret gjelder for tallene 1\,000, 10\,000 osv.
	\end{itemize}
}
\eks[1]{\vsb \vsb
	\alg{
		6\cdot \colb{10} &= 6\colb{0}\vn
		79\cdot \colb{10} &= 79\colb{0} \vn
		802\cdot\colb{10}&=802\colb{0}
	}
}
\eks[2]{ \vsb \vsb
	\alg{ 
		6\cdot\colb{100} &= 6\colb{00} \vn
		79\cdot\colb{100} &= 7\,9\colb{00} \vn
		802\cdot\colb{100} &=80\,2\colb{00}
	}
}
\eks[3]{ \vsb \vsb
	\alg{ 
		6\cdot\colb{1\,000} &= 6\,\colb{000} \vn
		79\cdot\colb{10\,000} &= 79\colb{0\,000} \vn
		802\cdot\colb{100\,000} &=80\,2\colb{00\,000}
	}
}
\newpage
\reg[Å gange desimaltall med 10, 100 osv. \label{gangdesmed10100}]{
	\vs
	\begin{itemize}
		\item Når man ganger et desimaltall med 10, får man svaret ved å flytte komma en plass til høgre.
		\item Når man ganger et heltall med 100, får man svaret ved å flytte komma to plasser til høgre.
		\item Det samme mønsteret gjelder for tallene 1\,000, 10\,000 osv.
	\end{itemize}
}
\eks[1]{\vsb \vsb
	\alg{
		7\colr{,}9\cdot 10 &= 79\colr{,}=79 \vn
		38\colr{,}02\cdot10&=380\colr{,}2 \vn
		0\colr{,}57\cdot 10 &=05\colr{,}7=5\colr{,}7 \vn
		0\colr{,}194\cdot 10&= 01\colr{,}94=1\colr{,}94
	}
}
\eks[2]{ \vsb \vsb
	\alg{
		7\colr{,}9\cdot 100 &= 790\colr{,}=790 \vn
		38\colr{,}02\cdot100&=3802\colr{,}=3\,802 \vn
		0\colr{,}57\cdot 100 &=057\colr{,}=57 \vn
		0\colr{,}194\cdot 100&= 019\colr{,}4=19\colr{,}4
	}
}
\eks[3]{ \vsb \vsb
	\alg{
		7\colr{,}9\cdot 1\,000 &= 7900\colr{,}=7\,900 \vn
		38\colr{,}02\cdot10\,000&=380020\colr{,}=380\,200 \vn
		0\colr{,}57\cdot 100\,000 &=57000\colr{,}=57\,000
	}
}
\info{Merk}{
	\hrs{gangheltallmed10100}{Regel} er bare et spesialtilfelle av \rref{gangdesmed10100}. For eksempel, å bruke \rref{gangheltallmed10100} på regnestykket $ {7\cdot 10} $ gir samme resultat som å bruke \rref{gangdesmed10100} på regnestykket $ {7,0\cdot 10} $. 
}
\newpage
\fork{Å gange tall med 10, 100 osv.}{
	Titallsystemet baserer seg på grupper av ti, hundre, tusen osv., og tideler, hundredeler og tusendeler osv (se \rref{titalsys}). Når man ganger et tall med 10, vil alle enere i tallet bli til tiere, alle tiere bli til hundrere osv. Hvert siffer forskyves altså én plass mot venstre. Tilsvarende forskyves hvert siffer to plasser mot venstre når man ganger med 100, tre plasser når man ganger med 1\,000 osv.
}
\subsection*{\delmedtihundre}
\reg[Deling med 10, 100, 1\,000 osv. \label{deledesmed10100}]{
	Når man deler et desimaltall med 10, får man svaret ved å flytte komma én plass til venstre.\vsk
	
	Når man deler et desimaltall med 100, får man svaret ved å flytte komma to plasser til venstre.\vsk
	
	Det samme mønsteret gjelder for tallene 1\,000, 10\,000 osv.
}
\eks[1]{ \vsb \vsb
	\alg{
		200:10&=200\colr{,}0:10 \\&=20\colr{,}00\\&=20	\vn
		45:10&=45\colr{,}0:10 \\&= 4\colr{,}50 \\&=4\colr{,}5
	}
}
\eks[2]{ \vsb \vsb
	\alg{
		200:100&=200\colr{,}0:100 \\&=2\colr{,}000\\&=2	\vn
		45:100&=45\colr{,}0:100 \\&= 0\colr{,}450 \\&=0\colr{,}45
	}
}
\newpage
\eks[3]{ \vsb \vsb
	\alg{
		143\colr{,}7 :10 &= 14\colr{,}37 \vn
		143\colr{,}7 :100 &= 1\colr{,}437 \vn
		143\colr{,}7 :1\,000 &= 0\colr{,}1437 
	}
}
\eks[4]{ \vsb \vsb
	\alg{
		93\colr{,}6:10 &= 9\colr{,}36 \vn
		93\colr{,}6:100 &= 0\colr{,}936 \vn
		93\colr{,}6:1\,000 &= 0\colr{,}0936
	}
}
\fork{Deling med 10, 100, 1\,000 osv.}{
	Titallsystemet baserer seg på grupper av ti, hundre, tusen osv., og tideler, hundredeler og tusendeler osv (se \rref{titalsys}). Når man deler et tall med 10, vil alle enere i tallet bli til tideler, alle tiere bli til enere osv. Hvert siffer forskyves altså én plass mot høgre. Tilsvarende forskyves hvert siffer to plasser mot høgre når man deler med 100, tre plasser når man deler med 1\,000 osv.
}

\subsubsection{Overslagsregning}
Det er ikke alltid vi trenger å vite svaret på regnestykker helt nøyaktig, ofte er det viktigere at vi fort kan avgjøre hva svaret \textsl{omtrent} er det samme som, aller helst ved hoderegning. Når vi finner svar som omtrent er riktige, sier vi at vi gjør et \outl{overslag}. Et overslag innebærer at vi avrunder\footnote{\textit{Obs!} Avrunding ved overslag trenger ikke å innebære avrunding til nærmeste tier og lignende.} tallene som inngår i et regnestykke slik at utregningen blir enklere. \vsk


\spr{
	At noe er ''omtrent det samme som'' skriver vi ofte som ''cirka'' (''ca.''). Tegnet for ''cirka'' er \sym{$ \approx $}.
} 

\subsubsection{Overslag ved addisjon og ganging}
La oss gjøre et overslag på regnestykket
\[ 98,2+24,6 \]
Vi ser at $ 98,2 \approx 100 $. Skriver vi 100 istedenfor 98,2 i regnestykket vårt, får vi noe som er litt mer enn det nøyaktige svaret. Skal vi endre på 24,6 bør vi derfor gjøre det til et tall som er litt mindre. 24,6 er ganske nærme 20, så vi kan skrive
\[ 98,2+24,6 \approx 100 + 20 = 120 \]
Når vi gjør overslag på tall som legges sammen, bør vi altså prøve å gjøre det ene tallet større (runde opp) og et tall mindre (runde ned).\\

\linje
Det samme gjelder også hvis vi har ganging, for eksempel
\[ 1\,689\cdot12 \]
Her avrunder vi 12 til 10. For å ''veie opp'' for at svaret da blir litt mindre enn det egentlige, avrunder vi 1\,689 opp til 1\,700. Da får vi
\[ 1\,689\cdot12\approx 1\,700\cdot 10 =17\,000 \]
\subsubsection{Overslag ved subtraskjon og deling}
Skal et tall trekkes fra et annet, blir det litt annerledes. La oss gjøre et overslag på
\[ 186,4-28,9 \]
Hvis vi runder 186,4 opp til 190 får vi et svar som er større enn det egentlige, derfor bør vi også trekke ifra noe. Det kan vi gjøre ved også å runde 28,9 oppover (til 30):
\alg{
	186,4-28,9&\approx 190-30 \\&=160
}
Samme prinsippet gjelder for deling: 
\[ 145:17 \]
Vi avrunder 17 opp til 20. Deler vi noe med 20 istedenfor 17, blir svaret mindre. Derfor bør vi også runde 145 oppover (til 150):
\[ 145:17 \approx 150:20 = 75 \]

\subsubsection{Overslagsregning oppsummert}
\reg[Overslagsregning \label{tipsoverslag}]{ \vs
	\begin{itemize}
		\item Ved addisjon eller multiplikasjon mellom to tall, avrund gjerne et tall opp og et tall ned.
		\item Ved subtraksjon eller deling mellom to tall, avrund gjerne begge tall ned eller begge tall opp.
	\end{itemize}	
}
\eks[]{
	Rund av og finn omtrentlig svar for regnestykkene.\os
	
	\abch{
		\item $ {23,1+174,7} $ 
		\item $ {11,8\cdot107,2} $ 		
	} \os
	\abchs{3}{
		\item $ {37,4-18,9} $  \ \ 
		\item $ {1054:209} $
	}
	\vspace{-2pt}
	
	\sv  \vspace{-7pt}
	\abc{
		\item $ 32,1 + 174,7 \approx 30+170 = 200 $
		\item $ 11,8 \cdot 107,2 \approx 10\cdot110 = 1\,100 $
		\item $ 37,4 - 18,9 \approx 40-20 = 20 $
		\item $ 1\,054:209 \approx 1\,000:200 = 5 $
	}
} \vsk

\info{Kommentar
}{
	Det finnes ingen konkrete regler for hva man \textsl{kan} eller ikke \textsl{kan} tillate seg av forenklinger når man gjør et overslag, det som er kalt \rref{tipsoverslag} er strengt tatt ikke en regel, men et nyttig tips.\vsk
	
	Man kan også spørre seg hvor langt unna det faktiske svaret man kan tillate seg å være ved overslagsregning. Heller ikke dette er det noe fasitsvar på, men en grei føring er at overslaget og det faktiske svaret skal være av samme \outl{størrelsesorden}. Litt enkelt sagt betyr dette at hvis det faktiske svaret har med tusener å gjøre, bør også overslaget ha med tusener å gjøre. Mer nøyaktig sagt betyr det av det faktiske svaret og ditt overslag bør ha samme tierpotens når de er skrevet på standardform\footnote{Se \refsec{Standardform}}.
}
\newpage



\section{Potenser \label{Potensar}}
\fig{pot}
En potens består av et \outl{grunntall}\index{grunntall} og en \outl{eksponent}\index{eksponent}. For eksempel er $2^{3}$ en potens med grunntall 2 og
eksponent 3. En positiv, heltalls eksponent sier hvor mange eksemplar
av grunntallet som skal ganges sammen, altså er
\[ 2^3 =2\cdot2\cdot2 \]

\reg[Potenstall]{
	$ {a^n} $ er et potenstall med grunntall $ a $ og eksponent $ n $. 
	\vsk
	
	Hvis $ n $ er et naturlig tall, vil $ a^n $ svare til $ n $ eksemplar av $ a $\\ multiplisert med hverandre.
}
\eks[1]{\vs \vs
	\algv{
		5^3 &= 5\cdot5\cdot5 \\
		&= 125
	}
}
\eks[2]{\vs \vs
	\[ c^4 = c\cdot c \cdot c \cdot c \]
}
\eks[3]{ \vs \vs
	\algv{
		(-7)^2 &= (-7)\cdot(-7) \\
		&= 49
	}
} 
\eks[4]{\vs \vs
	\[ a^1=a \]
}
\spr{
	Vanlige måter å si $ 2^3 $ på er
	\begin{itemize}
		\item ''2 i tredje''
		\item ''2 opphøyd i 3''
	\end{itemize}
	I programmeringsspråk brukes gjerne symbolet \sym{\^{}} eller symbolene \sym{**} mellom grunntall og eksponent.\vsk
	
	Å opphøye et tall i 2 kalles ''å kvadrere'' tallet.
}
\newpage
\info{Merk}{
	De kommende sidene vil inneholde regler for potenser med tilhørende forklaringer. Selv om det er ønskelig at de har en så generell form som mulig, har vi i forklaringene valgt å bruke eksempel der eksponentene ikke er variabler. Å bruke variabler som eksponenter ville gitt mye mindre leservennlige uttrykk, og vi vil påstå at de generelle tilfellene kommer godt til synes også ved å studere konkrete tilfeller. 
} \vsk \vsk

\reg[Multiplikasjon mellom potenser \label{potgang}]{
	\begin{equation}
		a^{m}\cdot a^{n}=a^{m+n}	
	\end{equation}
}
\eks[1]{\vs \vs
	\algv{3^{5}\cdot3^{2}&=3^{5+2}\\&=3^{7}}
}
\eks[2]{\vs \vs
	\algv{
		b^4\cdot b^{11}&= b^{3+11}\\
		&=b^{14}
	}
}
\eks[3]{ \vs \vs
	\algv{
		a^5\cdot a^{-7} &= a^{5+(-7)} \\
		&=a^{5-7} \\
		&= a^{-2} 
	}
	(Se \rref{potneg} for hvordan en potens med negativ eksponent kan tolkes.)	
} 
\newpage
\fork{\ref{potgang} Multiplikasjon mellom potenser}{
	La oss se på tilfellet 
	\[ a^{2}\cdot a^{3} \]
	Vi har at
	\algv{
		a^{2} & =2\cdot2\vn
		a^{3} & =2\cdot2\cdot2
	}
	Med andre ord kan vi skrive 
	\begin{align*}
		a^{2}\cdot a^{3} & =\overbrace{a \cdot a}^{a^{2}}\cdot\overbrace{a\cdot a\cdot a}^{a^{3}}\\
		& =a^{5}
	\end{align*}
}
\reg[Divisjon mellom potenser \label{potdivpot}]{\vs
	\[ \frac{a^{m}}{a^{n}}=a^{m-n} \] }

\eks[1]{\vspace{-20 pt}
	\[
	\frac{3^{5}}{3^{2}}=3^{5-2}=3^{3}
	\]
} 
\eks[2]{ \vs \vsb
	\alg{
		\frac{2^{4}\cdot a^{7}}{a^{6}\cdot2^{2}}&=2^{4-2}\cdot a^{7-6}\\
		&=2^{2}a \\
		&=4a
	}
}
\newpage
\fork{\ref{potdivpot} Divisjon mellom potenser}{
	La
	oss undersøke brøken
	\[ \frac{a^{5}}{a^{2}} \]
	Vi skriver
	ut potensene i teller og nevner: 
	\begin{align*}
		\frac{a^{5}}{a^{2}} & =\frac{a\cdot a\cdot a\cdot a\cdot a}{a\cdot a}\br
		& =\frac{\bcancel{a}\cdot\bcancel{a}\cdot a\cdot a\cdot a}{\bcancel{a}\cdot\bcancel{a}}\\
		& =a\cdot a\cdot a\\
		& =a^{3}
	\end{align*}
	Dette kunne vi ha skrevet som
	\begin{align*}
		\frac{a^{5}}{a^{2}} & =a^{5-2}\\
		& =a^{3}
	\end{align*}
} \vsk \vsk

\reg[Potens med 0 som eksponent \label{pota0}]{\vs \vs
	\[
	a^{0}=1
	\]
}
\eks[1]{\vs \vs\[
	1000^{0}=1
	\]}
\eks[2]{\vs \vs\[
	(-b)^{0}=1
	\]}
\fork{\ref{pota0} Potens med 0 som eksponent}{
	Et tall delt på seg selv er alltid lik 1, derfor er 
	\[
	\frac{a^{n}}{a^{n}}=1
	\]
	Av dette, og \rref{potdivpot}, har vi at
	\algv{
		1&=\frac{a^{n}}{a^{n}}
		\\& =a^{n-n}\\
		& =a^{0}
	}
} \vsk \vsk

\reg[Potenser med negativ eksponent \label{potneg}]{
	\[ a^{-n}=\frac{1}{a^n} \]
}
\eks[1]{ \vs \vs
	\alg{
		a^{-8}&=\frac{1}{a^8}  
	}	
}
\eks[2]{ \vs \vs
	\alg{
		(-4)^{-3}&=\frac{1}{(-4)^3} 
		=-\frac{1}{64}
	}
}
\fork{\ref{potneg} Potenser med negativ eksponent}{
	Av \rref{pota0} har vi at $ a^0=1 $. Altså er
	\alg{
		\frac{1}{a^n}=\frac{a^0}{a^n}
	}
	Av \rref{potdivpot}  er
	\algv{
		\frac{a^0}{a^n}&=a^{0-n} \\
		&=a^{-n}
	}
} \vsk \vsk


\reg[Brøker med potenser \label{potbr}]{\vs
	\begin{equation}\label{pbrg}
		\left(\frac{a}{b}\right)^{m}=\frac{a^{m}}{b^{m}}
\end{equation}} 
\eks[1]{ \vs \vs
	\alg{
		\left(\frac{3}{4}\right)^2=\frac{3^2}{4^2} 
		=\frac{9}{16}
	}
}
\eks[2]{ \vs \vs
	\alg{
		\left(\frac{a}{7}\right)^3=\frac{a^3}{7^3} 
		=\frac{a^3}{343}
	}
}
\fork{\ref{potbr} Brøker med potenser}{
	La oss studere
	\[ \left(\frac{a}{b}\right)^3 \]
	Vi har at
	\begin{align*}
		\left(\frac{a}{b}\right)^3 	&=\frac{a}{b}\cdot \frac{a}{b}\cdot \frac{a}{b}\br
		& =\frac{a\cdot a\cdot a}{b\cdot b\cdot b}\br
		& =\frac{a^{3}}{b^{3}}
	\end{align*}
}\vsk \vsk

\reg[\faktgr \label{faktgr}]{
	\begin{equation}\label{key}
		\left(ab\right)^{m}=a^{m}b^{m}
	\end{equation}
}
\eks[1]{ \vs \vs \vs
	\alg{
		(3a)^5&=3^5a^5 \\
		&=243a^5 
	}	
}
\eks[2]{\vs\vs
	\[
	(ab)^{4}=a^{4}b^{4}
	\]
}
\fork{\ref{faktgr} \faktgr}{
	La oss
	bruke ${(a\cdot b)^{3}}$ som eksempel. Vi har at
	\alg{
		(a\cdot b)^{3}&=(a\cdot b)\cdot(a\cdot b)\cdot(a\cdot b) \\
		&=a\cdot a\cdot a \cdot b \cdot b \cdot b \\
		&=a^3b^3
	}
}\vsk \vsk

\newpage
\reg[\potsomgrunn \label{potsomgrunn}]{
	\begin{equation}
		\left(a^{m}\right)^{n}=a^{m\cdot n}
\end{equation}}
\eks[1]{ \vs \vs
	\alg{
		\left(c^4\right)^5&=c^{4\cdot5}\\
		&=c^{20}	
	}	
}
\eks[2]{ \vs \vs 
	\alg{
		\left(3^\frac{5}{4}\right)^8&=3^{\frac{5}{4}\cdot8} \\
		&=3^{10}
	}	
}
\fork{\ref{potsomgrunn} \potsomgrunn}{
	La oss bruke $\left(a^{3}\right)^{4}$ som eksempel. Vi har at
	\begin{align*}
		\left(a^{3}\right)^{4} & =a^{3}\cdot a^{3}\cdot a^{3}\cdot a^{3}
	\end{align*}
	
	
	Av \rref{potgang} er
	\algv{
		a^{3}\cdot a^{3}\cdot a^{3}\cdot a^{3} & =a^{3+3+3+3}\\
		& =a^{3\cdot4}\\
		&=a^{12}
	}	
}

\newpage
\reg[\textit{n}-rot]{ \vs
	\[ a^\frac{1}{n}=\sqrt[n]{a} \]
	Symbolet \sym{$ \sqrt{\phantom{a}} $} kalles et \outl{rottegn}\index{rottegn}. For eksponenten $ \frac{1}{2} $ er det vanlig å utelate 2 i rottegnet:
	\[ a^\frac{1}{2}=\sqrt{a} \]
}
\eks{
	Av \rref{potsomgrunn} har vi at
	\alg{
		\left(a^b\right)^\frac{1}{b}&=a^{b\cdot \frac{1}{b}} \\
		&=a	
	}
	For eksempel er	
	\algv{
		9^\frac{1}{2}=\sqrt{9}=3 &\text{, siden } 3^2 =9 \vn
		125^\frac{1}{3}=\sqrt[3]{125}=5 &\text{, siden } 5^3 =125 \vn	
		16^\frac{1}{4}=\sqrt[4]{16}=2 &\text{, siden } 2^4 =16
	}	
}
\spr{
	$\sqrt{9} $ kalles ''kvadratrota til 9'' \vsk
	
	$ \sqrt[5]{9} $ kalles ''femterota til 9''.
}
\newpage
\section{Standardform} \label{Standardform}
Vi kan utnytte \rref{gangdesmed10100} og \rref{deledesmed10100}, og det vi kan om potenser, til å skrive tall på \outl{standardform}. \vsk

La oss se på tallet 6\,700. Av \rref{gangdesmed10100} vet vi at
\[ 6\,700=6,7\cdot1\,000 \]
Og siden $ 1000=10^3 $, er
\[ 6\,700=6,7\cdot1\,000=6,7\cdot 10^3 \]
\st{
	$ 6,7\cdot10^3 $ er 6\,700 skrevet på standardform fordi
	\begin{itemize}
		\item 6,7 er større eller lik 1 og mindre enn 10.
		\item $ 10^3 $ er en potens med grunntall 10 og eksponent 3, som er et heltall.
		\item 6,7 og $ 10^3 $ er ganget sammen.
	\end{itemize}
}
\linje \\[12pt]

La oss også se på tallet  0,093. Av \rref{deledesmed10100} har vi at
\[ 0,093=9,3: 100 \]
Men å dele med 100 er det samme som å gange med $ 10^{-2} $, altså er
\[ 0,093=9,3: 100=9,3\cdot10^{-2} \]
\st{
	$ 9,3\cdot10^{-2} $ er 0,093 skrevet på standardform fordi	
	\begin{itemize}
		\item 9,3 er større eller lik 1 og mindre enn 10.
		\item $ 10^{-2} $ er en potens med grunntall 10 og eksponent $ -2 $, som er et heltall.
		\item $ 9,3 $ og $ 10^{-2} $ er ganget sammen.
	\end{itemize} 
}
\reg[Standardform]{
	Et tall skrevet som
	\[ a\cdot 10^n \]
	hvor $ {1\leq|a|<10} $ og $ n $ er et heltall, er et tall skrevet på standardform.
}
\eks[1]{
	Skriv 980 på standardform.
	
	\sv \vsb
	\[ 980 = 9,8\cdot 10^2 \]
}
\eks[2]{
	Skriv 0,00671 på standardform.
	
	\sv \vsb
	\[ 0,00671 = 6,71\cdot 10^{-3} \]
}
\info{Tips}{
	For å skrive om tall på standardform kan du gjøre følgende:
	\begin{enumerate}
		\item Flytt komma slik at du får et tall som ligger mellom 0 og 10.
		\item Gang dette tallet med en tierpotens som har eksponent med tallverdi lik antallet plasser du flyttet komma. \qquad  Flyttet du komma mot venstre/høgre, er eksponenten positiv/negativ. 
	\end{enumerate}
}
\eks[3]{
	Skriv 9\,761\,432 på standardform.
	
	\sv \vs
	\begin{enumerate}
		\item 	Vi flytter komma 6 plasser til venstre, og får $ 9\colr{,}761432 $
		\item Vi ganger dette tallet med $ 10^6 $, og får at 
		\[ 9\,761\,432=9,761432\cdot 10^6 \] 
	\end{enumerate}
}
\newpage
\eks[4]{
	Skriv 0,00039 på standardform.
	
	\sv \vs
	\begin{enumerate}
		\item Vi flytter komma 4 plasser til høgre, og får $ 3,9 $.
		\item Vi ganger dette tallet med $ 10^{-4} $, og får at
		\[ 0,00039=3,9\cdot10^{-4} \]
	\end{enumerate}
}

\end{document}

