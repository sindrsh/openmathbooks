\documentclass[english,hidelinks, 11 pt, class=report,crop=false]{standalone}
\usepackage[T1]{fontenc}
%\usepackage[utf8]{inputenc}
\usepackage{lmodern} % load a font with all the characters
\usepackage{geometry}
\geometry{verbose,paperwidth=16.1 cm, paperheight=24 cm, inner=2.3cm, outer=1.8 cm, bmargin=2cm, tmargin=1.8cm}
\setlength{\parindent}{0bp}
\usepackage{import}
\usepackage[subpreambles=false]{standalone}
\usepackage{amsmath}
\usepackage{amssymb}
\usepackage{esint}
\usepackage{babel}
\usepackage{tabu}
\makeatother
\makeatletter

\usepackage{titlesec}
\usepackage{ragged2e}
\RaggedRight
\raggedbottom
\frenchspacing

\usepackage{graphicx}
\usepackage{float}
\usepackage{subfig}
\usepackage{placeins}
\usepackage{cancel}
\usepackage{framed}
\usepackage{wrapfig}
\usepackage[subfigure]{tocloft}
\usepackage[font=footnotesize,labelfont=sl]{caption} % Figure caption
\usepackage{bm}
\usepackage[dvipsnames, table]{xcolor}
\definecolor{shadecolor}{rgb}{0.105469, 0.613281, 1}
\colorlet{shadecolor}{Emerald!15} 
\usepackage{icomma}
\makeatother
\usepackage[many]{tcolorbox}
\usepackage{multicol}
\usepackage{stackengine}

\usepackage{esvect} %For vectors with capital letters

% For tabular
\usepackage{array}
\usepackage{multirow}
\usepackage{longtable} %breakable table

% Ligningsreferanser
\usepackage{mathtools} % for mathclap
%\mathtoolsset{showonlyrefs}

% sections without numbering in toc
\newcommand\tsec[1]{\phantomsection \addcontentsline{toc}{section}{#1}
	\section*{#1}}

% index
\usepackage{imakeidx}
\makeindex[title=Indeks]

%Footnote:
\usepackage[bottom, hang, flushmargin]{footmisc}
\usepackage{perpage} 
\MakePerPage{footnote}
\addtolength{\footnotesep}{2mm}
\renewcommand{\thefootnote}{\arabic{footnote}}
\renewcommand\footnoterule{\rule{\linewidth}{0.4pt}}
\renewcommand{\thempfootnote}{\arabic{mpfootnote}}

%colors
\definecolor{c1}{cmyk}{0,0.5,1,0}
\definecolor{c2}{cmyk}{1,0.25,1,0}
\definecolor{n3}{cmyk}{1,0.,1,0}
\definecolor{neg}{cmyk}{1,0.,0.,0}


\newcommand{\nreq}[1]{
\begin{equation}
	#1
\end{equation}
}


% Equation comments
\newcommand{\cm}[1]{\llap{\color{blue} #1}}


\usepackage[inline]{enumitem}
\newcounter{rg}
\numberwithin{rg}{chapter}


\newcommand{\reg}[2][]{\begin{tcolorbox}[boxrule=0.3 mm,arc=0mm,colback=blue!3] {\refstepcounter{rg}\phantomsection \large \textbf{\therg \;#1} \vspace{5 pt}}\newline #2  \end{tcolorbox}\vspace{-5pt}}
\newcommand{\regdef}[2][]{\begin{tcolorbox}[boxrule=0.3 mm,arc=0mm,colback=blue!3] {\refstepcounter{rg}\phantomsection \large \textbf{\therg \;#1} \vspace{5 pt}}\newline #2  \end{tcolorbox}\vspace{-5pt}}
\newcommand{\words}[1]{\begin{tcolorbox}[boxrule=0.3 mm,arc=0mm,colback=teal!3] #1  \end{tcolorbox}\vspace{-5pt}}

\newcommand\alg[1]{\begin{align*} #1 \end{align*}}

\newcommand\eks[2][]{\begin{tcolorbox}[boxrule=0.3 mm,arc=0mm,enhanced jigsaw,breakable,colback=green!3] {\large \textbf{\ekstitle #1} \vspace{5 pt}\\} #2 \end{tcolorbox}\vspace{-5pt} }

\newcommand{\st}[1]{\begin{tcolorbox}[boxrule=0.0 mm,arc=0mm,enhanced jigsaw,breakable,colback=yellow!12]{ #1} \end{tcolorbox}}

\newcommand{\spr}[1]{\begin{tcolorbox}[boxrule=0.3 mm,arc=0mm,enhanced jigsaw,breakable,colback=yellow!7] {\large \textbf{\sprtitle} \vspace{5 pt}\\} #1 \end{tcolorbox}\vspace{-5pt} }

\newcommand{\sym}[1]{\colorbox{blue!15}{#1}}

\newcommand{\info}[2]{\begin{tcolorbox}[boxrule=0.3 mm,arc=0mm,enhanced jigsaw,breakable,colback=cyan!6] {\large \textbf{#1} \vspace{5 pt}\\} #2 \end{tcolorbox}\vspace{-5pt} }

\newcommand\algv[1]{\vspace{-11 pt}\begin{align*} #1 \end{align*}}

\newcommand{\regv}{\vspace{5pt}}
\newcommand{\mer}{\textsl{\note}: }
\newcommand{\mers}[1]{{\footnotesize \mer #1}}
\newcommand\vsk{\vspace{11pt}}
\newcommand{\tbs}{\vspace{5pt}}
\newcommand\vs{\vspace{-11pt}}
\newcommand\vsb{\vspace{-16pt}}
\newcommand\br{\\[5 pt]}
\newcommand{\figp}[1]{../fig/#1}
\newcommand\algvv[1]{\vs\vs\begin{align*} #1 \end{align*}}
\newcommand{\y}[1]{$ {#1} $}
\newcommand{\os}{\\[5 pt]}
\newcommand{\prbxl}[2]{
\parbox[l][][l]{#1\linewidth}{#2
	}}
\newcommand{\prbxr}[2]{\parbox[r][][l]{#1\linewidth}{
		\setlength{\abovedisplayskip}{5pt}
		\setlength{\belowdisplayskip}{5pt}	
		\setlength{\abovedisplayshortskip}{0pt}
		\setlength{\belowdisplayshortskip}{0pt} 
		\begin{shaded}
			\footnotesize	#2 \end{shaded}}}
\newcommand{\fgbxr}[2]{
	\parbox[r][][l]{#1\linewidth}{#2
}}		

\renewcommand{\cfttoctitlefont}{\Large\bfseries}
\setlength{\cftaftertoctitleskip}{0 pt}
\setlength{\cftbeforetoctitleskip}{0 pt}

\newcommand{\bs}{\\[3pt]}
\newcommand{\vn}{\\[6pt]}
\newcommand{\fig}[1]{\begin{figure}[H]
		\centering
		\includegraphics[]{\figp{#1}}
\end{figure}}

\newcommand{\figc}[2]{\begin{figure}
		\centering
		\includegraphics[]{\figp{#1}}
		\caption{#2}
\end{figure}}
\newcommand{\arc}[1]{{
		\setbox9=\hbox{#1}%
		\ooalign{\resizebox{\wd9}{\height}{\texttoptiebar{\phantom{A}}}\cr\textit{#1}}}}

\newcommand{\sectionbreak}{\clearpage} % New page on each section

\newcommand{\nn}[1]{
\begin{equation*}
	#1
\end{equation*}
}

\newcommand{\enh}[1]{\,\textrm{#1}}

%asin, atan, acos
\DeclareMathOperator{\atan}{atan}
\DeclareMathOperator{\acos}{acos}
\DeclareMathOperator{\asin}{asin}

% Comments % old cm, ggb cm is new
%\newcommand{\cm}[1]{\llap{\color{blue} #1}}

%%%

\newcommand\fork[2]{\begin{tcolorbox}[boxrule=0.3 mm,arc=0mm,enhanced jigsaw,breakable,colback=yellow!7] {\large \textbf{#1 (\expl)} \vspace{5 pt}\\} #2 \end{tcolorbox}\vspace{-5pt} }
 
%colors
\newcommand{\colr}[1]{{\color{red} #1}}
\newcommand{\colb}[1]{{\color{blue} #1}}
\newcommand{\colo}[1]{{\color{orange} #1}}
\newcommand{\colc}[1]{{\color{cyan} #1}}
\definecolor{projectgreen}{cmyk}{100,0,100,0}
\newcommand{\colg}[1]{{\color{projectgreen} #1}}

% Methods
\newcommand{\metode}[2]{
	\textsl{#1} \\[-8pt]
	\rule{#2}{0.75pt}
}

%Opg
\newcommand{\abc}[1]{
	\begin{enumerate}[label=\alph*),leftmargin=18pt]
		#1
	\end{enumerate}
}
\newcommand{\abcs}[2]{
	\begin{enumerate}[label=\alph*),start=#1,leftmargin=18pt]
		#2
	\end{enumerate}
}
\newcommand{\abcn}[1]{
	\begin{enumerate}[label=\arabic*),leftmargin=18pt]
		#1
	\end{enumerate}
}
\newcommand{\abch}[1]{
	\hspace{-2pt}	\begin{enumerate*}[label=\alph*), itemjoin=\hspace{1cm}]
		#1
	\end{enumerate*}
}
\newcommand{\abchs}[2]{
	\hspace{-2pt}	\begin{enumerate*}[label=\alph*), itemjoin=\hspace{1cm}, start=#1]
		#2
	\end{enumerate*}
}

% Exercises


\newcounter{opg}
\numberwithin{opg}{section}

\newcounter{grub}
\numberwithin{opg}{section}
\newcommand{\op}[1]{\vspace{15pt} \refstepcounter{opg}\large \textbf{\color{blue}\theopg} \vspace{2 pt} \label{#1} \\}
\newcommand{\eksop}[2]{\vspace{15pt} \refstepcounter{opg}\large \textbf{\color{blue}\theopg} (#1) \vspace{2 pt} \label{#2} \\}

\newcommand{\nes}{\stepcounter{section}
	\setcounter{opg}{0}}
\newcommand{\opr}[1]{\vspace{3pt}\textbf{\ref{#1}}}
\newcommand{\oeks}[1]{\begin{tcolorbox}[boxrule=0.3 mm,arc=0mm,colback=white]
		\textit{\ekstitle: } #1	  
\end{tcolorbox}}
\newcommand\opgeks[2][]{\begin{tcolorbox}[boxrule=0.1 mm,arc=0mm,enhanced jigsaw,breakable,colback=white] {\footnotesize \textbf{\ekstitle #1} \\} \footnotesize #2 \end{tcolorbox}\vspace{-5pt} }


% tag exercises
\newcommand{\tagop}[1]{ 
{\small \color{Gray} #1} \os
}

% License
\newcommand{\lic}{
This book is part of the \net{https://sindrsh.github.io/openmathbooks/}{OpenMathBooks} project. OpenMathBooks © 2022 by Sindre Sogge Heggen is licensed under CC BY-NC-SA 4.0. To view a copy of this license, visit \net{http://creativecommons.org/licenses/by-nc-sa/4.0/}{http://creativecommons.org/licenses/by-nc-sa/4.0/}}

%referances
\newcommand{\net}[2]{{\color{blue}\href{#1}{#2}}}
\newcommand{\hrs}[2]{\hyperref[#1]{\color{blue}#2 \ref*{#1}}}
\newcommand{\refunnbr}[2]{\hyperref[#1]{\color{blue}#2}}


\newcommand{\openmath}{\net{https://sindrsh.github.io/openmathbooks/}{OpenMathBooks}}
\newcommand{\am}{\net{https://sindrsh.github.io/FirstPrinciplesOfMath/}{AM1}}
\newcommand{\mb}{\net{https://sindrsh.github.io/FirstPrinciplesOfMath/}{MB}}
\newcommand{\tmen}{\net{https://sindrsh.github.io/FirstPrinciplesOfMath/}{TM1}}
\newcommand{\tmto}{\net{https://sindrsh.github.io/FirstPrinciplesOfMath/}{TM2}}
\newcommand{\amto}{\net{https://sindrsh.github.io/FirstPrinciplesOfMath/}{AM2}}
\newcommand{\eksbm}{
\footnotesize
Dette er opppgaver som har blitt gitt ved sentralt utformet eksamen i Norge. Oppgavene er laget av Utdanningsdirektoratet. Forkortelser i parantes viser til følgende:
\begin{center}
	\begin{tabular}{c|c}
		E & Eksempeloppgave \\
		V/H & Eksamen fra vårsemesteret/høstsemesteret\\
		G/1P/1T/R1/R2 & Fag  \\
		XX & År 20XX \\
		D1/D2 & Del 1/Del 2
	\end{tabular}
\end{center}
Tekst og innhold kan her være noe endret i forhold til originalen.
}

%Excel og GGB:

\newcommand{\g}[1]{\begin{center} {\tt #1} \end{center}}
\newcommand{\gv}[1]{\begin{center} \vspace{-11 pt} {\tt #1}  \end{center}}
\newcommand{\cmds}[2]{{\tt #1}\\
	#2}

% outline word
\newcommand{\outl}[1]{{\boldmath \color{teal}\textbf{#1}}}
%line to seperate examples
\newcommand{\linje}{\rule{\linewidth}{1pt} }


%Vedlegg
\newcounter{vedl}
\newcounter{vedleq}
\renewcommand\thevedl{\Alph{vedl}}	
\newcommand{\nreqvd}{\refstepcounter{vedleq}\tag{\thevedl \thevedleq}}

%%% Writing code

\usepackage{listings}


\definecolor{codegreen}{rgb}{0,0.6,0}
\definecolor{codegray}{rgb}{0.5,0.5,0.5}
\definecolor{codepurple}{rgb}{0.58,0,0.82}
\definecolor{backcolour}{rgb}{0.95,0.95,0.92}

\newcommand{\pymet}[1]{{\ttfamily\color{magenta} #1}}
\newcommand{\pytype}[1]{{\ttfamily\color{codepurple} #1}}

\lstdefinestyle{mystyle}{
	backgroundcolor=\color{backcolour},   
	commentstyle=\color{codegreen},
	keywordstyle=\color{magenta},
	numberstyle=\tiny\color{codegray},
	stringstyle=\color{codepurple},
	basicstyle=\ttfamily\footnotesize,
	breakatwhitespace=false,         
	breaklines=true,                 
	captionpos=b,                    
	keepspaces=true,                 
	numbers=left,                    
	numbersep=5pt,                  
	showspaces=false,                
	showstringspaces=false,
	showtabs=false,                  
	tabsize=2,
	inputencoding=utf8,
	extendedchars=true,
	literate= {
		{å}{{\aa}}1 
		{æ}{{\ae}}1 
		{ø}{{\o}}1
	}
}

\lstset{style=mystyle}

\newcommand{\python}[1]{
\begin{tcolorbox}[boxrule=0.3 mm,arc=0mm,colback=white]
\lstinputlisting[language=Python]{#1}
\end{tcolorbox}}
\newcommand{\pythonut}[2]{
\begin{tcolorbox}[boxrule=0.3 mm,arc=0mm,colback=white]
\small 
%\textbf{Kode}
\lstinputlisting[language=Python]{#1}	
\vspace{11pt}
\textbf{Utdata} \\ \ttfamily
#2
\end{tcolorbox}}
%%%

%page number
%\usepackage{fancyhdr}
%\pagestyle{fancy}
%\fancyhf{}
%\renewcommand{\headrule}{}
%\fancyhead[RO, LE]{\thepage}

\usepackage{datetime2}
%%\usepackage{sansmathfonts} for dyslexia-friendly math
\usepackage[]{hyperref}

\newcommand{\texandasy}{Teksten er skrevet i \LaTeX\ og figurene er lagd ved hjelp av Asymptote.}

\newcommand{\expl}{forklaring}

\newcommand{\note}{Merk}
\newcommand{\notesm}[1]{{\footnotesize \textsl{\note:} #1}}
\newcommand{\ekstitle}{Eksempel }
\newcommand{\sprtitle}{Språkboksen}
\newcommand{\vedlegg}[1]{\refstepcounter{vedl}\section*{Vedlegg \thevedl: #1}  \setcounter{vedleq}{0}}

\newcommand\sv{\vsk \textbf{Svar} \vspace{4 pt}\\}

% exercises
\newcommand{\opgt}{\newpage \phantomsection \addcontentsline{toc}{section}{Oppgaver} \section*{Oppgaver for kapittel \thechapter}\vs \setcounter{section}{1}}

\newcommand{\grubop}[1]{\vspace{15pt} \refstepcounter{grub}\large \textbf{\color{blue} Gruble \thegrub} \vspace{2 pt} \label{#1} \\}
\newcommand{\grubr}[1]{\vspace{3pt}\textbf{Gruble \ref{#1}}}

%references
\newcommand{\reftab}[1]{\hrs{#1}{tabell}}
\newcommand{\rref}[1]{\hrs{#1}{regel}}
\newcommand{\dref}[1]{\hrs{#1}{definisjon}}
\newcommand{\refkap}[1]{\hrs{#1}{kapittel}}
\newcommand{\refsec}[1]{\hrs{#1}{seksjon}}
\newcommand{\refdsec}[1]{\hrs{#1}{delseksjon}}
\newcommand{\refved}[1]{\hrs{#1}{vedlegg}}
\newcommand{\eksref}[1]{\textsl{#1}}
\newcommand\fref[2][]{\hyperref[#2]{\textsl{figur \ref*{#2}#1}}}
\newcommand{\refop}[1]{{\color{blue}Oppgave \ref{#1}}}
\newcommand{\refops}[1]{{\color{blue}oppgave \ref{#1}}}

% Solutions manual
\newcommand{\selos}{Se løsningsforslag.}

\newcommand{\ompref}{Omgjøring av prefikser}



\begin{document}
\section{Introduction to Python}
Python is a programming language for \outl{text-based coding}. This means that the actions we want to be executed must be coded as text. The file containing all the code is referred to as a \outl{script}. The visible result of running the script is termed \outl{output}\footnote{\outl{Output} in English.}. There are various ways to run one's script; for example, one can use an online compiler like \net{https://www.programiz.com/python-programming/online-compiler/}{programmiz.com}.
\subsection{Object, Type, Function, and Expression}
Our first script consists of just one line of code:

\pythonut{helloworld.py}{
	Hello world!
}

In the upcoming sections, the terms \outl{object}, \outl{type}, \outl{function}, and \outl{expression} will frequently be discussed.
\begin{itemize}
	\item Almost everything in Python is objects. In the above script, both \pymet{print()} and \pytype{"Hello world"} are objects.
	\item Objects come in different types. \pymet{print()} is of the \pytype{function} type, while \pytype{"Hello world"} is of the \pytype{str} type\footnote{'str' is an abbreviation for the English word 'string'.}. The operations that can be executed with various objects depend on their types.
	\item Functions can accept \outl{arguments} and then perform operations. In the script above, the \pymet{print()} function takes the argument \pytype{"Hello world"} and displays the text as output.
	\item Expressions have strong similarities with functions, but they don't accept arguments.
\end{itemize}

\newpage
\subsubsection{Assignment and Calculation}
Text and numbers can be seen as some of the smallest building blocks (objects). Python has one type for text and two types for real numbers:
\begin{center}
	\begin{tabular}{r|l} \rowcolor{gray!10}
		\pytype{str} & text \\
		\pytype{int} & integer \\ \rowcolor{gray!10}
		\pytype{float} & decimal
	\end{tabular} 
\end{center}
It is usually useful to give our objects names. We do this by writing the name followed by \texttt{=} and the object. \outl{Comments} are text that is not treated as code. We can write comments by starting the sentence with \texttt{\color{codegreen} \#}.
\python{strintfloat.py} \vsk

With Python, we can of course perform classic arithmetic operations: \vspace{4pt}

\pythonut{opr.py}{
	a+b =  7\\
	a-b =  3\\
	a*b =  10\\
	a/b =  2.5\\
	a**b =  25\\
	a//b =  2 \\
	a\%b =  1
} \vsk
\newpage
The functions \pymet{str()}, \pymet{int()} and \pymet{float()} can be used to convert objects to types \pytype{int} or \pytype{float}: 
\pythonut{strintfloatfunk.py}{
	32.0\\
	6\\
	6.0
} \vsk

One important thing to be aware of is that \texttt{=} in Python \textsl{does not} mean the same as \sym{$ = $} in mathematics. While $ \sym{=} $ can be translated to ''equals'', we can say that \texttt{=} can be translated to 'is assigned to'.
\pythonut{assign.py}{
	5 \\
	6 
} \newpage
For an object to add itself and another value is so common in programming that Python has its own operator for it:
\pythonut{aplus1.py}{
	5\\
	6
}\vsk

Although computers are extremely fast at performing calculations, they have a limitation that is important to be aware of: rounding errors. One reason for this is that computers can only use a certain number of decimals to represent numbers. Another reason is that computers use the \net{https://en.wikipedia.org/wiki/Binary_number}{binary system}. There are many values that we can write exactly in the decimal system that cannot be written exactly in the binary system. To address this, we can use the \texttt{round()}\label{round()} function:\regv

\pythonut{rnd.py}{
	0.8300000000000001\\
	0.83
}
\newpage
\subsection{Custom Functions}
Using the method \pymet{def}, you can create your own functions. A function can perform actions, and it can \outl{return} one or more objects. It can also accept arguments. The code we write inside a function is only executed if we \outl{call} the function. \regv

\pythonut{func.py}{
	Hi. Someone called function b. The argument given was: Hello! \\
	5
}

\subsection{Boolean Values and Conditions}
The values \pytype{True} and \pytype{False} are called \outl{boolean values}. These will be the result when we check if objects are equal or different. To check this, we have the \outl{comparative operators}:
\begin{center}
	\begin{tabular}{c|c}
		\textbf{operator} & \textbf{meaning} \\ \hline
		\texttt{==}	& is equal to \\ \rowcolor{gray!10}
		\texttt{!=} & is \textsl{not} equal to\\
		\texttt{>} & is greater than \\ \rowcolor{gray!10}
		\texttt{>=} & is greater than, or equal to \\
		\texttt{<} & is less than \\ \rowcolor{gray!10}
		\texttt{<=} & is less than, or equal to \\		
	\end{tabular}
\end{center}
\pythonut{bool1.py}{
	False\\
	True\\
	True\\
	False\\
}\vsk

In addition to the comparative operators, we can use the \outl{logical operators} \pytype{and}, \pytype{or}, and \pytype{not}.
\pythonut{bool2.py}{
	False\\
	True\\
	True
}
\spr{
	Checks that use both comparative and logical operators will henceforth be called \outl{conditions}.
}
\subsection{Expressions \pymet{if, else}, and \pymet{elif}}
When we want to perform actions only \textsl{if} a condition is true (\pytype{True}), we use the expression \pymet{if} in front of the condition. The code we write indented under the \pymet{if} line will only be executed if the condition evaluates to \pytype{True}.
\pythonut{if.py}{
	Yep, c is greater than b
} 
If you first want to check if a condition is true, and then perform actions if it's \textsl{not}, you can use the expression \pymet{else}:
\pythonut{else.py}{
	But this comes because the condition in the if-line above was False
}
The expression \pymet{else} only considers (and doesn't make sense without) the \pymet{if} expression right above it. If we want actions to be performed \textsl{only} if no previous \pymet{if} expressions produced any result, we must use\footnote{\pymet{elif} is a shortcut for \pymet{else if}, which can also be used.} the expression \pymet{elif}. This is an \pymet{if} expression that takes effect if the \pymet{if} expression above did \textsl{not} take effect.
\pythonut{elif.py}{
	Now we are sure that 1 < b < 3
}
\info{Note}{
	When working with numbers, some conditions you expect to be \pytype{True} might turn out to be \pytype{False}. This often deals with rounding errors, as mentioned on page \pageref{round()}.
}
\newpage

\subsection{Lists}
Lists can be used to collect objects. The objects in the list are called the \outl{elements} of the list.
\python{list1.py}
The elements in lists are \outl{indexed}. The first object has index 0, the second object has index 1, and so on:
\pythonut{list1.py}{
	96 \\
	99 \\
	98
}
Using the built-in function \texttt{append()} we can add an object to the end of the list. This is an \outl{in-built function}\footnote{In short, it means that only certain types of objects can use this function.}, which we write at the end of the list name, preceded by a dot.
\pythonut{list3.py}{
	[] \newline
	[3] \newline
	[3, 7]
}
\newpage
With the \texttt{pop()} function, we can retrieve an object from the list.
\pythonut{list4.py}{
	a = 19 \\
	min\_liste = [6, 10, 15] \\
	a = 10 \\
	min\_liste = [6, 15]
}

\info{Explain to yourself}{
	What's the difference between writing \texttt{a = min\_liste[1]} and \texttt{a = min\_liste.pop(1)}?
} \vsk

With the \texttt{sort()} function, we can sort the elements in the list.
\pythonut{list5.py}{
	[0, 1, 3, 4, 7, 8, 9] 
	\\
	
	['a', 'b', 'c', 'd', 'e']
}\vsk

\newpage
With the \texttt{count()} function, we can count repeated elements in the list.
\pythonut{list6.py}{
	4\\
	1\\
	2\\
	0
} \vsk

With the \texttt{len()} function, we can find the number of elements in a list, and with the \texttt{sum()} function, we can find the sum of lists with numbers as elements.
\pythonut{list7.py}{
	5\\
	3\\
	15
}
\newpage
With the \texttt{in} expression, we can check if an element is in a list.
\pythonut{list8.py}{
	True\\
	False
}

\newpage
\subsection{Loops; \pymet{for} and \pymet{while}}
\subsubsection{\pymet{for} loop}
For objects containing multiple elements, we can use \pymet{for} loops to perform actions for each element. The actions must be written with an indentation after the \pymet{for} statement:
\pythonut{for.py}{
	5 \\
	50 \\[12pt]
	10\\
	100\\[12pt]
	15\\
	150
}
\spr{
	Going through each element in (for example) a list is called "iterating over the list".
}\vsk

Often, it's desired to iterate over the integers $0, 1, 2$ and so forth. For this, we can use \pymet{range()}:
\pythonut{forrang.py}{
	0 \\
	1 \\
	2
}
\subsubsection{\pymet{while} loop}
If we want actions to be performed until a condition is met, we can use a \pymet{while} loop: \regv
\pythonut{while1.py}{
	1\\
	2\\
	3\\
	4\\
}

\subsection{\pymet{input()}}
We can use the \pymet{input()} function to enter text while the script is running:
\python{input0.py}
The text written inside \pymet{input()} in the script above is the text we want displayed before the text to be entered. Line 2 of this code will not execute until text is entered.
\pythonut{input0.py}{
	Enter text here: OK\\
	OK
}
\newpage
The object provided by an \pymet{input()} function will always be of type \pytype{str}. One must always ensure to convert objects to the correct type:
\pythonut{input1.py}{
	Let's calculate a*b\\
	a = 3.7\\
	b = 4\\
	a*b = 14.8
}
\subsection{Error Messages}
Claim: All programmers will experience that the script does not run because we haven't written the code correctly. This is called a \outl{syntax error}. With a syntax error, you will be informed about which line the error is on and what the error is. The most common errors are:
\begin{itemize}
	\item \textbf{Forgetting indentation when using methods like \pymet{def}, \pymet{for}, \pymet{while}, and \pymet{if}}
	\pythonut{erindent.py}{
		line 5, in <module>\\
		print("a*b is greater than 48000")\\
		\^{} \\
		IndentationError: expected an indented block after 'if' statement on line 4
	}
	\item \textbf{Performing operations on types where it doesn't make sense}
	\pythonut{ertype.py}{
		line 2, in <module> \\
		b\_raised\_to\_second = b**2\\
		TypeError: unsupported operand type(s) for ** or\\ pow(): 'str' and 'int'\\
	}
\end{itemize}




\end{document}