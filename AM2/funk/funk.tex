\documentclass[english,hidelinks,pdftex, 11 pt, class=report,crop=false]{standalone}
\input{../am2_bm}
\usepackage[T1]{fontenc}
\usepackage[utf8]{luainputenc}
\usepackage{lmodern} % load a font with all the characters
\usepackage{geometry}
\geometry{verbose,a4paper, inner=0cm, outer=0 cm, bmargin=2cm, tmargin=1cm}
%\textwidth=12cm
\setlength{\parindent}{0bp}
\usepackage{import}
\usepackage[subpreambles=false]{standalone}
\usepackage{amsmath}
\usepackage{amssymb}
\usepackage{esint}
\usepackage{babel}
\usepackage{tabu}
\usepackage[dvipsnames, table]{xcolor}
\usepackage{cancel}
\makeatother
\makeatletter
\usepackage{datetime2}
\usepackage{titlesec}
\usepackage[many]{tcolorbox}

% Eheter
\newcommand{\enh}[1]{\,\textrm{#1}}
%referances
\newcommand{\net}[2]{{\color{blue}\href{#1}{#2}}}

%Spaces
\newcommand{\vsk}{\\[12pt]}
\newcommand{\vs}{\vspace{-12pt}}

% Tabell for opplegg

\newcommand{\ovlist}[1]{
\vspace{-16pt}
\begin{itemize}
	#1
\end{itemize}
}

% Chapters and sections
\titleformat{\section}[block]{\bfseries}{\hspace{3cm}\thesection}{5pt}{}
\titleformat{\subsection}[block]{\bfseries}{\hspace{3cm}\thesection}{5pt}{}
\newcommand{\sectionbreak}{\clearpage} % New page on each section
 

\newlength{\mywidth}
\setlength{\mywidth}{14cm}

\newcommand{\cont}[1]{
\begin{tcolorbox}[center, boxrule=0.0 mm, width=\mywidth,arc=0mm,enhanced jigsaw,,colback=white,breakable]
#1	
\end{tcolorbox}
}

\newcommand{\info}[5]{
\begin{tcolorbox}[center, boxrule=0.1 mm, width=\mywidth,arc=0mm,enhanced jigsaw,breakable,colback=yellow!5]	
	
	\footnotesize
	\textbf{Øvingsområde}\\[5pt] #1 
	
	\textbf{Utstyr}\\ #2  \\
	
	\begin{tabular}{@{} p{4cm} p{4cm} l} 
		\textbf{Tid} & \textbf{Elevinndeling} & \textbf{Læringsarena} \\
		#3  & #4 & #5
	\end{tabular} 
\end{tcolorbox}	
}

\newcommand{\gjen}[1]{\begin{tcolorbox}[center,boxrule=0.1 mm, width=\mywidth,arc=0mm,colback=blue!3] {\large \textbf{Gjennomføring} \vspace{5 pt}}\newline #1  \end{tcolorbox}\vspace{-5pt}}
\newcommand{\eks}[1]{\begin{tcolorbox}[center,boxrule=0.1 mm, width=\mywidth,arc=0mm,colback=green!3] {\large \textbf{Eksempel} \vspace{5 pt}}\newline #1  \end{tcolorbox}\vspace{-5pt}}

\newcounter{opl}
%\numberwithin{opl}{article}


\newcommand{\opl}[1]{
\newpage
{\refstepcounter{opl} %\phantomsection 
\large \textbf{\theopl \;#1} \vsk}
}

% Headlines
\newcommand{\fork}{\textbf{Forkunnskapar}\\}
\newcommand{\forb}{\textbf{Forberedelsar}\\}
\newcommand{\opgvr}{\textbf{Oppgaver}}



%colors
\newcommand{\colr}[1]{{\color{red} #1}}
\newcommand{\colb}[1]{{\color{blue} #1}}
\newcommand{\colo}[1]{{\color{orange} #1}}
\newcommand{\colc}[1]{{\color{cyan} #1}}
\definecolor{projectgreen}{cmyk}{100,0,100,0}
\newcommand{\colg}[1]{{\color{projectgreen} #1}}

% Lister med bokstavar
\usepackage[inline]{enumitem}
% Opg
\newcommand{\abc}[1]{
	\begin{enumerate}[label=\alph*),leftmargin=18pt]
		#1
	\end{enumerate}
}

\usepackage[]{hyperref}


\begin{document}
\subsection*{Motvirkende krefter for kjørende bil}
\#algebra \#modellering \#andregradsfunksjon \#omgjøring av enheter \\\#proporsjonalitet \#digital kompetanse\vsk

La $ F $ være summen av kreftene som virker i motsatt retning av en bils kjøreretning. Ifølge en rapport\footnote{\net{https://sintef.brage.unit.no/sintef-xmlui/handle/11250/2468761}{https://sintef.brage.unit.no/sintef-xmlui/handle/11250/2468761}} fra SINTEF kan\footnote{Det er er her forutsatt flatt strekke, og sett vekk ifra motstand ved akselerasjon.} $ F $ tilnærmes som
\[ F(v)= mgC_r+\frac{1}{2}\rho v^2 D_m\qquad,\qquad v\geq10\]
\begin{center}
	\begin{tabular}{c|c|c|c}
		& \textbf{betydning} & \textbf{verdi}&\textbf{enhet}  \\ \hline
		$ v $ & bilens hastighet & variabel& m/s \\
		$ m $& bilens masse\footnotemark & 1409 & kg\\
		$ g $& tyngdeakselerasjonen & 9.81 & m/$ \text{s}^2 $ \\
		$ C_r $ & koeffisient for bilens rullemotstand & 0.015\\
		$ \rho $ & tettheten til luft & 1.25 & kg/$ \text{m}^3 $ \\
		$ D_m $& koeffisient for bilens luftmotstandsareal\footnotemark &0.74
	\end{tabular}
\end{center}

\footnotetext[3]{Det er tatt ugangspunkt i gjennomsnittsvekten til en norsk personbil.}
\footnotetext[4]{Verdien er hentet fra \net{en.wikipedia.org/wiki/Automobile\_drag\_coefficient\#Drag\_area}{en.wikipedia.org/wiki/Automobile\_drag\_coefficient\#Drag\_area}}

\abc{
\item Tegn grafen til $ F $ for $ v\in [10, 35] $
\item På intervallet gitt i oppgave a, for hvilken hastighet er det at 
\begin{itemize}
	\item rullemotstanden gir det største bidraget til $ F $?
	\item luftmotstanden gir det største bidraget til $ F $?
\end{itemize} 
Oppgi svarene rundet av til nærmeste heltall og målt i km/h.
\item Med ''energiforbruk''\footnote{Den totale energimengden en bil bruker på en kjørelengde vil være høyere enn det vi har kalt ''energiforbruket''. Som regel vil den totale energimengden som kreves for å kjøre en strekning være høyere jo høyere hastighet man har. Slik kan man anta at differansen i energiforbruk vi finner i denne oppgaven er et minimum for den reelle differansen i total energimengde.} mener vi her den energien som må til for å motvirke $ F $ over en viss kjørelengde.
Ved konstant hastighet er energiforbruket etter kjørt lengde proporsjonal med $ F $. På norske motorveier er 90\,km/h og 110\,km/h vanlige fartsgrenser. Hvor stor økning i energiforbruk vil en økning fra 90\,km/h til 110\,km/h innebære?\os

\item Lag en funksjon $ F_1 $ som gir $ F $ ut ifra bilens hastighet målt i km/h.
}

\end{document}