\documentclass[english,hidelinks,pdftex, 11 pt, class=report,crop=false]{standalone}
\usepackage[T1]{fontenc}
\usepackage[utf8]{luainputenc}
\usepackage{lmodern} % load a font with all the characters
\usepackage{geometry}
\geometry{verbose,a4paper, inner=0cm, outer=0 cm, bmargin=2cm, tmargin=1cm}
%\textwidth=12cm
\setlength{\parindent}{0bp}
\usepackage{import}
\usepackage[subpreambles=false]{standalone}
\usepackage{amsmath}
\usepackage{amssymb}
\usepackage{esint}
\usepackage{babel}
\usepackage{tabu}
\usepackage[dvipsnames, table]{xcolor}
\usepackage{cancel}
\makeatother
\makeatletter
\usepackage{datetime2}
\usepackage{titlesec}
\usepackage[many]{tcolorbox}

% Eheter
\newcommand{\enh}[1]{\,\textrm{#1}}
%referances
\newcommand{\net}[2]{{\color{blue}\href{#1}{#2}}}

%Spaces
\newcommand{\vsk}{\\[12pt]}
\newcommand{\vs}{\vspace{-12pt}}

% Tabell for opplegg

\newcommand{\ovlist}[1]{
\vspace{-16pt}
\begin{itemize}
	#1
\end{itemize}
}

% Chapters and sections
\titleformat{\section}[block]{\bfseries}{\hspace{3cm}\thesection}{5pt}{}
\titleformat{\subsection}[block]{\bfseries}{\hspace{3cm}\thesection}{5pt}{}
\newcommand{\sectionbreak}{\clearpage} % New page on each section
 

\newlength{\mywidth}
\setlength{\mywidth}{14cm}

\newcommand{\cont}[1]{
\begin{tcolorbox}[center, boxrule=0.0 mm, width=\mywidth,arc=0mm,enhanced jigsaw,,colback=white,breakable]
#1	
\end{tcolorbox}
}

\newcommand{\info}[5]{
\begin{tcolorbox}[center, boxrule=0.1 mm, width=\mywidth,arc=0mm,enhanced jigsaw,breakable,colback=yellow!5]	
	
	\footnotesize
	\textbf{Øvingsområde}\\[5pt] #1 
	
	\textbf{Utstyr}\\ #2  \\
	
	\begin{tabular}{@{} p{4cm} p{4cm} l} 
		\textbf{Tid} & \textbf{Elevinndeling} & \textbf{Læringsarena} \\
		#3  & #4 & #5
	\end{tabular} 
\end{tcolorbox}	
}

\newcommand{\gjen}[1]{\begin{tcolorbox}[center,boxrule=0.1 mm, width=\mywidth,arc=0mm,colback=blue!3] {\large \textbf{Gjennomføring} \vspace{5 pt}}\newline #1  \end{tcolorbox}\vspace{-5pt}}
\newcommand{\eks}[1]{\begin{tcolorbox}[center,boxrule=0.1 mm, width=\mywidth,arc=0mm,colback=green!3] {\large \textbf{Eksempel} \vspace{5 pt}}\newline #1  \end{tcolorbox}\vspace{-5pt}}

\newcounter{opl}
%\numberwithin{opl}{article}


\newcommand{\opl}[1]{
\newpage
{\refstepcounter{opl} %\phantomsection 
\large \textbf{\theopl \;#1} \vsk}
}

% Headlines
\newcommand{\fork}{\textbf{Forkunnskapar}\\}
\newcommand{\forb}{\textbf{Forberedelsar}\\}
\newcommand{\opgvr}{\textbf{Oppgaver}}



%colors
\newcommand{\colr}[1]{{\color{red} #1}}
\newcommand{\colb}[1]{{\color{blue} #1}}
\newcommand{\colo}[1]{{\color{orange} #1}}
\newcommand{\colc}[1]{{\color{cyan} #1}}
\definecolor{projectgreen}{cmyk}{100,0,100,0}
\newcommand{\colg}[1]{{\color{projectgreen} #1}}

% Lister med bokstavar
\usepackage[inline]{enumitem}
% Opg
\newcommand{\abc}[1]{
	\begin{enumerate}[label=\alph*),leftmargin=18pt]
		#1
	\end{enumerate}
}

\usepackage[]{hyperref}

\newcommand{\note}{Merk}
\newcommand{\notesm}[1]{{\footnotesize \textsl{\note:} #1}}
\newcommand{\ekstitle}{Eksempel }
\newcommand{\sprtitle}{Språkboksen}
\newcommand{\expl}{forklaring}
\newcommand{\pyt}{Pytagoras' setning}
\newcommand\sv{\vsk \textbf{Svar} \vspace{4 pt}\\}

%references
\newcommand{\reftab}[1]{\hrs{#1}{tabell}}
\newcommand{\rref}[1]{\hrs{#1}{regel}}
\newcommand{\dref}[1]{\hrs{#1}{definisjon}}
\newcommand{\refkap}[1]{\hrs{#1}{kapittel}}
\newcommand{\refsec}[1]{\hrs{#1}{seksjon}}
\newcommand{\refdsec}[1]{\hrs{#1}{delseksjon}}
\newcommand{\refved}[1]{\hrs{#1}{vedlegg}}
\newcommand{\eksref}[1]{\textsl{#1}}
\newcommand\fref[2][]{\hyperref[#2]{\textsl{figur \ref*{#2}#1}}}
\newcommand{\refop}[1]{{\color{blue}Oppgave \ref{#1}}}
\newcommand{\refops}[1]{{\color{blue}oppgave \ref{#1}}}


%Algebra
\newcommand{\kvadset}{Kvadratsetningene}
\newcommand{\aenato}{Sum-produkt-metoden}

% Geometry
\newcommand{\hlikb}{Midtnormalen i en likebeint trekant}
\newcommand{\arealsetn}{Arealsetningen}
\newcommand{\trkmedian}{Median}
\newcommand{\midtrk}{Midtnormal (i trekant)}
\newcommand{\innskrsirk}{Innskrevet sirkel}
\newcommand{\cossetn}{Cosinussetningen}
\newcommand{\perfvink}{Sentral- og periferivinkel}
\newcommand{\tang}{Tangent}

% Derivative
\newcommand{\derel}{Den deriverte av elementære funksjoner}
\newcommand{\divder}{Divisjonsregelen}
\newcommand{\kjernereg}{Kjerneregelen}
\newcommand{\prodregder}{Produktregelen}
\newcommand{\lhop}{L'Hopitals regel}

% Funksjonsdrofting
\newcommand{\monder}{Monotoniegenskaper og den deriverte}
\newcommand{\fderekstr}{$ \bm{f'=0} $ for lokale ektstremalpunkt}
\newcommand{\andredertest}{Andrederiverttesten}

% Vectors
\newcommand{\detar}{Arealformler med determinanter}
\newcommand{\avstpunktlin}{Avstand mellom punkt og linje}

%Appendix
\newcommand{\rolle}{Rolles teorem}
\newcommand{\meanval}{Middelverdisetningen}

% Solutions manual
\newcommand{\selos}{Se løsningsforslag.}

\begin{document}
	
\section{Vekstfart, fart og akselerasjon}
\regdef[Gjennomsnittlig og momentan vekstfart]{
Gitt en funksjon $ f(x) $. Da har vi at
\begin{itemize}
	\item stigningstallet til linja som går gjennom punktene $ (a, f(a)) $ og $ (b, f(b)) $ kalles den \outl{gjennomsnittlige vekstfarten} til $ f $ på intervallet $ [a, b] $.
	
	\item $ f'(a) $ kalles den
	\outl{momentane vekstfarten} til $ f $ i $ a $. 
\end{itemize}
}
\info{En praktisk tolkning av begrepene}{
I \mb\ har vi sett at stigningstallet til linja som går gjennom punktene $ (a, f(a)) $ og $ (b, f(b)) $ er gitt ved uttrykket
\[ \frac{f(b)-f(a)}{b-a} \]
Hvis vi antar at dette forholdet er konstant for $ {x\in[a, b]} $, antar vi at $ f $ og $ x $ representerer proporsjonale størrelser\footnote{Se \am.} på intervallet.\vsk

Hvis vi ser tilbake til (den alternative) definisjonen av den deriverte i \tmen, innser vi at $ f'(a) $ er den gjennomsnittlige vekstfaktoren til $ f $ på intervallet $ [a, b] $ når $ {b\to a} $. Da går $ [a, b] $ mot å innheholde bare ett element, som er $ a $.
}
\newpage
\eks[]{
Se for deg at vi slipper en ball fra 60 meter over bakken, og lar den falle fritt nedover. Når vi slipper ballen, starter vi også en stoppeklokke. Antall meter $ h $ ballen er over bakken etter $ t $ sekunder kan da tilnærmes ved funksjonen
\[ h(t)=5\left(10-t^2\right)\quad,\quad x\in\left[0, \sqrt{10}\right] \]
\abc{
\item Finn den gjennomsnittlige vekstfarten til $ h $ på intervallet $ t\in[1, 3] $. Gi en praktisk tolkning av denne verdien.
\item Finn den momentane vekstfarten til $ h $ i 3. Gi en praktisk tolkning av denne verdien.
} \vs
\sv
\vs
\abc{
\item $ {h(1)=5\left(10-1\right)=45}  $ og $ {h(3)=5\left(10-3^2\right) =5} $. Det betyr at stigningstallet til linja mellom $ (1, f(1)) $ og $ (3, f(3)) $ er
\[ \frac{5-45}{3-1}=-20 \]

Altså er vekstfarten til $ h $ på intervallet $ [1, 3] $ lik $ -20 $. Siden $ h $ representerer antall meter, og $ t $ representerer antall sekunder, representerer vekstfarten en størrelse med enheten 'm/s'. Dette er en enhet for fart. Hvis ballen skulle falt 40 meter \textsl{nedover} i løpet av 2 sekunder med konstant fart, måtte denne farten vært $ 20\enh{m/s} $.
\fig{ballb}
\item Siden $ {h'(t)=-10t }$, er $ {h'(3)=-30} $. Dette betyr at etter å ha falt i 3 sekunder, har ballen oppnådd farten $ 30\enh{m/s} $, i retning \textsl{nedover}.
}
}
\reg[\fartogaksvekt \label{fartogaksvekt}]{
Gitt en vektorfunksjon $ \vec{r}(t) $, hvor $ \vec{r} $ representerer en posisjon og $ t $ representerer tid. Da har vi at
\begin{itemize}
	\item $ \vec{r}\,'(t) $ kalles \outl{fartsvektoren} og $ | \vec{r}\,'(t)|$ kalles \outl{banefarten}.
	\item $ \vec{r}\,''(t) $ kalles \outl{akselerasjonsvektoren} og	 $ | \vec{r}\,''(t)|$ kalles\\ \outl{akselerasjonsvektoren}.
\end{itemize}
}
\fork{\ref{fartogaksvekt} \fartogaksvekt}{
Hvis $ \vec{r}(t) $ representerer en posisjon (altså en relativ lengde fra et referansepunkt), og $ t $ en tid, vil $ \vec{r}\,'(t) $ innebære en lengde delt på en tid. Da vil $ \vec{r}\,'(t) $ representere en størrelse med en enhet for fart. $ \vec{r}\,''(t) $  vil innebære en fart delt på en tid, som da vil representere en akselerasjon.
}

\end{document}