\documentclass[english,hidelinks,pdftex, 11 pt, class=report,crop=false]{standalone}
\usepackage[T1]{fontenc}
\usepackage[utf8]{luainputenc}
\usepackage{lmodern} % load a font with all the characters
\usepackage{geometry}
\geometry{verbose,a4paper, inner=0cm, outer=0 cm, bmargin=2cm, tmargin=1cm}
%\textwidth=12cm
\setlength{\parindent}{0bp}
\usepackage{import}
\usepackage[subpreambles=false]{standalone}
\usepackage{amsmath}
\usepackage{amssymb}
\usepackage{esint}
\usepackage{babel}
\usepackage{tabu}
\usepackage[dvipsnames, table]{xcolor}
\usepackage{cancel}
\makeatother
\makeatletter
\usepackage{datetime2}
\usepackage{titlesec}
\usepackage[many]{tcolorbox}

% Eheter
\newcommand{\enh}[1]{\,\textrm{#1}}
%referances
\newcommand{\net}[2]{{\color{blue}\href{#1}{#2}}}

%Spaces
\newcommand{\vsk}{\\[12pt]}
\newcommand{\vs}{\vspace{-12pt}}

% Tabell for opplegg

\newcommand{\ovlist}[1]{
\vspace{-16pt}
\begin{itemize}
	#1
\end{itemize}
}

% Chapters and sections
\titleformat{\section}[block]{\bfseries}{\hspace{3cm}\thesection}{5pt}{}
\titleformat{\subsection}[block]{\bfseries}{\hspace{3cm}\thesection}{5pt}{}
\newcommand{\sectionbreak}{\clearpage} % New page on each section
 

\newlength{\mywidth}
\setlength{\mywidth}{14cm}

\newcommand{\cont}[1]{
\begin{tcolorbox}[center, boxrule=0.0 mm, width=\mywidth,arc=0mm,enhanced jigsaw,,colback=white,breakable]
#1	
\end{tcolorbox}
}

\newcommand{\info}[5]{
\begin{tcolorbox}[center, boxrule=0.1 mm, width=\mywidth,arc=0mm,enhanced jigsaw,breakable,colback=yellow!5]	
	
	\footnotesize
	\textbf{Øvingsområde}\\[5pt] #1 
	
	\textbf{Utstyr}\\ #2  \\
	
	\begin{tabular}{@{} p{4cm} p{4cm} l} 
		\textbf{Tid} & \textbf{Elevinndeling} & \textbf{Læringsarena} \\
		#3  & #4 & #5
	\end{tabular} 
\end{tcolorbox}	
}

\newcommand{\gjen}[1]{\begin{tcolorbox}[center,boxrule=0.1 mm, width=\mywidth,arc=0mm,colback=blue!3] {\large \textbf{Gjennomføring} \vspace{5 pt}}\newline #1  \end{tcolorbox}\vspace{-5pt}}
\newcommand{\eks}[1]{\begin{tcolorbox}[center,boxrule=0.1 mm, width=\mywidth,arc=0mm,colback=green!3] {\large \textbf{Eksempel} \vspace{5 pt}}\newline #1  \end{tcolorbox}\vspace{-5pt}}

\newcounter{opl}
%\numberwithin{opl}{article}


\newcommand{\opl}[1]{
\newpage
{\refstepcounter{opl} %\phantomsection 
\large \textbf{\theopl \;#1} \vsk}
}

% Headlines
\newcommand{\fork}{\textbf{Forkunnskapar}\\}
\newcommand{\forb}{\textbf{Forberedelsar}\\}
\newcommand{\opgvr}{\textbf{Oppgaver}}



%colors
\newcommand{\colr}[1]{{\color{red} #1}}
\newcommand{\colb}[1]{{\color{blue} #1}}
\newcommand{\colo}[1]{{\color{orange} #1}}
\newcommand{\colc}[1]{{\color{cyan} #1}}
\definecolor{projectgreen}{cmyk}{100,0,100,0}
\newcommand{\colg}[1]{{\color{projectgreen} #1}}

% Lister med bokstavar
\usepackage[inline]{enumitem}
% Opg
\newcommand{\abc}[1]{
	\begin{enumerate}[label=\alph*),leftmargin=18pt]
		#1
	\end{enumerate}
}

\usepackage[]{hyperref}

\newcommand{\note}{Merk}
\newcommand{\notesm}[1]{{\footnotesize \textsl{\note:} #1}}
\newcommand{\ekstitle}{Eksempel }
\newcommand{\sprtitle}{Språkboksen}
\newcommand{\expl}{forklaring}
\newcommand{\pyt}{Pytagoras' setning}
\newcommand\sv{\vsk \textbf{Svar} \vspace{4 pt}\\}

%references
\newcommand{\reftab}[1]{\hrs{#1}{tabell}}
\newcommand{\rref}[1]{\hrs{#1}{regel}}
\newcommand{\dref}[1]{\hrs{#1}{definisjon}}
\newcommand{\refkap}[1]{\hrs{#1}{kapittel}}
\newcommand{\refsec}[1]{\hrs{#1}{seksjon}}
\newcommand{\refdsec}[1]{\hrs{#1}{delseksjon}}
\newcommand{\refved}[1]{\hrs{#1}{vedlegg}}
\newcommand{\eksref}[1]{\textsl{#1}}
\newcommand\fref[2][]{\hyperref[#2]{\textsl{figur \ref*{#2}#1}}}
\newcommand{\refop}[1]{{\color{blue}Oppgave \ref{#1}}}
\newcommand{\refops}[1]{{\color{blue}oppgave \ref{#1}}}


%Algebra
\newcommand{\kvadset}{Kvadratsetningene}
\newcommand{\aenato}{Sum-produkt-metoden}

% Geometry
\newcommand{\hlikb}{Midtnormalen i en likebeint trekant}
\newcommand{\arealsetn}{Arealsetningen}
\newcommand{\trkmedian}{Median}
\newcommand{\midtrk}{Midtnormal (i trekant)}
\newcommand{\innskrsirk}{Innskrevet sirkel}
\newcommand{\cossetn}{Cosinussetningen}
\newcommand{\perfvink}{Sentral- og periferivinkel}
\newcommand{\tang}{Tangent}

% Derivative
\newcommand{\derel}{Den deriverte av elementære funksjoner}
\newcommand{\divder}{Divisjonsregelen}
\newcommand{\kjernereg}{Kjerneregelen}
\newcommand{\prodregder}{Produktregelen}
\newcommand{\lhop}{L'Hopitals regel}

% Funksjonsdrofting
\newcommand{\monder}{Monotoniegenskaper og den deriverte}
\newcommand{\fderekstr}{$ \bm{f'=0} $ for lokale ektstremalpunkt}
\newcommand{\andredertest}{Andrederiverttesten}

% Vectors
\newcommand{\detar}{Arealformler med determinanter}
\newcommand{\avstpunktlin}{Avstand mellom punkt og linje}

%Appendix
\newcommand{\rolle}{Rolles teorem}
\newcommand{\meanval}{Middelverdisetningen}

% Solutions manual
\newcommand{\selos}{Se løsningsforslag.}


\begin{document}
\section{Python}
\mers{Denne teksten bygger på teksten om Python i \am}.
\subsection{NumPy}
I Python kan man importere det som kalles \outl{bibliotek} for å få tilgang til enda flere typer objekter, funksjoner og liknende. NumPy er et bibliotek som inneholder typen \pymet{numpy.ndarray}. Denne typen har mange fellestrekk med en liste, men skiller seg ut ved at den inneholder et bestemt antall elemener. Dette gjør blant annet at prossesser som bruker NumPy-arrays istedenfor lister går raskere, og at NumPy-arrays egner seg bedre til regneoperasjoner.\vsk

For å lage NumPy-arrays må vi importere NumPy-biblioteket: \regv
\pythonut{array1.py}{
	[1 2] \newline
	[0 1 2 3]\newline
	[0. 0. 0.] \newline
	[ 2.  5.  8. 11.]
	
}
\info{Merk}{
	Til forskjell fra lister, er elementene skilt bare med mellomrom når de printes.
}

\newpage
\subsubsection{Klassisek regnearter}
Regneoperasjoner mellom NumPy-arrays blir utført elementvis:
\pythonut{arrayopr.py}{
	[12 24] \newline
	[20 80]
}
\subsubsection{Vektoroperasjoner}
NumPy-arrays fungerer ypperlig til å representere vektorer, og har innebygde metoder for å finne skalarprodukt, kryssprodukt og determinanter:\regv

\pythonut{skalkryss.py}{
	-33 \newline
	[-5  1 17] \newline
	17.0
}	
	
\section{Newtons metode}
Gitt en funskjon $ f(x) $ si at vi ønsker å finne et tall $ a $ slik at $ {f(a)=0} $. Ved \outl{Newtons metode} gjør vi denne antakelsen for å en tilnærming $ a $: \regv
\st{
La $ x_1 $ være skjæringspunktet mellom $ x $-aksen og tangenten til $ f $ i $ x_0 $. Vi antar da at $ |x_1-a|<|x_0-a| $. Sagt med ord antar vi at $ x_1 $ gir en bedre tilnærming for $ a $ enn det $ x_0 $ gjør.
}
Siden $ x_1 $ er skjæringspunktet mellom $ x $-aksen og tangenten til $ f $ i $ x_0 $, har vi  at\footnote{Se oppgave??}
\alg{
f'(x_0)(x_1-x_0)+f(x_0)&=0 \\
f'(x_0)x_1 &= f'(x_0)x_0-f(x_0) \\
x_1 &= x_0-\frac{f(x_0)}{f'(x_0)}
}
\begin{figure}
\subfloat[]{
\includegraphics{\figp{newt1a}}
}
\subfloat[]{
	\includegraphics{\figp{newt1b}}
}
\end{figure}
La $ x_2 $ være skjæringspunktet mellom $ x $-aksen og tangenten til $ f $ i $ x_1 $. Ved å gjenta prosedyren vi brukte for å finne $ x_1 $, kan vi finne $ x_2 $, som vi antar er en enda bedre tilnærming for $ a $ enn $ x_1 $.
Prosedyren kan vi gjenta fram til vi har funnet en $ x $-verdi som gir en tilstrekkelig\footnote{Hva som er en \textit{tilstrekkelig tilnærming} er det opp til oss selv å bestemme.} tilnærming til $ a$.
\newpage
\reg[Newtons metode]{
Gitt en funskjon $ f(x) $ si at vi ønsker å finne et tall $ a $ slik at $ {f(a)=0} $. Gitt $ x $-verdiene $ x_{n} $ og $ x_{n+1} $ for $ {n\in\mathbb{N}} $. Ved å bruke formelen
\nn{
x_{n+1} = x_n-\frac{f(x_n)}{f'(x_n)} 
}
antas det at $ x_{n+1} $ gir en bedre tilnærming for $ a $ enn $ x_{n} $.
}
\spr{
\outl{Newtons metode} kalles også \outl{Newton-Rhapsos metode}.
}
\info{Når er tilmærmingen god nok?}{
Newtons metode beskriver en iterasjonsprossess som man håper at nærmer seg en verdi. Hvis meotden lykkes, vil $ x_{n+1}$ og $ x_n $ etterhvert være veldig like, og slik kan en grense for hvor liten $ {|x_{n+1}-x_n|} $ kan være fungere som et godt mål for når iterasjonsprossessen skal stoppe.
}
\section{Trapesmetoden} \label{Trapesmetoden}
Gitt en funksjone $ f(x) $. Integralet $ \int_a^b f \,dx $ kan vi tilnærme ved å 
\st{
\begin{enumerate}
	\item Dele intervallet $ [a, b] $ inn i mindre intervall. Disse kaller vi \outl{delintervall}.
	\item Finne en tilnærmet verdi for integralet av $ f $ på hvert\\ delintervall.
	\item Summere verdiene fra punkt 2.
\end{enumerate}
}\regv

I \fref[a]{trapmetfig} har vi 3 like store delintervaller. Hvis vi setter $ {a=x_0} $ og $ {\Delta x=\frac{b-a}{3}} $, betyr dette at 
\alg{
x_1&=x_0+\Delta x & x_2&=x_0+2\Delta x & x_3&=x_3+3\Delta x=b
}
En tilnæret verdi for $ \int_{a}^{x_1}f\,dx $ får vi ved å finne arealet til trapeset med hjørner (husk at $ x_0=a $)
\alg{
(x_0, 0) && (x_1, 0) && (x_1, f(x_1)) && (x_0, f(a))
}
Dette arealet er gitt ved uttrykket
\[ \frac{1}{2}(x_1-x_0)\left[f(x_0)+f(x_1)\right]=
\frac{\Delta x}{2}\left[f(x_0)-f(x_1)\right] \]
Ved å tilnærme integralet for hvert delintervall på denne måten, kan vi skrive
\[ \int_{a}^{b} f\,dx \approx \frac{\Delta x}{2}\sum_{i=0}^{2} \left[f(x_i)+f(x_{i+1})\right] \]
\begin{figure}
	\centering
	\subfloat[Tilnærming med 3\\ \hspace{0.55cm}delintervaller.]{\includegraphics{\figp{trapmet}}}\qquad
	\subfloat[Tilnærming med 20\\ \hspace{0.55cm}delintervaller]{\includegraphics{\figp{trapmetb}}}
	\caption{\label{trapmetfig}}
\end{figure}

\reg[Trapesmetoden \label{trapmet}]{
Gitt en integrerbar funksjon $ f $. En tilnærmet verdi for $ \int_{a}^{b} f\,dx $ er da gitt som
\begin{equation}\label{trapmeteq}
	\int_{a}^{b} f\,dx \approx \frac{\Delta x}{2}\sum_{i=0}^{n}\left[f(x_i)+f(x_{i+1})\right]
\end{equation}
hvor
\algv{
n&\in \mathbb{\hat{N}}\vn
a&=x_0 \vn
b&=x_n \vn
\Delta x &= \frac{b-a}{n+1}\vn
x_{n+1}&=x_n+i\Delta x
}

}
\info{\note}{
Slik \rref{trapmet} er formulert, vil $ {[a, b]} $ være delt inn i $ {n+1} $ delintervaller.
}

\end{document}