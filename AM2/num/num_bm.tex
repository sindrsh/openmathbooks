\documentclass[english,hidelinks,pdftex, 11 pt, class=report,crop=false]{standalone}
\input{../am2_bm}
\usepackage[T1]{fontenc}
\usepackage[utf8]{luainputenc}
\usepackage{lmodern} % load a font with all the characters
\usepackage{geometry}
\geometry{verbose,a4paper, inner=0cm, outer=0 cm, bmargin=2cm, tmargin=1cm}
%\textwidth=12cm
\setlength{\parindent}{0bp}
\usepackage{import}
\usepackage[subpreambles=false]{standalone}
\usepackage{amsmath}
\usepackage{amssymb}
\usepackage{esint}
\usepackage{babel}
\usepackage{tabu}
\usepackage[dvipsnames, table]{xcolor}
\usepackage{cancel}
\makeatother
\makeatletter
\usepackage{datetime2}
\usepackage{titlesec}
\usepackage[many]{tcolorbox}

% Eheter
\newcommand{\enh}[1]{\,\textrm{#1}}
%referances
\newcommand{\net}[2]{{\color{blue}\href{#1}{#2}}}

%Spaces
\newcommand{\vsk}{\\[12pt]}
\newcommand{\vs}{\vspace{-12pt}}

% Tabell for opplegg

\newcommand{\ovlist}[1]{
\vspace{-16pt}
\begin{itemize}
	#1
\end{itemize}
}

% Chapters and sections
\titleformat{\section}[block]{\bfseries}{\hspace{3cm}\thesection}{5pt}{}
\titleformat{\subsection}[block]{\bfseries}{\hspace{3cm}\thesection}{5pt}{}
\newcommand{\sectionbreak}{\clearpage} % New page on each section
 

\newlength{\mywidth}
\setlength{\mywidth}{14cm}

\newcommand{\cont}[1]{
\begin{tcolorbox}[center, boxrule=0.0 mm, width=\mywidth,arc=0mm,enhanced jigsaw,,colback=white,breakable]
#1	
\end{tcolorbox}
}

\newcommand{\info}[5]{
\begin{tcolorbox}[center, boxrule=0.1 mm, width=\mywidth,arc=0mm,enhanced jigsaw,breakable,colback=yellow!5]	
	
	\footnotesize
	\textbf{Øvingsområde}\\[5pt] #1 
	
	\textbf{Utstyr}\\ #2  \\
	
	\begin{tabular}{@{} p{4cm} p{4cm} l} 
		\textbf{Tid} & \textbf{Elevinndeling} & \textbf{Læringsarena} \\
		#3  & #4 & #5
	\end{tabular} 
\end{tcolorbox}	
}

\newcommand{\gjen}[1]{\begin{tcolorbox}[center,boxrule=0.1 mm, width=\mywidth,arc=0mm,colback=blue!3] {\large \textbf{Gjennomføring} \vspace{5 pt}}\newline #1  \end{tcolorbox}\vspace{-5pt}}
\newcommand{\eks}[1]{\begin{tcolorbox}[center,boxrule=0.1 mm, width=\mywidth,arc=0mm,colback=green!3] {\large \textbf{Eksempel} \vspace{5 pt}}\newline #1  \end{tcolorbox}\vspace{-5pt}}

\newcounter{opl}
%\numberwithin{opl}{article}


\newcommand{\opl}[1]{
\newpage
{\refstepcounter{opl} %\phantomsection 
\large \textbf{\theopl \;#1} \vsk}
}

% Headlines
\newcommand{\fork}{\textbf{Forkunnskapar}\\}
\newcommand{\forb}{\textbf{Forberedelsar}\\}
\newcommand{\opgvr}{\textbf{Oppgaver}}



%colors
\newcommand{\colr}[1]{{\color{red} #1}}
\newcommand{\colb}[1]{{\color{blue} #1}}
\newcommand{\colo}[1]{{\color{orange} #1}}
\newcommand{\colc}[1]{{\color{cyan} #1}}
\definecolor{projectgreen}{cmyk}{100,0,100,0}
\newcommand{\colg}[1]{{\color{projectgreen} #1}}

% Lister med bokstavar
\usepackage[inline]{enumitem}
% Opg
\newcommand{\abc}[1]{
	\begin{enumerate}[label=\alph*),leftmargin=18pt]
		#1
	\end{enumerate}
}

\usepackage[]{hyperref}


\begin{document}
\section{Newtons metode}
Gitt en funskjon $ f(x) $ og likningen 
\[ f=0 \]
Ved Newtons metode følger man følgende resonnement for å finne en løsning av likningen: \regv
\st{
Gitt at $ {f(x_a)=0} $.
Vi starter med en $ x $-verdi $ x_0 $. Skjæringspunktet mellom $ x $-aksen og tangenten til $ f $ i $ x_0 $ kaller vi $ x_1 $. Vi antar at $ |x_1-x_a|<|x_0-x_a| $. 
}
Siden $ x_1 $ er skjæringspunktet mellom $ x $-aksen og tangenten til $ f $ i $ x_0 $, har vi  at\footnote{Se oppgave??}
\alg{
f'(x_0)(x_1-x_0)+f(x_0)&=0 \\
f'(x_0)x_1 &= f'(x_0)x_0-f(x_0) \\
x_1 &= x_0-\frac{f(x_0)}{f'(x_0)}
}
\begin{figure}
\subfloat[]{
\includegraphics{\figp{newt1a}}
}
\subfloat[]{
	\includegraphics{\figp{newt1b}}
}
\end{figure}
Ved å bruke tangenten til $ f $ i $ x_1 $, kan vi finne enda en ny $ x $-verdi, som vi antar gir en bedre tilnærming til $ x_a $ enn det $ x_1 $ gir. Denne prosedyren kan vi gjenta fram til vi har funnet en $ x $-verdi som gir en tilstrekkelig tilnærming til $ x_a $.
\newpage
\reg[Newtons metode (Newton-Rhapson metoden)]{
Gitt en funskjon $ f(x) $ og likningen 
\[ f=0 \]
hvor ${f(x_a)=0 }$. For et passende valg av $ x_n $ antas det da at $ {|x_{n+1}-x_a|<|x_n-x_a|} $, hvor
\nn{
x_{n+1} = x_n-\frac{f(x_n)}{f'(x_n)} 
}
}

\end{document}