\documentclass[english,hidelinks,pdftex, 11 pt, class=report,crop=false]{standalone}
\usepackage[T1]{fontenc}
\usepackage[utf8]{luainputenc}
\usepackage{lmodern} % load a font with all the characters
\usepackage{geometry}
\geometry{verbose,a4paper, inner=0cm, outer=0 cm, bmargin=2cm, tmargin=1cm}
%\textwidth=12cm
\setlength{\parindent}{0bp}
\usepackage{import}
\usepackage[subpreambles=false]{standalone}
\usepackage{amsmath}
\usepackage{amssymb}
\usepackage{esint}
\usepackage{babel}
\usepackage{tabu}
\usepackage[dvipsnames, table]{xcolor}
\usepackage{cancel}
\makeatother
\makeatletter
\usepackage{datetime2}
\usepackage{titlesec}
\usepackage[many]{tcolorbox}

% Eheter
\newcommand{\enh}[1]{\,\textrm{#1}}
%referances
\newcommand{\net}[2]{{\color{blue}\href{#1}{#2}}}

%Spaces
\newcommand{\vsk}{\\[12pt]}
\newcommand{\vs}{\vspace{-12pt}}

% Tabell for opplegg

\newcommand{\ovlist}[1]{
\vspace{-16pt}
\begin{itemize}
	#1
\end{itemize}
}

% Chapters and sections
\titleformat{\section}[block]{\bfseries}{\hspace{3cm}\thesection}{5pt}{}
\titleformat{\subsection}[block]{\bfseries}{\hspace{3cm}\thesection}{5pt}{}
\newcommand{\sectionbreak}{\clearpage} % New page on each section
 

\newlength{\mywidth}
\setlength{\mywidth}{14cm}

\newcommand{\cont}[1]{
\begin{tcolorbox}[center, boxrule=0.0 mm, width=\mywidth,arc=0mm,enhanced jigsaw,,colback=white,breakable]
#1	
\end{tcolorbox}
}

\newcommand{\info}[5]{
\begin{tcolorbox}[center, boxrule=0.1 mm, width=\mywidth,arc=0mm,enhanced jigsaw,breakable,colback=yellow!5]	
	
	\footnotesize
	\textbf{Øvingsområde}\\[5pt] #1 
	
	\textbf{Utstyr}\\ #2  \\
	
	\begin{tabular}{@{} p{4cm} p{4cm} l} 
		\textbf{Tid} & \textbf{Elevinndeling} & \textbf{Læringsarena} \\
		#3  & #4 & #5
	\end{tabular} 
\end{tcolorbox}	
}

\newcommand{\gjen}[1]{\begin{tcolorbox}[center,boxrule=0.1 mm, width=\mywidth,arc=0mm,colback=blue!3] {\large \textbf{Gjennomføring} \vspace{5 pt}}\newline #1  \end{tcolorbox}\vspace{-5pt}}
\newcommand{\eks}[1]{\begin{tcolorbox}[center,boxrule=0.1 mm, width=\mywidth,arc=0mm,colback=green!3] {\large \textbf{Eksempel} \vspace{5 pt}}\newline #1  \end{tcolorbox}\vspace{-5pt}}

\newcounter{opl}
%\numberwithin{opl}{article}


\newcommand{\opl}[1]{
\newpage
{\refstepcounter{opl} %\phantomsection 
\large \textbf{\theopl \;#1} \vsk}
}

% Headlines
\newcommand{\fork}{\textbf{Forkunnskapar}\\}
\newcommand{\forb}{\textbf{Forberedelsar}\\}
\newcommand{\opgvr}{\textbf{Oppgaver}}



%colors
\newcommand{\colr}[1]{{\color{red} #1}}
\newcommand{\colb}[1]{{\color{blue} #1}}
\newcommand{\colo}[1]{{\color{orange} #1}}
\newcommand{\colc}[1]{{\color{cyan} #1}}
\definecolor{projectgreen}{cmyk}{100,0,100,0}
\newcommand{\colg}[1]{{\color{projectgreen} #1}}

% Lister med bokstavar
\usepackage[inline]{enumitem}
% Opg
\newcommand{\abc}[1]{
	\begin{enumerate}[label=\alph*),leftmargin=18pt]
		#1
	\end{enumerate}
}

\usepackage[]{hyperref}

\newcommand{\note}{Merk}
\newcommand{\notesm}[1]{{\footnotesize \textsl{\note:} #1}}
\newcommand{\ekstitle}{Eksempel }
\newcommand{\sprtitle}{Språkboksen}
\newcommand{\expl}{forklaring}
\newcommand{\pyt}{Pytagoras' setning}
\newcommand\sv{\vsk \textbf{Svar} \vspace{4 pt}\\}

%references
\newcommand{\reftab}[1]{\hrs{#1}{tabell}}
\newcommand{\rref}[1]{\hrs{#1}{regel}}
\newcommand{\dref}[1]{\hrs{#1}{definisjon}}
\newcommand{\refkap}[1]{\hrs{#1}{kapittel}}
\newcommand{\refsec}[1]{\hrs{#1}{seksjon}}
\newcommand{\refdsec}[1]{\hrs{#1}{delseksjon}}
\newcommand{\refved}[1]{\hrs{#1}{vedlegg}}
\newcommand{\eksref}[1]{\textsl{#1}}
\newcommand\fref[2][]{\hyperref[#2]{\textsl{figur \ref*{#2}#1}}}
\newcommand{\refop}[1]{{\color{blue}Oppgave \ref{#1}}}
\newcommand{\refops}[1]{{\color{blue}oppgave \ref{#1}}}


%Algebra
\newcommand{\kvadset}{Kvadratsetningene}
\newcommand{\aenato}{Sum-produkt-metoden}

% Geometry
\newcommand{\hlikb}{Midtnormalen i en likebeint trekant}
\newcommand{\arealsetn}{Arealsetningen}
\newcommand{\trkmedian}{Median}
\newcommand{\midtrk}{Midtnormal (i trekant)}
\newcommand{\innskrsirk}{Innskrevet sirkel}
\newcommand{\cossetn}{Cosinussetningen}
\newcommand{\perfvink}{Sentral- og periferivinkel}
\newcommand{\tang}{Tangent}

% Derivative
\newcommand{\derel}{Den deriverte av elementære funksjoner}
\newcommand{\divder}{Divisjonsregelen}
\newcommand{\kjernereg}{Kjerneregelen}
\newcommand{\prodregder}{Produktregelen}
\newcommand{\lhop}{L'Hopitals regel}

% Funksjonsdrofting
\newcommand{\monder}{Monotoniegenskaper og den deriverte}
\newcommand{\fderekstr}{$ \bm{f'=0} $ for lokale ektstremalpunkt}
\newcommand{\andredertest}{Andrederiverttesten}

% Vectors
\newcommand{\detar}{Arealformler med determinanter}
\newcommand{\avstpunktlin}{Avstand mellom punkt og linje}

%Appendix
\newcommand{\rolle}{Rolles teorem}
\newcommand{\meanval}{Middelverdisetningen}

% Solutions manual
\newcommand{\selos}{Se løsningsforslag.}

\begin{document}
\section{GeoGebra}	
\subsection{CAS}	
\subsubsection{Definere variabler \label{defvar}}
Hvis vi ønsker å definere variabler som vi skal bruke i andre celler, må vi skrive \texttt{:=} . I figuren under er forskjellen mellom \texttt{=} og \texttt{:=} demonstrert med et forsøk på å finne $ f'(x) $ til funksjonen $ f(x)=x^2 $:
\begin{figure}
	\centering
	\includegraphics[scale=0.5]{fig/fder}
\end{figure}
Av figuren legger vi også merke til at celle 3 og celle 4 er markert med en hvit runding. Dette indikerer at størrelsen vil vises i \textsl{Grafikkfelt} eller \textsl{Grafikkfelt 3D} hvis man trykker på markøren (den skal da bli blå).
\subsubsection{Celle-referanser}

Ofte kommer vi ut for situasjoner der vi ønsker å bruke uttrykket vi har funnet i tidligere celler. Som eksempel har vi i celle 1 skrevet inn volumet $ v $ av en kule med radius $ r $, mens i celle 2 har vi volumet $ V $ av en kule med radius $ R $. Ønsker vi å finne forholdet mellom disse, kan vi bruke cellereferanser som hjelpemiddel. For å referere til celle 1 skriver vi \texttt{\$1} og for celle 2 skriver vi \texttt{\$2}. Forholdet $ \frac{v}{V} $ kan vi da skriver som \texttt{\$1/\$2}:
\begin{figure}
	\centering
	\includegraphics[scale=0.5]{fig/ref}
\end{figure}

\subsubsection{Lister}
Når et uttrykk står inni sløyfeparanteser \{\}, betyr det at det er laget en liste. En liste inneholder flere elementer som vi kan hente ut. Dette gjør vi ved å skrive paranteser bak listen, hvor vi angir nummeret til elementet i listen.
\begin{figure}
	\centering
	\includegraphics[scale=0.5]{fig/liste}
\end{figure}

Lister bruker vi også når vi skal løse liginger med flere ukjente:
\begin{figure}
	\centering
	\includegraphics[scale=0.5]{fig/lign}
\end{figure}
\subsubsection{Høyre- og venstresiden \label{hogl}}
De fleste uttrykkene vi jobber med i CAS inneholder et $ = $ tegn. Disse uttrykkene er en ligning med en venstre- og høyreside. Ofte ønsker vi å bruke uttrykket på bare én av disse sidene, og oftest høyresiden. Som eksempel har vi løst ligningen $ (a+b)x = c $ og definert funksjonen $ f(x)= d x^2 $. Vi ønsker så å sette løsningen av ligningen inn i funksjonen. Dette gjør vi ved hjelp av \texttt{HøyreSide}-kommandoen (resulatet uten bruken av denne er vist i celle 4).
\begin{figure}
	\centering
	\includegraphics[scale=0.5]{fig/hoyre}
\end{figure}
\subsubsection{ByttUt}
Noen ganger ønsker vi å endre en variabel i et utrykk. For å gjøre dette kan vi anvende \cms{ByttUt}{<Uttrykk>, <Liste med forandringer>}. La oss se på uttrykket
\[ \frac{a+b}{c} \]
Vi ønsker nå å sette $ {a=d }$, $ {b=2} $ og $ {c=f} $. Dette kan vi gjør ved å skrive følgende:
\begin{figure}
	\centering
	\includegraphics[scale=0.5]{fig/cas2}
\end{figure}	
\newpage
\subsection{Knapper og kommandoer}
\subsubsection*{Grafikkfelt}
Knappene velges fra rullemenyer på verktøylinjen. Nummereringen av menyene er fra venstre.\vsk

\begin{tabular}{@{}l}
	\,\includegraphics[scale=0.4]{fig/pkt} Lager et nytt punkt. (Meny nr. 1) \\
	\,\includegraphics[scale=0.4]{fig/lin} Lager linje mellom to punkt. (Meny nr. 2)\\	
	\,\includegraphics[scale=0.4]{fig/ekst} Finner topp- og bunnpunkt til en funksjon. (Meny nr. 2)\\
	\,\includegraphics[scale=0.4]{fig/nul} Finner nullpunktene til en funksjon. (Meny nr. 2)	\\
	\,\includegraphics[scale=0.4]{fig/skj} Finner skjæringspunkt mellom to objekt. (Meny nr. 3)\\	
	\,\includegraphics[scale=0.4]{fig/vek} Lager vektoren mellom to punkt (Meny nr. 3)\\		
	\,\includegraphics[scale=0.4]{fig/tekst} Lager en tekstboks. (Meny nr. 10)\\		
	\,\includegraphics[scale=0.4]{fig/flytt} Flytter grafikkfeltet. Endrer verdiavstanden hvis man peker på aksene. \\
	\hspace{1cm}(Meny nr. 10)\\			
\end{tabular}
\subsubsection*{CAS}
\begin{tabular}{@{}l}
	\;\includegraphics[scale=0.4]{fig/erlik} Gjengir uttrykket som er inntastet, ofte i forkortet form.\\	
	\;\includegraphics[scale=0.4]{fig/brin} Gjengir uttrykket som er inntastet.\\	
	\;\includegraphics[scale=0.4]{fig/caerlik} Gir tilnærmet verdi av et uttrykk (som desimaltall). \\	
	\;\includegraphics[scale=0.4]{fig/x} Gir eksaktløsningen av en ligning.\\
	\;\includegraphics[scale=0.4]{fig/xca} Gir tilnærmet løsning av en ligning som desimaltall.\\
	
\end{tabular}

\subsubsection{Hurtigtaster}
\begin{tabular}{@{}c | c |c | c }
	&\textbf{Beskrivelse} & \textbf{PC }& \textbf{Mac} \\ \hline
	$ \sqrt{} $	& kvadratrot& \texttt{alt\,+\,r} &\texttt{alt\,+\,r} \\\hline
	$ \pi $	& pi& \texttt{alt\,+\,p} & \texttt{alt\,+\,p}\\\hline
	$ \infty $ &uendelig& \texttt{alt\,+\,u} &\texttt{alt\,+\,,}  \\\hline
	$ \otimes $&kryssprodukt & \texttt{alt\,+\,shift\,+\,8}&\texttt{ctrl\,+\,shift\,+\,8} \\\hline
	$ e $&eulers tall & \texttt{alt\,+\,e}& \texttt{alt\,+\,e}\\\hline
	$ {}^\circ $&gradtegnet ($ \frac{\pi}{180} $) & \texttt{alt\,+\,o}& \texttt{alt\,+\,o}
	\\\hline	
\end{tabular}
\newpage

\subsubsection{Kommandoliste}
\cm{abs}{<x>}
{Finner lengden til et objekt \textit{x}.
}

\cm{Asymptote}{<Funksjon>}
{Finner asymptotene til en funksjon.}

\cm{Avstand}{<Punkt>, <Objekt>}
{Gir avstanden fra et punkt til et objekt.}

\cmc{ByttUt}{<Uttrykk>, <Liste med forandringer>}
{Viser et gitt uttrykk etter endring av variabler, gitt i en liste.}

\cm{Deriverte}{<Funksjon>}	
{Gir den deriverte av en funksjon.
	
	\mers{
		For en definert funksjon $ f(x) $, kan man like gjerne skrive \texttt{f'(x)}.
}}

\cm{Ekstremalpunkt}{<Funksjon>, <Start>, <Slutt> } {Finner lokale ekstremalpunkt og ekstremalverdier for en funksjon $ f $ på et gitt intervall.}

\cm{Ekstremalpunkt}{Polynom}
{Finner lokale ekstremalpunkt og ekstremalverdier til et polynom.}

\cm{Funksjon}{<Funksjon>, <Start>, <Slutt>}
{Tegner en funksjon på et gitt intervall.}

\cm{Høyde}{<Objekt>}
{Gir avstanden fra toppunkt til grunnflate i et objekt. 
	\mers{Avstanden har retning, og derfor kan den noen ganger være negativ. Tallverdien er den geometriske høyden.}}

\cmc{HøyreSide}{<Likning>}
{Gir høyresiden til en likning.}

\cmc{HøyreSide}{<Liste med likninger>}
{Gir en liste med høyresidene i en liste med ligninger.}

\cm{Integral}{<Funksjon>}
{Gir uttrykket til det ubestemte integralet av en funksjon. (Merk: Hvis kommandoen skrives i inntastingsfeltet, blir konstantleddet utelatt).}

\cm{Integral}{<Funksjon>, <Start>, <Slutt>}
{Gir det bestemte integralet av en funksjon på et intervall.}

\cmc{Integral}{<Variabel>}
{Gir uttrykket til det ubestemte integralet til en funksjon av gitt variabel. (Brukes dersom man ønsker å integrere funksjoner avhengig av en annen variabel enn $ x $).}

\cm{Kule}{<Punkt>, <Radius>}
{Viser en kule i Grafikkfelt 3D med sentrum i et gitt punkt og med en gitt radius.}

\cm{Kurve}{<Uttrykk>, <Uttrykk>, <Uttrykk>, <Parametervariabel>, <Start>, <Slutt>}
{Viser parameteriseringen av en kurve i Grafikkfelt 3D på et gitt intervall. Uttrykkene er henholdsvis uttrykkene for $ {x, y} $ og $ z $-koordinatene, bestemt av en gitt parametervariabel.
	
\mers{
Med mindre et bestemt intervall av kurven er ønsket, er det bedre å skrive parameteriseringen direkte inn i inntastingsfeltet som \texttt{A+t*u}, hvor $ A $ er et punkt på linja og $ u$ er en retningsvektor.
}}

\cm{Linje}{<Punkt>, <Punkt>}{ Gir uttrykket til en linje mellom to punkt. Hvis punktene har tre koordinater besår uttrykket av et punkt på linja og en fri variabel $ \}lambda $ mulitplisert med en retningsvektor.}

\cmc{Løs}{<Likning med x>}{
Løser en likning med $ x $ som ukjent.
}

\cmc{Løs}{<Liste med likninger>, <Liste med variabler>}
{Finner alle løsninger av en liste med ligninger med gitte variabel som ukjente.}

\cmc{Løs}{<Likning>, <Variabel>}
{Finner alle løsninger av en gitt likning med en gitt variabel som ukjent.}

\cm{Maks}{<Funksjon>, <Start x-verdi>, <Slutt x-verdi>} {Finner absolutt maksimum og maskimalpunkt for en funksjon $ f $ på et gitt intervall.}
\newpage
\cm{Min}{<Funksjon>, <Start x-verdi>, <Slutt x-verdi>} {Finner absolutt minimum og minimumspunkt for en funksjon $ f $ på et gitt intervall.}
\newpage
\cm{Nullpunkt}{<Polynom> }
{Finner alle nullpunkter til et polynom.}

\cm{NullpunktIntervall}{<Funksjon>, <Start>, <Slutt>}
{Finner alle nullpunkter på et gitt intervall til en hvilken som helst funksjon.}

\cm{Plan}{<Punkt>, <Punkt>, <Punkt>}
{Viser et plan i Grafikkfelt 3D, utspent av to av vektorene mellom tre gitte punkt.}

\cm{Prisme}{<Punkt>, <Punkt>, ...}
{Framstiller en prisme i Grafikkfelt 3D. {\tt Prisme[A,B,C,D]} lager en prisme med grunnflate $ ABC $ og tak $ DEF $, {\tt Prisme[A,B,C,D,E]} har grunnflate $ ABCD $ og tak $ EFG $. $ F, G$ og eventelt $ E $ blir konstruert av GeoGebra slik at hver sideflate er et parallellogram. Under kategorien \textsl{Prisme} i algebrafaltet finner man en konstant som oppgir volumet til pyramiden.}

\cm{Punkt}{<Liste>} {Lager et punkt med koordinater gitt som liste. \\
	
	\mers{For å lage punktet $ (x, y) $, kan man liksågodt skrive \texttt{(x,y)} i inntastingsfeltet. Skriver man \texttt{(x,y)} i CAS lager man vektoren $ [x, y] $.}
}

\cm{Pyramide}{<Punkt>, <Punkt>, ...}
{Framstiller en pyramide i Grafikkfelt 3D. {\tt Pyramide[A,B,C,D]} lager en pyramide med grunnflate ${ A, B, C} $ og toppunkt $ D $, mens {\tt Pyramide[A,B,C,D, E]} har grunnflate ${ A, B, C, D }$ og toppunkt $ E $. Under kategorien \textsl{Pyramide} i algebrafaltet finner man en konstant som oppgir volumet til pyramiden.}

\cm{RegLin}{<Liste>}
{Bruker regresjon med en rett linje for å tilpasse punkt gitt i en liste.}

\cm{RegEksp}{<Liste>}
{Bruker regresjon med en eksponentialfunksjon for å tilpasse punkt gitt i en liste.}

\cm{RegPoly}{<Liste>, <Grad>}
{Bruker regresjon med et polynom av gitt grad for å tilpasse punkt gitt i en liste.}

\cm{RegPot}{<Liste>}
{Bruker regresjon med en potensfunksjon for å tilpasse punkt gitt i en liste.}

\cm{RegSin}{<Liste>}
{Bruker regresjon med en sinusfunksjon for å tilpasse punkt gitt i en liste.	}

\cm{Skalarprodukt}{<Vektor>, <Vektor>} 
{Finner skalarproduktet av to vektorer. 
	
	\mers{For to vektorer $u$ og $v$ kan man like gjerne skrive \texttt{u*v}.}}

\cm{Skjæring}{<Objekt>, <Objekt>}
{Finner skjæringspunktene mellom to objekter. 
}

\cm{Skjæring}{<Funksjon>, <Funksjon>, <Start>, <Slutt>}
{Finner skjæringspunktene mellom to funksjoner på et gitt intervall.}

\cmc{Sum}{<Uttrykk>, <Variabel>, <Start>, <Slutt>} {
Finner summen av en rekke med en løpende variabel på et intervall.
}

\cm{TrigKombiner}{<Funksjon>}
{Skriver om et uttrykk på formen $ a\sin (kx) + b\cos (kx) $ til et kombinert uttrykk på formen $ r\cos (kx-c) $.}

\cm{TrigKombiner}{<Funksjon>, sin(x)}
{Skriver om en funksjon på formen $ {a\sin (kx) + b\cos (kx) }$ til et kombinert uttrykk på formen $ r\sin (kx+c) $.}


\cm{Vektor}{<Punkt>} {
Lager vektoren fra origo til et gitt punkt. \\
\mers{I CAS kan man lage vektoren $ {[x, y]} $ ved å skrive \texttt{(x,y)}, dette anbefales.}
}

\cmc{Vektorprodukt}{<Vektor>, <Vektor>}
{Finner vektorproduktet av to vektorer. \\
\mers{Merk: For to vektorer $u$ og $v$ kan man like gjerne skrive \texttt{u}$ \otimes $\texttt{v}. }
}

\cm{Vendepunkt}{<Polynom>}
{Finner vendepunktene til et polynom.}

\cmc{VenstreSide}{<Likning>}
{Gir venstresiden til en likning.}

\cmc{VenstreSide}{<Liste med likninger>}
{Gir en liste med venstresidene i en liste med ligninger.}

\cm{Vinkel}{<Vektor>, <Vektor>}
{Gir vinkelen mellom to vektorer. Kan også brukes for vinkel mellom plan/linjer, plan/plan og linje/linje}
\end{document}






\regsin

\retn

\skalar



\summ



\trikomb

\vektor

\vekpro

\vend

