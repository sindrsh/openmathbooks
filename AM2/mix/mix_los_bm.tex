\documentclass[english,hidelinks,pdftex, 11 pt, class=report,crop=false]{standalone}
\usepackage[T1]{fontenc}
\usepackage[utf8]{luainputenc}
\usepackage{lmodern} % load a font with all the characters
\usepackage{geometry}
\geometry{verbose,a4paper, inner=0cm, outer=0 cm, bmargin=2cm, tmargin=1cm}
%\textwidth=12cm
\setlength{\parindent}{0bp}
\usepackage{import}
\usepackage[subpreambles=false]{standalone}
\usepackage{amsmath}
\usepackage{amssymb}
\usepackage{esint}
\usepackage{babel}
\usepackage{tabu}
\usepackage[dvipsnames, table]{xcolor}
\usepackage{cancel}
\makeatother
\makeatletter
\usepackage{datetime2}
\usepackage{titlesec}
\usepackage[many]{tcolorbox}

% Eheter
\newcommand{\enh}[1]{\,\textrm{#1}}
%referances
\newcommand{\net}[2]{{\color{blue}\href{#1}{#2}}}

%Spaces
\newcommand{\vsk}{\\[12pt]}
\newcommand{\vs}{\vspace{-12pt}}

% Tabell for opplegg

\newcommand{\ovlist}[1]{
\vspace{-16pt}
\begin{itemize}
	#1
\end{itemize}
}

% Chapters and sections
\titleformat{\section}[block]{\bfseries}{\hspace{3cm}\thesection}{5pt}{}
\titleformat{\subsection}[block]{\bfseries}{\hspace{3cm}\thesection}{5pt}{}
\newcommand{\sectionbreak}{\clearpage} % New page on each section
 

\newlength{\mywidth}
\setlength{\mywidth}{14cm}

\newcommand{\cont}[1]{
\begin{tcolorbox}[center, boxrule=0.0 mm, width=\mywidth,arc=0mm,enhanced jigsaw,,colback=white,breakable]
#1	
\end{tcolorbox}
}

\newcommand{\info}[5]{
\begin{tcolorbox}[center, boxrule=0.1 mm, width=\mywidth,arc=0mm,enhanced jigsaw,breakable,colback=yellow!5]	
	
	\footnotesize
	\textbf{Øvingsområde}\\[5pt] #1 
	
	\textbf{Utstyr}\\ #2  \\
	
	\begin{tabular}{@{} p{4cm} p{4cm} l} 
		\textbf{Tid} & \textbf{Elevinndeling} & \textbf{Læringsarena} \\
		#3  & #4 & #5
	\end{tabular} 
\end{tcolorbox}	
}

\newcommand{\gjen}[1]{\begin{tcolorbox}[center,boxrule=0.1 mm, width=\mywidth,arc=0mm,colback=blue!3] {\large \textbf{Gjennomføring} \vspace{5 pt}}\newline #1  \end{tcolorbox}\vspace{-5pt}}
\newcommand{\eks}[1]{\begin{tcolorbox}[center,boxrule=0.1 mm, width=\mywidth,arc=0mm,colback=green!3] {\large \textbf{Eksempel} \vspace{5 pt}}\newline #1  \end{tcolorbox}\vspace{-5pt}}

\newcounter{opl}
%\numberwithin{opl}{article}


\newcommand{\opl}[1]{
\newpage
{\refstepcounter{opl} %\phantomsection 
\large \textbf{\theopl \;#1} \vsk}
}

% Headlines
\newcommand{\fork}{\textbf{Forkunnskapar}\\}
\newcommand{\forb}{\textbf{Forberedelsar}\\}
\newcommand{\opgvr}{\textbf{Oppgaver}}



%colors
\newcommand{\colr}[1]{{\color{red} #1}}
\newcommand{\colb}[1]{{\color{blue} #1}}
\newcommand{\colo}[1]{{\color{orange} #1}}
\newcommand{\colc}[1]{{\color{cyan} #1}}
\definecolor{projectgreen}{cmyk}{100,0,100,0}
\newcommand{\colg}[1]{{\color{projectgreen} #1}}

% Lister med bokstavar
\usepackage[inline]{enumitem}
% Opg
\newcommand{\abc}[1]{
	\begin{enumerate}[label=\alph*),leftmargin=18pt]
		#1
	\end{enumerate}
}

\usepackage[]{hyperref}

\newcommand{\note}{Merk}
\newcommand{\notesm}[1]{{\footnotesize \textsl{\note:} #1}}
\newcommand{\ekstitle}{Eksempel }
\newcommand{\sprtitle}{Språkboksen}
\newcommand{\expl}{forklaring}
\newcommand{\pyt}{Pytagoras' setning}
\newcommand\sv{\vsk \textbf{Svar} \vspace{4 pt}\\}

%references
\newcommand{\reftab}[1]{\hrs{#1}{tabell}}
\newcommand{\rref}[1]{\hrs{#1}{regel}}
\newcommand{\dref}[1]{\hrs{#1}{definisjon}}
\newcommand{\refkap}[1]{\hrs{#1}{kapittel}}
\newcommand{\refsec}[1]{\hrs{#1}{seksjon}}
\newcommand{\refdsec}[1]{\hrs{#1}{delseksjon}}
\newcommand{\refved}[1]{\hrs{#1}{vedlegg}}
\newcommand{\eksref}[1]{\textsl{#1}}
\newcommand\fref[2][]{\hyperref[#2]{\textsl{figur \ref*{#2}#1}}}
\newcommand{\refop}[1]{{\color{blue}Oppgave \ref{#1}}}
\newcommand{\refops}[1]{{\color{blue}oppgave \ref{#1}}}


%Algebra
\newcommand{\kvadset}{Kvadratsetningene}
\newcommand{\aenato}{Sum-produkt-metoden}

% Geometry
\newcommand{\hlikb}{Midtnormalen i en likebeint trekant}
\newcommand{\arealsetn}{Arealsetningen}
\newcommand{\trkmedian}{Median}
\newcommand{\midtrk}{Midtnormal (i trekant)}
\newcommand{\innskrsirk}{Innskrevet sirkel}
\newcommand{\cossetn}{Cosinussetningen}
\newcommand{\perfvink}{Sentral- og periferivinkel}
\newcommand{\tang}{Tangent}

% Derivative
\newcommand{\derel}{Den deriverte av elementære funksjoner}
\newcommand{\divder}{Divisjonsregelen}
\newcommand{\kjernereg}{Kjerneregelen}
\newcommand{\prodregder}{Produktregelen}
\newcommand{\lhop}{L'Hopitals regel}

% Funksjonsdrofting
\newcommand{\monder}{Monotoniegenskaper og den deriverte}
\newcommand{\fderekstr}{$ \bm{f'=0} $ for lokale ektstremalpunkt}
\newcommand{\andredertest}{Andrederiverttesten}

% Vectors
\newcommand{\detar}{Arealformler med determinanter}
\newcommand{\avstpunktlin}{Avstand mellom punkt og linje}

%Appendix
\newcommand{\rolle}{Rolles teorem}
\newcommand{\meanval}{Middelverdisetningen}

% Solutions manual
\newcommand{\selos}{Se løsningsforslag.}

\usepackage{xr}
\externaldocument{../../TM2/TM2_bm}
\externaldocument{../AM2_bm}
\begin{document}	
	
\opr{opganubevis}	
\abc{
\item Vi har at
\begin{equation}\label{restbelop}
	\text{resterende lånebeløp}=\text{forrige lånebeløp}-\text{avdrag} 
\end{equation}
Ved et annuitetslån er terminbeløpet likt ved hver nedbetaling av lånet. Dette betyr at
\begin{align}
	\text{terminbeløp}&=\text{avdrag}+\text{rentebeløp} \nonumber \\ 
	\text{avdrag}&=\text{terminbeløp}-\text{rentebeløp} \label{avdrag}
\end{align}
Av \eqref{restbelop} og \eqref{avdrag} er
\alg{
\text{resterende lånebeløp}&=\text{forrige lånebeløp}-(\text{terminbeløp}-\text{rentebeløp}) \\
&= \text{forrige lånebeløp}+\text{rentebeløp}-\text{terminbeløp}
}
Da $ \text{rentebeløp}=\text{forrige lånebeløp}\cdot r $, har vi at
\[ \text{resterende lånebeløp}=(1+r)\cdot\text{forrige lånebeløp}-\text{terminbeløp} \]
Med størrelsene slik de er definert i oppgaven kan dette skrives som
\[ L_n=(1+r)L_{n-1}-T \]
\item Vi har at
\alg{
	L_1&=(1+r)L_0 - T\vn
	L_2&= (1+r)L_1 - T\\
	&= (1+r)\left[(1+r)L_0 - T\right]\\
	&=(1+r)^2L_0+(1+r)T-T \vn
	L_3 &= (1+r)L_2 - T \\
	&= (1+r)\left[(1+r)^2L_0-(1+r)T-T\right] \\
	&=(1+r)^3L_0-(1+r)^2T-(1+r)T-T \\
}
Følgelig er
\[ L_n=(1+r)^nL_0-T-(1+r)T-(1+r)^2-...-(1+r)^{n-1}T \]
}
Etter $ t $ år er $ L_t=0 $, og dermed er
\[T+(1+r)T+(1+r)^2+...+(1+r)^{t-1}T =(1+r)^t \tag{*}\]
Venstresiden i likningen over er summen av en geometrisk rekke med første ledd $ T $ og kvotient $ 1+r $, og av \eqref{sumg} i \tmto\ har vi at
\alg{
T\frac{1-(1+r)^t}{-r} &= (1+r)^tL_0 \\
T&= \frac{-r}{1-(1+r)^t}(1+r)^tL_0 
}
Da $ (1+r)^t=\frac{1}{(1+r)^{-t}} $, kan vi skrive
\alg{
T&=\frac{-r}{1-(1+r)^t}\cdot\frac{1}{(1+r)^{-t}}L_0\br
&=\frac{-r}{(1+r)^{-t}-1}L_0 \br
&= \frac{r}{1-(1+r)^{-t}}L_0
}

	
\opr{opggeospar}
\abc{
\item Når du har spart i 5 måneder betyr det at første innskudd har forrentet seg 4 ganger, andre beløp 3 ganger osv. Forrentingen tilsvarer en økning med $ 1.02 $. Medregnet det ferske innskuddet blir regnestykket
\[ 1000\cdot1.02^4+1000\cdot1.02^3+1000\cdot1.02^2+1000\cdot1.02^1+1000 \]
\item Av oppgave a) innser vi at $ P(n) $ er summen av en geometrisk rekke med $ a_1 = 1000 $ og $ k=1.02 $:
\alg{
	P(n)&= 1000\cdot\frac{1-1.02^n}{1-1.02} \\
	&= -50000(1-1.02^n) \\
	&= 50000(1.02^n-1)
}
}


\opr{opgfloogfjere}
\abc{
\item En cosinusfunksjon har lavest verdi når cosinusuttrykket har verdien $ -1 $. Den laveste verdien til $ f $ er derfor $ {128-80=48} $. Når det er fjære er altså vannstanden 48\,cm over sjåkartnull.\vsk

\item Av \eqref{ker2pioverP} i \tmto\ har vi at perioden $ P $ er gitt som
\[ P=\frac{2\pi}{k} \]
I dette tilfellet er $ k=\frac{3\pi}{37} $, altså er
\alg{
	P&=\frac{37}{3}\\
	&=12+\frac{1}{3}
}
Følgelig er det 12 timer og 20 minutter mellom to etterfølgende tipspunkt for flo. Dette betyr at det er 6 timer og 10 minutter mellom flo og fjære.	
}
\end{document}

