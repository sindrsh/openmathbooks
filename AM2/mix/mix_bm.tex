\documentclass[english,hidelinks,pdftex, 11 pt, class=report,crop=false]{standalone}
\usepackage[T1]{fontenc}
\usepackage[utf8]{luainputenc}
\usepackage{lmodern} % load a font with all the characters
\usepackage{geometry}
\geometry{verbose,a4paper, inner=0cm, outer=0 cm, bmargin=2cm, tmargin=1cm}
%\textwidth=12cm
\setlength{\parindent}{0bp}
\usepackage{import}
\usepackage[subpreambles=false]{standalone}
\usepackage{amsmath}
\usepackage{amssymb}
\usepackage{esint}
\usepackage{babel}
\usepackage{tabu}
\usepackage[dvipsnames, table]{xcolor}
\usepackage{cancel}
\makeatother
\makeatletter
\usepackage{datetime2}
\usepackage{titlesec}
\usepackage[many]{tcolorbox}

% Eheter
\newcommand{\enh}[1]{\,\textrm{#1}}
%referances
\newcommand{\net}[2]{{\color{blue}\href{#1}{#2}}}

%Spaces
\newcommand{\vsk}{\\[12pt]}
\newcommand{\vs}{\vspace{-12pt}}

% Tabell for opplegg

\newcommand{\ovlist}[1]{
\vspace{-16pt}
\begin{itemize}
	#1
\end{itemize}
}

% Chapters and sections
\titleformat{\section}[block]{\bfseries}{\hspace{3cm}\thesection}{5pt}{}
\titleformat{\subsection}[block]{\bfseries}{\hspace{3cm}\thesection}{5pt}{}
\newcommand{\sectionbreak}{\clearpage} % New page on each section
 

\newlength{\mywidth}
\setlength{\mywidth}{14cm}

\newcommand{\cont}[1]{
\begin{tcolorbox}[center, boxrule=0.0 mm, width=\mywidth,arc=0mm,enhanced jigsaw,,colback=white,breakable]
#1	
\end{tcolorbox}
}

\newcommand{\info}[5]{
\begin{tcolorbox}[center, boxrule=0.1 mm, width=\mywidth,arc=0mm,enhanced jigsaw,breakable,colback=yellow!5]	
	
	\footnotesize
	\textbf{Øvingsområde}\\[5pt] #1 
	
	\textbf{Utstyr}\\ #2  \\
	
	\begin{tabular}{@{} p{4cm} p{4cm} l} 
		\textbf{Tid} & \textbf{Elevinndeling} & \textbf{Læringsarena} \\
		#3  & #4 & #5
	\end{tabular} 
\end{tcolorbox}	
}

\newcommand{\gjen}[1]{\begin{tcolorbox}[center,boxrule=0.1 mm, width=\mywidth,arc=0mm,colback=blue!3] {\large \textbf{Gjennomføring} \vspace{5 pt}}\newline #1  \end{tcolorbox}\vspace{-5pt}}
\newcommand{\eks}[1]{\begin{tcolorbox}[center,boxrule=0.1 mm, width=\mywidth,arc=0mm,colback=green!3] {\large \textbf{Eksempel} \vspace{5 pt}}\newline #1  \end{tcolorbox}\vspace{-5pt}}

\newcounter{opl}
%\numberwithin{opl}{article}


\newcommand{\opl}[1]{
\newpage
{\refstepcounter{opl} %\phantomsection 
\large \textbf{\theopl \;#1} \vsk}
}

% Headlines
\newcommand{\fork}{\textbf{Forkunnskapar}\\}
\newcommand{\forb}{\textbf{Forberedelsar}\\}
\newcommand{\opgvr}{\textbf{Oppgaver}}



%colors
\newcommand{\colr}[1]{{\color{red} #1}}
\newcommand{\colb}[1]{{\color{blue} #1}}
\newcommand{\colo}[1]{{\color{orange} #1}}
\newcommand{\colc}[1]{{\color{cyan} #1}}
\definecolor{projectgreen}{cmyk}{100,0,100,0}
\newcommand{\colg}[1]{{\color{projectgreen} #1}}

% Lister med bokstavar
\usepackage[inline]{enumitem}
% Opg
\newcommand{\abc}[1]{
	\begin{enumerate}[label=\alph*),leftmargin=18pt]
		#1
	\end{enumerate}
}

\usepackage[]{hyperref}

\newcommand{\note}{Merk}
\newcommand{\notesm}[1]{{\footnotesize \textsl{\note:} #1}}
\newcommand{\ekstitle}{Eksempel }
\newcommand{\sprtitle}{Språkboksen}
\newcommand{\expl}{forklaring}
\newcommand{\pyt}{Pytagoras' setning}
\newcommand\sv{\vsk \textbf{Svar} \vspace{4 pt}\\}

%references
\newcommand{\reftab}[1]{\hrs{#1}{tabell}}
\newcommand{\rref}[1]{\hrs{#1}{regel}}
\newcommand{\dref}[1]{\hrs{#1}{definisjon}}
\newcommand{\refkap}[1]{\hrs{#1}{kapittel}}
\newcommand{\refsec}[1]{\hrs{#1}{seksjon}}
\newcommand{\refdsec}[1]{\hrs{#1}{delseksjon}}
\newcommand{\refved}[1]{\hrs{#1}{vedlegg}}
\newcommand{\eksref}[1]{\textsl{#1}}
\newcommand\fref[2][]{\hyperref[#2]{\textsl{figur \ref*{#2}#1}}}
\newcommand{\refop}[1]{{\color{blue}Oppgave \ref{#1}}}
\newcommand{\refops}[1]{{\color{blue}oppgave \ref{#1}}}


%Algebra
\newcommand{\kvadset}{Kvadratsetningene}
\newcommand{\aenato}{Sum-produkt-metoden}

% Geometry
\newcommand{\hlikb}{Midtnormalen i en likebeint trekant}
\newcommand{\arealsetn}{Arealsetningen}
\newcommand{\trkmedian}{Median}
\newcommand{\midtrk}{Midtnormal (i trekant)}
\newcommand{\innskrsirk}{Innskrevet sirkel}
\newcommand{\cossetn}{Cosinussetningen}
\newcommand{\perfvink}{Sentral- og periferivinkel}
\newcommand{\tang}{Tangent}

% Derivative
\newcommand{\derel}{Den deriverte av elementære funksjoner}
\newcommand{\divder}{Divisjonsregelen}
\newcommand{\kjernereg}{Kjerneregelen}
\newcommand{\prodregder}{Produktregelen}
\newcommand{\lhop}{L'Hopitals regel}

% Funksjonsdrofting
\newcommand{\monder}{Monotoniegenskaper og den deriverte}
\newcommand{\fderekstr}{$ \bm{f'=0} $ for lokale ektstremalpunkt}
\newcommand{\andredertest}{Andrederiverttesten}

% Vectors
\newcommand{\detar}{Arealformler med determinanter}
\newcommand{\avstpunktlin}{Avstand mellom punkt og linje}

%Appendix
\newcommand{\rolle}{Rolles teorem}
\newcommand{\meanval}{Middelverdisetningen}

% Solutions manual
\newcommand{\selos}{Se løsningsforslag.}

\begin{document}	
\section{Oppgaver med tall og situasjoner fra virkelig-heten}	
\st{Se også oppgaver på \net{https://ektedata.uib.no/}{ekte.data.uib.no}}
\linje

\op{opggeospar}
\tagop{
\#rekker \#øknomi 
}
Du ønsker å spare penger i en bank som gir 2\,\% månedlig rente. Du sparer ved å gjøre et innskudd på 1000\enh{kr} hver måned.
\abc{
\item Skriv rekken som viser hvor mye penger du har i banken etter 5 måneder med sparing. Innskuddet i 5. måned skal tas med.
\item Sett opp et uttrykk $ P(n) $ som viser hvor mye penger du har i banken $ n $ måneder etter at sparingen startet. Innskuddet i \textit{n}-te måned skal tas med.
}
\newpage
\op{opganu}
\tagop{
\#rekker \#øknonomi \# programmering
}
Si at du låner $ 1\,500\,000 $ kroner av en bank. Lånet er et annuitetslån (se \am) med 3\% årlig rente, og lånet skal betales ned i løpet av 20 år med årlige fradrag og renter. For å beregne terminbeløpet $ x $ kan man tenke som følger:\os
\st{
	Tenk deg at din bank sparer penger i en annen bank, som tilbyr 3\% årlig sparerente. Da skal banken ende opp med det samme sparebeløpet ved begge disse tilfellene:
	\begin{itemize}
		\item I løpet av 20 år tilføres sparekontoen et årlig innskudd på $ x $ kroner.
		\item 1\,500\,000 kroner settes på sparekonto og forrentes i 20 år.
	\end{itemize}
}
\abc{
\item Finn verdien til terminbeløpet $ x $.
\item Lag et script som printer terminbeløp, avdrag og renter for hele nedbetalingstiden, og som bekrefter at svaret ditt fra a) er rett.
\item Sammenlign svaret ditt med en lånekalkulator på internett. (Sett alle gebyrer lik 0).
\item Sett opp en formel som viser det årlige terminbeløpet $ x $ ved et annuitetslån, uttrykt ved lånesummen $ L $, den årlige renten $ r $, og nedbetalingstiden $ t $.
}


\newpage
\op{opgstorlbolt}
\tagop{\#regresjon \#funksjonsdrøfting \#omgjøring av enheter}
Usain Bolt har verdensrekorden for 100\enh{m} sprint. I tabellen under ser du hva tidtakeren viste ved hver 10. meter under dette rekordløpet. \vs
\begin{center}\small
	\begin{tabular}{l|c|c|c|c|c|c|c|c|c|c}
		meter & 10 & 20 & 30 & 40 & 50 & 60 & 70 & 80 & 90 & 100\\
		sekunder& 1.89 & 2.88 & 3.78 & 4.64 & 5.47 & 6.29 & 7.1 & 7.92 & 8.75 & 9.58
	\end{tabular}
\end{center}
\abc{
	\item I figuren under har vi brukt datasettet fra tabellen til å utføre regresjon med et fjerdegradspolynom. Hva er det som er helt feil med denne tilnærmingen?
	\begin{figure}
		\includegraphics[scale=0.2]{\figp{100mreg}}
	\end{figure}
	\item I datasettet kan vi legge til et punkt som vil hjelpe med å korrigere feilen poengtert i a). Hvilket punkt er dette?
	\item Bruk regresjon med et fjerdegradspolynom på datasettet fra b).
	\item Ut ifra funksjonen du fant i c), hva var toppfarten til Bolt under dette løpet?
	\item Bruk datasettet fra b) til å finne gjennomsnittsfarten til Bolt for $ {t\in[0, 1.89]} $ og for $ {t\in[1.89, 9.58]} $. Sammenlikn disse hastighetene med svaret fra oppgave d), og drøft årsaken til ulikhetene/\\likhetene.
}
\newpage
\op{opghorkurv}
\tagop{
\#funksjoner \#regresjon \#derivasjon \#vektorer i planet
}
På side 26  i dokumentet \net{https://www.vegvesen.no/globalassets/fag/handboker/hb-v120-mai-2019.pdf}{Premisser for geometrisk utforming av veger} (utformet av Statens vegvesen) er minste \outl{horisontalkurveradius} $ R_{h, \text{min}} $ gitt ved formelen 
\[ R_{h, \text{min}}=\frac{V^2}{127(\text{e}_\text{maks}+f_k)} \]
hvor 
\alg{
V &= \text{fartsgrense} \br
\text{e}_\text{maks}& = \text{maksimal overhøyde }\br
f_k &= \text{dimensjonerende sidefriksjonsfaktor}
}
Si at en veibane er beskrevet av grafen en funksjon $ f(x) $. I vedlegg ?? i \tmen\ introduserte vi sirkelen som beksriver krumningen til $ f $. Vektoren mellom sentrum $ S $ i denne sirkelen og et punkt $ A= (x, f(x)) $ på grafen til $ f $ er gitt som
\[ \vv{AS}= \frac{1}{f''}\left[-f\cdot(1+(f')^2), 1+(f')^2\right] \]
La $ r $ være radien til sirkelen som beskriver krumningen til $ f $. Statens vegvesens krav tilsier at 
\[ r< R_{h, \text{min}} \]
Bruk et digitalt kart og regresjon til å finne en polynomfunksjon som gir en god tilnærming for utvalgte veistykker hvor fartsgrensen er kjent. Sett $ \textrm{e}_\text{maks}=0 $, og bruk tabellen\footnote{Hentet fra side 22 fra nevnte dokument.} under for å velge verdien til $ f_k $. Undersøk om krumningen til veistykket oppfyller kravet til Statens vegvesen i alle punkt.
\begin{figure}
	\includegraphics[scale=0.32]{fig/tab2_7}
\end{figure}
\newpage
\op{opgstorstar} 
\tagop{
	\# modellering \# areal \# derivasjon
}
Gitt et rektangel med omkrets $ O $, og la $ x $ være den éne sidelengden.
\abc{
	\item  Finn uttrykket til funksjonen $ A(x) $, som viser aralet til rektangelet.
	\item Hvilken form har rektangelet når arealet er størst?
}

\op{opgmagnskal}
\tagop{
\#logaritmer \#overslag
}
\outl{Momentmagnitudeskalaen} er en skala som brukes til å representere styrken på jordskjelv. Hvis $ S $ er det målte \outl{seismiske momentet} til jordskjelvet, er massemagnituden $ M_w $ gitt som\footnote{Kilde: \net{https://en.wikipedia.org/wiki/Moment_magnitude_scale}{Wikipedia}.}
\[ M_w=\frac{2}{3} \log S-10.7 \]
Energien som jordskjelvet utløser er tilnærmet proporsjonal med $ S $.\os

Gitt to jordskjelv, jordskjelv $ A $ og jordskjelv $ B $, med henholdsvis seismisk moment $ S_A $ og $ S_B $. Si videre at proporsjonalitetskonstanten for energi utløst av det seismiske momentet er likt for begge jordskjelvene. Hvis jordskjelv $ A $ er målt til 1 mer enn jordskjelv $ B $ på momentmagnitudeskalaen, hva er da forholdet mellom energi utløst av jordskjelv $ A $ og energi utløst av jordskjelv $ B $? 
\newpage
\op{opgkast}
Du skal prøve å kaste en ball så langt som mulig langs et flatt strekke. Posisjonen ballen har idét den forlater handen din setter du til $ (0, 0) $. Ved å anta at tyngdekraften deretter er den eneste kraften som virker på ballen, er posisjonen til ballen godt tilnærmet ved uttrykket
\[ \vec{p}_g(t)=\vec{v}t-[0, 5t^2] \]
hvor $ \vec{v}=[v_0 \cos \theta, v_0 \sin \theta] $ er hastighetsvektoren til ballen idét den forlot handen, og $ t $ er antall tidsenheter etter at ballen har forlatt handen. Idét ballen forlater handen din har den farten $ v_0 $, $ \vec{v} $ danner vinkelen $ \theta $ med horisontallinjen.\vsk 

Ut ifra $ \vec{p}_g $, hvilken verdi må $ \theta $ ha for at kastet skal bli lengst mulig?

\op{opgveiformel}
\tagop{
\# integrasjon \# derivasjon
}
La funksjonen $ s(t) $ beskrive hvor langt et objekt har bevegd seg etter tiden $ t $. Hvi objektet har konstant akselerasjon $ a $, har vi at
\[ s''(t)=a \] 
\abc{
\item Integrer $ s''(t) $ to ganger slik at du ender opp med et uttrykk for $ s(t) $.
\item Bestem uttrykkene for $ s(t) $ og $ s'(t) $ når du vet at $ s(t)=0 $ og $ s'(t)=v_0 $.
\item Hvilken fysisk størrelse representerer $ s'(t) $?
\item Undersøk begrepet ''bevegelsesligninger''\footnote{''Eqautions of motion på engelsk.} (også kalt ''veiformler'') i en fysikkbok eller på internett. Sammenlign uttrykkene du finner med uttrykket du fant for $ s(t) $. 
}
\newpage
\op{opgbane}
\tagop{
\#vektorer i planet \#derivasjon
}
Posisjonen $ \vec{s} $ til et objekt som beveger seg i en sirkelbane kan uttrykkes som 
\[ \vec{s} = r[\cos\left(2\pi f t\right), \sin\left(2\pi f t\right)] \]
hvor $ r $ er radien til sirkelbanen, $ t $ er tiden og $ f $ er frekvensen. 
\abc{
\item $ f $ beskriver antall runder objektet fullfører per tidsenhet\footnote{Hvis tidsenheten er 'sekund', har $ f $ benevningen '1/sekund'.}. Forklar hvorfor $ 2\pi f $ kalles \outl{vinkelfarten} til objektet.
\item Finn $ \vec{s}\,'(t) $. 
\item Hva er vinkelen mellom $ \vec{s}(t) $ og $ \vec{s}\,'(t) $?
\item Bestem lengden til $ \vec{s}\,'(t) $
\item Finn $ \vec{s}\,''(t) $.
\item Bestem lengden til $ \vec{s}\,''(t) $
\item  Hva er vinkelen mellom $ \vec{s}(t) $ og $ \vec{s}\,''(t) $? Peker $ \vec{s}\,''(t) $ innover i sirkelbanen eller ut fra sirkelbanen?
\item Bruk en fysikkbok eller internett til å undersøke begrepet \outl{sentripetalakselerasjon}. Sammenlign funnene dine i denne oppgaven med informasjonen du finner.
}

\newpage
\op{opgfloogfjere}
\tagop{\# trigonometri}
En tilnærming for høy- og lavvann i Molde er gitt ved \\funksjonen 
\[ f(x)=128+80\cos\left(\frac{3\pi}{37}x\right) \]
hvor $ f $ angir cm over sjøkartnull\footnote{Sjøkartnull er som regel satt til den laveste vannstanden som kan oppnås ut ifra astronomiske betingelser (flo og fjære er i stor grad betinget av hvordan jorda, sola og månen står i forhold til hverandre).} $ t $ timer etter et gitt referanse-tidspunkt. Referansetidspunktet er valgt slik at det ved $ {t=0} $ var høyvann (flo). \os
\abc{
\item Hva er vannstanden i Molde når det er lavvann (fjøre)?\os
\item Hvor lang tid er det mellom flo og fjøre? 
}
\mers{Denne oppgaven kan med fordel løses uten digitale hjelpemidler.}
\section{Praktiske oppgaver}
\section{Eksamensoppgaver}
\end{document}

