\documentclass[english,hidelinks, 11 pt, class=report,crop=false]{standalone}
\usepackage[T1]{fontenc}
%\usepackage[utf8]{inputenc}
\usepackage{lmodern} % load a font with all the characters
\usepackage{geometry}
\geometry{verbose,paperwidth=16.1 cm, paperheight=24 cm, inner=2.3cm, outer=1.8 cm, bmargin=2cm, tmargin=1.8cm}
\setlength{\parindent}{0bp}
\usepackage{import}
\usepackage[subpreambles=false]{standalone}
\usepackage{amsmath}
\usepackage{amssymb}
\usepackage{esint}
\usepackage{babel}
\usepackage{tabu}
\makeatother
\makeatletter

\usepackage{titlesec}
\usepackage{ragged2e}
\RaggedRight
\raggedbottom
\frenchspacing

\usepackage{graphicx}
\usepackage{float}
\usepackage{subfig}
\usepackage{placeins}
\usepackage{cancel}
\usepackage{framed}
\usepackage{wrapfig}
\usepackage[subfigure]{tocloft}
\usepackage[font=footnotesize,labelfont=sl]{caption} % Figure caption
\usepackage{bm}
\usepackage[dvipsnames, table]{xcolor}
\definecolor{shadecolor}{rgb}{0.105469, 0.613281, 1}
\colorlet{shadecolor}{Emerald!15} 
\usepackage{icomma}
\makeatother
\usepackage[many]{tcolorbox}
\usepackage{multicol}
\usepackage{stackengine}

\usepackage{esvect} %For vectors with capital letters

% For tabular
\usepackage{array}
\usepackage{multirow}
\usepackage{longtable} %breakable table

% Ligningsreferanser
\usepackage{mathtools} % for mathclap
%\mathtoolsset{showonlyrefs}

% sections without numbering in toc
\newcommand\tsec[1]{\phantomsection \addcontentsline{toc}{section}{#1}
	\section*{#1}}

% index
\usepackage{imakeidx}
\makeindex[title=Indeks]

%Footnote:
\usepackage[bottom, hang, flushmargin]{footmisc}
\usepackage{perpage} 
\MakePerPage{footnote}
\addtolength{\footnotesep}{2mm}
\renewcommand{\thefootnote}{\arabic{footnote}}
\renewcommand\footnoterule{\rule{\linewidth}{0.4pt}}
\renewcommand{\thempfootnote}{\arabic{mpfootnote}}

%colors
\definecolor{c1}{cmyk}{0,0.5,1,0}
\definecolor{c2}{cmyk}{1,0.25,1,0}
\definecolor{n3}{cmyk}{1,0.,1,0}
\definecolor{neg}{cmyk}{1,0.,0.,0}


\newcommand{\nreq}[1]{
\begin{equation}
	#1
\end{equation}
}


% Equation comments
\newcommand{\cm}[1]{\llap{\color{blue} #1}}


\usepackage[inline]{enumitem}
\newcounter{rg}
\numberwithin{rg}{chapter}


\newcommand{\reg}[2][]{\begin{tcolorbox}[boxrule=0.3 mm,arc=0mm,colback=blue!3] {\refstepcounter{rg}\phantomsection \large \textbf{\therg \;#1} \vspace{5 pt}}\newline #2  \end{tcolorbox}\vspace{-5pt}}
\newcommand{\regdef}[2][]{\begin{tcolorbox}[boxrule=0.3 mm,arc=0mm,colback=blue!3] {\refstepcounter{rg}\phantomsection \large \textbf{\therg \;#1} \vspace{5 pt}}\newline #2  \end{tcolorbox}\vspace{-5pt}}
\newcommand{\words}[1]{\begin{tcolorbox}[boxrule=0.3 mm,arc=0mm,colback=teal!3] #1  \end{tcolorbox}\vspace{-5pt}}

\newcommand\alg[1]{\begin{align*} #1 \end{align*}}

\newcommand\eks[2][]{\begin{tcolorbox}[boxrule=0.3 mm,arc=0mm,enhanced jigsaw,breakable,colback=green!3] {\large \textbf{\ekstitle #1} \vspace{5 pt}\\} #2 \end{tcolorbox}\vspace{-5pt} }

\newcommand{\st}[1]{\begin{tcolorbox}[boxrule=0.0 mm,arc=0mm,enhanced jigsaw,breakable,colback=yellow!12]{ #1} \end{tcolorbox}}

\newcommand{\spr}[1]{\begin{tcolorbox}[boxrule=0.3 mm,arc=0mm,enhanced jigsaw,breakable,colback=yellow!7] {\large \textbf{\sprtitle} \vspace{5 pt}\\} #1 \end{tcolorbox}\vspace{-5pt} }

\newcommand{\sym}[1]{\colorbox{blue!15}{#1}}

\newcommand{\info}[2]{\begin{tcolorbox}[boxrule=0.3 mm,arc=0mm,enhanced jigsaw,breakable,colback=cyan!6] {\large \textbf{#1} \vspace{5 pt}\\} #2 \end{tcolorbox}\vspace{-5pt} }

\newcommand\algv[1]{\vspace{-11 pt}\begin{align*} #1 \end{align*}}

\newcommand{\regv}{\vspace{5pt}}
\newcommand{\mer}{\textsl{\note}: }
\newcommand{\mers}[1]{{\footnotesize \mer #1}}
\newcommand\vsk{\vspace{11pt}}
\newcommand{\tbs}{\vspace{5pt}}
\newcommand\vs{\vspace{-11pt}}
\newcommand\vsb{\vspace{-16pt}}
\newcommand\br{\\[5 pt]}
\newcommand{\figp}[1]{../fig/#1}
\newcommand\algvv[1]{\vs\vs\begin{align*} #1 \end{align*}}
\newcommand{\y}[1]{$ {#1} $}
\newcommand{\os}{\\[5 pt]}
\newcommand{\prbxl}[2]{
\parbox[l][][l]{#1\linewidth}{#2
	}}
\newcommand{\prbxr}[2]{\parbox[r][][l]{#1\linewidth}{
		\setlength{\abovedisplayskip}{5pt}
		\setlength{\belowdisplayskip}{5pt}	
		\setlength{\abovedisplayshortskip}{0pt}
		\setlength{\belowdisplayshortskip}{0pt} 
		\begin{shaded}
			\footnotesize	#2 \end{shaded}}}
\newcommand{\fgbxr}[2]{
	\parbox[r][][l]{#1\linewidth}{#2
}}		

\renewcommand{\cfttoctitlefont}{\Large\bfseries}
\setlength{\cftaftertoctitleskip}{0 pt}
\setlength{\cftbeforetoctitleskip}{0 pt}

\newcommand{\bs}{\\[3pt]}
\newcommand{\vn}{\\[6pt]}
\newcommand{\fig}[1]{\begin{figure}[H]
		\centering
		\includegraphics[]{\figp{#1}}
\end{figure}}

\newcommand{\figc}[2]{\begin{figure}
		\centering
		\includegraphics[]{\figp{#1}}
		\caption{#2}
\end{figure}}
\newcommand{\arc}[1]{{
		\setbox9=\hbox{#1}%
		\ooalign{\resizebox{\wd9}{\height}{\texttoptiebar{\phantom{A}}}\cr\textit{#1}}}}

\newcommand{\sectionbreak}{\clearpage} % New page on each section

\newcommand{\nn}[1]{
\begin{equation*}
	#1
\end{equation*}
}

\newcommand{\enh}[1]{\,\textrm{#1}}

%asin, atan, acos
\DeclareMathOperator{\atan}{atan}
\DeclareMathOperator{\acos}{acos}
\DeclareMathOperator{\asin}{asin}

% Comments % old cm, ggb cm is new
%\newcommand{\cm}[1]{\llap{\color{blue} #1}}

%%%

\newcommand\fork[2]{\begin{tcolorbox}[boxrule=0.3 mm,arc=0mm,enhanced jigsaw,breakable,colback=yellow!7] {\large \textbf{#1 (\expl)} \vspace{5 pt}\\} #2 \end{tcolorbox}\vspace{-5pt} }
 
%colors
\newcommand{\colr}[1]{{\color{red} #1}}
\newcommand{\colb}[1]{{\color{blue} #1}}
\newcommand{\colo}[1]{{\color{orange} #1}}
\newcommand{\colc}[1]{{\color{cyan} #1}}
\definecolor{projectgreen}{cmyk}{100,0,100,0}
\newcommand{\colg}[1]{{\color{projectgreen} #1}}

% Methods
\newcommand{\metode}[2]{
	\textsl{#1} \\[-8pt]
	\rule{#2}{0.75pt}
}

%Opg
\newcommand{\abc}[1]{
	\begin{enumerate}[label=\alph*),leftmargin=18pt]
		#1
	\end{enumerate}
}
\newcommand{\abcs}[2]{
	\begin{enumerate}[label=\alph*),start=#1,leftmargin=18pt]
		#2
	\end{enumerate}
}
\newcommand{\abcn}[1]{
	\begin{enumerate}[label=\arabic*),leftmargin=18pt]
		#1
	\end{enumerate}
}
\newcommand{\abch}[1]{
	\hspace{-2pt}	\begin{enumerate*}[label=\alph*), itemjoin=\hspace{1cm}]
		#1
	\end{enumerate*}
}
\newcommand{\abchs}[2]{
	\hspace{-2pt}	\begin{enumerate*}[label=\alph*), itemjoin=\hspace{1cm}, start=#1]
		#2
	\end{enumerate*}
}

% Exercises


\newcounter{opg}
\numberwithin{opg}{section}

\newcounter{grub}
\numberwithin{opg}{section}
\newcommand{\op}[1]{\vspace{15pt} \refstepcounter{opg}\large \textbf{\color{blue}\theopg} \vspace{2 pt} \label{#1} \\}
\newcommand{\eksop}[2]{\vspace{15pt} \refstepcounter{opg}\large \textbf{\color{blue}\theopg} (#1) \vspace{2 pt} \label{#2} \\}

\newcommand{\nes}{\stepcounter{section}
	\setcounter{opg}{0}}
\newcommand{\opr}[1]{\vspace{3pt}\textbf{\ref{#1}}}
\newcommand{\oeks}[1]{\begin{tcolorbox}[boxrule=0.3 mm,arc=0mm,colback=white]
		\textit{\ekstitle: } #1	  
\end{tcolorbox}}
\newcommand\opgeks[2][]{\begin{tcolorbox}[boxrule=0.1 mm,arc=0mm,enhanced jigsaw,breakable,colback=white] {\footnotesize \textbf{\ekstitle #1} \\} \footnotesize #2 \end{tcolorbox}\vspace{-5pt} }


% tag exercises
\newcommand{\tagop}[1]{ 
{\small \color{Gray} #1} \os
}

% License
\newcommand{\lic}{
This book is part of the \net{https://sindrsh.github.io/openmathbooks/}{OpenMathBooks} project. OpenMathBooks © 2022 by Sindre Sogge Heggen is licensed under CC BY-NC-SA 4.0. To view a copy of this license, visit \net{http://creativecommons.org/licenses/by-nc-sa/4.0/}{http://creativecommons.org/licenses/by-nc-sa/4.0/}}

%referances
\newcommand{\net}[2]{{\color{blue}\href{#1}{#2}}}
\newcommand{\hrs}[2]{\hyperref[#1]{\color{blue}#2 \ref*{#1}}}
\newcommand{\refunnbr}[2]{\hyperref[#1]{\color{blue}#2}}


\newcommand{\openmath}{\net{https://sindrsh.github.io/openmathbooks/}{OpenMathBooks}}
\newcommand{\am}{\net{https://sindrsh.github.io/FirstPrinciplesOfMath/}{AM1}}
\newcommand{\mb}{\net{https://sindrsh.github.io/FirstPrinciplesOfMath/}{MB}}
\newcommand{\tmen}{\net{https://sindrsh.github.io/FirstPrinciplesOfMath/}{TM1}}
\newcommand{\tmto}{\net{https://sindrsh.github.io/FirstPrinciplesOfMath/}{TM2}}
\newcommand{\amto}{\net{https://sindrsh.github.io/FirstPrinciplesOfMath/}{AM2}}
\newcommand{\eksbm}{
\footnotesize
Dette er opppgaver som har blitt gitt ved sentralt utformet eksamen i Norge. Oppgavene er laget av Utdanningsdirektoratet. Forkortelser i parantes viser til følgende:
\begin{center}
	\begin{tabular}{c|c}
		E & Eksempeloppgave \\
		V/H & Eksamen fra vårsemesteret/høstsemesteret\\
		G/1P/1T/R1/R2 & Fag  \\
		XX & År 20XX \\
		D1/D2 & Del 1/Del 2
	\end{tabular}
\end{center}
Tekst og innhold kan her være noe endret i forhold til originalen.
}

%Excel og GGB:

\newcommand{\g}[1]{\begin{center} {\tt #1} \end{center}}
\newcommand{\gv}[1]{\begin{center} \vspace{-11 pt} {\tt #1}  \end{center}}
\newcommand{\cmds}[2]{{\tt #1}\\
	#2}

% outline word
\newcommand{\outl}[1]{{\boldmath \color{teal}\textbf{#1}}}
%line to seperate examples
\newcommand{\linje}{\rule{\linewidth}{1pt} }


%Vedlegg
\newcounter{vedl}
\newcounter{vedleq}
\renewcommand\thevedl{\Alph{vedl}}	
\newcommand{\nreqvd}{\refstepcounter{vedleq}\tag{\thevedl \thevedleq}}

%%% Writing code

\usepackage{listings}


\definecolor{codegreen}{rgb}{0,0.6,0}
\definecolor{codegray}{rgb}{0.5,0.5,0.5}
\definecolor{codepurple}{rgb}{0.58,0,0.82}
\definecolor{backcolour}{rgb}{0.95,0.95,0.92}

\newcommand{\pymet}[1]{{\ttfamily\color{magenta} #1}}
\newcommand{\pytype}[1]{{\ttfamily\color{codepurple} #1}}

\lstdefinestyle{mystyle}{
	backgroundcolor=\color{backcolour},   
	commentstyle=\color{codegreen},
	keywordstyle=\color{magenta},
	numberstyle=\tiny\color{codegray},
	stringstyle=\color{codepurple},
	basicstyle=\ttfamily\footnotesize,
	breakatwhitespace=false,         
	breaklines=true,                 
	captionpos=b,                    
	keepspaces=true,                 
	numbers=left,                    
	numbersep=5pt,                  
	showspaces=false,                
	showstringspaces=false,
	showtabs=false,                  
	tabsize=2,
	inputencoding=utf8,
	extendedchars=true,
	literate= {
		{å}{{\aa}}1 
		{æ}{{\ae}}1 
		{ø}{{\o}}1
	}
}

\lstset{style=mystyle}

\newcommand{\python}[1]{
\begin{tcolorbox}[boxrule=0.3 mm,arc=0mm,colback=white]
\lstinputlisting[language=Python]{#1}
\end{tcolorbox}}
\newcommand{\pythonut}[2]{
\begin{tcolorbox}[boxrule=0.3 mm,arc=0mm,colback=white]
\small 
%\textbf{Kode}
\lstinputlisting[language=Python]{#1}	
\vspace{11pt}
\textbf{Utdata} \\ \ttfamily
#2
\end{tcolorbox}}
%%%

%page number
%\usepackage{fancyhdr}
%\pagestyle{fancy}
%\fancyhf{}
%\renewcommand{\headrule}{}
%\fancyhead[RO, LE]{\thepage}

\usepackage{datetime2}
%%\usepackage{sansmathfonts} for dyslexia-friendly math
\usepackage[]{hyperref}




\begin{document}
{\Large Eksamen R2 Våren 2024\hfill {\footnotesize Løsning fra \color{blue} \href{https://sindreheggen.wordpress.com/}{OpenMathBooks prosjektet}}}	
\subsection*{Oppgave 1}	
\abc{
\item \alg{
\int_{-1}^{0} -x^3+3x \, dx &= \left[-\frac{1}{4}x^4+\frac{3}{2}x^2\right]_{-1}^{0} \\
&= -\frac{1}{4}0^4+\frac{3}{2}\cdot0^2-\left(\frac{1}{4}\cdot(-1)^4+\frac{3}{2}\cdot(-1)^2\right) \\
&= \frac{1}{4}-\frac{3}{2}\\
&= -\frac{5}{4}
}
\item Vi har at $ f'(x)=-2x^2+3 $, som betyr at $ f $ har maksimalpunkt og minimumspunkt når $ x=\pm\sqrt{\dfrac{3}{2}} $. Dette betyr at $ f $ er enten voksende eller synkende på hele intervallet $ [-1, 1] $, og dermed vil tallverdien til $ \int_{-1}^0 f\, dx $ representere arealet avgrenset av grafen til $ f $, $ x $-aksen og linjene $ x=-1 $ og $ x=0 $. Tilsvarende får vi for tallverdien til $ \int_0^{1} f \,dx $. Tallverdien til det bestemte integralet i oppgave a) er den samme for $ {x=\pm 1} $, fordi alle leddene med $ x $ har partalls eksponent. Dette betyr at $ |\int_{-1}^0 f\,dx| = |\int_{0}^1 f\,dx|  $, og da er arealet spurt om i oppgaven lik $ 2\cdot\frac{5}{4}=\frac{5}{2} $.


\subsection*{Oppgave 2}
Vi setter $ u=\sin x $, da er $ u'=\cos x$. Av kjerneregelen har vi da at
\alg{
\int \sin^3 x \cos x \,dx &= \int u^3 u' \,dx \\
&= \int u^3 \,du \\
&= \frac{1}{4}u^4 + C \\
&= \frac{1}{4}\sin^4 x + C
}
}

\newpage
\subsection*{Oppgave 3}
\abc{
\item Eleven prøver å finne ut hva $ n $ er når summen av den aritmetiske rekka
\[ 2+6+10+...+a_n \]
passerer 200.
\item Da koden ser ut til å gi det rett svar, gjenstår det for oss å finne $ n $. Den eksplisitte formelen til rekka er $ a_i=2+4(i-1) $. Av formelen for summen $ S_n $ til en aritmetisk rekke har vi at
\[ S_{n}= n\cdot\frac{2+a_n}{2}=n\frac{(2+2+4(n-1))}{2}=2n^2 \]
Når $ S_n>200 $ er altså
\alg{
2n^2> 200 \\
n^2 > 100 \\
n > 10
}
Dette betyr at når $ n=11 $ så passerer summen 200.
}

\subsection*{Oppgave 4}
\abc{
\item Vi har at
\alg{
	\vv{AB}&= [4-1, 1- 1, 1-0] = [3, 0, 1]\vn
	\vv{AC}&= [1, -1, -1]
}
\alg{
	\vv{AB}\times\vv{AC} \left|\begin{matrix}
		\vec{e}_x & \vec{e}_y & \vec{e}_z \\
		3 & 0 & 1 \\
		1 & -1 & -1
	\end{matrix}\right|&= [0(-1)-(-1)1, - (3(-1)-1\cdot1), 3(-1)-1\cdot0]\\
	&= [1, 4, -3]
}
Arealet til trekanten er gitt som
\alg{
	\frac{1}{2}|\vv{AB}\times\vv{AC}|&=\frac{1}{2}\sqrt{1^2+4^2+(-3)^2}=\frac{\sqrt{26}}{2}
}
\item Avstanden fra punktet $ C $ til linja gjennom $ A $ og $ B $ tilsvarer høyden i trekanten med segmentet $ AB $ som grunnlinje. Vi har at $ |\vv{AB}|=\sqrt{3^2+1^2}=\sqrt{10} $. Altså er høyden $ h $ gitt ved
\alg{
\frac{\sqrt{26}}{2}&=\frac{\sqrt{10}\cdot h}{2} \\
h &= \sqrt{\frac{13}{5}}
}
\item En linje med $ \vv{AB}\times\vv{AC} $ som retningsvektor vil stå vinkelrett på $ \alpha $. Dermed kan $ l $ parameteriseres som
\[ l: \left\lbrace{
	\begin{array}{l}
		x=-2-1t  \\
		y= 1+ 4t   \\
		z= 4-3t 
	\end{array}
}\right. \]
\item 
\textbf{Løsningsmetode 1} \\
Skal $ m $ være parallell med $ \alpha $  må vektoren $ P-(0, 0, z)=[-2, 1, 4-z] $ stå vinkelrett på $ \vv{AB}\times\vv{AC} $, om er en normalvektor til $ \alpha $. Dette betyr at
\alg{
[-2, 1, 4-z]\cdot[1, 4, -3]  &= 0\\
-2+4-12+3z &= 0 \\
z &= \frac{10}{3}
}
Altså er $ D=(0, 0, \frac{3}{10}) $.\\[12pt]

\textbf{Løsningsmetode 2} \\
$ \alpha $ kan parameteriseres som 
\[ \alpha: \left\lbrace{
	\begin{array}{l}
		x=1+3t+s  \\
		y= 1-s   \\
		z= t-s 
	\end{array}
}\right. \]
Når $ \alpha $ skjærer $ z $-aksen har vi at
\alg{
0 &= 1+3t+s \\
0 &= 1-s
}
Altså er $ s=1 $ og $ t= -\frac{2}{3} $. $ \alpha $ skjærer altså $ z $-aksen i $ E=(0, 0, -\frac{5}{3}) $. Avstanden fra $ D $ til $ E $ tilsvarer avstanden fra $ P $ til $ \alpha $ langs $ z $-aksen. Når $ x=-2 $ og $ y=1 $ har vi 
at
\alg{
	-2 &= 1+3t+s \\
	1 &= 1-s
}
Altså er $ s=0 $ og $ t=-1 $, og da er $ z=-1 $. Avstanden fra $ P $ til $ \alpha $ langs $ z $-aksen er derfor $ 4-(-1)=5 $, og da er $ D=(0, 0, \frac{10}{3}) $
}


\subsection*{Oppgave 5}
\abc{
\item Når $ f(x)=0 $ er
\alg{
2\cdot\sin\left(\frac{\pi}{6}x-\frac{\pi}{3}\right)-1 &= 0 \\
\sin\left(\frac{\pi}{6}x-\frac{\pi}{3}\right)&= \frac{1}{2}
}
Da $ \asin \frac{1}{2}=\frac{\pi}{6} $ har vi at  
\[ \frac{\pi}{6}x-\frac{\pi}{3}=\frac{\pi}{6}+2\pi n\qquad\vee\qquad  \frac{\pi}{6}x-\frac{\pi}{3}=\pi-\frac{\pi}{6}+2\pi n\]
hvor $ n\in\mathbb{Z} $. Altså er
\[ x=3+12 n\qquad\vee\qquad  x=7+12 n\]
Følgelig er $ x\in\lbrace 3, 7, 15, 19\rbrace $.
\item En sinusfunksjon $ g(x)=a\sin(kx+c)+d $ har amplitude $ |a| $, periode $ P=\frac{2\pi}{k} $, fase $ c $ og likevektslinje $ d $. For $ f $ har vi at
\alg{
|a| &= 2 \\
P &= 2\pi \cdot \frac{6}{\pi}=12 \\
c &= -\frac{\pi}{3} \\
d &= -1
}
Forskyvningen langs likevektslinja tilsvarer avstanden mellom et punkt på grafen til $ h(x)=2\sin\left(\frac{\pi}{6}x\right)-1 $ og et samsvarende punkt på grafen til $ f $. Kjernen til $ h $ blir 0 når $ x>0 $. Det samsvarende punktet på grafen til $ f $ får vi i det første tilfellet når kjernen til $ f $ blir 0 for $ x\geq0 $, altså når
\alg{
\frac{\pi}{6}x-\frac{\pi}{3}&=0 \\
x &= 2
}
Altså er forskyvningen langs likevektslinja lik 2.
}
\newpage
\subsection*{Oppgave 1}
\abc{
\item Farten er ca. 31.21 m/s. idet ballen blir skutt (celle 2).
\item Når ballen treffer banen er $ z=0 $, og da er $ t\approx1.43 $ (celle 3). Da er ballen ca 43.5 meter unna hjørnet (celle 4).
\item Ballen er på sitt høyeste når $ z $ har sin høyeste verdi, som er når $ t\approx0.71 $ (celle 5). Da er farten ca. 30.41 m/s (celle 6) og høyden 2.5 m (celle 5)
}
\localfig{opg1}{0.2}

\subsection*{Oppgave 2}
Ut i fra figuren leser vi av noen omtrentlige punkt, og bruker regresjon. Her nøyer vi oss med å se på grafen for å avgjøre hva som er en god tilnærming, og ender da med et 4. grads polynom. Bruker så formelen for volumet til et omdreiningslegeme, og finner at volumer ca. er 327 $ \text{cm}^3 $. 
\localfig{opg2}{0.3}


\newpage

\subsection*{Oppgave 3}
\abc{
\item Ut ifra figuren ser vi at $ T $ er tilnærmet periodisk mellom 0 til 365, altså i løpet av et år, og at $ V_f $ er tilnærmet $ [15, 19] $. I tillegg er $ T(90)\approx19 $ (rad 7), hvor $ x=90 $ tilsvarer 1. april.
\item  $ T'(x) $ gir endringen minutter per dag og 3 minutter tilsvarer $ \frac{3}{60} $ time. Denne endringen skjer i løpet av dag $ x=46 $ og dag $ x=135 $, som tilsvarer  15. februar og 15. mai (rad 3 og rad 4). (Vi har bare tatt hensyn til den positive endringen. Fremgangsmåten for å finne den negative endringen er helt lik).
\item Den største endringen er når $ T'(x) $ har sitt toppunkt, som er i løpet av dag $ x=90 $ (rad 5). Da er endringen på ca. 4 minutter per dag. (rad 8). (Vi har bare tatt hensyn til den positive endringen. Fremgangsmåten for å finne den negative endringen er helt lik.)
}
\localfig{opg3}{0.2}
\newpage
\subsection*{Oppgave 4}
\abc{
\item Kubikktall nr $ n $ er lik $ n^3 $. Altså er rekursiv formel
\[ S_{n+1}=S_n+(n+1)^3 \]
Regresjonsanalysen gir at $ R^2=1 $ når eksplisitt formel er
\[ S_n =\frac{1}{4}n^4+\frac{1}{2}n^3+\frac{1}{4}n^2 \]


\item \phantom{text}\\\pythonut{opg4.py}{
1625625
}
\item Uttrykket stemmer for $ S_1=1 $ (celle 2). Vi antar at uttrykker stemmer for $ S_n $, og sjekker om det da stemmer for $ S_{n+1} $, som det gjør (celle 3). 
}
\localfig{opg4}{0.24}
\newpage
\subsection*{Oppgave 5}
\abc{
\item I celle 3 finner vi sentrum i sirkelen, og i celle 4 finner vi radien til sirkelen. I celle 5 definerer vi ligningen til $ \gamma $. Den minste avstanden mellom sirkelen og planet er avstanden mellom sentrum i sirkelen og planet, fratrekt radien til sirkelen. Vi definerer normalvektoren til $ \gamma $ i celle 5, og bruker formelen for avstanden mellom et punkt og et plan. Den minste avstanden mellom sirkelen og planet er $ 4-\sqrt{6} $ (celle 6)
\item Da $ \gamma $ og $ \alpha $ er parallelle, er en normalvektor til $ \gamma $ også en normalvektor til $ \alpha $. I avstandsformelen ersatter vi $ -14 $ med $ d $, og finner verdien til $ d $ som gir at avstanden er den samme (celle 7). Da får vi at $ \alpha $ er gitt ved ligningen
\[ x+2y+2z+10=0 \]
\localfig{opg5}{0.4}
}

\end{document}