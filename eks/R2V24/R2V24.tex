\documentclass[english,hidelinks,pdftex, 11 pt, class=report,crop=false]{standalone}
\usepackage[T1]{fontenc}
\usepackage[utf8]{luainputenc}
\usepackage{lmodern} % load a font with all the characters
\usepackage{geometry}
\geometry{verbose,a4paper, inner=0cm, outer=0 cm, bmargin=2cm, tmargin=1cm}
%\textwidth=12cm
\setlength{\parindent}{0bp}
\usepackage{import}
\usepackage[subpreambles=false]{standalone}
\usepackage{amsmath}
\usepackage{amssymb}
\usepackage{esint}
\usepackage{babel}
\usepackage{tabu}
\usepackage[dvipsnames, table]{xcolor}
\usepackage{cancel}
\makeatother
\makeatletter
\usepackage{datetime2}
\usepackage{titlesec}
\usepackage[many]{tcolorbox}

% Eheter
\newcommand{\enh}[1]{\,\textrm{#1}}
%referances
\newcommand{\net}[2]{{\color{blue}\href{#1}{#2}}}

%Spaces
\newcommand{\vsk}{\\[12pt]}
\newcommand{\vs}{\vspace{-12pt}}

% Tabell for opplegg

\newcommand{\ovlist}[1]{
\vspace{-16pt}
\begin{itemize}
	#1
\end{itemize}
}

% Chapters and sections
\titleformat{\section}[block]{\bfseries}{\hspace{3cm}\thesection}{5pt}{}
\titleformat{\subsection}[block]{\bfseries}{\hspace{3cm}\thesection}{5pt}{}
\newcommand{\sectionbreak}{\clearpage} % New page on each section
 

\newlength{\mywidth}
\setlength{\mywidth}{14cm}

\newcommand{\cont}[1]{
\begin{tcolorbox}[center, boxrule=0.0 mm, width=\mywidth,arc=0mm,enhanced jigsaw,,colback=white,breakable]
#1	
\end{tcolorbox}
}

\newcommand{\info}[5]{
\begin{tcolorbox}[center, boxrule=0.1 mm, width=\mywidth,arc=0mm,enhanced jigsaw,breakable,colback=yellow!5]	
	
	\footnotesize
	\textbf{Øvingsområde}\\[5pt] #1 
	
	\textbf{Utstyr}\\ #2  \\
	
	\begin{tabular}{@{} p{4cm} p{4cm} l} 
		\textbf{Tid} & \textbf{Elevinndeling} & \textbf{Læringsarena} \\
		#3  & #4 & #5
	\end{tabular} 
\end{tcolorbox}	
}

\newcommand{\gjen}[1]{\begin{tcolorbox}[center,boxrule=0.1 mm, width=\mywidth,arc=0mm,colback=blue!3] {\large \textbf{Gjennomføring} \vspace{5 pt}}\newline #1  \end{tcolorbox}\vspace{-5pt}}
\newcommand{\eks}[1]{\begin{tcolorbox}[center,boxrule=0.1 mm, width=\mywidth,arc=0mm,colback=green!3] {\large \textbf{Eksempel} \vspace{5 pt}}\newline #1  \end{tcolorbox}\vspace{-5pt}}

\newcounter{opl}
%\numberwithin{opl}{article}


\newcommand{\opl}[1]{
\newpage
{\refstepcounter{opl} %\phantomsection 
\large \textbf{\theopl \;#1} \vsk}
}

% Headlines
\newcommand{\fork}{\textbf{Forkunnskapar}\\}
\newcommand{\forb}{\textbf{Forberedelsar}\\}
\newcommand{\opgvr}{\textbf{Oppgaver}}



%colors
\newcommand{\colr}[1]{{\color{red} #1}}
\newcommand{\colb}[1]{{\color{blue} #1}}
\newcommand{\colo}[1]{{\color{orange} #1}}
\newcommand{\colc}[1]{{\color{cyan} #1}}
\definecolor{projectgreen}{cmyk}{100,0,100,0}
\newcommand{\colg}[1]{{\color{projectgreen} #1}}

% Lister med bokstavar
\usepackage[inline]{enumitem}
% Opg
\newcommand{\abc}[1]{
	\begin{enumerate}[label=\alph*),leftmargin=18pt]
		#1
	\end{enumerate}
}

\usepackage[]{hyperref}



\begin{document}
{\Large Eksamen R2 Våren 2024\hfill {\footnotesize Løsning fra \color{blue} \href{https://sindreheggen.wordpress.com/}{OpenMathBooks prosjektet}}}	
\subsection*{Oppgave 1}	
\abc{
\item \alg{
\int_{-1}^{0} -x^3+3x \, dx &= \left[-\frac{1}{4}x^4+\frac{3}{2}x^2\right]_{-1}^{0} \\
&= -\frac{1}{4}0^4+\frac{3}{2}\cdot0^2-\left(\frac{1}{4}\cdot(-1)^4+\frac{3}{2}\cdot(-1)^2\right) \\
&= \frac{1}{4}-\frac{3}{2}\\
&= -\frac{5}{4}
}
\item Vi har at $ f'(x)=-2x^2+3 $, som betyr at $ f $ har maksimalpunkt og minimumspunkt når $ x=\pm\sqrt{\dfrac{3}{2}} $. Dette betyr at $ f $ er enten voksende eller synkende på hele intervallet $ [-1, 1] $, og dermed vil tallverdien til $ \int_{-1}^0 f\, dx $ representere arealet avgrenset av grafen til $ f $, $ x $-aksen og linjene $ x=-1 $ og $ x=0 $. Tilsvarende får vi for tallverdien til $ \int_0^{1} f \,dx $. Tallverdien til det bestemte integralet i oppgave a) er den samme for $ {x=\pm 1} $, fordi alle leddene med $ x $ har partalls eksponent. Dette betyr at $ |\int_{-1}^0 f\,dx| = |\int_{0}^1 f\,dx|  $, og da er arealet spurt om i oppgaven lik $ 2\cdot\frac{5}{4}=\frac{5}{2} $.


\subsection*{Oppgave 2}
Vi setter $ u=\sin x $, da er $ u'=\cos x$. Av kjerneregelen har vi da at
\alg{
\int \sin^3 x \cos x \,dx &= \int u^3 u' \,dx \\
&= \int u^3 \,du \\
&= \frac{1}{4}u^4 + C \\
&= \frac{1}{4}\sin^4 x + C
}
}

\newpage
\subsection*{Oppgave 3}
\abc{
\item Eleven prøver å finne ut hva $ n $ er når summen av den aritmetiske rekka
\[ 2+6+10+...+a_n \]
passerer 200.
\item Da koden ser ut til å gi det rett svar, gjenstår det for oss å finne $ n $. Den eksplisitte formelen til rekka er $ a_i=2+4(i-1) $. Av formelen for summen $ S_n $ til en aritmetisk rekke har vi at
\[ S_{n}= n\cdot\frac{2+a_n}{2}=n\frac{(2+2+4(n-1))}{2}=2n^2 \]
Når $ S_n>200 $ er altså
\alg{
2n^2> 200 \\
n^2 > 100 \\
n > 10
}
Dette betyr at når $ n=11 $ så passerer summen 200.
}

\subsection*{Oppgave 4}
\abc{
\item Vi har at
\alg{
	\vv{AB}&= [4-1, 1- 1, 1-0] = [3, 0, 1]\vn
	\vv{AC}&= [1, -1, -1]
}
\alg{
	\vv{AB}\times\vv{AC} \left|\begin{matrix}
		\vec{e}_x & \vec{e}_y & \vec{e}_z \\
		3 & 0 & 1 \\
		1 & -1 & -1
	\end{matrix}\right|&= [0(-1)-(-1)1, - (3(-1)-1\cdot1), 3(-1)-1\cdot0]\\
	&= [1, 4, -3]
}
Arealet til trekanten er gitt som
\alg{
	\frac{1}{2}|\vv{AB}\times\vv{AC}|&=\frac{1}{2}\sqrt{1^2+4^2+(-3)^2}=\frac{\sqrt{26}}{2}
}
\item Avstanden fra punktet $ C $ til linja gjennom $ A $ og $ B $ tilsvarer høyden i trekanten med segmentet $ AB $ som grunnlinje. Vi har at $ |\vv{AB}|=\sqrt{3^2+1^2}=\sqrt{10} $. Altså er høyden $ h $ gitt ved
\alg{
\frac{\sqrt{26}}{2}&=\frac{\sqrt{10}\cdot h}{2} \\
h &= \sqrt{\frac{13}{5}}
}
\item En linje med $ \vv{AB}\times\vv{AC} $ som retningsvektor vil stå vinkelrett på $ \alpha $. Dermed kan $ l $ parameteriseres som
\[ l: \left\lbrace{
	\begin{array}{l}
		x=-2-1t  \\
		y= 1+ 4t   \\
		z= 4-3t 
	\end{array}
}\right. \]
\item 
\textbf{Løsningsmetode 1} \\
Skal $ m $ være parallell med $ \alpha $  må vektoren $ P-(0, 0, z)=[-2, 1, 4-z] $ stå vinkelrett på $ \vv{AB}\times\vv{AC} $, om er en normalvektor til $ \alpha $. Dette betyr at
\alg{
[-2, 1, 4-z]\cdot[1, 4, -3]  &= 0\\
-2+4-12+3z &= 0 \\
z &= \frac{10}{3}
}
Altså er $ D=(0, 0, \frac{3}{10}) $.\\[12pt]

\textbf{Løsningsmetode 2} \\
$ \alpha $ kan parameteriseres som 
\[ \alpha: \left\lbrace{
	\begin{array}{l}
		x=1+3t+s  \\
		y= 1-s   \\
		z= t-s 
	\end{array}
}\right. \]
Når $ \alpha $ skjærer $ z $-aksen har vi at
\alg{
0 &= 1+3t+s \\
0 &= 1-s
}
Altså er $ s=1 $ og $ t= -\frac{2}{3} $. $ \alpha $ skjærer altså $ z $-aksen i $ E=(0, 0, -\frac{5}{3}) $. Avstanden fra $ D $ til $ E $ tilsvarer avstanden fra $ P $ til $ \alpha $ langs $ z $-aksen. Når $ x=-2 $ og $ y=1 $ har vi 
at
\alg{
	-2 &= 1+3t+s \\
	1 &= 1-s
}
Altså er $ s=0 $ og $ t=-1 $, og da er $ z=-1 $. Avstanden fra $ P $ til $ \alpha $ langs $ z $-aksen er derfor $ 4-(-1)=5 $, og da er $ D=(0, 0, \frac{10}{3}) $
}


\subsection*{Oppgave 5}
\abc{
\item Når $ f(x)=0 $ er
\alg{
2\cdot\sin\left(\frac{\pi}{6}x-\frac{\pi}{3}\right)-1 &= 0 \\
\sin\left(\frac{\pi}{6}x-\frac{\pi}{3}\right)&= \frac{1}{2}
}
Da $ \asin \frac{1}{2}=\frac{\pi}{6} $ har vi at  
\[ \frac{\pi}{6}x-\frac{\pi}{3}=\frac{\pi}{6}+2\pi n\qquad\vee\qquad  \frac{\pi}{6}x-\frac{\pi}{3}=\pi-\frac{\pi}{6}+2\pi n\]
hvor $ n\in\mathbb{Z} $. Altså er
\[ x=3+12 n\qquad\vee\qquad  x=7+12 n\]
Følgelig er $ x\in\lbrace 3, 7, 15, 19\rbrace $.
\item En sinusfunksjon $ g(x)=a\sin(kx+c)+d $ har amplitude $ |a| $, periode $ P=\frac{2\pi}{k} $, fase $ c $ og likevektslinje $ d $. For $ f $ har vi at
\alg{
|a| &= 2 \\
P &= 2\pi \cdot \frac{6}{\pi}=12 \\
c &= -\frac{\pi}{3} \\
d &= -1
}
Forskyvningen langs likevektslinja tilsvarer avstanden mellom et punkt på grafen til $ h(x)=2\sin\left(\frac{\pi}{6}x\right)-1 $ og et samsvarende punkt på grafen til $ f $. Kjernen til $ h $ blir 0 når $ x>0 $. Det samsvarende punktet på grafen til $ f $ får vi i det første tilfellet når kjernen til $ f $ blir 0 for $ x\geq0 $, altså når
\alg{
\frac{\pi}{6}x-\frac{\pi}{3}&=0 \\
x &= 2
}
Altså er forskyvningen langs likevektslinja lik 2.
}
\newpage
\subsection*{Oppgave 1}
\abc{
\item Farten er ca. 31.21 m/s. idet ballen blir skutt (celle 2).
\item Når ballen treffer banen er $ z=0 $, og da er $ t\approx1.43 $ (celle 3). Da er ballen ca 43.5 meter unna hjørnet (celle 4).
\item Ballen er på sitt høyeste når $ z $ har sin høyeste verdi, som er når $ t\approx0.71 $ (celle 5). Da er farten ca. 30.41 m/s (celle 6) og høyden 2.5 m (celle 5)
}
\localfig{opg1}{0.2}

\subsection*{Oppgave 2}
Ut i fra figuren leser vi av noen omtrentlige punkt, og bruker regresjon. Her nøyer vi oss med å se på grafen for å avgjøre hva som er en god tilnærming, og ender da med et 4. grads polynom. Bruker så formelen for volumet til et omdreiningslegeme, og finner at volumer ca. er 327 $ \text{cm}^3 $. 
\localfig{opg2}{0.3}


\newpage

\subsection*{Oppgave 3}
\abc{
\item Ut ifra figuren ser vi at $ T $ er tilnærmet periodisk mellom 0 til 365, altså i løpet av et år, og at $ V_f $ er tilnærmet $ [15, 19] $. I tillegg er $ T(90)\approx19 $ (rad 7), hvor $ x=90 $ tilsvarer 1. april.
\item  $ T'(x) $ gir endringen minutter per dag og 3 minutter tilsvarer $ \frac{3}{60} $ time. Denne endringen skjer i løpet av dag $ x=46 $ og dag $ x=135 $, som tilsvarer  15. februar og 15. mai (rad 3 og rad 4). (Vi har bare tatt hensyn til den positive endringen. Fremgangsmåten for å finne den negative endringen er helt lik).
\item Den største endringen er når $ T'(x) $ har sitt toppunkt, som er i løpet av dag $ x=90 $ (rad 5). Da er endringen på ca. 4 minutter per dag. (rad 8). (Vi har bare tatt hensyn til den positive endringen. Fremgangsmåten for å finne den negative endringen er helt lik.)
}
\localfig{opg3}{0.2}
\newpage
\subsection*{Oppgave 4}
\abc{
\item Kubikktall nr $ n $ er lik $ n^3 $. Altså er rekursiv formel
\[ S_{n+1}=S_n+(n+1)^3 \]
Regresjonsanalysen gir at $ R^2=1 $ når eksplisitt formel er
\[ S_n =\frac{1}{4}n^4+\frac{1}{2}n^3+\frac{1}{4}n^2 \]


\item \phantom{text}\\\pythonut{opg4.py}{
1625625
}
\item Uttrykket stemmer for $ S_1=1 $ (celle 2). Vi antar at uttrykker stemmer for $ S_n $, og sjekker om det da stemmer for $ S_{n+1} $, som det gjør (celle 3). 
}
\localfig{opg4}{0.24}
\newpage
\subsection*{Oppgave 5}
\abc{
\item I celle 3 finner vi sentrum i sirkelen, og i celle 4 finner vi radien til sirkelen. I celle 5 definerer vi ligningen til $ \gamma $. Den minste avstanden mellom sirkelen og planet er avstanden mellom sentrum i sirkelen og planet, fratrekt radien til sirkelen. Vi definerer normalvektoren til $ \gamma $ i celle 5, og bruker formelen for avstanden mellom et punkt og et plan. Den minste avstanden mellom sirkelen og planet er $ 4-\sqrt{6} $ (celle 6)
\item Da $ \gamma $ og $ \alpha $ er parallelle, er en normalvektor til $ \gamma $ også en normalvektor til $ \alpha $. I avstandsformelen ersatter vi $ -14 $ med $ d $, og finner verdien til $ d $ som gir at avstanden er den samme (celle 7). Da får vi at $ \alpha $ er gitt ved ligningen
\[ x+2y+2z+10=0 \]
\localfig{opg5}{0.4}
}

\end{document}