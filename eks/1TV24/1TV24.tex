\documentclass[english,hidelinks,pdftex, 11 pt, class=report,crop=false]{standalone}
\usepackage[T1]{fontenc}
\usepackage[utf8]{luainputenc}
\usepackage{lmodern} % load a font with all the characters
\usepackage{geometry}
\geometry{verbose,a4paper, inner=0cm, outer=0 cm, bmargin=2cm, tmargin=1cm}
%\textwidth=12cm
\setlength{\parindent}{0bp}
\usepackage{import}
\usepackage[subpreambles=false]{standalone}
\usepackage{amsmath}
\usepackage{amssymb}
\usepackage{esint}
\usepackage{babel}
\usepackage{tabu}
\usepackage[dvipsnames, table]{xcolor}
\usepackage{cancel}
\makeatother
\makeatletter
\usepackage{datetime2}
\usepackage{titlesec}
\usepackage[many]{tcolorbox}

% Eheter
\newcommand{\enh}[1]{\,\textrm{#1}}
%referances
\newcommand{\net}[2]{{\color{blue}\href{#1}{#2}}}

%Spaces
\newcommand{\vsk}{\\[12pt]}
\newcommand{\vs}{\vspace{-12pt}}

% Tabell for opplegg

\newcommand{\ovlist}[1]{
\vspace{-16pt}
\begin{itemize}
	#1
\end{itemize}
}

% Chapters and sections
\titleformat{\section}[block]{\bfseries}{\hspace{3cm}\thesection}{5pt}{}
\titleformat{\subsection}[block]{\bfseries}{\hspace{3cm}\thesection}{5pt}{}
\newcommand{\sectionbreak}{\clearpage} % New page on each section
 

\newlength{\mywidth}
\setlength{\mywidth}{14cm}

\newcommand{\cont}[1]{
\begin{tcolorbox}[center, boxrule=0.0 mm, width=\mywidth,arc=0mm,enhanced jigsaw,,colback=white,breakable]
#1	
\end{tcolorbox}
}

\newcommand{\info}[5]{
\begin{tcolorbox}[center, boxrule=0.1 mm, width=\mywidth,arc=0mm,enhanced jigsaw,breakable,colback=yellow!5]	
	
	\footnotesize
	\textbf{Øvingsområde}\\[5pt] #1 
	
	\textbf{Utstyr}\\ #2  \\
	
	\begin{tabular}{@{} p{4cm} p{4cm} l} 
		\textbf{Tid} & \textbf{Elevinndeling} & \textbf{Læringsarena} \\
		#3  & #4 & #5
	\end{tabular} 
\end{tcolorbox}	
}

\newcommand{\gjen}[1]{\begin{tcolorbox}[center,boxrule=0.1 mm, width=\mywidth,arc=0mm,colback=blue!3] {\large \textbf{Gjennomføring} \vspace{5 pt}}\newline #1  \end{tcolorbox}\vspace{-5pt}}
\newcommand{\eks}[1]{\begin{tcolorbox}[center,boxrule=0.1 mm, width=\mywidth,arc=0mm,colback=green!3] {\large \textbf{Eksempel} \vspace{5 pt}}\newline #1  \end{tcolorbox}\vspace{-5pt}}

\newcounter{opl}
%\numberwithin{opl}{article}


\newcommand{\opl}[1]{
\newpage
{\refstepcounter{opl} %\phantomsection 
\large \textbf{\theopl \;#1} \vsk}
}

% Headlines
\newcommand{\fork}{\textbf{Forkunnskapar}\\}
\newcommand{\forb}{\textbf{Forberedelsar}\\}
\newcommand{\opgvr}{\textbf{Oppgaver}}



%colors
\newcommand{\colr}[1]{{\color{red} #1}}
\newcommand{\colb}[1]{{\color{blue} #1}}
\newcommand{\colo}[1]{{\color{orange} #1}}
\newcommand{\colc}[1]{{\color{cyan} #1}}
\definecolor{projectgreen}{cmyk}{100,0,100,0}
\newcommand{\colg}[1]{{\color{projectgreen} #1}}

% Lister med bokstavar
\usepackage[inline]{enumitem}
% Opg
\newcommand{\abc}[1]{
	\begin{enumerate}[label=\alph*),leftmargin=18pt]
		#1
	\end{enumerate}
}

\usepackage[]{hyperref}



\begin{document}
{\Large Eksamen 1T Våren 2024\hfill {\footnotesize Løsning fra \color{blue} \href{https://sindreheggen.wordpress.com/}{OpenMathBooks prosjektet}}}	
\subsection*{Oppgave 1}	
\abc{
\item $ \tan u = \frac{6}{8} $ og $ \tan v = \frac{8}{6} $, dermed er $ \tan u\cdot \tan v =\frac{6}{8}\cdot\frac{8}{6}=1 $. 
\item Vi setter $ a $ og $ b $ som vist i figuren under. Da er $ {\tan u= \frac{a}{b}} $ og $ {\tan v = \frac{b}{a}} $. Følgelig er
\[ \tan u \cdot \tan v = \frac{a}{b}\cdot\frac{b}{a}=1 \]
Altså gjelder dette for alle rettvinklede trekanter.
\localfig{opg1bd1}{1}
}

\subsection*{Oppgave 2}
Hun kan ha utført polynomdivisjon på regnestykket $ \frac{2x^3+3x^2-11x-6}{2x^2+7x+3}  $ eller $ \frac{2x^3+3x^2-11x-6}{x-2}  $.
\alg{
	\phantom{-}&(2x^3+3x^2-11x-6):(2x^2+7x+3) = x-2\\ 
	-&\underline{(2x^3+7x^2+3x)} \\
	&\phantom{000\,\,x\;}10x^2-14x-6  \\
	&\phantom{xx}-\underline{(10x^2-14x-6)}\\
	&\phantom{aaaaaaaaaaaaaaaa''}0 \\
}
\alg{
	\phantom{-}&(2x^3+3x^2-11x-6):(x-2) = 2x^2+7x+3\\ 
	-&\underline{(2x^3-4x^2)} \\
	&\phantom{000\,\,x-}7x^2-11x-6  \\
	&\phantom{xxx}-\underline{(7x^2-14x)}\\
	&\phantom{aaaaaaaaaaa''\,}3x-6 \\
	&\phantom{aaaaaaaax}-\underline{(3x-6)}\\
	&\phantom{aaaaaaaaaaaaaaaaa} 0
}
Polynomdivisjonene viser at både $ x-2 $ og $ 2x^2+7x+3 $ er faktorer i $ 2x^3+2x^2-11x-6 $.

\subsection*{Oppgave 3}
Vi setter to uttrykk for arealet til det grønne området lik hverandre:
\alg{
a(a-b+b)-b^2\cdot &= a\cdot(a-b) + b(a-b) \\
a^2-b^2&= (a+b)(a-b)
}


\subsection*{Oppgave 4}
\alg{
f(0) &= 0^2-3\cdot0+7 = 7\\
f(5) &= 5^2-3\cdot5+7 = 17\\
\frac{f(5)-f(0)}{5-0}&=\frac{17-7}{5}=2
}
Verdien 2 vil bli skrevet ut, og dette forteller at den gjennomsnittlige endringen til $ f $ er 2 på intervallet $ [0, 5] $.

\newpage
\subsection*{Oppgave 5}
\abc{
\item Av nullpunktene vet vi at vi kan skrive $ f(x)=a(x+3)(x-4) $. Videre har vi at
\alg{
	f(0)&= a(0+3)(0-4) \\
	24 &= -12a \\
	a &= -2
}
Dermed er $ f(x)=-2(x+3)(x-4) $
\item
\alg{
-2(x+3)(x-4) &> 12 \\
(x+3)(x-4) &< -6 \\
x^2-4x+3x-12+6 &< 0\\
x^2-x-6&<0
}
Siden $ (-3)\cdot2=6 $ og $ -3+2=-1 $, har vi at
\alg{
(x-3)(x+2)< 0
}
Av fortegnsskjemaet ser vi at ulikheten over er oppfylt når \\$ x\in(-2, 3) $.
\localfig{opg5bd1}{1}
}
\newpage
\subsubsection{Oppgave 1}
\abc{
\item Skriver tallene inn i regnearket i GeoGebra, lager liste med punkt, og bruker regresjon med andregradspolynom. Får da $ O(x) $ som samsvarer med $ O(x) $ i oppgaven. I grafikkfeltet ser vi at grafen til $ O $ (den grønne kurven) tilnærmet skjærer alle punktene, og derfor er en god modell.
\item Det største overskuddet får vi i toppunktet til $ O $, som vi finner ved kommandoen \texttt{Ekstremalpunkt(O)}. Da får vi at det største overskuddet oppstår ved å selge 282-283 baguetter i uka.
\item Vi skriver inn punktene som $ I $ og $ J $, finner linja mellom dem med kommandoen \texttt{Linje(I, J)}. Da får vi at stigningstallet til linja er $ 23.96 $. Dette betyr at på intervallet $ [100, 200] $, så har overskuddet i gjennomsnitt endret seg med $ 23.96 $ kroner per solgte baguett.
\item Finner den momentane vekstfarten ved å skrive \texttt{O'(235)}, som gir at $ O(235)=8.61 $. Dette betyr at akkurat når salget nådde 235 baguetter, så endret overskuddet seg meg $ 8.61 $ kronger per solgte baguett.
}
\localfig{opg1}{0.25}

\newpage
\subsection*{Oppgave 2}
\abc{
\item Vinkelen må være ca. $ 59^\circ $ (celle 1).
\item Når $ u $ går mot $ 90^\circ $, går $ v $ mot ca. $ 48.75^\circ $ (celle 2).
\localfig{opg2}{0.4}
\item Si at $ u=v=t $. For $ \sin t\neq 0 $ har vi at
\alg{
\sin t &= 1.33\sin t \\
1 &= 1.33
} 
Dette fører altså til en selvmotsigelse. Hvis derimot $ \sin t=0^\circ $, er ligningen over oppfylt. Dermed er $ u $ og $ v $ bare like når $ t= \text{asind}(0) $, som gir at $ t=0^\circ $.
}

\subsection*{Oppgave 3}
Bruker arealsetningen, og finner at $ AC=2 $ (celle 1). Bruker cosinussetningen, og finner at $ BC= 2\sqrt{21} $ (celle 2).
\localfig{opg3}{0.5}

\newpage
\subsection*{Oppgave 4}
\abc{
\item Oddetall nr $ i $ er gitt ved formelen $ 2i-1 $.
\pythonut{opg4.py}{
	S1: 1\\
	S2: 4\\
	S3: 9\\
	S4: 16\\
	S5: 25\\
	S6: 36\\
	S7: 49\\
	S8: 64\\
	S9: 81\\
	S10: 100\\
	S11: 121\\
	S12: 144\\
	S13: 169\\
	S14: 196\\
	S15: 225\\
	S16: 256\\
	S17: 289\\
	S18: 324\\
	S19: 361\\
	S20: 400
}
\item At summene ser vi at $ S_i=i^2 $. Summen av $ i $ oddetall danner et kvadrat med lengde $ i $, og dermed er summen lik $ i^2 $.
}

\newpage
\subsection*{Oppgave 5}
\abc{
\item Skriver inn tabellen i rengearket i GeoGebra, lager liste med punkt, og bruker regresjon med potensfunksjon. Får da
\[ K(x)=7.56x^{0.38} \]
\item Aris modell:
\[ f(x)=1000\cdot0.88^x \]
der $ f $ er lufttrykket og $ x $ er km over havet. \os
Lisas modell:
\[ g(x)=1000\cdot0.5^\frac{x}{5.5} \]
der $ g $ er lufttrykket og $ x $ er km over havet.
\item Ved å skrive \texttt{Skjæring(K, y=85)}, finner vi at lufttrykket har verdi $ 578.34 $ når kokepunktet er $ 85^\circ $. Vi skriver så modellene til Ari og Lisa inn i GeoGebra, og finner at begge modellene gir en høyde på ca. 4.3 km over havet.
}
\localfig{opg5ac}{0.3}
\localfig{opg5c}{0.3}

\newpage
\subsection*{Oppgave 6}
Vi definerer $ g(x)=ax+b $ som tangeringslinja til $ f $ i punktet $ P=(1, 2) $. Av figuren ser vi at for $ x=1 $ er stigningstallet til $ g $ lik $ -2 $. Siden grafen til $ g $ skjærer grafen til $ f $ i $ P $, har vi at $ g(1)=2 $. Ved å løse denne ligningen får vi at
\[ g(x)=-2x+4 \] 
\localfig{opg6}{0.4}

\subsection*{Oppgave 7}
Ut ifra figur og krav ser vi at det kan passe med en tredjegradsfunksjon, en andregradsfunksjon og en lineær funksjon. Vi setter
\[ f(x)=ax^3+bx^2+cx+d \]
I bunnpunktet til en funksjon er den deriverte lik 0, og da er
\[ f'(x)=3ax^2+2bx+c=0 \]
Siden bunnpunktet ligger på $ y $-aksen har vi at
\alg{
f'(0) &= c =0
}
Vi velger oss derfor
\[ f(x)=x^2(x+2) \]
Da har $ (0, f(0)) $ og $ (-2, f(-2)) $ samme $ y $-verdi, nemlig 0. For at andregradsfunksjonen vår skal ha bunnpunkt i $ x=0 $, må vi også kreve at leddet proporsjonalt med $ x $ er lik 0 (på samme måte som vi fikk $ c=0 $ for $ f $). For at den i tillegg skal ha $ (-2, 0) $ på grafen, tar vi med faktoren $ x+2 $. Disse to kravene gir oss at 
\[ g(x)=(x+2)(x-2) \]
\newpage
Til slutt finner vi linja mellom to punkt på grafene til $ f $ og $ g $, og får funksjonen
\[ h(x)=-3x+6 \]
\localfig{opg7}{0.25}
\end{document}