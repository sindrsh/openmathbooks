\documentclass[english,hidelinks, 11 pt, class=report,crop=false]{standalone}
\usepackage[T1]{fontenc}
%\usepackage[utf8]{inputenc}
\usepackage{lmodern} % load a font with all the characters
\usepackage{geometry}
\geometry{verbose,paperwidth=16.1 cm, paperheight=24 cm, inner=2.3cm, outer=1.8 cm, bmargin=2cm, tmargin=1.8cm}
\setlength{\parindent}{0bp}
\usepackage{import}
\usepackage[subpreambles=false]{standalone}
\usepackage{amsmath}
\usepackage{amssymb}
\usepackage{esint}
\usepackage{babel}
\usepackage{tabu}
\makeatother
\makeatletter

\usepackage{titlesec}
\usepackage{ragged2e}
\RaggedRight
\raggedbottom
\frenchspacing

\usepackage{graphicx}
\usepackage{float}
\usepackage{subfig}
\usepackage{placeins}
\usepackage{cancel}
\usepackage{framed}
\usepackage{wrapfig}
\usepackage[subfigure]{tocloft}
\usepackage[font=footnotesize,labelfont=sl]{caption} % Figure caption
\usepackage{bm}
\usepackage[dvipsnames, table]{xcolor}
\definecolor{shadecolor}{rgb}{0.105469, 0.613281, 1}
\colorlet{shadecolor}{Emerald!15} 
\usepackage{icomma}
\makeatother
\usepackage[many]{tcolorbox}
\usepackage{multicol}
\usepackage{stackengine}

\usepackage{esvect} %For vectors with capital letters

% For tabular
\usepackage{array}
\usepackage{multirow}
\usepackage{longtable} %breakable table

% Ligningsreferanser
\usepackage{mathtools} % for mathclap
%\mathtoolsset{showonlyrefs}

% sections without numbering in toc
\newcommand\tsec[1]{\phantomsection \addcontentsline{toc}{section}{#1}
	\section*{#1}}

% index
\usepackage{imakeidx}
\makeindex[title=Indeks]

%Footnote:
\usepackage[bottom, hang, flushmargin]{footmisc}
\usepackage{perpage} 
\MakePerPage{footnote}
\addtolength{\footnotesep}{2mm}
\renewcommand{\thefootnote}{\arabic{footnote}}
\renewcommand\footnoterule{\rule{\linewidth}{0.4pt}}
\renewcommand{\thempfootnote}{\arabic{mpfootnote}}

%colors
\definecolor{c1}{cmyk}{0,0.5,1,0}
\definecolor{c2}{cmyk}{1,0.25,1,0}
\definecolor{n3}{cmyk}{1,0.,1,0}
\definecolor{neg}{cmyk}{1,0.,0.,0}


\newcommand{\nreq}[1]{
\begin{equation}
	#1
\end{equation}
}


% Equation comments
\newcommand{\cm}[1]{\llap{\color{blue} #1}}


\usepackage[inline]{enumitem}
\newcounter{rg}
\numberwithin{rg}{chapter}


\newcommand{\reg}[2][]{\begin{tcolorbox}[boxrule=0.3 mm,arc=0mm,colback=blue!3] {\refstepcounter{rg}\phantomsection \large \textbf{\therg \;#1} \vspace{5 pt}}\newline #2  \end{tcolorbox}\vspace{-5pt}}
\newcommand{\regdef}[2][]{\begin{tcolorbox}[boxrule=0.3 mm,arc=0mm,colback=blue!3] {\refstepcounter{rg}\phantomsection \large \textbf{\therg \;#1} \vspace{5 pt}}\newline #2  \end{tcolorbox}\vspace{-5pt}}
\newcommand{\words}[1]{\begin{tcolorbox}[boxrule=0.3 mm,arc=0mm,colback=teal!3] #1  \end{tcolorbox}\vspace{-5pt}}

\newcommand\alg[1]{\begin{align*} #1 \end{align*}}

\newcommand\eks[2][]{\begin{tcolorbox}[boxrule=0.3 mm,arc=0mm,enhanced jigsaw,breakable,colback=green!3] {\large \textbf{\ekstitle #1} \vspace{5 pt}\\} #2 \end{tcolorbox}\vspace{-5pt} }

\newcommand{\st}[1]{\begin{tcolorbox}[boxrule=0.0 mm,arc=0mm,enhanced jigsaw,breakable,colback=yellow!12]{ #1} \end{tcolorbox}}

\newcommand{\spr}[1]{\begin{tcolorbox}[boxrule=0.3 mm,arc=0mm,enhanced jigsaw,breakable,colback=yellow!7] {\large \textbf{\sprtitle} \vspace{5 pt}\\} #1 \end{tcolorbox}\vspace{-5pt} }

\newcommand{\sym}[1]{\colorbox{blue!15}{#1}}

\newcommand{\info}[2]{\begin{tcolorbox}[boxrule=0.3 mm,arc=0mm,enhanced jigsaw,breakable,colback=cyan!6] {\large \textbf{#1} \vspace{5 pt}\\} #2 \end{tcolorbox}\vspace{-5pt} }

\newcommand\algv[1]{\vspace{-11 pt}\begin{align*} #1 \end{align*}}

\newcommand{\regv}{\vspace{5pt}}
\newcommand{\mer}{\textsl{\note}: }
\newcommand{\mers}[1]{{\footnotesize \mer #1}}
\newcommand\vsk{\vspace{11pt}}
\newcommand{\tbs}{\vspace{5pt}}
\newcommand\vs{\vspace{-11pt}}
\newcommand\vsb{\vspace{-16pt}}
\newcommand\br{\\[5 pt]}
\newcommand{\figp}[1]{../fig/#1}
\newcommand\algvv[1]{\vs\vs\begin{align*} #1 \end{align*}}
\newcommand{\y}[1]{$ {#1} $}
\newcommand{\os}{\\[5 pt]}
\newcommand{\prbxl}[2]{
\parbox[l][][l]{#1\linewidth}{#2
	}}
\newcommand{\prbxr}[2]{\parbox[r][][l]{#1\linewidth}{
		\setlength{\abovedisplayskip}{5pt}
		\setlength{\belowdisplayskip}{5pt}	
		\setlength{\abovedisplayshortskip}{0pt}
		\setlength{\belowdisplayshortskip}{0pt} 
		\begin{shaded}
			\footnotesize	#2 \end{shaded}}}
\newcommand{\fgbxr}[2]{
	\parbox[r][][l]{#1\linewidth}{#2
}}		

\renewcommand{\cfttoctitlefont}{\Large\bfseries}
\setlength{\cftaftertoctitleskip}{0 pt}
\setlength{\cftbeforetoctitleskip}{0 pt}

\newcommand{\bs}{\\[3pt]}
\newcommand{\vn}{\\[6pt]}
\newcommand{\fig}[1]{\begin{figure}[H]
		\centering
		\includegraphics[]{\figp{#1}}
\end{figure}}

\newcommand{\figc}[2]{\begin{figure}
		\centering
		\includegraphics[]{\figp{#1}}
		\caption{#2}
\end{figure}}
\newcommand{\arc}[1]{{
		\setbox9=\hbox{#1}%
		\ooalign{\resizebox{\wd9}{\height}{\texttoptiebar{\phantom{A}}}\cr\textit{#1}}}}

\newcommand{\sectionbreak}{\clearpage} % New page on each section

\newcommand{\nn}[1]{
\begin{equation*}
	#1
\end{equation*}
}

\newcommand{\enh}[1]{\,\textrm{#1}}

%asin, atan, acos
\DeclareMathOperator{\atan}{atan}
\DeclareMathOperator{\acos}{acos}
\DeclareMathOperator{\asin}{asin}

% Comments % old cm, ggb cm is new
%\newcommand{\cm}[1]{\llap{\color{blue} #1}}

%%%

\newcommand\fork[2]{\begin{tcolorbox}[boxrule=0.3 mm,arc=0mm,enhanced jigsaw,breakable,colback=yellow!7] {\large \textbf{#1 (\expl)} \vspace{5 pt}\\} #2 \end{tcolorbox}\vspace{-5pt} }
 
%colors
\newcommand{\colr}[1]{{\color{red} #1}}
\newcommand{\colb}[1]{{\color{blue} #1}}
\newcommand{\colo}[1]{{\color{orange} #1}}
\newcommand{\colc}[1]{{\color{cyan} #1}}
\definecolor{projectgreen}{cmyk}{100,0,100,0}
\newcommand{\colg}[1]{{\color{projectgreen} #1}}

% Methods
\newcommand{\metode}[2]{
	\textsl{#1} \\[-8pt]
	\rule{#2}{0.75pt}
}

%Opg
\newcommand{\abc}[1]{
	\begin{enumerate}[label=\alph*),leftmargin=18pt]
		#1
	\end{enumerate}
}
\newcommand{\abcs}[2]{
	\begin{enumerate}[label=\alph*),start=#1,leftmargin=18pt]
		#2
	\end{enumerate}
}
\newcommand{\abcn}[1]{
	\begin{enumerate}[label=\arabic*),leftmargin=18pt]
		#1
	\end{enumerate}
}
\newcommand{\abch}[1]{
	\hspace{-2pt}	\begin{enumerate*}[label=\alph*), itemjoin=\hspace{1cm}]
		#1
	\end{enumerate*}
}
\newcommand{\abchs}[2]{
	\hspace{-2pt}	\begin{enumerate*}[label=\alph*), itemjoin=\hspace{1cm}, start=#1]
		#2
	\end{enumerate*}
}

% Exercises


\newcounter{opg}
\numberwithin{opg}{section}

\newcounter{grub}
\numberwithin{opg}{section}
\newcommand{\op}[1]{\vspace{15pt} \refstepcounter{opg}\large \textbf{\color{blue}\theopg} \vspace{2 pt} \label{#1} \\}
\newcommand{\eksop}[2]{\vspace{15pt} \refstepcounter{opg}\large \textbf{\color{blue}\theopg} (#1) \vspace{2 pt} \label{#2} \\}

\newcommand{\nes}{\stepcounter{section}
	\setcounter{opg}{0}}
\newcommand{\opr}[1]{\vspace{3pt}\textbf{\ref{#1}}}
\newcommand{\oeks}[1]{\begin{tcolorbox}[boxrule=0.3 mm,arc=0mm,colback=white]
		\textit{\ekstitle: } #1	  
\end{tcolorbox}}
\newcommand\opgeks[2][]{\begin{tcolorbox}[boxrule=0.1 mm,arc=0mm,enhanced jigsaw,breakable,colback=white] {\footnotesize \textbf{\ekstitle #1} \\} \footnotesize #2 \end{tcolorbox}\vspace{-5pt} }


% tag exercises
\newcommand{\tagop}[1]{ 
{\small \color{Gray} #1} \os
}

% License
\newcommand{\lic}{
This book is part of the \net{https://sindrsh.github.io/openmathbooks/}{OpenMathBooks} project. OpenMathBooks © 2022 by Sindre Sogge Heggen is licensed under CC BY-NC-SA 4.0. To view a copy of this license, visit \net{http://creativecommons.org/licenses/by-nc-sa/4.0/}{http://creativecommons.org/licenses/by-nc-sa/4.0/}}

%referances
\newcommand{\net}[2]{{\color{blue}\href{#1}{#2}}}
\newcommand{\hrs}[2]{\hyperref[#1]{\color{blue}#2 \ref*{#1}}}
\newcommand{\refunnbr}[2]{\hyperref[#1]{\color{blue}#2}}


\newcommand{\openmath}{\net{https://sindrsh.github.io/openmathbooks/}{OpenMathBooks}}
\newcommand{\am}{\net{https://sindrsh.github.io/FirstPrinciplesOfMath/}{AM1}}
\newcommand{\mb}{\net{https://sindrsh.github.io/FirstPrinciplesOfMath/}{MB}}
\newcommand{\tmen}{\net{https://sindrsh.github.io/FirstPrinciplesOfMath/}{TM1}}
\newcommand{\tmto}{\net{https://sindrsh.github.io/FirstPrinciplesOfMath/}{TM2}}
\newcommand{\amto}{\net{https://sindrsh.github.io/FirstPrinciplesOfMath/}{AM2}}
\newcommand{\eksbm}{
\footnotesize
Dette er opppgaver som har blitt gitt ved sentralt utformet eksamen i Norge. Oppgavene er laget av Utdanningsdirektoratet. Forkortelser i parantes viser til følgende:
\begin{center}
	\begin{tabular}{c|c}
		E & Eksempeloppgave \\
		V/H & Eksamen fra vårsemesteret/høstsemesteret\\
		G/1P/1T/R1/R2 & Fag  \\
		XX & År 20XX \\
		D1/D2 & Del 1/Del 2
	\end{tabular}
\end{center}
Tekst og innhold kan her være noe endret i forhold til originalen.
}

%Excel og GGB:

\newcommand{\g}[1]{\begin{center} {\tt #1} \end{center}}
\newcommand{\gv}[1]{\begin{center} \vspace{-11 pt} {\tt #1}  \end{center}}
\newcommand{\cmds}[2]{{\tt #1}\\
	#2}

% outline word
\newcommand{\outl}[1]{{\boldmath \color{teal}\textbf{#1}}}
%line to seperate examples
\newcommand{\linje}{\rule{\linewidth}{1pt} }


%Vedlegg
\newcounter{vedl}
\newcounter{vedleq}
\renewcommand\thevedl{\Alph{vedl}}	
\newcommand{\nreqvd}{\refstepcounter{vedleq}\tag{\thevedl \thevedleq}}

%%% Writing code

\usepackage{listings}


\definecolor{codegreen}{rgb}{0,0.6,0}
\definecolor{codegray}{rgb}{0.5,0.5,0.5}
\definecolor{codepurple}{rgb}{0.58,0,0.82}
\definecolor{backcolour}{rgb}{0.95,0.95,0.92}

\newcommand{\pymet}[1]{{\ttfamily\color{magenta} #1}}
\newcommand{\pytype}[1]{{\ttfamily\color{codepurple} #1}}

\lstdefinestyle{mystyle}{
	backgroundcolor=\color{backcolour},   
	commentstyle=\color{codegreen},
	keywordstyle=\color{magenta},
	numberstyle=\tiny\color{codegray},
	stringstyle=\color{codepurple},
	basicstyle=\ttfamily\footnotesize,
	breakatwhitespace=false,         
	breaklines=true,                 
	captionpos=b,                    
	keepspaces=true,                 
	numbers=left,                    
	numbersep=5pt,                  
	showspaces=false,                
	showstringspaces=false,
	showtabs=false,                  
	tabsize=2,
	inputencoding=utf8,
	extendedchars=true,
	literate= {
		{å}{{\aa}}1 
		{æ}{{\ae}}1 
		{ø}{{\o}}1
	}
}

\lstset{style=mystyle}

\newcommand{\python}[1]{
\begin{tcolorbox}[boxrule=0.3 mm,arc=0mm,colback=white]
\lstinputlisting[language=Python]{#1}
\end{tcolorbox}}
\newcommand{\pythonut}[2]{
\begin{tcolorbox}[boxrule=0.3 mm,arc=0mm,colback=white]
\small 
%\textbf{Kode}
\lstinputlisting[language=Python]{#1}	
\vspace{11pt}
\textbf{Utdata} \\ \ttfamily
#2
\end{tcolorbox}}
%%%

%page number
%\usepackage{fancyhdr}
%\pagestyle{fancy}
%\fancyhf{}
%\renewcommand{\headrule}{}
%\fancyhead[RO, LE]{\thepage}

\usepackage{datetime2}
%%\usepackage{sansmathfonts} for dyslexia-friendly math
\usepackage[]{hyperref}




\begin{document}
{\Large Eksamen 1T Våren 2024\hfill {\footnotesize Løsning fra \color{blue} \href{https://sindreheggen.wordpress.com/}{OpenMathBooks prosjektet}}}	
\subsection*{Oppgave 1}	
\abc{
\item $ \tan u = \frac{6}{8} $ og $ \tan v = \frac{8}{6} $, dermed er $ \tan u\cdot \tan v =\frac{6}{8}\cdot\frac{8}{6}=1 $. 
\item Vi setter $ a $ og $ b $ som vist i figuren under. Da er $ {\tan u= \frac{a}{b}} $ og $ {\tan v = \frac{b}{a}} $. Følgelig er
\[ \tan u \cdot \tan v = \frac{a}{b}\cdot\frac{b}{a}=1 \]
Altså gjelder dette for alle rettvinklede trekanter.
\localfig{opg1bd1}{1}
}

\subsection*{Oppgave 2}
Hun kan ha utført polynomdivisjon på regnestykket $ \frac{2x^3+3x^2-11x-6}{2x^2+7x+3}  $ eller $ \frac{2x^3+3x^2-11x-6}{x-2}  $.
\alg{
	\phantom{-}&(2x^3+3x^2-11x-6):(2x^2+7x+3) = x-2\\ 
	-&\underline{(2x^3+7x^2+3x)} \\
	&\phantom{000\,\,x\;}10x^2-14x-6  \\
	&\phantom{xx}-\underline{(10x^2-14x-6)}\\
	&\phantom{aaaaaaaaaaaaaaaa''}0 \\
}
\alg{
	\phantom{-}&(2x^3+3x^2-11x-6):(x-2) = 2x^2+7x+3\\ 
	-&\underline{(2x^3-4x^2)} \\
	&\phantom{000\,\,x-}7x^2-11x-6  \\
	&\phantom{xxx}-\underline{(7x^2-14x)}\\
	&\phantom{aaaaaaaaaaa''\,}3x-6 \\
	&\phantom{aaaaaaaax}-\underline{(3x-6)}\\
	&\phantom{aaaaaaaaaaaaaaaaa} 0
}
Polynomdivisjonene viser at både $ x-2 $ og $ 2x^2+7x+3 $ er faktorer i $ 2x^3+2x^2-11x-6 $.

\subsection*{Oppgave 3}
Vi setter to uttrykk for arealet til det grønne området lik hverandre:
\alg{
a(a-b+b)-b^2\cdot &= a\cdot(a-b) + b(a-b) \\
a^2-b^2&= (a+b)(a-b)
}


\subsection*{Oppgave 4}
\alg{
f(0) &= 0^2-3\cdot0+7 = 7\\
f(5) &= 5^2-3\cdot5+7 = 17\\
\frac{f(5)-f(0)}{5-0}&=\frac{17-7}{5}=2
}
Verdien 2 vil bli skrevet ut, og dette forteller at den gjennomsnittlige endringen til $ f $ er 2 på intervallet $ [0, 5] $.

\newpage
\subsection*{Oppgave 5}
\abc{
\item Av nullpunktene vet vi at vi kan skrive $ f(x)=a(x+3)(x-4) $. Videre har vi at
\alg{
	f(0)&= a(0+3)(0-4) \\
	24 &= -12a \\
	a &= -2
}
Dermed er $ f(x)=-2(x+3)(x-4) $
\item
\alg{
-2(x+3)(x-4) &> 12 \\
(x+3)(x-4) &< -6 \\
x^2-4x+3x-12+6 &< 0\\
x^2-x-6&<0
}
Siden $ (-3)\cdot2=6 $ og $ -3+2=-1 $, har vi at
\alg{
(x-3)(x+2)< 0
}
Av fortegnsskjemaet ser vi at ulikheten over er oppfylt når \\$ x\in(-2, 3) $.
\localfig{opg5bd1}{1}
}
\newpage
\subsubsection{Oppgave 1}
\abc{
\item Skriver tallene inn i regnearket i GeoGebra, lager liste med punkt, og bruker regresjon med andregradspolynom. Får da $ O(x) $ som samsvarer med $ O(x) $ i oppgaven. I grafikkfeltet ser vi at grafen til $ O $ (den grønne kurven) tilnærmet skjærer alle punktene, og derfor er en god modell.
\item Det største overskuddet får vi i toppunktet til $ O $, som vi finner ved kommandoen \texttt{Ekstremalpunkt(O)}. Da får vi at det største overskuddet oppstår ved å selge 282-283 baguetter i uka.
\item Vi skriver inn punktene som $ I $ og $ J $, finner linja mellom dem med kommandoen \texttt{Linje(I, J)}. Da får vi at stigningstallet til linja er $ 23.96 $. Dette betyr at på intervallet $ [100, 200] $, så har overskuddet i gjennomsnitt endret seg med $ 23.96 $ kroner per solgte baguett.
\item Finner den momentane vekstfarten ved å skrive \texttt{O'(235)}, som gir at $ O(235)=8.61 $. Dette betyr at akkurat når salget nådde 235 baguetter, så endret overskuddet seg meg $ 8.61 $ kronger per solgte baguett.
}
\localfig{opg1}{0.25}

\newpage
\subsection*{Oppgave 2}
\abc{
\item Vinkelen må være ca. $ 59^\circ $ (celle 1).
\item Når $ u $ går mot $ 90^\circ $, går $ v $ mot ca. $ 48.75^\circ $ (celle 2).
\localfig{opg2}{0.4}
\item Si at $ u=v=t $. For $ \sin t\neq 0 $ har vi at
\alg{
\sin t &= 1.33\sin t \\
1 &= 1.33
} 
Dette fører altså til en selvmotsigelse. Hvis derimot $ \sin t=0^\circ $, er ligningen over oppfylt. Dermed er $ u $ og $ v $ bare like når $ t= \text{asind}(0) $, som gir at $ t=0^\circ $.
}

\subsection*{Oppgave 3}
Bruker arealsetningen, og finner at $ AC=2 $ (celle 1). Bruker cosinussetningen, og finner at $ BC= 2\sqrt{21} $ (celle 2).
\localfig{opg3}{0.5}

\newpage
\subsection*{Oppgave 4}
\abc{
\item Oddetall nr $ i $ er gitt ved formelen $ 2i-1 $.
\pythonut{opg4.py}{
	S1: 1\\
	S2: 4\\
	S3: 9\\
	S4: 16\\
	S5: 25\\
	S6: 36\\
	S7: 49\\
	S8: 64\\
	S9: 81\\
	S10: 100\\
	S11: 121\\
	S12: 144\\
	S13: 169\\
	S14: 196\\
	S15: 225\\
	S16: 256\\
	S17: 289\\
	S18: 324\\
	S19: 361\\
	S20: 400
}
\item At summene ser vi at $ S_i=i^2 $. Summen av $ i $ oddetall danner et kvadrat med lengde $ i $, og dermed er summen lik $ i^2 $.
}

\newpage
\subsection*{Oppgave 5}
\abc{
\item Skriver inn tabellen i rengearket i GeoGebra, lager liste med punkt, og bruker regresjon med potensfunksjon. Får da
\[ K(x)=7.56x^{0.38} \]
\item Aris modell:
\[ f(x)=1000\cdot0.88^x \]
der $ f $ er lufttrykket og $ x $ er km over havet. \os
Lisas modell:
\[ g(x)=1000\cdot0.5^\frac{x}{5.5} \]
der $ g $ er lufttrykket og $ x $ er km over havet.
\item Ved å skrive \texttt{Skjæring(K, y=85)}, finner vi at lufttrykket har verdi $ 578.34 $ når kokepunktet er $ 85^\circ $. Vi skriver så modellene til Ari og Lisa inn i GeoGebra, og finner at begge modellene gir en høyde på ca. 4.3 km over havet.
}
\localfig{opg5ac}{0.3}
\localfig{opg5c}{0.3}

\newpage
\subsection*{Oppgave 6}
Da $ f' $ er en lineær funksjon, må $ f $ være en andregradsfunksjon. I celle 1 definerer vi en generell andregradsfunksjon. Ut ifra figuren ser vi at funksjonen må oppfylle at $ g'(2)=0 $ og $ g'(4)=4 $. Dette gir oss et ligningssett for $ a $ og $ b $, som er løst i celle 4. Vi definerer en ny funksjon $ h $ med de gitte verdiene for $ a $ og $ b $, og løser ligningen som er gitt av at punktet $ (1, 2) $ ligger på grafen til $ f $ (celle 5 og 4). I algebrafeltet skriver vi inn $ f $ med de gitte verdiene for $ a $, $ b $ og $ c $, og ser videre at grafen til $ f' $ stemmer overens med grafen gitt i oppgaven.
 
\localfig{opg6}{0.3}
\end{document}