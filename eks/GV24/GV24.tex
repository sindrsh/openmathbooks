\documentclass[english,hidelinks,pdftex, 11 pt, class=report,crop=false]{standalone}
\usepackage[T1]{fontenc}
\usepackage[utf8]{luainputenc}
\usepackage{lmodern} % load a font with all the characters
\usepackage{geometry}
\geometry{verbose,a4paper, inner=0cm, outer=0 cm, bmargin=2cm, tmargin=1cm}
%\textwidth=12cm
\setlength{\parindent}{0bp}
\usepackage{import}
\usepackage[subpreambles=false]{standalone}
\usepackage{amsmath}
\usepackage{amssymb}
\usepackage{esint}
\usepackage{babel}
\usepackage{tabu}
\usepackage[dvipsnames, table]{xcolor}
\usepackage{cancel}
\makeatother
\makeatletter
\usepackage{datetime2}
\usepackage{titlesec}
\usepackage[many]{tcolorbox}

% Eheter
\newcommand{\enh}[1]{\,\textrm{#1}}
%referances
\newcommand{\net}[2]{{\color{blue}\href{#1}{#2}}}

%Spaces
\newcommand{\vsk}{\\[12pt]}
\newcommand{\vs}{\vspace{-12pt}}

% Tabell for opplegg

\newcommand{\ovlist}[1]{
\vspace{-16pt}
\begin{itemize}
	#1
\end{itemize}
}

% Chapters and sections
\titleformat{\section}[block]{\bfseries}{\hspace{3cm}\thesection}{5pt}{}
\titleformat{\subsection}[block]{\bfseries}{\hspace{3cm}\thesection}{5pt}{}
\newcommand{\sectionbreak}{\clearpage} % New page on each section
 

\newlength{\mywidth}
\setlength{\mywidth}{14cm}

\newcommand{\cont}[1]{
\begin{tcolorbox}[center, boxrule=0.0 mm, width=\mywidth,arc=0mm,enhanced jigsaw,,colback=white,breakable]
#1	
\end{tcolorbox}
}

\newcommand{\info}[5]{
\begin{tcolorbox}[center, boxrule=0.1 mm, width=\mywidth,arc=0mm,enhanced jigsaw,breakable,colback=yellow!5]	
	
	\footnotesize
	\textbf{Øvingsområde}\\[5pt] #1 
	
	\textbf{Utstyr}\\ #2  \\
	
	\begin{tabular}{@{} p{4cm} p{4cm} l} 
		\textbf{Tid} & \textbf{Elevinndeling} & \textbf{Læringsarena} \\
		#3  & #4 & #5
	\end{tabular} 
\end{tcolorbox}	
}

\newcommand{\gjen}[1]{\begin{tcolorbox}[center,boxrule=0.1 mm, width=\mywidth,arc=0mm,colback=blue!3] {\large \textbf{Gjennomføring} \vspace{5 pt}}\newline #1  \end{tcolorbox}\vspace{-5pt}}
\newcommand{\eks}[1]{\begin{tcolorbox}[center,boxrule=0.1 mm, width=\mywidth,arc=0mm,colback=green!3] {\large \textbf{Eksempel} \vspace{5 pt}}\newline #1  \end{tcolorbox}\vspace{-5pt}}

\newcounter{opl}
%\numberwithin{opl}{article}


\newcommand{\opl}[1]{
\newpage
{\refstepcounter{opl} %\phantomsection 
\large \textbf{\theopl \;#1} \vsk}
}

% Headlines
\newcommand{\fork}{\textbf{Forkunnskapar}\\}
\newcommand{\forb}{\textbf{Forberedelsar}\\}
\newcommand{\opgvr}{\textbf{Oppgaver}}



%colors
\newcommand{\colr}[1]{{\color{red} #1}}
\newcommand{\colb}[1]{{\color{blue} #1}}
\newcommand{\colo}[1]{{\color{orange} #1}}
\newcommand{\colc}[1]{{\color{cyan} #1}}
\definecolor{projectgreen}{cmyk}{100,0,100,0}
\newcommand{\colg}[1]{{\color{projectgreen} #1}}

% Lister med bokstavar
\usepackage[inline]{enumitem}
% Opg
\newcommand{\abc}[1]{
	\begin{enumerate}[label=\alph*),leftmargin=18pt]
		#1
	\end{enumerate}
}

\usepackage[]{hyperref}



\begin{document}
{\Large Eksamen Matematikk 10. årstrinn våren 2024\hfill \\ {\footnotesize Løsning fra \color{blue} \href{https://sindreheggen.wordpress.com/}{OpenMathBooks prosjektet}}}	

\subsection*{Oppgave 1}
\alg{
\text{Pris for voksen} = x && \text{Pris for barn} = y
}
\begin{align}
	2x + y = 400 \label{opg1a}\\
	x + y = 260 \label{opg1b}
\end{align}
Vi trekker \eqref{opg1b} i fra \eqref{opg1a}:
\alg{
2x+y-(x+y) &= 400 -260\\
x &= 140
}
En barnebillett koster 140 kroner.

\subsection*{Oppgave 2}
\abc{
\item \phantom{ }\\ \begin{tabular}{c|m{1.5cm}|m{1.5cm}|m{1.5cm}|m{1.5cm}}
	& Figur 1 & Figur 2 & Figur 3 & Figur 4 \\ \hline
Tegning av figur	&\includegraphics[]{opg2d1_1} & \includegraphics[]{opg2d1_2} & \includegraphics[]{opg2d1_3} & \begin{figure}
		\includegraphics[]{opg2d1_4}
	\end{figure} \\ \hline
	Antall ruter i figuren &\centering 2 &\centering 6 &\centering 12 & \centering 20
\end{tabular}
\item Rutene utgjør et rektangel med bredde $ n $ og høgde $ n-1 $. Altså er det $ n(n-1) $ ruter i figuren. 
}

\subsection*{Oppgave 3}
\begin{tabular}{l|c|c}
\textbf{Påstand} & \textbf{Sann} & \textbf{Usann} \\ \hline
Lese bøker er det ungdommer bruker mest tid på & & x	\\ \hline 
Det er flere som spiller dataspill/TV-spill & & x \\ 
enn som ser på TV & &\\ \hline
Nesten tre firedeler av ungdommene spiller på & x & \\
telefon/nettbrett& &\\ \hline
Omtrent 40\% av ungdommene bruker mindre enn & & x \\
en time daglig på sosiale medier & &
\end{tabular}


\subsection*{Oppgave 4}
Både punktet $ (10, 145) $ og $ (16, 169) $ ser ut til å ligge på linja.
\[ \frac{169-145}{16-10}=\frac{24}{6}=4 \]
Stigningstallet er 4. Dette betyr at Kristin i gjennomsnitt har vokst 4 cm i året.

\subsection*{Oppgave 5}
Løsningen til Ruben er ikke riktig. For å finne avslaget i prosent, må man dele avslaget på originalprisen. Rugen har i stedet delt på den nye prisen, og da blir ikke svaret rett.

\subsection*{Oppgave 6}
Foreslår en matboks med lengde 10\enh{cm}, bredde 8\enh{cm} og høyde 7\enh{cm}. $ 1\enh{cm}=0,1\enh{dm} $ og $ 1\enh{dm}^3=1\enh{L} $. $ 10\cdot8\cdot7=560 $. Dette betyr at volumet er $ 0,56\enh{L} $.

\newpage
\subsection*{Oppgave 7}
\textbf{Løsningsmetode 1} \\
Arealet er lik $ (a+b)^2-b^2=a^2+2ab $
\begin{figure}
	\includegraphics[]{opg7d1_a}
\end{figure}

\textbf{Løsningsmetode 2}\\
Arealet er lik $ (a+b)a + ab = a^2+2ab$.
\begin{figure}
	\includegraphics[]{opg7d1_b}
\end{figure}

\newpage
\subsection*{Oppgave 1}
Ida har valgt å la $ y$-aksen starte på 0\,kr. Siden prisforskjellene er ganske små i forhold til den totale prisen, blir høydeforskjellen på søylene også ganske små. I tillegg blir intervallene mellom horisontallinjene store, dette gjør at det også er vanskelig å se verdiene til prisforskjellene.\vsk

Juan har valgt å la $ y $-aksen starte på 180\,kr. Da blir høydeforskjellen på søylene større og intervallene mellom horisontallinjene mindre. Dette gjør at det er lettere å se hvilken butikk som er billigst og hvor mye som skiller butikkene i pris.
\end{document}