\documentclass[english,hidelinks,pdftex, 11 pt, class=report,crop=false]{standalone}
\usepackage[T1]{fontenc}
\usepackage[utf8]{luainputenc}
\usepackage{lmodern} % load a font with all the characters
\usepackage{geometry}
\geometry{verbose,a4paper, inner=0cm, outer=0 cm, bmargin=2cm, tmargin=1cm}
%\textwidth=12cm
\setlength{\parindent}{0bp}
\usepackage{import}
\usepackage[subpreambles=false]{standalone}
\usepackage{amsmath}
\usepackage{amssymb}
\usepackage{esint}
\usepackage{babel}
\usepackage{tabu}
\usepackage[dvipsnames, table]{xcolor}
\usepackage{cancel}
\makeatother
\makeatletter
\usepackage{datetime2}
\usepackage{titlesec}
\usepackage[many]{tcolorbox}

% Eheter
\newcommand{\enh}[1]{\,\textrm{#1}}
%referances
\newcommand{\net}[2]{{\color{blue}\href{#1}{#2}}}

%Spaces
\newcommand{\vsk}{\\[12pt]}
\newcommand{\vs}{\vspace{-12pt}}

% Tabell for opplegg

\newcommand{\ovlist}[1]{
\vspace{-16pt}
\begin{itemize}
	#1
\end{itemize}
}

% Chapters and sections
\titleformat{\section}[block]{\bfseries}{\hspace{3cm}\thesection}{5pt}{}
\titleformat{\subsection}[block]{\bfseries}{\hspace{3cm}\thesection}{5pt}{}
\newcommand{\sectionbreak}{\clearpage} % New page on each section
 

\newlength{\mywidth}
\setlength{\mywidth}{14cm}

\newcommand{\cont}[1]{
\begin{tcolorbox}[center, boxrule=0.0 mm, width=\mywidth,arc=0mm,enhanced jigsaw,,colback=white,breakable]
#1	
\end{tcolorbox}
}

\newcommand{\info}[5]{
\begin{tcolorbox}[center, boxrule=0.1 mm, width=\mywidth,arc=0mm,enhanced jigsaw,breakable,colback=yellow!5]	
	
	\footnotesize
	\textbf{Øvingsområde}\\[5pt] #1 
	
	\textbf{Utstyr}\\ #2  \\
	
	\begin{tabular}{@{} p{4cm} p{4cm} l} 
		\textbf{Tid} & \textbf{Elevinndeling} & \textbf{Læringsarena} \\
		#3  & #4 & #5
	\end{tabular} 
\end{tcolorbox}	
}

\newcommand{\gjen}[1]{\begin{tcolorbox}[center,boxrule=0.1 mm, width=\mywidth,arc=0mm,colback=blue!3] {\large \textbf{Gjennomføring} \vspace{5 pt}}\newline #1  \end{tcolorbox}\vspace{-5pt}}
\newcommand{\eks}[1]{\begin{tcolorbox}[center,boxrule=0.1 mm, width=\mywidth,arc=0mm,colback=green!3] {\large \textbf{Eksempel} \vspace{5 pt}}\newline #1  \end{tcolorbox}\vspace{-5pt}}

\newcounter{opl}
%\numberwithin{opl}{article}


\newcommand{\opl}[1]{
\newpage
{\refstepcounter{opl} %\phantomsection 
\large \textbf{\theopl \;#1} \vsk}
}

% Headlines
\newcommand{\fork}{\textbf{Forkunnskapar}\\}
\newcommand{\forb}{\textbf{Forberedelsar}\\}
\newcommand{\opgvr}{\textbf{Oppgaver}}



%colors
\newcommand{\colr}[1]{{\color{red} #1}}
\newcommand{\colb}[1]{{\color{blue} #1}}
\newcommand{\colo}[1]{{\color{orange} #1}}
\newcommand{\colc}[1]{{\color{cyan} #1}}
\definecolor{projectgreen}{cmyk}{100,0,100,0}
\newcommand{\colg}[1]{{\color{projectgreen} #1}}

% Lister med bokstavar
\usepackage[inline]{enumitem}
% Opg
\newcommand{\abc}[1]{
	\begin{enumerate}[label=\alph*),leftmargin=18pt]
		#1
	\end{enumerate}
}

\usepackage[]{hyperref}



\begin{document}
{\Large Eksamen Matematikk 10. årstrinn våren 2024\hfill \\ {\footnotesize Løsning fra \color{blue} \href{https://sindreheggen.wordpress.com/}{OpenMathBooks prosjektet}}}	

\subsection*{Oppgave 1}
\alg{
\text{Pris for voksen} = x && \text{Pris for barn} = y
}
\begin{align}
	2x + y = 400 \label{opg1a}\\
	x + y = 260 \label{opg1b}
\end{align}
Vi trekker \eqref{opg1b} i fra \eqref{opg1a}:
\alg{
2x+y-(x+y) &= 400 -260\\
x &= 140
}
En barnebillett koster 140 kroner.

\subsection*{Oppgave 2}
\abc{
\item \phantom{ }\\ \begin{tabular}{c|m{1.5cm}|m{1.5cm}|m{1.5cm}|m{1.5cm}}
	& Figur 1 & Figur 2 & Figur 3 & Figur 4 \\ \hline
Tegning av figur	&\includegraphics[]{opg2d1_1} & \includegraphics[]{opg2d1_2} & \includegraphics[]{opg2d1_3} & \begin{figure}
		\includegraphics[]{opg2d1_4}
	\end{figure} \\ \hline
	Antall ruter i figuren &\centering 2 &\centering 6 &\centering 12 & \centering 20
\end{tabular}
\item Rutene utgjør et rektangel med bredde $ n $ og høgde $ n-1 $. Altså er det $ n(n-1) $ ruter i figuren. 
}

\subsection*{Oppgave 3}
\begin{tabular}{l|c|c}
\textbf{Påstand} & \textbf{Sann} & \textbf{Usann} \\ \hline
Lese bøker er det ungdommer bruker mest tid på & & x	\\ \hline 
Det er flere som spiller dataspill/TV-spill & & x \\ 
enn som ser på TV & &\\ \hline
Nesten tre firedeler av ungdommene spiller på & x & \\
telefon/nettbrett& &\\ \hline
Omtrent 40\% av ungdommene bruker mindre enn & & x \\
en time daglig på sosiale medier & &
\end{tabular}


\subsection*{Oppgave 4}
Både punktet $ (10, 145) $ og $ (16, 169) $ ser ut til å ligge på linja.
\[ \frac{169-145}{16-10}=\frac{24}{6}=4 \]
Stigningstallet er 4. Dette betyr at Kristin i gjennomsnitt har vokst 4 cm i året.

\subsection*{Oppgave 5}
Løsningen til Ruben er ikke riktig. For å finne avslaget i prosent, må man dele avslaget på originalprisen. Ruben har i stedet delt på den nye prisen, og da blir ikke svaret rett.

\subsection*{Oppgave 6}
Foreslår en matboks med lengde 10\enh{cm}, bredde 8\enh{cm} og høyde 7\enh{cm}. $ 1\enh{cm}=0,1\enh{dm} $ og $ 1\enh{dm}^3=1\enh{L} $. $ 10\cdot8\cdot7=560 $. Dette betyr at volumet er $ 0,56\enh{L} $.

\newpage
\subsection*{Oppgave 7}
\textbf{Løsningsmetode 1} \\
Arealet er lik $ (a+b)^2-b^2=a^2+2ab $
\begin{figure}
	\includegraphics[]{opg7d1_a}
\end{figure}

\textbf{Løsningsmetode 2}\\
Arealet er lik $ (a+b)a + ab = a^2+2ab$.
\begin{figure}
	\includegraphics[]{opg7d1_b}
\end{figure}

\newpage
\subsection*{Oppgave 1}
Ida har valgt å la $ y$-aksen starte på 0\,kr. Siden prisforskjellene er ganske små i forhold til den totale prisen, blir høydeforskjellen på søylene også ganske små. I tillegg blir intervallene mellom horisontallinjene store, dette gjør at det også er vanskelig å se verdiene til prisforskjellene.\vsk

Juan har valgt å la $ y $-aksen starte på 180\,kr. Da blir høydeforskjellen på søylene større og intervallene mellom horisontallinjene mindre. Dette gjør at det er lettere å se hvilken butikk som er billigst og hvor mye som skiller butikkene i pris.

\subsection*{Oppgave 2}
\abc{
\item \phantom{text}\\
\begin{figure}
	\includegraphics[scale=0.2]{opg2}
\end{figure}
\item $ f $ er en lineær funksjon med stigningstalll 200 og konstantledd 40. Det betyr at grafen til funksjonen er en rett linje som skjærer $ y $-aksen når $ y=40 $, og at $ y $-verdien øker med 200 hver gang $ x $-verdien øker med 1.\os

$ g $ er en rasjonal funksjon, fordi den kan skrives $ g(x)=\frac{1000+80x}{x} $. Siden uttrykket har en brøk med $ x $ som nevner, er funksjonen ikke definert for $ x=0 $. $ g $ får en veldig høy tallverdi når $ x $ er i nærheten av  $ 0 $, mens den blir tilnærmet lik 80 når $ x $ har veldig høy tallverdi.\os

$ h $ er en lineær funksjon med stigningstalll $ -100 $. Det betyr at grafen til funksjonen er en rett linje som skjærer $ y $-aksen når $ y=0 $, og at $ y $-verdien minker med 100 hver gang $ x $-verdien øker med 1. At $ h $ er en lineær funksjon uten konstantledd gjør at $ h $ og $ x $ er proporsjonale størrelser.
\item Si at en taxi har startpris 40 kr, og deretter 200 per kilometer. Da er $ f $ prisen for å kjøre $ x $ kilometer.
}

\subsection*{Oppgave 3}
\abc{
\item \phantom{text}\\
\begin{figure}
	\includegraphics[scale=0.3]{opg3a} \;
	\includegraphics[scale=0.3]{opg3b}
\end{figure}
\item $ \frac{9461,66}{7500}\approx1,26  $, altså er effektiv rente ca 26\%.
}

\subsection*{Oppgave 4}
\begin{figure}
	\includegraphics[scale=0.3]{opg4a} \;
	\includegraphics[scale=0.3]{opg4b}
\end{figure}
Skål 4 gir størst sannsynlighet.


\subsection*{Oppgave 5}
$ g $ ser ut til å være den lineære funksjonen som går gjennom punktene $ (0, 30) $ og $ (6, 30000) $. Grafen til $ g $ illustrerer den gjennomsnittlige økningen av følgere per måned. \os

$ f $ er trolig en funksjon som godt beskriver antall følgere for hver måned. Den illustrerer at veksten av følgere var lav de første 4 månedene, mens den ble høy de 2 siste månedene.

\newpage
\subsection*{Oppgave 6}
\abc{
\item Forklaring av tekst-skript linje for linje:
\renewcommand{\labelenumii}{\arabic{enumii}}
\begin{enumerate}
	\item Bruker skriver inn radius til minste kule. Input en string som må bli gjort om til en float for å brukes til regning.
	\item Bruker skriver inn radius til største kule. \addtocounter{enumii}{1}
	\item Utregning av vulumet til liten kule ved bruk av volumformel og $ \pi\approx3,14 $.
	\item Utregning av vulumet til stor kule. \addtocounter{enumii}{1}
	\item Definering av variabel som gir forholdet mellom volumet til den største kula og volumet til den minste kula. \addtocounter{enumii}{1}
	\item Printer variabelen fra linje 7.
\end{enumerate}
\item Volumet $ V $ til en kule med radius $ a $ er gitt som
\[ V=\frac{4\pi a^3}{3} \]
La $ V_R $ og $ V_r $ være de respektive volumene til kuler med radius $ R $ og $ r $. Da er 
\[ \frac{V_R}{V_r}=\frac{4\pi R^3}{3}\cdot \frac{3}{4\pi r^3}=\frac{R^3}{r^3}=\left(\frac{R}{r}\right)^3 \]
Hvis radius dobles vil altså volumet ganges med $ 2^3=8 $.
}

\subsection*{Oppgave 7}
\begin{figure}
\centering
\includegraphics[]{opg7}
\end{figure}
\[ (a+1)(a+5)-a(a+6)=a^2+5a+a+5-(a^2+6a)=5 \]
Prossessen de to jentene beskriver vil alltid gi en differanse lik 5.

\newpage
\subsection*{Oppgave 8}
\textbf{Tid brukt på å dusje}
\begin{figure}
	\centering
	\includegraphics[scale=0.3]{opg8a} \;
	\includegraphics[scale=0.3]{opg8b}
\end{figure}
Sett ut i fra de to sentralmålene gjennomsnitt og median kan man si at jentene dusjer noe lengre enn gutter, men forskjellen er veldig liten. For jentene er typetallet 15, som betyr at 15 er svaret som har høyest frekvens blant jentene. For guttene er det ikke noe typetall. At jentene har høyest varians forteller oss at det blant jentene at svarene i sum avviker mest fra gjennomsnittet. \vsk

\textbf{Vannforbruk} \os
$ \text{pris for døgnlig vannforbruk}=140\enh{L}\cdot 0,02\enh{kr}/L=2,8\enh{kr} $.\vsk

\textbf{Hva koster det å dusje?} \os
$ 1\enh{L}\text{ vann}\approx1\enh{kg}\text{ vann} $. Det blir ikke oppgitt hvor mange liter per minutt det er rimelig å anta at man bruker ved en dusj. Vi setter denne størrelsen lik $ a $. Videre antar vi at 60\% av dette vannforbruket består av varmtvann hvis dusjen har en behagelig temperatur. Det er dette vannet som er varmet opp fra 10 grader til 70 grader, altså 60 graders temperaturforskjell. Prisen for å dusje i $ m $ minutter vil da være gitt ved prisen for oppvarming av vann addert med prisen for forbruk av vann:
\[ \text{prisen for å dusje i } m \text{ minutter}= \frac{0,6a\cdot60\cdot4,2}{3600}\cdot1\cdot m+0,02\cdot a\cdot m=0,062am\]
Ut i fra undersøkelsen er det rimelig å si at en dusj varer i 10 minutter.
\[ \text{prisen for å dusje i } 10 \text{ minutter}= 0,62a\]
\textit{Kommentar: Et internettsøk viser at $ a=16 $ er et rimelig tall. I så fall vil en 10 minutts dusj koste ca 10\enh{kr}. }

\end{document} 