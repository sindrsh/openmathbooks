\documentclass[english,hidelinks,pdftex, 11 pt, class=report,crop=false]{standalone}
\usepackage[T1]{fontenc}
\usepackage[utf8]{luainputenc}
\usepackage{lmodern} % load a font with all the characters
\usepackage{geometry}
\geometry{verbose,a4paper, inner=0cm, outer=0 cm, bmargin=2cm, tmargin=1cm}
%\textwidth=12cm
\setlength{\parindent}{0bp}
\usepackage{import}
\usepackage[subpreambles=false]{standalone}
\usepackage{amsmath}
\usepackage{amssymb}
\usepackage{esint}
\usepackage{babel}
\usepackage{tabu}
\usepackage[dvipsnames, table]{xcolor}
\usepackage{cancel}
\makeatother
\makeatletter
\usepackage{datetime2}
\usepackage{titlesec}
\usepackage[many]{tcolorbox}

% Eheter
\newcommand{\enh}[1]{\,\textrm{#1}}
%referances
\newcommand{\net}[2]{{\color{blue}\href{#1}{#2}}}

%Spaces
\newcommand{\vsk}{\\[12pt]}
\newcommand{\vs}{\vspace{-12pt}}

% Tabell for opplegg

\newcommand{\ovlist}[1]{
\vspace{-16pt}
\begin{itemize}
	#1
\end{itemize}
}

% Chapters and sections
\titleformat{\section}[block]{\bfseries}{\hspace{3cm}\thesection}{5pt}{}
\titleformat{\subsection}[block]{\bfseries}{\hspace{3cm}\thesection}{5pt}{}
\newcommand{\sectionbreak}{\clearpage} % New page on each section
 

\newlength{\mywidth}
\setlength{\mywidth}{14cm}

\newcommand{\cont}[1]{
\begin{tcolorbox}[center, boxrule=0.0 mm, width=\mywidth,arc=0mm,enhanced jigsaw,,colback=white,breakable]
#1	
\end{tcolorbox}
}

\newcommand{\info}[5]{
\begin{tcolorbox}[center, boxrule=0.1 mm, width=\mywidth,arc=0mm,enhanced jigsaw,breakable,colback=yellow!5]	
	
	\footnotesize
	\textbf{Øvingsområde}\\[5pt] #1 
	
	\textbf{Utstyr}\\ #2  \\
	
	\begin{tabular}{@{} p{4cm} p{4cm} l} 
		\textbf{Tid} & \textbf{Elevinndeling} & \textbf{Læringsarena} \\
		#3  & #4 & #5
	\end{tabular} 
\end{tcolorbox}	
}

\newcommand{\gjen}[1]{\begin{tcolorbox}[center,boxrule=0.1 mm, width=\mywidth,arc=0mm,colback=blue!3] {\large \textbf{Gjennomføring} \vspace{5 pt}}\newline #1  \end{tcolorbox}\vspace{-5pt}}
\newcommand{\eks}[1]{\begin{tcolorbox}[center,boxrule=0.1 mm, width=\mywidth,arc=0mm,colback=green!3] {\large \textbf{Eksempel} \vspace{5 pt}}\newline #1  \end{tcolorbox}\vspace{-5pt}}

\newcounter{opl}
%\numberwithin{opl}{article}


\newcommand{\opl}[1]{
\newpage
{\refstepcounter{opl} %\phantomsection 
\large \textbf{\theopl \;#1} \vsk}
}

% Headlines
\newcommand{\fork}{\textbf{Forkunnskapar}\\}
\newcommand{\forb}{\textbf{Forberedelsar}\\}
\newcommand{\opgvr}{\textbf{Oppgaver}}



%colors
\newcommand{\colr}[1]{{\color{red} #1}}
\newcommand{\colb}[1]{{\color{blue} #1}}
\newcommand{\colo}[1]{{\color{orange} #1}}
\newcommand{\colc}[1]{{\color{cyan} #1}}
\definecolor{projectgreen}{cmyk}{100,0,100,0}
\newcommand{\colg}[1]{{\color{projectgreen} #1}}

% Lister med bokstavar
\usepackage[inline]{enumitem}
% Opg
\newcommand{\abc}[1]{
	\begin{enumerate}[label=\alph*),leftmargin=18pt]
		#1
	\end{enumerate}
}

\usepackage[]{hyperref}



\begin{document}
{\Large Eksamen P1 våren 2024\hfill {\footnotesize Løsning fra \color{blue} \href{https://sindreheggen.wordpress.com/}{OpenMathBooks prosjektet}}}	
\subsection*{Oppgave 1}
$ 2\text{\textperthousand}=0,002 $.
\alg{
2\text{\textperthousand\, av } x &= 18 \\
0.002\cdot x &= 18 \\
x &= \frac{18}{0.002} \\
&= 9\,000
}
9000 millioner er lik 9 milliarder. Befolkningen ville vært 9 milliarder.

\subsection*{Oppgave 2}
\abc{
\item 20\,000 er antall kroner Ada setter inn i banken. 1,0485 erd den årlige vekstfaktoren, som betyr at Ada får 4,85\% årlig rente. 
\item $ v $ i koden til Ada tilsvarer $ \frac{f(10)-f(0)}{10-0} $, som vil gi den gjennomsnittlige endringen til $ f $ på intervallet $ [0, 10] $. Dette vil fortelle hvor mange kroner sparepengene i snitt har økt med per år i løpet av 10 år. 
}

\subsection*{Oppgave 3}
\begin{itemize}
	\item For at to størrelser $ F $ og $ x $ skal være proporsjonale, må grafen til funksjonen $ F(x) $ være en rett linje som går gjennom origo. Dermed er $ f $ og $ x $ proporsjonale størrelser.
	\item Hvis $ F $ og $ x $ er omvendt proporsjonale størrelser, kan vi skrive $ F(x)=\frac{a}{x} $, hvor $ a $ er en konstant. Det betyr at $ F $ vil ha veldig høy verdi når $ x $ er i nærheten av 0. Når $ x $ har høy verdi, vil $ F $ nærme seg 0. $ F $ er ikke definert når $ x=0 $. Av dette ser vi at det bare er $ p $ og $ x $ som kan være omvendt proporsjonale størrelser.
\end{itemize}

\newpage
\subsection*{Oppgave 4}
\abc{
\item $ \frac{70}{10}=7 $. $ B=\frac{7^2}{2}=24,5 $.
\item \alg{
40,5 &= \frac{x^2}{2}\\
81 &= x^2 \\
\pm 9 &= x
}
$ x=9 $ tilsvarer en fart på $ 90\enh{km/h} $ ($ x $ ganget med 10).
}


\end{document}