\documentclass[english,hidelinks,pdftex, 11 pt, class=report,crop=false]{standalone}
\usepackage[T1]{fontenc}
\usepackage[utf8]{luainputenc}
\usepackage{lmodern} % load a font with all the characters
\usepackage{geometry}
\geometry{verbose,a4paper, inner=0cm, outer=0 cm, bmargin=2cm, tmargin=1cm}
%\textwidth=12cm
\setlength{\parindent}{0bp}
\usepackage{import}
\usepackage[subpreambles=false]{standalone}
\usepackage{amsmath}
\usepackage{amssymb}
\usepackage{esint}
\usepackage{babel}
\usepackage{tabu}
\usepackage[dvipsnames, table]{xcolor}
\usepackage{cancel}
\makeatother
\makeatletter
\usepackage{datetime2}
\usepackage{titlesec}
\usepackage[many]{tcolorbox}

% Eheter
\newcommand{\enh}[1]{\,\textrm{#1}}
%referances
\newcommand{\net}[2]{{\color{blue}\href{#1}{#2}}}

%Spaces
\newcommand{\vsk}{\\[12pt]}
\newcommand{\vs}{\vspace{-12pt}}

% Tabell for opplegg

\newcommand{\ovlist}[1]{
\vspace{-16pt}
\begin{itemize}
	#1
\end{itemize}
}

% Chapters and sections
\titleformat{\section}[block]{\bfseries}{\hspace{3cm}\thesection}{5pt}{}
\titleformat{\subsection}[block]{\bfseries}{\hspace{3cm}\thesection}{5pt}{}
\newcommand{\sectionbreak}{\clearpage} % New page on each section
 

\newlength{\mywidth}
\setlength{\mywidth}{14cm}

\newcommand{\cont}[1]{
\begin{tcolorbox}[center, boxrule=0.0 mm, width=\mywidth,arc=0mm,enhanced jigsaw,,colback=white,breakable]
#1	
\end{tcolorbox}
}

\newcommand{\info}[5]{
\begin{tcolorbox}[center, boxrule=0.1 mm, width=\mywidth,arc=0mm,enhanced jigsaw,breakable,colback=yellow!5]	
	
	\footnotesize
	\textbf{Øvingsområde}\\[5pt] #1 
	
	\textbf{Utstyr}\\ #2  \\
	
	\begin{tabular}{@{} p{4cm} p{4cm} l} 
		\textbf{Tid} & \textbf{Elevinndeling} & \textbf{Læringsarena} \\
		#3  & #4 & #5
	\end{tabular} 
\end{tcolorbox}	
}

\newcommand{\gjen}[1]{\begin{tcolorbox}[center,boxrule=0.1 mm, width=\mywidth,arc=0mm,colback=blue!3] {\large \textbf{Gjennomføring} \vspace{5 pt}}\newline #1  \end{tcolorbox}\vspace{-5pt}}
\newcommand{\eks}[1]{\begin{tcolorbox}[center,boxrule=0.1 mm, width=\mywidth,arc=0mm,colback=green!3] {\large \textbf{Eksempel} \vspace{5 pt}}\newline #1  \end{tcolorbox}\vspace{-5pt}}

\newcounter{opl}
%\numberwithin{opl}{article}


\newcommand{\opl}[1]{
\newpage
{\refstepcounter{opl} %\phantomsection 
\large \textbf{\theopl \;#1} \vsk}
}

% Headlines
\newcommand{\fork}{\textbf{Forkunnskapar}\\}
\newcommand{\forb}{\textbf{Forberedelsar}\\}
\newcommand{\opgvr}{\textbf{Oppgaver}}



%colors
\newcommand{\colr}[1]{{\color{red} #1}}
\newcommand{\colb}[1]{{\color{blue} #1}}
\newcommand{\colo}[1]{{\color{orange} #1}}
\newcommand{\colc}[1]{{\color{cyan} #1}}
\definecolor{projectgreen}{cmyk}{100,0,100,0}
\newcommand{\colg}[1]{{\color{projectgreen} #1}}

% Lister med bokstavar
\usepackage[inline]{enumitem}
% Opg
\newcommand{\abc}[1]{
	\begin{enumerate}[label=\alph*),leftmargin=18pt]
		#1
	\end{enumerate}
}

\usepackage[]{hyperref}
\begin{document}
\newpage
\section*{Forord til lærere}
\textbf{Bokas bruksområde}\\
Sammen med \net{arg1}{Matematikkens byggesteiner} (MB) dekker denne boka matematikk for 5.-10. klassse og for VGS-fagen 1P og 2P. Mens MB tar for seg de teoretiske grunnprinsippene matte er bygd på, er denne boka ment for å vise hvordan matte kan anvendes i det daglige. Det er likevel med en viss ambivalens jeg bruker ordet ''anvendt''. Jeg er hellig overbevist om at de aller fleste har behov å bruke matematikk i konkrete, praktiske situasjoner for å få opplevelsen av at matematikk blir anvendt. Jeg håper derfor disse gratis-bøkene kan frigi midler for skuler, som da kan investere i utstyr som gjør at elever (og lærere) får måle, estimere, kalkulere og vurdere ut i fra reelle situasjonar.\vsk

\textbf{Bokas disponering} \\
Da boka gaper over matematikk for 5. klasse og helt til VGS, vil kanskje mange mene at språket er noe avansert, spesielt for de yngste. Men forenklinger fører ofte til at man stadig må vende tilbake til tema for å kommentere nye utvidelser og/eller unntak, og da dannes det fort et unødig kronglete og innviklet bilde av matematikkens struktur. Jeg tror man i lengden er tjent med å presentere temaene så utfyllende som mulig, og heller bruke god tid på å forstå dem én gang for alle.\vsk

Noen vil kanskje også reagere på at eksemplene er veldig enkle, at de viser få sammensatte problemer. Én av grunnene til dette er at slik vil det faktisk være for de aller fleste etter endt skolegang; det handler om å bruke formler direkte. En annen grunn er at jeg mener det å mestre likninger er den overlegent beste måten å løse sammensatte problemer på, og derfor handler nesten hele kapittel 6 om problemløsing.\vsk

\textbf{Tilbakemeldinger og eventuelle endringer} \\
Jeg håper å høre fra deg med tilbakemeldinger om boka. Merk likevel at alle har sine tanker om hvordan ei lærebok ideelt sett bør utformes, så ikke tolk det som utakknemlighet hvis tilbakemeldinger ikke tas til etterretning. Husk at kodekilden til både denne \net{https://sindrsh.github.io/AppliedMath/}{boka} og \mb\;ligger åpen for alle på GitHub; med litt kunnskaper om Git og \LaTeX\;kan du enkelt gjøre endringer slik det passer deg og din klasse!


\newpage
\textbf{Gjøreliste} \\
Prosjektet som denne boka er en viktig del av er under stadig utvikling. Her er en liste med kommende gjøremål, i prioritert rekkefølge:
\begin{itemize}
	\item Korrigere skrivefeil. Dette gjøres kontinuerlig, gir du beskjed om feil funnet til {\tt sindre.heggen@gmail.com}, vil korrigering som oftest bli utført samme dag.
	\item Legge til flere oppgaver både i denne boka og i \mb.
	\item Legge til fasit 
	\item Lage en pensumoversikt for denne boka og \mb\,sett opp mot kompetansemålene f.o.m. 5. klasse og t.o.m. 2P. 
	\item Videreutvikle \net{https://hellandmatte.netlify.app/}{nettside} med læringsvideoer, undervisningsopplegg og mer. 
\end{itemize}
\newpage

\end{document}