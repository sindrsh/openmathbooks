\documentclass[english,hidelinks,pdftex, 11 pt, class=report,crop=false]{standalone}
\usepackage[T1]{fontenc}
\usepackage[utf8]{luainputenc}
\usepackage{lmodern} % load a font with all the characters
\usepackage{geometry}
\geometry{verbose,a4paper, inner=0cm, outer=0 cm, bmargin=2cm, tmargin=1cm}
%\textwidth=12cm
\setlength{\parindent}{0bp}
\usepackage{import}
\usepackage[subpreambles=false]{standalone}
\usepackage{amsmath}
\usepackage{amssymb}
\usepackage{esint}
\usepackage{babel}
\usepackage{tabu}
\usepackage[dvipsnames, table]{xcolor}
\usepackage{cancel}
\makeatother
\makeatletter
\usepackage{datetime2}
\usepackage{titlesec}
\usepackage[many]{tcolorbox}

% Eheter
\newcommand{\enh}[1]{\,\textrm{#1}}
%referances
\newcommand{\net}[2]{{\color{blue}\href{#1}{#2}}}

%Spaces
\newcommand{\vsk}{\\[12pt]}
\newcommand{\vs}{\vspace{-12pt}}

% Tabell for opplegg

\newcommand{\ovlist}[1]{
\vspace{-16pt}
\begin{itemize}
	#1
\end{itemize}
}

% Chapters and sections
\titleformat{\section}[block]{\bfseries}{\hspace{3cm}\thesection}{5pt}{}
\titleformat{\subsection}[block]{\bfseries}{\hspace{3cm}\thesection}{5pt}{}
\newcommand{\sectionbreak}{\clearpage} % New page on each section
 

\newlength{\mywidth}
\setlength{\mywidth}{14cm}

\newcommand{\cont}[1]{
\begin{tcolorbox}[center, boxrule=0.0 mm, width=\mywidth,arc=0mm,enhanced jigsaw,,colback=white,breakable]
#1	
\end{tcolorbox}
}

\newcommand{\info}[5]{
\begin{tcolorbox}[center, boxrule=0.1 mm, width=\mywidth,arc=0mm,enhanced jigsaw,breakable,colback=yellow!5]	
	
	\footnotesize
	\textbf{Øvingsområde}\\[5pt] #1 
	
	\textbf{Utstyr}\\ #2  \\
	
	\begin{tabular}{@{} p{4cm} p{4cm} l} 
		\textbf{Tid} & \textbf{Elevinndeling} & \textbf{Læringsarena} \\
		#3  & #4 & #5
	\end{tabular} 
\end{tcolorbox}	
}

\newcommand{\gjen}[1]{\begin{tcolorbox}[center,boxrule=0.1 mm, width=\mywidth,arc=0mm,colback=blue!3] {\large \textbf{Gjennomføring} \vspace{5 pt}}\newline #1  \end{tcolorbox}\vspace{-5pt}}
\newcommand{\eks}[1]{\begin{tcolorbox}[center,boxrule=0.1 mm, width=\mywidth,arc=0mm,colback=green!3] {\large \textbf{Eksempel} \vspace{5 pt}}\newline #1  \end{tcolorbox}\vspace{-5pt}}

\newcounter{opl}
%\numberwithin{opl}{article}


\newcommand{\opl}[1]{
\newpage
{\refstepcounter{opl} %\phantomsection 
\large \textbf{\theopl \;#1} \vsk}
}

% Headlines
\newcommand{\fork}{\textbf{Forkunnskapar}\\}
\newcommand{\forb}{\textbf{Forberedelsar}\\}
\newcommand{\opgvr}{\textbf{Oppgaver}}



%colors
\newcommand{\colr}[1]{{\color{red} #1}}
\newcommand{\colb}[1]{{\color{blue} #1}}
\newcommand{\colo}[1]{{\color{orange} #1}}
\newcommand{\colc}[1]{{\color{cyan} #1}}
\definecolor{projectgreen}{cmyk}{100,0,100,0}
\newcommand{\colg}[1]{{\color{projectgreen} #1}}

% Lister med bokstavar
\usepackage[inline]{enumitem}
% Opg
\newcommand{\abc}[1]{
	\begin{enumerate}[label=\alph*),leftmargin=18pt]
		#1
	\end{enumerate}
}

\usepackage[]{hyperref}

\newcommand{\note}{Merk}
\newcommand{\notesm}[1]{{\footnotesize \textsl{\note:} #1}}
\newcommand{\ekstitle}{Eksempel }
\newcommand{\sprtitle}{Språkboksen}
\newcommand{\expl}{forklaring}
\newcommand{\pyt}{Pytagoras' setning}
\newcommand\sv{\vsk \textbf{Svar} \vspace{4 pt}\\}

%references
\newcommand{\reftab}[1]{\hrs{#1}{tabell}}
\newcommand{\rref}[1]{\hrs{#1}{regel}}
\newcommand{\dref}[1]{\hrs{#1}{definisjon}}
\newcommand{\refkap}[1]{\hrs{#1}{kapittel}}
\newcommand{\refsec}[1]{\hrs{#1}{seksjon}}
\newcommand{\refdsec}[1]{\hrs{#1}{delseksjon}}
\newcommand{\refved}[1]{\hrs{#1}{vedlegg}}
\newcommand{\eksref}[1]{\textsl{#1}}
\newcommand\fref[2][]{\hyperref[#2]{\textsl{figur \ref*{#2}#1}}}
\newcommand{\refop}[1]{{\color{blue}Oppgave \ref{#1}}}
\newcommand{\refops}[1]{{\color{blue}oppgave \ref{#1}}}


%Algebra
\newcommand{\kvadset}{Kvadratsetningene}
\newcommand{\aenato}{Sum-produkt-metoden}

% Geometry
\newcommand{\hlikb}{Midtnormalen i en likebeint trekant}
\newcommand{\arealsetn}{Arealsetningen}
\newcommand{\trkmedian}{Median}
\newcommand{\midtrk}{Midtnormal (i trekant)}
\newcommand{\innskrsirk}{Innskrevet sirkel}
\newcommand{\cossetn}{Cosinussetningen}
\newcommand{\perfvink}{Sentral- og periferivinkel}
\newcommand{\tang}{Tangent}

% Derivative
\newcommand{\derel}{Den deriverte av elementære funksjoner}
\newcommand{\divder}{Divisjonsregelen}
\newcommand{\kjernereg}{Kjerneregelen}
\newcommand{\prodregder}{Produktregelen}
\newcommand{\lhop}{L'Hopitals regel}

% Funksjonsdrofting
\newcommand{\monder}{Monotoniegenskaper og den deriverte}
\newcommand{\fderekstr}{$ \bm{f'=0} $ for lokale ektstremalpunkt}
\newcommand{\andredertest}{Andrederiverttesten}

% Vectors
\newcommand{\detar}{Arealformler med determinanter}
\newcommand{\avstpunktlin}{Avstand mellom punkt og linje}

%Appendix
\newcommand{\rolle}{Rolles teorem}
\newcommand{\meanval}{Middelverdisetningen}

% Solutions manual
\newcommand{\selos}{Se løsningsforslag.}

\begin{document}


\op{gjeromtilm}
Gjør om til antall meter. \os
\abch{
	\item 484\enh{km}
	\item 91\enh{km}
	\item 2\,402\enh{km}
}

\op{gjeromtilg}
Gjør om til antall gram. \os
\abch{
	\item 484\enh{kg}
	\item 91\enh{hg}
	\item 2\,402\enh{hg}
}

\op{gjeromtilL}
Gjør om til antall liter \os
\abch{
	\item 480\enh{dl}
	\item 9100\enh{cl}
	\item 24\,000\enh{cl}
}

\op{gjeromblanda}
Gjør om 
\begin{multicols}{3}
	\abc{
		\item 12,4\enh{m} \\til antall km.
		\item 42\enh{dm} \\til antall m.
		\item 58,15\enh{cm} \\til antall mm.
		\item 0,0074\enh{km}\\ til antall m.
		\item 0,15\enh{m} \\til antall cm.
		\item 9,7\enh{g} \\til antall hg.
		\item 0,15\enh{mg}\\ til antall g.
		\item 1,419\enh{hg}\\ til antall mg.
		\item 31\enh{mg} \\til antall hg.
		\item 64\,039\enh{mg}\\ til antall kg.
		\item 89\enh{dL} \\til antall L.
		\item 691,4\enh{L}\\ til antall cL.
		\item 15\enh{L} \\til antall mL.
		\item 918\enh{cL} \\til antall L.
		\item 0,55\enh{dL} \\til antall mL.
	}
\end{multicols}

\nes

\op{volpris}
En prisme har lengde 9, bredde 10 og høgde 8.
\abc{
	\item Finn grunnflaten til prismen.
	\item Finn volumet til prismen.
}

\op{volprisl}
En prisme har lengde 9\enh{cm}, bredde 10\enh{cm} og høgde 8\enh{cm}.\os

Finn volumet til prismen.


\op{volkjegl}
En kjegle har radius 10\enh{dm} og høgde 4\enh{dm}.
\abc{
	\item Finn volumet til kjeglen.
	\item Hvor mange liter rommer kjeglen?
}

\op{volpyrl}
En firkantet pyramide har lengde 4\enh{cm}, bredde 9\enh{cm} og høgde 10\enh{cm}.
\abc{
	\item Finn volumet til kjeglen.
	\item Hvor mange liter rommer kjeglen?
}

\nes

\op{opgstorlkilopris}
I matbutikker er som regel både pris og kilopris oppgitt for en vare. Vekten til varen finner man på forpakningen til varen. Gå i din lokale matbutikk og velg ut fem varer. Sjekk om kiloprisen som butikken oppgir er rett for disse varene.

\op{opgstorlbolt}
Usain Bolt har verdensrekorden for 100\enh{m} sprint. I tabellen under ser du hva tidtakeren viste ved hver 10. meter under dette rekordløpet. \vs
\begin{center}\small
	\begin{tabular}{l|c|c|c|c|c|c|c|c|c|c}
meter & 10 & 20 & 30 & 40 & 50 & 60 & 70 & 80 & 90 & 100\\
sekunder& 1,89 & 2,88 & 3,78 & 4,64 & 5,47 & 6,29 & 7,1 & 7,92 & 8,75 & 9,58
\end{tabular}
\end{center}
\abc{
\item Hvis Bolts fart hadde vært den samme under hele løpet, hva hadde farten hans vært da?
\item Hvorfor er det \textit{ikke} rimelig å anta at Bolt hadde den samme farten under hele løpet?
\item Anta at Bolt under dette løpet nådde den høgste farten et menneske har sprunget. Finn en tilnærming til denne farten.
}

\nes
\op{opgstorlbolt2}
Gjør om svarene fra \refops{opgstorlbolt} a) og b) til hastigheter oppgitt i '\enh{km/h}'. 

\op{opgstorlfinn} 
Skriv ned eksempel på et dyr, et insekt, en gjenstand eller annet som veier mellom 1-100\,mg, cg, dg, g, dag, hg og kg.
\end{document}


