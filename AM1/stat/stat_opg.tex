\documentclass[english,hidelinks,pdftex, 11 pt, class=report,crop=false]{standalone}
\usepackage[T1]{fontenc}
\usepackage[utf8]{luainputenc}
\usepackage{lmodern} % load a font with all the characters
\usepackage{geometry}
\geometry{verbose,a4paper, inner=0cm, outer=0 cm, bmargin=2cm, tmargin=1cm}
%\textwidth=12cm
\setlength{\parindent}{0bp}
\usepackage{import}
\usepackage[subpreambles=false]{standalone}
\usepackage{amsmath}
\usepackage{amssymb}
\usepackage{esint}
\usepackage{babel}
\usepackage{tabu}
\usepackage[dvipsnames, table]{xcolor}
\usepackage{cancel}
\makeatother
\makeatletter
\usepackage{datetime2}
\usepackage{titlesec}
\usepackage[many]{tcolorbox}

% Eheter
\newcommand{\enh}[1]{\,\textrm{#1}}
%referances
\newcommand{\net}[2]{{\color{blue}\href{#1}{#2}}}

%Spaces
\newcommand{\vsk}{\\[12pt]}
\newcommand{\vs}{\vspace{-12pt}}

% Tabell for opplegg

\newcommand{\ovlist}[1]{
\vspace{-16pt}
\begin{itemize}
	#1
\end{itemize}
}

% Chapters and sections
\titleformat{\section}[block]{\bfseries}{\hspace{3cm}\thesection}{5pt}{}
\titleformat{\subsection}[block]{\bfseries}{\hspace{3cm}\thesection}{5pt}{}
\newcommand{\sectionbreak}{\clearpage} % New page on each section
 

\newlength{\mywidth}
\setlength{\mywidth}{14cm}

\newcommand{\cont}[1]{
\begin{tcolorbox}[center, boxrule=0.0 mm, width=\mywidth,arc=0mm,enhanced jigsaw,,colback=white,breakable]
#1	
\end{tcolorbox}
}

\newcommand{\info}[5]{
\begin{tcolorbox}[center, boxrule=0.1 mm, width=\mywidth,arc=0mm,enhanced jigsaw,breakable,colback=yellow!5]	
	
	\footnotesize
	\textbf{Øvingsområde}\\[5pt] #1 
	
	\textbf{Utstyr}\\ #2  \\
	
	\begin{tabular}{@{} p{4cm} p{4cm} l} 
		\textbf{Tid} & \textbf{Elevinndeling} & \textbf{Læringsarena} \\
		#3  & #4 & #5
	\end{tabular} 
\end{tcolorbox}	
}

\newcommand{\gjen}[1]{\begin{tcolorbox}[center,boxrule=0.1 mm, width=\mywidth,arc=0mm,colback=blue!3] {\large \textbf{Gjennomføring} \vspace{5 pt}}\newline #1  \end{tcolorbox}\vspace{-5pt}}
\newcommand{\eks}[1]{\begin{tcolorbox}[center,boxrule=0.1 mm, width=\mywidth,arc=0mm,colback=green!3] {\large \textbf{Eksempel} \vspace{5 pt}}\newline #1  \end{tcolorbox}\vspace{-5pt}}

\newcounter{opl}
%\numberwithin{opl}{article}


\newcommand{\opl}[1]{
\newpage
{\refstepcounter{opl} %\phantomsection 
\large \textbf{\theopl \;#1} \vsk}
}

% Headlines
\newcommand{\fork}{\textbf{Forkunnskapar}\\}
\newcommand{\forb}{\textbf{Forberedelsar}\\}
\newcommand{\opgvr}{\textbf{Oppgaver}}



%colors
\newcommand{\colr}[1]{{\color{red} #1}}
\newcommand{\colb}[1]{{\color{blue} #1}}
\newcommand{\colo}[1]{{\color{orange} #1}}
\newcommand{\colc}[1]{{\color{cyan} #1}}
\definecolor{projectgreen}{cmyk}{100,0,100,0}
\newcommand{\colg}[1]{{\color{projectgreen} #1}}

% Lister med bokstavar
\usepackage[inline]{enumitem}
% Opg
\newcommand{\abc}[1]{
	\begin{enumerate}[label=\alph*),leftmargin=18pt]
		#1
	\end{enumerate}
}

\usepackage[]{hyperref}

\newcommand{\note}{Merk}
\newcommand{\notesm}[1]{{\footnotesize \textsl{\note:} #1}}
\newcommand{\ekstitle}{Eksempel }
\newcommand{\sprtitle}{Språkboksen}
\newcommand{\expl}{forklaring}
\newcommand{\pyt}{Pytagoras' setning}
\newcommand\sv{\vsk \textbf{Svar} \vspace{4 pt}\\}

%references
\newcommand{\reftab}[1]{\hrs{#1}{tabell}}
\newcommand{\rref}[1]{\hrs{#1}{regel}}
\newcommand{\dref}[1]{\hrs{#1}{definisjon}}
\newcommand{\refkap}[1]{\hrs{#1}{kapittel}}
\newcommand{\refsec}[1]{\hrs{#1}{seksjon}}
\newcommand{\refdsec}[1]{\hrs{#1}{delseksjon}}
\newcommand{\refved}[1]{\hrs{#1}{vedlegg}}
\newcommand{\eksref}[1]{\textsl{#1}}
\newcommand\fref[2][]{\hyperref[#2]{\textsl{figur \ref*{#2}#1}}}
\newcommand{\refop}[1]{{\color{blue}Oppgave \ref{#1}}}
\newcommand{\refops}[1]{{\color{blue}oppgave \ref{#1}}}


%Algebra
\newcommand{\kvadset}{Kvadratsetningene}
\newcommand{\aenato}{Sum-produkt-metoden}

% Geometry
\newcommand{\hlikb}{Midtnormalen i en likebeint trekant}
\newcommand{\arealsetn}{Arealsetningen}
\newcommand{\trkmedian}{Median}
\newcommand{\midtrk}{Midtnormal (i trekant)}
\newcommand{\innskrsirk}{Innskrevet sirkel}
\newcommand{\cossetn}{Cosinussetningen}
\newcommand{\perfvink}{Sentral- og periferivinkel}
\newcommand{\tang}{Tangent}

% Derivative
\newcommand{\derel}{Den deriverte av elementære funksjoner}
\newcommand{\divder}{Divisjonsregelen}
\newcommand{\kjernereg}{Kjerneregelen}
\newcommand{\prodregder}{Produktregelen}
\newcommand{\lhop}{L'Hopitals regel}

% Funksjonsdrofting
\newcommand{\monder}{Monotoniegenskaper og den deriverte}
\newcommand{\fderekstr}{$ \bm{f'=0} $ for lokale ektstremalpunkt}
\newcommand{\andredertest}{Andrederiverttesten}

% Vectors
\newcommand{\detar}{Arealformler med determinanter}
\newcommand{\avstpunktlin}{Avstand mellom punkt og linje}

%Appendix
\newcommand{\rolle}{Rolles teorem}
\newcommand{\meanval}{Middelverdisetningen}

% Solutions manual
\newcommand{\selos}{Se løsningsforslag.}

\begin{document}
\opgt

\nes \nes

\op{opgstatsentrodd1}
Gitt datasettet
\[ 2\quad12\quad 3\quad 0\quad 2\quad 5\quad 8\quad2\quad 11 \]
Finn \os
\abch{
	\item typetallet \item medianen \item gjennomsnittet 
}

\op{opgstatsentrodd2}
Gitt datasettet
	\[ 9\quad6\quad 3\quad 0\quad 8\quad 5\quad 8\quad4\quad 10\quad 8 \quad 6 \]
Finn \os
\abch{
\item typetallet \item medianen \item gjennomsnittet 
}

\op{opgstatsentrpar1}
Gitt datasettet
\[ 11\quad7\quad 16\quad 0\quad 8\quad 9\quad 8\quad5\quad 16\quad 5 \]
Finn \os
\abch{
	\item typetallet \item medianen \item gjennomsnittet 
}

\op{opgstatsentrpar2}
Gitt datasettet
\[ 6\quad11\quad 14\quad 5\quad 6\quad 9\quad 8\quad5\quad 11\quad 5\quad 11\quad 17 \]
Finn \os
\abch{
	\item typetallet \item medianen \item gjennomsnittet 
}
\newpage
\op{opgrikest}
Du ønsker å finne ut hva nordmenn flest har i formue\footnote{Enkelt sagt er formue summen av penger du har i banken, verdier av hus, bil etc., fratrekt gjeld o.l.}, og bestemmer deg for å finne ut av dette ved å spørre fem tilfeldige personer du møter i gata. De fire første svarene (i kr) er disse:
\[ 3,2\text{ millioner}\qquad 2,9\text{ millioner}\qquad 1,8\text{ millioner}\qquad 4,2\text{ millioner}  \]
Den siste personen du tilfeldigvis møter er personen i Norge med høyest formue. Hens svar er dette\footnote{Tallet er basert på personen i Norge med høyest formue ut ifra ligningstallene for 2019.}:
\[ 20\,915,3\text{ millioner} \]
\abc{
\item Finn medianen i datasettet.
\item Finn gjennomsnittet i datasettet.
\item Er det medianen eller gjennomsnittet som trolig best representerer hva nordmenn flest har i formue?
}

\op{opgfrkvtb1}
Lag en frekvenstabell for datasettet under. (La tittelen til venstre kolonne være ''frukt''.)
\begin{center}
	banan\quad eple \quad eple\quad eple \quad pære \quad banan \quad eple \quad pære \quad appelsin \quad eple \quad pære \quad pære
\end{center}

\op{opgfrkvtb2}
Lag en frekvenstabell for datasettet fra \refops{opgstatsentrpar2}. (La tittelen til venstre kolonne være ''tall''.)

\op{opgstatsoyl} \vs
\abc{
\item Lag et søylediagram for datasettet fra \refops{opgfrkvtb1}.
\item Lag et søylediagram for datasettet fra \refops{opgfrkvtb2}.
}

\op{opgstatsoylrgn} \vs
\abc{
	\item Lag et søylediagram for datasettet fra \refops{opgfrkvtb1}.
	\item Lag et søylediagram for datasettet fra \refops{opgfrkvtb2}.
}
\newpage

\op{opgstathvorfor}
Av de fire undersøkelsene på side \pageref{undersok}, hvorfor har vi
\abc{
	\item vist frekvenstabell bare for undersøkelse 2?
	\item vist søylediagram bare for undersøkelse  2 og 3?
	\item vist sektordiagram bare for undersøkelse 2 og ?
	\item vist linjediagram bare for undersøkelse 4?
}

\op{opgstat1}
Hvis datasettet har partalls antall svar kan man også finne medianen slik: \regv
\st{
\begin{enumerate}
	\item Finn de to tallene i midten.
	\item Finn differansen mellom tallene, og del denne med 2.
	\item Legg resultatet fra punkt 2 til det laveste av de to tallene i midten.
\end{enumerate}
} 
Prøv metoden på datasettet fra \refops{opgstatsentrpar1}.


\op{opgstatsnittmobil}
Tenk deg at noen i 2014 kom med følgende opplysning:
\begin{center}
''Mellom 2006 og 2014 ble det i gjennomsnitt solgt flere mobiltelefoner uten smartfunksjon enn med.''
\end{center}
Hvorfor ville denne opplysningen være lite til hjelp hvis man skulle spå framtidige salg av mobiltelefoner med og uten smartfunksjoner?

\op{opgstathvorfor2}
Av de fire undersøkelsene på side \pageref{undersok}, hvorfor har vi ikke funnet sentral- og spredningsmål for undersøkelse 3?

\newpage
\eksop{GV23D1}{gv23d1opg4}
Tabellen nedenfor viser hastigheter målt i en
fartskontroll. Alle hastighetene er mål i km/h.
\begin{center}
	\begin{tabular}{|c|c|c|c|c|c|c|c|c|c|}
		62 & 20 & 62 & 18 & 55 & 62 & 65 & 54 & 62 & 60
	\end{tabular}
\end{center}
\abc{
\item Avgjør gjennomsnitt, median og typetall.
\item Begrunn hvilket av sentralmålene du ville valgt for å beskrive bilenes hastighet.
}

\end{document}

