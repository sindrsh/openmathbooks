\documentclass[english,hidelinks,pdftex, 11 pt, class=report,crop=false]{standalone}
\usepackage[T1]{fontenc}
\usepackage[utf8]{luainputenc}
\usepackage{lmodern} % load a font with all the characters
\usepackage{geometry}
\geometry{verbose,a4paper, inner=0cm, outer=0 cm, bmargin=2cm, tmargin=1cm}
%\textwidth=12cm
\setlength{\parindent}{0bp}
\usepackage{import}
\usepackage[subpreambles=false]{standalone}
\usepackage{amsmath}
\usepackage{amssymb}
\usepackage{esint}
\usepackage{babel}
\usepackage{tabu}
\usepackage[dvipsnames, table]{xcolor}
\usepackage{cancel}
\makeatother
\makeatletter
\usepackage{datetime2}
\usepackage{titlesec}
\usepackage[many]{tcolorbox}

% Eheter
\newcommand{\enh}[1]{\,\textrm{#1}}
%referances
\newcommand{\net}[2]{{\color{blue}\href{#1}{#2}}}

%Spaces
\newcommand{\vsk}{\\[12pt]}
\newcommand{\vs}{\vspace{-12pt}}

% Tabell for opplegg

\newcommand{\ovlist}[1]{
\vspace{-16pt}
\begin{itemize}
	#1
\end{itemize}
}

% Chapters and sections
\titleformat{\section}[block]{\bfseries}{\hspace{3cm}\thesection}{5pt}{}
\titleformat{\subsection}[block]{\bfseries}{\hspace{3cm}\thesection}{5pt}{}
\newcommand{\sectionbreak}{\clearpage} % New page on each section
 

\newlength{\mywidth}
\setlength{\mywidth}{14cm}

\newcommand{\cont}[1]{
\begin{tcolorbox}[center, boxrule=0.0 mm, width=\mywidth,arc=0mm,enhanced jigsaw,,colback=white,breakable]
#1	
\end{tcolorbox}
}

\newcommand{\info}[5]{
\begin{tcolorbox}[center, boxrule=0.1 mm, width=\mywidth,arc=0mm,enhanced jigsaw,breakable,colback=yellow!5]	
	
	\footnotesize
	\textbf{Øvingsområde}\\[5pt] #1 
	
	\textbf{Utstyr}\\ #2  \\
	
	\begin{tabular}{@{} p{4cm} p{4cm} l} 
		\textbf{Tid} & \textbf{Elevinndeling} & \textbf{Læringsarena} \\
		#3  & #4 & #5
	\end{tabular} 
\end{tcolorbox}	
}

\newcommand{\gjen}[1]{\begin{tcolorbox}[center,boxrule=0.1 mm, width=\mywidth,arc=0mm,colback=blue!3] {\large \textbf{Gjennomføring} \vspace{5 pt}}\newline #1  \end{tcolorbox}\vspace{-5pt}}
\newcommand{\eks}[1]{\begin{tcolorbox}[center,boxrule=0.1 mm, width=\mywidth,arc=0mm,colback=green!3] {\large \textbf{Eksempel} \vspace{5 pt}}\newline #1  \end{tcolorbox}\vspace{-5pt}}

\newcounter{opl}
%\numberwithin{opl}{article}


\newcommand{\opl}[1]{
\newpage
{\refstepcounter{opl} %\phantomsection 
\large \textbf{\theopl \;#1} \vsk}
}

% Headlines
\newcommand{\fork}{\textbf{Forkunnskapar}\\}
\newcommand{\forb}{\textbf{Forberedelsar}\\}
\newcommand{\opgvr}{\textbf{Oppgaver}}



%colors
\newcommand{\colr}[1]{{\color{red} #1}}
\newcommand{\colb}[1]{{\color{blue} #1}}
\newcommand{\colo}[1]{{\color{orange} #1}}
\newcommand{\colc}[1]{{\color{cyan} #1}}
\definecolor{projectgreen}{cmyk}{100,0,100,0}
\newcommand{\colg}[1]{{\color{projectgreen} #1}}

% Lister med bokstavar
\usepackage[inline]{enumitem}
% Opg
\newcommand{\abc}[1]{
	\begin{enumerate}[label=\alph*),leftmargin=18pt]
		#1
	\end{enumerate}
}

\usepackage[]{hyperref}

\newcommand{\note}{Merk}
\newcommand{\notesm}[1]{{\footnotesize \textsl{\note:} #1}}
\newcommand{\ekstitle}{Eksempel }
\newcommand{\sprtitle}{Språkboksen}
\newcommand{\expl}{forklaring}
\newcommand{\pyt}{Pytagoras' setning}
\newcommand\sv{\vsk \textbf{Svar} \vspace{4 pt}\\}

%references
\newcommand{\reftab}[1]{\hrs{#1}{tabell}}
\newcommand{\rref}[1]{\hrs{#1}{regel}}
\newcommand{\dref}[1]{\hrs{#1}{definisjon}}
\newcommand{\refkap}[1]{\hrs{#1}{kapittel}}
\newcommand{\refsec}[1]{\hrs{#1}{seksjon}}
\newcommand{\refdsec}[1]{\hrs{#1}{delseksjon}}
\newcommand{\refved}[1]{\hrs{#1}{vedlegg}}
\newcommand{\eksref}[1]{\textsl{#1}}
\newcommand\fref[2][]{\hyperref[#2]{\textsl{figur \ref*{#2}#1}}}
\newcommand{\refop}[1]{{\color{blue}Oppgave \ref{#1}}}
\newcommand{\refops}[1]{{\color{blue}oppgave \ref{#1}}}


%Algebra
\newcommand{\kvadset}{Kvadratsetningene}
\newcommand{\aenato}{Sum-produkt-metoden}

% Geometry
\newcommand{\hlikb}{Midtnormalen i en likebeint trekant}
\newcommand{\arealsetn}{Arealsetningen}
\newcommand{\trkmedian}{Median}
\newcommand{\midtrk}{Midtnormal (i trekant)}
\newcommand{\innskrsirk}{Innskrevet sirkel}
\newcommand{\cossetn}{Cosinussetningen}
\newcommand{\perfvink}{Sentral- og periferivinkel}
\newcommand{\tang}{Tangent}

% Derivative
\newcommand{\derel}{Den deriverte av elementære funksjoner}
\newcommand{\divder}{Divisjonsregelen}
\newcommand{\kjernereg}{Kjerneregelen}
\newcommand{\prodregder}{Produktregelen}
\newcommand{\lhop}{L'Hopitals regel}

% Funksjonsdrofting
\newcommand{\monder}{Monotoniegenskaper og den deriverte}
\newcommand{\fderekstr}{$ \bm{f'=0} $ for lokale ektstremalpunkt}
\newcommand{\andredertest}{Andrederiverttesten}

% Vectors
\newcommand{\detar}{Arealformler med determinanter}
\newcommand{\avstpunktlin}{Avstand mellom punkt og linje}

%Appendix
\newcommand{\rolle}{Rolles teorem}
\newcommand{\meanval}{Middelverdisetningen}

% Solutions manual
\newcommand{\selos}{Se løsningsforslag.}

\begin{document}
\section{GeoGebra}
\subsection{Introduksjon}
Når du åpner GeoGebra får du et bilde som dette:
\begin{figure}[H]
	\centering
	\includegraphics[scale=0.1]{ggbalgoggraf}
\end{figure}
Feltet hvor det står ''Skriv inn'' kalles \textit{inntastingsfeltet}. Dette feltet og det blanke feltet under utgjør \textit{algebrafeltet}. Koordinatsystemet til høyre kalles \textit{grafikkfeltet}.

\subsection{Å skrive inn punkt, funksjoner og linjer}
\subsubsection{Punkt}
Si at vi ønsker å få punktene $ (1,3) $ og $ (4,5) $ til å vises i grafikkfeltet. I inntastingsfeltet skriver vi da
\g{(1,3)}
og \vs
\g{(4,5)}
GeoGebra kaller da punktene $ A $ og $ B $, og tegner dem inn i grafikfeltet:
\begin{figure}[H]
	\centering
	\includegraphics[scale=0.15]{pointAandB}
\end{figure}
Ønsker vi å selv et punkts navn kan vi f. eks skrive
\g{P=(2,4)}
\begin{figure}[H]
	\centering
	\includegraphics[scale=0.15]{pointP}
\end{figure}
\subsubsection{Funksjoner}
Si vi har funksjonen 
\[f(x)= \frac{3}{2} x^2 + 3x \]
For å bruke $ f(x) $ i GeoGebra, skriver vi:
\g{3/2*x\^{}2+3x}
Når vi ikke gir funksjonen noen navn, vil GeoGebra automatisk gi funksjonen navnet $ f $. I algebrafeltet får vi derfor
\begin{figure}[H]
	\centering
	\includegraphics[scale=0.5]{skrivf}
\end{figure}
I grafikkfeltet får vi grafen til $ f $. \vsk

Hvis vi isteden har funksjonen
\[ P(x)= 0,15x^3 - 0,4 x\]
er det to ting vi må passe på. Det første er at \textsl{alle desimaltall må skrives med punktum istedenfor komma} i GeoGebra
. Det andre er at vi ønsker å gi funksjonen navnet $ P(x) $. Vi skriver da
\g{P(x) = 0.15x\^{}3 - 0.4x}
og får \vspace{-5pt}
\begin{figure}[H]
	\centering
	\includegraphics[scale=0.5]{pfig}
\end{figure}
\info{Obs!}{
Man kan aldri gi funksjoner navnet $ {y(x)} $ i GeoGebra. $ y $ kan bare brukes når man skriver inn uttrykk for en rett linje, altså $ {y=ax +b} $, hvor $ a $ og $ b $ er to valgfrie tall.
}

\subsubsection{Vannette og loddrette linjer}

Ønser vi å lage ei linje som går vannrett gjennom verdien 3 på $ y $-aksen og ei linje som går loddrett gjennom verdien 2 på $ x $-aksen skriver vi:
\g{y = 3}
og 
\g{x = 2 }
Da får vi denne figuren:
\begin{figure}[H]
	\centering
	\includegraphics[scale=0.5]{23}
\end{figure}

\subsection{Å finne verdien til funksjoner og linjer}
\subsubsection{Funksjoner}
Si vi har funksjonen
\[H(x)= x^2 + 3x -3 \]
Hvis vi ønsker å vite hva $ H(2) $ er, skriver vi
\g{H(2)}
som resulterer i dette
\begin{figure}[H]
	\centering
	\includegraphics[scale=0.5]{H}
\end{figure}
Da vet vi at $ H(2)=7 $.
\subsubsection{Linjer}
Det anbefales på det sterkeste at du bruker funksjonsuttrykk når du behandler linjer i GeoGebra, men i noen tilfeller kommer man ikke utenom linjer på former $ y=ax+b $. \vsk

La oss se på de to linjene \vs
\alg{
	y&= x-3 \vn
	y&= -2x+1
}
Vi skriver disse inn i GeoGebra, og får
\begin{figure}[H]
	\centering
	\includegraphics[scale=0.5]{fglin1}
\end{figure}
Ønsker vi nå å finne hva verdien til $ {y=x-3} $ er når $ {x=2} $, må vi legge merke til at GeoGebra har kalt denne linja for $ f $. Svaret vi søker får vi da ved å skrive $ f(2) $. Ønsker vi samtidig å vite hva $ {y=-2x+1} $ er når $ {x=0} $, må vi skrive $ g(0) $:
\begin{figure}[H]
	\centering
	\includegraphics[scale=0.6]{fglin2}
\end{figure}

\newpage
\subsection{Knapper og kommandoer}

\subsubsection*{Grafikkfelt}
Knappene velges fra rullemenyer på verktøylinjen. Nummereringen av menyene er fra venstre.\vsk

\begin{tabular}{@{}l}
	\,\includegraphics[scale=0.4]{fig/pkt} Lager et nytt punkt. (Meny nr. 1) \\
	\,\includegraphics[scale=0.4]{fig/lin} Lager linje mellom to punkt. (Meny nr. 2)\\	
	\,\includegraphics[scale=0.4]{fig/ekst} Finner topp- og bunnpunkt til en funksjon. (Meny nr. 2)\\
	\,\includegraphics[scale=0.4]{fig/nul} Finner nullpunktene til en funksjon. (Meny nr. 2)	\\
	\,\includegraphics[scale=0.4]{fig/skj} Finner skjæringspunkt mellom to objekt. (Meny nr. 3)\\	
	\,\includegraphics[scale=0.4]{fig/vek} Lager vektoren mellom to punkt (Meny nr. 3)\\		
	\,\includegraphics[scale=0.4]{fig/tekst} Lager en tekstboks. (Meny nr. 10)\\		
	\,\includegraphics[scale=0.4]{fig/flytt} Flytter grafikkfeltet. Endrer verdiavstanden hvis man peker på aksene. \\
	\hspace{1cm}(Meny nr. 10)\\			
\end{tabular}

\subsubsection{Hurtigtaster}
\begin{tabular}{@{}c | c |c | c }
	&\textbf{Beskrivelse} & \textbf{PC }& \textbf{Mac} \\ \hline
	$ \sqrt{} $	& kvadratrot& \texttt{alt\,+\,r} &\texttt{alt\,+\,r} \\\hline
	$ \pi $	& pi& \texttt{alt\,+\,p} & \texttt{alt\,+\,p}\\\hline
	$ \infty $ &uendelig& \texttt{alt\,+\,u} &\texttt{alt\,+\,,}  \\\hline
	$ \otimes $&kryssprodukt & \texttt{alt\,+\,shift\,+\,8}&\texttt{ctrl\,+\,shift\,+\,8} \\\hline
	$ e $&eulers tall & \texttt{alt\,+\,e}& \texttt{alt\,+\,e}\\\hline
	$ {}^\circ $&gradtegnet ($ \frac{\pi}{180} $) & \texttt{alt\,+\,o}& \texttt{alt\,+\,o}
	\\\hline	
\end{tabular}
\newpage
\subsubsection{Videoer}
\begin{itemize}
\item \net{https://drive.google.com/file/d/1u9u6eZFW8yqtM2ukdW9716i1_YH0NMOa/view?usp=sharing}{Finne nullpunktene til en graf}
\item \net{https://drive.google.com/file/d/11B7o3WMrbR5cpfshkv9FSBpQ5khBCwLS/view?usp=sharing}{Finne lokale bunnpunkt (eller toppunkt) til en graf}
\item \net{https://drive.google.com/file/d/1koy-ffEejtIt-YbIohoFN4gbMLi-95cb/view?usp=sharing}{Finne skjæringspunktene til to funksjoner}
\item \net{https://drive.google.com/file/d/1Z8v05XjlqsSFpHEed5q4s0nGDjdQUfqs/view?usp=sharing}{Justere akser}
\item \net{https://drive.google.com/file/d/1oKKDk084IEhy11rNhDk0-VykzXLuhthy/view?usp=sharing}{Endre tykkelse, farge o.l på graf}
\item \net{https://drive.google.com/file/d/1568rRj2PxzwK6wSm7v4w3GWzDnGtSuuX/view?usp=sharing}{Tegne graf på gitt intervall}	\\
{\footnotesize I videoen tegner vi $f(x)= x^2-3x+2 $ på intervallet $ 0\leq x \leq 5 $}.
\item \net{https://drive.google.com/file/d/1x7y7DJMJ-7rfpgapB48e5Nlq6UnXXcI2/view?usp=sharing}{Lage linje mellom to punkt}	\\
{\footnotesize Legg merke til hva som gjøres mot slutten av videoen for å få det vante uttrykket $ y=ax+b $}.
\item \net{https://drive.google.com/file/d/18cNeX3vJFEY7BpgrH_BFpI1q4c-bczyH/view?usp=sharing}{Utføre regresjon}\\
{\footnotesize I videoen har vi på forhånd skrevet inn tallene i tabellen under, som viser elbilsalget i Norge antall år etter 2010. Disse tallene ble også brukt i \refsec{Regresjon}. \os
Det utføres regresjon med en linje, en kvadratisk funksjon og en 4. grads funksjon.
	
\begin{tabular}{c|c}
	\textbf{antall år} & \textbf{elbiler} \\ \hline
	0 &	3347\\
	1 &	5381\\
	2 &	9565\\
	3 &	19678\\
	4 &	42356\\
	5 &	73312\\
	6 &	101126\\
	7 &	138477\\
	8 &	194900\\
	9 &	260688\\
	10 &337201\\
	11 &	455271\\
\end{tabular}
}



\end{itemize}
\newpage
\subsubsection{Kommandoliste}
\mers{Mange av kommandoene har egne knapper, som blant annet vist i videoene over.}
\begin{itemize}
	\item \cmds{abs( <x> )}{Gir lengden til $ x $ (et tall, et linjestykke o.l.). Alternativt kan man skrive {\tt{|x|}}}.
	
	\item \cmds{Linje( <Punkt>, <Punkt> )}{Gir linjen mellom to punkt.}
	
	\item \cmds{Ekstremalpunkt( <Funksjon>, <Start>, <Slutt> )}{Finner lokale topp- og bunnpunkt for en funksjon på et gitt intervall.}
	
	\item \cmds{Funksjon( <Funksjon>, <Start>, <Slutt> )}{Tegner en funksjon innenfor et gitt intervall.}
	
	\item \cmds{Mangekant( <Punkt>, ..., <Punkt> )}{Tegner mangekanten mellom gitte punkt.}
	
	\item \cmds{Nullpunkt( <Funksjon>, <Start>, <Slutt> )}{Gir nullpunktene til en funksjon innenfor et gitt intervall}
	
	\item \cmds{RegLin( <Liste> )}
	{Bruker regresjon med en rett linje for å tilpasse punkt gitt i en liste.}
	
	\item \cmds{RegEksp( <Liste> )}
	{Bruker regresjon med en eksponentialfunksjon for å tilpasse punkt gitt i en liste.}
	
	\item \cmds{RegPoly( <Liste>, <Grad> )}
	{Bruker regresjon med et polynom av gitt grad for å tilpasse punkt gitt i en liste.}
	
	\item \cmds{RegPot( <Liste> )}
	{Bruker regresjon med en potensfunksjon for å tilpasse punkt gitt i en liste.}
	
	\item \cmds{Skjæring( <Objekt>, <Objekt> )}{Finner skjæringspunktene til to objekt (funksjoner, linjer o.l.)}
	
\end{itemize}



\end{document}

