\documentclass[english,hidelinks,pdftex, 11 pt, class=report,crop=false]{standalone}
\usepackage[T1]{fontenc}
\usepackage[utf8]{luainputenc}
\usepackage{lmodern} % load a font with all the characters
\usepackage{geometry}
\geometry{verbose,a4paper, inner=0cm, outer=0 cm, bmargin=2cm, tmargin=1cm}
%\textwidth=12cm
\setlength{\parindent}{0bp}
\usepackage{import}
\usepackage[subpreambles=false]{standalone}
\usepackage{amsmath}
\usepackage{amssymb}
\usepackage{esint}
\usepackage{babel}
\usepackage{tabu}
\usepackage[dvipsnames, table]{xcolor}
\usepackage{cancel}
\makeatother
\makeatletter
\usepackage{datetime2}
\usepackage{titlesec}
\usepackage[many]{tcolorbox}

% Eheter
\newcommand{\enh}[1]{\,\textrm{#1}}
%referances
\newcommand{\net}[2]{{\color{blue}\href{#1}{#2}}}

%Spaces
\newcommand{\vsk}{\\[12pt]}
\newcommand{\vs}{\vspace{-12pt}}

% Tabell for opplegg

\newcommand{\ovlist}[1]{
\vspace{-16pt}
\begin{itemize}
	#1
\end{itemize}
}

% Chapters and sections
\titleformat{\section}[block]{\bfseries}{\hspace{3cm}\thesection}{5pt}{}
\titleformat{\subsection}[block]{\bfseries}{\hspace{3cm}\thesection}{5pt}{}
\newcommand{\sectionbreak}{\clearpage} % New page on each section
 

\newlength{\mywidth}
\setlength{\mywidth}{14cm}

\newcommand{\cont}[1]{
\begin{tcolorbox}[center, boxrule=0.0 mm, width=\mywidth,arc=0mm,enhanced jigsaw,,colback=white,breakable]
#1	
\end{tcolorbox}
}

\newcommand{\info}[5]{
\begin{tcolorbox}[center, boxrule=0.1 mm, width=\mywidth,arc=0mm,enhanced jigsaw,breakable,colback=yellow!5]	
	
	\footnotesize
	\textbf{Øvingsområde}\\[5pt] #1 
	
	\textbf{Utstyr}\\ #2  \\
	
	\begin{tabular}{@{} p{4cm} p{4cm} l} 
		\textbf{Tid} & \textbf{Elevinndeling} & \textbf{Læringsarena} \\
		#3  & #4 & #5
	\end{tabular} 
\end{tcolorbox}	
}

\newcommand{\gjen}[1]{\begin{tcolorbox}[center,boxrule=0.1 mm, width=\mywidth,arc=0mm,colback=blue!3] {\large \textbf{Gjennomføring} \vspace{5 pt}}\newline #1  \end{tcolorbox}\vspace{-5pt}}
\newcommand{\eks}[1]{\begin{tcolorbox}[center,boxrule=0.1 mm, width=\mywidth,arc=0mm,colback=green!3] {\large \textbf{Eksempel} \vspace{5 pt}}\newline #1  \end{tcolorbox}\vspace{-5pt}}

\newcounter{opl}
%\numberwithin{opl}{article}


\newcommand{\opl}[1]{
\newpage
{\refstepcounter{opl} %\phantomsection 
\large \textbf{\theopl \;#1} \vsk}
}

% Headlines
\newcommand{\fork}{\textbf{Forkunnskapar}\\}
\newcommand{\forb}{\textbf{Forberedelsar}\\}
\newcommand{\opgvr}{\textbf{Oppgaver}}



%colors
\newcommand{\colr}[1]{{\color{red} #1}}
\newcommand{\colb}[1]{{\color{blue} #1}}
\newcommand{\colo}[1]{{\color{orange} #1}}
\newcommand{\colc}[1]{{\color{cyan} #1}}
\definecolor{projectgreen}{cmyk}{100,0,100,0}
\newcommand{\colg}[1]{{\color{projectgreen} #1}}

% Lister med bokstavar
\usepackage[inline]{enumitem}
% Opg
\newcommand{\abc}[1]{
	\begin{enumerate}[label=\alph*),leftmargin=18pt]
		#1
	\end{enumerate}
}

\usepackage[]{hyperref}

\newcommand{\note}{Merk}
\newcommand{\notesm}[1]{{\footnotesize \textsl{\note:} #1}}
\newcommand{\ekstitle}{Eksempel }
\newcommand{\sprtitle}{Språkboksen}
\newcommand{\expl}{forklaring}
\newcommand{\pyt}{Pytagoras' setning}
\newcommand\sv{\vsk \textbf{Svar} \vspace{4 pt}\\}

%references
\newcommand{\reftab}[1]{\hrs{#1}{tabell}}
\newcommand{\rref}[1]{\hrs{#1}{regel}}
\newcommand{\dref}[1]{\hrs{#1}{definisjon}}
\newcommand{\refkap}[1]{\hrs{#1}{kapittel}}
\newcommand{\refsec}[1]{\hrs{#1}{seksjon}}
\newcommand{\refdsec}[1]{\hrs{#1}{delseksjon}}
\newcommand{\refved}[1]{\hrs{#1}{vedlegg}}
\newcommand{\eksref}[1]{\textsl{#1}}
\newcommand\fref[2][]{\hyperref[#2]{\textsl{figur \ref*{#2}#1}}}
\newcommand{\refop}[1]{{\color{blue}Oppgave \ref{#1}}}
\newcommand{\refops}[1]{{\color{blue}oppgave \ref{#1}}}


%Algebra
\newcommand{\kvadset}{Kvadratsetningene}
\newcommand{\aenato}{Sum-produkt-metoden}

% Geometry
\newcommand{\hlikb}{Midtnormalen i en likebeint trekant}
\newcommand{\arealsetn}{Arealsetningen}
\newcommand{\trkmedian}{Median}
\newcommand{\midtrk}{Midtnormal (i trekant)}
\newcommand{\innskrsirk}{Innskrevet sirkel}
\newcommand{\cossetn}{Cosinussetningen}
\newcommand{\perfvink}{Sentral- og periferivinkel}
\newcommand{\tang}{Tangent}

% Derivative
\newcommand{\derel}{Den deriverte av elementære funksjoner}
\newcommand{\divder}{Divisjonsregelen}
\newcommand{\kjernereg}{Kjerneregelen}
\newcommand{\prodregder}{Produktregelen}
\newcommand{\lhop}{L'Hopitals regel}

% Funksjonsdrofting
\newcommand{\monder}{Monotoniegenskaper og den deriverte}
\newcommand{\fderekstr}{$ \bm{f'=0} $ for lokale ektstremalpunkt}
\newcommand{\andredertest}{Andrederiverttesten}

% Vectors
\newcommand{\detar}{Arealformler med determinanter}
\newcommand{\avstpunktlin}{Avstand mellom punkt og linje}

%Appendix
\newcommand{\rolle}{Rolles teorem}
\newcommand{\meanval}{Middelverdisetningen}

% Solutions manual
\newcommand{\selos}{Se løsningsforslag.}

\usepackage{xr}
\externaldocument{../AM_bm}

\begin{document}
\subsubsection*{Kapittel\ref{Storlogenh}}
\opr{opggjeromtilm}
\abch{
	\item 484\,000\enh{m}
	\item 91\,000\enh{m}
	\item 2\,402\,000\enh{m}
}

\opr{opggjeromtilg}
\abch{
	\item 484\,000\enh{g}
	\item 9\,100\enh{g}
	\item 240\,200\enh{g}
}

\opr{opggjeromtilL}
\abch{
	\item 48\enh{l}
	\item 91\enh{l}
	\item 240\enh{cl}
}

\opr{opggjeromblanda} \vs
\begin{multicols}{3}
	\abc{
		\item 0,0124\enh{km}
		\item 4,2\enh{m}
		\item 581,5\enh{mm}
		\item 7,4\enh{m}
		\item 15\enh{cm}
		
		\item 0,097\enh{hg}
		\item 0,00015\enh{g}
		\item 141\,900\enh{mg}
		\item 0,00031\enh{hg}
		\item 0,064039\enh{kg}
		\item 8,9\enh{l}
		\item 69\,140\enh{cl}
		\item 15000\enh{ml}
		\item 9,18\enh{l}
		\item 55\enh{ml}
	}
\end{multicols}

\opr{opgvolprisl} $ 720\enh{cm}^3 $


\opr{opgvolkjegl}
\abch{
	\item $ 32\enh{dm}^3 $
	\item $ 32\enh{l} $
}

\opr{opgvolpyrl}
\abch{
	\item $ 120\enh{cm}^3 $
	\item $ 0,12\enh{l} $
}

\nes

\opr{opgstorlbolt}
\abch{
\item Ca. $ 10,19\enh{m/s} $
\item Han startet med $ 0\enh{m/s} $ som fart, og trengte de første metrene til å akselerere.
\item Ca. $ 12,35\enh{m/s} $.
}

\nes
\opr{opgstorlbolt2} Ca.  $ 36,68\enh{m/s} $ og ca. $ 44.46\enh{m/s} $

\opr{opgstorlfinn} 
Skriv ned eksempel på et dyr, et insekt, en gjenstand eller annet som veier mellom 1-100\,mg, cg, dg, g, dag, hg og kg.


\subsubsection*{Kapittel \ref{Statistikk}}
\opr{opgstatsentrodd1}
\abch{
	\item 2
	\item 3
	\item 5 
}

\opr{opgstatsentrodd2}
\abch{
\item 8 
\item 6 
\item $ \frac{67}{11}$
}

\opr{opgstatsentrpar1}
\abch{
\item 5, 8 og 16
\item 8
\item $ 8,5 $
}

\opr{opgstatsentrpar2}
\abch{
\item 5 og 11
\item 8.5
\item 9
}

\opr{opgrikest}
\abch{
\item 3,2
\item 4185.48
\item Medianen
}


\opr{opgfrkvtb1} \selos

\opr{opgfrkvtb2} \selos

\opr{opgstatsoyl}

\opr{statsoylrgn} 
\newpage

\opr{stathvorfor}
Av de fire undersøkelsene på side \pageref{undersok}, hvorfor har vi
\abc{
	\item I undersøkelse 1 har hver verdi frekvens lik 1, og da er det unødvendig å lage en frekvenstabell. Punktene i undersøkelse 3 gir samme informasjon som en frekvenstabell. Informasjonen gitt i undersøkelse 4 er allerede gitt i form av en frekvenstabell.
	\item vist søylediagram bare for undersøkelse  2 og 3?
	\item vist sektordiagram bare for undersøkelse 2 og ?
	\item vist linjediagram bare for undersøkelse 4?
}

\opr{stathvorfor2} Spredningsmål gir bare mening for tallverdier.

\subsection*{Kapittel \label{Brok}}
\opr{finbrokdel}
\abch{
	\item  6
	\item  15
	\item  42
	\item  80
}

\opr{finnbrokdel2}
\abch{
	\item $ \frac{8}{15} $
	\item $ \frac{48}{77} $
	\item $ \frac{9}{65} $ 
}

\opr{brokfirma} 320\,000\,kr


\opr{sktilpro}
\abch{
	\item $ 78\% $ 
	\item $ 91,2\%$
	\item $ 0,7\% $
	\item $ 193,54\% $ 
}

\opr{proverdi}
\abch{
	\item 0,57 
	\item 0,981
	\item 2,19
	\item 0,003
}

\opr{sktilpro}
\abch{
	\item $ 70\% $
	\item $ 22\% $ 
	\item $ 36\% $
	\item $ 145\% $
}

\op{finnpro}
Finn \os
\abch{
	\item 100
	\item 250
	\item 63
	\item 560
	\item 30
}

\newpage
\op{proav}
\abch{
	\item 40\%
	\item 25\%
	\item ca 42,86\%
	\item ca 22,22\%
}

\opr{proundsk2}

\opr{prook} 
\abch{
	\item 44
	\item 325
	\item 1008
	\item 649
	\item 200 
}

\op{prored} \vs
\abc{
	\item 36
	\item 175
	\item 112
}
\op{bitcoin}
Du kjøper en hest for 20\,000\enh{kr}, og håper at verdien til hesten vil stige med 8\% i løpet av et år. Hvor mye er den i så fall verd da?

\op{pckjop}
Du kjøper en ny gaming-PC til 20\,000\enh{kr}, og regner med at verdien til PCen vil synke med 12\% i løpet av et år. Hvor mye er den i så fall verd da?

\op{opgbrokpp}
Si at originalprisen på en bukse er 500\enh{kr}. Først ble det gitt 20\% rabatt på denne prisen, men etter en stund ble det gitt 30\% rabatt. Avgjør hvilke av utsagnene under som er sann/ikke sann
\begin{enumerate}[label=(\roman*)]
	\item Når rabatten gikk fra å være 20\% til å være 30\%, ble originalprisen redusert med 10\%.
	\item Når rabatten gikk fra å være 20\% til å være 30\%, økte rabatten med 10\%.
	\item Når rabatten gikk fra å være 20\% til å være 30\%, økte rabatten med 10 prosentpoeng.	
\end{enumerate}

\nes

\op{finnvekstf1} \vs
\abc{
	\item Finn vekstfaktoren fra oppgave \ref{prook}a).
	\item Finn vekstfaktoren fra oppgave \ref{prook}b).
	\item Finn vekstfaktoren fra oppgave \ref{prook}c).
}

\op{finnvesktf2} \vs
\abc{
	\item Finn vekstfaktoren fra oppgave \ref{prored}a).
	\item Finn vekstfaktoren fra oppgave \ref{prored}b).
	\item Finn vekstfaktoren fra oppgave \ref{prored}c).
}

\opr{forh2}
Finn forholdet og forholdstallet mellom antall hester og griser når vi har:\os
\begin{tabular}{@{}l l l}	
	\textbf{a)} 5 hester og 2 griser. &\textbf{b)} 12 griser og 4 hester.
\end{tabular}

\newpage
\opr{forh2}


\vsk \vspace{12pt}
\begin{comment}
Oppgave om hvilket dyr som er sterkest i forhold til vekten. Skaraben er verdens sterkeste.
\end{comment}




\op{forh}
De fleste brus inneholder ca 10\enh{g} karbohydrater per 100\enh{g}. En type saftsirup inneholder 44\enh{g} karbohydrater per 100\enh{g}. Saften skal lages med 2 deler sirup og 9 deler vann. \os

Inneholder saften mer eller mindre karbohydrater per 100\,g enn en brus? \os

\mers{I denne oppgaven går vi ut ifra at både 1\,dl vann og 1\,dl saftsirup veier 100\,g.}


\end{document}

