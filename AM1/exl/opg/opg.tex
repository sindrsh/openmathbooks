\documentclass[english,hidelinks,pdftex, 11 pt, class=report,crop=false]{standalone}
\usepackage[T1]{fontenc}
\usepackage[utf8]{luainputenc}
\usepackage{lmodern} % load a font with all the characters
\usepackage{geometry}
\geometry{verbose,a4paper, inner=0cm, outer=0 cm, bmargin=2cm, tmargin=1cm}
%\textwidth=12cm
\setlength{\parindent}{0bp}
\usepackage{import}
\usepackage[subpreambles=false]{standalone}
\usepackage{amsmath}
\usepackage{amssymb}
\usepackage{esint}
\usepackage{babel}
\usepackage{tabu}
\usepackage[dvipsnames, table]{xcolor}
\usepackage{cancel}
\makeatother
\makeatletter
\usepackage{datetime2}
\usepackage{titlesec}
\usepackage[many]{tcolorbox}

% Eheter
\newcommand{\enh}[1]{\,\textrm{#1}}
%referances
\newcommand{\net}[2]{{\color{blue}\href{#1}{#2}}}

%Spaces
\newcommand{\vsk}{\\[12pt]}
\newcommand{\vs}{\vspace{-12pt}}

% Tabell for opplegg

\newcommand{\ovlist}[1]{
\vspace{-16pt}
\begin{itemize}
	#1
\end{itemize}
}

% Chapters and sections
\titleformat{\section}[block]{\bfseries}{\hspace{3cm}\thesection}{5pt}{}
\titleformat{\subsection}[block]{\bfseries}{\hspace{3cm}\thesection}{5pt}{}
\newcommand{\sectionbreak}{\clearpage} % New page on each section
 

\newlength{\mywidth}
\setlength{\mywidth}{14cm}

\newcommand{\cont}[1]{
\begin{tcolorbox}[center, boxrule=0.0 mm, width=\mywidth,arc=0mm,enhanced jigsaw,,colback=white,breakable]
#1	
\end{tcolorbox}
}

\newcommand{\info}[5]{
\begin{tcolorbox}[center, boxrule=0.1 mm, width=\mywidth,arc=0mm,enhanced jigsaw,breakable,colback=yellow!5]	
	
	\footnotesize
	\textbf{Øvingsområde}\\[5pt] #1 
	
	\textbf{Utstyr}\\ #2  \\
	
	\begin{tabular}{@{} p{4cm} p{4cm} l} 
		\textbf{Tid} & \textbf{Elevinndeling} & \textbf{Læringsarena} \\
		#3  & #4 & #5
	\end{tabular} 
\end{tcolorbox}	
}

\newcommand{\gjen}[1]{\begin{tcolorbox}[center,boxrule=0.1 mm, width=\mywidth,arc=0mm,colback=blue!3] {\large \textbf{Gjennomføring} \vspace{5 pt}}\newline #1  \end{tcolorbox}\vspace{-5pt}}
\newcommand{\eks}[1]{\begin{tcolorbox}[center,boxrule=0.1 mm, width=\mywidth,arc=0mm,colback=green!3] {\large \textbf{Eksempel} \vspace{5 pt}}\newline #1  \end{tcolorbox}\vspace{-5pt}}

\newcounter{opl}
%\numberwithin{opl}{article}


\newcommand{\opl}[1]{
\newpage
{\refstepcounter{opl} %\phantomsection 
\large \textbf{\theopl \;#1} \vsk}
}

% Headlines
\newcommand{\fork}{\textbf{Forkunnskapar}\\}
\newcommand{\forb}{\textbf{Forberedelsar}\\}
\newcommand{\opgvr}{\textbf{Oppgaver}}



%colors
\newcommand{\colr}[1]{{\color{red} #1}}
\newcommand{\colb}[1]{{\color{blue} #1}}
\newcommand{\colo}[1]{{\color{orange} #1}}
\newcommand{\colc}[1]{{\color{cyan} #1}}
\definecolor{projectgreen}{cmyk}{100,0,100,0}
\newcommand{\colg}[1]{{\color{projectgreen} #1}}

% Lister med bokstavar
\usepackage[inline]{enumitem}
% Opg
\newcommand{\abc}[1]{
	\begin{enumerate}[label=\alph*),leftmargin=18pt]
		#1
	\end{enumerate}
}

\usepackage[]{hyperref}

\begin{document}
\phantomsection 
\addcontentsline{toc}{section}{Oppgaver} 

\opgt

(Oppgavene under gjøres i regneark) \vsk

\op{gjorop}
Gjør oppgave \ref{borge4} og \ref{nora}.

\op{serielaan}
\textbf{a)} Sett opp et serielån hvor:
\begin{itemize}
	\item Lånesummen er 300\,000\,kr
	\item Renten er 2,1\%
	\item Lånet skal betales med 15 årlige terminbeløp.
\end{itemize}
Avrund alle kronebeløp til hele kroner.\os

\textbf{b)} Hvor mye koster lånet totalt? (Summen av alle terminbeløpene.)

\op{anulaan}
\textbf{a)} Sett opp et annuitetslån hvor:
\begin{itemize}
	\item Lånesummen er 300\,000\,kr
	\item Renten er 2,1\%
	\item Lånet skal betales med 15 årlige terminbeløp, som er 23\,523\,kr.
\end{itemize}
Avrund alle kronebeløp til hele kroner.\os

\textbf{b)} Hvor mye koster lånet totalt? \os
\textbf{c)} Sammenlign svaret du fikk i oppgave b) med svaret fra oppgave E\ref{serielaan}b, hvilket lån koster mest penger?\os

\op{sjekk}
Sjekk at du i oppgave E\ref{serielaan} og E\ref{anulaan} har fåt samme svar som nettsiden \net{https://www.laanekalkulator.no/}{laanekalkulator.no}. (Velg \textsl{Tinglysning: Ingen} og sett alle gebyrer til 0).
\newpage
\op{h17_d2_9}
(Oppgaven er hentet fra del 2, eksamen høsten 2017.)\vsk \\
\includegraphics[scale=0.9]{h17_d2_9}

\newpage
\op{h16_d2_5}
(Oppgaven er hentet fra del 2, eksamen våren 2016.)
\begin{figure}
	\centering
	\includegraphics[scale=0.75]{v16_d2_5}
\end{figure}
\newpage
\op{h16_d2_6}
(Oppgaven er hentet fra del 2, eksamen våren 2016.)
\begin{figure}
	\centering
	\includegraphics[scale=0.75]{v16_d2_6}
\end{figure}
\end{document}