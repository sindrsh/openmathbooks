\documentclass[english,hidelinks,pdftex, 11 pt, class=report,crop=false]{standalone}
\usepackage[T1]{fontenc}
\usepackage[utf8]{luainputenc}
\usepackage{lmodern} % load a font with all the characters
\usepackage{geometry}
\geometry{verbose,a4paper, inner=0cm, outer=0 cm, bmargin=2cm, tmargin=1cm}
%\textwidth=12cm
\setlength{\parindent}{0bp}
\usepackage{import}
\usepackage[subpreambles=false]{standalone}
\usepackage{amsmath}
\usepackage{amssymb}
\usepackage{esint}
\usepackage{babel}
\usepackage{tabu}
\usepackage[dvipsnames, table]{xcolor}
\usepackage{cancel}
\makeatother
\makeatletter
\usepackage{datetime2}
\usepackage{titlesec}
\usepackage[many]{tcolorbox}

% Eheter
\newcommand{\enh}[1]{\,\textrm{#1}}
%referances
\newcommand{\net}[2]{{\color{blue}\href{#1}{#2}}}

%Spaces
\newcommand{\vsk}{\\[12pt]}
\newcommand{\vs}{\vspace{-12pt}}

% Tabell for opplegg

\newcommand{\ovlist}[1]{
\vspace{-16pt}
\begin{itemize}
	#1
\end{itemize}
}

% Chapters and sections
\titleformat{\section}[block]{\bfseries}{\hspace{3cm}\thesection}{5pt}{}
\titleformat{\subsection}[block]{\bfseries}{\hspace{3cm}\thesection}{5pt}{}
\newcommand{\sectionbreak}{\clearpage} % New page on each section
 

\newlength{\mywidth}
\setlength{\mywidth}{14cm}

\newcommand{\cont}[1]{
\begin{tcolorbox}[center, boxrule=0.0 mm, width=\mywidth,arc=0mm,enhanced jigsaw,,colback=white,breakable]
#1	
\end{tcolorbox}
}

\newcommand{\info}[5]{
\begin{tcolorbox}[center, boxrule=0.1 mm, width=\mywidth,arc=0mm,enhanced jigsaw,breakable,colback=yellow!5]	
	
	\footnotesize
	\textbf{Øvingsområde}\\[5pt] #1 
	
	\textbf{Utstyr}\\ #2  \\
	
	\begin{tabular}{@{} p{4cm} p{4cm} l} 
		\textbf{Tid} & \textbf{Elevinndeling} & \textbf{Læringsarena} \\
		#3  & #4 & #5
	\end{tabular} 
\end{tcolorbox}	
}

\newcommand{\gjen}[1]{\begin{tcolorbox}[center,boxrule=0.1 mm, width=\mywidth,arc=0mm,colback=blue!3] {\large \textbf{Gjennomføring} \vspace{5 pt}}\newline #1  \end{tcolorbox}\vspace{-5pt}}
\newcommand{\eks}[1]{\begin{tcolorbox}[center,boxrule=0.1 mm, width=\mywidth,arc=0mm,colback=green!3] {\large \textbf{Eksempel} \vspace{5 pt}}\newline #1  \end{tcolorbox}\vspace{-5pt}}

\newcounter{opl}
%\numberwithin{opl}{article}


\newcommand{\opl}[1]{
\newpage
{\refstepcounter{opl} %\phantomsection 
\large \textbf{\theopl \;#1} \vsk}
}

% Headlines
\newcommand{\fork}{\textbf{Forkunnskapar}\\}
\newcommand{\forb}{\textbf{Forberedelsar}\\}
\newcommand{\opgvr}{\textbf{Oppgaver}}



%colors
\newcommand{\colr}[1]{{\color{red} #1}}
\newcommand{\colb}[1]{{\color{blue} #1}}
\newcommand{\colo}[1]{{\color{orange} #1}}
\newcommand{\colc}[1]{{\color{cyan} #1}}
\definecolor{projectgreen}{cmyk}{100,0,100,0}
\newcommand{\colg}[1]{{\color{projectgreen} #1}}

% Lister med bokstavar
\usepackage[inline]{enumitem}
% Opg
\newcommand{\abc}[1]{
	\begin{enumerate}[label=\alph*),leftmargin=18pt]
		#1
	\end{enumerate}
}

\usepackage[]{hyperref}

\newcommand{\note}{Merk}
\newcommand{\notesm}[1]{{\footnotesize \textsl{\note:} #1}}
\newcommand{\ekstitle}{Eksempel }
\newcommand{\sprtitle}{Språkboksen}
\newcommand{\expl}{forklaring}
\newcommand{\pyt}{Pytagoras' setning}
\newcommand\sv{\vsk \textbf{Svar} \vspace{4 pt}\\}

%references
\newcommand{\reftab}[1]{\hrs{#1}{tabell}}
\newcommand{\rref}[1]{\hrs{#1}{regel}}
\newcommand{\dref}[1]{\hrs{#1}{definisjon}}
\newcommand{\refkap}[1]{\hrs{#1}{kapittel}}
\newcommand{\refsec}[1]{\hrs{#1}{seksjon}}
\newcommand{\refdsec}[1]{\hrs{#1}{delseksjon}}
\newcommand{\refved}[1]{\hrs{#1}{vedlegg}}
\newcommand{\eksref}[1]{\textsl{#1}}
\newcommand\fref[2][]{\hyperref[#2]{\textsl{figur \ref*{#2}#1}}}
\newcommand{\refop}[1]{{\color{blue}Oppgave \ref{#1}}}
\newcommand{\refops}[1]{{\color{blue}oppgave \ref{#1}}}


%Algebra
\newcommand{\kvadset}{Kvadratsetningene}
\newcommand{\aenato}{Sum-produkt-metoden}

% Geometry
\newcommand{\hlikb}{Midtnormalen i en likebeint trekant}
\newcommand{\arealsetn}{Arealsetningen}
\newcommand{\trkmedian}{Median}
\newcommand{\midtrk}{Midtnormal (i trekant)}
\newcommand{\innskrsirk}{Innskrevet sirkel}
\newcommand{\cossetn}{Cosinussetningen}
\newcommand{\perfvink}{Sentral- og periferivinkel}
\newcommand{\tang}{Tangent}

% Derivative
\newcommand{\derel}{Den deriverte av elementære funksjoner}
\newcommand{\divder}{Divisjonsregelen}
\newcommand{\kjernereg}{Kjerneregelen}
\newcommand{\prodregder}{Produktregelen}
\newcommand{\lhop}{L'Hopitals regel}

% Funksjonsdrofting
\newcommand{\monder}{Monotoniegenskaper og den deriverte}
\newcommand{\fderekstr}{$ \bm{f'=0} $ for lokale ektstremalpunkt}
\newcommand{\andredertest}{Andrederiverttesten}

% Vectors
\newcommand{\detar}{Arealformler med determinanter}
\newcommand{\avstpunktlin}{Avstand mellom punkt og linje}

%Appendix
\newcommand{\rolle}{Rolles teorem}
\newcommand{\meanval}{Middelverdisetningen}

% Solutions manual
\newcommand{\selos}{Se løsningsforslag.}


\begin{document}
\section{Introduksjon til Python}
\pythonut{helloworld.py}{
Hello world!
}

Python deler talll inn i tre typer:
\begin{center}
	\begin{tabular}{r|l}
		\texttt{int} & reelle heltall \\
		\texttt{float} & relle tall \\
		\texttt{complex} & komplekse tall
	\end{tabular} 
\end{center}
Vi skal i denne boka konsentrere oss om \texttt{int} og \texttt{float}. Tallypene definerer vi ved å ekskludere eller inkludere punktum:
\python{intfloat.py}
\pythonut{opr.py}{
a+b =  7\\
a-b =  3\\
a*b =  10\\
a/b =  2.5\\
a**b =  25\\
a//b =  2 \\
a\%b =  1
}
\subsection*{Boolske verdier og vilkår}
De boolske verdiene er verdiene \pytype{True} og \pytype{False}. Disse vil være resultatet når vi sjekker om objekter er like eller ulike. For å sjekke dette har vi de \outl{sammenlignende operatorene}:
\begin{center}
	\begin{tabular}{c|c}
		\textbf{operator} & \textbf{betydning} \\ \hline
		\texttt{==}	& er lik \\ \rowcolor{gray!10}
		\texttt{!=} & er \textsl{ikke} lik\\
		\texttt{>} & er større enn \\ \rowcolor{gray!10}
		\texttt{>=} & er større enn, eller lik \\
		\texttt{<} & er mindre enn \\ \rowcolor{gray!10}
		\texttt{<=} & er mindre enn, eller lik \\		
	\end{tabular}
\end{center}
\pythonut{bool1.py}{
False\\
True\\
True\\
False\\
}\vsk

I tillegg til de sammenlignende operatorene kan vi bruke de \outl{logiske operatorene} \pytype{and}, \pytype{or} og \pytype{not}
\pythonut{bool2.py}{
False\\
True\\
True

}
\spr{
Sjekker som bruker de sammenlignende og de logiske operatorene, skal vi heretter kalle \outl{vilkår}.
}
\subsection*{\pymet{if, else og elif} uttrykkene}
Når vi ønsker å utføre handlinger bare \textsl{hvis} et vilkår er sant (\pytype{True}), bruker vi \pymet{if} uttrykket foran vilkåret. Koden vi skriver med innrykk under \pymet{if}-linjen, vil bare bli utført hvis vilkåret gir \pytype{True}. 
\pythonut{if.py}{
 Jepp, c er større enn b
} 
Hvis man først vil sjekke om et vilkår er sant, og så utføre handlinger hvis det \textsl{ikke} er det, kan vi bruke \pymet{else} uttrykket:
\pythonut{else.py}{
Men denne kommer, fordi vilkåret i if-linja over var False
}
\pymet{else} uttrykket tar bare hensyn til (og gir ikke mening uten) \pymet{if} uttrykket like over seg. Hvis vi vil at handlinger skal utføres \textsl{bare} hvis ingen tidligere \pymet{if} uttrykk ga noe utslag, må vi bruke\footnote{\pymet{elif} er en forkortelse for \pymet{else if}, som også kan brukes.} \pymet{elif} uttrykket. Dette er et \pymet{if} uttrykk som slår inn hvis \pymet{if} uttrykket over \textsl{ikke} ga utslag.
\pythonut{elif.py}{
Nå er vi sikre på at 1 < b < 3
}

\subsection*{Lister}
Lister kan vi bruke for å samle objekter. Objektene som er i listen kalles \outl{elementene} til listen.
\python{list1.py}
Elementene i lister er \outl{indekserte}. Første objekt har indeks 0, andre objekt har indeks 1 og så videre:
\pythonut{list1.py}{
96 \\
99 \\
98
}
Med den innebygde funksjonen \texttt{append()} kan vi legge til et objekt i enden av listen. Dette er en \outl{innebygd funksjon}\footnote{Kort fortalt betyr det at det bare er noen typer objekter som kan bruke denne funksjonen.}, som vi skriver i enden av navnet på listen, med et punktum foran.
\pythonut{list3.py}{
[] \newline
[3] \newline
[3, 7]
}
Med funksjonen \texttt{pop()} kan vi hente ut et objekt fra listen
\pythonut{list4.py}{
a = 19 \\
min\_liste = [6, 10, 15] \\
a = 10 \\
min\_liste = [6, 15]
}
\info{Forklar for deg selv}{
Hva er forskjellen på å skive \texttt{a = min\_liste[1]} og å skrive \texttt{a = min\_liste.pop(1)}?
}

\subsection*{\pymet{for} looper}
For objekter som inneholder flere elementer, kan vi bruke \pymet{for}-looper til å utføre handlinger for hvert element. Handlingene må vi skrive med et innrykk etter \pymet{for}-uttrykket:
\pythonut{for.py}{
5 \\
50 \\[12pt]

10\\
100\\[12pt]

15\\
150
}
\spr{
Å gå gjennom hvert element i (for eksempel) en liste kalles \outl{å iterere over listen}.
}\vsk

Ofte er det ønskelig å iterere over heltallene $ 0, 1, 2 $ og så videre. Til dette kan vi bruke \pymet{range()}-funksjonen:
\pythonut{forrang.py}{
0 \\
1 \\
2
}
\subsection*{\pymet{while} loop}
Hvis vi ønsker at handlinger skal utføres fram til et vilkår er sant, kan vi bruke \pymet{while} metoden:
\pythonut{while1.py}{
1\\
2\\
3\\
4\\
}
\newpage

\info{Obs!}{
	Bruk aldri innebygde funksjoner som navn på variabler.
}


\end{document}