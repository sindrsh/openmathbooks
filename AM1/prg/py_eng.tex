\documentclass[english,hidelinks,pdftex, 11 pt, class=report,crop=false]{standalone}
\usepackage[T1]{fontenc}
\usepackage[utf8]{luainputenc}
\usepackage{lmodern} % load a font with all the characters
\usepackage{geometry}
\geometry{verbose,a4paper, inner=0cm, outer=0 cm, bmargin=2cm, tmargin=1cm}
%\textwidth=12cm
\setlength{\parindent}{0bp}
\usepackage{import}
\usepackage[subpreambles=false]{standalone}
\usepackage{amsmath}
\usepackage{amssymb}
\usepackage{esint}
\usepackage{babel}
\usepackage{tabu}
\usepackage[dvipsnames, table]{xcolor}
\usepackage{cancel}
\makeatother
\makeatletter
\usepackage{datetime2}
\usepackage{titlesec}
\usepackage[many]{tcolorbox}

% Eheter
\newcommand{\enh}[1]{\,\textrm{#1}}
%referances
\newcommand{\net}[2]{{\color{blue}\href{#1}{#2}}}

%Spaces
\newcommand{\vsk}{\\[12pt]}
\newcommand{\vs}{\vspace{-12pt}}

% Tabell for opplegg

\newcommand{\ovlist}[1]{
\vspace{-16pt}
\begin{itemize}
	#1
\end{itemize}
}

% Chapters and sections
\titleformat{\section}[block]{\bfseries}{\hspace{3cm}\thesection}{5pt}{}
\titleformat{\subsection}[block]{\bfseries}{\hspace{3cm}\thesection}{5pt}{}
\newcommand{\sectionbreak}{\clearpage} % New page on each section
 

\newlength{\mywidth}
\setlength{\mywidth}{14cm}

\newcommand{\cont}[1]{
\begin{tcolorbox}[center, boxrule=0.0 mm, width=\mywidth,arc=0mm,enhanced jigsaw,,colback=white,breakable]
#1	
\end{tcolorbox}
}

\newcommand{\info}[5]{
\begin{tcolorbox}[center, boxrule=0.1 mm, width=\mywidth,arc=0mm,enhanced jigsaw,breakable,colback=yellow!5]	
	
	\footnotesize
	\textbf{Øvingsområde}\\[5pt] #1 
	
	\textbf{Utstyr}\\ #2  \\
	
	\begin{tabular}{@{} p{4cm} p{4cm} l} 
		\textbf{Tid} & \textbf{Elevinndeling} & \textbf{Læringsarena} \\
		#3  & #4 & #5
	\end{tabular} 
\end{tcolorbox}	
}

\newcommand{\gjen}[1]{\begin{tcolorbox}[center,boxrule=0.1 mm, width=\mywidth,arc=0mm,colback=blue!3] {\large \textbf{Gjennomføring} \vspace{5 pt}}\newline #1  \end{tcolorbox}\vspace{-5pt}}
\newcommand{\eks}[1]{\begin{tcolorbox}[center,boxrule=0.1 mm, width=\mywidth,arc=0mm,colback=green!3] {\large \textbf{Eksempel} \vspace{5 pt}}\newline #1  \end{tcolorbox}\vspace{-5pt}}

\newcounter{opl}
%\numberwithin{opl}{article}


\newcommand{\opl}[1]{
\newpage
{\refstepcounter{opl} %\phantomsection 
\large \textbf{\theopl \;#1} \vsk}
}

% Headlines
\newcommand{\fork}{\textbf{Forkunnskapar}\\}
\newcommand{\forb}{\textbf{Forberedelsar}\\}
\newcommand{\opgvr}{\textbf{Oppgaver}}



%colors
\newcommand{\colr}[1]{{\color{red} #1}}
\newcommand{\colb}[1]{{\color{blue} #1}}
\newcommand{\colo}[1]{{\color{orange} #1}}
\newcommand{\colc}[1]{{\color{cyan} #1}}
\definecolor{projectgreen}{cmyk}{100,0,100,0}
\newcommand{\colg}[1]{{\color{projectgreen} #1}}

% Lister med bokstavar
\usepackage[inline]{enumitem}
% Opg
\newcommand{\abc}[1]{
	\begin{enumerate}[label=\alph*),leftmargin=18pt]
		#1
	\end{enumerate}
}

\usepackage[]{hyperref}

\newcommand{\note}{Merk}
\newcommand{\notesm}[1]{{\footnotesize \textsl{\note:} #1}}
\newcommand{\ekstitle}{Eksempel }
\newcommand{\sprtitle}{Språkboksen}
\newcommand{\expl}{forklaring}
\newcommand{\pyt}{Pytagoras' setning}
\newcommand\sv{\vsk \textbf{Svar} \vspace{4 pt}\\}

%references
\newcommand{\reftab}[1]{\hrs{#1}{tabell}}
\newcommand{\rref}[1]{\hrs{#1}{regel}}
\newcommand{\dref}[1]{\hrs{#1}{definisjon}}
\newcommand{\refkap}[1]{\hrs{#1}{kapittel}}
\newcommand{\refsec}[1]{\hrs{#1}{seksjon}}
\newcommand{\refdsec}[1]{\hrs{#1}{delseksjon}}
\newcommand{\refved}[1]{\hrs{#1}{vedlegg}}
\newcommand{\eksref}[1]{\textsl{#1}}
\newcommand\fref[2][]{\hyperref[#2]{\textsl{figur \ref*{#2}#1}}}
\newcommand{\refop}[1]{{\color{blue}Oppgave \ref{#1}}}
\newcommand{\refops}[1]{{\color{blue}oppgave \ref{#1}}}


%Algebra
\newcommand{\kvadset}{Kvadratsetningene}
\newcommand{\aenato}{Sum-produkt-metoden}

% Geometry
\newcommand{\hlikb}{Midtnormalen i en likebeint trekant}
\newcommand{\arealsetn}{Arealsetningen}
\newcommand{\trkmedian}{Median}
\newcommand{\midtrk}{Midtnormal (i trekant)}
\newcommand{\innskrsirk}{Innskrevet sirkel}
\newcommand{\cossetn}{Cosinussetningen}
\newcommand{\perfvink}{Sentral- og periferivinkel}
\newcommand{\tang}{Tangent}

% Derivative
\newcommand{\derel}{Den deriverte av elementære funksjoner}
\newcommand{\divder}{Divisjonsregelen}
\newcommand{\kjernereg}{Kjerneregelen}
\newcommand{\prodregder}{Produktregelen}
\newcommand{\lhop}{L'Hopitals regel}

% Funksjonsdrofting
\newcommand{\monder}{Monotoniegenskaper og den deriverte}
\newcommand{\fderekstr}{$ \bm{f'=0} $ for lokale ektstremalpunkt}
\newcommand{\andredertest}{Andrederiverttesten}

% Vectors
\newcommand{\detar}{Arealformler med determinanter}
\newcommand{\avstpunktlin}{Avstand mellom punkt og linje}

%Appendix
\newcommand{\rolle}{Rolles teorem}
\newcommand{\meanval}{Middelverdisetningen}

% Solutions manual
\newcommand{\selos}{Se løsningsforslag.}


\begin{document}
\section{Introduction to Python}
Python is a programming language for \outl{text-based coding}. This means that the actions we want to be executed must be coded as text. The file containing all the code is referred to as a \outl{script}. The visible result of running the script is termed \outl{output}\footnote{\outl{Output} in English.}. There are various ways to run one's script; for example, one can use an online compiler like \net{https://www.programiz.com/python-programming/online-compiler/}{programmiz.com}.
\subsection{Object, Type, Function, and Expression}
Our first script consists of just one line of code:

\pythonut{helloworld.py}{
	Hello world!
}

In the upcoming sections, the terms \outl{object}, \outl{type}, \outl{function}, and \outl{expression} will frequently be discussed.
\begin{itemize}
	\item Almost everything in Python is objects. In the above script, both \pymet{print()} and \pytype{"Hello world"} are objects.
	\item Objects come in different types. \pymet{print()} is of the \pytype{function} type, while \pytype{"Hello world"} is of the \pytype{str} type\footnote{'str' is an abbreviation for the English word 'string'.}. The operations that can be executed with various objects depend on their types.
	\item Functions can accept \outl{arguments} and then perform operations. In the script above, the \pymet{print()} function takes the argument \pytype{"Hello world"} and displays the text as output.
	\item Expressions have strong similarities with functions, but they don't accept arguments.
\end{itemize}

\newpage
\subsubsection{Assignment and Calculation}
Text and numbers can be seen as some of the smallest building blocks (objects). Python has one type for text and two types for real numbers:
\begin{center}
	\begin{tabular}{r|l} \rowcolor{gray!10}
		\pytype{str} & text \\
		\pytype{int} & integer \\ \rowcolor{gray!10}
		\pytype{float} & decimal
	\end{tabular} 
\end{center}
It is usually useful to give our objects names. We do this by writing the name followed by \texttt{=} and the object. \outl{Comments} are text that is not treated as code. We can write comments by starting the sentence with \texttt{\color{codegreen} \#}.
\python{strintfloat.py} \vsk

With Python, we can of course perform classic arithmetic operations: \vspace{4pt}

\pythonut{opr.py}{
	a+b =  7\\
	a-b =  3\\
	a*b =  10\\
	a/b =  2.5\\
	a**b =  25\\
	a//b =  2 \\
	a\%b =  1
} \vsk
\newpage
The functions \pymet{str()}, \pymet{int()} and \pymet{float()} can be used to convert objects to types \pytype{int} or \pytype{float}: 
\pythonut{strintfloatfunk.py}{
	32.0\\
	6\\
	6.0
} \vsk

One important thing to be aware of is that \texttt{=} in Python \textsl{does not} mean the same as \sym{$ = $} in mathematics. While $ \sym{=} $ can be translated to ''equals'', we can say that \texttt{=} can be translated to 'is assigned to'.
\pythonut{assign.py}{
	5 \\
	6 
} \newpage
For an object to add itself and another value is so common in programming that Python has its own operator for it:
\pythonut{aplus1.py}{
	5\\
	6
}\vsk

Although computers are extremely fast at performing calculations, they have a limitation that is important to be aware of: rounding errors. One reason for this is that computers can only use a certain number of decimals to represent numbers. Another reason is that computers use the \net{https://en.wikipedia.org/wiki/Binary_number}{binary system}. There are many values that we can write exactly in the decimal system that cannot be written exactly in the binary system. To address this, we can use the \texttt{round()}\label{round()} function:\regv

\pythonut{rnd.py}{
	0.8300000000000001\\
	0.83
}
\newpage
\subsection{Custom Functions}
Using the method \pymet{def}, you can create your own functions. A function can perform actions, and it can \outl{return} one or more objects. It can also accept arguments. The code we write inside a function is only executed if we \outl{call} the function. \regv

\pythonut{func.py}{
	Hi. Someone called function b. The argument given was: Hello! \\
	5
}

\subsection{Boolean Values and Conditions}
The values \pytype{True} and \pytype{False} are called \outl{boolean values}. These will be the result when we check if objects are equal or different. To check this, we have the \outl{comparative operators}:
\begin{center}
	\begin{tabular}{c|c}
		\textbf{operator} & \textbf{meaning} \\ \hline
		\texttt{==}	& is equal to \\ \rowcolor{gray!10}
		\texttt{!=} & is \textsl{not} equal to\\
		\texttt{>} & is greater than \\ \rowcolor{gray!10}
		\texttt{>=} & is greater than, or equal to \\
		\texttt{<} & is less than \\ \rowcolor{gray!10}
		\texttt{<=} & is less than, or equal to \\		
	\end{tabular}
\end{center}
\pythonut{bool1.py}{
	False\\
	True\\
	True\\
	False\\
}\vsk

In addition to the comparative operators, we can use the \outl{logical operators} \pytype{and}, \pytype{or}, and \pytype{not}.
\pythonut{bool2.py}{
	False\\
	True\\
	True
}
\spr{
	Checks that use both comparative and logical operators will henceforth be called \outl{conditions}.
}
\subsection{Expressions \pymet{if, else}, and \pymet{elif}}
When we want to perform actions only \textsl{if} a condition is true (\pytype{True}), we use the expression \pymet{if} in front of the condition. The code we write indented under the \pymet{if} line will only be executed if the condition evaluates to \pytype{True}.
\pythonut{if.py}{
	Yep, c is greater than b
} 
If you first want to check if a condition is true, and then perform actions if it's \textsl{not}, you can use the expression \pymet{else}:
\pythonut{else.py}{
	But this comes because the condition in the if-line above was False
}
The expression \pymet{else} only considers (and doesn't make sense without) the \pymet{if} expression right above it. If we want actions to be performed \textsl{only} if no previous \pymet{if} expressions produced any result, we must use\footnote{\pymet{elif} is a shortcut for \pymet{else if}, which can also be used.} the expression \pymet{elif}. This is an \pymet{if} expression that takes effect if the \pymet{if} expression above did \textsl{not} take effect.
\pythonut{elif.py}{
	Now we are sure that 1 < b < 3
}
\info{Note}{
	When working with numbers, some conditions you expect to be \pytype{True} might turn out to be \pytype{False}. This often deals with rounding errors, as mentioned on page \pageref{round()}.
}
\newpage

\subsection{Lister}
Lister kan vi bruke for å samle objekter. Objektene som er i listen kalles \outl{elementene} til listen.
\python{list1.py}
Elementene i lister er \outl{indekserte}. Første objekt har indeks 0, andre objekt har indeks 1 og så videre:
\pythonut{list1.py}{
96 \\
99 \\
98
}
Med den innebygde funksjonen \texttt{append()} kan vi legge til et objekt i enden av listen. Dette er en \outl{innebygd funksjon}\footnote{Kort fortalt betyr det at det bare er noen typer objekter som kan bruke denne funksjonen.}, som vi skriver i enden av navnet på listen, med et punktum foran.
\pythonut{list3.py}{
[] \newline
[3] \newline
[3, 7]
}
\newpage
Med funksjonen \texttt{pop()} kan vi hente ut et objekt fra listen
\pythonut{list4.py}{
a = 19 \\
min\_liste = [6, 10, 15] \\
a = 10 \\
min\_liste = [6, 15]
}
\info{Forklar for deg selv}{
Hva er forskjellen på å skive \texttt{a = min\_liste[1]} og å skrive \texttt{a = min\_liste.pop(1)}?
} \vsk

Med funksjonen \texttt{sort()} kan vi sortere elementene i listen.
\pythonut{list5.py}{
[0, 1, 3, 4, 7, 8, 9]
\\

['a', 'b', 'c', 'd', 'e']
}\vsk
\newpage
Med funksjonene \texttt{count()} kan vi telle gjentatte elementer i listen.
\pythonut{list6.py}{
4\\
1\\
2\\
0
} \vsk

Med funksjonen \texttt{len()} kan vi finne antall elementer i en liste, og med funksjonen \texttt{sum()} kan vi finne summen av lister med tall som elementer.
\pythonut{list7.py}{
5\\
3\\
15
}
\newpage
Med uttrykket \texttt{in} kan vi sjekke om et element er i en liste.
\pythonut{list8.py}{
True\\
False
}


\newpage
\subsection{Looper; \pymet{for} og \pymet{while}}
\subsubsection{\pymet{for} loop}
For objekter som inneholder flere elementer, kan vi bruke \pymet{for}\,-\,looper til å utføre handlinger for hvert element. Handlingene må vi skrive med et innrykk etter \pymet{for}\,-\,uttrykket:
\pythonut{for.py}{
5 \\
50 \\[12pt]

10\\
100\\[12pt]

15\\
150
}
\spr{
Å gå gjennom hvert element i (for eksempel) en liste kalles ''å iterere over listen''.
}\vsk

Ofte er det ønskelig å iterere over heltallene $ 0, 1, 2 $ og så videre. Til dette kan vi bruke \pymet{range()}:
\pythonut{forrang.py}{
0 \\
1 \\
2
}
\subsubsection{\pymet{while} loop}
Hvis vi ønsker at handlinger skal utføres fram til et vilkår er sant, kan vi bruke en \pymet{while}\,-\,loop: \regv
\pythonut{while1.py}{
1\\
2\\
3\\
4\\
}

\subsection{\pymet{input()}}
Vi kan bruke funksjonen \pymet{input()} til å skrive inn tekst mens skriptet kjører:
\python{input0.py}
Teksten vi har skrevet inni \pymet{input()} i skriptet over er teksten vi ønsker vist foran teksten som ønskes innskrevet. Linje 2 i denne koden vil ikke kjøres før en tekst er innskrevet.
\pythonut{input0.py}{
Skriv inn tekst her: OK\\
OK
}
\newpage
Objektet gitt av en \pymet{input()}-funksjon vil alltid være av typen \pytype{str}. Man må alltid passe på å gjøre om objekter til rett type:
\pythonut{input1.py}{
La oss regne ut a*b\\
a = 3.7\\
b = 4\\
a*b = 14.8
}
\subsection{Feilmeldinger}
Påstand: Alle programmerere vil erfare at skriptet ikke kjører fordi vi ikke har skrevet koden på rett måte. Dette kalles en \outl{syntax error}. Ved syntax error vil man få beskjed om på hvilken linje feilen befinner seg, og hva som er feil. De vanligste feilene er
\begin{itemize}
\item \textbf{Å glemme innrykk når man bruker metoder som \pymet{def}, \pymet{for}, \pymet{while}, og \pymet{if} }
\pythonut{erindent.py}{
line 5, in <module>\\
print("a*b er større enn 48000")\\
 \^{} \\
IndentationError: expected an indented block after 'if' statement on line 4
}
\item \textbf{Å utføre operasjoner på typer det ikke gir mening for}
\pythonut{ertype.py}{
line 2, in <module> \\
b\_opphøyd\_i\_andre = b**2\\
TypeError: unsupported operand type(s) for ** or\\ pow(): 'str' and 'int'\\
}
\end{itemize}

\newpage



\end{document}