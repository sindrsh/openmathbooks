\documentclass[english,hidelinks,pdftex, 11 pt, class=article,crop=false]{standalone}
\usepackage[T1]{fontenc}
\usepackage[utf8]{luainputenc}
\usepackage{lmodern} % load a font with all the characters
\usepackage{geometry}
\geometry{verbose,a4paper, inner=0cm, outer=0 cm, bmargin=2cm, tmargin=1cm}
%\textwidth=12cm
\setlength{\parindent}{0bp}
\usepackage{import}
\usepackage[subpreambles=false]{standalone}
\usepackage{amsmath}
\usepackage{amssymb}
\usepackage{esint}
\usepackage{babel}
\usepackage{tabu}
\usepackage[dvipsnames, table]{xcolor}
\usepackage{cancel}
\makeatother
\makeatletter
\usepackage{datetime2}
\usepackage{titlesec}
\usepackage[many]{tcolorbox}

% Eheter
\newcommand{\enh}[1]{\,\textrm{#1}}
%referances
\newcommand{\net}[2]{{\color{blue}\href{#1}{#2}}}

%Spaces
\newcommand{\vsk}{\\[12pt]}
\newcommand{\vs}{\vspace{-12pt}}

% Tabell for opplegg

\newcommand{\ovlist}[1]{
\vspace{-16pt}
\begin{itemize}
	#1
\end{itemize}
}

% Chapters and sections
\titleformat{\section}[block]{\bfseries}{\hspace{3cm}\thesection}{5pt}{}
\titleformat{\subsection}[block]{\bfseries}{\hspace{3cm}\thesection}{5pt}{}
\newcommand{\sectionbreak}{\clearpage} % New page on each section
 

\newlength{\mywidth}
\setlength{\mywidth}{14cm}

\newcommand{\cont}[1]{
\begin{tcolorbox}[center, boxrule=0.0 mm, width=\mywidth,arc=0mm,enhanced jigsaw,,colback=white,breakable]
#1	
\end{tcolorbox}
}

\newcommand{\info}[5]{
\begin{tcolorbox}[center, boxrule=0.1 mm, width=\mywidth,arc=0mm,enhanced jigsaw,breakable,colback=yellow!5]	
	
	\footnotesize
	\textbf{Øvingsområde}\\[5pt] #1 
	
	\textbf{Utstyr}\\ #2  \\
	
	\begin{tabular}{@{} p{4cm} p{4cm} l} 
		\textbf{Tid} & \textbf{Elevinndeling} & \textbf{Læringsarena} \\
		#3  & #4 & #5
	\end{tabular} 
\end{tcolorbox}	
}

\newcommand{\gjen}[1]{\begin{tcolorbox}[center,boxrule=0.1 mm, width=\mywidth,arc=0mm,colback=blue!3] {\large \textbf{Gjennomføring} \vspace{5 pt}}\newline #1  \end{tcolorbox}\vspace{-5pt}}
\newcommand{\eks}[1]{\begin{tcolorbox}[center,boxrule=0.1 mm, width=\mywidth,arc=0mm,colback=green!3] {\large \textbf{Eksempel} \vspace{5 pt}}\newline #1  \end{tcolorbox}\vspace{-5pt}}

\newcounter{opl}
%\numberwithin{opl}{article}


\newcommand{\opl}[1]{
\newpage
{\refstepcounter{opl} %\phantomsection 
\large \textbf{\theopl \;#1} \vsk}
}

% Headlines
\newcommand{\fork}{\textbf{Forkunnskapar}\\}
\newcommand{\forb}{\textbf{Forberedelsar}\\}
\newcommand{\opgvr}{\textbf{Oppgaver}}



%colors
\newcommand{\colr}[1]{{\color{red} #1}}
\newcommand{\colb}[1]{{\color{blue} #1}}
\newcommand{\colo}[1]{{\color{orange} #1}}
\newcommand{\colc}[1]{{\color{cyan} #1}}
\definecolor{projectgreen}{cmyk}{100,0,100,0}
\newcommand{\colg}[1]{{\color{projectgreen} #1}}

% Lister med bokstavar
\usepackage[inline]{enumitem}
% Opg
\newcommand{\abc}[1]{
	\begin{enumerate}[label=\alph*),leftmargin=18pt]
		#1
	\end{enumerate}
}

\usepackage[]{hyperref}


\begin{document}
\section{Omkrets}
\subsection*{Generelt}
Når ein måler kor langt det er rundt ein figur, finn ein \textit{omkretsen} til figuren. Lat oss starte med å finne omkretsen til dette rektangelet:
\fig{geo1}
Me ser at rektangelet har to sider som har lengde 4, og to sider som har lengde 5:
\fig{geo1a}
Dette betyr at:
\alg{
\text{Omkrets til rektangel} &= 4+4+5+5 \\
&= 18
}

\subsection*{Omkrets når sidene er kjente}
Nokre gonger er det ikkje så lett å sjå kor lange sidene er, men vi har fått vite det i staden. Som for eksempel i denne figuren, som har sider med lengdene 2, 1, 3 og 2:
\fig{geo2}
\alg{
\text{Omkrets} &= 2+1+3+2 \\
&= 8
}
\newpage
\subsection*{Omkrets med eining}
Når me måler lengder med linjal eller liknande må me passe på å ta med eininga i svaret vårt:
\begin{figure}
	\centering
	\includegraphics[scale=0.06]{\asym{2t5}}
\end{figure}
\alg{
\text{Omkrets til rektangel} &= 5\enh{cm}+2\enh{cm}+5\enh{cm}+2\enh{cm} \\
&= 14\enh{cm}
}

\newpage
\section{Areal}
\begin{center}
	\includegraphics[scale=0.2]{\asym{golv}}
	\includegraphics[scale=0.117]{\asym{papir}}
\end{center}
\subsection*{Alle overflater har eit areal}
Overalt rundt oss kan me sjå \textit{overflater}, som for eksempel eit golv eller arket me skriv på. Når me ønsker å seie noko om kor store overflater er, må me finne \textit{arealet} deira. Idéen bak omgrepet areal er denne:\vsk

Me tenker oss eit kvadrat med bredde 1 og høgde 1, som me kan kalle for ''einarkvadradet'':
\fig{tri_10}
Så ser me på overflata me ønsker å finne arealet av og spør:\os
\begin{center}
	''Kor mange einarkvadrat er det plass til på denne overflata?''
\end{center}
Lat oss prøve dette med eit rektangel som er 5 breidt og 4 høgt:
\fig{tri_11a}
Me kan da telle oss fram til at rektangelet har plass til 20 einarkvadrat:
\[ \text{Arealet til rektangelet}=20 \]
\fig{tri_11}
Me kan også legge merke til at figuren vår er den samme me brukte da me teikna gangestykket $ 4\cdot5 $:
\alg{
	\text{Arealet til rektangelet} &= 4\cdot 5 \\
	&= 20 
}
\fig{tri_11}
\reg[Arealet av rektangel \label{arfir}]{
	\[ \text{Areal av rektangel}=\text{høgde}\cdot\text{bredde} \]
	\fig{tri_12}
}\vsk

\subsection*{Areal med eining}
Når me måler lengder med linjal eller liknande, må me også for areal passe på å ta med einingane:
\begin{figure}
	\centering
	\includegraphics[scale=0.06]{\asym{2t5}}
\end{figure}
\[ \text{Areal av rektangel}=2\enh{cm}\cdot5\enh{cm} \]
$ 2\enh{cm}\cdot5\enh{cm} $ reknar ein ut med å gange 2 med 5, og så skrive \,cm$ ^2 $ bakom:
\prbxl{0.69}{\alg{
		\text{Areal av rektangel}&=2\enh{cm}\cdot5\enh{cm} \\
		&= 2\cdot 5\enh{cm}^2\\
		&= 10\enh{cm}^2
}}
\prbxr{0.3}{Vi skriv cm$ ^2 $ fordi vi har ganga saman 2 lengder som vi har målt i cm.}

\end{document}

