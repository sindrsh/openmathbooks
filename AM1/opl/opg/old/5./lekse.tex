\documentclass[english,hidelinks,pdftex, 11 pt, class=report,crop=false]{standalone}
\usepackage[T1]{fontenc}
\usepackage[utf8]{luainputenc}
\usepackage{lmodern} % load a font with all the characters
\usepackage{geometry}
\geometry{verbose,a4paper, inner=0cm, outer=0 cm, bmargin=2cm, tmargin=1cm}
%\textwidth=12cm
\setlength{\parindent}{0bp}
\usepackage{import}
\usepackage[subpreambles=false]{standalone}
\usepackage{amsmath}
\usepackage{amssymb}
\usepackage{esint}
\usepackage{babel}
\usepackage{tabu}
\usepackage[dvipsnames, table]{xcolor}
\usepackage{cancel}
\makeatother
\makeatletter
\usepackage{datetime2}
\usepackage{titlesec}
\usepackage[many]{tcolorbox}

% Eheter
\newcommand{\enh}[1]{\,\textrm{#1}}
%referances
\newcommand{\net}[2]{{\color{blue}\href{#1}{#2}}}

%Spaces
\newcommand{\vsk}{\\[12pt]}
\newcommand{\vs}{\vspace{-12pt}}

% Tabell for opplegg

\newcommand{\ovlist}[1]{
\vspace{-16pt}
\begin{itemize}
	#1
\end{itemize}
}

% Chapters and sections
\titleformat{\section}[block]{\bfseries}{\hspace{3cm}\thesection}{5pt}{}
\titleformat{\subsection}[block]{\bfseries}{\hspace{3cm}\thesection}{5pt}{}
\newcommand{\sectionbreak}{\clearpage} % New page on each section
 

\newlength{\mywidth}
\setlength{\mywidth}{14cm}

\newcommand{\cont}[1]{
\begin{tcolorbox}[center, boxrule=0.0 mm, width=\mywidth,arc=0mm,enhanced jigsaw,,colback=white,breakable]
#1	
\end{tcolorbox}
}

\newcommand{\info}[5]{
\begin{tcolorbox}[center, boxrule=0.1 mm, width=\mywidth,arc=0mm,enhanced jigsaw,breakable,colback=yellow!5]	
	
	\footnotesize
	\textbf{Øvingsområde}\\[5pt] #1 
	
	\textbf{Utstyr}\\ #2  \\
	
	\begin{tabular}{@{} p{4cm} p{4cm} l} 
		\textbf{Tid} & \textbf{Elevinndeling} & \textbf{Læringsarena} \\
		#3  & #4 & #5
	\end{tabular} 
\end{tcolorbox}	
}

\newcommand{\gjen}[1]{\begin{tcolorbox}[center,boxrule=0.1 mm, width=\mywidth,arc=0mm,colback=blue!3] {\large \textbf{Gjennomføring} \vspace{5 pt}}\newline #1  \end{tcolorbox}\vspace{-5pt}}
\newcommand{\eks}[1]{\begin{tcolorbox}[center,boxrule=0.1 mm, width=\mywidth,arc=0mm,colback=green!3] {\large \textbf{Eksempel} \vspace{5 pt}}\newline #1  \end{tcolorbox}\vspace{-5pt}}

\newcounter{opl}
%\numberwithin{opl}{article}


\newcommand{\opl}[1]{
\newpage
{\refstepcounter{opl} %\phantomsection 
\large \textbf{\theopl \;#1} \vsk}
}

% Headlines
\newcommand{\fork}{\textbf{Forkunnskapar}\\}
\newcommand{\forb}{\textbf{Forberedelsar}\\}
\newcommand{\opgvr}{\textbf{Oppgaver}}



%colors
\newcommand{\colr}[1]{{\color{red} #1}}
\newcommand{\colb}[1]{{\color{blue} #1}}
\newcommand{\colo}[1]{{\color{orange} #1}}
\newcommand{\colc}[1]{{\color{cyan} #1}}
\definecolor{projectgreen}{cmyk}{100,0,100,0}
\newcommand{\colg}[1]{{\color{projectgreen} #1}}

% Lister med bokstavar
\usepackage[inline]{enumitem}
% Opg
\newcommand{\abc}[1]{
	\begin{enumerate}[label=\alph*),leftmargin=18pt]
		#1
	\end{enumerate}
}

\usepackage[]{hyperref}
\geometry{verbose,a4paper, inner=2.05cm, outer=2.05 cm, bmargin=2cm, tmargin=1.8cm}

\begin{document}
\begin{huge}
	\textbf{Oppgåveark i matematatikk 22.-25. oktober}\vsk
\end{huge}

Skriv alle svar i kladdeboka di og vis utrekning.\vsk

\textbf{Oppgave 1}\bs
Under ser du to metodar å rekne ut $ 59+87 $ på:
\begin{figure}
	\centering
	\includegraphics[scale=0.5]{plus}
\end{figure}
Bruk éin av desse metodane til å rekne ut:
\begin{enumerate}
	\item $ 137+49 $ \\
	\item $ 53+78 $ \\
	\item $ 64,6+103,9 $
\end{enumerate}
\vsk
\textbf{Oppgave 2}\\
Nokon påstår dette:\vsk
\begin{center}
	\begin{minipage}[c]{0.8\linewidth}
		\itshape
		\textit{For å finne ut kva $8\cdot9  $ er, kan du tenke slik:}
		\begin{itemize}
			\item Éin mindre enn $ 8 $ er $\color{blue} 7 $
			\item $ 9-\color{blue}7\color{black} =\color{cyan}2$
		\end{itemize}
		Altså er $ 8\cdot9=\color{blue}7\color{cyan}2 $
	\end{minipage}
\end{center}
\vsk
\begin{center}
	\begin{minipage}[c]{0.8\linewidth}
		\itshape
		\textit{For å finne ut kva $7\cdot9  $ er, kan du tenke slik:}
		\begin{itemize}
			\item Éin mindre enn $ 7 $ er $\color{blue} 6 $
			\item $ 9-\color{blue}6\color{black} =\color{cyan}3$
		\end{itemize}
		Altså er $ 7\cdot9=\color{blue}6\color{cyan}3 $
	\end{minipage}
\end{center}\vsk

Sjekk om dette trikset verkar for heile 9-gangen!
\newpage\textbf{Oppgave 3}\bs
\textit{Påminning: \textsl{Arealat} av ein firkant er antal ruter som er inni firkanten}\vsk

\textbf{a)} Finn arealet til firkanten under ved å bruke eit reknestykke med ganging. (Du kan telle for å sjekke svaret ditt).
\fig{gang}
\textbf{b)} Finn arealet til firkanten under ved å bruke eit reknestykke med ganging. (Du kan telle for å sjekke svaret ditt).
\fig{gang2}
\textbf{c)} Gjer oppgave a) og b) opp att, men bruk eit anna gangestykke enn det du har skrive. \vsk

\textbf{d)} Lag to forskjellige figurar med areal lik 11. (Du bestemmer sjølv kor stor éi rute skal vere).
\end{document}

