\documentclass[english,openright ,hidelinks,pdftex, 12 pt, class=report,crop=false]{standalone}
\usepackage[T1]{fontenc}
\usepackage[utf8]{luainputenc}
\usepackage{lmodern} % load a font with all the characters
\usepackage{geometry}
\geometry{verbose,a4paper, inner=0cm, outer=0 cm, bmargin=2cm, tmargin=1cm}
%\textwidth=12cm
\setlength{\parindent}{0bp}
\usepackage{import}
\usepackage[subpreambles=false]{standalone}
\usepackage{amsmath}
\usepackage{amssymb}
\usepackage{esint}
\usepackage{babel}
\usepackage{tabu}
\usepackage[dvipsnames, table]{xcolor}
\usepackage{cancel}
\makeatother
\makeatletter
\usepackage{datetime2}
\usepackage{titlesec}
\usepackage[many]{tcolorbox}

% Eheter
\newcommand{\enh}[1]{\,\textrm{#1}}
%referances
\newcommand{\net}[2]{{\color{blue}\href{#1}{#2}}}

%Spaces
\newcommand{\vsk}{\\[12pt]}
\newcommand{\vs}{\vspace{-12pt}}

% Tabell for opplegg

\newcommand{\ovlist}[1]{
\vspace{-16pt}
\begin{itemize}
	#1
\end{itemize}
}

% Chapters and sections
\titleformat{\section}[block]{\bfseries}{\hspace{3cm}\thesection}{5pt}{}
\titleformat{\subsection}[block]{\bfseries}{\hspace{3cm}\thesection}{5pt}{}
\newcommand{\sectionbreak}{\clearpage} % New page on each section
 

\newlength{\mywidth}
\setlength{\mywidth}{14cm}

\newcommand{\cont}[1]{
\begin{tcolorbox}[center, boxrule=0.0 mm, width=\mywidth,arc=0mm,enhanced jigsaw,,colback=white,breakable]
#1	
\end{tcolorbox}
}

\newcommand{\info}[5]{
\begin{tcolorbox}[center, boxrule=0.1 mm, width=\mywidth,arc=0mm,enhanced jigsaw,breakable,colback=yellow!5]	
	
	\footnotesize
	\textbf{Øvingsområde}\\[5pt] #1 
	
	\textbf{Utstyr}\\ #2  \\
	
	\begin{tabular}{@{} p{4cm} p{4cm} l} 
		\textbf{Tid} & \textbf{Elevinndeling} & \textbf{Læringsarena} \\
		#3  & #4 & #5
	\end{tabular} 
\end{tcolorbox}	
}

\newcommand{\gjen}[1]{\begin{tcolorbox}[center,boxrule=0.1 mm, width=\mywidth,arc=0mm,colback=blue!3] {\large \textbf{Gjennomføring} \vspace{5 pt}}\newline #1  \end{tcolorbox}\vspace{-5pt}}
\newcommand{\eks}[1]{\begin{tcolorbox}[center,boxrule=0.1 mm, width=\mywidth,arc=0mm,colback=green!3] {\large \textbf{Eksempel} \vspace{5 pt}}\newline #1  \end{tcolorbox}\vspace{-5pt}}

\newcounter{opl}
%\numberwithin{opl}{article}


\newcommand{\opl}[1]{
\newpage
{\refstepcounter{opl} %\phantomsection 
\large \textbf{\theopl \;#1} \vsk}
}

% Headlines
\newcommand{\fork}{\textbf{Forkunnskapar}\\}
\newcommand{\forb}{\textbf{Forberedelsar}\\}
\newcommand{\opgvr}{\textbf{Oppgaver}}



%colors
\newcommand{\colr}[1]{{\color{red} #1}}
\newcommand{\colb}[1]{{\color{blue} #1}}
\newcommand{\colo}[1]{{\color{orange} #1}}
\newcommand{\colc}[1]{{\color{cyan} #1}}
\definecolor{projectgreen}{cmyk}{100,0,100,0}
\newcommand{\colg}[1]{{\color{projectgreen} #1}}

% Lister med bokstavar
\usepackage[inline]{enumitem}
% Opg
\newcommand{\abc}[1]{
	\begin{enumerate}[label=\alph*),leftmargin=18pt]
		#1
	\end{enumerate}
}

\usepackage[]{hyperref}

\geometry{a4paper,verbose,tmargin=1cm,bmargin=1cm,lmargin=4cm,rmargin=4cm,headheight=3cm,headsep=1cm,footskip=1cm}
\newcounter{inl}
\newcounter{abc}
%\numberwithin{inl}{}
\newcommand{\inl}[1]{\vspace{15pt} \setcounter{abc}{0} \refstepcounter{inl} {\large \textbf{Oppgave \theinl \label{#1}}} \vspace{5 pt}\\}
\renewcommand\theabc{\alph{abc}}
\newcommand{\ab}{\refstepcounter{abc} \textbf{\theabc}) }
\usepackage{array}


\begin{document}
	\pagenumbering{gobble}
{\Large Løsningsforslag} \\[15pt]	
{\Large \textbf{Del 1 - Uten hjelpemidler}} 	
	
\inl{p1}

\textbf{a)} Av figuren ser vi at når vi går 1 bort, må vi gå 3 opp for å komme tilbake til grafen til $ f(x) $. Dette betyr at stigningstallet til $ x $ er $ {\frac{3}{1}=3 }$. Videre ser vi at grafen til $ f(x) $ skjærer verdien $ -2 $ på vertikalaksen, derfor er konstantleddet også $ -2 $. Derfor har vi at:
\[ f(x)=3x-2 \]
Når vi går 1 bort, må vi gå 1 ned for å komme tilbake til grafen til $ g(x) $, som derfor har stigningstall $ \frac{-1}{1}=-21$. Videre ser vi at konstantleddet må være 6, derfor har vi at:
\[ g(x)=-x+6 \]
\begin{figure}
	\centering
	\includegraphics[]{Fig1b}
\end{figure}


\textbf{b)} Av grafen ser vi at skjæringspunktet er (2, 4) (markert med rød prikk på bildet over).\os

\inl{p7}
For å leie buss med Matteergøy Busselskap må man betale 4000 kr for buss og sjåfør, i tillegg til 20 kr for hver mil bussen skal kjøre.\os
\textbf{a)} Siden vi må betale 20 kr for hver mil som kjæres, må vi gange 10 med antall mil. Siden $ x $ betyr mil må vi gange 20 med $ x $. I tillegg må vi legge på 4000 kr etterpå for buss og sjåfør:
\[ S(x) = 20x+4000 \]

\textbf{b)} \textit{Løsningsmetode 1}: Siden $ S(x) $ er hvor mye vi må betale for en tur, og vi kan betale 6000 kr kan vi skrive $ {S(x)=6000} $. Da får vi en ligning vi kan løse:
\alg{
6000 &= 20x + 4000 \\
6000-4000&= 20 x \\
\frac{200\cancel{0}}{\cancel{0}} &= \frac{\cancel{20}x}{\cancel{10}}\\
100&= x
}
Vi får altså kjørt 100 mil for 6000 kr.\os

\textit{Løsningsmetode 2}: For å få 6000, må vi legge 2000 til 4000. For at $ 20x $ skal bli 2000, ser vi at $ x $ må være 100:
\alg{
20\cdot100+4000 &= 2000+4000\\
&= 6000
}
Altså kan vi kjøre 100 mil for 6000 kr.


\inl{p3}
I skjæringspunktet må $ f(x) $ og $ g(x) $ være like:
\alg{
	f(x)&=g(x) \\
	2x+1 &= -x+10 \\
	2x+x &= 10-1 \\
	\frac{\cancel{3}x}{\cancel{3}} &= \frac{9}{3} \\
	 x &= 3
}
For å finne funksjonsverdien når $ {x=3} $ kan vi selv velge om vi vil finne $ f(3) $ eller $ g(3) $, vi bruker her $ f(3) $:
\alg{
f(3) &= 2\cdot3+1 \\ 
&=7
}
Skjæringspunktet er altså $ (3,7) $

\inl{p2}
\textbf{a)} Vi finner to punkt på grafen til $ f(x) $ ved selv å velge ut to $ x $-verdier, i vårt tilfelle bruker vi $ x=0 $ og $ x=3 $ (det er lurt å ha litt avstand mellom $ x $-verdiene).

\parbox{0.65\linewidth}{\alg{
		f(0) &= 2\cdot0-1\\ &=-1\\
		f(3)&= 2\cdot3-1 \\ &= 6-1 \\&=5
}}
\parbox{0.25\linewidth}{
\begin{tabular}{|c|c|}
	\boldmath $ x $ &  \boldmath $ f(x) $ \\ \hline
	0 & -1 \\ 
	3 & 5
\end{tabular}
}
\begin{figure}
	\centering
	\includegraphics[scale=0.94]{fig2b}
\end{figure}


\textbf{b)} \vs \vs
\alg{
f(-2) &= (-2)^2 - 2(-2)+1 = 4+4+1 =9 \\
f(-1) &= (-1)^2 - 2(-1)+1 = 1+2+1=7 \\
f(0) &= 0^2-2(0)+1 =2 \\
f(1) &= 1^2-2(1)+1 =1-2+1=0
}
\begin{center}
	\begin{tabular}{*{5}{|p{0.7cm}}|}
		$ x $ & $ -2 $ & $ -1 $ & $ 0 $ & $ 1 $\\ \hline
		$ f(x) $& 9 & 7 &2 &0 
	\end{tabular}
\end{center}

\textbf{c)} \vs \vs 
\alg{
	f(-2) &= -(-2)^2 -(-2)+2 = -4+2+2 =0 \\
	f(2) &= -2^2-(2)+2 =-4-2+2=-4
}

\begin{center}
	\begin{tabular}{*{3}{|p{0.7cm}}|}
		$ x $ & $ -2 $ & $ 2 $ \\ \hline
		$ f(x) $& $ -7 $ & $ -3 $ 
	\end{tabular}
\end{center}

\inl{p6}
Av figuren ser vi at når vi går 4 bort, må vi gå 1 opp for å komme tilbake til grafen, derfor er stigningstallet $ \frac{1}{4} $. Grafen til $ f(x) $ skjærer vertikalaksen i verdien 1, som derfor er konstantleddet. Da har vi at:
\[ f(x)= \frac{1}{4}x+1 \]
\begin{figure}
	\centering
	\includegraphics[]{Fig3b}
\end{figure}

\newpage
{\Large \textbf{Del 2 - Med hjelpemidler}} 
	
\inl{p4}
Gitt funksjonen
\[ f(x)=x^2-2x-3 \]
\textbf{a)} Finn verdien til $ f(x) $ når $ {x=10} $. \os
\textbf{b)} Finn toppunktet/bunnpunktet til $ f $.\os
\textbf{c)} Finn nullpunktene til $ f $. \os
\textbf{d)} Hva er $ x $ når $ {f(x)=12} $?

\inl{p5}
Funksjonen $ D(x) $ er en tilnærming for hvor mange timer dagslys Ålesund har $ x $ måneder etter 1. januar.
\[ D(x)= 0.0129x^4-0.2912x^3+1.6250x^2+0.2189x+5.414\quad, \quad 0\leq x\leq12 \]
\textbf{a)} Tegn grafen til $ D $.\os
\textbf{b)} I hvilken måned er dagen lengst, ifølge funksjonen?\os
\textbf{c)} I hvilken måned er dagen kortest, ifølge funksjonen?\os
\textbf{d)} \textit{Vårjevndøgn} kalles dagene i året hvor det er mørkt og lyst like lenge. Hvilke måneder er det vårjevndøgn, ifølge funksjonen?


\end{document}


