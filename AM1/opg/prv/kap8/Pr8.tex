\documentclass[english,openright ,hidelinks,pdftex, 12 pt, class=report,crop=false]{standalone}
\usepackage[T1]{fontenc}
\usepackage[utf8]{luainputenc}
\usepackage{lmodern} % load a font with all the characters
\usepackage{geometry}
\geometry{verbose,a4paper, inner=0cm, outer=0 cm, bmargin=2cm, tmargin=1cm}
%\textwidth=12cm
\setlength{\parindent}{0bp}
\usepackage{import}
\usepackage[subpreambles=false]{standalone}
\usepackage{amsmath}
\usepackage{amssymb}
\usepackage{esint}
\usepackage{babel}
\usepackage{tabu}
\usepackage[dvipsnames, table]{xcolor}
\usepackage{cancel}
\makeatother
\makeatletter
\usepackage{datetime2}
\usepackage{titlesec}
\usepackage[many]{tcolorbox}

% Eheter
\newcommand{\enh}[1]{\,\textrm{#1}}
%referances
\newcommand{\net}[2]{{\color{blue}\href{#1}{#2}}}

%Spaces
\newcommand{\vsk}{\\[12pt]}
\newcommand{\vs}{\vspace{-12pt}}

% Tabell for opplegg

\newcommand{\ovlist}[1]{
\vspace{-16pt}
\begin{itemize}
	#1
\end{itemize}
}

% Chapters and sections
\titleformat{\section}[block]{\bfseries}{\hspace{3cm}\thesection}{5pt}{}
\titleformat{\subsection}[block]{\bfseries}{\hspace{3cm}\thesection}{5pt}{}
\newcommand{\sectionbreak}{\clearpage} % New page on each section
 

\newlength{\mywidth}
\setlength{\mywidth}{14cm}

\newcommand{\cont}[1]{
\begin{tcolorbox}[center, boxrule=0.0 mm, width=\mywidth,arc=0mm,enhanced jigsaw,,colback=white,breakable]
#1	
\end{tcolorbox}
}

\newcommand{\info}[5]{
\begin{tcolorbox}[center, boxrule=0.1 mm, width=\mywidth,arc=0mm,enhanced jigsaw,breakable,colback=yellow!5]	
	
	\footnotesize
	\textbf{Øvingsområde}\\[5pt] #1 
	
	\textbf{Utstyr}\\ #2  \\
	
	\begin{tabular}{@{} p{4cm} p{4cm} l} 
		\textbf{Tid} & \textbf{Elevinndeling} & \textbf{Læringsarena} \\
		#3  & #4 & #5
	\end{tabular} 
\end{tcolorbox}	
}

\newcommand{\gjen}[1]{\begin{tcolorbox}[center,boxrule=0.1 mm, width=\mywidth,arc=0mm,colback=blue!3] {\large \textbf{Gjennomføring} \vspace{5 pt}}\newline #1  \end{tcolorbox}\vspace{-5pt}}
\newcommand{\eks}[1]{\begin{tcolorbox}[center,boxrule=0.1 mm, width=\mywidth,arc=0mm,colback=green!3] {\large \textbf{Eksempel} \vspace{5 pt}}\newline #1  \end{tcolorbox}\vspace{-5pt}}

\newcounter{opl}
%\numberwithin{opl}{article}


\newcommand{\opl}[1]{
\newpage
{\refstepcounter{opl} %\phantomsection 
\large \textbf{\theopl \;#1} \vsk}
}

% Headlines
\newcommand{\fork}{\textbf{Forkunnskapar}\\}
\newcommand{\forb}{\textbf{Forberedelsar}\\}
\newcommand{\opgvr}{\textbf{Oppgaver}}



%colors
\newcommand{\colr}[1]{{\color{red} #1}}
\newcommand{\colb}[1]{{\color{blue} #1}}
\newcommand{\colo}[1]{{\color{orange} #1}}
\newcommand{\colc}[1]{{\color{cyan} #1}}
\definecolor{projectgreen}{cmyk}{100,0,100,0}
\newcommand{\colg}[1]{{\color{projectgreen} #1}}

% Lister med bokstavar
\usepackage[inline]{enumitem}
% Opg
\newcommand{\abc}[1]{
	\begin{enumerate}[label=\alph*),leftmargin=18pt]
		#1
	\end{enumerate}
}

\usepackage[]{hyperref}

\geometry{a4paper,verbose,tmargin=1cm,bmargin=2cm,lmargin=4cm,rmargin=4cm,headheight=3cm,headsep=1cm,footskip=1cm}
\newcounter{inl}
\newcounter{abc}
%\numberwithin{inl}{}
\newcommand{\inl}[1]{\vspace{15pt} \setcounter{abc}{0} \refstepcounter{inl} {\large \textbf{Oppgave \theinl \label{#1}}} \vspace{5 pt}\\}
\renewcommand\theabc{\alph{abc}}
\newcommand{\ab}{\refstepcounter{abc} \textbf{\theabc}) }
\usepackage{array}


\begin{document}

\inl{p1}

\textbf{a)} Finn uttrykkene til $ f(x) $ og $ g(x) $ på bildet under.
\begin{figure}
	\centering
	\includegraphics[]{Fig1}
\end{figure}

\textbf{b)} Finn skjæringspunktet mellom $ f(x) $ og $ g(x) $ grafisk.\os

\inl{p7}
For å leie buss med Mattergøy Busselskap må man betale 4000 kr for buss og sjåfør, i tillegg til 20 kr for hver mil bussen skal kjøre.\os
\textbf{a)} Sett opp et uttrykk $ S(x) $ som viser hvor mye man må betale hvis bussen skal kjøre $ x $ mil.\os
\textbf{b)} Hvor langt får man kjørt for 6000 kr?




\inl{p3}
Gitt funksjonene:\vs
\alg{
f(x) &= 2x+1 \\[5pt]
g(x) &= -x+10
}
Finn skjæringspunktet mellom $ f(x) $ og $ g(x) $.
\newpage
\inl{p2}
\textbf{a)} Tegn grafen til $ {f(x)=2x-1} $ i koordinatsystemet under:
\begin{figure}
	\centering
	\includegraphics[]{fig2}
\end{figure}
\textbf{b)} Gitt funksjonen $ {f(x)=x^2-2x+1} $. Fyll ut tabellen:\os
\begin{center}
	\begin{tabular}{*{5}{|p{0.7cm}}|}
		$ x $ & $ -2 $ & $ -1 $ & $ 0 $ & $ 1 $\\ \hline
		$ f(x) $& & & & 
	\end{tabular}
\end{center}

\textbf{c)} Gitt funksjonen $ {f(x)=-x^2-x+2} $. Fyll ut tabellen:\os
\begin{center}
	\begin{tabular}{*{3}{|p{0.7cm}}|}
		$ x $ & $ -2 $ & $ 2 $ \\ \hline
		$ f(x) $& & 
	\end{tabular}
\end{center}

\inl{p6}
Finn uttrykket til $ f(x) $.
\begin{figure}
	\centering
	\includegraphics[]{Fig3}
\end{figure}

\newpage
{\Large \textbf{Ekstra: Eksamensoppgaver}} \\[20pt]
\begin{centering}
\includegraphics[scale=0.7]{eks1}\\[20pt]
\includegraphics[scale=0.7]{eks2}
\includegraphics[scale=0.7]{eks3}
\end{centering}
\end{document}


