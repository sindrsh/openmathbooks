\documentclass[english,hidelinks,pdftex, 11 pt, class=report,crop=false]{standalone}
\usepackage[T1]{fontenc}
\usepackage[utf8]{luainputenc}
\usepackage{lmodern} % load a font with all the characters
\usepackage{geometry}
\geometry{verbose,a4paper, inner=0cm, outer=0 cm, bmargin=2cm, tmargin=1cm}
%\textwidth=12cm
\setlength{\parindent}{0bp}
\usepackage{import}
\usepackage[subpreambles=false]{standalone}
\usepackage{amsmath}
\usepackage{amssymb}
\usepackage{esint}
\usepackage{babel}
\usepackage{tabu}
\usepackage[dvipsnames, table]{xcolor}
\usepackage{cancel}
\makeatother
\makeatletter
\usepackage{datetime2}
\usepackage{titlesec}
\usepackage[many]{tcolorbox}

% Eheter
\newcommand{\enh}[1]{\,\textrm{#1}}
%referances
\newcommand{\net}[2]{{\color{blue}\href{#1}{#2}}}

%Spaces
\newcommand{\vsk}{\\[12pt]}
\newcommand{\vs}{\vspace{-12pt}}

% Tabell for opplegg

\newcommand{\ovlist}[1]{
\vspace{-16pt}
\begin{itemize}
	#1
\end{itemize}
}

% Chapters and sections
\titleformat{\section}[block]{\bfseries}{\hspace{3cm}\thesection}{5pt}{}
\titleformat{\subsection}[block]{\bfseries}{\hspace{3cm}\thesection}{5pt}{}
\newcommand{\sectionbreak}{\clearpage} % New page on each section
 

\newlength{\mywidth}
\setlength{\mywidth}{14cm}

\newcommand{\cont}[1]{
\begin{tcolorbox}[center, boxrule=0.0 mm, width=\mywidth,arc=0mm,enhanced jigsaw,,colback=white,breakable]
#1	
\end{tcolorbox}
}

\newcommand{\info}[5]{
\begin{tcolorbox}[center, boxrule=0.1 mm, width=\mywidth,arc=0mm,enhanced jigsaw,breakable,colback=yellow!5]	
	
	\footnotesize
	\textbf{Øvingsområde}\\[5pt] #1 
	
	\textbf{Utstyr}\\ #2  \\
	
	\begin{tabular}{@{} p{4cm} p{4cm} l} 
		\textbf{Tid} & \textbf{Elevinndeling} & \textbf{Læringsarena} \\
		#3  & #4 & #5
	\end{tabular} 
\end{tcolorbox}	
}

\newcommand{\gjen}[1]{\begin{tcolorbox}[center,boxrule=0.1 mm, width=\mywidth,arc=0mm,colback=blue!3] {\large \textbf{Gjennomføring} \vspace{5 pt}}\newline #1  \end{tcolorbox}\vspace{-5pt}}
\newcommand{\eks}[1]{\begin{tcolorbox}[center,boxrule=0.1 mm, width=\mywidth,arc=0mm,colback=green!3] {\large \textbf{Eksempel} \vspace{5 pt}}\newline #1  \end{tcolorbox}\vspace{-5pt}}

\newcounter{opl}
%\numberwithin{opl}{article}


\newcommand{\opl}[1]{
\newpage
{\refstepcounter{opl} %\phantomsection 
\large \textbf{\theopl \;#1} \vsk}
}

% Headlines
\newcommand{\fork}{\textbf{Forkunnskapar}\\}
\newcommand{\forb}{\textbf{Forberedelsar}\\}
\newcommand{\opgvr}{\textbf{Oppgaver}}



%colors
\newcommand{\colr}[1]{{\color{red} #1}}
\newcommand{\colb}[1]{{\color{blue} #1}}
\newcommand{\colo}[1]{{\color{orange} #1}}
\newcommand{\colc}[1]{{\color{cyan} #1}}
\definecolor{projectgreen}{cmyk}{100,0,100,0}
\newcommand{\colg}[1]{{\color{projectgreen} #1}}

% Lister med bokstavar
\usepackage[inline]{enumitem}
% Opg
\newcommand{\abc}[1]{
	\begin{enumerate}[label=\alph*),leftmargin=18pt]
		#1
	\end{enumerate}
}

\usepackage[]{hyperref}

\newcommand{\note}{Merk}
\newcommand{\notesm}[1]{{\footnotesize \textsl{\note:} #1}}
\newcommand{\ekstitle}{Eksempel }
\newcommand{\sprtitle}{Språkboksen}
\newcommand{\expl}{forklaring}
\newcommand{\pyt}{Pytagoras' setning}
\newcommand\sv{\vsk \textbf{Svar} \vspace{4 pt}\\}

%references
\newcommand{\reftab}[1]{\hrs{#1}{tabell}}
\newcommand{\rref}[1]{\hrs{#1}{regel}}
\newcommand{\dref}[1]{\hrs{#1}{definisjon}}
\newcommand{\refkap}[1]{\hrs{#1}{kapittel}}
\newcommand{\refsec}[1]{\hrs{#1}{seksjon}}
\newcommand{\refdsec}[1]{\hrs{#1}{delseksjon}}
\newcommand{\refved}[1]{\hrs{#1}{vedlegg}}
\newcommand{\eksref}[1]{\textsl{#1}}
\newcommand\fref[2][]{\hyperref[#2]{\textsl{figur \ref*{#2}#1}}}
\newcommand{\refop}[1]{{\color{blue}Oppgave \ref{#1}}}
\newcommand{\refops}[1]{{\color{blue}oppgave \ref{#1}}}


%Algebra
\newcommand{\kvadset}{Kvadratsetningene}
\newcommand{\aenato}{Sum-produkt-metoden}

% Geometry
\newcommand{\hlikb}{Midtnormalen i en likebeint trekant}
\newcommand{\arealsetn}{Arealsetningen}
\newcommand{\trkmedian}{Median}
\newcommand{\midtrk}{Midtnormal (i trekant)}
\newcommand{\innskrsirk}{Innskrevet sirkel}
\newcommand{\cossetn}{Cosinussetningen}
\newcommand{\perfvink}{Sentral- og periferivinkel}
\newcommand{\tang}{Tangent}

% Derivative
\newcommand{\derel}{Den deriverte av elementære funksjoner}
\newcommand{\divder}{Divisjonsregelen}
\newcommand{\kjernereg}{Kjerneregelen}
\newcommand{\prodregder}{Produktregelen}
\newcommand{\lhop}{L'Hopitals regel}

% Funksjonsdrofting
\newcommand{\monder}{Monotoniegenskaper og den deriverte}
\newcommand{\fderekstr}{$ \bm{f'=0} $ for lokale ektstremalpunkt}
\newcommand{\andredertest}{Andrederiverttesten}

% Vectors
\newcommand{\detar}{Arealformler med determinanter}
\newcommand{\avstpunktlin}{Avstand mellom punkt og linje}

%Appendix
\newcommand{\rolle}{Rolles teorem}
\newcommand{\meanval}{Middelverdisetningen}

% Solutions manual
\newcommand{\selos}{Se løsningsforslag.}

\begin{document}

\st{Se også oppgaver på \net{https://ektedata.uib.no/}{ekte.data.uib.no}}	\newpage	

\eksop{GV21D1}{GV21D1opg2}
\tagop{
\#regnerekkefølge \#potenser
}
Regn ut $ 3(2+5)-3^2  $

\eksop{GV21D1}{GV21D1opg3}
\tagop{
	\# pi \#rottuttrykk \#desimaltall \#brøk
}
Sorter tallene fra størst til minst
\[ 3,1\qquad\sqrt{9}\qquad2,9\qquad\pi \qquad \frac{32}{10} \]

\eksop{GV21D1}{GV21D1opg6}
\tagop{
\#kobinasjonsregning \#logikk
}
Det er 29 bokstaver i alfabetet vårt, og tallsystemet vi vanligvis bruker, har 10 siffer.
Du skal lage en kode med seks tegn. De to første skal være bokstaver, og de fire neste skal være
siffer. En slik kode kan for eksempel være YA6505. \os

Hvilket av tallene under gir antall ulike koder det er mulig å lage? \os
\abch{
\item 290
\item 2320
\item 4\,092\,480
\item 8\,410\,000
}

\eksop{GV21D1}{GV21D1opg9}
\tagop{
\#ligninger \#formler
}
Sett $ n=120 $, og bestem verdien av $ p $ i formelen nedenfor. 
\[ n = 12p + 48 \]

\eksop{GV21D1}{GV21D1opg1} 
\tagop{
\#enheter \#regning
}
En løpebane på en friidrettsbane er 400 meter.
Cornelia løper 6 runder. Hvor langt løper Cornelia? \os

\mers{I eksamenssettet er dette den første av to deloppgaver. Den andre deloppgaven er altså utelatt her.}

\op{opgtorden}
\tagop{
\#overslag \#proporsjonalitet
}
Når det er lyn og torden kan du bruke følgende metode for å finne ut omtrent hvor langt unna du er uværet:\os

\textsl{Start med å telle sekunder straks du ser et lyn. Stopp tellingen når du hører torden. Gang antall sekunder med 300, da har du et overslag på hvor mange meter du er unna uværet.}\os

Bruk internett til undersøke hastigheten til lys (lyn) og lyd (torden) i luft, og forklar hva denne metoden baserer seg på.

\eksop{1PV22D1}{1PV22D1opg5}
\tagop{
\#programmering \#prosentregning
}
\python{python/1pv22d1opg5.py}
En elev har skrevet programkoden ovenfor.
Hva ønsker eleven å finne ut?
Forklar hva som skjer når programmet kjøres.
\newpage
\eksop{1PV22D1}{1PV22D1opg2}
\tagop{\#prosentregning \#statistikk \#tallforståelse}
\fig{1pv22d1opg2}	
Diagrammet viser antall elever ved en videregående skole de fire siste årene.\os

Når var det størst prosentvis økning i antall elever fra et år til det neste?

\op{opgstorstar} 
\tagop{
	\# modellering \# areal
}
Gitt et rektangel med omkrets 4, og la $ x $ være den éne sidelengden.
\abc{
\item  Finn uttrykket til funksjonen $ A(x) $, som viser aralet til rektangelet.
\item Hva er $ x $ når rektangelet har størst areal? Hvilken form har rektangelet da?
}
\mers{I !!amto finner du en \textit{generalisert} utgave av denne oppgaven. Med andre ord kan det vises at formen du finner i oppgave b) alltid vil være den formen et rektangel må ha for å ha et størst mulig areal i forhold til omkretsen.}
\newpage		
\op{opgdusj}
\tagop{
\# omgjøring av enheter \# standardform \\\# proporsjonale størrelser 
}
Det har det blitt populært å regne ut hva det koster å ta seg en dusj. Til et slikt reknestykke kan man gjøre følgende antakelser:
\begin{itemize}
	\item Energien som kreves er energien som må til for å varme opp vannet som gikk med til dusjingen fra 7$^\circ $ til $ 35^\circ $.
	\item For å øke temperaturen til 1 liter vann med 1$ ^\circ $, kreves det $ 4,2\cdot10^3\enh{J} $.
\end{itemize}
Ifølge \net{https://www.vg.no/spesial/2022/stromprisene/}{vg.no} er 395,41 øre/kWh den høyeste (gjennomsnittlige) strømprisen registrert i Oslo. 
\abc{
\item Regn ut hva en dusj på 10 minutter ville kostet med denne prisen. 
\item Bruk internett til å finne strømprisene for din region i dag. Sjekk hva en 10 minutters dusj vil koste deg.
}
\textit{Obs! I denne oppgaven ser vi bort ifra \outl{nettleie}. Alle strømforbrukere må betale nettleie for frakt av strøm, og jo mer strøm man forbruker, jo høyere vil nettleien være, men strømprisen vil ha mest å si for hvor mye en dusj koster.}
\newpage
\op{opgtannhjul}
\tagop{
\#faktorisering \#primtall
}
I et system hvor to tannhjul virker sammen, er det ønskelig at en tann og et kammer møtes så sjeldent som mulig. Dette for å unngå slitasje. \vsk

\abc{
\item Et tannhjul med 12 tenner er koblet til et tannhjul med 21 tenner, som vist i figuren under. Hvor mange omdreininger må det røde tannhjulet ta for at tann A og kammer 1 skal møtes igjen?
\item Gjenta spørsmålet fra oppgave a), men hvor det blå tannhjulet er erstattet med et tannhjul som har 11 tenner.
}
\fig{tannhjul}
\newpage
\op{opgamazonas}
\tagop{
\#proporsjonale størrelser \#volum \#avrunding
}
\outl{Vannføringen} i en elv viser til volumet vann en elv frakter per tidsenhet. I en \net{https://en.wikipedia.org/wiki/List_of_rivers_by_discharge}{Wikipedia}-artikkel om verdens største elver kan man lese følgende:\regv

\st{
(...) The average flow rate at the mouth of the Amazon is sufficient to fill more than 83 such pools each second.\vsk

\textsl{\textbf{Norsk oversettelse}\\ (...) Den gjennomsnittlige vannføringen i munningen av Amazonas-elven er nok til å fylle mer enn 83 slike per sekund.}
}

Gjengi utregningene som er brukt for å finne de to tallene i teksten over, når du vet at artikkelen har brukt følgende som utgangspunkt:
\begin{itemize}
\item Bassenget det er snakk om er et olympisk svømmebasseng, som har har lengde 50\enh{m}, bredde 25\enh{m} og dybde 2\enh{m}.
\item Den gjennomsnittlige vannføringen i Amazonas-elven er 209\,000\enh{m}$ ^3 $/\text{s}
\end{itemize}
\newpage
\op{luftmotstand}
\tagop{
	\#algebra \#modellering \#andregradsfunksjon \\
	\#omgjøring av enheter \#proporsjonalitet
}

La $ F $ være summen av kreftene som virker i motsatt retning av en bils kjøreretning. Ifølge en rapport\footnote{\net{https://sintef.brage.unit.no/sintef-xmlui/handle/11250/2468761}{https://sintef.brage.unit.no/sintef-xmlui/handle/11250/2468761}} fra SINTEF kan\footnote{Det er er her forutsatt flatt strekke, og sett vekk ifra motstand ved akselerasjon.} $ F $ tilnærmes som
\[ F(v)= mgC_r+\frac{1}{2}\rho v^2 D_m\qquad,\qquad v\geq10\]
\begin{center}
	\begin{tabular}{c|c|c|c}
		& \textbf{betydning} & \textbf{verdi}&\textbf{enhet}  \\ \hline
		$ v $ & bilens hastighet & variabel& m/s \\
		$ m $& bilens masse\footnotemark & 1409 & kg\\
		$ g $& tyngdeakselerasjonen & 9.81 & m/$ \text{s}^2 $ \\
		$ C_r $ & koeffisient for bilens rullemotstand & 0.015\\
		$ \rho $ & tettheten til luft & 1.25 & kg/$ \text{m}^3 $ \\
		$ D_m $& koeffisient for bilens luftmotstandsareal\footnotemark &0.74
	\end{tabular}
\end{center}

\footnotetext[3]{Det er tatt ugangspunkt i gjennomsnittsvekten til en norsk personbil.}
\footnotetext[4]{Verdien er hentet fra \net{en.wikipedia.org/wiki/Automobile\_drag\_coefficient\#Drag\_area}{en.wikipedia.org/wiki/Automobile\_drag\_coefficient\#Drag\_area}}

\abc{
	\item Tegn grafen til $ F $ for $ v\in [10, 35] $
	\item På intervallet gitt i oppgave a, for hvilken hastighet er det at 
	\begin{itemize}
		\item rullemotstanden gir det største bidraget til $ F $?
		\item luftmotstanden gir det største bidraget til $ F $?
	\end{itemize} 
	Oppgi svarene rundet av til nærmeste heltall og målt i km/h.\newpage
	\item Med ''energiforbruk''\footnote{Den totale energimengden en bil bruker på en kjørelengde vil være høyere enn det vi har kalt ''energiforbruket''. Som regel vil den totale energimengden som kreves for å kjøre en strekning være høyere jo høyere hastighet man har. Slik kan man anta at differansen i energiforbruk vi finner i denne oppgaven er et minimum for den reelle differansen i total energimengde.} mener vi her den energien som må til for å motvirke $ F $ over en viss kjørelengde.
	Ved konstant hastighet er energiforbruket etter kjørt lengde proporsjonal med $ F $. På norske motorveier er 90\,km/h og 110\,km/h vanlige fartsgrenser. Hvor stor økning i energiforbruk vil en økning fra 90\,km/h til 110\,km/h innebære?\os
	
	\item Lag en funksjon $ F_1 $ som gir $ F $ ut ifra bilens hastighet målt i km/h.
}

\end{document}

