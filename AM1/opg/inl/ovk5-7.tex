\documentclass[english,hidelinks,pdftex, 11 pt, class=report,crop=false]{standalone}
\usepackage[T1]{fontenc}
\usepackage[utf8]{luainputenc}
\usepackage{lmodern} % load a font with all the characters
\usepackage{geometry}
\geometry{verbose,a4paper, inner=0cm, outer=0 cm, bmargin=2cm, tmargin=1cm}
%\textwidth=12cm
\setlength{\parindent}{0bp}
\usepackage{import}
\usepackage[subpreambles=false]{standalone}
\usepackage{amsmath}
\usepackage{amssymb}
\usepackage{esint}
\usepackage{babel}
\usepackage{tabu}
\usepackage[dvipsnames, table]{xcolor}
\usepackage{cancel}
\makeatother
\makeatletter
\usepackage{datetime2}
\usepackage{titlesec}
\usepackage[many]{tcolorbox}

% Eheter
\newcommand{\enh}[1]{\,\textrm{#1}}
%referances
\newcommand{\net}[2]{{\color{blue}\href{#1}{#2}}}

%Spaces
\newcommand{\vsk}{\\[12pt]}
\newcommand{\vs}{\vspace{-12pt}}

% Tabell for opplegg

\newcommand{\ovlist}[1]{
\vspace{-16pt}
\begin{itemize}
	#1
\end{itemize}
}

% Chapters and sections
\titleformat{\section}[block]{\bfseries}{\hspace{3cm}\thesection}{5pt}{}
\titleformat{\subsection}[block]{\bfseries}{\hspace{3cm}\thesection}{5pt}{}
\newcommand{\sectionbreak}{\clearpage} % New page on each section
 

\newlength{\mywidth}
\setlength{\mywidth}{14cm}

\newcommand{\cont}[1]{
\begin{tcolorbox}[center, boxrule=0.0 mm, width=\mywidth,arc=0mm,enhanced jigsaw,,colback=white,breakable]
#1	
\end{tcolorbox}
}

\newcommand{\info}[5]{
\begin{tcolorbox}[center, boxrule=0.1 mm, width=\mywidth,arc=0mm,enhanced jigsaw,breakable,colback=yellow!5]	
	
	\footnotesize
	\textbf{Øvingsområde}\\[5pt] #1 
	
	\textbf{Utstyr}\\ #2  \\
	
	\begin{tabular}{@{} p{4cm} p{4cm} l} 
		\textbf{Tid} & \textbf{Elevinndeling} & \textbf{Læringsarena} \\
		#3  & #4 & #5
	\end{tabular} 
\end{tcolorbox}	
}

\newcommand{\gjen}[1]{\begin{tcolorbox}[center,boxrule=0.1 mm, width=\mywidth,arc=0mm,colback=blue!3] {\large \textbf{Gjennomføring} \vspace{5 pt}}\newline #1  \end{tcolorbox}\vspace{-5pt}}
\newcommand{\eks}[1]{\begin{tcolorbox}[center,boxrule=0.1 mm, width=\mywidth,arc=0mm,colback=green!3] {\large \textbf{Eksempel} \vspace{5 pt}}\newline #1  \end{tcolorbox}\vspace{-5pt}}

\newcounter{opl}
%\numberwithin{opl}{article}


\newcommand{\opl}[1]{
\newpage
{\refstepcounter{opl} %\phantomsection 
\large \textbf{\theopl \;#1} \vsk}
}

% Headlines
\newcommand{\fork}{\textbf{Forkunnskapar}\\}
\newcommand{\forb}{\textbf{Forberedelsar}\\}
\newcommand{\opgvr}{\textbf{Oppgaver}}



%colors
\newcommand{\colr}[1]{{\color{red} #1}}
\newcommand{\colb}[1]{{\color{blue} #1}}
\newcommand{\colo}[1]{{\color{orange} #1}}
\newcommand{\colc}[1]{{\color{cyan} #1}}
\definecolor{projectgreen}{cmyk}{100,0,100,0}
\newcommand{\colg}[1]{{\color{projectgreen} #1}}

% Lister med bokstavar
\usepackage[inline]{enumitem}
% Opg
\newcommand{\abc}[1]{
	\begin{enumerate}[label=\alph*),leftmargin=18pt]
		#1
	\end{enumerate}
}

\usepackage[]{hyperref}

\newcounter{inl}
\numberwithin{inl}{chapter}
\newcommand{\inl}[1]{\vspace{15pt} \refstepcounter{inl} \textbf{Oppgave \theinl}\label{#1} \vspace{2 pt}\\}
\renewcommand\theinl{\arabic{inl}}

\newcommand{\inr}[1]{\vspace{3pt}\textbf{\ref{#1}}}
\begin{document}
{\textbf{\Large Øving for prøve i kapittel 5-7 (fredag 3. nov.)}}\\ \vspace{5 pt}

\inl{in1}
Gjør om:\os
\begin{tabular}{@{}l l l l}
	\textbf{a)} 14\,m til lengde målt i km.\\
	\textbf{b)} 25000\,m til lengde målt i mil.\\
	\textbf{c)} 23,5\,mm til lengde målt i dm.
\end{tabular}

\inl{in2}

Gjør om:\os
\begin{tabular}{@{}l l l l}
	\textbf{a)} 145\,m$ ^2 $ til areal målt i km$ ^2 $.\\
	\textbf{b)} 28000\,m$ ^2 $ til areal målt i dm$ ^2 $.\\
	\textbf{c)} 223,5\,mm$ ^2 $ til areal målt i dm$ ^2 $.
\end{tabular}

\inl{in3}

I en klasse er det 21 personer som kjører til skolen og 10 som tar båt. Hva er forholdet mellom antall personer som kjører og tar båt?

\inl{in4}
Du lager et lotteri og ønsker at forholdet mellom antall vinnerlodd og taperlodd skal være $ {2:7} $. Hvis du lager 12 vinnerlodd, hvor mange taperlodd må du da lage?

\inl{in5}
I en twistpose er det 15 sjokolader igjen, alle av de to beste sortene, som er \textsl{Marsipan} og \textsl{Cocos}. Forholdet mellom \textsl{Marsipan} og \textsl{Cocos} i posen er $ {1:4} $.\os
\textbf{a)} Hvor mange \textsl{Marsipan} og hvor mange \textsl{Cocos} ligger i posen?\os
\textbf{b)} Etter en stund har forholdet endret seg til $ {1:3} $. Hva kan ha skjedd?

\inl{in6}
Trekant $ \triangle ABC $ inneholder vinklene  $ 35^\circ $ og $ 60^\circ $, mens $ \triangle DEF $ inneholder vinklene  $ 85^\circ $ og $ 40^\circ $.\os
\textbf{a)} Finn den resterende vinkelen i begge trekantene.\os
\textbf{b)} Er trekantene formlike?
\newpage
\inl{in7}
\fig{tri4}
\textbf{a)} Forklar (NØYE!) hvorfor trekantene $ \triangle ABC $ og $ \triangle ADC $ er formlike. Lag en tegning som viser hvor på figuren man kan finne $ \angle A, \angle B $ og $ \angle C $.\os
\textbf{b)} Hvilke sider i trekantene er samsvarende?\os
\textbf{c)} \y{AB=10}, \y{BC=6} og \y{AC=8}. Forklar hvorfor høyden til $ \triangle ABC $ er 4,8.\os
\textbf{d)} Hva er arealet til $ \triangle ABC $?\os
\textbf{e)} Hva er arealet til $ \triangle ADC $?

\inl{in8}
\textbf{a)} Skriv om arealformelen for en trekant til en formel for grunnlinjen $ g $. Hva er $ g $ hvis $ {h=4} $ og \y{A=12}?\os
\textbf{b)} Skriv om arealformelen for et trapes til en formel for $ a $. Hva er $ a $ hvis $ {h=3, b=3} $ og $ A= 15$?\os
\textbf{c)} Skriv om arealformelen for et trapes til en formel for $ h $. Hva er $ h $ hvis $ {a=3, b=7} $ og $ {A=25} $?\os
\textbf{d)} Skriv om omkretsformelen for en sirkel til en formel for radiusen $ r $. Hva er $ r $ hvis $ {O=12\pi} $?
\newpage
\inl{in9}
Linjalen på bildet er en klassisk linjal med cm-mål.
\begin{figure}
	\centering
	\includegraphics[scale=0.08]{kart}
\end{figure}
På kartet over er huset til Sindre markert med den røde prikken til høyre, og Helland skule (ungdomsskolen i Vestnes) markert med den røde prikken til venstre. Kartet er i målestokken $ {1:50\,000} $.\os
\textbf{a)} Hvor langt er det mellom huset til Sindre og Helland skule?\os

\textbf{b)} Etter at dette kartet ble lagd, har en bro blitt bygget over Tresfjorden (fjorden på kartet). Broen er ca 2\,km lang i virkeligheten. Hvor lang blir denne broen på kartet?

\inl{in10}
I denne oppgaven bruker vi at $ \pi\approx 3 $.\os

En idrettsbane har mål som vist i figuren under:
\fig{tri25}
\textbf{a)} Finn omkretsen til idrettsbanen. Vurder om svaret du finner virker rimelig.\os
\textbf{b)} Finn arealet til idrettsbanen.
\newpage
\inl{in11}
\textbf{a)} Skriv om arealformelen for en sirkel til en formel for radiusen $ r $.\os
\textbf{d)} Hvis en sirkel har arealet $ 36\pi $, hva er da radiusen til sirkelen?

\inl{in12}
I en bøtte med 21\,L maling er det blandet grønn og rød maling i forholdet $ {2:5} $.\os
\textbf{a)} Hvor mye grønn maling er det i bøtten?\os

\textbf{b)} Hvor mye rød maling er det i bøtten?\os

\textbf{c)} Hva kan du gjøre for å endre forholdet til $ 2:9 $?\os
\textbf{d)} Hva kan du gjøre for å endre forholdet til $ 1:2 $?

\newpage
\section*{Løsningsforslag}
\inr{in1} \os

\begin{tabular}{@{}l l l l}
	\textbf{a)} $ 14\enh{m}=0,014\enh{km} $\\
	\textbf{b)} $ 25000\enh{m}=2,5\enh{mil} $\\
	\textbf{c)} $ 23,5\enh{mm}=0,235\enh{dm} $
\end{tabular}\vsk

\inr{in2}\os
\begin{tabular}{@{}l l l l}
	\textbf{a)} $ 145\enh{m}^2=0,000145\enh{km}^2 $.\\
	\textbf{b)} $ 28\,000\enh{m}^2=2\,800\,000\enh{dm}^2 $\\
	\textbf{c)} $ 223,5\enh{mm}^2=0,02235\enh{dm}^2 $.
\end{tabular}\vsk

\inr{in3}
Forholdet er:
\alg{
\frac{\text{antall som kjører til skolen}}{\text{antall som tar båt}}&=	
\frac{21}{10}\\&=2,1
}

\inr{in4}
Vi vet at:
\alg{
\frac{\text{vinnerlodd}}{\text{taperlodd}}&= \frac{2}{7}
}
12 vinnerlodd er 6 ganger mer enn 2. Det betyr at vi må ha 6 ganger flere taperlodd for at forholdet skal bli det samme:
\alg{
\frac{2\cdot 6}{7\cdot 6}&= \frac{12}{42}
}
Vi må altså lage 42 taperlodd.\vsk

\inr{in5}\\
\textbf{a)}
Siden forholdet er $ {1:4} $ er det $ {1+4=5} $ deler i alt. Det betyr at det er $ {15\cdot\frac{1}{5}=}3 $ \textsl{Marsipan} og $ {15\cdot\frac{4}{5}=12} $ \textsl{Cocos}.\os
\textbf{b)} 
\begin{itemize}
	\item Det kan ha blitt spist 3 \textsl{Cocos}. For da er forholdet $ {\frac{3}{9}=\frac{1}{3}} $.
	\item Det kan ha blitt spist 2 \textsl{Marsipan} og 9 \textsl{Cocos}. For da er forholdet $ {\frac{1}{3}} $.	
	\item Det kan ha blitt spist én \textsl{Marsipan} og 6 \textsl{Cocos}. For da er forholdet $ {\frac{2}{6}=\frac{1}{3}} $
\end{itemize} \vsk

\inr{in6}\\
\textbf{a)} For $ \triangle ABC $:
\alg{
180^\circ-35^\circ-60^\circ&= 85^\circ
}
For $ \triangle DEF $:
\alg{
	180^\circ-85^\circ-40^\circ&= 55^\circ
}
\textbf{b)} Trekantene har forskjellige vinkelverdier og er derfor ikke formlike.\vsk

\inr{in7}\\
(Noen ganger kan det være lurt å tegne trekantene hver for seg for et tydeligere bilde:)
\begin{figure}
	\centering
	\subfloat{\includegraphics[scale=0.7]{\asym{tri4a}}}\quad
	\subfloat{\includegraphics[scale=0.7]{\asym{tri4b}}}
\end{figure}
\textbf{a)} 
\begin{itemize}
	\item Trekantene deler $ \angle A $
	\item Begge trekantene har en $ 90^\circ $ vinkel.
	\item Trekantene har derfor to samsvarende vinkelverdier, og er da formlike.
\end{itemize}

\textbf{b)}\vs \vs
\begin{figure}
	\centering
	\subfloat{\includegraphics[scale=0.7]{\asym{tri4c}}}\quad
	\subfloat{\includegraphics[scale=0.7]{\asym{tri4d}}}
\end{figure}
\begin{itemize}
	\item $ AB $ og $ AC $ er samsvarende (hører til $ 90^\circ $-graderen).
	\item $ BC $ og $ DC $ er samsvarende (hører til $ \angle A $).	
	\item $ AC $ og $ AD $ er samsvarende (hører til $ \angle B $).		
\end{itemize}
\textbf{c)} Høyden i $ \triangle ABC $ er lengden av $ DC $. Av det vi fant i oppgave b) og \hr{forform} vet vi at:
\alg{
\frac{AC}{AB} &= \frac{DC}{BC} \\
\frac{8}{10}&= \frac{DC}{6} \\
\frac{8\cdot 6}{10}&= \frac{DC\cdot \cancel{6}}{\cancel{6}} \\
\frac{48}{10}&= DC \\
4,8 &= DC
}
Derfor er høyden 4,8.\os

\textbf{c)} $ \triangle ABC $ har grunnlinjen $ {AB=10} $ og høyden $ {DC=4,8} $:
\alg{
A &= \frac{g\cdot h}{2}\br
&= \frac{10\cdot4,8}{2}\br
&= \frac{48}{2}\\
&= 24
}

\textbf{d)} Hvis vi velger oss $ {AD} $ som grunnlinje i $ \triangle DEF $ blir høyden $ {DC=4,8} $. Lengde til $ AD $ kan vi finne på lignende måte som i opg b):
\alg{
\frac{AD}{AC}&= \frac{DC}{BC} \br
\frac{AD}{8}&= \frac{4,8}{6} \br
\frac{AD\cdot \cancel{8}}{\cancel{8}}&= \frac{4,8\cdot8}{6} \br
AD &= \frac{38,4}{6} \\
 &= 6,4
}
Arealet blir derfor: \vs
\alg{
	A &= \frac{6,4\cdot 4,8}{2}\br
	&= \frac{10\cdot4,8}{2}\br
	&= \frac{48}{2}\\
	&= 24
}\vsk

\inr{in8}\\
\textbf{a)} Vi skriver om arealformelen slik at $ h $ står alene på én side: 
\alg{
A &= \frac{g\cdot h}{2} \br
2\cdot A &= \frac{\cancel{2}\cdot g\cdot h}{\cancel{2}} \br
\frac{2A}{h} &= \frac{g\cdot \cancel{h}}{\cancel{h}} \br
\frac{2A}{h} &= g
}
Når vi vet at $ {h=4} $ og \y{A=12} kan vi bruke formelen over tl å finne $ g $:
\alg{
g &= \frac{2\cdot 12}{4} \\
 &= 6
}

\textbf{b)} Vi skriver om arealformelen slik at $ a $ står alene på én side: 
\alg{
	A &= \frac{a+b}{2} \br
	2\cdot A &= \frac{\cancel{2}\cdot g\cdot h}{\cancel{2}} \br
	\frac{2A}{h} &= \frac{g\cdot \cancel{h}}{\cancel{h}} \br
	\frac{2A}{h} &= g
}
Når vi vet at $ {h=4} $ og \y{A=12} kan vi bruke formelen over tl å finne $ g $:
\alg{
	g &= \frac{2\cdot 12}{4} \\
	&= 6
}
\textbf{b)} Vi skriver om arealformelen slik at $ a $ står alene på én side: 
\alg{
	A &= \frac{(a+b)h}{2} \br
	2\cdot A &=\cancel{2}\cdot\frac{(a+b)h}{\cancel{2}} \br
	\frac{2A}{h} &= \frac{(a+b)\cancel{h}}{\cancel{h}}\br
	\frac{2A}{h}-b &= a
}
Når vi vet at $ {h=3, b=3} $ og $ A= 15$ kan vi bruke formelen over til å finne $ a $:
\alg{
 a &= \frac{2\cdot 15}{3}-3 \\
 &= \frac{30}{3}-3 \\
 &= 10-3\\
 &= 7
}\os

\textbf{c)} Vi skriver om arealformelen slik at $ h $ står alene på én side: 
\alg{
	A &= \frac{(a+b)h}{2} \br
	2\cdot A &=\cancel{2}\cdot\frac{(a+b)h}{\cancel{2}} \br
	\frac{2A}{(a+b)} &= \frac{\bcancel{(a+b)}h}{{\bcancel{(a+b)}}}\br
	\frac{2A}{(a+b)} &= h
}
Når vi vet at $ {a=3, b=7} $ og $ A= 25$ kan vi bruke formelen over til å finne $ h $:
\alg{
h &= \frac{2\cdot 25}{(3+7)} \\
&= \frac{50}{10} \\
&= 5
}

\inr{in9} \\
\textbf{a)} På karter er det 10\,cm mellom huset og skolen. Målestokken sier at \textsl{én cm på kartet er 50\,000} i virkeligheten, altså at:
\alg{
	\text{10\,cm i virkeligheten}&= 10\cdot 50\,000\text{\,cm i virkeligheten} \\ 
	&= 500\,000\text{\,cm i virkeligheten} \\
	&= 5\enh{km}
}
Det er altså 5\,km mellom skolen og huset.\os

\textbf{b)} Målestokken forteller at:
\alg{
	\frac{\text{lengde på kart}}{\text{lengde i virkeligheten}}&= \frac{1}{50\,000} \br
	\frac{\text{lengde på kart}}{\text{2\,km}}&= \frac{1}{50\,000} \br
	\text{lengde på kart} &= \frac{200\,000\enh{cm}}{50\,000}\\
	&= 4\enh{cm}
}

\inr{in10}
Idrettsbanen består av to halvsirkler, begge med 40\,m radius, og to lengder, begge 80\,m. Da to halvsirklene kan vi slå sammen til én sirkel, som har omkretsen $O_s= 2\pi r $. Altså er:
\alg{
	O_s&=2\pi\cdot40 \\
	&\approx 2\cdot3\cdot40 \\
	&= 240
}
Omkretsen av hele løpebanen blir derfor:
\[ 240+80+80=400 \]
Det er derfor 400\,m rundt banen, noe som gir mening siden det er en idrettsbane. \vsk

\inr{in12}\\
\textbf{a)} Siden forholdet er $ {2:5} $ er det i alt $ {2+5=7} $ deler. Den grønne malingen utgjør derfor $ \frac{2}{7} $ av 21\,L som er:
\[ \frac{2}{7}\cdot21\,L =6\,L \] 
\textbf{b)} Siden det er 6\,L grønn maling må det være $ 21\,L-6\,L=15\,L $ rød maling.\os
\textbf{c)} Skal forholdet bli $ {2:9} $ trengervi 4 deler til med rød maling. Hver del er $ 21\,L:7=3\,L $, derfor må vi helle $ 3\,L\cdot4=12\,L $ maling i bøtta.\os
\textbf{d)} Hvis jeg har $ 3 $ deler grønn maling og 6 deler rød maling blir forholdet $ \frac{3}{6}=\frac{1}{2} $. Derfor må jeg tilsette $ 1\cdot3\,L=3\,L $ grønn maling og $ 1\cdot3\,L=3\,L $ rød maling. \vsk

\end{document}

