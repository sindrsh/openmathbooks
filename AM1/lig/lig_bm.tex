\documentclass[english,hidelinks,pdftex, 11 pt, class=report,crop=false]{standalone}
\usepackage[T1]{fontenc}
\usepackage[utf8]{luainputenc}
\usepackage{lmodern} % load a font with all the characters
\usepackage{geometry}
\geometry{verbose,a4paper, inner=0cm, outer=0 cm, bmargin=2cm, tmargin=1cm}
%\textwidth=12cm
\setlength{\parindent}{0bp}
\usepackage{import}
\usepackage[subpreambles=false]{standalone}
\usepackage{amsmath}
\usepackage{amssymb}
\usepackage{esint}
\usepackage{babel}
\usepackage{tabu}
\usepackage[dvipsnames, table]{xcolor}
\usepackage{cancel}
\makeatother
\makeatletter
\usepackage{datetime2}
\usepackage{titlesec}
\usepackage[many]{tcolorbox}

% Eheter
\newcommand{\enh}[1]{\,\textrm{#1}}
%referances
\newcommand{\net}[2]{{\color{blue}\href{#1}{#2}}}

%Spaces
\newcommand{\vsk}{\\[12pt]}
\newcommand{\vs}{\vspace{-12pt}}

% Tabell for opplegg

\newcommand{\ovlist}[1]{
\vspace{-16pt}
\begin{itemize}
	#1
\end{itemize}
}

% Chapters and sections
\titleformat{\section}[block]{\bfseries}{\hspace{3cm}\thesection}{5pt}{}
\titleformat{\subsection}[block]{\bfseries}{\hspace{3cm}\thesection}{5pt}{}
\newcommand{\sectionbreak}{\clearpage} % New page on each section
 

\newlength{\mywidth}
\setlength{\mywidth}{14cm}

\newcommand{\cont}[1]{
\begin{tcolorbox}[center, boxrule=0.0 mm, width=\mywidth,arc=0mm,enhanced jigsaw,,colback=white,breakable]
#1	
\end{tcolorbox}
}

\newcommand{\info}[5]{
\begin{tcolorbox}[center, boxrule=0.1 mm, width=\mywidth,arc=0mm,enhanced jigsaw,breakable,colback=yellow!5]	
	
	\footnotesize
	\textbf{Øvingsområde}\\[5pt] #1 
	
	\textbf{Utstyr}\\ #2  \\
	
	\begin{tabular}{@{} p{4cm} p{4cm} l} 
		\textbf{Tid} & \textbf{Elevinndeling} & \textbf{Læringsarena} \\
		#3  & #4 & #5
	\end{tabular} 
\end{tcolorbox}	
}

\newcommand{\gjen}[1]{\begin{tcolorbox}[center,boxrule=0.1 mm, width=\mywidth,arc=0mm,colback=blue!3] {\large \textbf{Gjennomføring} \vspace{5 pt}}\newline #1  \end{tcolorbox}\vspace{-5pt}}
\newcommand{\eks}[1]{\begin{tcolorbox}[center,boxrule=0.1 mm, width=\mywidth,arc=0mm,colback=green!3] {\large \textbf{Eksempel} \vspace{5 pt}}\newline #1  \end{tcolorbox}\vspace{-5pt}}

\newcounter{opl}
%\numberwithin{opl}{article}


\newcommand{\opl}[1]{
\newpage
{\refstepcounter{opl} %\phantomsection 
\large \textbf{\theopl \;#1} \vsk}
}

% Headlines
\newcommand{\fork}{\textbf{Forkunnskapar}\\}
\newcommand{\forb}{\textbf{Forberedelsar}\\}
\newcommand{\opgvr}{\textbf{Oppgaver}}



%colors
\newcommand{\colr}[1]{{\color{red} #1}}
\newcommand{\colb}[1]{{\color{blue} #1}}
\newcommand{\colo}[1]{{\color{orange} #1}}
\newcommand{\colc}[1]{{\color{cyan} #1}}
\definecolor{projectgreen}{cmyk}{100,0,100,0}
\newcommand{\colg}[1]{{\color{projectgreen} #1}}

% Lister med bokstavar
\usepackage[inline]{enumitem}
% Opg
\newcommand{\abc}[1]{
	\begin{enumerate}[label=\alph*),leftmargin=18pt]
		#1
	\end{enumerate}
}

\usepackage[]{hyperref}

\newcommand{\note}{Merk}
\newcommand{\notesm}[1]{{\footnotesize \textsl{\note:} #1}}
\newcommand{\ekstitle}{Eksempel }
\newcommand{\sprtitle}{Språkboksen}
\newcommand{\expl}{forklaring}
\newcommand{\pyt}{Pytagoras' setning}
\newcommand\sv{\vsk \textbf{Svar} \vspace{4 pt}\\}

%references
\newcommand{\reftab}[1]{\hrs{#1}{tabell}}
\newcommand{\rref}[1]{\hrs{#1}{regel}}
\newcommand{\dref}[1]{\hrs{#1}{definisjon}}
\newcommand{\refkap}[1]{\hrs{#1}{kapittel}}
\newcommand{\refsec}[1]{\hrs{#1}{seksjon}}
\newcommand{\refdsec}[1]{\hrs{#1}{delseksjon}}
\newcommand{\refved}[1]{\hrs{#1}{vedlegg}}
\newcommand{\eksref}[1]{\textsl{#1}}
\newcommand\fref[2][]{\hyperref[#2]{\textsl{figur \ref*{#2}#1}}}
\newcommand{\refop}[1]{{\color{blue}Oppgave \ref{#1}}}
\newcommand{\refops}[1]{{\color{blue}oppgave \ref{#1}}}


%Algebra
\newcommand{\kvadset}{Kvadratsetningene}
\newcommand{\aenato}{Sum-produkt-metoden}

% Geometry
\newcommand{\hlikb}{Midtnormalen i en likebeint trekant}
\newcommand{\arealsetn}{Arealsetningen}
\newcommand{\trkmedian}{Median}
\newcommand{\midtrk}{Midtnormal (i trekant)}
\newcommand{\innskrsirk}{Innskrevet sirkel}
\newcommand{\cossetn}{Cosinussetningen}
\newcommand{\perfvink}{Sentral- og periferivinkel}
\newcommand{\tang}{Tangent}

% Derivative
\newcommand{\derel}{Den deriverte av elementære funksjoner}
\newcommand{\divder}{Divisjonsregelen}
\newcommand{\kjernereg}{Kjerneregelen}
\newcommand{\prodregder}{Produktregelen}
\newcommand{\lhop}{L'Hopitals regel}

% Funksjonsdrofting
\newcommand{\monder}{Monotoniegenskaper og den deriverte}
\newcommand{\fderekstr}{$ \bm{f'=0} $ for lokale ektstremalpunkt}
\newcommand{\andredertest}{Andrederiverttesten}

% Vectors
\newcommand{\detar}{Arealformler med determinanter}
\newcommand{\avstpunktlin}{Avstand mellom punkt og linje}

%Appendix
\newcommand{\rolle}{Rolles teorem}
\newcommand{\meanval}{Middelverdisetningen}

% Solutions manual
\newcommand{\selos}{Se løsningsforslag.}
\begin{document}

\section{Å finne størrelser}
Likninger, formler og funksjoner (og utttrykk) er begrep som dukker opp i forskjellige sammenhenger, men som i bunn og grunn handler om det samme; \textsl{de uttrykker relasjoner mellom størrelser}. Når alle størrelsene utenom den éne er kjent, kan vi finne denne enten direkte eller indirekte.\vsk

\subsection{Å finne størrelser direkte}
Mange av regelboksene i boka inneholder en formel. Når en størrelse står alene på én side av formelen, sier vi at det er en formel for \textsl{den} størrelsen. For eksempel inneholder \rref{maalstk} en formel for 'målestokk'. Når de andre størrelsene er gitt, er det snakk om å sette verdiene inn i formelen og regne ut for å finne den ukjente,  'målestokk'.  \vsk

Men ofte har vi bare en beskrivelse av en situasjon, og da må vi selv lage formlene. Da gjelder det å først identifisere hvilke størrelser som er til stede, og så finne relasjonen mellom dem.\regv
\eks[1]{
For en taxi er det følgende kostnader:
\begin{itemize}
	\item Du må betale 50\enh{kr} uansett hvor langt du blir kjørt.
	\item I tillegg betaler du 15\enh{kr} for hver kilometer du blir kjørt.
\end{itemize}
\abc{
\item Sett opp et uttrykk for hvor mye taxituren koster for hver kilometer du blir kjørt.
\item Hva koster en taxitur på 17\enh{km}?
}

\sv \vs
\abc{
\item Her er det to ukjente størrelser; 'kostnaden for taxituren' og 'antall kilometer kjørt'. Relasjonen mellom dem er denne:
	\[ \text{kostnaden for taxituren}=50+15\cdot\text{antall kilometer kjørt} \]
\item Vi har nå at
\[ \text{kostnaden for taxituren}=50+15\cdot17= 305 \]
Taxituren koster altså 305\enh{kr}.
}
}
\info{Tips}{
Ved å la enkeltbokstaver representere størrelser, får man kortere uttrykk. La $ k $ stå for 'kostnad for taxituren' og $ x $ for 'antall kilometer kjørt'. Da blir uttrykket fra \textsl{Eksempel 1} over dette:
\[ k=50+15x \]
I tillegg kan man gjerne bruke skrivemåten for funksjoner:
\[ k(x)=50+15x \] 
}
\subsection{Å finne størrelser indirekte}
\subsubsection{Når formlene er kjente}
\eks[1]{
	Vi har sett at strekningen $ s $ vi har kjørt, farten $ f $ vi har holdt, og tiden $ t $ vi har brukt kan settes i sammenheng via formelen\footnote{$ \text{strekning}=\text{fart}\cdot \text{tid} $}:
	\[ s = f\cdot t \] 
	Dette er altså en formel for $ s $. Ønsker vi i stedet en formel for $ f $, kan vi gjøre om formelen ved å følge prinsippene for likninger\footnote{Se \mb, s. 121.}:
	\alg{
		s &= f\cdot t \br
		\frac{s}{t}&=\frac{f\cdot \bcancel{t}}{\bcancel{t}} \br
		\frac{s}{t}&=f
	}
}
\newpage
\eks[2]{
	\textit{Ohms lov} sier at strømmen $ I $ gjennom en metallisk leder (med konstant temeperatur) er gitt ved formelen
	\[ I = \frac{U}{R} \]
	hvor $ U $ er spenningen og $ R $ er resistansen.
	
	\abc{
	\item Skriv om formelen til en formel for $ R $.	
}
Strøm måles i Ampere (A), spenning i Volt (V) og motstand i Ohm ($ \Omega $).
\abcs{2}{
\item Hvis strømmen er 2\,A og spenningen 12\,V, hva er da resistansen?
}

	
	\sv \vs
\abc{
	\item Vi gjør om formelen slik at $ R $ står alene på én side av likhetstegnet:\vs
\alg{
	I\cdot R&=\frac{U\cdot \cancel{R}}{\cancel{R}} \br
	I\cdot R &= U \br
	\frac{\cancel{I}\cdot R}{\cancel{I}} &= \frac{U}{I}\br 
	R &= \frac{U}{I}
}
\item Vi bruker formelen vi fant i a), og får at
\alg{
	R &= \frac{U}{I} \br
	&= \frac{12}{2} \\
	&= 6
}
Resistansen er altså $ 6\,\Omega $.
}
}
\newpage
\eks[3]{
	Gitt en temperatur $ T_C $ målt i antall grader Celsius ($ ^\circ C $). Temperaturen $ T_F $ målt i antall grader Fahrenheit ($ ^\circ F $) er da gitt ved formelen
	\[ T_F = \frac{9}{5}\cdot T_C+32 \]
\abc{
\item Skriv om formelen til en formel for $ T_C $.
\item Hvis en temperatur er målt til 59$ ^\circ F $, hva er da temperaturen målt i $ ^\circ C $?
}
	
	\sv \vs
	
	\abc{
	\item Vi isolerer $ T_C $ på én side av likhetstegnet:
	\alg{
		T_F &= \frac{9}{5}\cdot T_C+32 \\
		T_F-32 &= \frac{9}{5}\cdot T_C \\
		5(T_F-32) &= \cancel{5}\cdot\frac{9}{\cancel{5}}\cdot F_C \\
		5(T_F-32) &= 9T_C \\
		\frac{5(T_F-32)}{9} &= \frac{\cancel{9}T_C}{\cancel{9}} \\
		\frac{5(T_F-32)}{9} &= T_C
	}
	\item Vi bruker formelen fra a), og finner at
	\alg{
		T_C&= \frac{5(59-32)}{9} \br
		&= \frac{5(27)}{9} \br
		&= 5\cdot 3 \\
		&= 15
	}
}
}

\subsubsection{Når formlene er ukjente}
\eks[1]{
Tenk at klassen ønsker å dra på en klassetur som til sammen koster 11\,000\,kr. For å dekke utgiftene har dere allerede skaffet 2\,000\,kr, resten skal skaffes gjennom loddsalg. For hvert lodd som selges, tjener dere 25\,kr.\os

\abc{
\item Lag en likning for hvor mange lodd klassen må selge for å få råd til klasseturen.
\item Løs likningen.
}

\sv
\abc{
\item Vi starter med å tenke oss regnestykket i ord:
\small
\[ \text{penger allerede skaffet}+\text{antall lodd}\cdot\text{penger per lodd}=\text{prisen på turen} \]
\normalsize
Den eneste størrelsen vi ikke vet om er 'antall lodd'. Vi erstatter\footnote{Dette gjør vi bare fordi det da blir mindre for oss å skrive.} \textit{antall lodd} med $ x $, og setter verdiene til de andre størrelsene inn i likningen:
\[ 2\,000+x\cdot25 = 11\,000 \]
\item \ \vs \vs
\alg{
	25x &= 11\,000-2\,000\\
	25 x &= 9\,000\\
	\frac{\cancel{25} x}{\cancel{25}} &= \frac{9\,000}{25} \\
	x &= 360
}
}
}
\newpage
\eks[2]{En vennegjeng ønsker å spleise på en bil som koster 50\,000 kr, men det er usikkert hvor mange personer som skal være med på å spleise.\os 
	\textbf{a)} Kall 'antall personer som blir med på å spleise' for $ P $ og 'utgift per person' for $ U $,  og lag en formel for $ U $.\os
	
	\textbf{b)} Finn utgiften per person hvis 20 personer blir med.
	
	\sv
	\textbf{a)} Siden prisen på bilen skal deles på antall personer som er med i spleiselaget, må formelen bli
	\[ U = \frac{50\,000}{P} \]
	
	\textbf{b)} Vi erstatter $ P $ med 20, og får
	\alg{
		U &= \frac{50\,000}{20}\\
		&= 2\,500
	}
	Utgiften per person er altså 2\,500\enh{kr}.
}
\eks[2]{
	En klasse planlegger en tur som krever bussreise. De får tilbud fra to busselskap:
	\begin{itemize}
		\item \textbf{Busselskap 1} \\
		Klassen betaler 10\,000\,kr uansett, og 10\enh{kr} per km.
		\item \textbf{Busselskap 2} \\
		Klassen betaler 4\,000\,kr uansett, og 30\enh{kr} per km.
	\end{itemize}
	For hvilken lengde kjørt tilbyr busselskapene samme pris?
	
	\sv
	
	Vi innfører følgende variabler:
	\begin{itemize}
		\item $ x=\text{antall kilometer kjørt} $ \\
		\item $ f(x)=\text{pris for Busselskap 1} $ \\
		\item $ g(x)=\text{pris for Busselskap 2} $
	\end{itemize}
	Da er
	\alg{
		f(x)&=10x+10\,000\vn
		g(x)&=30x+4\,000
	}
	Videre løser vi nå oppgaven både med en grafisk og en algebraisk metode. \vsk
	
	\metode{Grafisk metode}{0.6\linewidth} \os
	Vi tegner grafene til funksjonene inn i samme koordinatsystem:
	\fig{lig2}
	Vi leser av at funksjonene har samme verdi når $ {x=200} $. Dette betyr at busselskapene tilbyr samme pris hvis klassen skal kjøre 200\enh{km}.\vsk \vsk
	
	\metode{Algebraisk metode}{0.6\linewidth} \os
	Busselskapene har samme pris når
	\alg{
		f(x)&=g(x) \\
		10x+10\,000&=30x+6\,000 \\
		4\,000&=20x \\
		x&=200
	}
	Busselskapene tilbyr altså samme pris hvis klassen skal kjøre 200\enh{km}.
}
\eks[4]{
	''Broren min er dobbelt så gammel som meg. Til sammen er vi 9 år gamle. Hvor gammel er jeg?''.
	
	\sv
	''Broren min er dobbelt så gammel som meg.'' betyr at
	\[ \text{brors alder}=2\cdot \text{min alder} \]
	''Til sammen er vi 9 år gamle.'' betyr at
	\[ \text{brors alder}+\text{min alder}=\text{9} \]
	Erstatter vi 'brors alder' med ''$2\cdot\text{min alder} $'', får vi
	\[ 2\cdot\text{min alder}+\text{min alder}=\text{9} \]
	Altså er
	\algv{
		3\cdot \text{min alder} &= 9 \\
		\frac{\cancel{3}\cdot \text{min alder}}{\cancel{3}}&= \frac{9}{3} \\
		\text{min alder} &= 3
	}
	''Jeg'' er altså 3 år gammel.
}
\subsection{Grafisk metode}
\reg[Grafisk løsning av likningssett]{
Et lineært likningssett bestående av to ukjente, $ x $ og $ y $, kan løses ved å 
	\begin{enumerate}
		\item omskrive de to likningene til uttrykk for to linjer.
		\item finne skjæringspunktet til linjene.
\end{enumerate}
}
\section{Regresjon}
Å forsøke å beskrive hvordan noe vil \textsl{utvikle} seg er en av de viktigste anvendelsene for funksjoner. Hvis vi har et datasett som beskriver tidligere hendelser, kan vi prøve å finne den funksjonen som passer best til datasettet. Dette kalles å utføre \outl{regresjon}. \vsk

Grafen under viser\footnote{Tall hentet fra \net{https://elbil.no/om-elbil/elbilstatistikk/}{elbil.no}} antall elbiler i Norge etter år 2010.
\fig{elbilsalg1}
Vi ønsker nå å finne en funksjon som
\begin{enumerate}[label=(\roman*)]
	\item så godt som mulig skjærer hvert punkt.
	\item har en graf som passer til situasjonen vi modellerer.
\end{enumerate}
Hvis vi utfører regresjon med en lineær funksjon i GeoGebra (se side \pageref{ggbreg}), får vi denne grafen:
\fig{elbilsalg2}
Utfører vi regresjon også med en andregradsfunksjon, får vi følgende resultat:
\fig{elbilsalg3}
I figuren over kan vi merke oss at
\begin{itemize}
	\item begge modellene (funksjonene) ''oppfører'' seg feilaktig i starten. Den lineære funksjonen starter med et negativt antall biler, mens den kvadratiske funksjonen starter med at antallet synker fra år 0 til år 1.
	\item Grafen til den kvadratiske passer punktene mye bedre enn grafen til den lineære funksjonen.
\end{itemize}
Hvis vi hadde antatt at den lineære funksjonen ga en god beskrivelse av antallet elbiler fremover i tid, kunne vi lest av fra grafen at antall elbiler i 2021 var ca. 350\,000. Hadde vi i stedet antatt det samme om den kvadratiske funksjonen, kunne vi lest av fra grafen at antall elbiler i 2021 var litt over 425\,000. Fasit er at antall elbiler i 2021 var 455\,271.
\fig{elbilsalg4}
\newpage
\end{document}


