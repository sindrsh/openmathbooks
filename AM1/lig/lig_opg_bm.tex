\documentclass[english,hidelinks,pdftex, 11 pt, class=report,crop=false]{standalone}
\usepackage[T1]{fontenc}
\usepackage[utf8]{luainputenc}
\usepackage{lmodern} % load a font with all the characters
\usepackage{geometry}
\geometry{verbose,a4paper, inner=0cm, outer=0 cm, bmargin=2cm, tmargin=1cm}
%\textwidth=12cm
\setlength{\parindent}{0bp}
\usepackage{import}
\usepackage[subpreambles=false]{standalone}
\usepackage{amsmath}
\usepackage{amssymb}
\usepackage{esint}
\usepackage{babel}
\usepackage{tabu}
\usepackage[dvipsnames, table]{xcolor}
\usepackage{cancel}
\makeatother
\makeatletter
\usepackage{datetime2}
\usepackage{titlesec}
\usepackage[many]{tcolorbox}

% Eheter
\newcommand{\enh}[1]{\,\textrm{#1}}
%referances
\newcommand{\net}[2]{{\color{blue}\href{#1}{#2}}}

%Spaces
\newcommand{\vsk}{\\[12pt]}
\newcommand{\vs}{\vspace{-12pt}}

% Tabell for opplegg

\newcommand{\ovlist}[1]{
\vspace{-16pt}
\begin{itemize}
	#1
\end{itemize}
}

% Chapters and sections
\titleformat{\section}[block]{\bfseries}{\hspace{3cm}\thesection}{5pt}{}
\titleformat{\subsection}[block]{\bfseries}{\hspace{3cm}\thesection}{5pt}{}
\newcommand{\sectionbreak}{\clearpage} % New page on each section
 

\newlength{\mywidth}
\setlength{\mywidth}{14cm}

\newcommand{\cont}[1]{
\begin{tcolorbox}[center, boxrule=0.0 mm, width=\mywidth,arc=0mm,enhanced jigsaw,,colback=white,breakable]
#1	
\end{tcolorbox}
}

\newcommand{\info}[5]{
\begin{tcolorbox}[center, boxrule=0.1 mm, width=\mywidth,arc=0mm,enhanced jigsaw,breakable,colback=yellow!5]	
	
	\footnotesize
	\textbf{Øvingsområde}\\[5pt] #1 
	
	\textbf{Utstyr}\\ #2  \\
	
	\begin{tabular}{@{} p{4cm} p{4cm} l} 
		\textbf{Tid} & \textbf{Elevinndeling} & \textbf{Læringsarena} \\
		#3  & #4 & #5
	\end{tabular} 
\end{tcolorbox}	
}

\newcommand{\gjen}[1]{\begin{tcolorbox}[center,boxrule=0.1 mm, width=\mywidth,arc=0mm,colback=blue!3] {\large \textbf{Gjennomføring} \vspace{5 pt}}\newline #1  \end{tcolorbox}\vspace{-5pt}}
\newcommand{\eks}[1]{\begin{tcolorbox}[center,boxrule=0.1 mm, width=\mywidth,arc=0mm,colback=green!3] {\large \textbf{Eksempel} \vspace{5 pt}}\newline #1  \end{tcolorbox}\vspace{-5pt}}

\newcounter{opl}
%\numberwithin{opl}{article}


\newcommand{\opl}[1]{
\newpage
{\refstepcounter{opl} %\phantomsection 
\large \textbf{\theopl \;#1} \vsk}
}

% Headlines
\newcommand{\fork}{\textbf{Forkunnskapar}\\}
\newcommand{\forb}{\textbf{Forberedelsar}\\}
\newcommand{\opgvr}{\textbf{Oppgaver}}



%colors
\newcommand{\colr}[1]{{\color{red} #1}}
\newcommand{\colb}[1]{{\color{blue} #1}}
\newcommand{\colo}[1]{{\color{orange} #1}}
\newcommand{\colc}[1]{{\color{cyan} #1}}
\definecolor{projectgreen}{cmyk}{100,0,100,0}
\newcommand{\colg}[1]{{\color{projectgreen} #1}}

% Lister med bokstavar
\usepackage[inline]{enumitem}
% Opg
\newcommand{\abc}[1]{
	\begin{enumerate}[label=\alph*),leftmargin=18pt]
		#1
	\end{enumerate}
}

\usepackage[]{hyperref}

\newcommand{\note}{Merk}
\newcommand{\notesm}[1]{{\footnotesize \textsl{\note:} #1}}
\newcommand{\ekstitle}{Eksempel }
\newcommand{\sprtitle}{Språkboksen}
\newcommand{\expl}{forklaring}
\newcommand{\pyt}{Pytagoras' setning}
\newcommand\sv{\vsk \textbf{Svar} \vspace{4 pt}\\}

%references
\newcommand{\reftab}[1]{\hrs{#1}{tabell}}
\newcommand{\rref}[1]{\hrs{#1}{regel}}
\newcommand{\dref}[1]{\hrs{#1}{definisjon}}
\newcommand{\refkap}[1]{\hrs{#1}{kapittel}}
\newcommand{\refsec}[1]{\hrs{#1}{seksjon}}
\newcommand{\refdsec}[1]{\hrs{#1}{delseksjon}}
\newcommand{\refved}[1]{\hrs{#1}{vedlegg}}
\newcommand{\eksref}[1]{\textsl{#1}}
\newcommand\fref[2][]{\hyperref[#2]{\textsl{figur \ref*{#2}#1}}}
\newcommand{\refop}[1]{{\color{blue}Oppgave \ref{#1}}}
\newcommand{\refops}[1]{{\color{blue}oppgave \ref{#1}}}


%Algebra
\newcommand{\kvadset}{Kvadratsetningene}
\newcommand{\aenato}{Sum-produkt-metoden}

% Geometry
\newcommand{\hlikb}{Midtnormalen i en likebeint trekant}
\newcommand{\arealsetn}{Arealsetningen}
\newcommand{\trkmedian}{Median}
\newcommand{\midtrk}{Midtnormal (i trekant)}
\newcommand{\innskrsirk}{Innskrevet sirkel}
\newcommand{\cossetn}{Cosinussetningen}
\newcommand{\perfvink}{Sentral- og periferivinkel}
\newcommand{\tang}{Tangent}

% Derivative
\newcommand{\derel}{Den deriverte av elementære funksjoner}
\newcommand{\divder}{Divisjonsregelen}
\newcommand{\kjernereg}{Kjerneregelen}
\newcommand{\prodregder}{Produktregelen}
\newcommand{\lhop}{L'Hopitals regel}

% Funksjonsdrofting
\newcommand{\monder}{Monotoniegenskaper og den deriverte}
\newcommand{\fderekstr}{$ \bm{f'=0} $ for lokale ektstremalpunkt}
\newcommand{\andredertest}{Andrederiverttesten}

% Vectors
\newcommand{\detar}{Arealformler med determinanter}
\newcommand{\avstpunktlin}{Avstand mellom punkt og linje}

%Appendix
\newcommand{\rolle}{Rolles teorem}
\newcommand{\meanval}{Middelverdisetningen}

% Solutions manual
\newcommand{\selos}{Se løsningsforslag.}

\begin{document}
\opgt

\op{lig6}
Ola og Kari tilbyr et kurs i svømming. For kurset tjener de til sammen 12\,000\,kr. Ola er assistenten til Kari, og Kari skal ha dobbelt så mye av inntekten som Ola. \os

Hvor mye tjener Ola og hvor mye tjener Kari for kurset?

\op{lig7}
Du skal snekre et gjerde som er 3,4 m langt. For å lage gjerdet skal du bruke 8 planker som er 0,25 m breie, som vist i figuren under. Det skal være den samme avstanden mellom alle plankene. 
\fig{gj}
\abch{
\item Sett opp en ligning ut ifra beskrivelsen over. La $ x $ være avstanden mellom plankene.
\item Løs ligningen fra a).
}

\op{ligvol} \vs
\abc{
\item Skriv dette som en ligning: ''Volumet til en firkantet prisme med bredde 4, lengde 7 og høgde $ x $ er 252.''
\item Løs ligningen fra oppgave a).
}

\op{ligpro1} \vs
\abc{
\item Skriv dette som en ligning:
''25\% av $ x $ er lik 845''.
\item Løs ligningen fra oppgave a).
}
\newpage
\op{ligpro2}
Det gis 360\enh{kr} rabatt på en vare, og dette tilsvarer 20\% av \\ originalprisen.
\abc{
\item La $ x $ være originalprisen på varen. Sett opp en ligning som beskriver informasjonen gitt over.
\item Finn originalprisen til varen.
}

\eksop{GV23D1}{gv23d1opg7}
Marco kjøpte et headset til 779\enh{kr}. Før rabatten på 200 kroner, kostet headsettet 979 kroner.\os

Omtrent hvor mange prosent rabatt fikk Marco?

\eksop{GV2023D1}{GV2023D1opg1}
To slikkepinner og to sjokolader koster 32 kr.\\
Fire slikkepinner og to sjokolader koster 44 kr.\os
Hvor mye koster en slikkepinne?


\newpage


\eksop{E22}{eksu22opg7}
Arne har 120 kr, mens de fem søsknene hans har 30 kr hver. Arne og søsknene skal fordele pengene slik at alle har like mye. Hvor mange kroner må Arne gi til hver av søsknene sine?

\op{frm4}
Effekten $ P $ (målt i Watt) i en elektrisk krets er gitt ved formelen:
\[ P=R\cdot I^2 \]
hvor $ R $ er motstanden og $ I $ er strømmen i kretsen.\os
\textbf{a)} Hvis $ {R=5\,\Omega} $ og $ {I=10\,A} $, hva er da effekten?\os
\textbf{b)} Skriv om formelen til en formel for $ I^2 $.

\op{frmtra} 
Skriv om arealformelen for et trapes (se \mb, s. 143) til en formel for høgden.

\op{frm5}
På
\net{http://www.klikk.no/foreldre/smabarn/regn-ut-barnets-hoyde-som-voksen-2446226}{klikk.no} finner man disse formelene for å regne ut hvor høy et barn kommer til å bli:\os

\textit{For jenter:}
\begin{enumerate}
	\item Regn ut mors høyde i cm + fars høyde i cm
	\item Trekk fra 13 cm
	\item Del med 2.
\end{enumerate}

\textit{For gutter:}
\begin{enumerate}
	\item Regn ut mors høyde i cm + fars høyde i cm
	\item Legg til 13\enh{cm} 
	\item Del med 2.
\end{enumerate}
Kall barnets (fremtidige) høyde for $ B $, mors høyde for $ M $, og fars høde for $ F $.
\abc{
\item Lag en formel for $ B $ når barnet er ei jente.
\item Lag en formel for $ B $ når barnet er en gutt.
\item Gjør om formelen fra a) til en formel for $ F $.
\item Ei jente har en mor som er 165\,cm. Formelen fra oppgave a) sier at jenta vil bli 171\,cm høy. Hvor høy er faren til jenta?}

\op{ligindks}
I 2005 kostet en sykkel 1\,500\,kr, mens den i 2014 ville kostet 1\,784\,kr om prisen hadde fulgt konsumprisindeksen. \os

I 2005 var KPI 82,3, hva var den i 2014?


\op{ligskj1}
Gitt de to funksjonene
\algv{
f(x)&=3x-7 \vn
g(x)&=x+5
}
Finn skjæringspunktet til funksjonene.

\op{ligskj2}
Gitt de to funksjonene
\algv{
	f(x)&=-2x-3 \vn
	g(x)&=4x+9
}
Finn skjæringspunktet til funksjonene.

\op{ligmob}
Si at du kan velge mellom disse to månedsabonnementene for mobil:
\begin{itemize}
	\item \textbf{Abonnement A} \\
	300\enh{kr} i fast pris og 50\enh{kr} per GB data brukt.
	\item \textbf{Abonnement B} \\
	Fast pris på 500\enh{kr} og 10\enh{kr} per GB data brukt.
\end{itemize}
\abc{
\item For hvilket databruk vil abonnementene koste det samme?
\item Hvis du bruker ca. 7 GB data i måneden, hvilket abonnement bør du da velge?
} 

\eksop{GV23D1}{gv23d1opg6}
Jenny kjørte fra hjemmet sitt til hytta. Nedenfor er en grafisk framstilling av sammenhengen
mellom tiden (timer) og \\ strekningen (km) for turen til Jenny. \vs
\fig{gv23d1opg6}
Bestem stigningstallet til funksjonen, og forklar sammenhengen mellom stigningstallet og Jennys gjennomsnittsfart.

\eksop{1PV23D1}{1PV22D1opg4}
Tabellen nedenfor viser høgda til Klara noen år fra hun var 4 år, til hun var 10 år.
\begin{center}
	\begin{tabular}{|c|c|c|c|c|}
		\hline
		Alder (år) & 4 & 5 & 8 & 10 \\ \hline
		Høgde (cm) & 100 & 107 & 128 & 142\\
		\hline
	\end{tabular}
\end{center}
\abc{
\item Lag en modell som viser sammenhengen mellom høgda og alderen til Klara basert på tallene i tabellen.
\item Hvor høg vil Klara være når hun fyller 19 år, ifølge modellen?
}
Klara var 50 cm høg då ho blei fødd.
\abcs{3}{
\item Gjør beregninger og vurder gyldighetsområdet\footnote{Se \vedlegg{funk} for ordbeskrivelse} til modellen du fant i oppgave a).
}
\newpage
\eksop{1PV22D1}{1PV22D1opg4}
Siri har et stykke papp og vil lage en eske. Hun har satt opp en modell som viser
volumet $ V(x) $ cm$ ^3 $ av esken dersom hun lager den $ x $ cm høy
\[ 4x^3-100x^2+600x\quad,\quad 0<x<10\]
\abc{
\item Hvor stort volum får esken dersom Siri lager den 5\enh{cm} høy?
\item Hva finner Siri ut dersom hun løser ligningen $ V(x)=500 $ ?
}

\eksop{1PV22D1}{1PV22D1opg6}
Et rektangel er tre ganger så langt som det er bredt. Arealet av rektangelet er 432\enh{cm}$ ^2 $ .\os
Hvor bredt er rektangelet?

\newpage
\op{ligpunkt}
I denne oppgaven kan du anta at punkt på grafen som \textit{ser} ut til å ha heltalls koordinater har det.
\fig{ligo1}

\abc{
	\item Finn koordinatene til toppunktet til $ f(x) $.
	\item Finn koordinatene til minst ett av skjæringspunktene til $ f(x) $ og $ g(x) $.
	\item Finn nullpunktene til $ g(x) $.
}

\op{ligseto1}
Løs ligningssettet
\algv{
3b-2a&=15 \vn
5a-b&=8
}
\op{ligseto2}
Løs ligningssettet
\algv{
8x-3y&=4x-3 \\
x+8y&=y-2x
}
\newpage
\eksop{GEV22}{GEV22opg1}
To sjokolader og én vannflaske koster 40 kr.\\
Fire sjokolader og tre vannflasker koster 98 kr.

\newpage
\grubop{opgligprored}
Orinigalprisen på en vare er først senket med 20\%, og så er den nye prisen senket med 50\%. Etter dette koster varen 400\enh{kr}. Hva kostet varen opprinnelig?

\grubop{opgligmedian}
La $ a $ og $ b $ være de to midterste verdiene i et datasett med partalls antall verdier. Vis at metoden for å finne medianen slik den er beskrevet i \rref{median} er likegyldig med metoden som er beskrevet i \refops{opgstat1}.


\end{document}

