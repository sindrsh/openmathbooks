\documentclass[english,hidelinks,pdftex, 11 pt, class=report,crop=false]{standalone}
\usepackage[T1]{fontenc}
\usepackage[utf8]{luainputenc}
\usepackage{lmodern} % load a font with all the characters
\usepackage{geometry}
\geometry{verbose,a4paper, inner=0cm, outer=0 cm, bmargin=2cm, tmargin=1cm}
%\textwidth=12cm
\setlength{\parindent}{0bp}
\usepackage{import}
\usepackage[subpreambles=false]{standalone}
\usepackage{amsmath}
\usepackage{amssymb}
\usepackage{esint}
\usepackage{babel}
\usepackage{tabu}
\usepackage[dvipsnames, table]{xcolor}
\usepackage{cancel}
\makeatother
\makeatletter
\usepackage{datetime2}
\usepackage{titlesec}
\usepackage[many]{tcolorbox}

% Eheter
\newcommand{\enh}[1]{\,\textrm{#1}}
%referances
\newcommand{\net}[2]{{\color{blue}\href{#1}{#2}}}

%Spaces
\newcommand{\vsk}{\\[12pt]}
\newcommand{\vs}{\vspace{-12pt}}

% Tabell for opplegg

\newcommand{\ovlist}[1]{
\vspace{-16pt}
\begin{itemize}
	#1
\end{itemize}
}

% Chapters and sections
\titleformat{\section}[block]{\bfseries}{\hspace{3cm}\thesection}{5pt}{}
\titleformat{\subsection}[block]{\bfseries}{\hspace{3cm}\thesection}{5pt}{}
\newcommand{\sectionbreak}{\clearpage} % New page on each section
 

\newlength{\mywidth}
\setlength{\mywidth}{14cm}

\newcommand{\cont}[1]{
\begin{tcolorbox}[center, boxrule=0.0 mm, width=\mywidth,arc=0mm,enhanced jigsaw,,colback=white,breakable]
#1	
\end{tcolorbox}
}

\newcommand{\info}[5]{
\begin{tcolorbox}[center, boxrule=0.1 mm, width=\mywidth,arc=0mm,enhanced jigsaw,breakable,colback=yellow!5]	
	
	\footnotesize
	\textbf{Øvingsområde}\\[5pt] #1 
	
	\textbf{Utstyr}\\ #2  \\
	
	\begin{tabular}{@{} p{4cm} p{4cm} l} 
		\textbf{Tid} & \textbf{Elevinndeling} & \textbf{Læringsarena} \\
		#3  & #4 & #5
	\end{tabular} 
\end{tcolorbox}	
}

\newcommand{\gjen}[1]{\begin{tcolorbox}[center,boxrule=0.1 mm, width=\mywidth,arc=0mm,colback=blue!3] {\large \textbf{Gjennomføring} \vspace{5 pt}}\newline #1  \end{tcolorbox}\vspace{-5pt}}
\newcommand{\eks}[1]{\begin{tcolorbox}[center,boxrule=0.1 mm, width=\mywidth,arc=0mm,colback=green!3] {\large \textbf{Eksempel} \vspace{5 pt}}\newline #1  \end{tcolorbox}\vspace{-5pt}}

\newcounter{opl}
%\numberwithin{opl}{article}


\newcommand{\opl}[1]{
\newpage
{\refstepcounter{opl} %\phantomsection 
\large \textbf{\theopl \;#1} \vsk}
}

% Headlines
\newcommand{\fork}{\textbf{Forkunnskapar}\\}
\newcommand{\forb}{\textbf{Forberedelsar}\\}
\newcommand{\opgvr}{\textbf{Oppgaver}}



%colors
\newcommand{\colr}[1]{{\color{red} #1}}
\newcommand{\colb}[1]{{\color{blue} #1}}
\newcommand{\colo}[1]{{\color{orange} #1}}
\newcommand{\colc}[1]{{\color{cyan} #1}}
\definecolor{projectgreen}{cmyk}{100,0,100,0}
\newcommand{\colg}[1]{{\color{projectgreen} #1}}

% Lister med bokstavar
\usepackage[inline]{enumitem}
% Opg
\newcommand{\abc}[1]{
	\begin{enumerate}[label=\alph*),leftmargin=18pt]
		#1
	\end{enumerate}
}

\usepackage[]{hyperref}
\newcommand{\nxt}{\\ \includegraphics[]{lina} \\[5 pt]}

\begin{document}
{\Large \textbf{Forkunnskaper til oppgavene}} \\[5pt]
\setcounter{chapter}{1}
\reg[Arealet av rektangler \label{arfir}]{
Arealet $ A $ av et rektangel med bredde $ b $ og høyde $ h $ er:
\[A= b\cdot h \]
\fig{tri12}
}
\reg[Arealet av rette trekanter]{
Arealet $ A $ av rette trekanter er:
\[ A = \frac{g\cdot h}{2} \]
\fig{tri16}
}
\reg[Omkretsen av en sirkel]{Omkretsen $ O $ av en sirkel med radius $ r $ er:
	\[ O = 2\pi r \]
	\fig{tri22}
}\regv
\newpage
{\Large \textbf{Oppgave 1}} \\[5pt]
Den blå trekanten i de to figurene under er den samme:
\fig{tri18}
\fig{tri18a}
\textbf{a)} Finn arealet av rektangelet som består av den oransje, den blå og den grønne trekanten.\os
\textbf{b)} Finn arealet av den oransje trekanten.\os
\textbf{c)} Finn arealet av den grønne trekanten.\os
\textbf{d)} Bruk arealene fra opg a)-c) til å finne arealet av den blå trekanten.\\[20pt]

{\Large \textbf{Oppgave 2}} \\[5pt]
Den blå trekanten i de to figurene under er den samme:
\begin{center}
	\includegraphics[]{\asym{tri20}}\quad\qquad
	\includegraphics[]{\asym{tri20a}}
\end{center}
\textbf{a)} Finn arealet av rektangelet som består av den oransje, den blå og den grønne trekanten.\os
\textbf{b)} Finn arealet av den oransje trekanten.\os
\textbf{c)} Finn arealet av den grønne trekanten.\os
\textbf{d)} Bruk arealene fra opg a)-c) til å finne arealet av den blå trekanten.\vsk
\newpage
{\Large \textbf{Oppgave 3}} \\[5pt]
\textbf{a)} Bruk det du fant i oppgave 1d og 2d til å skrive formelen:
\reg[Arealet av en trekant \label{artri}]{
Arealet $ A $ av en trekant med grunnlinje $ g $ og høyde $ h $ er:
\[ A = \;?\]
	\begin{figure}
		\centering
		\includegraphics[scale=0.8]{\asym{tri16}}
		\quad
		\includegraphics[scale=0.8]{\asym{tri16a}}
		\quad
		\includegraphics[scale=0.8]{\asym{tri16b}}
	\end{figure}
}\vsk
\textbf{b)} Bruk \hr{artri} til å finne formlene: (\textsl{Hint:} Lag egne tegninger av trekanter du ''ser'' inni firkantene):
\reg[Arealet av et parallellogram]{
Arealet $ A $ av et parallellogram med bredde $ b $ og høyde $ h $ er:
\[ A = \;? \]
\fig{tri21}
}
\reg[Arealet av et trapes]{
	Arealet $ A $ av et parallellogram med bredde $ b $ og høyde $ h $ er:
	\[ A = \;? \]
\fig{tri21a}
}\newpage
{\Large \textbf{Oppgave 4}} \\[5pt]
I figuren nedenfor har vi delt opp en sirkel i 4, 10 og 50 (like store) biter, og lagt disse bitene etter hverandre.
\fig{tri19}
\fig{tri19a}
\fig{tri19b}
\textbf{a)} Prøv å beskriv hva som skjer når man deler sirkelen inn i mindre og mindre biter.\os
\textbf{b)} Hvor lang er alle buebitene til sammen? Hva er utykket for denne lengden hvis radiusen til sirkelen er $ r $?\vsk

Se på figuren med de 50 bitene lagt etter hverandre.\os
\textbf{c)}  Ca. hvor brei er denne figuren? Ca. hvor høy er denne figuren?\os

\textbf{d)} Finn tilnærmingen til arealet av denne figuren.\os

\textbf{e)} Skriv inn formelen:
\reg[Arealet av en sirkel]{
Arealet $ A $ av en sirkel med radius $ r $ er:
\[ A =\;?\]
}\vsk

\begin{comment}
	{\Large \textbf{Oppgave 5}} \\[5pt]
	\begin{figure}
	\includegraphics[scale=0.5]{/home/sindre/Pictures/sin1}
	\end{figure}
	{\Large \textbf{Oppgave 6}} \\[5pt]
	\begin{figure}
	\includegraphics[scale=0.4]{/home/sindre/Pictures/sin2}
	\end{figure}
	\newpage
	{\Large \textbf{Oppgave 7}} \\[5pt]
	\begin{figure}
	\includegraphics[scale=0.4]{/home/sindre/Pictures/sin3}
	\end{figure}
\end{comment}
\end{document}


