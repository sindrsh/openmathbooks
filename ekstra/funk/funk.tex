\documentclass[english,hidelinks,pdftex, 11 pt, class=report,crop=false]{standalone}
\usepackage[T1]{fontenc}
\usepackage[utf8]{luainputenc}
\usepackage{lmodern} % load a font with all the characters
\usepackage{geometry}
\geometry{verbose,a4paper, inner=0cm, outer=0 cm, bmargin=2cm, tmargin=1cm}
%\textwidth=12cm
\setlength{\parindent}{0bp}
\usepackage{import}
\usepackage[subpreambles=false]{standalone}
\usepackage{amsmath}
\usepackage{amssymb}
\usepackage{esint}
\usepackage{babel}
\usepackage{tabu}
\usepackage[dvipsnames, table]{xcolor}
\usepackage{cancel}
\makeatother
\makeatletter
\usepackage{datetime2}
\usepackage{titlesec}
\usepackage[many]{tcolorbox}

% Eheter
\newcommand{\enh}[1]{\,\textrm{#1}}
%referances
\newcommand{\net}[2]{{\color{blue}\href{#1}{#2}}}

%Spaces
\newcommand{\vsk}{\\[12pt]}
\newcommand{\vs}{\vspace{-12pt}}

% Tabell for opplegg

\newcommand{\ovlist}[1]{
\vspace{-16pt}
\begin{itemize}
	#1
\end{itemize}
}

% Chapters and sections
\titleformat{\section}[block]{\bfseries}{\hspace{3cm}\thesection}{5pt}{}
\titleformat{\subsection}[block]{\bfseries}{\hspace{3cm}\thesection}{5pt}{}
\newcommand{\sectionbreak}{\clearpage} % New page on each section
 

\newlength{\mywidth}
\setlength{\mywidth}{14cm}

\newcommand{\cont}[1]{
\begin{tcolorbox}[center, boxrule=0.0 mm, width=\mywidth,arc=0mm,enhanced jigsaw,,colback=white,breakable]
#1	
\end{tcolorbox}
}

\newcommand{\info}[5]{
\begin{tcolorbox}[center, boxrule=0.1 mm, width=\mywidth,arc=0mm,enhanced jigsaw,breakable,colback=yellow!5]	
	
	\footnotesize
	\textbf{Øvingsområde}\\[5pt] #1 
	
	\textbf{Utstyr}\\ #2  \\
	
	\begin{tabular}{@{} p{4cm} p{4cm} l} 
		\textbf{Tid} & \textbf{Elevinndeling} & \textbf{Læringsarena} \\
		#3  & #4 & #5
	\end{tabular} 
\end{tcolorbox}	
}

\newcommand{\gjen}[1]{\begin{tcolorbox}[center,boxrule=0.1 mm, width=\mywidth,arc=0mm,colback=blue!3] {\large \textbf{Gjennomføring} \vspace{5 pt}}\newline #1  \end{tcolorbox}\vspace{-5pt}}
\newcommand{\eks}[1]{\begin{tcolorbox}[center,boxrule=0.1 mm, width=\mywidth,arc=0mm,colback=green!3] {\large \textbf{Eksempel} \vspace{5 pt}}\newline #1  \end{tcolorbox}\vspace{-5pt}}

\newcounter{opl}
%\numberwithin{opl}{article}


\newcommand{\opl}[1]{
\newpage
{\refstepcounter{opl} %\phantomsection 
\large \textbf{\theopl \;#1} \vsk}
}

% Headlines
\newcommand{\fork}{\textbf{Forkunnskapar}\\}
\newcommand{\forb}{\textbf{Forberedelsar}\\}
\newcommand{\opgvr}{\textbf{Oppgaver}}



%colors
\newcommand{\colr}[1]{{\color{red} #1}}
\newcommand{\colb}[1]{{\color{blue} #1}}
\newcommand{\colo}[1]{{\color{orange} #1}}
\newcommand{\colc}[1]{{\color{cyan} #1}}
\definecolor{projectgreen}{cmyk}{100,0,100,0}
\newcommand{\colg}[1]{{\color{projectgreen} #1}}

% Lister med bokstavar
\usepackage[inline]{enumitem}
% Opg
\newcommand{\abc}[1]{
	\begin{enumerate}[label=\alph*),leftmargin=18pt]
		#1
	\end{enumerate}
}

\usepackage[]{hyperref}
%%


\newcommand{\regu}[2][]{\begin{tcolorbox}[boxrule=0.3 mm,arc=0mm,colback=blue!3] {\large \textbf{#1} \vspace{5 pt}}\newline #2  \end{tcolorbox}\vspace{-5pt}}

\begin{document}
\fork{L'hopital 2}{
	Vi har at
	\alg{
		\lim\limits_{x\to a} \frac{g}{f}&=\lim\limits_{x\to a} \frac{\frac{1}{f}}{\frac{1}{g}} 
	}
	Da $ {\lim\limits_{x\to a} f=\lim\limits_{x\to a} g=\infty} $, må $ {\lim\limits_{x\to a} \frac{1}{f}=\lim\limits_{x\to a}\frac{1}{g}=0} $. Av Lhopital1?? har vi da at 
	\alg{
		\lim\limits_{x\to a} \frac{g}{f}&=\lim\limits_{x\to a} \frac{\frac{1}{f^2}f'}{\frac{1}{g^2}g'} 
	}
	Multipliserer vi begge sider med $ \lim\limits_{x\to a} \frac{f^2}{g^2} $, får vi at
	\alg{
		\lim\limits_{x\to a} \frac{f}{g}=\lim\limits_{x\to a}\frac{f'}{g'}
	}	
}

\eks{
	Gitt funksjonen
	\[ f(x)=\sin x\quad,\quad x\in[-2, 4] \]	
	\textbf{a)} Finn infleksjonspunktene til $ f $.
	
	\textbf{b)} Finn vendepunktene til $ f $. \\
	
	\sv
	\textbf{a)} Infleksjonspunktene finner vi der hvor $ f''(x)=0 $:
	\alg{f''(x) &= 0\\
		(\sin x)'' &= 0 \\
		-\sin x &= 0 	}
	Av $ {x\in D_f} $ er det $ {x=0} $ og $ {x=\pi }$ som oppfyller kravet fra ligningen over. For å finne ut om $ f'' $ skifter fortegn i disse punktene, setter vi opp et fortegnsskjema:
	\begin{figure}[H]
		\centering
		\begin{tikzpicture}[scale=2]	
			\draw[color=black] (0,-0.25) -- (0,1.25);
			\node[anchor=south] at (0,1.25) { $-2$};  
			\draw[color=black] (2,-0.25) -- (2,1.25);
			\node[anchor=south] at (2,1.25) { $\pi$};	
			\draw[color=black] (3,-0.25) -- (3,1.25);
			\node[anchor=south] at (3,1.25) { $4$};		
			\draw[dashed,color=black] (0,1) -- (3,1);
			\draw[dashed,color=black] (0,0.5) -- (1,0.5);
			\draw[dashed,color=black] (2,0.5) -- (3,0.5);	
			\draw[color=black] (1,0.5) -- (2,0.5);	
			\node[anchor=east] at (0,0.5) { $\sin x$};  		
			\node[anchor=east] at (0,1) { $-1$};  
			\draw[color=black] (1,-0.25) -- (1,1.25);
			\node[anchor=south] at (1,1.25) { $0$};    
			\draw[color=black] (0,0) -- (1,0);
			\draw[color=black] (2,0) -- (3,0);	    
			\draw[dashed,color=black] (1,0) -- (2,0);    
			\node[anchor=east] at (0,0) {$f''$};     
			\filldraw (1,0) circle[radius=1pt] ;   	
			\filldraw (2,0) circle[radius=1pt] ; 	
		\end{tikzpicture}
	\end{figure}
	
	$ f'' $ går altså fra positiv til negativ i $ {x=0} $ og fra negativ til positiv i $ {x=\pi} $. Dette betyr at $ f $ går fra konveks til konkav i $ {x=0} $ og fra konkav til konveks i $ {x=\pi }$.
}
\end{document}


