\documentclass[english,hidelinks,pdftex, 11 pt, class=report,crop=false]{standalone}
\usepackage[T1]{fontenc}
\usepackage[utf8]{luainputenc}
\usepackage{lmodern} % load a font with all the characters
\usepackage{geometry}
\geometry{verbose,a4paper, inner=0cm, outer=0 cm, bmargin=2cm, tmargin=1cm}
%\textwidth=12cm
\setlength{\parindent}{0bp}
\usepackage{import}
\usepackage[subpreambles=false]{standalone}
\usepackage{amsmath}
\usepackage{amssymb}
\usepackage{esint}
\usepackage{babel}
\usepackage{tabu}
\usepackage[dvipsnames, table]{xcolor}
\usepackage{cancel}
\makeatother
\makeatletter
\usepackage{datetime2}
\usepackage{titlesec}
\usepackage[many]{tcolorbox}

% Eheter
\newcommand{\enh}[1]{\,\textrm{#1}}
%referances
\newcommand{\net}[2]{{\color{blue}\href{#1}{#2}}}

%Spaces
\newcommand{\vsk}{\\[12pt]}
\newcommand{\vs}{\vspace{-12pt}}

% Tabell for opplegg

\newcommand{\ovlist}[1]{
\vspace{-16pt}
\begin{itemize}
	#1
\end{itemize}
}

% Chapters and sections
\titleformat{\section}[block]{\bfseries}{\hspace{3cm}\thesection}{5pt}{}
\titleformat{\subsection}[block]{\bfseries}{\hspace{3cm}\thesection}{5pt}{}
\newcommand{\sectionbreak}{\clearpage} % New page on each section
 

\newlength{\mywidth}
\setlength{\mywidth}{14cm}

\newcommand{\cont}[1]{
\begin{tcolorbox}[center, boxrule=0.0 mm, width=\mywidth,arc=0mm,enhanced jigsaw,,colback=white,breakable]
#1	
\end{tcolorbox}
}

\newcommand{\info}[5]{
\begin{tcolorbox}[center, boxrule=0.1 mm, width=\mywidth,arc=0mm,enhanced jigsaw,breakable,colback=yellow!5]	
	
	\footnotesize
	\textbf{Øvingsområde}\\[5pt] #1 
	
	\textbf{Utstyr}\\ #2  \\
	
	\begin{tabular}{@{} p{4cm} p{4cm} l} 
		\textbf{Tid} & \textbf{Elevinndeling} & \textbf{Læringsarena} \\
		#3  & #4 & #5
	\end{tabular} 
\end{tcolorbox}	
}

\newcommand{\gjen}[1]{\begin{tcolorbox}[center,boxrule=0.1 mm, width=\mywidth,arc=0mm,colback=blue!3] {\large \textbf{Gjennomføring} \vspace{5 pt}}\newline #1  \end{tcolorbox}\vspace{-5pt}}
\newcommand{\eks}[1]{\begin{tcolorbox}[center,boxrule=0.1 mm, width=\mywidth,arc=0mm,colback=green!3] {\large \textbf{Eksempel} \vspace{5 pt}}\newline #1  \end{tcolorbox}\vspace{-5pt}}

\newcounter{opl}
%\numberwithin{opl}{article}


\newcommand{\opl}[1]{
\newpage
{\refstepcounter{opl} %\phantomsection 
\large \textbf{\theopl \;#1} \vsk}
}

% Headlines
\newcommand{\fork}{\textbf{Forkunnskapar}\\}
\newcommand{\forb}{\textbf{Forberedelsar}\\}
\newcommand{\opgvr}{\textbf{Oppgaver}}



%colors
\newcommand{\colr}[1]{{\color{red} #1}}
\newcommand{\colb}[1]{{\color{blue} #1}}
\newcommand{\colo}[1]{{\color{orange} #1}}
\newcommand{\colc}[1]{{\color{cyan} #1}}
\definecolor{projectgreen}{cmyk}{100,0,100,0}
\newcommand{\colg}[1]{{\color{projectgreen} #1}}

% Lister med bokstavar
\usepackage[inline]{enumitem}
% Opg
\newcommand{\abc}[1]{
	\begin{enumerate}[label=\alph*),leftmargin=18pt]
		#1
	\end{enumerate}
}

\usepackage[]{hyperref}
%%


\newcommand{\regu}[2][]{\begin{tcolorbox}[boxrule=0.3 mm,arc=0mm,colback=blue!3] {\large \textbf{#1} \vspace{5 pt}}\newline #2  \end{tcolorbox}\vspace{-5pt}}

\begin{document}
	\pagenumbering{roman}
\regu[Å sette prøve]{Når vi erstattar $ x $ med eit tal i ei likning, \textit{set me prøve } på likninga. \vsk

Viss vi får lik verdi på begge sider av likninga, \textbf{er} talet løysinga til likninga. Viss vi får ulike verdiar, \textbf{er ikkje} talet løysinga.
}
\eks[1]{ \vs
\abc{
\item Undersøk om $ {x=3} $ er løysinga til likninga $ \colb{x}+4= 7 $.
\item Undersøk om $ {x=5} $ er løysinga til likninga $ 2\colb{x} = 20 $.
}
\sv
\abc{
	\item I likninga
	\[ \colb{x}+4=7 \]
	erstattar vi $ \colb{x} $ med \colb{3}:
	\alg{
		\colb{3}+4 &= 7 \\
		7 &= 7
	}
Altså \textbf{er} $ {x=3} $ løysinga til likninga.
\item I likninga (hugs at $ 2\colb{x}=2\cdot\colb{x} $)
\[ 2\cdot\colb{x}=20 \]
erstattar vi $ \colb{x} $ med \colb{5}:
\alg{
2\cdot \colb{5}&=20 \\
10 &= 20
}
Altså \textbf{er ikkje} $ x=5 $ løysinga til likninga. 
} 
}
\newpage
\eks[2]{
\abc{
\item Løys likninga $ 5x - 7 = 4x + 2  $.
\item Set prøve på svaret ditt.
}

\sv

\abc{ 
\item \phantom{} \vsb 
\begin{align} 
	5x-4x &= 2+7 \\
	x &= 9
\end{align}
\item  I den originale likninga har vi at
\nn{
5\cdot{\colb{x}} - 7 = 4\cdot\colb{x} + 2 
}
Når vi erstattar  $ \colb{x} $ med $ \colb{9} $, får vi
\alg{
5\cdot{\colb{9}} - 7 &= 4\cdot\colb{9} + 2 \\
45-7 &= 36+2 \\
38 &= 38
}
Verdien til venstre er lik den til høgre, og da er vi sikre på at $ x=9 $ er løysinga til likninga.
}
}
\newpage
\eks[3]{
\abc{
\item Løys likninga $ 10x-6 = 8x+4 $.
\item Set prøve på svaret ditt.
}
\sv
\abc{
\item \phantom{} \vs
\alg{
10x-8x &= 4+6 \\
\frac{\cancel{2}x}{2} &= \frac{10}{2} \\
x &= 5
}
\item Vi set $ x=5 $ inn i den originale likninga:
\alg{
 10\cdot5-6 &= 8\cdot5+4 \\
 44 &= 44
}
Altså er vi sikre på at $ x=5 $ er løysinga til likninga.
}
}
\newpage
\eks[4]{
\abc{
\item Løys likninga $ 2x+41 = 9x-1 $.
\item Set prøve på svaret ditt.
}
\sv
\abc{
\item (Merk at vi samlar $ x $-ane på høgre side fordi $ 9 $ er større enn $ 2 $).
\alg{
41+1 &= 9x-2x \\
\frac{42}{7} &= \frac{\cancel{7}x}{\cancel{7}}\\
6 &= x 
}
\item Vi set $ x=6 $ inn i den originale likninga:
\alg{
2\cdot6 +41 &= 9\cdot6 -1 \\
53 &= 53
}
Altså er vi sikre på at $ x=6 $ er løysinga til likninga.
}
}

\eks[1]{
	Vi har sett at sammenhengen mellom strekningen $ s $ vi har kjørt, den konstante farten $ f $ vi har holdt, og tiden $ t $ vi har brukt er gitt ved formelen\footnote{$ \text{strekning}=\text{fart}\cdot \text{tid} $}:
	\[ s = f\cdot t \] 
	Dette er altså en formel for $ s $. Ønsker vi i stedet en formel for $ f $, kan vi gjøre om formelen ved å følge prinsippene for likninger\footnote{Se \mb, s. 121.}:
	\alg{
		s &= f\cdot t \br
		\frac{s}{t}&=\frac{f\cdot \bcancel{t}}{\bcancel{t}} \br
		\frac{s}{t}&=f
	}
}
\eks[4]{
	''Broren min er dobbelt så gammel som meg. Til sammen er vi 9 år gamle. Hvor gammel er jeg?''.
	
	\sv
	''Broren min er dobbelt så gammel som meg.'' betyr at
	\[ \text{brors alder}=2\cdot \text{min alder} \]
	''Til sammen er vi 9 år gamle.'' betyr at
	\[ \text{brors alder}+\text{min alder}=\text{9} \]
	Erstatter vi 'brors alder' med ''$2\cdot\text{min alder} $'', får vi
	\[ 2\cdot\text{min alder}+\text{min alder}=\text{9} \]
	Altså er
	\algv{
		3\cdot \text{min alder} &= 9 \\
		\frac{\cancel{3}\cdot \text{min alder}}{\cancel{3}}&= \frac{9}{3} \\
		\text{min alder} &= 3
	}
	''Jeg'' er altså 3 år gammel.
}

\end{document}


