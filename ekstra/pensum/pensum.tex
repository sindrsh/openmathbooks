\documentclass{article}
\usepackage[a4paper]{geometry}

\begin{document}
\section*{Kompetansemål}
I tabellene under vises de trinnvise kompetansemålene definert av Udir, og hvilke kapitler fra bøkene i OpenMathBooks\,-\,serien som dekker disse. Merk at begge ''Anvendt''-bøkene har et vedlegg som gir lenker til relevante tekster. I kompetansemål som nevner innhenting av informasjon fra tekster er det underforstått at disse vedleggene er med på å dekke kompetansemålet. I alle kompetansemål hvor anvendelse/praktiske situasjoner er det også underforstått at kapittel 8 og kapittel 4 i henholdsvis \textsl{Anvendt matematikk 1} og \textsl{2} er med på å dekke kompetansemålet.
\newpage
\begin{center}
	\begin{tabular}{p{10.5cm} | c | c |} 
	\textbf{Kompetansemål 5. trinn} & \textbf{MB} & \textbf{AM}\\ \hline
\shortstack[l]{\\utforske og forklare sammenhenger mellom brøker,\\ desimaltall og prosent og bruke det i hoderegning} &\shortstack{1 \\4} &\shortstack{3\\{}} \\ \hline
	
\shortstack[l]{\\beskrive brøk som del av en hel, som del av en mengde\\ og som tall på tallinjen og vurdere og navngi størrelsene} &\shortstack{1\\4} &\shortstack{3\\{}} \\ \hline

\shortstack[l]{\\representere brøker på ulike måter og oversette\\ mellom de ulike representasjonene
} &\shortstack{1 \\4} &\shortstack{3\\{}} \\ \hline

\shortstack[l]{\\utvikle og bruke ulike strategier for regning med positive \\tall og brøk og forklare tenkemåtene sine \\{}
} &\shortstack{1\\4\\6} &\shortstack{3\\{}} \\ \hline	

\shortstack[l]{\\formulere og løse problemer fra egen hverdag\\ som har med brøk å gjøre
} &\shortstack{} &\shortstack{3\\{}} \\ \hline

\shortstack[l]{\\diskutere tilfeldighet og sannsynlighet i spill og \\praktiske situasjoner og knytte det til brøk
} &\shortstack{} &\shortstack{5} \\ \hline

\shortstack[l]{\\løse ligninger og ulikheter gjennom logiske resonnementer og\\ forklare hva det vil si at et tall er en løsning på en ligning
} &\shortstack{9} &\shortstack{6} \\ \hline

\shortstack[l]{\\lage og løse oppgaver i regneark som \\omhandler personlig økonomi
} &\shortstack{} &\shortstack{4\\7} \\ \hline

\shortstack[l]{\\formulere og løse problemer fra egen hverdag\\ som har med tid å gjøre
} &\shortstack{6} &\shortstack{8} \\ \hline

\shortstack[l]{\\lage og programmere algoritmer med bruk av \\variabler, vilkår og løkker
} &\shortstack{} &\shortstack{7} \\ \hline
\end{tabular}
\end{center}\vspace{20pt}

\begin{center}
	\begin{tabular}{p{10.5cm} | c | c} 
		\textbf{Kompetansemål 6. trinn} & \textbf{MB} & \textbf{AM}\\ \hline
		\shortstack[l]{\\utforske, navngi og plassere desimaltall \\på tallinjen
		} &\shortstack{1} &\shortstack{} \\ \hline
	
	\shortstack[l]{\\utforske strategier for regning med desimaltall og \\sammenligne med regnestrategier for hele tall
	} &\shortstack{6} &\shortstack{} \\ \hline

\shortstack[l]{\\formulere og løse problemer fra sin egen hverdag som har med \\desimaltall, brøk og prosent å gjøre, og forklare egne\\ tenkemåter
} &\shortstack{} &\shortstack{1\\3\\8} \\ \hline

\shortstack[l]{\\beskrive egenskaper ved og minimumsdefinisjoner av to- og \\tredimensjonale figurer og forklare hvilke egenskaper figurene \\har felles, og hvilke egenskaper som skiller dem fra hverandre
} &\shortstack{7} &\shortstack{} \\ \hline

\shortstack[l]{\\utforske og beskrive symmetri i mønstre og utføre \\kongruensavbildninger med og uten koordinatsystem
} &\shortstack{1\\7} &\shortstack{} \\ \hline

\shortstack[l]{\\måle radius, diameter og omkrets i sirkler og utforske\\ og argumentere for sammenhengen
} &\shortstack{10} &\shortstack{} \\ \hline

\shortstack[l]{\\utforske mål for areal og volum i praktiske situasjoner \\og representere dem på ulike måter
} &\shortstack{10} &\shortstack{1} \\ \hline

\shortstack[l]{\\bruke ulike strategier for å regne ut areal og omkrets \\og utforske sammenhenger mellom disse
} &\shortstack{7} &\shortstack{3} \\ \hline

\shortstack[l]{\\bruke variabler og formler til å uttrykke sammenhenger\\ i praktiske situasjoner
} &\shortstack{7} &\shortstack{alle} \\ \hline
	\end{tabular}
\end{center}


\begin{center}
	\begin{tabular}{p{10.5cm} | c | c |} 
		\textbf{Kompetansemål 7. trinn} & \textbf{MB} & \textbf{AM}\\ \hline
		\shortstack[l]{\\utvikle og bruke hensiktsmessige strategier i regning med brøk,\\ desimaltall og prosent og forklare tenkemåtene sine
		} &\shortstack{1 \\4} &\shortstack{3\\{}} \\ \hline
	
	\shortstack[l]{\\representere og bruke brøk, desimaltall og prosent på ulike \\måter og utforske de matematiske sammenhengene mellom \\disse representasjonsformene
	} &\shortstack{1 \\4\\{}} &\shortstack{3\\{}} \\ \hline

	\shortstack[l]{\\utforske negative tall i praktiske situasjoner \\ \phantom{text}
} &\shortstack{} &\shortstack{4 \\ 8} \\ \hline

	\shortstack[l]{\\bruke tallinje i regning med positive og negative tall
} &\shortstack{5} &\shortstack{} \\ \hline

	\shortstack[l]{\\bruke sammensatte regneuttrykk til å beskrive\\ og utføre utregninger
} &\shortstack{3 \\ 6} &\shortstack{} \\ \hline

	\shortstack[l]{\\bruke ulike strategier for å løse lineære ligninger og ulikheter \\og vurdere om løsninger er gyldige
} &\shortstack{9} &\shortstack{6} \\ \hline

	\shortstack[l]{\\ utforske og bruke hensiktsmessige sentralmål i egne og andres\\ statistiske undersøkelser
} &\shortstack{} &\shortstack{2} \\ \hline

	\shortstack[l]{\\ logge, sortere, presentere og lese data i tabeller og diagrammer\\ og begrunne valget av framstilling
} &\shortstack{} &\shortstack{2 \\ 7} \\ \hline

	\shortstack[l]{\\ lage og vurdere budsjett og regnskap ved å bruke regneark \\med cellereferanser og formler
} &\shortstack{} &\shortstack{4\\6} \\ \hline

\shortstack[l]{\\ bruke programmering til å utforske data i tabeller og datasett
} &\shortstack{} &\shortstack{7} \\ \hline	
	\end{tabular}	
\end{center}\vspace{20pt}

\begin{center}
	\begin{tabular}{p{10.5cm} | c | c |} 
		\textbf{Kompetansemål 8. trinn} & \textbf{MB} & \textbf{AM}\\ \hline
		\shortstack[l]{\\ bruke potenser og kvadratrøtter i utforsking og problemløsing\\ og argumentere for framgangsmåter og resultater
		} &\shortstack{7\\ 11} &\shortstack{3 \\ 4} \\ \hline
		
		\shortstack[l]{\\utvikle og kommunisere strategiar for hovudrekning \\i utrekningar \\{}
		} &\shortstack{1 \\ 4\\6} &\shortstack{} \\ \hline
		
		\shortstack[l]{\\ utforske og beskrive primtalsfaktorisering \\og bruke det i brøkrekning
		} &\shortstack{3\\4} &\shortstack{} \\ \hline
		
		\shortstack[l]{\\ utforske algebraiske reknereglar
		} &\shortstack{7} &\shortstack{} \\ \hline
		
		\shortstack[l]{\\ beskrive og generalisere mønster med eigne ord og algebraisk \\ \phantom{a}
		} &\shortstack{8\\10} &\shortstack{} \\ \hline
		
		\shortstack[l]{\\lage og løyse problem som omhandlar samansette måleiningar
		} &\shortstack{} &\shortstack{1} \\ \hline
		
		\shortstack[l]{\\ lage og forklare rekneuttrykk med tal, variablar og konstantar \\knytte til praktiske situasjonar
		} &\shortstack{} &\shortstack{alle} \\ \hline
		
		\shortstack[l]{\\ lage, løyse og forklare likningar knytte til praktiske situasjonar \\\phantom{text}
		} &\shortstack{} &\shortstack{6} \\ \hline
		
		\shortstack[l]{\\ utforske, forklare og samanlikne funksjonar knytte til\\ praktiske situasjonar
		} &\shortstack{} &\shortstack{6} \\ \hline
		
		\shortstack[l]{\\ representere funksjonar på ulike måtar og vise samanhengar\\ mellom representasjonane
		} &\shortstack{10} & \\ \hline	

		\shortstack[l]{\\ utforske korleis algoritmar kan skapast, testast\\ og forbetrast ved hjelp av programmering
} &\shortstack{} &\shortstack{7\\{}} \\ \hline		
	\end{tabular}	
\end{center} \vspace{20pt}


\begin{center}
	\begin{tabular}{p{10.5cm} | c | c |} 
		\textbf{Kompetansemål 9. trinn} & \textbf{MB} & \textbf{AM}\\ \hline
		\shortstack[l]{\\ beskrive, forklare og presentere strukturer og utviklinger\\ i geometriske mønstre og i tallmønstre
		} &\shortstack{10\\{}} &\shortstack{} \\ \hline
		
		\shortstack[l]{\\ utforske egenskapene ved ulike polygoner og forklare\\ begrepene formlikhet og kongruens
		} &\shortstack{10\\{}} &\shortstack{} \\ \hline
		
		\shortstack[l]{\\ utforske, beskrive og argumentere for sammenhenger mellom\\ sidelengdene i trekanter
		} &\shortstack{10\\{}} &\shortstack{} \\ \hline
		
		\shortstack[l]{\\ utforske og argumentere for hvordan det å endre forutsetninger \\i geometriske problemstillinger påvirker løsninger
		} &\shortstack{10} &\shortstack{} \\ \hline
		
		\shortstack[l]{\\ utforske og argumentere for formler for areal og volum av\\ tredimensjonale figurer
		} &\shortstack{7\\10} &\shortstack{} \\ \hline
		
		\shortstack[l]{\\ tolke og kritisk vurdere statistiske \\framstillinger fra mediene og lokalsamfunnet
		} &\shortstack{} &\shortstack{2} \\ \hline
		
		\shortstack[l]{\\ finne og diskutere sentralmål og spredningsmål i reelle datasett \\\phantom{a}
		} &\shortstack{} &\shortstack{2\\7} \\ \hline
		
		\shortstack[l]{\\ utforske og argumentere for hvordan framstillinger av tall og \\data kan brukes for å fremme ulike synspunkter
		} &\shortstack{} &\shortstack{2\\{}} \\ \hline
		
		\shortstack[l]{\\ beregne og vurdere sannsynlighet i statistikk og spill
		} &\shortstack{} &\shortstack{5} \\ \hline
		
		\shortstack[l]{\\ simulere utfall i tilfeldige forsøk og beregne sannsynligheten \\for at noe skal inntreffe, ved å bruke programmering
		} &\shortstack{} &\shortstack{7\\{}} \\ \hline	
	\end{tabular}	
\end{center} \vspace{20pt}

\begin{center}
	\begin{tabular}{p{10.5cm} | c | c |} 
		\textbf{Kompetansemål 10. trinn} & \textbf{MB} & \textbf{AM}\\ \hline
		\shortstack[l]{\\ utforske og generalisere multiplikasjon av\\ polynomer algebraisk og geometrisk
		} &\shortstack{3} &\shortstack{} \\ \hline
		
		\shortstack[l]{\\ utforske og sammenligne egenskaper ved ulike funksjoner \\ved å bruke digitale verktøy
		} &\shortstack{} &\shortstack{7} \\ \hline
		
		\shortstack[l]{\\ lage, løse og forklare ligningssett knyttet til praktiske \\situasjoner
		} &\shortstack{9} &\shortstack{6} \\ \hline
		
		\shortstack[l]{\\ utforske sammenhengen mellom konstant prosentvis endring,\\ vekstfaktor og eksponentialfunksjoner
		} &\shortstack{} &\shortstack{3} \\ \hline
		
		\shortstack[l]{\\ hente ut og tolke relevant informasjon fra tekster om\\ kjøp og salg og ulike typer lån og bruke det til å formulere \\og løse problemer
		} &\shortstack{} &\shortstack{8} \\ \hline
		
		\shortstack[l]{\\ planlegge, utføre og presentere et utforskende arbeid\\ knyttet til personlig økonomi
		} &\shortstack{} &\shortstack{4 \\ 7} \\ \hline
		
		\shortstack[l]{\\ bruke funksjoner i modellering og argumentere\\ for framgangsmåter og resultater
		} &\shortstack{10} &\shortstack{6\\7} \\ \hline
		
		\shortstack[l]{\\ modellere situasjoner knyttet til reelle datasett, presentere\\ resultatene og argumentere for at modellene er gyldige
		} &\shortstack{} &\shortstack{6\\7} \\ \hline
		
		\shortstack[l]{\\ utforske matematiske egenskaper og sammenhenger ved\\ å bruke programmering
		} &\shortstack{} &\shortstack{7} \\ \hline
		
	\end{tabular}	
\end{center}

\newpage

\begin{center}
	\begin{tabular}{p{10.5cm} | c | c |} 
		\textbf{Kompetansemål 1P} & \textbf{MB} & \textbf{AM}\\ \hline
		\shortstack[l]{\\ lese, hente ut og vurdere matematikk i tekster om situasjoner fra \\lokalmiljøet, gjøre beregninger knyttet til dette og presentere og\\ argumentere for resultatene
		} &\shortstack{} &\shortstack{} \\ \hline
		
		\shortstack[l]{\\ utforske hvordan ulike premisser vil kunne påvirke hvordan \\matematiske problemer fra samfunnsliv og arbeidsliv løses
		} &\shortstack{} &\shortstack{} \\ \hline
		
		\shortstack[l]{\\ modellere situasjoner knyttet til temaer fra samfunnsliv og \\arbeidsliv, presentere og argumentere for resultatene\\ og for når modellene er gyldige
		} &\shortstack{} &\shortstack{} \\ \hline
		
		\shortstack[l]{\\ identifisere variable størrelser i ulike situasjoner og bruke dem \\til utforsking og generalisering
		} &\shortstack{10} &\shortstack{alle} \\ \hline
		
		\shortstack[l]{\\ tolke og bruke formler som gjelder samfunnsliv og arbeidsliv
		} &\shortstack{} &\shortstack{alle} \\ \hline
		
		\shortstack[l]{\\ bruke prosent, prosentpoeng, promille og vekstfaktor i utregninger \\og presentere og begrunne løsninger
		} &\shortstack{} &\shortstack{3\\4} \\ \hline
		
		\shortstack[l]{\\ utforske, beskrive og bruke begrepene proporsjonalitet\\ og omvendt proporsjonalitet
		} &\shortstack{} &\shortstack{1\\{}} \\ \hline
		
		\shortstack[l]{\\ tolke og bruke sammensatte måleenheter i praktiske sammenhenger\\ og velge egnet måleenhet
		} &\shortstack{} &\shortstack{1} \\ \hline
		
		\shortstack[l]{\\ tolke og bruke funksjoner i matematisk modellering og problemløsing \\\phantom{text}
		} &\shortstack{} &\shortstack{6\\7} \\ \hline
	
		\shortstack[l]{\\planlegge, utføre og presentere selvstendig arbeid knyttet til \\modellering og funksjoner innenfor samfunnsfaglige temaer\\ \phantom{text}
		} &\shortstack{} &\shortstack{2\\6\\7} \\ \hline
		
		
		\shortstack[l]{\\bruke digitale verktøy i utforsking og problemløsing knyttet\\ til egenskaper ved funksjoner, og diskutere løsningene
		} &\shortstack{10\\{}} &\shortstack{7\\{}} \\ \hline
				
		\shortstack[l]{\\tolke og regne med rotuttrykk, potenser og tall på standardform \\ \phantom{a}
		} &\shortstack{8\\11} &\shortstack{} \\ \hline
	\end{tabular}	
\end{center} \vspace{20pt}


\begin{center}
	\begin{tabular}{p{10.5cm} | c | c |} 
		\textbf{Kompetansemål 1T} &	TM1 &AM2 \\ \hline
	\shortstack[l]{\\ tolke og bruke funksjoner i matematisk modellering og problemløsing 
	} &\shortstack{7} &\shortstack{3} \\ \hline


	\shortstack[l]{\\ lese og forstå matematiske bevis og utforske og utvikle bevis i \\relevante matematiske emner
	} &\shortstack{alle} &\shortstack{} \\ \hline

	\shortstack[l]{\\identifisere variable størrelser i ulike situasjoner, sette opp formler\\ og utforske disse ved hjelp av digitale verktøy
	} &\shortstack{alle} &\shortstack{1} \\ \hline

	\shortstack[l]{\\ utforske strategier for å løse ligninger, ligningssystemer og ulikheter \\og argumentere for tenkemåtene sine
	} &\shortstack{2} &\shortstack{1} \\ \hline

	\shortstack[l]{\\ forklare forskjellen mellom en identitet, en ligning, et algebraisk \\uttrykk og en funksjon
	} &\shortstack{2\\7} &\shortstack{} \\ \hline

	\shortstack[l]{\\ utforske sammenhenger mellom andregradsligninger og\\ andregradsulikheter, andregradsfunksjoner og kvadratsetningene \\og bruke sammenhengene i problemløsing
	} &\shortstack{2 \\{} \\{}} &\shortstack{} \\ \hline

	\shortstack[l]{\\modellere situasjoner knyttet til ulike temaer, drøfte, presentere\\ og forklare resultatene og argumentere for om modellene er gyldige
	} &\shortstack{7\\{}} &\shortstack{1\\{}} \\ \hline

	\shortstack[l]{\\ lese, hente ut og vurdere matematikk i relevante tekster\\ om ulike temaer og presentere relevante beregninger og\\ analyser av resultatene
	} &\shortstack{} &\shortstack{1\\{}\\{}} \\ \hline		

	\shortstack[l]{\\ utforske og beskrive egenskapene ved polynomfunksjoner,\\ rasjonale funksjoner, eksponentialfunksjoner og potensfunksjoner
	} &\shortstack{7\\{}} &\shortstack{\\{}\\{}} \\ \hline

	\shortstack[l]{\\ bruke gjennomsnittlig og momentan vekstfart i konkrete eksempler \\og gjøre rede for den deriverte
	} &\shortstack{6\\{}} &\shortstack{\\{}\\{}} \\ \hline

	\shortstack[l]{\\ forklare polynomdivisjon og bruke det til å omskrive algebraiske \\uttrykk, drøfte funksjoner og løse ligninger og ulikheter
	} &\shortstack{2\\{}} &\shortstack{\\{}\\{}} \\ \hline

	\shortstack[l]{\\ gjøre rede for definisjonene av sinus, cosinus og tangens og bruke \\ trigonometri til å beregne lengder, vinkler og arealer i vilkårlige\\ trekanter
	} &\shortstack{3\\{}\\{}} &\shortstack{\\{}\\{}} \\ \hline

	\shortstack[l]{\\ begrunne sinus-, cosinus- og arealsetningen
	} &\shortstack{3} &\shortstack{} \\ \hline

	\shortstack[l]{\\ bruke trigonometri til å analysere og løse sammensatte teoretiske \\og praktiske problemer med lengder, vinkler og arealer
	} &\shortstack{3} &\shortstack{} \\ \hline
	\end{tabular}
\end{center} \vspace{20pt}

\begin{center}
	\begin{tabular}{p{10.5cm} | c | c |} 
		\textbf{Kompetansemål R1} &	TM1 &AM2 \\ \hline
		\shortstack[l]{\\ planlegge og gjennomføre et selvstendig arbeid med reelle datasett\\ knyttet til naturvitenskapelige temaer og forhold, og analysere\\ og presentere funn
		} &\shortstack{} &\shortstack{} \\ \hline
	
		\shortstack[l]{\\ forstå begrepene vekstfart, grenseverdi, derivasjon og kontinuitet,\\ og bruke disse for å løse praktiske problemer
		} &\shortstack{5\\6} &\shortstack{3} \\ \hline
	
		\shortstack[l]{\\ bruke ulike strategier for å utforske og bestemme grenseverdier til \\funksjoner, og utforske og argumentere for anvendelser av \\grenseverdier
		} &\shortstack{1\\5\\6} &\shortstack{1} \\ \hline		
	

		\shortstack[l]{\\ bestemme den deriverte i et punkt geometrisk, algebraisk og ved \\ numeriske metoder, og gi eksempler på funksjoner som ikke \\ er deriverbare i gitte punkter
		} &\shortstack{5\\6\\{}} &\shortstack{1\\{}\\{}} \\ \hline		
	
		\shortstack[l]{\\ analysere og tolke ulike funksjoner ved å bruke derivasjon \\ \phantom{a}
		} &\shortstack{6\\7} &\shortstack{} \\ \hline	
	
		\shortstack[l]{\\ anvende derivasjon til å analysere og tolke egne matematiske \\modeller av reelle datasett
		} &
		\shortstack{6 \\ 7} &\shortstack{1\\{}} \\ \hline	
		
		\shortstack[l]{\\ utforske og forstå regneregler for potenser og logaritmer, og bruke\\ ulike strategier for å løse eksponentialligninger og logaritmeligninger
		} &
		\shortstack{2 \\{}} &\shortstack{} \\ \hline
		
		\shortstack[l]{\\ modellere og analysere eksponentiell og logistisk vekst i reelle\\ datasett
		} &
		\shortstack{2 \\{}} &\shortstack{1\\{}} \\ \hline	
								
		\shortstack[l]{\\ gjøre rede for og argumentere for om en funksjon er kontinuerlig\\ eller diskontinuerlig i et punkt i et definisjonsområde,\\ og gi eksempler på anvendelser av diskontinuerlige funksjoner
		} &
		\shortstack{1 \\ 5 \\ 6} &\shortstack{} \\ \hline	
		
		\shortstack[l]{\\ utforske, analysere og derivere ulike funksjoner og deres omvendte \\funksjoner, og gjøre rede for egenskaper til og sammenhenger\\ mellom slike funksjoner
		} &
		\shortstack{7\\{}\\{}} &\shortstack{} \\ \hline	
		
		\shortstack[l]{\\ anvende parameterframstillinger til linjer og bruke\\ parameterframstillinger til å løse naturvitenskapelige problemer
		} &
		\shortstack{7\\{}} &\shortstack{3\\{}} \\ \hline	
		
		\shortstack[l]{\\ forstå begrepet vektor og regneregler for vektorer i planet, \\og bruke vektorer til å beregne ulike størrelser i planet
		} &
		\shortstack{4\\{}} &\shortstack{1\\{}} \\ \hline								
		
	\end{tabular}
\end{center}\vspace{20pt}


\begin{center}
\begin{tabular}{p{10.5cm} | c | c |} 
	\textbf{Kompetansemål R2} &	TM2 &AM2 \\ \hline
	\shortstack[l]{\\ utforske egenskaper ved ulike rekker og gjøre rede for \\ praktiske anvendelser av egenskaper ved rekker
	} &\shortstack{1\\{}} &\\ \hline
	
	
	\shortstack[l]{\\ utforske rekursive sammenhenger ved å bruke programmering og\\ presentere egne framgangsmåter
	} &\shortstack{} &\shortstack{1\\{}} \\ \hline
	
	\shortstack[l]{\\gjøre rede for integral som en grenseverdi av en følge av summer,\\ og tolke betydningen av denne grenseverdien i ulike situasjoner
	} &\shortstack{5} &\shortstack{} \\ \hline
	
	\shortstack[l]{\\ gjøre rede for analysens fundamentalteorem og gjøre rede for\\ konsekvenser av teoremet
	} &\shortstack{5\\{}} &\shortstack{} \\ \hline
	
	\shortstack[l]{\\ utvikle algoritmer for å beregne integraler numerisk, \\og bruke programmering til å utføre algoritmene
	} &\shortstack{5\\{}} &\shortstack{1\\{}} \\ \hline
	
	\shortstack[l]{\\ gi eksempler på ulike situasjoner som kan modelleres ved å bruke \\ matematiske funksjoner, og modellere og analysere slike \\ situasjoner ved å bruke reelle datasett
	} &\shortstack{ \\{} \\{}} &\shortstack{1\\{}\\{}} \\ \hline
	
	\shortstack[l]{\\ anvende derivasjon og integrasjon til å analysere og tolke\\ egne matematiske modeller av reelle datasett
	} &\shortstack{\\{}} &\shortstack{1\\{}} \\ \hline
	
	\shortstack[l]{\\ analysere og tolke ulike funksjoner ved å bruke derivasjon\\ og integrasjon, og anvende integrasjon til å beregne ulike\\ mål av omdreiningslegemer
	} &\shortstack{2\\5\\{}} &\shortstack{} \\ \hline		
	
	\shortstack[l]{\\ anvende parameterframstillinger til kurver og bruke\\ parameterframstillinger til å løse naturvitenskapelige problemer\\ inkludert problemer knyttet til fart og akselerasjon
	} &\shortstack{\\{}} &\shortstack{\\{3}\\{}} \\ \hline
	
	\shortstack[l]{\\ utforske og forstå regneregler for vektorer i rommet, og bruke\\ vektorer til å beregne ulike størrelser i rommet
	} &\shortstack{4\\{}} &\shortstack{\\{}\\{}} \\ \hline
	
	\shortstack[l]{\\ utforske egenskaper ved radianer og trigonometriske funksjoner og\\ identiteter og anvende disse egenskapene til å løse \\praktiske problemer
	} &\shortstack{2\\{}\\{}} &\shortstack{\\{}\\{}} \\ \hline
	
	\shortstack[l]{\\ analysere og forstå matematiske bevis, forklare de bærende ideene\\ i et matematisk bevis og utvikle egne bevis
	} &\shortstack{alle\\{}\\{}} &\shortstack{\\{}\\{}} \\ \hline
\end{tabular}
\end{center} 


\end{document}





