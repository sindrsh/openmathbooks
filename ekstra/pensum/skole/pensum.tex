\documentclass{article}
\usepackage[a4paper]{geometry}

\begin{document}
\section*{Kompetansemål}
I tabellene under vises de trinnvise kompetansemålene definert av Udir, og hvilke kapitler fra bøkene i OpenMathBooks\,-\,serien som dekker disse. Merk at begge ''Anvendt''-bøkene har et vedlegg som gir lenker til relevante tekster. I kompetansemål som nevner innhenting av informasjon fra tekster er det underforstått at disse vedleggene er med på å dekke kompetansemålet. I alle kompetansemål hvor anvendelse/praktiske situasjoner er det også underforstått at kapittel 8 og kapittel 4 i henholdsvis \textsl{Anvendt matematikk 1} og \textsl{2} er med på å dekke kompetansemålet.

\begin{center}
	\begin{tabular}{p{10.5cm} | c | c |} 
		\textbf{Kompetansemål 1P} & \textbf{MB} & \textbf{AM1}\\ \hline
		\shortstack[l]{\\ lese, hente ut og vurdere matematikk i tekster om situasjoner fra \\lokalmiljøet, gjøre beregninger knyttet til dette og presentere og\\ argumentere for resultatene
		} &\shortstack{} &\shortstack{} \\ \hline
		
		\shortstack[l]{\\ utforske hvordan ulike premisser vil kunne påvirke hvordan \\matematiske problemer fra samfunnsliv og arbeidsliv løses
		} &\shortstack{} &\shortstack{} \\ \hline
		
		\shortstack[l]{\\ modellere situasjoner knyttet til temaer fra samfunnsliv og \\arbeidsliv, presentere og argumentere for resultatene\\ og for når modellene er gyldige
		} &\shortstack{} &\shortstack{} \\ \hline
		
		\shortstack[l]{\\ identifisere variable størrelser i ulike situasjoner og bruke dem \\til utforsking og generalisering
		} &\shortstack{10} &\shortstack{alle} \\ \hline
		
		\shortstack[l]{\\ tolke og bruke formler som gjelder samfunnsliv og arbeidsliv
		} &\shortstack{} &\shortstack{alle} \\ \hline
		
		\shortstack[l]{\\ bruke prosent, prosentpoeng, promille og vekstfaktor i utregninger \\og presentere og begrunne løsninger
		} &\shortstack{} &\shortstack{3\\4} \\ \hline
		
		\shortstack[l]{\\ utforske, beskrive og bruke begrepene proporsjonalitet\\ og omvendt proporsjonalitet
		} &\shortstack{} &\shortstack{1\\{}} \\ \hline
		
		\shortstack[l]{\\ tolke og bruke sammensatte måleenheter i praktiske sammenhenger\\ og velge egnet måleenhet
		} &\shortstack{} &\shortstack{1} \\ \hline
		
		\shortstack[l]{\\ tolke og bruke funksjoner i matematisk modellering og problemløsing \\\phantom{text}
		} &\shortstack{} &\shortstack{6\\7} \\ \hline
	
		\shortstack[l]{\\planlegge, utføre og presentere selvstendig arbeid knyttet til \\modellering og funksjoner innenfor samfunnsfaglige temaer\\ \phantom{text}
		} &\shortstack{} &\shortstack{2\\6\\7} \\ \hline
		
		
		\shortstack[l]{\\bruke digitale verktøy i utforsking og problemløsing knyttet\\ til egenskaper ved funksjoner, og diskutere løsningene
		} &\shortstack{10\\{}} &\shortstack{7\\{}} \\ \hline
				
		\shortstack[l]{\\tolke og regne med rotuttrykk, potenser og tall på standardform \\ \phantom{a}
		} &\shortstack{8\\11} &\shortstack{} \\ \hline
	\end{tabular}	
\end{center} \vspace{20pt}
\newpage
Årsplan ut ifra Mønsters årsplan 
\begin{center}
	\begin{tabular}{|l | c | c | c|}
		\textbf{Uker} & \textbf{Tema} & \textbf{AM1} & \textbf{MB} \\ \hline 
		2 & Programmering, innføring i Python & 7.1	 & \\ \hline 
		4 & Tall og regning & 1 & 6 \\ \hline
		5 & Lineære funksjoner og modeller & 6 & \\ \hline
		4 & Brøk, forhold og prosent & 3 & \\ \hline
		6 & Potenser og formler & 6 & 8.1-8.2 \\ \hline
		7 & Funksjoner & 6, 7.3 & \\ \hline
		5 & Modellering & 6, 7.3 &
	\end{tabular}
\end{center}


\end{document}





